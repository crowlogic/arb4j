\documentclass[11pt]{article}
\usepackage{amsmath,amssymb,amsthm,mathtools}
\usepackage{enumitem}
\usepackage{hyperref}

\newtheorem{definition}{Definition}
\newtheorem{theorem}{Theorem}
\newtheorem{lemma}{Lemma}
\newtheorem{corollary}{Corollary}
\newtheorem{remark}{Remark}

\title{Oscillatory Processes Generated by Unitary Bijective Time Changes\\ of Stationary Gaussian Processes}

\author{Stephen Crowley\thanks{Email: \texttt{stephencrowley214@gmail.com}}}

\date{August 1, 2025}

\begin{document}

\maketitle

\begin{abstract}
This article establishes that Gaussian processes obtained via unitary, measure-preserving bijective time transformations of stationary processes are a subclass of oscillatory processes in the sense of Priestley. The central object is the unitary composition operator $M_\theta$ (and its inverse), which implements the time change at the Hilbert-space level and conjugates covariance and spectral structures. Comprehensive theorems and proofs are provided for all main statements, including the oscillatory spectral representation, $L^2$-isometry, evolutionary spectrum, and expected zero formula.
\end{abstract}

\section{Scaling Functions and Oscillatory Processes}

\begin{definition}[Scaling Functions]\label{def:scaling}
Let $\mathcal{F}$ denote the set of functions $\theta:\mathbb{R}\to\mathbb{R}$ such that:
\begin{enumerate}[label=(\alph*)]
    \item $\theta$ is continuously differentiable and $\theta'(t)>0$ for all $t$,
    \item $\theta$ is strictly increasing and bijective.
\end{enumerate}
\end{definition}

\begin{remark}
From the inverse function theorem, any $\theta \in \mathcal{F}$ has an everywhere differentiable inverse with
$(\theta^{-1})'(s) = 1/\theta'(\theta^{-1}(s))$ for all $s$ in the range of $\theta$.
\end{remark}

\begin{definition}[Oscillatory Process]\label{def:oscillatory}
A complex-valued, second-order stochastic process $\{X_t\}_{t\in\mathbb{R}}$ is called \emph{oscillatory} if there exists
\begin{enumerate}[label=(\roman*)]
    \item a family of functions $\{\phi_t(\omega)\}$ with $\phi_t(\omega)=A_t(\omega)e^{i\omega t}$ where $A_t(\cdot)\in L^2(\mu)$,
    \item a complex orthogonal-increment process $Z(\omega)$ with $E|dZ(\omega)|^2 = d\mu(\omega)$,
\end{enumerate}
such that
\begin{equation}\label{eq:oscrep}
    X_t = \int_{-\infty}^{\infty} \phi_t(\omega)\,dZ(\omega).
\end{equation}
\end{definition}

\section{Unitary Time-Change Operator and its Inverse}

\begin{definition}[Unitary Time-Change Operator]\label{def:unitary}
Let $\theta\in\mathcal{F}$. Define the operator $M_\theta : L^2(\mathbb{R}) \to L^2(\mathbb{R})$ by
\begin{equation}\label{eq:unitaryM}
    (M_\theta f)(t) := \sqrt{\theta'(t)}\,f(\theta(t)).
\end{equation}
\end{definition}

\begin{lemma}[Unitarity of $M_\theta$]\label{lem:unitary}
The operator $M_\theta$ is unitary; that is, for all $f \in L^2(\mathbb{R})$,
\begin{equation}\label{eq:unitnorm}
    \int_{-\infty}^{\infty} | (M_\theta f)(t) |^2 \, dt = \int_{-\infty}^{\infty} |f(s)|^2 ds.
\end{equation}
\end{lemma}

\begin{proof}
Substitute $s = \theta(t)$, and $ds = \theta'(t)dt$, then:
\begin{align*}
    \int_\mathbb{R} | (M_\theta f)(t) |^2 dt
    &= \int_\mathbb{R} | \sqrt{\theta'(t)} f(\theta(t)) |^2 dt  \\
    &= \int_\mathbb{R} \theta'(t) |f(\theta(t))|^2 dt \\
    &= \int_\mathbb{R} |f(s)|^2 ds.
\end{align*}
Because $\theta$ is bijective and smooth, this covers all of $\mathbb{R}$.
\end{proof}

\begin{definition}[Inverse Operator]\label{def:Mtheta_inv}
The inverse of $M_\theta$ is given by
\begin{equation}\label{eq:unitaryM_inverse}
    (M_\theta^{-1} g)(s) := \frac{g(\theta^{-1}(s))}{\sqrt{\theta'(\theta^{-1}(s))}}.
\end{equation}
\end{definition}

\begin{lemma}[Verification of Inverse]\label{lem:invert}
For any $f \in L^2(\mathbb{R})$, $M_\theta^{-1} M_\theta f = f$ and $M_\theta M_\theta^{-1} g = g$ for all $g \in L^2(\mathbb{R})$.
\end{lemma}

\begin{proof}
\noindent
For $f$:
\[
(M_\theta^{-1} M_\theta f)(s) = \frac{M_\theta f(\theta^{-1}(s))}{\sqrt{\theta'(\theta^{-1}(s))}} = \frac{\sqrt{\theta'(\theta^{-1}(s))} f(\theta(\theta^{-1}(s)))}{\sqrt{\theta'(\theta^{-1}(s))}} = f(s).
\]
For $g$:
\[
(M_\theta M_\theta^{-1} g)(t) = \sqrt{\theta'(t)} \cdot \frac{g(\theta(\theta^{-1}(\theta(t))))}{\sqrt{\theta'(\theta^{-1}(\theta(t)))}} = g(\theta(t)).
\]
But under the substitution $s = \theta(t)$, so $g(\theta(t))$ traverses all of $L^2$ as $t$ traverses $\mathbb{R}$, confirming mutual inverseness.
\end{proof}

\section{Oscillatory Representation of Unitary Time-Changed Stationary Processes}

Let $\{S_t\}$ be a stationary Gaussian process with continuous spectral representation
\begin{equation}\label{eq:stat_spectral}
    S_t = \int_{-\infty}^{\infty} e^{i\omega t}\, dZ(\omega).
\end{equation}
For $\theta\in\mathcal{F}$, define the transformed process
\begin{equation}\label{eq:osc_process}
    X_t := \sqrt{\theta'(t)}\,S_{\theta(t)}.
\end{equation}

\begin{theorem}[Oscillatory Spectral Representation]\label{thm:spectral}
The process defined by \eqref{eq:osc_process} has the oscillatory representation
\begin{equation}\label{eq:oscillatoryrep}
    X_t = \int_{-\infty}^{\infty} \sqrt{\theta'(t)}e^{i\omega \theta(t)} dZ(\omega),
\end{equation}
i.e., with $\phi_t(\omega) = \sqrt{\theta'(t)}e^{i\omega\theta(t)}$.
Moreover, $\phi_t(\omega)$ can be written in the standard form:
\begin{equation}
    \phi_t(\omega) = A_t(\omega)e^{i\omega t}, \qquad
    A_t(\omega) = \sqrt{\theta'(t)} e^{i\omega(\theta(t)-t)}.
\end{equation}
\end{theorem}

\begin{proof}
Substituting \eqref{eq:stat_spectral} into \eqref{eq:osc_process}:
\begin{align*}
    X_t &= \sqrt{\theta'(t)} S_{\theta(t)} = \sqrt{\theta'(t)} \int_{-\infty}^{\infty} e^{i\omega \theta(t)} dZ(\omega) 
    = \int_{-\infty}^{\infty} \sqrt{\theta'(t)} e^{i\omega \theta(t)} dZ(\omega).
\end{align*}
To write in oscillatory standard form:
\[
    \sqrt{\theta'(t)}e^{i\omega \theta(t)} = \sqrt{\theta'(t)} e^{i\omega (\theta(t)-t)}\, e^{i\omega t} = A_t(\omega) e^{i\omega t}.
\]
The square-integrability in $\omega$ follows because $\mu$ is finite and $\theta'$ is everywhere positive and finite.
\end{proof}

\begin{theorem}[Evolutionary Spectrum]\label{thm:evolve}
For the above process, the evolutionary power spectrum at time $t$ is
\begin{equation}
    dF_t(\omega) = |A_t(\omega)|^2 d\mu(\omega) = \theta'(t) d\mu(\omega).
\end{equation}
\end{theorem}

\begin{proof}
By definition, $dF_t(\omega) = |A_t(\omega)|^2 d\mu(\omega)$. Since 
\[
|A_t(\omega)|^2 = |\sqrt{\theta'(t)}|^2 \cdot |e^{i\omega(\theta(t)-t)}|^2 = \theta'(t),
\]
the result follows.
\end{proof}

\section{\texorpdfstring{$L^2$}{L2}-Norm Preservation by Unitary Time-Change}

\begin{theorem}[$L^2$-Norm Preservation]\label{thm:L2}
The unitary time-change preserves the $L^2$-norms of the stochastic processes: for any measurable $I\subseteq\mathbb{R}$,
\begin{equation}
    \int_I \mathbb{E}\left[|X_t|^2\right]dt = \int_{\theta(I)} \mathbb{E}\left[|S_s|^2\right]ds.
\end{equation}
\end{theorem}

\begin{proof}
Using $X_t = \sqrt{\theta'(t)} S_{\theta(t)}$ and stationarity of $S$:
\begin{align*}
    \mathbb{E}[|X_t|^2] &= \theta'(t)\,\mathbb{E}[|S_{\theta(t)}|^2]\\
    &= \theta'(t)\,\mathbb{E}[|S_0|^2]\\
    &= \theta'(t)\,\sigma^2 \quad\text{(where $\sigma^2$ is constant)}
\end{align*}
Thus,
\begin{align*}
    \int_I \mathbb{E}[|X_t|^2]dt = \sigma^2 \int_I \theta'(t) dt = \sigma^2 \int_{\theta(I)} ds = \int_{\theta(I)} \mathbb{E}[|S_s|^2] ds.
\end{align*}
\end{proof}

\section{Expected Zero Formula}

\begin{theorem}[Expected Zero Count]\label{thm:zerocount}
Let $K(\tau)$ be the covariance function of $S_t$ (assumed twice differentiable at $0$), and $\theta\in\mathcal{F}$. The expected number of real zeros of $X_t$ on $[a,b]$ is
\begin{equation}
    \mathbb{E}[N_{[a,b]}] = \sqrt{ -\ddot{K}(0) }\,(\theta(b)-\theta(a)).
\end{equation}
\end{theorem}

\begin{proof}
By the Kac-Rice formula:
\begin{equation}
    \mathbb{E}[N_{[a,b]}] = \int_a^b 
    \sqrt{ -\lim_{s\to t} \frac{\partial^2}{\partial s\partial t} \operatorname{cov}(X_s,X_t) } dt.
\end{equation}
For $X_t = \sqrt{\theta'(t)} S_{\theta(t)}$,
\[
    \operatorname{cov}(X_s, X_t) = \sqrt{\theta'(s)\theta'(t)} K(\theta(t)-\theta(s)).
\]
The relevant limit is:
\begin{align*}
    \lim_{s\to t} \frac{\partial^2}{\partial s\partial t} \left[ \sqrt{\theta'(s)\theta'(t)} K(\theta(t)-\theta(s)) \right]
    &= \theta'(t)^2 K''(0),
\end{align*}
since the cross derivatives act on $K(\theta(t)-\theta(s))$, and $\theta'(t)$ multiplies through. Therefore, 
\begin{align*}
    \mathbb{E}[N_{[a,b]}] &= \int_a^b \sqrt{ - K''(0)\, \theta'(t)^2 } dt
    = \sqrt{ -K''(0) } \int_a^b \theta'(t) dt\\
    &= \sqrt{ -K''(0) } (\theta(b)-\theta(a)).
\end{align*}
\end{proof}

\section{Conclusion}

The class of processes $X_t := \sqrt{\theta'(t)} S_{\theta(t)}$ forms a subclass of oscillatory processes corresponding to measure-preserving, unitary time changes of stationary Gaussian processes. The unitary composition operator $M_\theta$ implements this transformation at the Hilbert space level, preserving all $L^2$ inner products and yields oscillatory spectral representations with evolving spectra. The zero set and energy properties of the process are determined by the geometry of $\theta$ and the spectrum of the underlying stationary process, as proved above.

\begin{thebibliography}{99}
  \bibitem{priestley1965} M.B. Priestley, \emph{Evolutionary spectra and non-stationary processes}. J. Roy. Statist. Soc. Ser. B 27 (1965), 204--237.
  \bibitem{cramer1967} H. Cramér and M.R. Leadbetter, \emph{Stationary and Related Stochastic Processes}. Wiley, 1967.
  \bibitem{kac1943} M. Kac. On the average number of real roots of a random algebraic equation. \emph{Bulletin of the American Mathematical Society}, 49(4):314--320, 1943.
  \bibitem{rice1945} S. O. Rice, Mathematical analysis of random noise. \emph{Bell Syst. Tech. J.}, 24 (1945), 46--156.
\end{thebibliography}

\end{document}
