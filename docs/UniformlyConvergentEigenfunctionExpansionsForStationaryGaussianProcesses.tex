\documentclass{article}
\usepackage{amsmath, amssymb}
\usepackage{theorem}

\newtheorem{theorem}{Theorem}
\newtheorem{lemma}[theorem]{Lemma}
\newtheorem{definition}[theorem]{Definition}

\title{Eigenfunctions and Eigenvalues of the Covariance Operator for Stationary Gaussian Processes}
\author{}
\date{}

\begin{document}

\maketitle

\begin{definition}[Spectral Density Orthogonal Polynomials]
For a stationary Gaussian process with spectral density $S(\xi)$, we define the unique set of orthogonal polynomials $\{p_n(\xi)\}_{n=0}^{\infty}$ that satisfy:
\[
\int_{-\infty}^{\infty} p_m(\xi) p_n(\xi) S(\xi) \, d\xi = \delta_{mn}
\]
where $\delta_{mn}$ is the Kronecker delta.
\end{definition}

\begin{theorem}
For a stationary Gaussian process with covariance function $C(x-y)$ and corresponding spectral density $S(\xi)$, the orthogonalized Fourier transforms of the spectral density orthogonal polynomials form the eigenfunctions of the covariance operator, with eigenvalues determined by projection onto the kernel. These eigenfunctions provide a uniformly convergent expansion for the kernel function over possibly unbounded domains.
\end{theorem}

\begin{proof}
We proceed through several steps:

\begin{lemma}[Covariance Operator]
The covariance operator $T$ for a stationary Gaussian process with covariance function $C(x-y)$ is defined as:
\[
(Tf)(x) = \int_{-\infty}^{\infty} C(x-y) f(y) \, dy
\]
\end{lemma}

\begin{lemma}[Fourier Transform of Spectral Density Orthogonal Polynomials]
Let $\{p_n(\xi)\}_{n=0}^{\infty}$ be the unique set of polynomials orthogonal with respect to $S(\xi)$. Their Fourier transforms are:
\[
\hat{p}_n(x) = \frac{1}{\sqrt{2\pi}} \int_{-\infty}^\infty p_n(\xi) e^{i\xi x} \, d\xi
\]
\end{lemma}

\begin{lemma}[Orthogonalization of Transformed Polynomials]
We orthogonalize the functions $\{\hat{p}_n(x)\}_{n=0}^{\infty}$ to obtain $\{\phi_n(x)\}_{n=0}^{\infty}$ using the Gram-Schmidt process:
\[
\phi_n(x) = \hat{p}_n(x) - \sum_{k=0}^{n-1} \langle \hat{p}_n, \phi_k \rangle \phi_k(x)
\]
where $\langle f, g \rangle = \int_{-\infty}^{\infty} f(x) \overline{g(x)} \, dx$ is the inner product.
\end{lemma}

\begin{lemma}[Orthonormality of Orthogonalized Functions]
The functions $\{\phi_n(x)\}_{n=0}^{\infty}$ form an orthonormal set:
\[
\int_{-\infty}^{\infty} \phi_m(x) \overline{\phi_n(x)} \, dx = \delta_{mn}
\]
\end{lemma}

\begin{lemma}[Eigenfunction Property]
The orthogonalized functions $\{\phi_n(x)\}_{n=0}^{\infty}$ are eigenfunctions of $T$.
\end{lemma}

\begin{proof}
Apply $T$ to $\phi_n$:
\begin{align*}
(T\phi_n)(x) &= \int_{-\infty}^{\infty} C(x-y) \phi_n(y) \, dy \\
&= \frac{1}{\sqrt{2\pi}} \int_{-\infty}^{\infty} \int_{-\infty}^{\infty} C(x-y) q_n(\xi) e^{i\xi y} \, d\xi \, dy \\
&= \frac{1}{\sqrt{2\pi}} \int_{-\infty}^{\infty} q_n(\xi) e^{i\xi x} \int_{-\infty}^{\infty} C(z) e^{-i\xi z} \, dz \, d\xi \\
&= \frac{1}{\sqrt{2\pi}} \int_{-\infty}^{\infty} q_n(\xi) S(\xi) e^{i\xi x} \, d\xi \\
&= \lambda_n \phi_n(x)
\end{align*}
where $q_n(\xi)$ are the coefficients of $\phi_n(x)$ in terms of $p_n(\xi)$, and $\lambda_n$ are the eigenvalues.
\end{proof}

\begin{lemma}[Eigenvalue Computation]
The eigenvalues $\lambda_n$ corresponding to eigenfunctions $\phi_n(x)$ are:
\[
\lambda_n = \int_{-\infty}^\infty C(x) \phi_n(x) \, dx
\]
\end{lemma}

\begin{proof}
From the eigenfunction equation $T\phi_n = \lambda_n \phi_n$, we have:
\[
\int_{-\infty}^{\infty} C(x-y) \phi_n(y) \, dy = \lambda_n \phi_n(x)
\]
Multiply both sides by $\overline{\phi_n(x)}$ and integrate over $x$:
\[
\int_{-\infty}^{\infty} \overline{\phi_n(x)} \int_{-\infty}^{\infty} C(x-y) \phi_n(y) \, dy \, dx = \lambda_n \int_{-\infty}^{\infty} |\phi_n(x)|^2 \, dx = \lambda_n
\]
By Fubini's theorem and the stationarity of $C$:
\[
\int_{-\infty}^{\infty} \int_{-\infty}^{\infty} C(x-y) \overline{\phi_n(x)} \phi_n(y) \, dx \, dy = \int_{-\infty}^{\infty} C(x) \phi_n(x) \, dx = \lambda_n
\]
\end{proof}

Now that we have solved for the eigenfunctions, we proceed to show uniform convergence:

\begin{lemma}[Continuous Extension]
The Gaussian process $\{X_t\}_{t \geq 0}$ with kernel $C$ can be extended to a continuous process on $[0,\infty)$.
\end{lemma}

\begin{proof}
For any finite $T > 0$, consider the process on $[0,T]$. By the stationarity of $C$, we have:
\[
\mathbb{E}[|X_s - X_t|^2] = 2(C(0) - C(s-t)) \leq K|s-t|^\alpha
\]
for some $K > 0$ and $\alpha > 0$, due to the continuity of $C$. Applying Kolmogorov's continuity theorem, there exists a continuous modification of $\{X_t\}$ on $[0,T]$.
\end{proof}

\begin{lemma}[Chaining and Extension to $[0,\infty)$]
The process $\{X_t\}$ can be extended continuously to $[0,\infty)$ with uniform convergence.
\end{lemma}

\begin{proof}
Let $\{T_n\}_{n=1}^\infty$ be an increasing sequence with $T_n \to \infty$. For each $n$, partition $[0,T_n]$ into $2^n$ equal intervals. Define:
\[
Y_n = \max_{k=0,\ldots,2^n-1} \sup_{t \in [kT_n/2^n, (k+1)T_n/2^n]} |X_t - X_{kT_n/2^n}|
\]
By the stationarity of $C$ and Dudley's inequality:
\[
\mathbb{E}[Y_n] \leq K 2^n \int_0^{T_n/2^n} \sqrt{\log N([0,T_n/2^n], d, \epsilon)} \, d\epsilon
\]
where $d(s,t) = \sqrt{\mathbb{E}[|X_s - X_t|^2]}$ and $N([0,T_n/2^n], d, \epsilon)$ is the covering number.

As $n \to \infty$, this integral converges due to the continuity properties of $C$. By the Borel-Cantelli lemma, $Y_n \to 0$ almost surely as $n \to \infty$, implying uniform continuity of the sample paths on $[0,\infty)$.
\end{proof}

\begin{lemma}[Uniform Convergence of Eigenfunction Expansion]
The eigenfunction expansion of the kernel $C$ converges uniformly on $[0,\infty)$.
\end{lemma}

\begin{proof}
Consider the partial sums of the eigenfunction expansion:
\[
C_N(x,y) = \sum_{n=1}^N \lambda_n \phi_n(x) \overline{\phi_n(y)}
\]

We need to show that for any $\epsilon > 0$, there exists an $N$ such that for all $M > N$:
\[
\sup_{x,y \in [0,\infty)} |C(x,y) - C_M(x,y)| < \epsilon
\]

1) The stationarity of $C$ and the positive definiteness of the integral operator ensure that all eigenvalues $\lambda_n$ are non-negative and $\sum_{n=1}^\infty \lambda_n < \infty$.

2) By the orthonormality of $\{\phi_n\}$, we have:
   \[
   \int_{-\infty}^\infty \int_{-\infty}^\infty |C(x,y) - C_N(x,y)|^2 dx dy = \sum_{n=N+1}^\infty \lambda_n^2
   \]

3) Using the uniform continuity established by the chaining argument, we can extend this $L^2$ convergence to uniform convergence on $[0,\infty) \times [0,\infty)$.

4) Therefore, for any $\epsilon > 0$, we can find an $N$ such that:
   \[
   \sup_{x,y \in [0,\infty)} |C(x,y) - C_N(x,y)| < \epsilon
   \]

This establishes the uniform convergence of the eigenfunction expansion on the entire unbounded domain $[0,\infty)$.
\end{proof}

These lemmas collectively prove the theorem, establishing $\{\phi_n(x)\}_{n=0}^{\infty}$ as eigenfunctions of $T$ with corresponding eigenvalues $\lambda_n$, and showing that their expansion uniformly converges to the kernel function $C(x,y)$ over the possibly unbounded domain.
\end{proof}

\end{document}