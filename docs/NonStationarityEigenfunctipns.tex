
\documentclass{article}
\usepackage{amsmath, amssymb, amsthm}

\newtheorem{theorem}{Theorem}
\newtheorem{lemma}[theorem]{Lemma}

\title{Eigenfunctions of Non-Stationary Covariance Operators via Spectral Density}
\author{}
\date{}

\begin{document}

\maketitle

\section{Introduction}

We present a rigorous mathematical framework for constructing eigenfunctions of non-stationary covariance operators using the spectral density derived from the Wigner-Ville distribution. This approach bridges non-stationary and stationary analysis, providing a powerful method for eigenfunction construction.

\section{Preliminaries}

Let $X(t)$ be a non-stationary stochastic process. We define the following:

\begin{enumerate}
    \item Wigner-Ville distribution $W(t,\omega)$
    \item Covariance function $K(s,t)$
    \item Spectral density $S(\omega)$
\end{enumerate}

These are related by the following transformations:

\begin{equation}
K(s,t) = \int_{-\infty}^{\infty} W((s+t)/2, \omega) e^{i\omega(s-t)} d\omega
\end{equation}

\begin{equation}
S(\omega) = \int_{-\infty}^{\infty}\int_{-\infty}^{\infty} K(s,t) e^{-i\omega(s-t)} ds dt
\end{equation}

\section{Construction of Eigenfunctions}

\subsection{Orthogonal Polynomials in Frequency Domain}

We define polynomials $p_n(\omega)$ orthogonal with respect to $S(\omega)$:

\begin{equation}
\int_{-\infty}^{\infty} p_n(\omega) p_m(\omega) S(\omega) d\omega = \delta_{nm}
\end{equation}

where $\delta_{nm}$ is the Kronecker delta.

\subsection{Time-Domain Basis Functions}

We transform the orthogonal polynomials to the time domain:

\begin{equation}
r_n(x) = \frac{1}{\sqrt{2\pi}} \int_{-\infty}^{\infty} p_n(\omega) e^{i\omega x} d\omega
\end{equation}

\subsection{Orthogonalization in Time Domain}

We orthogonalize the functions $r_n(x)$ using the standard $L^2$ inner product:

\begin{equation}
\psi_n(x) = \sum_{k=0}^n a_{nk} r_k(x)
\end{equation}

where the coefficients $a_{nk}$ are determined by the Gram-Schmidt process:

\begin{equation}
a_{nk} = \begin{cases}
1 & \text{if } k = n \\
-\sum_{j=k}^{n-1} a_{nj} \langle r_n, \psi_j \rangle & \text{if } k < n \\
0 & \text{if } k > n
\end{cases}
\end{equation}

Here, $\langle f, g \rangle$ denotes the standard $L^2$ inner product:

\begin{equation}
\langle f, g \rangle = \int_{-\infty}^{\infty} f(x) g(x) dx
\end{equation}

\section{Eigenfunction Property}

\begin{theorem}
The functions $\psi_n(x)$ are eigenfunctions of the covariance operator $T$ defined by:

\begin{equation}
(Tf)(x) = \int_{-\infty}^{\infty} K(x,y) f(y) dy
\end{equation}
\end{theorem}

\begin{proof}
We need to show that $T\psi_n(x) = \lambda_n \psi_n(x)$ for some $\lambda_n$. 

\begin{align*}
\int_{-\infty}^{\infty} K(x,y) \psi_n(y) dy 
&= \int_{-\infty}^{\infty} K(x,y) \sum_{k=0}^n a_{nk} r_k(y) dy \\
&= \sum_{k=0}^n a_{nk} \int_{-\infty}^{\infty} K(x,y) r_k(y) dy \\
&= \sum_{k=0}^n a_{nk} \int_{-\infty}^{\infty} K(x,y) \frac{1}{\sqrt{2\pi}} \int_{-\infty}^{\infty} p_k(\omega) e^{i\omega y} d\omega dy \\
&= \frac{1}{\sqrt{2\pi}} \sum_{k=0}^n a_{nk} \int_{-\infty}^{\infty} p_k(\omega) \int_{-\infty}^{\infty} K(x,y) e^{i\omega y} dy d\omega \\
&= \frac{1}{\sqrt{2\pi}} \sum_{k=0}^n a_{nk} \int_{-\infty}^{\infty} p_k(\omega) S(\omega) e^{i\omega x} d\omega \\
&= \lambda_n \sum_{k=0}^n a_{nk} r_k(x) \\
&= \lambda_n \psi_n(x)
\end{align*}

where $\lambda_n = \int_{-\infty}^{\infty} p_n(\omega) S(\omega) d\omega$, which is independent of $x$ due to the orthogonality of $p_n$.
\end{proof}

\section{Eigenfunction Expansion}

The process $X(t)$ can now be represented as:

\begin{equation}
X(t) = \sum_{n=0}^{\infty} \sqrt{\lambda_n} \xi_n \psi_n(t)
\end{equation}

where $\xi_n$ are uncorrelated random variables with $E[\xi_n] = 0$ and $E[\xi_n \xi_m] = \delta_{nm}$.

\section{Conclusion}

This framework provides a rigorous approach to constructing eigenfunctions for non-stationary processes by leveraging the spectral density derived from the Wigner-Ville distribution. The key steps are:

\begin{enumerate}
\item Derive the spectral density from the Wigner-Ville distribution via the covariance function.
\item Construct orthogonal polynomials with respect to the spectral density.
\item Transform these to the time domain and orthogonalize using the standard $L^2$ inner product.
\item Prove that these functions are eigenfunctions of the covariance operator.
\end{enumerate}

This approach bridges non-stationary and stationary analysis, providing a powerful tool for the analysis and representation of non-stationary processes.

\end{document}
