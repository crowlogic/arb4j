\documentclass{article}
\usepackage{amsmath, amssymb, amsthm}

\newtheorem{theorem}{Theorem}
\newtheorem{lemma}[theorem]{Lemma}

\title{Eigenfunctions of Non-Stationary Covariance Operators via Wigner-Ville Distribution}
\author{}
\date{}

\begin{document}

\maketitle

\section{Introduction}

We present a rigorous mathematical framework for constructing eigenfunctions of non-stationary covariance operators using the Wigner-Ville distribution. This approach extends the eigenfunction method typically used for stationary processes to the non-stationary case.

\section{Preliminaries}

Let $X(t)$ be a non-stationary stochastic process with covariance function $K(s,t)$. The Wigner-Ville distribution $W(t,\omega)$ and the covariance function $K(s,t)$ are related by the following Fourier transform pair:

\begin{equation}
W(t,\omega) = \int_{-\infty}^{\infty} K(t+\tau/2, t-\tau/2) e^{-i\omega\tau} d\tau
\end{equation}

\begin{equation}
K(s,t) = \int_{-\infty}^{\infty} W((s+t)/2, \omega) e^{i\omega(s-t)} d\omega
\end{equation}

\section{Construction of Eigenfunctions}

\subsection{Orthogonal Polynomials in Time-Frequency Domain}

We define polynomials $p_n(t,\omega)$ orthogonal with respect to $W(t,\omega)$:

\begin{equation}
\int_{-\infty}^{\infty}\int_{-\infty}^{\infty} p_n(t,\omega) p_m(t,\omega) W(t,\omega) dt d\omega = \delta_{nm}
\end{equation}

where $\delta_{nm}$ is the Kronecker delta.

\subsection{Time-Domain Basis Functions}

We transform the orthogonal polynomials to the time domain:

\begin{equation}
r_n(s,t) = \int_{-\infty}^{\infty} p_n((s+t)/2, \omega) e^{i\omega(s-t)} d\omega
\end{equation}

\subsection{Orthogonalization in Time Domain}

We orthogonalize the functions $r_n(s,t)$ with respect to $K(s,t)$:

\begin{equation}
\psi_n(s,t) = \sum_{k=0}^n a_{nk} r_k(s,t)
\end{equation}

where the coefficients $a_{nk}$ are determined by:

\begin{equation}
a_{nk} = \begin{cases}
1 & \text{if } k = n \\
-\sum_{j=k}^{n-1} a_{nj} \langle r_n, \psi_j \rangle_K & \text{if } k < n \\
0 & \text{if } k > n
\end{cases}
\end{equation}

Here, $\langle f, g \rangle_K$ denotes the inner product with respect to $K$:

\begin{equation}
\langle f, g \rangle_K = \int_{-\infty}^{\infty}\int_{-\infty}^{\infty} f(s,t) g(s,t) K(s,t) ds dt
\end{equation}

\section{Eigenfunction Property}

\begin{theorem}
The functions $\psi_n(s,t)$ are eigenfunctions of the covariance operator $T$ defined by:

\begin{equation}
(Tf)(s) = \int_{-\infty}^{\infty} K(s,t) f(t) dt
\end{equation}
\end{theorem}

\begin{proof}
We need to show that $T\psi_n(s,u) = \lambda_n \psi_n(s,u)$ for some $\lambda_n$. 

\begin{align*}
\int_{-\infty}^{\infty} K(s,t) \psi_n(t,u) dt 
&= \int_{-\infty}^{\infty} K(s,t) \sum_{k=0}^n a_{nk} r_k(t,u) dt \\
&= \sum_{k=0}^n a_{nk} \int_{-\infty}^{\infty} K(s,t) r_k(t,u) dt \\
&= \sum_{k=0}^n a_{nk} \int_{-\infty}^{\infty} K(s,t) \int_{-\infty}^{\infty} p_k((t+u)/2, \omega) e^{i\omega(t-u)} d\omega dt \\
&= \sum_{k=0}^n a_{nk} \int_{-\infty}^{\infty} p_k((s+u)/2, \omega) e^{i\omega(s-u)} d\omega \int_{-\infty}^{\infty} K(s,t) e^{i\omega(t-s)} dt \\
&= \sum_{k=0}^n a_{nk} \int_{-\infty}^{\infty} p_k((s+u)/2, \omega) e^{i\omega(s-u)} W((s+u)/2, \omega) d\omega \\
&= \sum_{k=0}^n a_{nk} r_k(s,u) \int_{-\infty}^{\infty} p_k((s+u)/2, \omega) W((s+u)/2, \omega) d\omega \\
&= \lambda_n \sum_{k=0}^n a_{nk} r_k(s,u) \\
&= \lambda_n \psi_n(s,u)
\end{align*}

where $\lambda_n = \int_{-\infty}^{\infty} p_n((s+u)/2, \omega) W((s+u)/2, \omega) d\omega$, which is independent of $s$ and $u$ due to the orthogonality of $p_n$.
\end{proof}

\section{Eigenfunction Expansion}

The process $X(t)$ can now be represented as:

\begin{equation}
X(t) = \sum_{n=0}^{\infty} \sqrt{\lambda_n} \xi_n \psi_n(t,t)
\end{equation}

where $\xi_n$ are uncorrelated random variables with $E[\xi_n] = 0$ and $E[\xi_n \xi_m] = \delta_{nm}$.

\section{Conclusion}

This framework provides a rigorous approach to extending the eigenfunction method to non-stationary processes using the Wigner-Ville distribution. The key steps are:

\begin{enumerate}
\item Construct orthogonal polynomials in the time-frequency domain.
\item Transform these to the time domain.
\item Orthogonalize the resulting functions with respect to the covariance function.
\item Prove that these functions are eigenfunctions of the covariance operator.
\end{enumerate}

This approach allows us to construct eigenfunctions for non-stationary processes, providing a basis for analysis and representation of these processes.

\end{document}