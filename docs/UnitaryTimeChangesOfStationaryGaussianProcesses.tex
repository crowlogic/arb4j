\documentclass{article}
\usepackage{amsmath, amssymb, amsthm}
\usepackage[margin=1in]{geometry}

\newtheorem{theorem}{Theorem}
\newtheorem{definition}{Definition}
\newtheorem{lemma}{Lemma}

\title{Unitary Time Changes of Stationary Gaussian Processes}
\author{}
\date{}

\begin{document}
\maketitle

\begin{theorem}[Gain Function for Unitary Time Changes]
Let $X(t)$ be a zero-mean stationary Gaussian process and $\theta: \mathbb{R} \to \mathbb{R}$ be an absolutely continuous bijection with $\theta'(t) \neq 0$ almost everywhere. The unitary time change produces an oscillatory process $Z(t)$ with gain function
\[
A_t(\lambda) = \sqrt{|\theta'(t)|} \, e^{i\lambda(\theta(t) - t)}
\]
\end{theorem}

\begin{definition}[Unitary Time Change Operator]
Let $\theta: \mathbb{R} \to \mathbb{R}$ be an absolutely continuous bijection with $\theta'(t) \neq 0$ almost everywhere. The unitary time change operator $U_\theta$ on $L^2(\mathbb{R})$ is defined by
\[
(U_\theta f)(t) = \sqrt{|\theta'(t)|} \, f(\theta(t))
\]
\end{definition}

\begin{lemma}[Unitarity of Time Change Operator]
The operator $U_\theta$ defined above is unitary on $L^2(\mathbb{R})$.
\end{lemma}

\begin{proof}
For any $f \in L^2(\mathbb{R})$, compute
\begin{align}
\|U_\theta f\|_2^2 &= \int_{-\infty}^{\infty} |f(\theta(t))|^2 |\theta'(t)| \, dt
\end{align}
By the change of variables $s = \theta(t)$, we have $ds = \theta'(t) \, dt$, so
\begin{align}
\|U_\theta f\|_2^2 &= \int_{-\infty}^{\infty} |f(s)|^2 \, ds = \|f\|_2^2
\end{align}
Thus $U_\theta$ is an isometry. Since $\theta$ is a bijection, $U_\theta$ is surjective, hence unitary.
\end{proof}

\begin{definition}[Stationary Gaussian Process]
A zero-mean stationary Gaussian process $X(t)$ has the spectral representation
\[
X(t) = \int_{-\infty}^{\infty} e^{i\lambda t} \, d\Phi(\lambda)
\]
where $\Phi(\lambda)$ is a complex-valued orthogonal increment process with $E[|d\Phi(\lambda)|^2] = dF(\lambda)$ for some finite measure $F$.
\end{definition}

\begin{definition}[Oscillatory Process]
An oscillatory process in the sense of Priestley is a process $Z(t)$ with the representation
\[
Z(t) = \int_{-\infty}^{\infty} \varphi_t(\lambda) \, d\Phi(\lambda)
\]
where $\varphi_t(\lambda)$ is the oscillatory function and $\Phi(\lambda)$ is as in the stationary case. The gain function $A_t(\lambda)$ is defined by
\[
\varphi_t(\lambda) = A_t(\lambda) e^{i\lambda t}
\]
so that
\[
Z(t) = \int_{-\infty}^{\infty} A_t(\lambda) e^{i\lambda t} \, d\Phi(\lambda)
\]
\end{definition}

\begin{proof}[Proof of Main Theorem]
Start with the stationary process
\[
X(t) = \int_{-\infty}^{\infty} e^{i\lambda t} \, d\Phi(\lambda)
\]

Apply the unitary time change operator to obtain
\[
Z(t) = (U_\theta X)(t) = \sqrt{|\theta'(t)|} \, X(\theta(t))
\]

Substituting the spectral representation:
\begin{align}
Z(t) &= \sqrt{|\theta'(t)|} \int_{-\infty}^{\infty} e^{i\lambda \theta(t)} \, d\Phi(\lambda)\\
&= \int_{-\infty}^{\infty} \sqrt{|\theta'(t)|} \, e^{i\lambda \theta(t)} \, d\Phi(\lambda)
\end{align}

To express this in oscillatory form, factor out $e^{i\lambda t}$:
\begin{align}
Z(t) &= \int_{-\infty}^{\infty} \sqrt{|\theta'(t)|} \, e^{i\lambda(\theta(t) - t)} \, e^{i\lambda t} \, d\Phi(\lambda)
\end{align}

Comparing with the oscillatory representation $Z(t) = \int_{-\infty}^{\infty} A_t(\lambda) e^{i\lambda t} \, d\Phi(\lambda)$, we identify the gain function:
\[
A_t(\lambda) = \sqrt{|\theta'(t)|} \, e^{i\lambda(\theta(t) - t)}
\]

The oscillatory function is therefore
\[
\varphi_t(\lambda) = A_t(\lambda) e^{i\lambda t} = \sqrt{|\theta'(t)|} \, e^{i\lambda \theta(t)}
\]
\end{proof}

\begin{theorem}[Kernel Representation]
The covariance kernel of the oscillatory process $Z(t)$ is given by
\[
K_Z(s,t) = \sqrt{|\theta'(s)||\theta'(t)|} \int_{-\infty}^{\infty} e^{i\lambda(\theta(s) - \theta(t))} \, dF(\lambda)
\]
where $F(\lambda)$ is the spectral measure of the original stationary process.
\end{theorem}

\begin{proof}
The covariance is
\begin{align}
K_Z(s,t) &= E[Z(s)\overline{Z(t)}]\\
&= E\left[\int_{-\infty}^{\infty} A_s(\lambda) e^{i\lambda s} \, d\Phi(\lambda) \int_{-\infty}^{\infty} \overline{A_t(\mu) e^{i\mu t}} \, d\overline{\Phi(\mu)}\right]\\
&= \int_{-\infty}^{\infty} A_s(\lambda) \overline{A_t(\lambda)} e^{i\lambda(s-t)} \, dF(\lambda)
\end{align}

Substituting the gain function:
\begin{align}
K_Z(s,t) &= \int_{-\infty}^{\infty} \sqrt{|\theta'(s)|} e^{i\lambda(\theta(s) - s)} \sqrt{|\theta'(t)|} e^{-i\lambda(\theta(t) - t)} e^{i\lambda(s-t)} \, dF(\lambda)\\
&= \sqrt{|\theta'(s)||\theta'(t)|} \int_{-\infty}^{\infty} e^{i\lambda(\theta(s) - \theta(t))} \, dF(\lambda)
\end{align}
\end{proof}

\end{document}\

