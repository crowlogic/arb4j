\documentclass{article}
\usepackage[english]{babel}
\usepackage{geometry,amsmath,amssymb,latexsym,theorem}
\geometry{letterpaper}

%%%%%%%%%% Start TeXmacs macros
\newcommand{\cdummy}{\cdot}
\newcommand{\tmrsub}[1]{\ensuremath{_{\textrm{#1}}}}
\newcommand{\tmtextbf}[1]{\text{{\bfseries{#1}}}}
\newcommand{\tmtextit}[1]{\text{{\itshape{#1}}}}
\newenvironment{proof}{\noindent\textbf{Proof\ }}{\hspace*{\fill}$\Box$\medskip}
\newtheorem{corollary}{Corollary}
{\theorembodyfont{\rmfamily}\newtheorem{example}{Example}}
%%%%%%%%%% End TeXmacs macros

\begin{document}

\title{Energetic Hilbert Subspaces}

\maketitle

\section*{Introduction}

In mathematics, more precisely in functional analysis, an energetic space is,
intuitively, a subspace of a given real Hilbert space equipped with a new
"energetic" inner product. The motivation for the name comes from physics, as
in many physical problems the energy of a system can be expressed in terms of
the energetic inner product.

\section*{Energetic Space}

Formally, consider a real Hilbert space $X$ with the inner product $(\cdot |
\cdot)$ and the norm $\| \cdummy \|$. Let $Y$ be a linear subspace of $X$ and
$B : Y \to X$ be a strongly monotone symmetric linear operator, that is, a
linear operator satisfying:
\begin{itemize}
  \item $(Bu|v) = (u|Bv)$ {\forall} $u, v \in Y$
  
  \item $(Bu|u) \ge c \|u\|^2$ for some constant $c > 0$ and {\forall}$u \in
  Y$
\end{itemize}
The \tmtextbf{energetic inner product} is defined as:
\begin{equation}
  (u|v)_E = (Bu|v)
\end{equation}
for all $u, v \in Y$.

The \tmtextbf{energetic norm} is:
\begin{equation}
  \|u\|_E = \sqrt{(u|u)_E}
\end{equation}
for all $u \in Y$.

The set $Y$ together with the energetic inner product is a pre-Hilbert space.
The \tmtextbf{energetic space} $X_E$ is defined as the completion of $Y$ in
the energetic norm. $X_E$ can be considered a subset of the original Hilbert
space $X$, since any Cauchy sequence in the energetic norm is also Cauchy in
the norm of $X$ (this follows from the strong monotonicity property of $B$).

The energetic inner product is extended from $Y$ to $X_E$ by:
\begin{equation}
  (u|v)_E = \lim_{n \to \infty} (u_n |v_n)_E
\end{equation}
where $(u_n)$ and $(v_n)$ are sequences in $Y$ that converge to points in
$X_E$ in the energetic norm.

\section*{Energetic Extension}

The operator $B$ admits an \tmtextbf{energetic extension} $B_E$:
\begin{equation}
  B_E : X_E \to X_E^{\ast}
\end{equation}
defined on $X_E$ with values in the dual space $X_E^{\ast}$ and given by the
formula:
\begin{equation}
  \langle B_E u|v \rangle_E = (u|v)_E
\end{equation}
for all $u, v \in X_E$.

Here, $\langle \cdot | \cdot \rangle_E$ denotes the duality bracket between
$X_E^{\ast}$ and $X_E$, so $\langle B_E u|v \rangle_E$ actually denotes $(B_E
u) (v)$.

If $u$ and $v$ are elements in the original subspace $Y$, then:
\[ \langle B_E u|v \rangle_E = (u|v)_E = (Bu|v) = \langle u|B|v \rangle \]
by the definition of the energetic inner product. If one views $Bu$, which is
an element in $X$, as an element in the dual $X^{\ast}$ via the Riesz
representation theorem, then $Bu$ will also be in the dual $X_E^{\ast}$ (by
the strong monotonicity property of $B$). Via these identifications, it
follows that $B_E u = Bu$. In other words, the original operator $B : Y \to X$
can be viewed as an operator $B : Y \to X_E^{\ast}$, and then $B_E : X_E \to
X_E^{\ast}$ is simply the function extension of $B$ from $Y$ to $X_E$.

\section*{An Example from Physics}

Consider a string whose endpoints are fixed at two points $a < b$ on the real
line. Let the vertical outer force density at each point $x$ ($a \le x \le b$)
on the string be $f (x) \mathbf{e}$, where $\mathbf{e}$ is a unit vector
pointing vertically and $f : [a, b] \to \mathbb{R}$. Let $u (x)$ be the
deflection of the string at the point $x$ under the influence of the force.
Assuming that the deflection is small, the elastic energy of the string is:
\begin{equation}
  \frac{1}{2}  \int_a^b u' (x)^2  \hspace{0.17em} dx
\end{equation}
and the total potential energy of the string is:
\begin{equation}
  F (u) = \frac{1}{2}  \int_a^b u' (x)^2  \hspace{0.17em} dx - \int_a^b u (x)
  f (x)  \hspace{0.17em} dx
\end{equation}
The deflection $u (x)$ minimizing the potential energy will satisfy the
differential equation:
\begin{equation}
  - u'' = f
\end{equation}
with boundary conditions:
\begin{equation}
  u (a) = u (b) = 0
\end{equation}
To study this equation, consider the space $X = L^2 (a, b)$, the $L^2$ space
of all square-integrable functions $u : [a, b] \to \mathbb{R}$ with respect to
the Lebesgue measure. This space is Hilbert with the inner product:
\begin{equation}
  (u|v) = \int_a^b u (x) v (x)  \hspace{0.17em} dx
\end{equation}
and the norm given by:
\begin{equation}
  \|u\| = \sqrt{(u|u)}
\end{equation}
Let $Y$ be the set of all twice continuously differentiable functions $u : [a,
b] \to \mathbb{R}$ with the boundary conditions $u (a) = u (b) = 0$. Then $Y$
is a linear subspace of $X$.

Consider the operator $B : Y \to X$ given by:
\begin{equation}
  Bu = - u''
\end{equation}
so the deflection satisfies the equation $Bu = f$. Using integration by parts
and the boundary conditions, one can see that:
\begin{equation}
  (Bu|v) = - \int_a^b u'' (x) v (x)  \hspace{0.17em} dx = \int_a^b u' (x) v'
  (x)  \hspace{0.17em} dx = (u|Bv)
\end{equation}
for any $u$ and $v \in Y$. Therefore, $B$ is a symmetric linear operator.

$B$ is also strongly monotone, since by Friedrichs's inequality:
\begin{equation}
  \|u\|^2 = \int_a^b u^2 (x)  \hspace{0.17em} dx \le C \int_a^b u' (x)^2 
  \hspace{0.17em} dx = C (Bu|u)
\end{equation}
for some $C > 0$.

The energetic space with respect to the operator $B$ is the Sobolev space
$H^1_0 (a, b)$. The elastic energy of the string which motivated this study
is:
\begin{equation}
  \frac{1}{2}  \int_a^b u' (x)^2  \hspace{0.17em} dx = \frac{1}{2} (u|u)_E
\end{equation}
so it is half of the energetic inner product of $u$ with itself.

To calculate the deflection $u$ minimizing the total potential energy $F (u)$
of the string, one writes this problem in the form:
\begin{equation}
  (u|v)_E = (f|v) \forall v \in X_E
\end{equation}
Next, one usually approximates $u$ by some $u_h$, a function in a
finite-dimensional subspace of the true solution space. For example, one might
let $u_h$ be a continuous piecewise linear function in the energetic space,
which gives the finite element method. The approximation $u_h$ can be computed
by solving a system of linear equations.

The energetic norm turns out to be the natural norm in which to measure the
error between $u$ and $u_h$, see C{\'e}a's lemma.

\section*{References}

\begin{itemize}
  \item Zeidler, Eberhard. \tmtextit{Applied functional analysis: applications
  to mathematical physics}. New York: Springer-Verlag, 1995. ISBN
  0-387-94442-7.
  
  \item Johnson, Claes. \tmtextit{Numerical solution of partial differential
  equations by the finite element method}. Cambridge University Press, 1987.
  ISBN 0-521-34514-6.
\end{itemize}


{\noindent}\tmtextbf{Corollary 3.} If $\{\varphi_n \} \subset D (P)$ and the
sequence $\{P \varphi_n \}$ is complete in an initial Hilbert space $H$, then
the sequence $\{\varphi_n \}$ is complete in the energetic space $H_P$.

\subsection*{2.3. Examples}

The examples listed below present various ways of determining complete
sequences with the aid of the theorems proved. Having a given sequence, it is
sufficient to prove its completeness in the energetic space $H_P$, since every
sequence complete in $H_P$ is also complete in an initial Hilbert space $H$
{\cite{15}}. Sometimes, however, in order to prove the completeness of a
sequence in $H_P$ one should have shown its completeness in $H$ from the
start. Even when there is no need to do so, it is preferable to become
acquainted with direct proofs of the completeness of the sequences in the
initial space $H$ with the aim of understanding how the theorems proved above
are used.

In all examples, except one, $H$ is assumed to be the space of quadratically
integrable functions and $P$ is always the operator $- \Delta$ with the
homogeneous Dirichlet boundary condition. Therefore the domain $D (- \Delta)$
consists of those functions of the space $H$ which are twice continuously
differentiable in the considered area and vanish on its boundary.

\begin{example}
  Consider on the interval $D = (0, a)$ the sequence $\{\varphi_n \} = \{x^n
  (a - x) : n = 1, 2, \ldots\}$.
  
  Let $\mathcal{F}$ be the family consisting of $\varphi_1$ and $\varphi_2$.
  Raising the difference $a - x$ to $m$th power according to the binomial
  theorem and multiplying by $x^n  (a - x)$ we ascertain that $x^n  (a - x)^{m
  + 1}$ is a linear combination of the terms of the sequence $\{\varphi_n \}$.
  Thus after addition of the constant $\varphi_0 = 1$ to the considered
  sequence, it is obvious that the sequence AS given by the formula (10) is a
  subset of the linear space Lin$\{\varphi_n : n = 0, 1, 2, \ldots\}$. Taking
  two points $x'$ and $x''$ from the open interval $(0, a)$ such that
  $\varphi_1 (x') = \varphi_1 (x'')$ and $\varphi_2 (x') = \varphi_2 (x'')$,
  and dividing the second equation by the first, we obtain $x' = x''$.
  Therefore the family $\mathcal{F}$ separates the points of the interval $(0,
  a)$ and $R (\mathcal{F})$ is the two-point set $\{0, a\}$ being the boundary
  $\partial D$ of the set $D$. Let AS$_2 = \{\varphi_n \}$ and AS\tmrsub{$1$}
  will be the one-element sequence $\{\varphi_0 \}$. Of course, AS\tmrsub{$1$}
  consisting only of a constant function fulfills the assumptions of the
  corollary of Theorem 2 (taking advantage of it one should assume $k = 1$ and
  $S_1 = \partial D$). Thus according to this corollary the sequence
  $\{\varphi_n \}$ is complete in $L_2 (D)$.
\end{example}

To prove the completeness of the sequence $\{\varphi_n \}$ in $H_{- \Delta}$,
let us observe that
\begin{equation}
  1 = - \Delta \frac{\varphi}{2}
\end{equation}
\begin{equation}
  x^{n - 1} = \frac{[n (n - 1) ax^{n - 2} - 2 \Delta \varphi_n]}{(n^2 + n)}
  \forall n = 2, 3, \ldots
\end{equation}
, rom which it follows that every element of the sequence $\{\psi_n \} = \{x^n
: n = 0, 1, 2, \ldots\}$

\begin{proof}
  Let $g \in C (D)$. By virtue of condition (a) there exist $f_n \in
  \text{Lin} \{\varphi_n \}$ such that $\sup \{|f (u) P_n f (u) - f (u) | : u
  \in D\} \leq 1 / n$. Because $1 / f \in L_2 (D)$, the inverse image $f^{- 1}
  [0]$ is of measure zero. Therefore $|P_n f - g| \rightarrow 0$ almost
  everywhere and $|P_n f - g|^2 \leq 1 / |f|^2$. Consequently $\|P_n f - g\|
  \rightarrow 0$ by the Lebesgue dominated convergence theorem, which,
  together with the known fact that the set of continuous functions is dense
  in $L_2 (D)$, proves that the sequence $\{P \varphi_n \}$ is complete in
  $L_2 (D)$.
  
  {\noindent}If condition (b) is fulfilled, then for any $g \in L_2 (D)$ there
  exists a $g_n \in \text{Lin} \{\varphi_n \}$ such that $\|fP_n - fg\| \leq 1
  / n$. The inequality $|f|_{\inf} \cdot \|P_n - g\| \leq \|fP_n - fg\|$
  implies that $\|P_n - g\| \rightarrow 0$ since $|f|_{\inf} \neq 0$, from
  where it follows that the sequence $\{P \varphi_n \}$ is complete in $L_2
  (D)$ also for the case (b). From the two inequalities satisfied by every
  positive definite operator
  \begin{equation}
    \ 
  \end{equation}
  \begin{equation}
    \begin{array}{rl}
      \|Ph\| & \geq \mu \|h\|\\
      \|Ph\| \cdot \|h\| & \geq \|h\|^2
    \end{array} \quad \forall h \in D (P)
  \end{equation}
  {\noindent}where $\mu$ is a positive real number, and $\|h\|_P$ denotes the
  norm in $H_P$, we come to a conclusion that $\|h_n \|_P \rightarrow 0$ for
  any sequence $\{h_n \} \subset D (P)$ such that $\|Ph_n \| \rightarrow 0$.
  Thus by the completeness of the sequence $\{P \varphi_n \}$ in $L_2 (D)$,
  the sequence $\{\varphi_n \}$ is complete in $H_P$, which completes the
  proof.
\end{proof}

{\noindent}From the point (a) of the above theorem the following corollary
results directly:

\begin{corollary}
  If $\{\varphi_n \}$ is a sequence in $D (P)$, and there exists a sequence
  $\{\psi_n \}$ complete in $C (D)$ and a function $f \in C (D)$, $1 / f \in
  L_2 (D)$, such that each function $\psi_n$ can be uniformly approximated by
  linear combinations of the terms of the sequence $\{fP \varphi_n \}$ (in
  particular when $\{\psi_n \} \subset \text{Lin} \{fP \varphi_n \}$, i.e.,
  when each of the functions $\psi_n$ is a linear combination of the terms of
  the sequence $\{fP \varphi_n \}$), then the sequence $\{\varphi_n \}$ is
  complete in the energetic space $H_P$.
\end{corollary}

{\noindent}Assuming that $P$ is the identity operator $(Pu = u)$, choosing the
function $f = 1$ and taking advantage of the part of the proof of Theorem 6
concerning case (a) (where, among other things, we replace the limit $\|P_n f
- g\|^2 \rightarrow 0$ by the limit $|f_n - g|^p \rightarrow 0$) we obtain a
proof of the following corollary:

\begin{corollary}
  Every sequence complete in $C (D)$ is also complete in $L_p (D)$.
\end{corollary}

\begin{corollary}
  If $\{\varphi_n \} \subset D (P)$ and the sequence $\{P \varphi_n \}$ is
  complete in an initial Hilbert space $H$, then the sequence $\{\varphi_n \}$
  is complete in the energetic space $H_P$.
\end{corollary}

\subsection*{2.3. Examples}

The examples listed below present various ways of determining complete
sequences with the aid of the theorems proved. Having a given sequence, it is
sufficient to prove its completeness in the energetic space $H_P$, since every
sequence complete in $H_P$ is also complete in an initial Hilbert space $H$
{\cite{15}}. Sometimes, however, in order to prove the completeness of a
sequence in $H_P$ one should have shown its completeness in $H$ from the
start. Even when there is no need to do so, it is preferable to become
acquainted with direct proofs of the completeness of the sequences in the
initial space $H$ with the aim of understanding how the theorems proved above
are used.

In all examples, except one, $H$ is assumed to be the space of quadratically
integrable functions and $P$ is always the operator $- \Delta$ with the
homogeneous Dirichlet boundary condition. Therefore the domain $D (- \Delta)$
consists of those functions of the space $H$ which are twice continuously
differentiable in the considered area and vanish on its boundary.

\begin{example}
  Consider on the interval $D = (0, a)$ the sequence $\{\varphi_n \} = \{x^n
  (a - x) : n = 1, 2, \ldots\}$.
  
  Let $\mathcal{F}$ be the family consisting of $\varphi_1$ and $\varphi_2$.
  Raising the difference $a - x$ to $m$th power according to the binomial
  theorem and multiplying by $x^n  (a - x)$ we ascertain that $x^n  (a - x)^{m
  + 1}$ is a linear combination of the terms of the sequence $\{\varphi_n \}$.
  Thus after addition of the constant $\varphi_0 = 1$ to the considered
  sequence, it is obvious that the sequence AS given by the formula (10) is a
  subset of the linear space Lin$\{\varphi_n : n = 0, 1, 2, \ldots\}$. Taking
  two points $x'$ and $x''$ from the open interval $(0, a)$ such that
  $\varphi_1 (x') = \varphi_1 (x'')$ and $\varphi_2 (x') = \varphi_2 (x'')$,
  and dividing the second equation by the first, we obtain $x' = x''$.
  Therefore the family $\mathcal{F}$ separates the points of the interval $(0,
  a)$ and $R (\mathcal{F})$ is the two-point set $\{0, a\}$ being the boundary
  $\partial D$ of the set $D$. Let AS$_2 = \{\varphi_n \}$ and AS\tmrsub{$1$}
  will be the one-element sequence $\{\varphi_0 \}$. Of course, AS\tmrsub{$1$}
  consisting only of a constant function fulfills the assumptions of the
  corollary of Theorem 2 (taking advantage of it one should assume $k = 1$ and
  $S_1 = \partial D$). Thus according to this corollary the sequence
  $\{\varphi_n \}$ is complete in $L_2 (D)$.
  
  To prove the completeness of the sequence $\{\varphi_n \}$ in $H_{-
  \Delta}$, let us observe that $1 = - \Delta \varphi / 2, x^{n - 1} = [n (n -
  1) ax^{n - 2} - 2 \Delta \varphi_n] / (n^2 + n), n = 2, 3, \ldots$, from
  which it follows that every element of the sequence $\{\psi_n \} = \{x^n : n
  = 0, 1, 2, \ldots\}$
\end{example}

\end{document}
