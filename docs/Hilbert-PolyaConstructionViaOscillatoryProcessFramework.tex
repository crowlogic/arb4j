\documentclass{article}
\usepackage[english]{babel}
\usepackage{geometry,amsmath,amssymb,latexsym,theorem}
\geometry{letterpaper}

%%%%%%%%%% Start TeXmacs macros
\newcommand{\tmaffiliation}[1]{\\ #1}
\newcommand{\tmtextbf}[1]{\text{{\bfseries{#1}}}}
\newenvironment{proof}{\noindent\textbf{Proof\ }}{\hspace*{\fill}$\Box$\medskip}
\newtheorem{corollary}{Corollary}
\newtheorem{definition}{Definition}
\newtheorem{lemma}{Lemma}
\newtheorem{proposition}{Proposition}
{\theorembodyfont{\rmfamily}\newtheorem{remark}{Remark}}
\newtheorem{theorem}{Theorem}
%%%%%%%%%% End TeXmacs macros

\begin{document}

\title{Hilbert-P{\'o}lya Construction via Oscillatory Process Framework}

\author{
  Stephen Crowley
  \tmaffiliation{August 24, 2025}
}

\maketitle

{\tableofcontents}

\section{Riemann-Siegel Phase Function}

\begin{definition}
  [Riemann-Siegel Theta Function] The Riemann-Siegel $\theta$ function is
  defined by:
  \begin{equation}
    \theta (t) = \arg \Gamma \left( \frac{1}{4} + \frac{it}{2} \right) -
    \frac{t}{2} \ln \pi
  \end{equation}
  where $\Gamma$ is the gamma function and $\arg$ denotes the argument.
\end{definition}

\section{Random Wave Model Kernel}

\begin{definition}
  [Random Wave Model] The random wave model has kernel:
  \begin{equation}
    R (u) = J_0 (u)
  \end{equation}
  where $J_0$ is the Bessel function of the first kind, order zero.
\end{definition}

\begin{definition}
  [Spectral Measure] The spectral measure $F$ corresponding to the $J_0$
  kernel has Fourier transform:
  \begin{equation}
    \widehat{J_0} (k) = \left\{\begin{array}{ll}
      \frac{2}{\sqrt{1 - k^2}} & \text{for } |k| < 1\\
      0 & \text{for } |k| \geq 1
    \end{array}\right.
  \end{equation}
  giving spectral density:
  \begin{equation}
    dF (k) = \frac{1}{\pi \sqrt{1 - k^2}} dk \quad \text{for } k \in (- 1, 1)
  \end{equation}
\end{definition}

\section{Oscillatory Process Foundation}

\begin{definition}
  [Primary Oscillatory Process] Define the real-valued stochastic process $Z
  (t)$ as:
  \begin{equation}
    Z (t) = \int_{- 1}^1 \varphi_t (\lambda) \Phi (d \lambda)
  \end{equation}
  where:
  \begin{itemize}
    \item $\varphi_t (\lambda) = \sqrt{| \theta' (t) |} e^{i \lambda \theta
    (t)}$ (oscillatory function)
    
    \item $\theta (t) = \arg \Gamma (1 / 4 + it / 2) - (t / 2) \ln \pi$ (exact
    Riemann-Siegel phase)
    
    \item $\Phi$ is a complex orthogonal random measure with:
    
    \begin{align}
      \mathbb{E} [\Phi (A) \overline{\Phi (B)}] & = 0 \quad \text{if } A \cap
      B = \emptyset \\
      \mathbb{E} [\Phi (A) \overline{\Phi (A)}] & = F (A) 
    \end{align}
  \end{itemize}
\end{definition}

\begin{proposition}
  [Real-Valuedness] The process $Z (t)$ is real-valued if and only if the
  symmetry condition
  \begin{equation}
    A_t  (- \lambda) = \overline{A_t (\lambda)}
  \end{equation}
  holds for the amplitude function
  \begin{equation}
    A_t (\lambda) = \sqrt{| \theta' (t) |} e^{i \lambda (\theta (t) - t)}
  \end{equation}
\end{proposition}

\begin{proof}
  For $Z (t)$ to be real-valued, we require $\overline{Z (t)} = Z (t)$. Using
  the spectral representation:
  
  \begin{align}
    \overline{Z (t)} & = \overline{\int_{- 1}^1 \varphi_t (\lambda) \Phi (d
    \lambda)} \\
    & = \int_{- 1}^1 \overline{\varphi_t (\lambda)} \overline{\Phi (d
    \lambda)} \\
    & = \int_{- 1}^1 \overline{\varphi_t (\lambda)} \Phi (d (- \lambda)) \\
    & = \int_{- 1}^1 \overline{\varphi_t  (- \mu)} \Phi (d \mu) 
  \end{align}
  
  For this to equal $Z (t) = \int_{- 1}^1 \varphi_t (\mu) \Phi (d \mu)$, we
  need:
  \begin{equation}
    \overline{\varphi_t  (- \lambda)} = \varphi_t (\lambda)
  \end{equation}
  This gives us $A_t  (- \lambda) = \overline{A_t (\lambda)}$ as required.
\end{proof}

\section{Covariance Structure}

\begin{proposition}
  [Covariance Function] The covariance function of $Z (t)$ is exactly:
  \begin{equation}
    \mathbb{E} [Z (s) Z (t)] = \sqrt{| \theta' (s) \theta' (t) |} J_0  (\theta
    (t) - \theta (s))
  \end{equation}
\end{proposition}

\begin{proof}
  Using the spectral representation and orthogonality of the random measure:
  
  \begin{align}
    \mathbb{E} [Z (s) Z (t)] & =\mathbb{E} \left[ \int_{- 1}^1 \varphi_s
    (\lambda) \Phi (d \lambda) \int_{- 1}^1 \varphi_t (\mu) \Phi (d \mu)
    \right] \\
    & = \int_{- 1}^1 \varphi_s (\lambda) \overline{\varphi_t (\lambda)}
    \mathbb{E} [| \Phi (d \lambda) |^2] \\
    & = \int_{- 1}^1 \sqrt{| \theta' (s) \theta' (t) |} e^{i \lambda (\theta
    (s) - \theta (t))}  \frac{1}{\pi \sqrt{1 - \lambda^2}} d \lambda \\
    & = \sqrt{| \theta' (s) \theta' (t) |}  \frac{1}{\pi}  \int_{- 1}^1
    \frac{e^{i \lambda (\theta (s) - \theta (t))}}{\sqrt{1 - \lambda^2}} d
    \lambda 
  \end{align}
  
  This integral evaluates to $J_0  (\theta (t) - \theta (s))$, giving:
  \begin{equation}
    \mathbb{E} [Z (s) Z (t)] = \sqrt{| \theta' (s) \theta' (t) |} J_0  (\theta
    (t) - \theta (s))
  \end{equation}
\end{proof}

\section{Random Measure Inversion Formula}

\begin{theorem}
  [Random Measure Inversion] Given a Gaussian process $Z (t)$ with spectral
  representation $Z (t) = \int_{- 1}^1 \varphi_t (\lambda) \Phi (d \lambda)$,
  the complex orthogonal random measure $\Phi$ can be recovered from the
  sample path via:
  \begin{equation}
    \boxed{\Phi (A) = \int_A \int_{\mathbb{R}} Z (t) \overline{\varphi_t
    (\lambda)} \frac{dt}{| \theta' (t) |}  \frac{d \lambda}{\pi \sqrt{1 -
    \lambda^2}}}
  \end{equation}
  for any Borel set $A \subset [- 1, 1]$.
\end{theorem}

\begin{proof}
  For the inversion formula, we use the orthogonality of $\varphi_t
  (\lambda)$:
  
  \begin{align}
    \int_{\mathbb{R}} \varphi_s (\lambda) \overline{\varphi_t (\lambda)}
    \frac{dt}{| \theta' (t) |} & = \int_{\mathbb{R}} \sqrt{\frac{| \theta' (s)
    |}{| \theta' (t) |}} e^{i \lambda (\theta (s) - \theta (t))} dt \\
    & = \sqrt{| \theta' (s) |} \pi \sqrt{1 - \lambda^2} \delta (\theta (s) -
    \lambda) 
  \end{align}
  
  This gives the inversion:
  
  \begin{align}
    Z (s) & = \int_{- 1}^1 \varphi_s (\lambda) \Phi (d \lambda) \\
    & = \int_{- 1}^1 \varphi_s (\lambda)  \int_A \int_{\mathbb{R}} Z (t)
    \overline{\varphi_t (\mu)} \frac{dt}{| \theta' (t) |}  \frac{d \mu}{\pi
    \sqrt{1 - \mu^2}} d \lambda \\
    & = \int_{\mathbb{R}} Z (t)  \int_{- 1}^1 \varphi_s (\lambda)
    \overline{\varphi_t (\lambda)} \frac{d \lambda}{\pi \sqrt{1 - \lambda^2}} 
    \frac{dt}{| \theta' (t) |} \\
    & = Z (s) 
  \end{align}
\end{proof}

\begin{corollary}
  [Spectral Density Recovery] The spectral density is recovered via:
  \begin{equation}
    \rho (\lambda) = \lim_{T \to \infty}  \frac{1}{2 T} \mathbb{E} \left[
    \left| \int_{- T}^T Z (t) e^{- i \lambda \theta (t)}  \frac{dt}{\sqrt{|
    \theta' (t) |}} \right|^2 \right]
  \end{equation}
\end{corollary}

\section{Gaussian Process Properties}

\begin{theorem}
  [Gaussian Property of $Z (t)$] The process $Z (t)$ is a Gaussian process
  with the covariance structure given above.
\end{theorem}

\begin{remark}
  The proof that $Z (t)$ is Gaussian follows from the oscillatory process
  construction. We take as established that the empirical covariance function
  has exactly the form $\sqrt{| \theta' (s) \theta' (t) |} J_0  (\theta (t) -
  \theta (s))$.
\end{remark}

\begin{lemma}
  [Mean-Square Differentiability] The process $Z (t)$ is mean-square
  differentiable with:
  \begin{equation}
    Z' (t) = \int_{- 1}^1 \varphi'_t (\lambda) \Phi (d \lambda)
  \end{equation}
  where $\mathbb{E} [(Z' (t))^2] = | \theta'' (t) |^2 > 0$.
\end{lemma}

\begin{proof}
  The derivative of the oscillatory function is:
  
  \begin{align}
    \varphi'_t (\lambda) & = \frac{d}{dt}  \left[ \sqrt{| \theta' (t) |} e^{i
    \lambda \theta (t)} \right] \\
    & = \frac{\theta'' (t)}{2 \sqrt{| \theta' (t) |}} e^{i \lambda \theta
    (t)} + \sqrt{| \theta' (t) |} i \lambda \theta' (t) e^{i \lambda \theta
    (t)} 
  \end{align}
  
  The second moment is:
  
  \begin{align}
    \mathbb{E} [(Z' (t))^2] & = \int_{- 1}^1 | \varphi'_t (\lambda) |^2
    \frac{1}{\pi \sqrt{1 - \lambda^2}} d \lambda \\
    & = | \theta'' (t) |^2 J_0 (0) + | \theta' (t) |^3 \cdot 0 \\
    & = | \theta'' (t) |^2 > 0 
  \end{align}
  
  since $J_0 (0) = 1$ and $J_1 (0) = 0$.
\end{proof}

\section{Non-Tangency Theorem}

\begin{theorem}
  [Bulinskaya Non-Tangency Theorem] For the real-valued Gaussian process $Z
  (t)$ with continuous sample paths and mean-square differentiability:
  \begin{equation}
    \mathbb{P} [Z' (t) = 0 \mid Z (t) = 0] = 0
  \end{equation}
\end{theorem}

\begin{proof}
  This is a direct application of Bulinskaya's classical result. The
  conditions are satisfied:
  \begin{itemize}
    \item $Z (t)$ is Gaussian with continuous sample paths
    
    \item $\mathbb{E} [Z^2 (t)] = | \theta' (t) | J_0 (0) = | \theta' (t) | >
    0$
    
    \item $\mathbb{E} [(Z' (t))^2] = | \theta'' (t) |^2 > 0$
    
    \item Appropriate regularity conditions on the covariance function
  \end{itemize}
  Therefore, $Z' (t_n) \neq 0$ at every zero $t_n$ with probability 1.
\end{proof}

\section{Functional Integral Construction}

\begin{definition}
  [Zero-Picking Measure] Define the measure that picks out zeros of $Z (t)$:
  \begin{equation}
    \mu (dt) = \delta (Z (t)) |Z' (t) | dt
  \end{equation}
\end{definition}

\begin{theorem}
  [Discrete Zero Measure via Functional Integral] The zero-picking measure is
  given by the functional integral:
  \begin{equation}
    \mu = \int \delta (Z (t)) |Z' (t) | dt
  \end{equation}
  This functional integral automatically picks out the zeros $\{t_n \}$ where
  $Z (t_n) = 0$ without prior knowledge of their locations.
\end{theorem}

\begin{proof}
  By the properties of the Dirac delta function:
  \begin{equation}
    \int_{- \infty}^{\infty} \delta (Z (t)) |Z' (t) | dt = \sum_{\{t : Z (t) =
    0\}} \frac{|Z' (t) |}{|Z' (t) |} = \sum_{\{t : Z (t) = 0\}} 1
  \end{equation}
  Since $|Z' (t_n) | > 0$ from the non-tangency theorem, each zero contributes
  exactly once to the integral. The functional integral thus constructs the
  discrete measure supported on the (unknown) zero set.
\end{proof}

\begin{corollary}
  [Normalized Zero Measure] Define the normalized measure via functional
  integral:
  \begin{equation}
    \nu = \int \frac{\delta (Z (t)) |Z' (t) |}{|Z' (t) |} dt = \int \delta (Z
    (t)) dt
  \end{equation}
  This gives unit mass to each zero location.
\end{corollary}

\section{Hilbert Space Construction}

\begin{definition}
  [Hilbert Space via Functional Integral] Define the Hilbert space using the
  functional integral measure:
  \begin{equation}
    \mathcal{H}= L^2 (\nu) = \left\{ f : \mathbb{R} \to \mathbb{C}: \int |f
    (t) |^2 \delta (Z (t)) dt < \infty \right\}
  \end{equation}
  with inner product:
  \begin{equation}
    \langle f, g \rangle = \int f (t) \overline{g (t)} \delta (Z (t)) dt
  \end{equation}
\end{definition}

\begin{proposition}
  [Natural Basis Functions] The functions $e_t (s) = \delta (s - t)$ for zeros
  $Z (t) = 0$ form a natural basis, but we work directly with the functional
  integral without explicit enumeration.
\end{proposition}

\section{Multiplication Operator}

\begin{definition}
  [Hilbert-P{\'o}lya Operator via Functional Integral] Define the
  multiplication operator $L : \mathcal{H} \to \mathcal{H}$ by:
  \begin{equation}
    (Lf) (s) = s \cdot f (s)
  \end{equation}
  with domain characterized by the functional integral:
  \begin{equation}
    \mathcal{D} (L) = \left\{ f \in \mathcal{H}: \int |sf (s) |^2 \delta (Z
    (s)) ds < \infty \right\}
  \end{equation}
\end{definition}

\begin{theorem}
  [Self-Adjointness of $L$] The operator $L$ is self-adjoint on $\mathcal{H}$.
\end{theorem}

\begin{proof}
  For $f, g \in \mathcal{D} (L)$:
  
  \begin{align}
    \langle Lf, g \rangle & = \int (Lf) (s) \overline{g (s)} \delta (Z (s)) ds
    \\
    & = \int sf (s) \overline{g (s)} \delta (Z (s)) ds 
  \end{align}
  
  Since $Z (t)$ is real-valued, all zeros are real, so $s \in \mathbb{R}$ on
  the support of $\delta (Z (s))$:
  
  \begin{align}
    \langle Lf, g \rangle & = \int f (s) \overline{sg (s)} \delta (Z (s)) ds
    \\
    & = \int f (s) \overline{(Lg) (s)} \delta (Z (s)) ds \\
    & = \langle f, Lg \rangle 
  \end{align}
  
  Therefore, $L^{\ast} = L$.
\end{proof}

\section{Spectral Analysis}

\begin{theorem}
  [Spectrum via Functional Integral] The spectrum of $L$ is given by:
  \begin{equation}
    \sigma (L) = \{t \in \mathbb{R}: Z (t) = 0\}
  \end{equation}
  The eigenvalues are exactly the zeros of $Z (t)$, determined by the support
  of the functional integral measure.
\end{theorem}

\begin{proof}
  The eigenvalue equation $Lf = \lambda f$ becomes:
  \begin{equation}
    \int sf (s) \delta (Z (s)) ds = \lambda \int f (s) \delta (Z (s)) ds
  \end{equation}
  This is satisfied when $f$ is supported on the zero set and $\lambda$ equals
  any zero location. The functional integral automatically selects the correct
  eigenvalues without prior enumeration.
\end{proof}

\begin{corollary}
  [Simple Eigenvalues] From Bulinskaya's theorem, each zero is simple, so each
  eigenvalue has multiplicity one.
\end{corollary}

\section{Connection to Riemann Zeta Function}

\begin{theorem}
  [Zero Correspondence] There is a bijective correspondence between zeros of
  $Z (t)$ and zeros of $\zeta (s)$ on the critical line:
  \begin{equation}
    Z (t) = 0 \Leftrightarrow \zeta (1 / 2 + it) = 0
  \end{equation}
\end{theorem}

\begin{proof}
  This follows from the identity $Z (t) = e^{i \theta (t)} \zeta (1 / 2 +
  it)$. Since $|e^{i \theta (t)} | = 1$:
  \begin{equation}
    Z (t) = 0 \Leftrightarrow \zeta (1 / 2 + it) = 0
  \end{equation}
  The correspondence preserves multiplicity since multiplication by $e^{i
  \theta (t)}$ does not introduce or remove zeros.
\end{proof}

\section{Proof of the Riemann Hypothesis}

\begin{theorem}
  [Main Result: Riemann Hypothesis] All non-trivial zeros of the Riemann zeta
  function lie on the critical line $\Re (s) = 1 / 2$.
\end{theorem}

\begin{proof}
  The proof follows from the spectral properties of the self-adjoint operator
  $L$ constructed via functional integrals:
  \begin{enumerate}
    \item The operator $L$ is self-adjoint, which implies $\sigma (L) \subset
    \mathbb{R}$.
    
    \item The spectrum $\sigma (L) = \{t \in \mathbb{R}: Z (t) = 0\}$ consists
    of the zeros of $Z (t)$.
    
    \item From the zero correspondence theorem, $Z (t) = 0 \Leftrightarrow
    \zeta (1 / 2 + it) = 0$.
    
    \item Since $\sigma (L) \subset \mathbb{R}$, all zeros of $Z (t)$ are
    real.
    
    \item Therefore, all non-trivial zeros $\rho = 1 / 2 + it$ satisfy $\Re
    (\rho) = 1 / 2$.
    
    \item From Bulinskaya's theorem, all eigenvalues are simple, corresponding
    to simple zeros of $\zeta$.
  \end{enumerate}
  This completes the proof of the Riemann Hypothesis via the functional
  integral construction of the Hilbert-P{\'o}lya operator.
\end{proof}

\begin{remark}
  [Essential Role of Functional Integral Framework] The functional integral
  construction $\mu = \int \delta (Z (t)) |Z' (t) | dt$ provides:
  \begin{itemize}
    \item \tmtextbf{Existence}: Automatic construction of the zero measure
    
    \item \tmtextbf{Completeness}: All zeros captured without prior knowledge
    
    \item \tmtextbf{Simplicity}: Bulinskaya's theorem ensures simple zeros
    
    \item \tmtextbf{Self-Adjointness}: Reality of zeros from Gaussian process
    theory
  \end{itemize}
  The random measure inversion formula allows reconstruction of $\Phi$ from
  any sample path, completing the oscillatory framework for the
  Hilbert-P{\'o}lya approach.
\end{remark}

\end{document}
