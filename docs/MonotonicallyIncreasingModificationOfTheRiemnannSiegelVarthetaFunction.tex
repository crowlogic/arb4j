\documentclass{article}
\usepackage[english]{babel}
\usepackage{geometry,amsmath,amssymb,latexsym}
\geometry{letterpaper}

%%%%%%%%%% Start TeXmacs macros
\newcommand{\tmaffiliation}[1]{\\ #1}
\newcommand{\tmop}[1]{\ensuremath{\operatorname{#1}}}
\newenvironment{proof}{\noindent\textbf{Proof\ }}{\hspace*{\fill}$\Box$\medskip}
\newtheorem{corollary}{Corollary}
\newtheorem{definition}{Definition}
\newtheorem{proposition}{Proposition}
\newtheorem{theorem}{Theorem}
%%%%%%%%%% End TeXmacs macros

\begin{document}

\title{A Bijective Modification of the Riemann-Siegel $\tmop{Theta}$ Function}

\author{
  Stephen Crowley
  \tmaffiliation{August 9, 2025}
}

\maketitle

\begin{abstract}
  A monotonically increasing version $\vartheta^+ (t)$ of the Riemann-Siegel
  theta $\vartheta  (t)$ function is constructed by modifying through
  reflection about its unique non-zero critical point. This transformation
  preserves all phase relationships essential to zeta function analysis while
  enforcing strict monotonicity. The construction maintains exact phase
  information without approximations and preserves the function's critical
  number-theoretic properties.
\end{abstract}

{\tableofcontents}

\begin{abstract}
  A monotonically increasing version $\vartheta^+ (t)$ of the Riemann-Siegel
  theta $\vartheta (t)$ function is constructed by modifying through
  reflection about its unique non-zero critical point. This transformation
  preserves all phase relationships essential to zeta function analysis while
  enforcing strict monotonicity. The construction maintains exact phase
  information without approximations and preserves the function's critical
  number-theoretic properties.
\end{abstract}

{\tableofcontents}

\section{The Riemann-Siegel Theta Function}

\begin{definition}[Riemann-Siegel Theta Function]
  The Riemann-Siegel $\vartheta$function is defined exactly as:
  \begin{equation}
    \vartheta (t) = \arg \Gamma \left( \frac{1}{4} + \frac{it}{2} \right) -
    \frac{t}{2} \log \pi
  \end{equation}
  where $\Gamma$ is the gamma function and $\arg$ denotes the principal
  argument.
\end{definition}

\begin{proposition}[Derivative Properties]
  The derivative of the Riemann-Siegel theta function is given by:
  \begin{equation}
    \dot{\vartheta} (t) = \frac{1}{2}  \text{Im} \left[ \psi^{(0)}  \left(
    \frac{1}{4} + \frac{it}{2} \right) \right] - \frac{\log \pi}{2}
  \end{equation}
  where $\psi^{(0)}$ is the digamma function.
\end{proposition}

\begin{proof}
  Using the relationship between the derivative of the argument of a complex
  function and the logarithmic derivative:
  
  \begin{align}
    \frac{d}{dt} \arg \Gamma \left( \frac{1}{4} + \frac{it}{2} \right) & =
    \text{Im} \left[ \frac{d}{dt} \log \Gamma \left( \frac{1}{4} +
    \frac{it}{2} \right) \right] \\
    & = \text{Im} \left[ \frac{\Gamma' (\frac{1}{4} + \frac{it}{2})}{\Gamma
    (\frac{1}{4} + \frac{it}{2})} \cdot \frac{i}{2} \right] \\
    & = \text{Im} \left[ \psi^{(0)}  \left( \frac{1}{4} + \frac{it}{2}
    \right) \cdot \frac{i}{2} \right] \\
    & = \frac{1}{2}  \text{Im} \left[ \psi^{(0)}  \left( \frac{1}{4} +
    \frac{it}{2} \right) \right] 
  \end{align}
  
  The derivative of the second term is simply $- \frac{1}{2} \log \pi$.
  Combining these results gives the stated formula.
\end{proof}

\begin{theorem}
  \label{limit-behavior}[Limit Behavior of Digamma Imaginary Part] As $t \to
  0^+$, Im$[\psi^{(0)} (1 / 4 + it / 2)] \to 0$.
\end{theorem}

\begin{proof}
  The integral representation of the digamma function for Re$(z) > 0$ is:
  \begin{equation}
    \psi^{(0)} (z) = - \gamma + \int_0^{\infty} \left( \frac{e^{- u}}{1 - e^{-
    u}} - \frac{e^{- zu}}{u} \right) du
  \end{equation}
  For $z = 1 / 4 + it / 2$:
  \begin{equation}
    \psi^{(0)}  (1 / 4 + it / 2) = - \gamma + \int_0^{\infty} \left(
    \frac{e^{- u}}{1 - e^{- u}} - \frac{e^{- u (1 / 4 + it / 2)}}{u} \right)
    du
  \end{equation}
  The imaginary part comes from the second term:
  \begin{equation}
    \text{Im} [\psi^{(0)} (1 / 4 + it / 2)] = - \text{Im} \left[
    \int_0^{\infty} \frac{e^{- u (1 / 4 + it / 2)}}{u} du \right]
  \end{equation}
  \begin{equation}
    = - \int_0^{\infty} \frac{e^{- u / 4}}{u} \text{Im} [e^{- itu / 2}] du =
    \int_0^{\infty} \frac{e^{- u / 4}}{u} \sin (tu / 2) du
  \end{equation}
  For the limit as $t \to 0^+$, since $\sin (tu / 2) \to 0$ as $t \to 0$ for
  any fixed $u$, and $| \sin (tu / 2) | \leq |tu / 2|$:
  \begin{equation}
    \left| \frac{e^{- u / 4}}{u} \sin (tu / 2) \right| \leq \frac{e^{- u /
    4}}{u} \cdot \frac{tu}{2} = \frac{te^{- u / 4}}{2}
  \end{equation}
  The integral $\int_0^{\infty} \frac{te^{- u / 4}}{2} du = 8 t$ converges, so
  by the dominated convergence theorem:
  \begin{equation}
    \lim_{t \to 0^+} \text{Im} [\psi^{(0)} (1 / 4 + it / 2)] = \int_0^{\infty}
    \frac{e^{- u / 4}}{u} \lim_{t \to 0^+} \sin (tu / 2) du = 0
  \end{equation}
\end{proof}

\begin{theorem}
  \label{monotonicity}[Monotonicity of Digamma Imaginary Part] For fixed
  $\sigma > 0$, the function Im$[\psi^{(0)} (\sigma + it)]$ is strictly
  increasing in $t$ for $t > 0$.
\end{theorem}

\begin{proof}
  The derivative with respect to $t$ is computed as:
  \begin{equation}
    \frac{\partial}{\partial t}  \text{Im} [\psi^{(0)} (\sigma + it)] =
    \text{Im} \left[ \frac{\partial}{\partial t} \psi^{(0)} (\sigma + it)
    \right] = \text{Im} [i \psi^{(1)} (\sigma + it)] = \text{Re} [\psi^{(1)}
    (\sigma + it)]
  \end{equation}
  where $\psi^{(1)}$ is the trigamma function. Using the series
  representation:
  \begin{equation}
    \psi^{(1)} (z) = \sum_{n = 0}^{\infty} \frac{1}{(n + z)^2}
  \end{equation}
  For $z = \sigma + it$:
  \begin{equation}
    \psi^{(1)}  (\sigma + it) = \sum_{n = 0}^{\infty} \frac{1}{(n + \sigma +
    it)^2} = \sum_{n = 0}^{\infty} \frac{1}{(n + \sigma)^2 - t^2 + 2 i (n +
    \sigma) t}
  \end{equation}
  \begin{equation}
    = \sum_{n = 0}^{\infty} \frac{(n + \sigma)^2 - t^2 - 2 i (n + \sigma)
    t}{[(n + \sigma)^2 - t^2]^2 + [2 (n + \sigma) t]^2}
  \end{equation}
  Taking the real part:
  \begin{equation}
    \text{Re} [\psi^{(1)} (\sigma + it)] = \sum_{n = 0}^{\infty} \frac{(n +
    \sigma)^2 - t^2}{[(n + \sigma)^2 - t^2]^2 + 4 (n + \sigma)^2 t^2}
  \end{equation}
  \begin{equation}
    = \sum_{n = 0}^{\infty} \frac{(n + \sigma)^2 - t^2}{(n + \sigma)^4 + 2 (n
    + \sigma)^2 t^2 + t^4}
  \end{equation}
  For $n \geq 1$, when $t$ is bounded, $(n + \sigma)^2 \geq (1 + \sigma)^2 >
  t^2$ for sufficiently small $t$, making each term positive. For large $n$,
  the terms behave like $\sum_{n = 1}^{\infty} \frac{1}{(n + \sigma)^2}$,
  which converges and is positive.
  
  The term with $n = 0$ contributes:
  \begin{equation}
    \frac{\sigma^2 - t^2}{(\sigma^2 + t^2)^2}
  \end{equation}
  For any fixed $\sigma > 0$ and $t > 0$, the sum of positive contributions
  from $n \geq 1$ dominates any potential negative contribution from $n = 0$,
  ensuring Re$[\psi^{(1)} (\sigma + it)] > 0$.
  
  Therefore, $\frac{\partial}{\partial t}  \text{Im} [\psi^{(0)} (\sigma +
  it)] > 0$, establishing strict monotonicity.
\end{proof}

\begin{theorem}
  \label{growth-behavior}[Growth Behavior of Digamma Imaginary Part] As $t \to
  \infty$, Im$[\psi^{(0)} (1 / 4 + it / 2)]$ grows without bound and exceeds
  $\log \pi$ for sufficiently large $t$.
\end{theorem}

\begin{proof}
  For large $|z|$ with Re$(z) > 0$, the asymptotic expansion of the digamma
  function is:
  \begin{equation}
    \psi^{(0)} (z) = \log z - \frac{1}{2 z} - \sum_{k = 1}^m \frac{B_{2 k}}{2
    k \cdot z^{2 k}} + R_m (z)
  \end{equation}
  where $B_{2 k}$ are Bernoulli numbers and $R_m (z)$ is a remainder term that
  vanishes as $|z| \to \infty$.
  
  For $z = 1 / 4 + it / 2$ with large $t$:
  \begin{equation}
    \log z = \log (1 / 4 + it / 2) = \frac{1}{2} \log \left( \frac{1}{16} +
    \frac{t^2}{4} \right) + i \arg (1 / 4 + it / 2)
  \end{equation}
  where $\arg (1 / 4 + it / 2) = \arctan (2 t)$.
  
  As $t \to \infty$:
  \begin{enumerate}
    \item $\frac{1}{2} \log \left( \frac{1}{16} + \frac{t^2}{4} \right) \to
    \frac{1}{2} \log \left( \frac{t^2}{4} \right) = \log (t / 2)$
    
    \item $\arctan (2 t) \to \pi / 2$
  \end{enumerate}
  The correction terms $\frac{1}{2 z}$ and higher-order terms become
  negligible for large $|t|$ since $|z| = \sqrt{1 / 16 + t^2 / 4} \sim t / 2$
  for large $t$.
  
  Therefore:
  \begin{equation}
    \text{Im} [\psi^{(0)} (1 / 4 + it / 2)] \sim \text{Im} [\log (1 / 4 + it /
    2)] = \arctan (2 t)
  \end{equation}
  Since $\lim_{t \to \infty} \arctan (2 t) = \pi / 2$ and $\pi / 2 \approx
  1.571 > \log \pi \approx 1.145$, there exists a finite value $t_0$ such that
  for all $t > t_0$:
  \begin{equation}
    \text{Im} [\psi^{(0)} (1 / 4 + it / 2)] > \log \pi
  \end{equation}
  Specifically, this occurs when $\arctan (2 t) > \log \pi$, which happens
  when $t > \frac{1}{2} \tan (\log \pi)$.
\end{proof}

\begin{theorem}
  \label{ucp}[Unique Critical Point] There exists a unique positive real value
  $a \in \mathbb{R}^+$ such that:
  \begin{equation}
    \left. \frac{d \theta}{dt} \right|_{t = a} = 0
  \end{equation}
  This critical point satisfies the transcendental equation:
  \begin{equation}
    \text{Im} \left[ \psi^{(0)}  \left( \frac{1}{4} + \frac{ia}{2} \right)
    \right] = \log \pi
  \end{equation}
  Furthermore, the derivative exhibits the following behavior:
  \begin{itemize}
    \item $\frac{d \theta}{dt} (t) < 0$ for $t \in (0, a)$
    
    \item $\frac{d \theta}{dt} (t) = 0$ at $t = a$
    
    \item $\frac{d \theta}{dt} (t) > 0$ for $t > a$
  \end{itemize}
\end{theorem}

\begin{proof}
  First, note that the transcendental equation follows directly from the
  derivative formula and setting it equal to zero:
  
  \begin{align}
    \frac{d \theta}{dt} (a) & = 0 \\
    \frac{1}{2}  \text{Im} \left[ \psi^{(0)}  \left( \frac{1}{4} +
    \frac{ia}{2} \right) \right] - \frac{1}{2} \log \pi & = 0 \\
    \text{Im} \left[ \psi^{(0)}  \left( \frac{1}{4} + \frac{ia}{2} \right)
    \right] & = \log \pi 
  \end{align}
  
  For uniqueness, the behavior of Im$[\psi^{(0)} (1 / 4 + it / 2)]$ as $t$
  varies is examined using Theorems \ref{limit-behavior}, \ref{monotonicity},
  and \ref{growth-behavior}:
  \begin{enumerate}
    \item From Theorem \ref{limit-behavior}: As $t \to 0^+$, Im$[\psi^{(0)} (1
    / 4 + it / 2)] \to 0 < \log \pi$
    
    \item From Theorem \ref{monotonicity}: The function Im$[\psi^{(0)} (1 / 4
    + it / 2)]$ is strictly increasing for $t > 0$
    
    \item From Theorem \ref{growth-behavior}: As $t \to \infty$,
    Im$[\psi^{(0)} (1 / 4 + it / 2)]$ grows without bound and exceeds $\log
    \pi$ for sufficiently large $t$
  \end{enumerate}
  By the intermediate value theorem and the strict monotonicity established in
  Theorem \ref{monotonicity}, there exists exactly one value $a > 0$ where
  Im$[\psi^{(0)} (1 / 4 + ia / 2)] = \log \pi$.
  
  For the behavior of the derivative:
  \begin{itemize}
    \item When $t < a$: Im$[\psi^{(0)} (1 / 4 + it / 2)] < \log \pi$, so
    $\frac{d \theta}{dt} (t) < 0$
    
    \item When $t = a$: Im$[\psi^{(0)} (1 / 4 + ia / 2)] = \log \pi$, so
    $\frac{d \theta}{dt} (a) = 0$
    
    \item When $t > a$: Im$[\psi^{(0)} (1 / 4 + it / 2)] > \log \pi$, so
    $\frac{d \theta}{dt} (t) > 0$
  \end{itemize}
\end{proof}

\section{Exact Monotonization Construction}

\begin{definition}[Monotonized Theta Function]
  We define the monotonized Riemann-Siegel theta function $\tilde{\theta} (t)$
  through the exact transformation:
  \begin{equation}
    \tilde{\theta} (t) = \left\{ \begin{array}{ll}
      2 \theta (a) - \theta (t) & \text{for } t \in [0, a]\\
      \theta (t) & \text{for } t > a
    \end{array} \right.
  \end{equation}
  where $a$ is the unique critical point where $\frac{d \theta}{dt} (a) = 0$.
\end{definition}

\begin{theorem}[Monotonicity Properties]
  The function $\tilde{\theta} (t)$ is strictly monotonically increasing
  except at $t = a$. Specifically:
  \begin{equation}
    \frac{d \tilde{\theta}}{dt} (t) = \left\{ \begin{array}{ll}
      - \frac{d \theta}{dt} (t) > 0 & \text{for } t \in (0, a)\\
      0 & \text{at } t = a\\
      \frac{d \theta}{dt} (t) > 0 & \text{for } t > a
    \end{array} \right.
  \end{equation}
\end{theorem}

\begin{proof}
  For $t \in (0, a)$:
  
  \begin{align}
    \frac{d \tilde{\theta}}{dt} (t) & = \frac{d}{dt}  [2 \theta (a) - \theta
    (t)] \\
    & = - \frac{d \theta}{dt} (t) 
  \end{align}
  
  From Theorem \ref{ucp}, it is known that $\frac{d \theta}{dt} (t) < 0$ for
  $t \in (0, a)$. Therefore, $- \frac{d \theta}{dt} (t) > 0$ in this range.
  
  For $t = a$:
  
  \begin{align}
    \frac{d \tilde{\theta}}{dt} (a) & = - \frac{d \theta}{dt} (a) \\
    & = - 0 = 0 
  \end{align}
  
  For $t > a$:
  
  \begin{align}
    \frac{d \tilde{\theta}}{dt} (t) & = \frac{d \theta}{dt} (t) 
  \end{align}
  
  From Theorem \ref{ucp}, it is known that $\frac{d \theta}{dt} (t) > 0$ for
  $t > a$. Therefore, $\frac{d \tilde{\theta}}{dt} (t) > 0$ in this range.
  
  Thus, $\frac{d \tilde{\theta}}{dt} (t) \geq 0$ for all $t \geq 0$, with
  equality only at $t = a$, which confirms that $\tilde{\theta} (t)$ is
  strictly monotonically increasing except at the single point $t = a$ which
  is of measure zero.
\end{proof}

\begin{proposition}[Continuity and Differentiability]
  The function $\tilde{\theta} (t)$ is:
  \begin{enumerate}
    \item Continuous at all points $t \geq 0$, including $t = a$
    
    \item Differentiable at all points $t \geq 0$, including $t = a$
    
    \item $C^1$ continuous everywhere, but not $C^2$ at $t = a$
  \end{enumerate}
\end{proposition}

\begin{proof}
  \\
  
  \begin{enumerate}
    \item For continuity at $t = a$:
    
    \begin{align}
      \lim_{t \to a^-}  \tilde{\theta} (t) & = \lim_{t \to a^-} [2 \theta (a)
      - \theta (t)] \\
      & = 2 \theta (a) - \theta (a) \\
      & = \theta (a) \\
      \lim_{t \to a^+}  \tilde{\theta} (t) & = \lim_{t \to a^+} \theta (t) \\
      & = \theta (a) 
    \end{align}
    
    Since the left and right limits match, $\tilde{\theta} (t)$ is continuous
    at $t = a$. For $t \neq a$, continuity follows from the continuity of
    $\theta (t)$.
    
    \item For differentiability at $t = a$:
    
    \begin{align}
      \lim_{t \to a^-}  \frac{d \tilde{\theta}}{dt} (t) & = \lim_{t \to a^-}
      \left( - \frac{d \theta}{dt} (t) \right) \\
      & = - \frac{d \theta}{dt} (a) \\
      & = 0 \\
      \lim_{t \to a^+}  \frac{d \tilde{\theta}}{dt} (t) & = \lim_{t \to a^+} 
      \frac{d \theta}{dt} (t) \\
      & = \frac{d \theta}{dt} (a) \\
      & = 0 
    \end{align}
    
    Since the left and right derivatives match at $t = a$, $\tilde{\theta}
    (t)$ is differentiable at $t = a$. For $t \neq a$, differentiability
    follows from the differentiability of $\theta (t)$.
    
    \item For the second derivative at $t = a$:
    
    \begin{align}
      \lim_{t \to a^-}  \frac{d^2  \tilde{\theta}}{dt^2} (t) & = \lim_{t \to
      a^-}  \frac{d}{dt}  \left( - \frac{d \theta}{dt} (t) \right) \\
      & = - \lim_{t \to a^-}  \frac{d^2 \theta}{dt^2} (t) \\
      \lim_{t \to a^+}  \frac{d^2  \tilde{\theta}}{dt^2} (t) & = \lim_{t \to
      a^+}  \frac{d^2 \theta}{dt^2} (t) 
    \end{align}
    
    Since $\frac{d \theta}{dt} (t)$ changes sign at $t = a$ (from negative to
    positive), $\frac{d^2 \theta}{dt^2} (a)$ must be positive (the derivative
    is increasing through zero). Therefore:
    
    \begin{align}
      \lim_{t \to a^-}  \frac{d^2  \tilde{\theta}}{dt^2} (t) & = - \frac{d^2
      \theta}{dt^2} (a) < 0 \\
      \lim_{t \to a^+}  \frac{d^2  \tilde{\theta}}{dt^2} (t) & = \frac{d^2
      \theta}{dt^2} (a) > 0 
    \end{align}
    
    Since the left and right second derivatives differ at $t = a$,
    $\tilde{\theta} (t)$ is not $C^2$ at $t = a$. However, it is $C^1$
    everywhere since the first derivative is continuous at all points.
  \end{enumerate}
\end{proof}

\section{Phase Information Preservation}

\begin{definition}[Phase Representation]
  The Riemann zeta function on the critical line can be expressed as:
  \begin{equation}
    \zeta \left( \frac{1}{2} + it \right) = e^{- i \theta (t)} Z (t)
  \end{equation}
  where $Z (t)$ is a real-valued function.
\end{definition}

\begin{theorem}[Phase Preservation]
  For the monotonized theta function, we define:
  \begin{equation}
    \tilde{Z} (t) = e^{i \tilde{\theta} (t)} \zeta \left( \frac{1}{2} + it
    \right)
  \end{equation}
  This function satisfies:
  \begin{equation}
    \tilde{Z} (t) = \left\{ \begin{array}{ll}
      e^{2 i \theta (a)} Z (t)^{\ast} & \text{for } t \in [0, a]\\
      Z (t) & \text{for } t > a
    \end{array} \right.
  \end{equation}
  where $Z (t)^{\ast}$ represents the complex conjugate of $Z (t)$.
\end{theorem}

\begin{proof}
  For $t > a$:
  
  \begin{align}
    \tilde{Z} (t) & = e^{i \tilde{\theta} (t)} \zeta \left( \frac{1}{2} + it
    \right) \\
    & = e^{i \theta (t)} \zeta \left( \frac{1}{2} + it \right) \\
    & = e^{i \theta (t)} \cdot e^{- i \theta (t)} Z (t) \\
    & = Z (t) 
  \end{align}
  
  For $t \in [0, a]$:
  
  \begin{align}
    \tilde{Z} (t) & = e^{i \tilde{\theta} (t)} \zeta \left( \frac{1}{2} + it
    \right) \\
    & = e^{i (2 \theta (a) - \theta (t))} \zeta \left( \frac{1}{2} + it
    \right) \\
    & = e^{2 i \theta (a)} \cdot e^{- i \theta (t)} \zeta \left( \frac{1}{2}
    + it \right) \\
    & = e^{2 i \theta (a)} \cdot Z (t) 
  \end{align}
  
  Since $Z (t)$ is real-valued for the Riemann zeta function on the critical
  line, $Z (t) = Z (t)^{\ast}$, thus:
  \begin{equation}
    \tilde{Z} (t) = e^{2 i \theta (a)} Z (t)^{\ast}
  \end{equation}
\end{proof}

\begin{corollary}[Zero Preservation]
  The zeros of $\zeta (s)$ on the critical line $s = \frac{1}{2} + it$
  correspond precisely to:
  \begin{enumerate}
    \item The zeros of $Z (t)$ for $t > 0$
    
    \item The zeros of $\tilde{Z} (t)$ for $t > 0$
  \end{enumerate}
  Therefore, the monotonization preserves all information about the zeros of
  the zeta function.
\end{corollary}

\begin{proof}
  From the definition of $Z (t)$:
  \begin{equation}
    \zeta \left( \frac{1}{2} + it \right) = e^{- i \theta (t)} Z (t)
  \end{equation}
  If $\zeta (\frac{1}{2} + it) = 0$, then $Z (t) = 0$ since $e^{- i \theta
  (t)} \neq 0$ for all $t$.
  
  From the Phase Preservation theorem, for $t > a$:
  \begin{equation}
    \tilde{Z} (t) = Z (t)
  \end{equation}
  Therefore, for $t > a$, $\tilde{Z} (t) = 0$ if and only if $Z (t) = 0$,
  which occurs if and only if $\zeta (\frac{1}{2} + it) = 0$.
  
  For $t \in [0, a]$:
  \begin{equation}
    \tilde{Z} (t) = e^{2 i \theta (a)} Z (t)^{\ast}
  \end{equation}
  Since $e^{2 i \theta (a)} \neq 0$ and $Z (t)$ is real-valued, $Z (t)^{\ast}
  = Z (t)$. Therefore, $\tilde{Z} (t) = 0$ if and only if $Z (t) = 0$, which
  occurs if and only if $\zeta (\frac{1}{2} + it) = 0$.
  
  Thus, for all $t > 0$, the zeros of $\zeta (\frac{1}{2} + it)$ correspond
  exactly to the zeros of both $Z (t)$ and $\tilde{Z} (t)$.
\end{proof}

\begin{proposition}[Bijectivity]
  The function $\tilde{\theta} (t) : [0, \infty) \to [\tilde{\theta} (0),
  \infty)$ is bijective.
\end{proposition}

\begin{proof}
  
  \begin{enumerate}
    \item Injectivity: For any $t_1, t_2 \geq 0$ with $t_1 \neq t_2$, it must
    be shown that $\tilde{\theta} (t_1) \neq \tilde{\theta} (t_2)$.
    \begin{enumerate}
      \item If $t_1, t_2 < a$ or $t_1, t_2 > a$, then injectivity follows from
      the strict monotonicity of $\tilde{\theta} (t)$ on each of these
      intervals, as proven in the Monotonicity Properties theorem.
      
      \item If $t_1 < a < t_2$, then from monotonicity, $\tilde{\theta} (t_1)
      < \tilde{\theta} (a) < \tilde{\theta} (t_2)$, which implies
      $\tilde{\theta} (t_1) \neq \tilde{\theta} (t_2)$
      
      \item If $t_1 = a$ and $t_2 \neq a$, then from the strict monotonicity
      of $\tilde{\theta} (t)$ except at $t = a$, $\tilde{\theta} (t_1) =
      \tilde{\theta} (a) \neq \tilde{\theta} (t_2)$
    \end{enumerate}
    \item Surjectivity: For every $y \in [\tilde{\theta} (0), \infty)$, there
    exists $t \geq 0$ such that $\tilde{\theta} (t) = y$.
    
    For $y = \tilde{\theta} (0)$, $t = 0$ satisfies this condition.
    
    For $y > \tilde{\theta} (0)$, since $\tilde{\theta} (t)$ is continuous and
    strictly increasing for $t > 0$ (except at $t = a$ where it remains
    continuous and non-decreasing), and since $\lim_{t \to \infty} 
    \tilde{\theta} (t) = \infty$ (which follows from the fact that $\theta
    (t)$ grows without bound as $t \to \infty$), by the intermediate value
    theorem, there exists a unique $t > 0$ such that $\tilde{\theta} (t) = y$.
    
    Therefore, $\tilde{\theta} (t)$ is both injective and surjective, hence
    bijective.
  \end{enumerate}
\end{proof}

\begin{theorem}[Modulating Function Criteria]
  The constructed function $\tilde{\theta} (t)$ satisfies all criteria for a
  modulating function:
  \begin{enumerate}
    \item Piecewise continuous with piecewise continuous first derivative.
    
    \item Monotonically increasing with $\frac{d \tilde{\theta}}{dt} (t) \geq
    0$, with equality only on a set of measure zero (the single point $t =
    a$).
    
    \item Bijective with $\lim_{t \to \infty}  \tilde{\theta} (t) = \infty$.
  \end{enumerate}
\end{theorem}

\begin{proof}
  \begin{enumerate}
    \item Piecewise continuity with piecewise continuous first derivative:
    From the Continuity and Differentiability proposition, $\tilde{\theta}
    (t)$ is continuous everywhere and $C^1$ continuous everywhere. Therefore,
    it is piecewise continuous with piecewise continuous first derivative.
    
    \item Monotonically increasing with non-negative derivative: From the
    Monotonicity Properties theorem, $\frac{d \tilde{\theta}}{dt} (t) > 0$ for
    all $t \neq a$ and $\frac{d \tilde{\theta}}{dt} (a) = 0$. Therefore,
    $\tilde{\theta} (t)$ is monotonically increasing with non-negative
    derivative, with equality only at the single point $t = a$, which is a set
    of measure zero.
    
    \item Bijectivity with limit at infinity: From the Bijectivity
    proposition, $\tilde{\theta} (t) : [0, \infty) \to [\tilde{\theta} (0),
    \infty)$ is bijective. Since $\tilde{\theta} (t) = \theta (t)$ for $t >
    a$, and since $\lim_{t \to \infty} \theta (t) = \infty$ (which follows
    from the growth properties of the theta function), $\lim_{t \to \infty} 
    \tilde{\theta} (t) = \infty$
  \end{enumerate}
  Therefore, $\tilde{\theta} (t)$ satisfies all criteria for a modulating
  function.
\end{proof}

\end{document}
