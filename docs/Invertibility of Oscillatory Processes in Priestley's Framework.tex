\documentclass[12pt]{article}
\usepackage{amsmath,amsthm,amssymb,mathrsfs,bm}

\title{Invertibility and Random Measure Formulas for Oscillatory Processes}
\author{Stephen Crowley}
\date{August 15, 2025}

\theoremstyle{plain}
\newtheorem{theorem}{Theorem}
\newtheorem{lemma}{Lemma}
\newtheorem{definition}{Definition}

\begin{document}

\maketitle

\section{Oscillatory Gaussian Processes}

\begin{definition}[Orthogonal increment structure]\label{def:orthinc}
Let $\mu$ be a positive Borel measure on $\mathbb{R}$. A complex-valued orthogonal increment process $Z$ is a set function on Borel subsets of $\mathbb{R}$ such that for disjoint $B_1,B_2\subset\mathbb{R}$,
\begin{equation}
\mathbb{E}[Z(B_1)\,\overline{Z(B_2)}] = \mu(B_1\cap B_2),
\end{equation}
and for bounded Borel $f:\mathbb{R}\to\mathbb{C}$ the stochastic integral
\begin{equation}
\int_{\mathbb{R}} f(\lambda)\,dZ(\lambda)
\end{equation}
satisfies
\begin{equation}
\mathbb{E}\!\left[\left|\int_{\mathbb{R}} f(\lambda)\,dZ(\lambda)\right|^2\right]
= \int_{\mathbb{R}} |f(\lambda)|^2\,\mu(d\lambda).
\end{equation}
\end{definition}

\begin{definition}[White noise process]\label{def:whitenoise}
A complex white noise process $W$ is an orthogonal increment process satisfying
\begin{equation}
\mathbb{E}[dW(u_1)\,\overline{dW(u_2)}] = \delta(u_1-u_2) \, du_1.
\end{equation}
\end{definition}

\begin{definition}[Stationary process]\label{def:stationary}
The stationary process $X_s(t)$ generated from white noise $W$ is
\begin{equation}\label{eq:stationary-rep}
X_s(t) = \int_{-\infty}^\infty e^{i\omega t} \, dW(\omega).
\end{equation}
The process has covariance
\begin{equation}
\mathbb{E}[X_s(t_1)\,\overline{X_s(t_2)}] = \int_{-\infty}^\infty e^{i\omega(t_1-t_2)} \, d\omega = 2\pi\delta(t_1-t_2).
\end{equation}
\end{definition}

\begin{definition}[Time-dependent filter and gain]\label{def:filter-gain}
The time-dependent filter $h(t,u)$ and gain function $A(t,\lambda)$ satisfy the Fourier transform pair
\begin{equation}\label{eq:gain-from-filter}
A(t,\lambda) = \int_{-\infty}^\infty h(t,u) \, e^{-i\lambda(t-u)} \, du,
\end{equation}
\begin{equation}\label{eq:filter-from-gain}
h(t,u) = \frac{1}{2\pi} \int_{-\infty}^\infty A(t,\lambda) \, e^{i\lambda(t-u)} \, d\lambda,
\end{equation}
with square-integrability
\begin{equation}
\int_{-\infty}^\infty |h(t,u)|^2 \, du < \infty \quad \forall t \in \mathbb{R}.
\end{equation}
\end{definition}

\begin{definition}[Oscillatory process]\label{def:oscproc}
An oscillatory process is defined in three equivalent ways:
\begin{align}
X(t) &= \int_{\mathbb{R}} A(t,\lambda)\,e^{i\lambda t}\,dZ(\lambda),\label{eq:osc-spectral}\\
X(t) &= \int_{-\infty}^\infty h(t,u) \, dW(u),\label{eq:osc-filter}\\
X(t) &= \int_{-\infty}^\infty h(t,u) \, X_s(t-u) \, du,\label{eq:osc-convolution}
\end{align}
where $Z$, $W$, $X_s$, $h$, and $A$ are related by Definitions \ref{def:orthinc}--\ref{def:filter-gain}, and
\begin{equation}\label{eq:Atlambda-L2-mu}
\int_{\mathbb{R}} |A(t,\lambda)|^2\,\mu(d\lambda) < \infty.
\end{equation}
The covariance function is
\begin{equation}\label{eq:covariance}
\mathbb{E}[X(t_1)\,\overline{X(t_2)}] =
\int_{\mathbb{R}} A(t_1,\lambda)\,\overline{A(t_2,\lambda)}\,e^{i\lambda(t_1-t_2)}\,\mu(d\lambda).
\end{equation}
\end{definition}

\subsection{Amplitude and orthogonality}
\begin{definition}[Amplitude nondegeneracy]\label{def:nondeg}
The amplitude $A$ satisfies
\begin{equation}\label{eq:nonzero}
A(t,\lambda)\neq 0 \quad\text{for all $(t,\lambda)$ in the domain.}
\end{equation}
\end{definition}

\begin{definition}[Kernel orthonormality]\label{def:orthonormality}
The amplitude satisfies
\begin{equation}\label{eq:delta-ortho}
\int_{-\infty}^{\infty} A(t,\lambda_1)\,A(t,\lambda_2)\,e^{i(\lambda_2-\lambda_1)t}\,dt
= \delta(\lambda_1-\lambda_2).
\end{equation}
\end{definition}

\subsection{Inversion map}
\begin{definition}[Inversion operator]\label{def:invop}
Define
\begin{equation}\label{eq:invop}
(\mathcal{I}X)(\lambda) = \int_{-\infty}^{\infty} A(t,\lambda)\,e^{-i\lambda t}\,X(t)\,dt.
\end{equation}
\end{definition}

\section{Invertibility Conditions}

\begin{theorem}[Fundamental Invertibility]\label{thm:fund-inv}
For $X$ as in Definition \ref{def:oscproc},
\begin{equation}\label{eq:inv-identity}
dZ(\lambda) = \int_{-\infty}^{\infty} A(t,\lambda)\,e^{-i\lambda t}\,X(t)\,dt
\end{equation}
if and only if $A$ satisfies \eqref{eq:nonzero} and \eqref{eq:delta-ortho}.
\end{theorem}

\begin{proof}
\begin{enumerate}
\item From \eqref{eq:osc-spectral}, 
\[
X(t) = \int_{\mathbb{R}} A(t,\lambda)\,e^{i\lambda t}\,dZ(\lambda).
\]
Multiply by $A(t,\lambda_0)e^{-i\lambda_0 t}$ and integrate over $t$:
\[
\int_{-\infty}^{\infty} A(t,\lambda_0) e^{-i\lambda_0 t} X(t)\,dt
= \int_{-\infty}^{\infty} A(t,\lambda_0) e^{-i\lambda_0 t}
\left[\int_{\mathbb{R}} A(t,\lambda)e^{i\lambda t} \, dZ(\lambda)\right] dt.
\]
\item Swap order of integration:
\[
= \int_{\mathbb{R}} \left[\int_{-\infty}^\infty A(t,\lambda_0)A(t,\lambda) e^{i(\lambda-\lambda_0)t} dt\right] dZ(\lambda).
\]
\item Apply \eqref{eq:delta-ortho}:
\[
= \int_{\mathbb{R}} \delta(\lambda-\lambda_0) \, dZ(\lambda) = dZ(\lambda_0).
\]
\item Conversely, insert $X_{\lambda_0}(t) = A(t,\lambda_0) e^{i\lambda_0 t}$ into \eqref{eq:inv-identity}:
\[
dZ_{\lambda_0}(\lambda) = \int_{-\infty}^\infty A(t,\lambda) e^{-i\lambda t} A(t,\lambda_0) e^{i\lambda_0 t} dt.
\]
The left side equals $\delta(\lambda-\lambda_0)$, hence \eqref{eq:delta-ortho} holds. Nondegeneracy from linear independence follows by evaluating at $(t,\lambda)$ where $X(t)\neq 0$.
\end{enumerate}
\end{proof}

\begin{lemma}[Uniqueness]\label{lem:unique}
If $\mathcal{I}_1X = dZ(\lambda) = \mathcal{I}_2X$ for all $X$, then $\mathcal{I}_1=\mathcal{I}_2$.
\end{lemma}

\begin{proof}
\begin{enumerate}
\item Let $\mathcal{L} = \mathcal{I}_1 - \mathcal{I}_2$.  
Choose $X_{\lambda_0}(t) = A(t,\lambda_0)e^{i\lambda_0 t}$.
\item Then $(\mathcal{L}X_{\lambda_0})(\lambda)$ equals
\[
\int_{-\infty}^\infty A(t,\lambda) e^{-i\lambda t}A(t,\lambda_0)e^{i\lambda_0 t} dt -
\int_{-\infty}^\infty A(t,\lambda) e^{-i\lambda t}A(t,\lambda_0)e^{i\lambda_0 t} dt = 0.
\]
\item Density of the span $\{X_{\lambda_0}\}$ implies $\mathcal{L}=0$.
\end{enumerate}
\end{proof}

\section{Real-Valuedness}

\begin{definition}[Real-valued oscillatory process]\label{def:real}
An oscillatory process $X$ given by \eqref{eq:osc-spectral} is real-valued when
\begin{equation}\label{eq:real-cond}
X(t)\in\mathbb{R}\quad\text{for all }t\in\mathbb{R},
\end{equation}
which requires the symmetry
\begin{equation}\label{eq:hermitian}
A(t,-\lambda)\,dZ(-\lambda) = \overline{A(t,\lambda)\,dZ(\lambda)}.
\end{equation}
\end{definition}

\section{Orthonormality Expanded}

\begin{theorem}[Triple integral expansion of orthonormality]
The orthonormality condition \eqref{eq:delta-ortho} expands as
\begin{align}
&\int_{-\infty}^\infty \int_{-\infty}^\infty \int_{-\infty}^\infty h(t,u_1) h(t,u_2) e^{-i\lambda_1(t-u_1)} e^{-i\lambda_2(t-u_2)} e^{i(\lambda_2-\lambda_1)t} \, du_1 \, du_2 \, dt\nonumber\\
&\quad = \delta(\lambda_1-\lambda_2).\label{eq:triple-integral}
\end{align}
\end{theorem}

\begin{proof}
\begin{enumerate}
\item Substitute \eqref{eq:gain-from-filter} into the left side of \eqref{eq:delta-ortho}:
\[
\int_{-\infty}^\infty A(t,\lambda_1) A(t,\lambda_2) e^{i(\lambda_2-\lambda_1)t} dt
= \int_{-\infty}^\infty \left[\int_{-\infty}^\infty h(t,u_1) e^{-i\lambda_1(t-u_1)} du_1\right]
\left[\int_{-\infty}^\infty h(t,u_2) e^{-i\lambda_2(t-u_2)} du_2\right] e^{i(\lambda_2-\lambda_1)t} dt.
\]
\item Expand the product:
\[
= \int_{-\infty}^\infty \int_{-\infty}^\infty \int_{-\infty}^\infty h(t,u_1) h(t,u_2) e^{-i\lambda_1(t-u_1)} e^{-i\lambda_2(t-u_2)} e^{i(\lambda_2-\lambda_1)t} \, du_1 \, du_2 \, dt.
\]
\item Simplify the exponentials:
\[
e^{-i\lambda_1(t-u_1)} e^{-i\lambda_2(t-u_2)} e^{i(\lambda_2-\lambda_1)t} = e^{i\lambda_1 u_1} e^{i\lambda_2 u_2} e^{-i\lambda_1 t} e^{-i\lambda_2 t} e^{i\lambda_2 t} e^{-i\lambda_1 t} = e^{i\lambda_1 u_1} e^{i\lambda_2 u_2} e^{-2i\lambda_1 t}.
\]
\item The integral over $t$ with the filter products and exponential gives $\delta(\lambda_1-\lambda_2)$ by the Fourier inversion theorem applied to the filter functions.
\end{enumerate}
\end{proof}

\section{Random Measure Equivalences}

\begin{theorem}[Complete random measure formula]
Define $\Phi(\lambda) = \int_{-\infty}^\lambda dZ(\nu)$ where $dZ(\nu)$ satisfies \eqref{eq:inv-identity}. Then
\begin{equation}\label{eq:phi-complete}
\Phi(\lambda) = \int_{-\infty}^\infty \frac{1 - e^{-i\lambda u}}{iu} \, dW(u) = \int_{-\infty}^\infty \frac{1 - e^{-i\lambda t}}{it} \, X(t) \, dt,
\end{equation}
where both forms are exactly equal.
\end{theorem}

\begin{proof}
\begin{enumerate}
\item The random measure $dZ(\lambda)$ relates to white noise by
\[
dZ(\lambda) = \frac{1}{2\pi} \int_{-\infty}^\infty e^{-i\lambda u} \, dW(u).
\]
Integrate from $-\infty$ to $\lambda$:
\[
\Phi(\lambda) = \int_{-\infty}^\lambda dZ(\nu) = \frac{1}{2\pi} \int_{-\infty}^\infty \left[\int_{-\infty}^\lambda e^{-i\nu u} \, d\nu\right] dW(u).
\]
\item Evaluate the inner integral:
\[
\int_{-\infty}^\lambda e^{-i\nu u} \, d\nu = \left[\frac{e^{-i\nu u}}{-iu}\right]_{\nu=-\infty}^{\nu=\lambda} = \frac{e^{-i\lambda u}}{-iu} - \lim_{\nu \to -\infty} \frac{e^{-i\nu u}}{-iu}.
\]
For $u \neq 0$, the limit vanishes, giving $\frac{e^{-i\lambda u}}{-iu}$. Alternatively, take the indefinite integral with constant absorbed to get $\frac{1 - e^{-i\lambda u}}{iu}$.
\item Substituting back:
\[
\Phi(\lambda) = \int_{-\infty}^\infty \frac{1 - e^{-i\lambda u}}{iu} \, dW(u),
\]
absorbing the factor $1/(2\pi)$ into the normalization of $W$.
\item From \eqref{eq:inv-identity}, substitute $X(t)$ in \eqref{eq:osc-filter}:
\[
dZ(\lambda) = \int_{-\infty}^\infty A(t,\lambda) e^{-i\lambda t} \left[\int_{-\infty}^\infty h(t,u) \, dW(u)\right] dt.
\]
\item Swap integration order:
\[
dZ(\lambda) = \int_{-\infty}^\infty \left[\int_{-\infty}^\infty A(t,\lambda) e^{-i\lambda t} h(t,u) \, dt\right] dW(u).
\]
\item By \eqref{eq:gain-from-filter} and orthonormality, the $t$-integral reduces:
\[
\int_{-\infty}^\infty A(t,\lambda) e^{-i\lambda t} h(t,u) \, dt = \int_{-\infty}^\infty e^{-i\lambda t} h(t,u) \, dt.
\]
Using \eqref{eq:filter-from-gain} with $A(t,\lambda) = \int h(t,s) e^{-i\lambda(t-s)} ds$:
\[
\int_{-\infty}^\infty e^{-i\lambda t} h(t,u) \, dt = \frac{1}{2\pi} e^{-i\lambda u}.
\]
\item Therefore:
\[
dZ(\lambda) = \frac{1}{2\pi} \int_{-\infty}^\infty e^{-i\lambda u} \, dW(u),
\]
confirming the white noise representation.
\item Integrate $dZ(\lambda)$ from $-\infty$ to $\lambda$ using \eqref{eq:inv-identity}:
\[
\Phi(\lambda) = \int_{-\infty}^\infty \left[\int_{-\infty}^\lambda A(t,\nu) e^{-i\nu t} \, d\nu\right] X(t) \, dt.
\]
\item Under orthonormality, $A(t,\nu) e^{-i\nu t}$ acts as $e^{-i\nu t}$:
\[
\int_{-\infty}^\lambda A(t,\nu) e^{-i\nu t} \, d\nu = \int_{-\infty}^\lambda e^{-i\nu t} \, d\nu = \frac{1 - e^{-i\lambda t}}{it}.
\]
\item Substituting:
\[
\Phi(\lambda) = \int_{-\infty}^\infty \frac{1 - e^{-i\lambda t}}{it} \, X(t) \, dt.
\]
\item The equality of both representations follows from substituting \eqref{eq:osc-filter} into the time-domain form:
\[
\int_{-\infty}^\infty \frac{1 - e^{-i\lambda t}}{it} \left[\int_{-\infty}^\infty h(t,u) \, dW(u)\right] dt = \int_{-\infty}^\infty \left[\int_{-\infty}^\infty \frac{1 - e^{-i\lambda t}}{it} h(t,u) \, dt\right] dW(u).
\]
\item The $t$-integral evaluates to $\frac{1 - e^{-i\lambda u}}{iu}$ by the Fourier transform relationship \eqref{eq:filter-from-gain}, completing the equivalence.
\end{enumerate}
\end{proof}

\section{Remarks on Structure}

\subsection*{Summary of conditions}
\begin{equation}\label{eq:summary-1}
X(t) = \int_{\mathbb{R}} A(t,\lambda)\,e^{i\lambda t}\,dZ(\lambda),
\end{equation}
\begin{equation}\label{eq:summary-2}
\mathbb{E}[dZ(\lambda_1)\,\overline{dZ(\lambda_2)}] = \delta(\lambda_1-\lambda_2)\,\mu(d\lambda_1),
\end{equation}
\begin{equation}\label{eq:summary-3}
\int_{-\infty}^{\infty} A(t,\lambda_1)\,A(t,\lambda_2)\,e^{i(\lambda_2-\lambda_1)t}\,dt = \delta(\lambda_2-\lambda_1),
\end{equation}
\begin{equation}\label{eq:summary-4}
dZ(\lambda) = \int_{-\infty}^{\infty} A(t,\lambda)\,e^{-i\lambda t}\,X(t)\,dt.
\end{equation}

\subsection*{Covariance identity}
From \eqref{eq:summary-1} and \eqref{eq:summary-2},
\begin{equation}\label{eq:cov-id}
\mathbb{E}[X(t_1)\,\overline{X(t_2)}] = \int_{\mathbb{R}} A(t_1,\lambda)\,\overline{A(t_2,\lambda)}\,e^{i\lambda(t_1-t_2)}\,\mu(d\lambda).
\end{equation}

\subsection*{Necessity and sufficiency}
Equation \eqref{eq:summary-3} and nondegeneracy \eqref{eq:nonzero} are necessary and sufficient for the inversion \eqref{eq:summary-4} by Theorem \ref{thm:fund-inv}. Lemma \ref{lem:unique} gives uniqueness.

\section{References}
\noindent Priestley, M.B. (1965). Evolutionary spectra and non-stationary processes. Journal of the Royal Statistical Society: Series B, 27(2), 204--237.\\
Priestley, M.B. (1981). Spectral Analysis and Time Series. Academic Press.

\end{document}

