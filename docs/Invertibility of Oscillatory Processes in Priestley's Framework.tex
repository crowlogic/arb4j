\documentclass{article}
\usepackage[english]{babel}
\usepackage{geometry,amsmath,latexsym}
\geometry{letterpaper}

%%%%%%%%%% Start TeXmacs macros
\newcommand{\mathd}{\mathrm{d}}
\newcommand{\tmtextit}[1]{\text{{\itshape{#1}}}}
\newenvironment{proof}{\noindent\textbf{Proof\ }}{\hspace*{\fill}$\Box$\medskip}
\newtheorem{definition}{Definition}
\newtheorem{lemma}{Lemma}
\newtheorem{theorem}{Theorem}
%%%%%%%%%% End TeXmacs macros

\begin{document}

\title{Invertibility of Oscillatory Processes in Priestley's Framework}

\author{Stephen Crowley}

\date{}

\maketitle

{\tableofcontents}

\section{Oscillatory Gaussian Processes}

\begin{definition}
  An oscillatory process $X (t)$ in Priestley's sense has the integral
  representation
  \begin{equation}
    X (t) = \int_{\infty}^{\infty} A (t, \lambda) e^{i \lambda t} 
    \hspace{0.17em} dZ (\lambda)
  \end{equation}
  where $A (t, \lambda)$ is the time-varying amplitude function and $dZ
  (\lambda)$ is an orthogonal increment process with
  \begin{equation}
    E [dZ (\lambda_1) \overline{dZ (\lambda_2)}] = \delta (\lambda_1 -
    \lambda_2) \mu (d \lambda_1)
  \end{equation}
  for some measure $\mu$.
\end{definition}

\subsection{Invertibility Conditions}

\begin{theorem}
  [Fundamental Invertibility Theorem] The oscillatory process $X (t)$ with
  amplitude $A (t, \lambda)$ allows the expression of the associated complex
  orthogonal random measure
  \begin{equation}
    dZ (\lambda) = \int_{- \infty}^{\infty} \overline{A (t, \lambda)} e^{- i
    \lambda t} X (t)  \hspace{0.17em} dt
  \end{equation}
  from a sample path realization $X (t)$ if and only if:
  \begin{enumerate}
    \item $A (t, \lambda) \neq 0 \forall (t, \lambda)$ in the relevant domain
    
    \
    
    and
    
    \
    
    \item The orthogonality condition holds:
    \begin{equation}
      \int_{- \infty}^{\infty} \overline{A (t, \lambda_1)} A (t, \lambda_2)
      e^{i (\lambda_2 - \lambda_1) t}  \hspace{0.17em} dt = \delta (\lambda_1
      - \lambda_2) \label{ortho}
    \end{equation}
  \end{enumerate}
\end{theorem}

\begin{proof}
  From the representation
  \begin{equation}
    X (t) = \int_{- \infty}^{\infty} A (t, \lambda) e^{i \lambda t} 
    \hspace{0.17em} dZ (\lambda)
  \end{equation}
  , one seeks to obtain the expression for $dZ (\lambda)$. The orthogonality
  condition (\ref{ortho}) ensures that the kernel functions form an
  orthonormal system. This allows the projection of $X (t)$ onto each
  frequency component.Multiply both sides by $\overline{A (t, \lambda_0)} e^{-
  i \lambda_0 t}$ and integrate over $t$
  \begin{equation}
    \begin{array}{ll}
      \int_{- \infty}^{\infty} \overline{A (t, \lambda_0)} e^{- i \lambda_0 t}
      X (t)  \hspace{0.17em} dt & = \int_{- \infty}^{\infty} \overline{A (t,
      \lambda_0)} e^{- i \lambda_0 t}  \int_{- \pi}^{\pi} A (t, \lambda) e^{i
      \lambda t}  \hspace{0.17em} dZ (\lambda)  \hspace{0.17em} dt\\
      & = \int_{- \infty}^{\infty} \left[ \int_{- \infty}^{\infty}
      \overline{A (t, \lambda_0)} A (t, \lambda) e^{i (\lambda - \lambda_0) t}
      \hspace{0.17em} dt \right] dZ (\lambda)\\
      & = \int_{- \infty}^{\infty} \delta (\lambda - \lambda_0) dZ
      (\lambda)\\
      & = \mathd Z (\lambda_0)
    \end{array} \label{inv}
  \end{equation}
  where the second-to-last equality is due to
  \begin{equation}
    \int_{- \infty}^{\infty} \overline{A (t, \lambda_0)} A (t, \lambda) e^{i
    (\lambda - \lambda_0) t}  \hspace{0.17em} dt = \delta (\lambda -
    \lambda_0)
  \end{equation}
  which yields $dZ (\lambda_0)$ after application of the elementary Dirac
  delta function identity.
\end{proof}

\begin{lemma}
  [Uniqueness of Inversion] The inversion formula (\ref{inv}) is unique under
  the given conditions.
\end{lemma}

\begin{proof}
  Suppose two different inversion operators both recover $dZ (\lambda)$ from
  $X (t)$. Then their difference must annihilate all possible $X (t)$ while
  producing zero output, which implies they are identical by the
  non-degeneracy condition $A (t, \lambda) \neq 0$.
\end{proof}

\section{References}

\

Priestley, M.B. (1965). Evolutionary spectra and non-stationary processes.
\tmtextit{Journal of the Royal Statistical Society: Series B}, 27(2), 204-237.

\

Priestley, M.B. (1981). \tmtextit{Spectral Analysis and Time Series}.
Academic Press.

\

\end{document}
