\documentclass{article}
\usepackage{amsmath}
\usepackage{amssymb}
\usepackage{amsthm}

\newtheorem{theorem}{Theorem}
\newtheorem{definition}{Definition}

\begin{document}

\title{Eigenfunction Expansion for Translation-Invariant Kernels via Galerkin Method}
\author{}
\date{}

\maketitle

\begin{definition}
For a translation-invariant kernel $K(x-y)$ on $\mathbb{R}^d$, its Gram matrix $A$ with respect to a uniformly convergent orthonormal basis $\{\psi_j\}_{j=1}^{\infty}$ is:

\[A_{ij} = \int_{\mathbb{R}^d} \psi_i(x-y)\psi_j(y)dy\]
\end{definition}

\begin{theorem}
The Gram matrix $A$ can be expressed in terms of Fourier transforms:

\[A_{ij} = \mathcal{F}^{-1}[\mathcal{F}[\psi_i]^* \cdot \mathcal{F}[\psi_j]]\]

where $\mathcal{F}$ denotes the Fourier transform, $\mathcal{F}^{-1}$ the inverse Fourier transform, and $^*$ the complex conjugate.
\end{theorem}

\begin{proof}
By the convolution theorem and Parseval's identity:

\begin{align*}
A_{ij} &= \int_{\mathbb{R}^d} \psi_i(x-y)\psi_j(y)dy \\
&= (\psi_i * \psi_j)(x) \\
&= \mathcal{F}^{-1}[\mathcal{F}[\psi_i] \cdot \mathcal{F}[\psi_j]] \\
&= \mathcal{F}^{-1}[\mathcal{F}[\psi_i]^* \cdot \mathcal{F}[\psi_j]]
\end{align*}

The last step follows from the fact that $\psi_i$ is real-valued, so $\mathcal{F}[\psi_i] = \mathcal{F}[\psi_i]^*$.
\end{proof}

\begin{theorem}
For a kernel $K(x-y) = \sum_{j=1}^{\infty} a_j \psi_j(x-y)$, the eigenfunctions $\phi_k$ and their corresponding eigenvalues $\lambda_k$ are given by:

\[\phi_k(x) = \sum_{j=1}^{\infty} b_{kj} \psi_j(x)\]

where the coefficients $b_{kj}$ satisfy:

\[\sum_{j=1}^{\infty} a_i A_{ij} b_{kj} = \lambda_k b_{ki} \quad \text{for all } i\]
\end{theorem}

\begin{proof}
Let $\phi_k(x) = \sum_{j=1}^{\infty} b_{kj} \psi_j(x)$ be an eigenfunction of $K$. Then:

\begin{align*}
\lambda_k \phi_k(x) &= \int K(x-y)\phi_k(y)dy \\
&= \int \left(\sum_{i=1}^{\infty} a_i \psi_i(x-y)\right) \left(\sum_{j=1}^{\infty} b_{kj} \psi_j(y)\right) dy \\
&= \sum_{i=1}^{\infty} \sum_{j=1}^{\infty} a_i b_{kj} \int \psi_i(x-y)\psi_j(y)dy \\
&= \sum_{i=1}^{\infty} \sum_{j=1}^{\infty} a_i b_{kj} A_{ij} \\
&= \sum_{i=1}^{\infty} a_i \left(\sum_{j=1}^{\infty} A_{ij} b_{kj}\right)
\end{align*}

Equating coefficients of $\psi_i(x)$ on both sides:

\[\lambda_k b_{ki} = a_i \sum_{j=1}^{\infty} A_{ij} b_{kj}\]

This is equivalent to the equation:

\[\sum_{j=1}^{\infty} a_i A_{ij} b_{kj} = \lambda_k b_{ki} \quad \text{for all } i\]

Thus, the eigenfunctions are given by the solutions of this equation system.
\end{proof}

\begin{theorem}
The nth eigenfunction $\phi_n(x)$ of the kernel $K(x-y) = \sum_{j=1}^{\infty} a_j \psi_j(x-y)$ is given by:

\[\phi_n(x) = \sum_{j=1}^{\infty} b_{nj} \psi_j(x)\]

where the coefficients $b_{nj}$ satisfy:

\[\sum_{j=1}^{\infty} a_i A_{ij} b_{nj} = \lambda_n b_{ni} \quad \text{for all } i\]
\end{theorem}

\end{document}
