\documentclass{article}
\usepackage{amsmath}

\begin{document}

\section*{From Lagrangian to Hamiltonian Formulation}

\begin{enumerate}
    \item \textbf{Start with the Lagrangian:}
    \[
    L = L(q, \dot{q}, t)
    \]

    \item \textbf{Define the Conjugate Momenta:}
    \[
    p_i = \frac{\partial L}{\partial \dot{q}_i}
    \]

    \item \textbf{Perform the Legendre Transformation:}
    \[
    H = \sum_i p_i \dot{q}_i - L
    \]
    Here, \( \dot{q}_i \) should be expressed in terms of \( p_i \) and \( q_i \), making \( H \) a function of \( p \), \( q \), and possibly \( t \):
    \[
    H = H(q, p, t)
    \]

    \item \textbf{Hamiltonian Formulation:}
    In the Hamiltonian formulation, the equations of motion are derived from Hamilton's equations:
    \[
    \dot{q}_i = \frac{\partial H}{\partial p_i}, \quad \dot{p}_i = -\frac{\partial H}{\partial q_i}
    \]
\end{enumerate}

\section*{From Hamiltonian to Lagrangian Formulation}

\begin{enumerate}
    \item \textbf{Start with the Hamiltonian:}
    \[
    H = H(q, p, t)
    \]

    \item \textbf{Express the velocities:}
    Use Hamilton's equations to express the velocities \( \dot{q}_i \) in terms of \( p \) and \( q \):
    \[
    \dot{q}_i = \frac{\partial H}{\partial p_i}
    \]

    \item \textbf{Invert to find \( \dot{q} \) as functions of \( q \) and \( p \):}
    If possible, solve the expressions from Hamilton's equations to find \( \dot{q}_i \) as functions of \( q_i \) and \( p_i \).

    \item \textbf{Perform the inverse Legendre Transformation:}
    Compute the Lagrangian by inverting the Legendre transformation:
    \[
    L = \sum_i p_i \dot{q}_i - H
    \]
    where \( H \) and \( \dot{q}_i \) are expressed in terms of \( p_i \) and \( q_i \), recovering the Lagrangian description.
\end{enumerate}

\end{document}
