\documentclass{article}
\usepackage{amsmath, amssymb} % For math symbols like \mathbb, \mathscr

\begin{document}

\title{
On a Class of Asymptotically Stationary Harmonizable Processes
}

\author{
Dominique Dehay \\ Laboratoire de Statistique et Probabilités (M.2), Université des Sciences et Techniques de Lille Flandres Artois Cité Scientifique, 59655-Villeneuve D'Ascq Cedex, France \\ Communicated by P. R. Krishnaiah
}

\date{} % Suppress date printing if not specified or irrelevant

\maketitle % Place title and author

\begin{abstract}
We prove that every harmonizable process with \(\sigma\)-finite bimeasure is asymptotically stationary and we give its associated spectral measure. © 1987 Academic Press, Inc.
\end{abstract}

\section*{I. Introduction}

For stochastic processes, various extensions of the notion of stationarity have been developed such as asymptotic stationarity and harmonizability, which are related notions. For example, Rozanov [12] established that every strongly harmonizable process is asymptotically stationary.

In Section 2, we introduce a larger class of asymptotically stationary harmonizable processes, i.e., harmonizable processes which have \(\sigma\)-finite bimeasure, and we prove that they are uniform limits of a sequence of strongly harmonizable ones.

In Section 3, we show that these processes are indeed asymptotically stationary, and we exhibit the associated spectral measure using a stationary dilation of the harmonizable process under consideration [10].

% Metadata block - placement might vary depending on journal style
{
\small % Smaller font for metadata
\par\noindent % Ensure it starts on a new line
Received August 11, 1986; revised January 2, 1987. \\
AMS 1980 subject classifications: \(60 \mathrm{G} 10, 60 \mathrm{G} 12\). \\
Key words and phrases: harmonizable processes, asymptotic stationarity, spectral measures.
\par
}
\vspace{\baselineskip} % Add some space after metadata

\section*{5. Preliminaries} % Note: The section number is 5 in the source

Following Rozanov [12] (see also [1,6]), a process \(X: \mathbb{R} \rightarrow L_{\mathbb{C}}^{2}(S, \mathscr{F}, P)\) is said to be asymptotically stationary if there exists a continuous function \(r: \mathbb{R} \rightarrow \mathbb{C}\), such that for any \(h\) in \(\mathbb{R}\)
\[
r(h)=\lim _{t \rightarrow+\infty} \frac{1}{t} \int_{0}^{t} E(X(s+h) \cdot \overline{X(s)}) d s .
\]

In this case there exists a unique positive bounded measure \(m\) on \(\mathscr{B}(\mathbb{R})\), called the associated spectral measure of \(X\), which verifies for any \(h\) in \(\mathbb{R}\) :
\[
r(h)=\int e^{i h u} m(d u) .
\]

We recall that every weakly harmonizable process \(X: \mathbb{R} \rightarrow L_{\mathbb{C}}^{2}(S, \mathscr{F}, P)\) is the Fourier transform of a stochastic measure \(\mu: \mathscr{B}(\mathbb{R}) \rightarrow L_{\mathbb{C}}^{2}(S, \mathscr{F}, P)[8\), 11, 12], i.e., for any \(t\) in \(\mathbb{R}\) :
\[
X(t)=\int e^{i t u} \mu(d u)
\]

When the spectral bimeasure \(M\) of \(X\), defined on \(\mathscr{B}(\mathbb{R}) \times \mathscr{B}(\mathbb{R})\) by \(M(A, B)=E(\mu(A) \cdot \overline{\mu(B)})\), is extendable to a measure on \(\mathscr{B}\left(\mathbb{R}^{2}\right)\), the process is termed strongly harmonizable.

In this paper we use the concept of integration with respect to a spectral bimeasure as introduced by Moché [8, Chap. IV].
Rozanov has proved that every strongly harmonizable process is asymptotically stationary and, more precisely, one can establish the following:
\newtheorem{proposition}{Proposition}[section] % Define proposition environment if needed, numbering within section
\begin{proposition} % Assuming 2.1 is a Proposition based on context
Let \(X\) be a strongly harmonizable process with spectral measure \(M\), and let \(\Delta=\{(u, v) \mid u=v\}\) be the diagonal axis of \(\mathbb{R}^{2}\). Then uniformly with respect to \(h\) in \(\mathbb{R}\), we have:
\[
\lim _{t \rightarrow+\infty} \frac{1}{t} \int_{0}^{t} E(X(s+h) \cdot \overline{X(s)}) d s=\iint_{\Delta} e^{i h v} M(d u, d v)
\]
\end{proposition}

So in the weakly harmonizable case, one of the problems is: How can we define the restriction on the diagonal axis \(\Delta\) of the bimeasure \(M\) as a measure on \(\mathscr{B}(\mathbb{R})\) ?
\newtheorem{definition}{Definition}[section] % Define definition environment
\begin{definition} % Assuming 2.2 is a Definition
A spectral bimeasure \(M\) is said to be \(\sigma\)-finite if there exists a sequence \(\left(B_{n}\right)_{n \in \mathbb{N}}\) in \(\mathscr{B}(\mathbb{R})\) which verifies:
\begin{enumerate}
    \item[(1)] for any \(n \in \mathbb{N}, B_{n} \subset B_{n+1}\) ; and \(\bigcup_{n \in \mathbb{N}} B_{n}=\mathbb{R}\) ;
    \item[(2)] for any \(n, M\) has finite Vitali variation on \(B_{n} \times B_{n}\).
\end{enumerate}
\end{definition}

\newtheorem{example}{Example}[section] % Define example environment
\begin{example} % Assuming 2.3 is an Example
(a) Obviously, the spectral bimeasure of every strongly harmonizable process is \(\sigma\)-finite.
(b) Here is an example of weakly harmonizable process which is not strongly harmonizable. It is due to Niemi [9] following Edwards [5] (see also [2]).

Let us consider the positive definite family of real numbers defined by
\[
\begin{array}{ll}
c_{j j}=\frac{\pi}{2 j(\log (j+1))^{2}}, & j \in \mathbb{N} \setminus \{0\} \\
c_{j k}=\frac{\sin (\pi(j-k) / 2)}{(j-k) j^{1 / 2} k^{1 / 2} \log (j+1) \log (k+1)}, & j \neq k ; j, k \in \mathbb{N} \setminus \{0\} .
\end{array}
\]

Then there exist a probability space ( \(S, \mathscr{F}, P\) ) and a sequence \(\left(x_{j}\right)\) in \(L_{\mathbb{R}}^{2}(S, \mathscr{F}, P)\) such that \(E\left(x_{j} \cdot x_{k}\right)=c_{j k}\). We can use this sequence to define a stochastic measure \(\mu: \mathscr{B}(\mathbb{R}) \rightarrow L_{\mathbb{R}}^{2}(S, \mathscr{F}, P)\) by \(\mu(B)=\sum_{j \in B} x_{j}\), for every Borel set \(B\) of \(\mathbb{R}\).

Since \(\sum_{j} \sum_{k}\left|c_{j k}\right|=+\infty\), the Vitali variation of \(\mathbb{R}^{2}\) of its bimeasure \(M\) is infinite. Moreover, since \(\mu\) is discrete, \(M\) is obviously \(\sigma\)-finite. Therefore the Fourier transform of \(\mu\) has a \(\sigma\)-finite bimeasure but is not strongly harmonizable. So the class of harmonizable processes with \(\sigma\)-finite bimeasure contains strictly the class of strongly harmonizable ones.
\end{example}

\paragraph{2.4. Notations.} % Using paragraph for unnumbered subsection
Throughout the sequel, we consider a weakly harmonizable process \(X\) with \(\sigma\)-finite bimeasure \(M\), and spectral stochastic measure \(\mu\).

Let \(\left(B_{n}\right)_{n \in \mathbb{N}}\) be a sequence in \(\mathscr{B}(\mathbb{R})\) which satisfies (1) and (2) and for any \(n\) let \(\mu_{n}\) be the stochastic measure on \(\mathscr{B}(\mathbb{R})\) defined by \(\mu_{n}(B)=\mu\left(B \cap B_{n}\right)\), \(M_{n}\) be its spectral bimeasure which is of finite Vitali variation on \(\mathbb{R}^{2}\), and \(X_{n}\) be the associated strongly harmonizable process.... Niemi [9, Theorem 3.41] has proved that, for any weakly harmonizable process \(X\), there exists a sequence of strongly harmonizable processes which converges in q.m. to \(X\) uniformly on every compact subset of \(\mathbb{R}\). Recently, Moche and the author [3] showed that this property remains true if the process \(X\) is only continuous and bounded. Here we obtain another sharpening of Niemi's result.

\setcounter{proposition}{4} % Reset counter if needed, assuming 2.5 is the 5th proposition in section 2
\begin{proposition} % Assuming 2.5 is a Proposition
For every harmonizable process \(X\) with \(\sigma\)-finite bimeasure, there exists a bounded sequence of strongly harmonizable processes which converges in q.m. towards \(X\) uniformly on \(\mathbb{R}\).
\end{proposition}

\begin{proof}
With the previous notations, let \(B_{n}^{\prime}=\mathbb{R} \setminus B_{n}\) and let \(\|\mu\|\) be the semi-variation of the stochastic measure \(\mu\), [4, Definition IV.10.3]; from [4, Theorem IV.10.8] we estimate for every \(t\) :
\[
\begin{gathered}
E\left[\left|X_{n}(t)\right|^{2}\right] \leqslant\left(\|\mu\|\left(B_{n}\right)\right)^{2} \leqslant(\|\mu\|(\mathbb{R}))^{2} \\
E\left[\left|X(t)-X_{n}(t)\right|^{2}\right]=E\left(\left|\int_{B_{n}^{\prime}} e^{i t u} \mu(d u)\right|^{2}\right) \leqslant\left(\|\mu\|\left(B_{n}^{\prime}\right)\right)^{2}
\end{gathered}
\]
Since the sequence ( \(\left.B_{n}^{\prime}\right)_{n \in \mathbb{N}}\) decreases towards the empty set as \(n\) tends to infinity, then \(\|\mu\|\left(B_{n}^{\prime}\right)\) converges towards 0 [4; Lemma IV.10.5] and we can conclude that the bounded sequence \(\left(X_{n}\right)_{n \in \mathbb{N}}\) converges towards \(X\) in \(L_{\mathbb{C}}^{2}(S, \mathscr{F}, P)\) uniformly with respect to \(t\) on \(\mathbb{R}\).
\end{proof}

\section*{3. Main Result}
\newtheorem{theorem}{Theorem}[section] % Define theorem environment
\begin{theorem} % Assuming 3.1 is a Theorem
Every harmonizable process with \(\sigma\)-finite bimeasure is asymptotically stationary.
\end{theorem}

\begin{proof}
One can easily obtain that if a bounded sequence of asymptotically stationary processes ( \(X_{n}(t), t \in \mathbb{R}\) ) converges in q.m. towards a process \((X(t), t \in \mathbb{R})\) uniformly with respect to \(t\) in \(\mathbb{R}\), then the process \((X(t), t \in \mathbb{R})\) is asymptotically stationary. One can conclude using Proposition 2.5.

Now with a quite different proof, we are going to sharpen the previous result and to estimate the associated spectral measure of the harmonizable process under consideration.
\end{proof}

\begin{theorem} % Assuming 3.2 is a Theorem
For any harmonizable process with \(\sigma\)-finite bimeasure, uniformly with respect to \(h\) in \(\mathbb{R}\), we have
\[
\lim _{t \rightarrow+\infty} \frac{1}{t} \int_{0}^{t} E(X(s+h) \cdot \overline{X(s)}) d s=\int e^{i h u} m(d u),
\]
where the positive bounded measure \(m\) on \(\mathscr{B}(\mathbb{R})\) is defined by:
\[
\text { for every } B \text { in } \mathscr{B}(\mathbb{R}), \quad m(B)=\lim _{n \rightarrow+\infty} M_{n}((B \times B) \cap \Delta) .
\]
\end{theorem}

\begin{proof}
With Notations 2.4, let \(K(t, s)=E(X(t) \cdot \overline{X(s)})\) and \(K_{n}(t, s)=E\left(X_{n}(t) \cdot \overline{X_{n}(s)}\right)\). \\
(a) From Proposition 2.5, the sequence \(K_{n}(t, s)\) converges towards \(K(t, s)\) uniformly with respect to \((t, s)\) in \(\mathbb{R}^{2}\). So, given \(\varepsilon>0\), there exists \(N(\varepsilon)\) such that for \(n>N(\varepsilon)\) and for every \(t>0\) and every \(h\) we have
\begin{equation} \label{eq:a1} % Adding equation number for reference if needed internally
\left|\frac{1}{t} \int_{0}^{t} K(s+h, s) d s-\frac{1}{t} \int_{0}^{t} K_{n}(s+h, s) d s\right|<\varepsilon .
\end{equation}
Using the same notation for the spectral bimeasure \(M_{n}\) and its extension as a measure on \(\mathscr{B}\left(\mathbb{R}^{2}\right)\), we deduce from Proposition 2.1 that for every \(n\), there exists \(T(n, \varepsilon)\) such that for \(t>T(n, \varepsilon)\) and for every \(h\) one has:
\begin{equation} \label{eq:a2}
\left|\frac{1}{t} \int_{0}^{t} K_{n}(s+h, s) d s-\iint_{\Delta} e^{i u h} M_{n}(d u, d v)\right|<\varepsilon .
\end{equation}
Consequently for \(n>N(\varepsilon), t>T(n, \varepsilon)\) and for every \(h\) we obtain:
\begin{equation} \label{eq:a3}
\left|\frac{1}{t} \int_{0}^{t} K(s+h, s) d s-\iint_{\Delta} e^{i u h} M_{n}(d u, d v)\right|<2 \varepsilon .
\end{equation}
(b) We are going to prove that the sequence \(\left(m_{n}\right)\) of the restrictions on \(\Delta\) of the spectral measures ( \(M_{n}\) ) is convergent.

First of all, \(\left(m_{n}\right)\) is increasing since for any \(B\) in \(\mathscr{B}(\mathbb{R})\)
\[
m_{n}(B)=M_{n}((B \times B) \cap \Delta) = M((B \cap B_n) \times (B \cap B_n) \cap \Delta) % Clarifying the definition of m_n
\]
Let's re-evaluate the original text's argument: $m_n(B) = M_n((B \times B) \cap \Delta)$ and $m_{n+1}(B) = M_{n+1}((B \times B) \cap \Delta)$. Since $M_n(A, C) = M((A \cap B_n), (C \cap B_n))$ and $M_{n+1}(A, C) = M((A \cap B_{n+1}), (C \cap B_{n+1}))$. Also $B_n \subset B_{n+1}$. The measure $M$ restricted to the diagonal is positive. Let $m_{diag}$ be the measure $M$ restricted to the diagonal $\Delta$. Then $m_n(B) = m_{diag}(B \cap B_n)$ and $m_{n+1}(B) = m_{diag}(B \cap B_{n+1})$. Since $B_n \subset B_{n+1}$, $B \cap B_n \subset B \cap B_{n+1}$. Since $m_{diag}$ is a positive measure, $m_{diag}(B \cap B_n) \leq m_{diag}(B \cap B_{n+1})$, hence $m_n(B) \leq m_{n+1}(B)$.
\[
m_{n}(B) \leqslant m_{n+1}(B) .
\]
The only difficulty is to show that this sequence is bounded. Now Miamee and Salehi [7: Domination lemma] have proved that for every spectral bimeasure \(M\) on \(\mathscr{B}(\mathbb{R}) \times \mathscr{B}(\mathbb{R})\), there exists a positive bounded measure \(m_{d}\) on \(\mathscr{B}(\mathbb{R})\) such that for any bounded Borel function \(f: \mathbb{R} \rightarrow \mathbb{C}\) one has:
\[
0 \leqslant \iint f(t) \overline{f(s)} M(d t, d s) \leqslant \int|f(t)|^{2} m_{d}(d t) .
\]
So, for any Borel set \(B\) in \(\mathbb{R}\) we have:
\[
0 \leqslant M(B, B) \leqslant m_{d}(B) .
\]
Let us put \(I_{q}^{r}=\left(r / 2^{q},(r+1) / 2^{q}\right], r=\cdots-1,0,1, \ldots\) and \(q=0,1, \ldots\). Then, for any \(q\), the sets \(I_{q}^{r}, r \in \mathbb{Z}\), form a partition of \(\mathbb{R}\), and the sequence \(S_q = \bigcup_{r=-\infty}^{+\infty} I_{q}^{r} \times I_{q}^{r}\) decreases towards the diagonal axis \(\Delta\), as \(q\) becomes infinite.... Given \(B\) in \(\mathscr{B}(\mathbb{R}), n\), and \(q\), then the measure \(M_{n}\) verifies:
\[
\begin{aligned}
0 & \leqslant M_{n}\left(\bigcup_{r=-\infty}^{+\infty}\left(B \cap I_{q}^{r}\right) \times\left(B \cap I_{q}^{r}\right)\right) \\
& =\sum_{r=-\infty}^{+\infty} M\left(\left(B \cap I_{q}^{r} \cap B_{n}\right) \times\left(B \cap I_{q}^{r} \cap B_{n}\right)\right) \\
& \leqslant \sum_{r=-\infty}^{+\infty} m_{d}\left(B \cap I_{q}^{r} \cap B_{n}\right) \\
& =m_{d}\left(B \cap B_{n}\right) .
\end{aligned}
\]
Hence, when \(q\) tends to infinity we obtain (taking the limit inside the sum requires justification, perhaps using properties of measures on product spaces, or the definition of $m_n$ as the diagonal restriction):
\[
0 \leqslant m_{n}(B) \leqslant m_{d}\left(B \cap B_{n}\right) \leqslant m_{d}(\mathbb{R}) .
\]
So, for every Borel set \(B\), the increasing sequence \(\left(m_{n}(B)\right)\) converges towards a positive number \(m(B)\), and according to the Vitali-Hahn Saks theorem [4, Corollary III.7.3], \(m\) is a positive bounded measure on \(\mathscr{B}(\mathbb{R})\). It is estimated for all \(n\) and \(B\) by
\[
m_{n}(B) \leqslant m(B) \leqslant m_{d}(B) \leqslant m_{d}(\mathbb{R})<+\infty \quad \text { and } \quad m\left(B \cap B_{n}\right)=m_{n}(B)
\]
Moreover for any bounded Borel function \(f\) one has:
\[
\begin{aligned}
\left|\int f(u) m_{n}(d u)-\int f(u) m(d u)\right| & = \left|\int f(u) m(d u) - \int f(u) m_n(d u) \right| \\
& = \left| \int_{B} f(u) m(d u) - \int_{B \cap B_n} f(u) m(d u) \right| \\ % Using m(B \cap B_n) = m_n(B) is incorrect here.
% Let's use the fact that m is the limit of m_n.
& = \left| \int f(u) (m - m_n)(d u) \right| \\
& = \left| \int_{B_n^c} f(u) m(d u) \right| \quad (\text{since } m_n(A) = m(A \cap B_n)) \\
& \leqslant \int_{B_n^c} |f(u)| m(d u) \\
& \leqslant m\left(B_{n}^{\prime}\right) \cdot \sup_{u \in \mathbb{R}} (|f(u)|) .
\end{aligned}
\]
Since $m(B_n') \to 0$ as $n \to \infty$ (because $m$ is a finite measure and $B_n' \downarrow \emptyset$), the convergence $\int f dm_n \to \int f dm$ holds.
Consequently, given \(\varepsilon>0\), there exists \(N^{\prime}(\varepsilon)\) such that for \(n>N^{\prime}(\varepsilon)\) and for every \(h\) (taking \(f(u) = e^{iuh}\)):
\begin{equation} \label{eq:b1}
\left|\iint_{\Delta} e^{i u h} M_{n}(d u, d v)-\int e^{i u h} m(d u)\right| = \left|\int e^{i u h} m_{n}(d u)-\int e^{i u h} m(d u)\right| < \varepsilon .
\end{equation}
(c) From the relations (\ref{eq:a3}) and (\ref{eq:b1}) we deduce that for any \(\varepsilon>0\), there exists \(N = \max(N(\varepsilon), N'(\varepsilon))\) and \(T(\varepsilon) = T(N, \varepsilon)\) such that for \(t>T(\varepsilon)\) and for every \(h\) we have:
\[
\begin{aligned}
\left|\frac{1}{t} \int_{0}^{t} K(s+h, s) d s-\int e^{i u h} m(d u)\right| & \le \left|\frac{1}{t} \int_{0}^{t} K(s+h, s) d s-\iint_{\Delta} e^{i u h} M_{N}(d u, d v)\right| \\
& + \left|\iint_{\Delta} e^{i u h} M_{N}(d u, d v)-\int e^{i u h} m(d u)\right| \\
& < 2\varepsilon + \varepsilon = 3\varepsilon
\end{aligned}
\]
as was to be shown.
\end{proof}

\newtheorem{remark}{Remark}[section] % Define remark environment
\begin{remark} % Assuming 3.3 is a Remark
(a) There exist weakly harmonizable processes with non-\(\sigma\)-finite spectral bimeasure. Indeed, Niemi gave an example of a discrete time weakly harmonizable process which is not asymptotically stationary (cf. [11, Sect. 6]). As Theorems 3.1 and 3.2 still hold in the discrete time case, its spectral bimeasure is not \(\sigma\)-finite. Consequently, \(\mu\) denoting its spectral stochastic measure (defined on \(\mathscr{B}([-\pi, \pi])\) ), the spectral bimeasure of the (continuous time) weakly harmonizable process defined by
\[
X(t)=\int e^{i t x} \mu(d x), \quad t \in \mathbb{R},
\]
is not \(\sigma\)-finite. We do not know if \(X\) is asymptotically stationary.

More generally we do not know how to compare more precisely the class of weakly harmonizable processes and the class of asymptotically stationary processes.
(b) So we have:
\[
\begin{array}{c@{}c@{}c}
\{\text{stationary}\} & \varsubsetneqq & \{\text{strongly harmonizable}\} \\
& \varsubsetneqq & \{\text{harmonizable with } \sigma\text{-finite bimeasure}\} \\
& \varsubsetneqq & \{\text{weakly harmonizable}\} \\
& \subset & \{\text{asymptotically stationary}\}
\end{array}
\]
(Note: The layout in the original PDF is complex; this array attempts to capture the relationships shown.)
\end{remark}

\section*{Acknowledgments}

I would like to thank Professor M. M. Rao and the referee as well as Professor R. Moché for their precious advice.

\section*{References}
% Using a simple list for references as in the original text
\begin{enumerate}
    \item[1.] Blanc Lapierre, A., and Brard, R. (1949). Les fonctions aléatoires stationnaires et la loi des grands nombres. Bull. Soc. Math. France 74 102-105.
    \item[2.] Dehay, D. (1985). Quelques lois des grands nombres pour les processus harmonisables. Thèse U.E.R. Math., Université des Sciences et Techniques de Lille.
    \item[3.] Dehay, D., and MochE, R. (1986). Strongly harmonizable approximations of bounded continuous random fields. Stochastic Process. Appl. 23 327-331.
    \item[4.] Dunford, N., and Schwartz, J. T. (1957). Linear Operators, Part I, Interscience, New York.
    \item[5.] Edwards, D. A. (1955). Vector-valued measure and bounded variation in Hilbert space. Math. Scand. 3 90-96.
    \item[6.] Kampé de Fériet, J., and Frankiel, F. N. (1962). Correlation and spectra of nonstationary random functions. Math. Comp. 16 1-21.
    \item[7.] Miamee, A . G., and Salehi, H. (1978). Harmonizability, V-boundedness and stationary dilation of stochastic processes. Indiana Univ. Math. J. 27 37-50.
    \item[8.] Mocht, R. (1985). Introduction aux processus harmonisables. Lectures Notes, U.E.R. Math., Université des Sciences et Techniques de Lille.
    \item[9.] Niemi, H. (1975). Stochastic processes as Fourier transforms of stochastic measures. Ann. Acad. Sci. Fenn. Ser. A I Math. 591 1-47.
    \item[10.] Niemi, H. (1975). On stationary dilations and linear prediction of certain stochastic processes. Soc. Sci. Fenn. Comment. Phys.-Math. 45 111-130.
    \item[11.] Rao, M. M. (1985). Harmonizable, Cramér, and Karhunen classes of processes. In Handbook of Statistics, Vol. 5, pp. 279-310. Elsevier, New York/Amsterdam.
    \item[12.] Rozanov, Yu. A. (1959). Spectral analysis of abstract functions. Theory Probab. Appl. 4 271-287.
\end{enumerate}

\end{document}
