\documentclass{article}
\usepackage[english]{babel}
\usepackage{amsmath}

%%%%%%%%%% Start TeXmacs macros
\newcommand{\tmop}[1]{\ensuremath{\operatorname{#1}}}
%%%%%%%%%% End TeXmacs macros

\begin{document}

\title{Arb and Uncertainty Propagation in Financial Modeling}

\date{}

\maketitle

\section{Introduction}

Arbitrary-precision ball arithmetic (Arb) provides a rigorous foundation for
uncertainty propagation in complex numerical computations, particularly in the
context of financial modeling.

\section{Key Concepts}

\subsection{Uncertainty Propagation}

Arb is fundamentally based on uncertainty propagation. Every arithmetic
operation carries forward the precision of its operands:
\[ (x \pm \epsilon_x) \circ (y \pm \epsilon_y) = z \pm \epsilon_z \]
where $\circ$ represents any basic arithmetic operation and $\epsilon$ denotes
the associated uncertainty.

\subsection{Rigorous Error Bounds}

For any computation $f (x)$, Arb provides rigorous error bounds:
\[ f (x) \in [f_{\tmop{low}} (x), f_{\tmop{high}} (x)] \]
These bounds are mathematically guaranteed, not approximations.

\subsection{Representation of Numbers}

In Arb, a number $x$ is represented as a ball:
\[ x = [m - r, m + r] \]
where $m$ is the midpoint and $r$ is the radius, encapsulating the
uncertainty.

\section{Implications for Financial Modeling}

\subsection{Parameter Estimation}

For any model parameter $\theta$:
\[ \theta \in [\theta_{\tmop{low}}, \theta_{\tmop{high}}] \]
The goal is to minimize $(\theta_{\tmop{high}} - \theta_{\tmop{low}})$ through
precise calibration.

\subsection{Price Calculation}

For an option price $P$:
\[ P = f (\theta_1, \theta_2, ..., \theta_n) \in [P_{\tmop{low}},
   P_{\tmop{high}}] \]
The bounds $[P_{\tmop{low}}, P_{\tmop{high}}]$ rigorously account for all
parameter uncertainties.

\subsection{Model Robustness}

The use of Arb ensures:
\begin{enumerate}
  \item Prevention of silent error accumulation
  
  \item Clear quantification of result reliability
  
  \item Does not suffer from numerical instabilities 
\end{enumerate}

\section{Conclusion}

The integration of Arb in financial modeling provides a level of numerical
rigor and reliability crucial for advanced quantitative finance applications,
offering mathematically guaranteed bounds on all computations and results.

\end{document}
