\documentclass{article}
\usepackage[english]{babel}
\usepackage{amsmath,amssymb,enumerate,mathrsfs,theorem}

%%%%%%%%%% Start TeXmacs macros
\newcommand{\assign}{:=}
\newcommand{\cdummy}{\cdot}
\newcommand{\tmaffiliation}[1]{\\ #1}
\newcommand{\tmdummy}{$\mbox{}$}
\newcommand{\tmem}[1]{{\em #1\/}}
\newcommand{\tmemail}[1]{\\ \textit{Email:} \texttt{#1}}
\newcommand{\tmnote}[1]{\thanks{\textit{Note:} #1}}
\newenvironment{enumeratealpha}{\begin{enumerate}[a{\textup{)}}] }{\end{enumerate}}
{\theorembodyfont{\rmfamily}\newtheorem{remark}{Remark}}
\newtheorem{theorem}{Theorem}
%%%%%%%%%% End TeXmacs macros

\begin{document}

\title{Conditions for the completeness of the spectral domain of a
harmonizable process}

\author{
  Roland Averkamp
  \tmnote{Tel.: 49/761/203 5672; fax: 49/761/203 5661}
  \tmaffiliation{Institut f{\"u}r Mathematische Stochastik, Albert-Ludwigs
  Universit{\"a}t Freiburg, Eckerstrasse 1, 79104 Freiburg, Germany}
  \tmemail{averkamp@pollux.mathematik.uni-freiburg.de}
}

\date{Received 11 September 1995; received in revised form 23 June 1997}

\maketitle

\begin{abstract}
  We generalize a theorem of K{\"o}the and Toeplitz on unconditional bases in
  Hilbert spaces to Hilbert space-valued measures. This leads to a necessary
  and sufficient condition for the completeness of the spectral domain of a
  weakly harmonizable process whose shift operator exists and is invertible. A
  process in this class has a complete spectral domain if and only if it is
  the image of a stationary process under a topological isomorphism. (C) 1997
  Elsevier Science B.V.
\end{abstract}

{\noindent}Keywords: Weakly harmonizable process; Bimeasure; Completeness of
spectral domain; Shift operator

\section*{1. Introduction}

Let $L_0^2 (P)$ denote the zero mean square integrable functions on some
probability space. The stochastic processes in this paper are either discrete
time ($T =\mathbb{Z}$) or continuous time $(T =\mathbb{R})$ and they are
subsets of $L_0^2 (P)$. Let $\hat{T} = \Pi \assign \mathbb{R}/ 2 \pi
\mathbb{Z}= (- \pi, \pi]$ if $T =\mathbb{Z}$ and $\hat{T} =\mathbb{R}$ if $T
=\mathbb{R}$, and let $\mathbb{B}$ denote the Borel $\sigma$-field of either
$\Pi$ or $\mathbb{R}$.

Recall that a process $(X_t)_{t \in T}$ is weakly stationary if its covariance
function admits the representation
\begin{equation}
  \mathrm{Cov} (X_s, X_t) = \int_{\hat{T}} e^{i \lambda s} \overline{e^{i
  \lambda t}} \mu (d \lambda), \quad s, t \in T
\end{equation}
where $\mu$ is a finite positive measure on $\hat{T}$. The time domain of a
stationary process is isometrically isomorphic to its spectral domain $L^2
(\mu)$ and prediction and filtering of the process can thus be carried out in
the spectral domain using Fourier methods.

Harmonizable processes are an extension which allow also for these Fourier
methods. A stochastic process $(X_t)_{t \in T}$ is called strongly
harmonizable, if its covariance function admits the representation
\begin{equation}
  \mathrm{Cov} (X_s, X_t) = \int_{\hat{T}} \int_{\hat{T}} e^{is \lambda}
  \overline{e^{it \lambda'}} \hspace{0.27em} \beta (d \lambda, d \lambda'),
  \quad \forall s, t \in T
\end{equation}
for some measure $\beta$ on $\mathbb{B} \otimes \mathbb{B}$. These processes
were introduced by Lo{\`e}ve (1948) and Cram{\'e}r (1951). An important
example of strongly harmonizable processes are the periodically correlated
discrete time processes (Hurd, 1989, 1991; Dehay, 1994). In this case, the
mass of $\beta$ is concentrated on lines parallel to the diagonal $\{(\lambda,
\lambda), \lambda \in \hat{T} \}$ of $\hat{T} \times \hat{T}$. However, the
class of strongly harmonizable processes is still not large enough for some
applications. For example, it is not closed under linear transformations. A
wider extension of stationary processes are the weakly harmonizable processes.
These processes have the same covariance representation as strongly
harmonizable processes, but the conditions on $\beta$ are relaxed. $\beta$ is
not necessarily a measure, but can be a bimeasure (i.e., $\beta$ is a mapping
$\beta : \mathbb{B} \times \mathbb{B} \to \mathbb{C}$ and $\beta (\cdummy, A),
\beta (A, \cdot)$ are measures for all $A \in \mathbb{B}$), which is positive
definite (i.e., $\beta (A, B) = \overline{\beta (B, A)}$ and $\sum_{i, j} a_i
\beta (A_i, A_j) \overline{a_j} \geq 0$ for all $A_i \in \mathbb{B}, a_i \in
\mathbb{C}$). $\beta$ is called the spectral bimeasure of the process. The
class of weakly harmonizable processes is closed under linear transformations.
Indeed, another characterization of weakly harmonizability is that these
processes are coordinatewise images of stationary processes under linear
mappings (Niemi, 1975, 1977). In general, bimeasures cannot be extended to
measures on $\hat{T} \otimes \hat{T}$, so the integral in (2) is not a
Lebesgue integral, but a strict integral in the sense of Morse and Transue
(Chang and Rao, 1986).

Let $f, g : \hat{T} \to \mathbb{C}$ be measurable and $\beta$ a bimeasure on
$\mathbb{B} \times \mathbb{B}$. $(f, g)$ is strictly $\beta$-integrable, if
the following holds:
\begin{itemize}
  \item[(1)] $f$ is $\beta (\cdummy, A)$-integrable, $g$ is $\beta (A,
  \cdot)$-integrable for all $A \in \mathbb{B}$. If this holds, then
  $\beta_1^E : A \mapsto \int_E f (x) \beta (dx, A)$ and $\beta_2^F : A
  \mapsto \int_F g (y) \beta (A, dy)$ are measures for all $E, F \in \Sigma$
  (Houdr{\'e}, 1987, Lemma 2).
  
  \item[(2)] $f$ is $\beta_2^F$-integrable and $g$ is $\beta_1^E$-integrable
  for all $E, F \in \mathbb{B}$. One defines now
\end{itemize}
\begin{equation}
  \int_E \int_F^{\ast} (f, g)  \hspace{0.17em} d \beta \assign \int_E f (x) 
  \hspace{0.17em} \beta_2^F  (dx) = \int_F g (y) \beta_1^E  (dy)
\end{equation}
The equality of the integrals holds by Houdr{\'e} (1987, Theorem 4). If
$\beta$ is a measure on $\mathbb{B} \otimes \mathbb{B}$ and $f, g$ are bounded
then this integral is equal to the usual Lebesgue integral, i.e., $\int_E
\int_F f (x) g (y) \beta (dx, dy)$.

For a positive-definite bimeasure $\beta$ define
\begin{equation}
  \mathscr{L}_{\ast}^2 (\beta) \assign \{f : (f, \bar{f}) \text{is strict }
  \beta \text{-integrable} \}
\end{equation}
the spectral domain of $\beta$. This is a pre-Hilbert space with the inner
product $(f, g)_{\beta} \assign \iint^{\ast} (f, \bar{g})  \hspace{0.17em} d
\beta$ (cf. Chang and Rao, 1986).

Now let $(X_t)_{t \in T}$ be a weakly harmonizable process with spectral
bimeasure $\beta$. As for a stationary process, $(X_t)_{t \in T}$ has an
integral representation $X_t = \int_{\hat{T}} e^{i \lambda t} Z (d \lambda)$
where $Z$ is a stochastic measure (i.e., $L_0^2 (P)$-valued) and the integral
is in the sense of Dunford-Schwartz (1960, IV.10). $Z$ is called the
stochastic spectral measure of $(X_t)_{t \in T}$. It is related to the
spectral bimeasure by $\beta (A, B) = EZ (A) \overline{Z (B)}$,
$\mathscr{L}_{\ast}^2 (\beta) = L^1 (Z)$ and
\begin{equation}
  \mathrm{Cov} \left( \int f \hspace{0.17em} dZ, \int g \hspace{0.17em} dZ
  \right) = \iint^{\ast} (f, \bar{g})  \hspace{0.17em} d \beta
\end{equation}
(Chang and Rao, 1986). We then say that $\beta$ is induced by $Z$. Note that
$(X_t)_{t \in T}$ is stationary if $Z$ is orthogonally scattered (i.e., $A
\cap B = \emptyset$ then $Z (A) \perp Z (B)$) or equivalently the mass of
$\beta$ is concentrated on the diagonal, i.e., $\beta (A, B) = \beta (A \cap
B)$.

It is well known that
\begin{equation}
  H_Z \assign \overline{\mathrm{sp} \{Z (A), A \in \mathbb{B}\}} =
  \overline{\mathrm{sp} \left\{ \int_{\hat{T}} e^{it \cdot} dZ, t \in T
  \right\}}
\end{equation}
Indeed, stochastic measures on $\hat{T}$ are regular (this follows from
Dunford and Schwartz, 1960, IV.10.5) and hence the set of integrals of
continuous functions is dense in $H_Z$. The assertion now follows from the
fact that continuous functions can be approximated uniformly on compact sets
by trigonometric polynomials (Naimark, 1959, p. 406, Corollary 4).

As we have already seen, the mapping $V : \mathscr{L}_{\ast}^2 (\beta) \to H_Z
(= \overline{\mathrm{sp} \{X_t, t \in T\}}), f \mapsto \int f \hspace{0.17em}
dZ$ is an isometry. It is easy to see that $V$ is surjective if and only if
$\mathscr{L}_{\ast}^2 (\beta)$ is complete. Thus, the spectral domain and the
time domain of the process are isometrically isomorphic if and only if the
spectral domain is complete.

One of the reasons to introduce weakly harmonizable processes was to retain
the powerful Fourier methods of the stationary processes, so that, for
example, extrapolation and interpolation can be carried out in the spectral
domain of the process, using again the methods of Fourier analysis. But the
transition from the spectral domain to the time domain of a process can only
be made in full generality if the spectral domain is complete. For example,
Chang and Rao (1986) need this completeness to solve filter equations, and
Houdr{\'e} (1991) for obtaining autoregressive predictors. A solution to this
problem might be to take the algebraic completion of the spectral domain. But
the members of this space need not to be functions in the usual sense. Also
the isometry to the time domain can no longer be described by the elegant
integral representation of the process.

Cram{\'e}r, who defined the spectral domain, raised the problem of the
completeness of this spectral domain: "the set $\Lambda_2 (\rho)$ will have
all the properties of Hilbert space, except possibly the completeness
property" (Cram{\'e}r, 1951). In the 1980s, Rao et al. (Chang and Rao, 1986;
Rao, 1989a, b; Mehlmann, 1991) came up with proofs for the completeness of the
spectral domain, but these proofs worked only in special cases. Then in 1991,
Miamee and Salehi were able to show that, in general, $\mathscr{L}_{\ast}^2
(\beta)$ is not complete (cf. Miamee and Salehi, 1991; Mich{\'a}lek and
R{\"u}schendorf, 1994). In 1995, Miamee and Schr{\"o}der (1995) gave
conditions for the completeness of the spectral domain in certain cases (see
also Mich{\'a}lek and R{\"u}schendorf, 1994).

In this work, we represent a necessary and sufficient condition for the
completeness of the spectral domain for a broad class of weakly harmonizable
processes. This result is not only of interest for harmonizable processes, but
also in functional analysis, in fact we prove:\\
Let $Z$ be a Hilbert space-valued measure such that $\int f \hspace{0.17em} dZ
= 0$ implies $f = 0$ $Z$-almost everywhere. Then $\{ \int f \hspace{0.17em}
dZ, f \in L^1 (Z)\}$ is complete if and only if there exists an orthogonally
scattered measure $\hat{Z}$ and an isomorphism $V$ with $V (\hat{Z}) = Z$.

The following result due to Rao (Rao, 1989, p. 605, Proposition 3.2) gives a
sufficient condition for the completeness of the spectral domain of a weakly
harmonizable process (with $T =\mathbb{R}$ or $\mathbb{Z}$): Let $\beta$ be a
positive-definite bimeasure on $\mathbb{B} \times \mathbb{B}$. Then
$\mathscr{L}_{\ast}^2 (\beta)$ is complete if there is a positive finite
measure $\mu$ on $\mathbb{B}$ and a continuous linear mapping $W : L^2 (\mu)
\to L^2 (\mu)$ with closed range such that $L^2 (\mu) \subset
\mathscr{L}_{\ast}^2 (\beta)$ and
\begin{equation}
  \iint^{\ast} (f, \bar{g})  \hspace{0.17em} d \beta = \int W (f) \overline{W
  (g)} \hspace{0.17em} d \mu, \quad \forall f, g \in L^2 (\mu)
\end{equation}
There is always a finite measure $\mu$ and a linear continuous mapping from
$L^2 (\mu)$ to $L^2 (\mu)$ such that (7) holds, but its range is not
necessarily closed. This follows from the Grothendieck inequality (Rao, 1989,
Theorem 2). If $L^2 (\mu) = \mathscr{L}_{\ast}^2 (\beta)$, the closed range
condition is necessary. The closed range condition is clearly satisfied if
there are constants $C, c > 0$ such that
\begin{equation}
  C \|f\|_{\mu, 2} \geq \|f\|_{\beta} \geq c \|f\|_{\mu, 2}, \quad \forall f
  \in L^2 (\mu)
\end{equation}
holds. This is the sufficient condition for the completeness of the spectral
domain obtained by Truong-Van (1981) (Theorem 6, cf. also Mich{\'a}lek and
R{\"u}schendorf, 1994). In the next section we shall see that this condition
is close to being necessary for completeness.

Since the results of the next section do not depend on the structure of
$\mathbb{R}$ or $\Pi$, we will formulate and prove them for arbitrary
measurable spaces. So let $(\Omega, \Sigma)$ be a measurable space and $\beta$
a positive-definite bimeasure defined on $\Sigma \times \Sigma$. $\beta$ is
induced by the stochastic measure $Z$. Since the strict integral is not handy,
we shall make use of (3) and use instead the Dunford-Schwartz integral for
vector measures. By $\|Z\| (\cdummy)$, we will denote the semivariation of the
stochastic measure $Z$ (Dunford and Schwartz, 1960, IV.10.3). In this paper,
we will use only the Hilbert space properties of $L_0^2 (P)$. By $\| \cdummy
\|$ we refer to the norm of this space.

The next theorem is crucial for the next section and due to Drewsnowski
(1974a, p. 217, 3.10; 1974b, p. 799, Example 2.3.3).

\begin{theorem}
  {\tmdummy}
  
  \begin{enumeratealpha}
    \item $\| \cdummy \|_Z$ defined via $\|f\|_Z \assign \sup_{|g| \leq |f|}
    \left| \int g \hspace{0.17em} dZ \right|$, $f \in L^1 (Z)$, is a norm on
    $L^1 (Z)$. This norm is equivalent to the norm
    \begin{equation}
      \lVert |f| \rVert \assign \sup_{A \in \Sigma} \left| \int_A f
      \hspace{0.17em} dZ \right|  \left( \geq \frac{1}{4} \|f\|_Z \right)
    \end{equation}
    \item $(L^1 (Z), \| \cdot \|_Z)$ is a Banach space.
    
    \item If $f_n, f \in L^1 (Z)$ for all $n \in \mathbb{N}$ and $\lim_{n \to
    \infty} \|f - f_n \|_Z = 0$, then $(f_n)_{n \in \mathbb{N}}$ converges to
    $f$ in $Z$-measure (i.e., $\forall \varepsilon > 0$, $\lim_{n \to \infty}
    \|Z\| (\{|f_n - f| > \varepsilon\}) = 0$). Hence any subsequence of
    $(f_n)_{n \in \mathbb{N}}$ contains a subsequence converging $Z$-a.e. to
    $f$.
  \end{enumeratealpha}
\end{theorem}

Note that $\|f\|_Z = \sup_{|g| \leq |f|} \|g\|_{\beta}$! Finally, note that
the simple functions are dense in $(L^1 (Z), \| \cdot \|_Z)$. This follows
from the results in Kwapien and Woycy{\'n}ski (1992, Proposition 7.1.1,
Definition 7.3.1).

\section*{2. A condition for the completeness of the spectral domain}

Miamee and Schr{\"o}der (1995) obtained necessary conditions for the
completeness of the spectral domain of a weakly harmonizable process. In
particular, for the class of spectral bimeasures $\beta$ such that for some
positive finite measure $\mu$, $\mu (\{f \neq 0\}) = 0$ is equivalent to
$\|f\|_{\beta} = 0$. In this section we investigate the completeness of the
spectral domain of a bimeasure $\beta$ such that $\| \cdummy \|_Z = 0$ is
equivalent to $\| \cdummy \|_{\beta} = 0$, which is a weaker condition than
the one studied by Miamee and Schr{\"o}der.

\begin{theorem}
  Let $\| \cdummy \|_{\beta} = 0$ be equivalent to $\| \cdummy \|_Z = 0$. Then
  the following statements are equivalent:
  \begin{enumerate}
    \item $\mathscr{L}_{\ast}^2 (\beta)$ is complete.
    
    \item There exists a $K > 0$ such that for all $f \in L^1 (Z)$,
    \begin{equation}
      K \|f\|_{\beta} \geq \|f\|_Z
    \end{equation}
    \item There exists a positive finite measure $\mu$ and constants $c, C >
    0$ such that $L^2 (\mu) = L^1 (Z) (= \mathscr{L}_{\ast}^2 (\beta))$ and
    \begin{equation}
      C \| \cdummy \|_{\mu, 2} \geq \| \cdummy \|_{\beta} \geq c \| \cdummy
      \|_{\mu, 2}
    \end{equation}
  \end{enumerate}
\end{theorem}

{\tmem{Proof.}} $(1) \Rightarrow (2)$: Note that $T : L^1 (Z) \to
\mathscr{L}_{\ast}^2 (\beta), f \mapsto f$, is a linear, one-to-one mapping
and that $(L^1 (Z), \| \cdot \|_Z)$ and $\mathscr{L}_{\ast}^2 (\beta)$ are
Banach spaces. This implies that $T$ has a continuous linear inverse (Dunford
and Schwartz, 1960, II.2.2). Consequently, there exists $K > 0$ satisfying $K
\|f\|_{\beta} \geq \|f\|_Z$ for all $f \in L^1 (Z)$.

$(2) \Rightarrow (1)$: Since $\|f\|_Z \geq \|f\|_{\beta}$, both norms are
equivalent.

$(3) \Rightarrow (2)$: It follows directly from the definition of $\| \cdummy
\|_Z$ that $C \| \cdummy \|_{\mu, 2} \geq \| \cdummy \|_Z$, hence $C / c \|
\cdummy \|_{\beta} \geq \| \cdummy \|_Z$.

$(2) \Rightarrow (3)$: We define $|f|_Z \assign \inf_{|h| = |f|} \| \int h
\hspace{0.17em} dZ\| = \inf_{|h| = |f|} \|h\|_{\beta}$, and prove that $\|
\cdummy \|_Z^2$ is superadditive while $| \cdummy |_Z^2$ is subadditive:
Indeed, let $f, g \in L^1 (Z)$ and $fg = 0$. Then

\begin{align}
  \|f + g\|_Z^2 & = \sup_{|f' | \leq |f|, |g' | \leq |g|} \sup_{\alpha \in
  \mathbb{C}, | \alpha | = 1} (\alpha f', \alpha f')_{\beta} + (g',
  g')_{\beta} + 2 \mathrm{Re} (\alpha f', g')_{\beta} \\
  & \geq \sup_{|f' | \leq |f|, |g' | \leq |g|} (f', f')_{\beta} + (g',
  g')_{\beta} \\
  & = \|f\|_Z^2 + \|g\|_Z^2 
\end{align}

In a similar way one shows $|f + g|_Z^2 \leq |f|_Z^2 + |g|_Z^2$.

Next, we prove that
\begin{equation}
  \mu (A) \assign \sup \left\{ \sum_{i \in I} |1_{A_i} |_Z^2 ; (A_i)_{i \in I}
  \text{finite measurable partitions of } A \right\}
\end{equation}
is a positive finite measure. The properties $\mu (\emptyset) = 0$, $\mu \geq
0$, and the additivity follow readily from the definition of $\mu$.

Since $\|Z\| (\cdummy)$ is continuous at the empty set (Dunford and Schwartz,
1960, IV.10.5) and $\mu (A) \leq \|1_A \|_Z^2 = \|Z\| (A)^2$, we conclude that
$\mu$ is also continuous. Thus $\mu$ is a measure.

It is clear that $|1_A |_Z^2 \leq \mu (A) \leq \|1_A \|_Z^2$ for all $A \in
\Sigma$. Let $f$ be a simple function. From the last inequality, the
subadditivity of $| \cdummy |_Z^2$, and the superadditivity of $\| \cdummy
\|_Z^2$, it follows that
\begin{equation}
  |f|_Z^2 \leq \|f\|_{\mu, 2}^2 \leq \|f\|_Z^2
\end{equation}
By the assumptions there exists a $K > 0$ such that $K \| \cdummy \|_{\beta}
\geq \| \cdummy \|_Z$. Now let $f \in L^1 (Z)$; then
\begin{equation}
  K |f|_Z = K \inf_{|g| = |f|} \|g\|_{\beta} \geq \inf_{|g| = |f|} \|g\|_Z =
  \|f\|_Z
\end{equation}
where the last equality holds since $\| \cdummy \|_Z$ depends only on the
modulus of a function. Thus $K | \cdummy |_Z \geq \| \cdummy \|_Z$.

Hence, the norms $\| \cdummy \|_{\mu, 2}$ and $\| \cdummy \|_Z$ are equivalent
for the simple functions. Since $(L^1 (Z), \| \cdot \|_Z)$ and $L^2 (\mu)$ are
Banach spaces, since the simple functions are dense in both spaces and since
the convergence in both spaces implies the almost everywhere convergence of
some subsequence, the norms are equivalent for $L^1 (Z) = L^2 (\mu)$.

The equivalence of $\| \cdummy \|_{\beta}$ and $\| \cdummy \|_Z$ implies the
equivalence of $\| \cdummy \|_{\beta}$ and $\| \cdummy \|_{\mu, 2}$.

Let $\beta, Z$ and $\mu$ be as in the second condition of Theorem 1. Then
there is (up to isometric isomorphism) a unique orthogonally scattered Hilbert
space valued measure $\hat{Z}$ satisfying $\| \hat{Z} (A)\|^2 = \mu (A)$, $L^1
(\hat{Z}) = L^2 (\mu)$ and
\begin{equation}
  \left( \int f \hspace{0.17em} d \hat{Z}, \int g \hspace{0.17em} d \hat{Z}
  \right) = \int f \bar{g}  \hspace{0.17em} d \mu, \quad \forall f, g \in L^2
  (\mu)
\end{equation}
Take, for example, $\hat{Z} (A) \assign 1_A \in L^2 (\mu)$. Since $\| \cdummy
\|_{\mu, 2}$ and $\| \cdummy \|_{\beta}$ are equivalent, there is a continuous
and continuously invertible linear map $V : H_{\hat{Z}} \to H_Z$ with $V \circ
\hat{Z} = Z$.

Now, let $(x_i)_{i \in \mathbb{N}}$ be a normed sequence in some Hilbert
space, and let $Z$ be the measure on $\mathbb{N}$ with $Z (\{i\}) = x_i / i^2,
i \in \mathbb{N}$. It is easy to show that $(a_i)_{i \in \mathbb{N}} \in L^1
(Z)$ if and only if $\sum_{i = 1}^{\infty} a_i x_i  \left( = \int_{\mathbb{N}}
a \hspace{0.17em} dZ \right)$ is unconditionally convergent. Hence, it is easy
to see that Theorem 1 is a generalization of the theorem of K{\"o}the and
Toeplitz (Nikolskii, 1986, p. 137), which states that if $(x_n)_{n \in
\mathbb{N}}$ is a normed unconditional base of a Hilbert space, then there
exist constants $c, C > 0$ such that
\begin{equation}
  c \sum_{i = 1}^n |a_i |^2 \leq \left| \sum_{i = 1}^n a_i x_i \right|^2 \leq
  C \sum_{i = 1}^n |a_i |^2, \quad a_i \in \mathbb{C}, i = 1, \ldots, n \in
  \mathbb{N}
\end{equation}
In other words, any normed unconditional base of a separable Hilbert space is
topologically isomorphic to some orthonormal base.

\section*{3. The shift operator condition}

We now turn to the question of finding under which conditions $\| \cdummy
\|_{\beta} = 0$ and $\| \cdummy \|_Z = 0$ are equivalent, where $\beta$ is the
spectral bimeasure and $Z$ the stochastic spectral measure of some weakly
harmonizable process.

The next theorem gives a necessary and sufficient condition for this
equivalence.

\begin{theorem}
  Let $(X_t)_{t \in \mathbb{Z}}$ be a weakly harmonizable process with
  spectral bimeasure $\beta$ and stochastic spectral measure $Z$.
  
  Then the following two statements are equivalent:
  \begin{enumerate}
    \item $\| \cdummy \|_{\beta} = 0$ is equivalent to $\| \cdummy \|_Z = 0$.
    
    \item The shift operator $S : \mathscr{L}_{\ast}^2 (\beta) \to
    \mathscr{L}_{\ast}^2 (\beta)$, $S (f) \assign \exp (i \cdummy) f$ is a
    well-defined invertible linear mapping, i.e., $\|f\|_{\beta} = 0
    \Leftrightarrow \| \exp (i \cdot) f\|_{\beta} = 0$.
  \end{enumerate}
\end{theorem}

{\tmem{Proof.}} (1) $\Rightarrow$ (2): This follows immediately from $\|fe^{i
\cdot} \|_Z = \|f\|_Z$.

(2) $\Rightarrow$ (1): Let $\|f\|_{\beta} = 0$. It follows from the theorem
stated at the end of Section 1, that there is $A \in \mathbb{B}$ such that
$\|f\|_Z \leq 5 \|f 1_A \|_{\beta}$. Since $\|Z\|$ is regular, it is easy to
see that there exist trigonometric polynomials $(p_n)_{n \in \mathbb{N}}$ with
$\|p_n \|_{\infty} \leq 2$ for all $n \in \mathbb{N}$, such that $\lim_{n \to
\infty} \|p_n - 1_A \|_Z = 0$ (the trigonometric polynomials are dense in the
continuous functions with respect to the uniform convergence (Naimark, 1959,
p. 406, Corollary 4)). By the theorem previously cited there is a subsequence
$(p_{n_k})_{k \in \mathbb{N}}$ converging $Z$-almost everywhere to $1_A$.
Applying the theorem of dominated convergence (Dunford and Schwartz, 1960,
IV.10.10) we have $\lim_{k \to \infty}  \int p_{n_k} f \hspace{0.17em} dZ =
\int_A f \hspace{0.17em} dZ$. But if $p = \sum_{j = - m}^m a_j e^{ij \cdot}$,
then
\begin{equation}
  \left| \int pf \hspace{0.17em} dZ \right| = \|pf\|_{\beta} \leq \sum_{j = -
  m}^m |a_j |  | e^{ij \cdot} f |_{\beta} = \sum_{j = - m}^m |a_j | \|S^j
  (f)\|_{\beta} = 0
\end{equation}
Hence $0 = 5 \| \int_A f \hspace{0.17em} dZ\| \geq \|f\|_Z$.

\begin{remark}
  A sufficient condition for the existence of an invertible shift operator $S$
  of the spectral domain is the existence of an invertible shift operator
  $\hat{S}$ of the time domain $H_X \assign \mathrm{sp} \{X_t, t \in
  \mathbb{Z}\}$, i.e., there is a continuous linear and continuously
  invertible mapping $\hat{S} : H_X \to H_X$ with $\hat{S} (X_t) = X_{t + 1}$
  for all $t \in \mathbb{Z}$.
  
  Indeed, it is sufficient to show that
  \begin{equation}
    \hat{S} \left( \int f \hspace{0.17em} dZ \right) = \int S (f) 
    \hspace{0.17em} dZ, \quad \forall f \in L^1 (Z)
  \end{equation}
  $U : f \mapsto e^{i \cdot} f$ and $V : f \mapsto \int f \hspace{0.17em} dZ$
  are continuous linear mappings from $(L^1 (Z), \| \cdot \|_Z)$ to $(L^1 (Z),
  \| \cdot \|_Z)$, respectively $H_Z$. If $p$ is a trigonometric polynomial,
  then $\hat{S} (V (p)) = V (U (p))$ holds. By an argument similar to the one
  given in the proof of Theorem 2 it follows that the trigonometric
  polynomials are dense in $(L^1 (Z), \| \cdot \|_Z)$. Thus, (19) is true for
  all $f \in L^1 (Z)$.
\end{remark}

A similar shift operator criterion holds for weakly harmonizable
continuous-time processes.

Let $(X_t)_{t \in \mathbb{Z}}$ be a weakly harmonizable process whose spectral
domain is complete and assume that the shift operator $S$ of the process
exists and is invertible. Then by the last two theorems the operators $S^k, k
\in \mathbb{Z}$, are uniformly bounded. But is the uniform boundedness of the
operators $S^k, k \in \mathbb{Z}$, sufficient for the completeness of the
spectral domain?

The following theorem, due to Nagy (1948), gives an affirmative answer (see
also Truong-Van, 1981).

\begin{theorem}
  Let $(X_t)_{t \in \mathbb{Z}}$ be a stochastic process, and let $S$ be the
  shift operator of the process (i.e., $S$ is a linear continuous mapping on
  the time domain of $(X_t)_{t \in \mathbb{Z}}$, $S (X_t) = X_{t + 1}$ for all
  $t \in \mathbb{Z}$). If $S$ is invertible and moreover the mappings $S^k, k
  \in \mathbb{Z}$, are uniformly bounded (in the operator norm), then
  $(X_t)_{t \in \mathbb{Z}}$ is weakly harmonizable and there exists a finite
  positive measure $\mu$, such that $\| \cdummy \|_{\mu, 2}$ and $\| \cdummy
  \|_{\beta}$ are equivalent, where $\beta$ is the spectral bimeasure of
  $(X_t)_{t \in \mathbb{Z}}$. Hence $\mathscr{L}_{\ast}^2 (\beta)$ is
  complete.
\end{theorem}

\section*{Acknowledgements}

The author would like to thank C. Houdr{\'e}, J. Kallsen and the anonymous
referee for their valuable advice on the composition of this paper.

\section*{References}

Chang, D.K., Rao, M.M., 1986. Bimeasures and nonstationary processes. In: Rao,
M.M. (Ed.), Real and Stochastic Analysis. Wiley, New York, pp. 7-118.\\
Cram{\'e}r, H. 1951. A Contribution to the theory of stochastic processes. In:
Proc. 2nd Berkeley Symp. Math. Statist. and Prob., University of California,
Berkeley, CA, pp. 329-339.\\
Dehay, D., 1994. Spectral analysis of the covariance of the almost
periodically correlated processes. Stochastic Processes Appl. 50 (2),
315-330.\\
Drewsnowski, L., 1974a. On control submeasures and measures. Stud. Math. 50,
203-224.\\
Drewsnowski, L., 1974b. On Subseries Convergence in some function spaces.
Bull. Acad{\'e}mie Polonaise Sci. S{\'e}rie Sci. Math. Astronom. Phys. 22,
797-803.\\
Dunford, N., Schwartz, J.T., 1960. Linear operators, Part 1: General Theory.
Wiley Interscience, New York.\\
Houdr{\'e}, C., 1987. A vector bimeasure integral with some applications.
Technical Report 214, Center for Stochastic Processes, Univ. North Carolina.\\
Houdr{\'e}, C., 1991. On the linear prediction of multivariate (2, p)-bounded
processes. Ann. Probab. 19 (2), 843-867.\\
Hurd, H.L., 1989. Representation of strongly harmonizable periodically
correlated processes and their covariances. J. Multivariate Anal. 29 (1),
53-67.\\
Hurd, H.L., 1991. Correlation theory of almost periodically correlated
processes. J. Multivariate Anal. 37 (1) 24-25.\\
Kwapien, S., Woycy{\'n}ski, W.A., 1992. Random series and stochastic
integrals: single and multiple. Birkh{\"a}user, Boston.\\
Lo{\`e}ve, M., 1948. Fonctions al{\'e}atoires du second ordre. In: P. L{\'e}vy
(Ed.), Processus Stochastiques et Mouvement Brownien. Guthier-Villars,
Paris.\\
Mehlmann, M., 1991. Structure and moving average representation for
multidimensional strongly harmonizable processes. Stochastic Anal. Appl. 9 (3)
323-361.\\
Miamee, A.G., Salehi, H., 1991. An example of a harmonizable process whose
spectral domain is not complete. Scandinavian J. Statist. 18, 249-254.\\
Miamee, A.G., Schr{\"o}der, B.S.W., 1995. On completeness of the spectral
domain of harmonizable processes. Probab. Theory Related Fields 101,
303-309.\\
Mich{\'a}lek, J., R{\"u}schendorf, L., 1994. A remark on the spectral domain
of nonstationary processes. Stochastic Processes Appl. 53 (1), 55-64.\\
Naimark, M.A., 1959. Normed Rings. P. Noordhof, Ltd. Groningen, Niederlands.\\
Niemi, H., 1975. On stationary dilations and the linear prediction of certain
stochastic processes. Soc. Sci. Fenn. Comment. Phys.-Math. 45, 111-130.\\
Niemi, H., 1977. On orthogonally scattered dilations of bounded vector
measures. Ann. Acad. Sci. Fenn. Ser. AI Math. 593, 43-52.\\
Nikolskii, N.K., 1986. Treatise on the Shiftoperator. Springer, Berlin.\\
Rao, M.M., 1989a. Bimeasures and harmonizable processes (analysis,
classification, and representation). In: Probability Theory on Groups IX,
Proc. 9th Conf., Oberwolfach, 1988. Lecture Notes in Math., vol. 1379.
Springer, Berlin, pp. 254-298.\\
Rao, M.M., 1989b. A view on harmonizable processes. In: Statistical Data
Analysis and Inference, Pap. Int. Conf., Neuchatel, Switzerland, pp.
597-615.\\
Rozanov, Y.A., 1967. Stationary Random Processes. Holden-Day Inc., San
Francisco.\\
de Sz. Nagy, B., 1948. On uniformly bounded linear transformations in Hilbert
space, Acta Sci. Math. XI. Acta Universitatis Szegediensis, pp. 152-157.\\
Truong-Van, B., 1981. Une g{\'e}n{\'e}ralisation du th{\'e}or{\`e}me de
Kolmogorov-Aronszajn Processus V-born{\'e}s $q$-dimensionnels: domaine
spectral \& dilatations stationnaires. Ann. Inst. Henri Poincar{\'e} XVII (1),
31-49.

\end{document}
