\documentclass{article}
\usepackage{amsmath}
\usepackage{amssymb}

\title{Relationship Between Two Finite Fourier Transform Integrals}
\author{}
\date{}

\begin{document}
\maketitle

\section{The Integral Formulas}

We examine the relationship between two integral formulas that appear as Lemma 1.1 and Lemma 2 in the study of finite Fourier transforms of orthogonal polynomials.

\subsection{Lemma 1.1: Power Function Integral}

For $k \geq 0$ an integer and $\lambda$ an indeterminate:
\begin{equation}
\int_0^1 x^k e^{i\lambda x} \, dx = \frac{(-1)^k k!}{(i\lambda)^{k+1}}\left[e^{i\lambda}E_k(-i\lambda) - e^{-i\lambda}E_k(i\lambda)\right]
\end{equation}

An alternative form is:
\begin{equation}
\int_0^1 x^k e^{i\lambda x} \, dx = \frac{(-1)^k k!}{(i\lambda)^{k+1}}\left[e^{i\lambda}E_k(-i\lambda) - 1\right]
\end{equation}

where $E_n(x) = \sum_{j=0}^{n} \frac{x^j}{j!}$ represents the partial sums of the exponential function.

\subsection{Lemma 2: Shifted Polynomial Integral}

For $j \geq 0$:
\begin{equation}
\int_0^1 \left(\frac{1-x}{2}\right)^j e^{ixy} \, dx = \frac{e^{iy} \gamma(j+1,2iy)}{2^j (iy)^{j+1}}
\end{equation}

where $\gamma(s,x)$ denotes the lower incomplete gamma function.

\section{Relationship Between the Integrals}

The two integrals are related through a substitution transformation. Starting with the second integral, we can derive its formula using the approach similar to the first integral.

\section{Proof of Lemma 2}

Starting with the integral:
\begin{equation}
\int_0^1 \left(\frac{1-x}{2}\right)^j e^{ixy} \, dx
\end{equation}

We apply the substitution $t = \frac{1-x}{2}$, which gives $x = 1-2t$ and $dx = -2dt$. The limits change from $x \in [0,1]$ to $t \in [1/2,0]$, which we adjust:

\begin{align}
\int_0^1 \left(\frac{1-x}{2}\right)^j e^{ixy} \, dx &= \int_{1/2}^0 t^j e^{iy(1-2t)} (-2dt)\\
&= 2e^{iy}\int_0^{1/2} t^j e^{-2iyt} \, dt
\end{align}

Next, we use the substitution $u = 2iyt$, which gives $t = \frac{u}{2iy}$ and $dt = \frac{du}{2iy}$:

\begin{align}
2e^{iy}\int_0^{1/2} t^j e^{-2iyt} \, dt &= 2e^{iy}\int_0^{iy} \left(\frac{u}{2iy}\right)^j e^{-u} \frac{du}{2iy}\\
&= \frac{2e^{iy}}{(2iy)^{j+1}}\int_0^{2iy} u^j e^{-u} \, du\\
&= \frac{e^{iy}\gamma(j+1,2iy)}{2^j(iy)^{j+1}}
\end{align}

where we've used the definition of the lower incomplete gamma function:
\begin{equation}
\gamma(s,x) = \int_0^x t^{s-1}e^{-t} \, dt
\end{equation}

\section{Connection to the Lower Incomplete Gamma Function}

The relationship between the results can be further understood through the connection between the partial exponential sum $E_k(x)$ and the incomplete gamma function $\gamma(s,x)$:

\begin{equation}
\gamma(j+1,z) = j!\left(1 - e^{-z}\sum_{k=0}^j \frac{z^k}{k!}\right) = j!\left(1 - e^{-z}E_j(z)\right)
\end{equation}

This relationship provides the theoretical bridge between the results of both lemmas.

\end{document}
