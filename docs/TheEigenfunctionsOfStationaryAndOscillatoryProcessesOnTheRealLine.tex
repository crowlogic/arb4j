\documentclass{article}
\usepackage[english]{babel}
\usepackage{geometry,amsmath,amssymb,latexsym}
\geometry{letterpaper}

%%%%%%%%%% Start TeXmacs macros
\newenvironment{proof}{\noindent\textbf{Proof\ }}{\hspace*{\fill}$\Box$\medskip}
\newtheorem{definition}{Definition}
\newtheorem{lemma}{Lemma}
\newtheorem{theorem}{Theorem}
%%%%%%%%%% End TeXmacs macros

\begin{document}

\section*{Introduction}

Oscillatory processes generalize stationary stochastic processes by allowing
their spectral properties to evolve over time. Central to this representation
is the gain function $A (t, \omega)$, a complex-valued function that works in
conjunction with an underlying spectral density $S (\omega)$ to produce
time-varying spectral characteristics. The magnitude $\lvert A (t, \omega)
\rvert$ scales the spectral power at each frequency and time, while the
argument $\arg A (t, \omega)$ introduces frequency-dependent phase shifts. The
effective spectral density at time $t$ becomes $\lvert A (t, \omega) \rvert^2
S (\omega)$, showing how the gain function and underlying spectral density
work together multiplicatively.

\begin{definition}[Stationary Process]
  A stochastic process $\{X (t), \hspace{0.17em} t \in \mathbb{R}\}$ is
  stationary when its covariance $R (s, t)$ depends only on the lag: $R (s, t)
  = R (t - s)$ for all $s, t \in \mathbb{R}$.
\end{definition}

\begin{definition}[Complex orthogonal random measure]
  Let $(E, \mathcal{E})$ be a measurable space. A complex orthogonal random
  measure is a map $\Phi : \mathcal{E} \to L^2 (\Omega ; \mathbb{C})$ such
  that:
  \begin{enumerate}
    \item (Null and $\sigma$-additivity in $L^2$) $\Phi (\varnothing) = 0$,
    $\Phi (A \cup B) = \Phi (A) + \Phi (B)$ for disjoint $A, B \in
    \mathcal{E}$, and for any disjoint sequence $(A_n)_{n \ge 1} \subset
    \mathcal{E}$,
    \[ \Phi \hspace{-0.17em} \left( \bigcup_{n \ge 1} A_n \right) = \sum_{n
       \ge 1} \Phi (A_n)  \quad \text{in } L^2 . \]
    \item (Zero mean and covariance) There exists a finite measure $\mu$ on
    $(E, \mathcal{E})$ such that, for all $A, B \in \mathcal{E}$,
    \[ \mathbb{E} [\Phi (A)] = 0, \qquad \mathbb{E} \left[ \Phi (A)
       \hspace{0.17em} \overline{\Phi (B)} \right] = \mu (A \cap B) . \]
  \end{enumerate}
  In particular, for all $A \in \mathcal{E}$, $\mathbb{E} [| \Phi (A) |^2] =
  \mu (A)$, and for disjoint $A, B$ the increments are orthogonal in $L^2$.
\end{definition}

\begin{theorem}[Spectral Representation of Oscillatory Processes]
  A realization of an oscillatory process $Z (t)$ is one that satisfies
  \begin{equation}
    \label{eq:osc-rep} Z (t) \hspace{0.27em} = \hspace{0.27em} \int_{-
    \infty}^{\infty} A_t (\omega)  \hspace{0.17em} e^{i \omega t} 
    \hspace{0.17em} d \Phi (\omega) \hspace{0.27em} = \hspace{0.27em} \int_{-
    \infty}^{\infty} h (t, u)  \hspace{0.17em} X (u)  \hspace{0.17em} du,
  \end{equation}
  where $A_t (\omega)$ is a gain function and $\Phi$ is a complex orthogonal
  random measure. The stationary reference process is
  \begin{equation}
    \label{eq:X-stationary} X (u) \hspace{0.27em} = \hspace{0.27em} \int_{-
    \infty}^{\infty} e^{i \omega u}  \hspace{0.17em} d \Phi (\omega) .
  \end{equation}
  In the sense of Priestley's canonical definition, the oscillatory kernel $h$
  and the gain $A_t$ form a Fourier pair (in the sense of distributions) with
  the convention
  \begin{equation}
    \label{eq:h-A-pair} h (t, u) \hspace{0.27em} = \hspace{0.27em} \frac{1}{2
    \pi}  \int_{- \infty}^{\infty} A_t (\lambda)  \hspace{0.17em} e^{i \lambda
    (t - u)}  \hspace{0.17em} d \lambda, \qquad A_t (\omega) \hspace{0.27em} =
    \hspace{0.27em} \int_{- \infty}^{\infty} h (t, u)  \hspace{0.17em} e^{- i
    \omega (t - u)}  \hspace{0.17em} du.
  \end{equation}
  If $Z$ is real-valued, the conjugate symmetry conditions hold:
  \begin{equation}
    \label{eq:real-sym} A_t (\omega) \hspace{0.27em} = \hspace{0.27em}
    A_t^{\ast}  (- \omega), \qquad d \Phi (- \omega) \hspace{0.27em} =
    \hspace{0.27em} d \Phi^{\ast} (\omega) .
  \end{equation}
\end{theorem}

\begin{proof}
  Using \eqref{eq:X-stationary} and Fubini/Tonelli in $L^2$,
  
  \begin{align*}
    Z (t) & = \int_{- \infty}^{\infty} h (t, u)  \hspace{0.17em} X (u) 
    \hspace{0.17em} du = \int_{- \infty}^{\infty} h (t, u) \left( \int_{-
    \infty}^{\infty} e^{i \omega u}  \hspace{0.17em} d \Phi (\omega) \right)
    du\\
    & = \int_{- \infty}^{\infty} \int_{- \infty}^{\infty} h (t, u) 
    \hspace{0.17em} e^{i \omega u}  \hspace{0.17em} du \hspace{0.17em} d \Phi
    (\omega) .
  \end{align*}
  
  By the canonical Fourier relation \eqref{eq:h-A-pair},
  \[ \int_{- \infty}^{\infty} h (t, u)  \hspace{0.17em} e^{i \omega u} 
     \hspace{0.17em} du = \frac{1}{2 \pi}  \int_{- \infty}^{\infty}
     \hspace{-0.17em} \hspace{-0.17em} \hspace{-0.17em} A_t (\lambda) 
     \hspace{0.17em} e^{i \lambda t} \left( \int_{- \infty}^{\infty} e^{i
     (\omega - \lambda) u}  \hspace{0.17em} du \right) d \lambda = A_t
     (\omega)  \hspace{0.17em} e^{i \omega t} . \]
  Therefore $Z (t) = \int A_t (\omega)  \hspace{0.17em} e^{i \omega t} 
  \hspace{0.17em} d \Phi (\omega)$, proving \eqref{eq:osc-rep}.
  Real-valuedness follows from \eqref{eq:real-sym} by a standard change of
  variables.
\end{proof}

\begin{theorem}[Eigenfunction Property for Stationary Processes]
  Let $R (\tau)$ be a stationary covariance function and define the integral
  operator
  \begin{equation}
    (Kf) (t) \hspace{0.27em} = \hspace{0.27em} \int_{- \infty}^{\infty} R (t -
    s)  \hspace{0.17em} f (s)  \hspace{0.17em} ds.
  \end{equation}
  Then
  \begin{equation}
    K \hspace{0.17em} e^{i \omega t} \hspace{0.27em} = \hspace{0.27em} S
    (\omega)  \hspace{0.17em} e^{i \omega t},
  \end{equation}
  where the eigenvalue is the spectral density
  \begin{equation}
    S (\omega) \hspace{0.27em} = \hspace{0.27em} \int_{- \infty}^{\infty} R
    (\tau)  \hspace{0.17em} e^{- i \omega \tau}  \hspace{0.17em} d \tau .
  \end{equation}
\end{theorem}

\begin{proof}
  \begin{align*}
    (Ke^{i \omega \cdot}) (t) & = \int_{- \infty}^{\infty} R (t - s) 
    \hspace{0.17em} e^{i \omega s}  \hspace{0.17em} ds = \int_{-
    \infty}^{\infty} R (\tau)  \hspace{0.17em} e^{i \omega (t - \tau)} 
    \hspace{0.17em} d \tau\\
    & = e^{i \omega t}  \int_{- \infty}^{\infty} R (\tau)  \hspace{0.17em}
    e^{- i \omega \tau}  \hspace{0.17em} d \tau = S (\omega)  \hspace{0.17em}
    e^{i \omega t} .
  \end{align*}
\end{proof}

\begin{theorem}[Eigenfunction Property for Oscillatory Processes]
  Assume absolute continuity: the spectral measure $dF (\omega) = S (\omega) 
  \hspace{0.17em} d \omega$ with $S (\omega) \ge 0$. Let
  \begin{equation}
    C (s, t) \hspace{0.27em} = \hspace{0.27em} \int_{- \infty}^{\infty} A_s
    (\omega)  \hspace{0.17em} A_t^{\ast} (\omega)  \hspace{0.17em} S (\omega) 
    \hspace{0.17em} d \omega, \qquad (Kf) (t) \hspace{0.27em} =
    \hspace{0.27em} \int_{- \infty}^{\infty} C (t, s)  \hspace{0.17em} f (s) 
    \hspace{0.17em} ds.
  \end{equation}
  Define the oscillatory functions
  \begin{equation}
    \phi (t, \omega) \hspace{0.27em} = \hspace{0.27em} A_t (\omega) 
    \hspace{0.17em} e^{i \omega t} .
  \end{equation}
  Suppose the time-orthogonality identity (in the sense of distributions)
  \begin{equation}
    \label{eq:time-orth} \int_{- \infty}^{\infty} A_s^{\ast} (\lambda) 
    \hspace{0.17em} A_s (\omega)  \hspace{0.17em} e^{i \omega s} 
    \hspace{0.17em} ds \hspace{0.27em} = \hspace{0.27em} 2 \pi \hspace{0.17em}
    \delta (\omega - \lambda) .
  \end{equation}
  Then, for each $\omega$,
  \begin{equation}
    \label{eq:Kphi-eig} (K \phi (\cdot, \omega)) (t) \hspace{0.27em} =
    \hspace{0.27em} S (\omega)  \hspace{0.17em} \phi (t, \omega) .
  \end{equation}
\end{theorem}

\begin{proof}
  \begin{align*}
    (K \phi (\cdot, \omega)) (t) & = \int_{- \infty}^{\infty} C (t, s) 
    \hspace{0.17em} \phi (s, \omega)  \hspace{0.17em} ds\\
    & = \int_{- \infty}^{\infty} \left( \int_{- \infty}^{\infty} A_t
    (\lambda) \hspace{0.17em} A_s^{\ast} (\lambda) \hspace{0.17em} S (\lambda)
    \hspace{0.17em} d \lambda \right) A_s (\omega)  \hspace{0.17em} e^{i
    \omega s}  \hspace{0.17em} ds\\
    & = \int_{- \infty}^{\infty} A_t (\lambda)  \hspace{0.17em} S (\lambda)
    \hspace{0.17em} \left[ \int_{- \infty}^{\infty} A_s^{\ast} (\lambda)
    \hspace{0.17em} A_s (\omega) \hspace{0.17em} e^{i \omega s} 
    \hspace{0.17em} ds \right] d \lambda\\
    & \overset{\eqref{eq:time-orth}}{=} \int_{- \infty}^{\infty} A_t
    (\lambda)  \hspace{0.17em} S (\lambda)  \hspace{0.17em} (2 \pi) 
    \hspace{0.17em} \delta (\omega - \lambda)  \hspace{0.17em} d \lambda\\
    & = 2 \pi \hspace{0.17em} A_t (\omega)  \hspace{0.17em} S (\omega)
    \hspace{0.27em} = \hspace{0.27em} S (\omega)  \hspace{0.17em} \phi (t,
    \omega),
  \end{align*}
  
  where the last equality uses $\phi (t, \omega) = A_t (\omega) e^{i \omega
  t}$ and the $2 \pi$ factor matches the Fourier normalization implicit in
  \eqref{eq:time-orth} and \eqref{eq:h-A-pair}.
\end{proof}

\begin{lemma}[Orthogonality Property]
  With the Fourier convention used above,
  \[ \int_{- \infty}^{\infty} A_s^{\ast} (\lambda)  \hspace{0.17em} A_s
     (\omega)  \hspace{0.17em} e^{i \omega s}  \hspace{0.17em} ds
     \hspace{0.27em} = \hspace{0.27em} 2 \pi \hspace{0.17em} \delta (\lambda -
     \omega) . \]
\end{lemma}

\begin{proof}
  For the orthogonal random measure $\Phi$,
  \[ \mathbb{E} \hspace{-0.17em} \left[ d \Phi (\lambda) \hspace{0.17em} d
     \Phi^{\ast} (\omega) \right] \hspace{0.27em} = \hspace{0.27em} 2 \pi
     \hspace{0.17em} \delta (\lambda - \omega)  \hspace{0.17em} S (\lambda) 
     \hspace{0.17em} d \lambda, \]
  under the absolute continuity assumption $dF (\omega) = S (\omega) 
  \hspace{0.17em} d \omega$ and the chosen Fourier constants. The
  representation
  \[ Z (t) \hspace{0.27em} = \hspace{0.27em} \int_{- \infty}^{\infty} A_t
     (\omega)  \hspace{0.17em} e^{i \omega t}  \hspace{0.17em} d \Phi
     (\omega), \]
  combined with this covariance structure, yields the stated
  time-orthogonality identity for the modulating amplitudes, consistent with
  the normalization used in \eqref{eq:h-A-pair}.
\end{proof}

\begin{theorem}[Real-Valued Oscillatory Processes]
  The process $Z (t)$ is real-valued if and only if
  \begin{equation}
    A_t (\omega) \hspace{0.27em} = \hspace{0.27em} A_t^{\ast}  (- \omega) 
    \qquad \text{and} \qquad d \Phi (- \omega) \hspace{0.27em} =
    \hspace{0.27em} d \Phi^{\ast} (\omega) .
  \end{equation}
\end{theorem}

\begin{proof}
  Compute
  \[ Z^{\ast} (t) \hspace{0.27em} = \hspace{0.27em} \int_{- \infty}^{\infty}
     A_t^{\ast} (\omega)  \hspace{0.17em} e^{- i \omega t}  \hspace{0.17em} d
     \Phi^{\ast} (\omega) . \]
  Set $\omega = - \nu$ so $d \omega = - d \nu$, then
  \[ Z^{\ast} (t) \hspace{0.27em} = \hspace{0.27em} \int_{- \infty}^{\infty}
     A_t^{\ast}  (- \nu)  \hspace{0.17em} e^{i \nu t}  \hspace{0.17em} d
     \Phi^{\ast}  (- \nu) \hspace{0.27em} = \hspace{0.27em} \int_{-
     \infty}^{\infty} A_t^{\ast}  (- \omega)  \hspace{0.17em} e^{i \omega t} 
     \hspace{0.17em} d \Phi^{\ast}  (- \omega) . \]
  Thus $Z (t) = Z^{\ast} (t)$ for all $t$ holds if and only if $A_t (\omega) =
  A_t^{\ast}  (- \omega)$ and $d \Phi (\omega) = d \Phi^{\ast}  (- \omega)$
  for all $\omega$. The converse direction is immediate by substitution.
\end{proof}

\begin{theorem}[Eigenfunction Conjugate Pairs]
  With $\phi (t, \omega) = A_t (\omega) e^{i \omega t}$ and $A_t (\omega) =
  A_t^{\ast}  (- \omega)$,
  \[ \phi^{\ast} (t, \omega) \hspace{0.27em} = \hspace{0.27em} \phi (t, -
     \omega) . \]
\end{theorem}

\begin{proof}
  \[ \phi^{\ast} (t, \omega) \hspace{0.27em} = \hspace{0.27em} \left( A_t
     (\omega) \hspace{0.17em} e^{i \omega t} \right)^{\ast} \hspace{0.27em} =
     \hspace{0.27em} A_t^{\ast} (\omega)  \hspace{0.17em} e^{- i \omega t}
     \hspace{0.27em} = \hspace{0.27em} A_t  (- \omega)  \hspace{0.17em} e^{- i
     \omega t} \hspace{0.27em} = \hspace{0.27em} A_t  (- \omega) 
     \hspace{0.17em} e^{i (- \omega) t} \hspace{0.27em} = \hspace{0.27em} \phi
     (t, - \omega) . \]
\end{proof}

\begin{theorem}[Filter Kernel: Dual Fourier Formula]
  With the Fourier convention fixed above,
  \[ h (t, u) \hspace{0.27em} = \hspace{0.27em} \frac{1}{2 \pi}  \int_{-
     \infty}^{\infty} A_t (\omega)  \hspace{0.17em} e^{i \omega (t - u)} 
     \hspace{0.17em} d \omega \hspace{0.27em} = \hspace{0.27em} \frac{1}{2
     \pi}  \int_{- \infty}^{\infty} \phi (t, \omega)  \hspace{0.17em} e^{- i
     \omega u}  \hspace{0.17em} d \omega . \]
\end{theorem}

\begin{proof}
  \[ \frac{1}{2 \pi}  \int_{- \infty}^{\infty} \phi (t, \omega) 
     \hspace{0.17em} e^{- i \omega u}  \hspace{0.17em} d \omega
     \hspace{0.27em} = \hspace{0.27em} \frac{1}{2 \pi}  \int_{-
     \infty}^{\infty} A_t (\omega)  \hspace{0.17em} e^{i \omega t} 
     \hspace{0.17em} e^{- i \omega u}  \hspace{0.17em} d \omega
     \hspace{0.27em} = \hspace{0.27em} \frac{1}{2 \pi}  \int_{-
     \infty}^{\infty} A_t (\omega)  \hspace{0.17em} e^{i \omega (t - u)} 
     \hspace{0.17em} d \omega . \]
\end{proof}

\begin{theorem}[Inverse Relations]
  \begin{equation}
    A_t (\omega) \hspace{0.27em} = \hspace{0.27em} \int_{- \infty}^{\infty} h
    (t, u)  \hspace{0.17em} e^{- i \omega (t - u)}  \hspace{0.17em} du, \qquad
    \phi (t, \omega) \hspace{0.27em} = \hspace{0.27em} \int_{-
    \infty}^{\infty} h (t, u)  \hspace{0.17em} e^{- i \omega u} 
    \hspace{0.17em} du.
  \end{equation}
\end{theorem}

\begin{proof}
  Using the dual formula and the identity $\int_{- \infty}^{\infty} e^{i
  (\lambda - \omega) u}  \hspace{0.17em} du = 2 \pi \hspace{0.17em} \delta
  (\lambda - \omega)$,
  
  \begin{align*}
    \int_{- \infty}^{\infty} h (t, u)  \hspace{0.17em} e^{- i \omega (t - u)} 
    \hspace{0.17em} du & = \int_{- \infty}^{\infty} \left[ \frac{1}{2 \pi} 
    \int_{- \infty}^{\infty} A_t (\lambda) \hspace{0.17em} e^{i \lambda (t -
    u)}  \hspace{0.17em} d \lambda \right] e^{- i \omega (t - u)} 
    \hspace{0.17em} du\\
    & = \frac{1}{2 \pi}  \int_{- \infty}^{\infty} A_t (\lambda) 
    \hspace{0.17em} e^{i \lambda t} e^{- i \omega t} \left( \int_{-
    \infty}^{\infty} e^{- i (\lambda - \omega) u}  \hspace{0.17em} du \right)
    d \lambda\\
    & = \frac{1}{2 \pi}  \int_{- \infty}^{\infty} A_t (\lambda) 
    \hspace{0.17em} e^{i \lambda t} e^{- i \omega t}  \hspace{0.17em} 2 \pi
    \hspace{0.17em} \delta (\lambda - \omega)  \hspace{0.17em} d \lambda\\
    & = A_t (\omega) .
  \end{align*}
  
  The formula for $\phi (t, \omega)$ follows by multiplying both sides by
  $e^{i \omega t}$ or directly from the dual formula.
\end{proof}

\end{document}
