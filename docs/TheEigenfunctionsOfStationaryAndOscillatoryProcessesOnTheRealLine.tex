\documentclass{article}
\usepackage[english]{babel}
\usepackage{geometry,amssymb,latexsym}
\geometry{letterpaper}

%%%%%%%%%% Start TeXmacs macros
\newcommand{\cdummy}{\cdot}
\newcommand{\tmaffiliation}[1]{\\ #1}
\newenvironment{proof}{\noindent\textbf{Proof\ }}{\hspace*{\fill}$\Box$\medskip}
\newtheorem{definition}{Definition}
\newtheorem{lemma}{Lemma}
\newtheorem{theorem}{Theorem}
%%%%%%%%%% End TeXmacs macros

\begin{document}

\title{Oscillatory Processes}

\author{
  Stephen Crowley
  \tmaffiliation{August 7, 2025}
}

\maketitle

\begin{definition}[Stationary Process]
  A stochastic process $\{X (t), t \in \mathbb{R}\}$ is stationary when $R (s,
  t) = R (t - s)$ for all $s, t \in \mathbb{R}$.
\end{definition}

\begin{theorem}[Filter Representation of Nonstationary Process]
  Oscillatory processes $Z (t)$ satisfy
  \begin{equation}
    Z (t) = \int_{- \infty}^{\infty} A_t (\omega) e^{i \omega t} d \Phi
    (\omega) = \int_{- \infty}^{\infty} h (t, u) X (u) du
  \end{equation}
  where $A_t (\omega)$ is a gain function satisfying
  \begin{equation}
    A_t (\omega) = A_t^{\ast}  (- \omega)
  \end{equation}
  and $\Phi (\omega)$ is an orthogonal increment process.
  \begin{equation}
    X (u) = \int_{- \infty}^{\infty} e^{i \omega u} d \Phi (\omega)
  \end{equation}
\end{theorem}

\begin{proof}
  
  \begin{equation}
    \begin{array}{ll}
      Z (t) & = \int_{- \infty}^{\infty} h (t, u) X (u) du\\
      & = \int_{- \infty}^{\infty} h (t, u) \int_{- \infty}^{\infty} e^{i
      \omega u} d \Phi (\omega) du\\
      & = \int_{- \infty}^{\infty} \int_{- \infty}^{\infty} h (t, u) e^{i
      \omega u} dud \Phi (\omega)\\
      & = \int_{- \infty}^{\infty} \frac{1}{2 \pi}  \int_{- \infty}^{\infty}
      \int_{- \infty}^{\infty} A_t (\lambda) e^{i \lambda (t - u)} d \lambda
      \hspace{0.27em} e^{i \omega u} dud \Phi (\omega)\\
      & = \int_{- \infty}^{\infty} \frac{1}{2 \pi}  \int_{- \infty}^{\infty}
      A_t (\lambda) \int_{- \infty}^{\infty} e^{i \lambda (t - u)} e^{i \omega
      u} du d \lambda d \Phi (\omega)\\
      & = \int_{- \infty}^{\infty} \frac{1}{2 \pi}  \int_{- \infty}^{\infty}
      A_t (\lambda) e^{i \lambda t} \int_{- \infty}^{\infty} e^{i (\omega -
      \lambda) u} dud \lambda d \Phi (\omega)\\
      & = \int_{- \infty}^{\infty} \frac{1}{2 \pi}  \int_{- \infty}^{\infty}
      A_t (\lambda) e^{i \lambda t} 2 \pi \delta (\omega - \lambda) d \lambda
      d \Phi (\omega)\\
      & = \int_{- \infty}^{\infty} A_t (\omega) e^{i \omega t} d \Phi
      (\omega)
    \end{array}
  \end{equation}
  where the interchanges are justified by quadratic integrability of the
  time-dependent gain functions $A_t (\lambda)$ with respect to the spectral
  measure $S (\lambda) = d F (\lambda) \forall t \in \mathbb{R}$
\end{proof}

\begin{theorem}[Eigenfunction Property for Stationary Processes]
  Let $R (\tau)$ be a stationary covariance function. Let the corresponding
  integral coariance operator be defined
  \begin{equation}
    (Kf) (t) = \int_{- \infty}^{\infty} R (t - s) f (s) ds
  \end{equation}
  then
  \begin{equation}
    Ke^{i \omega t} = S (\omega) e^{i \omega t}
  \end{equation}
  where the eigenvalue is the correponding element of the spectral density
  \begin{equation}
    S (\omega) = \int_{- \infty}^{\infty} R (\tau) e^{- i \omega \tau} d \tau
  \end{equation}
\end{theorem}

\begin{proof}
  
  \begin{equation}
    \begin{array}{ll}
      (Kf) (t) & = \int_{- \infty}^{\infty} R (t - s) e^{i \omega s} ds\\
      & = \int_{- \infty}^{\infty} R (\tau) e^{i \omega (t - \tau)} d \tau\\
      & = e^{i \omega t}  \int_{- \infty}^{\infty} R (\tau) e^{- i \omega
      \tau} d \tau\\
      & = S (\omega) e^{i \omega t}
    \end{array}
  \end{equation}
  
\end{proof}

\begin{theorem}[Eigenfunction Property for Oscillatory Processes]
  Let
  \begin{equation}
    C (s, t) = \int_{- \infty}^{\infty} A_s (\omega) A_t^{\ast} (\omega) dF
    (\omega)
  \end{equation}
  and
  \begin{equation}
    (Kf) (t) = \int_{- \infty}^{\infty} C (t, s) f (s) ds
  \end{equation}
  then the oscillatory functions
  \begin{equation}
    \phi (t, \omega) = A_t (\omega) e^{i \omega t}
  \end{equation}
  are eigenfunction of $K$ with eigenvalues $S (\lambda) = dF (\omega) \forall
  \omega$
  \begin{equation}
    (K \phi (\cdummy, \omega)) (t) = \phi_t (\lambda) S (\lambda)
  \end{equation}
\end{theorem}

\begin{proof}
  
  \begin{equation}
    \begin{array}{ll}
      K \phi (\cdummy, \omega) (t) & = \int_{- \infty}^{\infty} C (t, s) \phi
      (s, \omega) ds\\
      & = \int_{- \infty}^{\infty} \left( \int_{- \infty}^{\infty} A_t
      (\lambda) A_s^{\ast} (\lambda) dF (\lambda) \right) A_s (\omega) e^{i
      \omega s} ds\\
      & = \int_{- \infty}^{\infty} A_t (\lambda) \left[ \int_{-
      \infty}^{\infty} A_s^{\ast} (\lambda) A_s (\omega) e^{i \omega s} ds
      \right] dF (\lambda)\\
      & = \int_{- \infty}^{\infty} A_t (\lambda) \delta (\lambda - \omega) dF
      (\lambda)\\
      & = A_t (\omega) dF (\omega)\\
      & = \phi (t, \omega) dF (\omega)
    \end{array}
  \end{equation}
  
\end{proof}

\begin{lemma}[Orthogonality Property]
  \[ \int_{- \infty}^{\infty} A_s^{\ast} (\lambda) A_s (\omega) e^{i \omega s}
     ds = \delta (\lambda - \omega) \]
\end{lemma}

\begin{proof}
  The orthogonality of $\Phi (\omega)$ is
  \[ \mathbb{E} [d \Phi (\lambda) d \Phi^{\ast} (\omega)] = \delta (\lambda -
     \omega) dF (\lambda) . \]
  The representation
  \[ Z (t) = \int_{- \infty}^{\infty} A_t (\omega) e^{i \omega t} d \Phi
     (\omega) \]
  with this covariance property, forces the stated orthogonality among the
  time-varying modulating amplitudes.
\end{proof}

\begin{theorem}[Real-Valued Oscillatory Processes]
  The process $Z (t)$ is real-valued if and only if
  \begin{equation}
    A_t (\omega) = A_t^{\ast}  (- \omega)
  \end{equation}
  and
  \begin{equation}
    d \Phi (- \omega) = d \Phi^{\ast} (\omega)
  \end{equation}
\end{theorem}

\begin{proof}
  Compute
  \[ Z^{\ast} (t) = \int_{- \infty}^{\infty} A_t^{\ast} (\omega) e^{- i \omega
     t} d \Phi^{\ast} (\omega) . \]
  Set $\omega = - \nu$, so $d \omega = - d \nu$,
  \[ Z^{\ast} (t) = \int_{+ \infty}^{- \infty} A_t^{\ast}  (- \nu) e^{i \nu t}
     d \Phi^{\ast}  (- \nu)  (- d \nu) = \int_{- \infty}^{\infty} A_t^{\ast} 
     (- \omega) e^{i \omega t} d \Phi^{\ast}  (- \omega) . \]
  For $Z (t)$ to be real-valued,
  \[ Z (t) = Z^{\ast} (t) \]
  for all $t$, so it is necessary that for all $\omega$,
  \[ A_t (\omega) = A_t^{\ast}  (- \omega), \qquad d \Phi (\omega) = d
     \Phi^{\ast}  (- \omega) . \]
  If these hold, then
  \[ Z^{\ast} (t) = \int_{- \infty}^{\infty} A_t^{\ast}  (- \omega) e^{i
     \omega t} d \Phi^{\ast}  (- \omega) = \int_{- \infty}^{\infty} A_t
     (\omega) e^{i \omega t} d \Phi (\omega) = Z (t) . \]
\end{proof}

\begin{theorem}[Eigenfunction Conjugate Pairs]
  $\phi^{\ast} (t, \omega) = \phi (t, - \omega)$.
\end{theorem}

\begin{proof}
  \[ \phi^{\ast} (t, \omega) = [A_t (\omega) e^{i \omega t}]^{\ast} =
     A_t^{\ast} (\omega) e^{- i \omega t} \]
  By the conjugate symmetry property,
  \[ A_t^{\ast} (\omega) e^{- i \omega t} = A_t  (- \omega) e^{- i \omega t} =
     A_t  (- \omega) e^{i (- \omega) t} = \phi (t, - \omega) \]
\end{proof}

\begin{theorem}[Filter Kernel: Dual Fourier Formula]
  \[ h (t, u) = \frac{1}{2 \pi}  \int_{- \infty}^{\infty} A_t (\omega) e^{i
     \omega (t - u)} d \omega = \frac{1}{2 \pi}  \int_{- \infty}^{\infty} \phi
     (t, \omega) e^{- i \omega u} d \omega \]
\end{theorem}

\begin{proof}
  \[ \frac{1}{2 \pi}  \int_{- \infty}^{\infty} \phi (t, \omega) e^{- i \omega
     u} d \omega = \frac{1}{2 \pi}  \int_{- \infty}^{\infty} [A_t (\omega)
     e^{i \omega t}] e^{- i \omega u} d \omega = \frac{1}{2 \pi}  \int_{-
     \infty}^{\infty} A_t (\omega) e^{i \omega (t - u)} d \omega \]
\end{proof}

\begin{theorem}[Inverse Relations]
  \begin{equation}
    A_t (\omega) = \int_{- \infty}^{\infty} h (t, u) e^{- i \omega (t - u)} du
  \end{equation}
  \begin{equation}
    \phi (t, \omega) = \int_{- \infty}^{\infty} h (t, u) e^{- i \omega u} du
  \end{equation}
\end{theorem}

\begin{proof}
  \[ \int_{- \infty}^{\infty} h (t, u) e^{- i \omega (t - u)} du = \int_{-
     \infty}^{\infty} \frac{1}{2 \pi}  \int_{- \infty}^{\infty} A_t (\lambda)
     e^{i \lambda (t - u)} d \lambda \hspace{0.27em} e^{- i \omega (t - u)} du
  \]
  \[ = \frac{1}{2 \pi}  \int_{- \infty}^{\infty} A_t (\lambda) \left[ \int_{-
     \infty}^{\infty} e^{i \lambda (t - u)} e^{- i \omega (t - u)} du \right]
     d \lambda \]
  \[ = \frac{1}{2 \pi}  \int_{- \infty}^{\infty} A_t (\lambda) e^{i \lambda t}
     e^{- i \omega t} \left[ \int_{- \infty}^{\infty} e^{- i (\lambda -
     \omega) u} du \right] d \lambda = \frac{1}{2 \pi}  \int_{-
     \infty}^{\infty} A_t (\lambda) e^{i \lambda t} e^{- i \omega t} 2 \pi
     \delta (\lambda - \omega) d \lambda \]
  \[ = \int_{- \infty}^{\infty} A_t (\lambda) e^{i \lambda t} e^{- i \omega t}
     \delta (\lambda - \omega) d \lambda = A_t (\omega) e^{i \omega t} e^{- i
     \omega t} = A_t (\omega) \]
  The formula for $\phi (t, \omega)$ is found similarly.
\end{proof}

\

\end{document}
