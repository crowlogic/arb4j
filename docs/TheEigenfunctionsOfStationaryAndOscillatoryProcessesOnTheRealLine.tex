\documentclass{article}
\usepackage[english]{babel}
\usepackage{geometry,amsmath,amssymb,latexsym}
\geometry{letterpaper}

%%%%%%%%%% Start TeXmacs macros
\newcommand{\tmaffiliation}[1]{\\ #1}
\newcommand{\tmtextbf}[1]{\text{{\bfseries{#1}}}}
\newenvironment{proof}{\noindent\textbf{Proof\ }}{\hspace*{\fill}$\Box$\medskip}
\newtheorem{definition}{Definition}
\newtheorem{lemma}{Lemma}
\newtheorem{theorem}{Theorem}
%%%%%%%%%% End TeXmacs macros

\begin{document}

\title{The Eigenfunctions of Stationary and Oscillatory Processes on the Real
Line}

\author{
  Stephen Crowley
  \tmaffiliation{August 5, 2025}
}

\date{}

\maketitle

\begin{definition}[Stationary Process]
  A stochastic process $\{X (t), t \in \mathbb{R}\}$ is called stationary if
  its covariance function satisfies
  \[ R (s, t) = R (t - s) \]
  for all $s, t \in \mathbb{R}$.
\end{definition}

\begin{definition}[Oscillatory Process (Priestley)]
  A stochastic process $\{X (t), t \in \mathbb{R}\}$ is called oscillatory if
  it possesses an evolutionary spectral representation
  \[ X (t) = \int_{- \infty}^{\infty} A (t, \omega) e^{i \omega t} dZ (\omega)
  \]
  where $A (t, \omega)$ is the evolutionary amplitude function and $Z
  (\omega)$ is an orthogonal increment process.
\end{definition}

\begin{theorem}[Eigenfunction Property for Stationary Processes]
  Let $\{X (t), t \in \mathbb{R}\}$ be a stationary process with covariance
  function $R (\tau)$ and covariance operator
  \[ (Kf) (t) = \int_{- \infty}^{\infty} R (t - s) f (s) ds \]
  Then the complex exponentials $e^{i \omega t}$ are eigenfunctions of $K$
  with eigenvalues equal to the power spectral density $S (\omega)$.
\end{theorem}

\begin{proof}
  Consider the action of $K$ on $e^{i \omega t}$:
  \[ (Ke^{i \omega t}) (t) = \int_{- \infty}^{\infty} R (t - s) e^{i \omega s}
     ds \]
  Substituting $\tau = t - s$:
  
  \begin{align*}
    & = e^{i \omega t}  \int_{- \infty}^{\infty} R (\tau) e^{- i \omega \tau}
    d \tau\\
    & = e^{i \omega t} \cdot S (\omega)
  \end{align*}
  
  where
  \[ S (\omega) = \int_{- \infty}^{\infty} R (\tau) e^{- i \omega \tau} d \tau
  \]
  is the power spectral density by the Wiener-Khintchine theorem.
\end{proof}

\begin{theorem}[Eigenfunction Property for Oscillatory Processes]
  Let $\{X (t), t \in \mathbb{R}\}$ be an oscillatory process with
  evolutionary spectral representation
  \[ X (t) = \int_{- \infty}^{\infty} A (t, \omega) e^{i \omega t} dZ (\omega)
  \]
  and covariance function
  \[ C (s, t) = \int_{- \infty}^{\infty} A (s, \omega) A^{\ast} (t, \omega) dF
     (\omega) \]
  where $F (\omega)$ is the spectral measure. Then the oscillatory functions
  \[ \phi (t, \omega) = A (t, \omega) e^{i \omega t} \]
  are eigenfunctions of the covariance operator
  \[ (Kf) (t) = \int_{- \infty}^{\infty} C (t, s) f (s) ds \]
  with eigenvalues $dF (\omega)$.
\end{theorem}

\begin{proof}
  Consider the action of $K$ on the oscillatory function $\phi (s, \omega) = A
  (s, \omega) e^{i \omega s}$:
  \[ (K \phi) (t) = \int_{- \infty}^{\infty} C (t, s) A (s, \omega) e^{i
     \omega s} ds \]
  Substitute $C (t, s) = \int A (t, \lambda) A^{\ast} (s, \lambda) dF
  (\lambda)$:
  
  \begin{align*}
    (K \phi) (t) & = \int_{- \infty}^{\infty} \left[ \int_{- \infty}^{\infty}
    A (t, \lambda) A^{\ast} (s, \lambda) dF (\lambda) \right] A (s, \omega)
    e^{i \omega s} ds\\
    & = \int_{- \infty}^{\infty} A (t, \lambda) \left[ \int_{-
    \infty}^{\infty} A^{\ast} (s, \lambda) A (s, \omega) e^{i \omega s} ds
    \right] dF (\lambda)
  \end{align*}
  
  By Fubini's theorem, the order of integration may be exchanged:
  \[ = \int_{- \infty}^{\infty} A (t, \lambda) \left[ \int_{- \infty}^{\infty}
     A^{\ast} (s, \lambda) A (s, \omega) e^{i \omega s} ds \right] dF
     (\lambda) \]
  The inner integral represents the orthogonality condition in the
  evolutionary spectral representation:
  \[ \int_{- \infty}^{\infty} A^{\ast} (s, \lambda) A (s, \omega) e^{i \omega
     s} ds = \delta (\lambda - \omega) \]
  Therefore
  \[ (K \phi) (t) = \int_{- \infty}^{\infty} A (t, \lambda) \delta (\lambda -
     \omega) dF (\lambda) = A (t, \omega) dF (\omega) = \phi (t, \omega) \cdot
     dF (\omega) \]
\end{proof}

\begin{lemma}[Orthogonality Property]
  For the evolutionary spectral representation, the orthogonality condition
  \[ \int_{- \infty}^{\infty} A^{\ast} (s, \lambda) A (s, \omega) e^{i \omega
     s} ds = \delta (\lambda - \omega) \]
  follows from the requirement that $dZ (\omega)$ be an orthogonal increment
  process.
\end{lemma}

\begin{proof}
  The orthogonality of $dZ (\omega)$ requires
  \[ \mathbb{E} [dZ (\lambda) dZ^{\ast} (\omega)] = \delta (\lambda - \omega)
     dF (\lambda) \]
  This condition, with the evolutionary spectral representation, directly
  implies the stated orthogonality property for the amplitude functions.
\end{proof}

\begin{theorem}[Real-Valued Oscillatory Processes]
  Let $Z (t)$ be a sample path realization of an oscillatory process (with
  evolutionary spectral representation)
  \begin{equation}
    X (t) = \int_{- \infty}^{\infty} A_{\lambda} (t) e^{i \lambda t} d \Phi
    (\lambda)
  \end{equation}
  where $A_t (\omega)$ is the gain function and $\Phi (\omega)$ is an
  orthogonal increment process. Then $X (t)$ is real-valued if and only if the
  following conditions hold:
  \begin{equation}
    A (t, \omega) = A^{\ast} (t, - \omega) \qquad \text{(Gain Conjugate
    Symmetry)}
  \end{equation}
  \begin{equation}
    dZ (- \omega) = dZ^{\ast} (\omega) \qquad \text{(Increment Conjugate
    Symmetry)}
  \end{equation}
\end{theorem}

\begin{proof}
  \tmtextbf{Necessity:} Assume $X (t)$ is real-valued, so
  \begin{equation}
    X (t) = X^{\ast} (t) \forall t \in \mathbb{R}
  \end{equation}
  Taking the complex conjugate of the evolutionary spectral representation:
  \begin{equation}
    X^{\ast} (t) = \left[ \int_{- \infty}^{\infty} A (t, \omega) e^{i \omega
    t} dZ (\omega) \right]^{\ast} = \int_{- \infty}^{\infty} A^{\ast} (t,
    \omega) e^{- i \omega t} dZ^{\ast} (\omega)
  \end{equation}
  Making the substitution $\omega \mapsto - \omega$ in this integral:
  \begin{equation}
    X^{\ast} (t) = \int_{- \infty}^{\infty} A^{\ast} (t, - \omega) e^{i \omega
    t} dZ^{\ast}  (- \omega)
  \end{equation}
  Since $X (t) = X^{\ast} (t)$, we have:
  \begin{equation}
    \int_{- \infty}^{\infty} A (t, \omega) e^{i \omega t} dZ (\omega) =
    \int_{- \infty}^{\infty} A^{\ast} (t, - \omega) e^{i \omega t} dZ^{\ast} 
    (- \omega)
  \end{equation}
  By the uniqueness of the evolutionary spectral representation, this equality
  holds for all $t$ if and only if:
  \begin{equation}
    A (t, \omega) = A^{\ast} (t, - \omega)
  \end{equation}
  \begin{equation}
    dZ (\omega) = dZ^{\ast}  (- \omega)
  \end{equation}
  \tmtextbf{Sufficiency:} Assume the two conjugate symmetry conditions hold.
  Then:
  \begin{equation}
    X^{\ast} (t) = \int_{- \infty}^{\infty} A^{\ast} (t, \omega) e^{- i \omega
    t} dZ^{\ast} (\omega)
  \end{equation}
  \begin{equation}
    = \int_{- \infty}^{\infty} A (t, - \omega) e^{- i \omega t} dZ (- \omega)
  \end{equation}
  Substituting $\omega \mapsto - \omega$:
  \[ X^{\ast} (t) = \int_{- \infty}^{\infty} A (t, \omega) e^{i \omega t} dZ
     (\omega) = X (t) \]
  Therefore, $X (t)$ is real-valued.
\end{proof}

\begin{theorem}[Eigenfunction Conjugate Pairs]
  Under the conditions for real-valued oscillatory processes, the
  eigenfunctions $\phi (t, \omega) = A (t, \omega) e^{i \omega t}$ satisfy the
  conjugate symmetry relation
  \begin{equation}
    \phi^{\ast} (t, \omega) = \phi (t, - \omega)
  \end{equation}
\end{theorem}

\begin{proof}
  Given that $A (t, \omega) = A^{\ast} (t, - \omega)$, we compute:
  \begin{equation}
    \begin{array}{ll}
      \phi^{\ast} (t, \omega) & = [A (t, \omega) e^{i \omega t}]^{\ast}\\
      & = A^{\ast} (t, \omega) e^{- i \omega t}\\
      & = A (t, - \omega) e^{- i \omega t} \quad \text{(by amplitude
      symmetry)}\\
      & = \phi (t, - \omega)
    \end{array}
  \end{equation}
  
\end{proof}

\begin{theorem}[Equivalence of Evolutionary Spectral and Filter
Representations]
  Let $X (t)$ be a stochastic process. The evolutionary spectral
  representation
  \begin{equation}
    X (t) = \int_{- \infty}^{\infty} A (t, \omega) e^{i \omega t} dZ (\omega)
  \end{equation}
  where $A (t, \omega)$ is the gain function and $dZ (\omega)$ is an
  orthogonal increment process, is equivalent to the time-domain filter
  representation
  \begin{equation}
    X (t) = \int_{- \infty}^{\infty} h_t  (t - s) dW (s)
  \end{equation}
  where $h_t  (t - s)$ is a time-dependent filter kernel and $dW (s)$ is an
  orthogonal increment process.
\end{theorem}

\begin{proof}
  The filter kernel $h_t  (t - s)$ is related to the gain function and
  oscillatory function by the Fourier transform relationships:
  
  \begin{align}
    h_t  (t - s) & = \int_{- \infty}^{\infty} \phi (t, \omega) e^{- i \omega
    (t - s)} d \omega \\
    & = \int_{- \infty}^{\infty} A (t, \omega) e^{i \omega t} e^{- i \omega
    (t - s)} d \omega \\
    & = \int_{- \infty}^{\infty} A (t, \omega) e^{i \omega s} d \omega 
  \end{align}
  
  where $\phi (t, \omega) = A (t, \omega) e^{i \omega t}$ is the oscillatory
  function.
  
  The inverse relationships are:
  \begin{equation}
    A (t, \omega) = \int_{- \infty}^{\infty} h_t  (t - s) e^{- i \omega s} ds
  \end{equation}
  \begin{equation}
    \phi (t, \omega) = \int_{- \infty}^{\infty} h_t (u) e^{- i \omega (t - u)}
    du
  \end{equation}
  To establish the equivalence of the two representations, substitute the
  orthogonal increment relationship $dZ (\omega) = \int_{- \infty}^{\infty}
  e^{- i \omega s} dW (s)$ into the evolutionary spectral representation:
  
  \begin{align}
    X (t) & = \int_{- \infty}^{\infty} A (t, \omega) e^{i \omega t} dZ
    (\omega) \\
    & = \int_{- \infty}^{\infty} A (t, \omega) e^{i \omega t} \left[ \int_{-
    \infty}^{\infty} e^{- i \omega s} dW (s) \right] d \omega \\
    & = \int_{- \infty}^{\infty} \left[ \int_{- \infty}^{\infty} A (t,
    \omega) e^{i \omega t} e^{- i \omega s} d \omega \right] dW (s) \\
    & = \int_{- \infty}^{\infty} \left[ \int_{- \infty}^{\infty} A (t,
    \omega) e^{i \omega (t - s)} d \omega \right] dW (s) \\
    & = \int_{- \infty}^{\infty} h_t  (t - s) dW (s) 
  \end{align}
  
  where the last equality follows from the definition of $h_t  (t - s)$ with
  $u = t - s$.
\end{proof}

\begin{theorem}[Fourier Transform Relationships]
  The gain function $A (t, \omega)$, oscillatory function $\phi (t, \omega)$,
  and filter kernel $h_t (u)$ satisfy the following Fourier transform
  relationships:
  
  \begin{align}
    A (t, \omega) & = \int_{- \infty}^{\infty} h_t  (t - s) e^{- i \omega s}
    ds \\
    \phi (t, \omega) & = A (t, \omega) e^{i \omega t} = \int_{-
    \infty}^{\infty} h_t (u) e^{- i \omega (t - u)} du \\
    h_t  (t - s) & = \int_{- \infty}^{\infty} A (t, \omega) e^{i \omega s} d
    \omega = \int_{- \infty}^{\infty} \phi (t, \omega) e^{- i \omega (t - s)}
    d \omega 
  \end{align}
\end{theorem}

\begin{proof}
  The proof establishes each transform relationship directly.
  
  For the first relationship, apply the inverse Fourier transform to $h_t  (t
  - s)$:
  
  \begin{align}
    A (t, \omega) & =\mathcal{F}_s^{- 1}  [h_t (t - s)] \\
    & = \int_{- \infty}^{\infty} h_t  (t - s) e^{- i \omega s} ds 
  \end{align}
  
  For the oscillatory function relationship, substitute the definition $\phi
  (t, \omega) = A (t, \omega) e^{i \omega t}$:
  
  \begin{align}
    \phi (t, \omega) & = A (t, \omega) e^{i \omega t} \\
    & = \left[ \int_{- \infty}^{\infty} h_t (t - s) e^{- i \omega s} ds
    \right] e^{i \omega t} \\
    & = \int_{- \infty}^{\infty} h_t  (t - s) e^{- i \omega s} e^{i \omega t}
    ds \\
    & = \int_{- \infty}^{\infty} h_t  (t - s) e^{- i \omega (s - t)} ds \\
    & = \int_{- \infty}^{\infty} h_t (u) e^{- i \omega (t - u)} du 
  \end{align}
  
  where $u = t - s$ in the last step.
  
  For the inverse relationships, apply the Fourier transform to recover $h_t 
  (t - s)$:
  
  \begin{align}
    h_t  (t - s) & =\mathcal{F}_{\omega}^{- 1}  [A (t, \omega) e^{i \omega s}]
    \\
    & = \int_{- \infty}^{\infty} A (t, \omega) e^{i \omega s} d \omega 
  \end{align}
  
  Similarly:
  
  \begin{align}
    h_t  (t - s) & =\mathcal{F}_{\omega}^{- 1}  [\phi (t, \omega) e^{- i
    \omega t}] \\
    & = \int_{- \infty}^{\infty} \phi (t, \omega) e^{- i \omega t} e^{i
    \omega (t - s)} d \omega \\
    & = \int_{- \infty}^{\infty} \phi (t, \omega) e^{- i \omega (t - s)} d
    \omega 
  \end{align}
\end{proof}

\end{document}
