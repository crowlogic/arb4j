\documentclass{article}
\usepackage{amsmath}
\usepackage{amssymb}
\usepackage{amsthm}

\newtheorem{theorem}{Theorem}
\newtheorem{lemma}{Lemma}
\newtheorem{definition}{Definition}

\begin{document}

\title{Eigenfunction Properties of Stationary and Oscillatory Stochastic Processes}
\author{}
\date{}
\maketitle

\begin{definition}[Stationary Process]
A stochastic process $\{X(t), t \in \mathbb{R}\}$ is called stationary if its covariance function satisfies $R(s,t) = R(t-s)$ for all $s,t \in \mathbb{R}$.
\end{definition}

\begin{definition}[Oscillatory Process (Priestley)]
A stochastic process $\{X(t), t \in \mathbb{R}\}$ is called oscillatory if it possesses an evolutionary spectral representation
\begin{equation}
X(t) = \int_{-\infty}^{\infty} A(t,\omega)e^{i\omega t} dZ(\omega)
\end{equation}
where $A(t,\omega)$ is the evolutionary amplitude function and $Z(\omega)$ is an orthogonal increment process.
\end{definition}

\begin{theorem}[Eigenfunction Property for Stationary Processes]
Let $\{X(t), t \in \mathbb{R}\}$ be a stationary process with covariance function $R(\tau)$ and covariance operator
\begin{equation}
(Kf)(t) = \int_{-\infty}^{\infty} R(t-s)f(s)ds
\end{equation}
Then the complex exponentials $e^{i\omega t}$ are eigenfunctions of $K$ with eigenvalues equal to the power spectral density $S(\omega)$.
\end{theorem}

\begin{proof}
Consider the action of $K$ on $e^{i\omega t}$:
\begin{align}
(Ke^{i\omega t})(t) &= \int_{-\infty}^{\infty} R(t-s)e^{i\omega s}ds
\end{align}
Substituting $\tau = t-s$:
\begin{align}
&= e^{i\omega t} \int_{-\infty}^{\infty} R(\tau)e^{-i\omega \tau}d\tau \\
&= e^{i\omega t} \cdot S(\omega)
\end{align}
where $S(\omega) = \int_{-\infty}^{\infty} R(\tau)e^{-i\omega \tau}d\tau$ is the power spectral density by the Wiener-Khintchine theorem.
\end{proof}

\begin{theorem}[Eigenfunction Property for Oscillatory Processes]
Let $\{X(t), t \in \mathbb{R}\}$ be an oscillatory process with evolutionary spectral representation
\begin{equation}
X(t) = \int_{-\infty}^{\infty} A(t,\omega)e^{i\omega t} dZ(\omega)
\end{equation}
and covariance function
\begin{equation}
C(s,t) = \int_{-\infty}^{\infty} A(s,\omega)A^*(t,\omega) dF(\omega)
\end{equation}
where $F(\omega)$ is the spectral measure. Then the oscillatory functions $\phi(t,\omega) = A(t,\omega)e^{i\omega t}$ are eigenfunctions of the covariance operator
\begin{equation}
(Kf)(t) = \int_{-\infty}^{\infty} C(t,s)f(s)ds
\end{equation}
with eigenvalues $dF(\omega)$.
\end{theorem}

\begin{proof}
Consider the action of $K$ on the oscillatory function $\phi(s,\omega) = A(s,\omega)e^{i\omega s}$:
\begin{align}
(K\phi)(t) &= \int_{-\infty}^{\infty} C(t,s)A(s,\omega)e^{i\omega s} ds \\
&= \int_{-\infty}^{\infty} \left[\int_{-\infty}^{\infty} A(t,\lambda)A^*(s,\lambda) dF(\lambda)\right] A(s,\omega)e^{i\omega s} ds
\end{align}

By Fubini's theorem, the order of integration may be exchanged:
\begin{align}
&= \int_{-\infty}^{\infty} A(t,\lambda) \left[\int_{-\infty}^{\infty} A^*(s,\lambda)A(s,\omega)e^{i\omega s} ds\right] dF(\lambda)
\end{align}

The inner integral represents the orthogonality condition in the evolutionary spectral representation. By the fundamental property of evolutionary spectral representations:
\begin{equation}
\int_{-\infty}^{\infty} A^*(s,\lambda)A(s,\omega)e^{i\omega s} ds = \delta(\lambda - \omega)
\end{equation}
where $\delta(\lambda - \omega)$ is the Dirac delta function.

Therefore:
\begin{align}
(K\phi)(t) &= \int_{-\infty}^{\infty} A(t,\lambda) \delta(\lambda - \omega) dF(\lambda) \\
&= A(t,\omega) dF(\omega) \\
&= \phi(t,\omega) \cdot \frac{dF(\omega)}{A(t,\omega)e^{i\omega t}} \cdot A(t,\omega)e^{i\omega t} \\
&= \phi(t,\omega) \cdot dF(\omega)
\end{align}

This establishes that $\phi(t,\omega) = A(t,\omega)e^{i\omega t}$ are eigenfunctions with eigenvalues $dF(\omega)$.
\end{proof}

\begin{lemma}[Orthogonality Property]
For the evolutionary spectral representation, the orthogonality condition
\begin{equation}
\int_{-\infty}^{\infty} A^*(s,\lambda)A(s,\omega)e^{i\omega s} ds = \delta(\lambda - \omega)
\end{equation}
follows from the requirement that $dZ(\omega)$ be an orthogonal increment process.
\end{lemma}

\begin{proof}
The orthogonality of $dZ(\omega)$ requires $\mathbb{E}[dZ(\lambda)dZ^*(\omega)] = \delta(\lambda - \omega)dF(\lambda)$. This condition, combined with the evolutionary spectral representation, directly implies the stated orthogonality property for the amplitude functions.
\end{proof}

\begin{theorem}[Correspondence Principle]
The eigenfunction properties of oscillatory processes reduce to those of stationary processes when the evolutionary amplitude function becomes constant: $A(t,\omega) = A(\omega)$.
\end{theorem}

\begin{proof}
When $A(t,\omega) = A(\omega)$ is independent of time, the oscillatory functions become $\phi(t,\omega) = A(\omega)e^{i\omega t}$, which are scalar multiples of the complex exponentials $e^{i\omega t}$. The covariance function reduces to
\begin{equation}
C(s,t) = \int_{-\infty}^{\infty} |A(\omega)|^2 e^{i\omega(s-t)} dF(\omega)
\end{equation}
which depends only on $s-t$, recovering the stationary case.
\end{proof}

\end{document}
