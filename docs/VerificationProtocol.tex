\documentclass{article}
\usepackage[english]{babel}
\usepackage{geometry,amsmath}
\geometry{letterpaper}

%%%%%%%%%% Start TeXmacs macros
\newcommand{\tmtextbf}[1]{\text{{\bfseries{#1}}}}
%%%%%%%%%% End TeXmacs macros

\begin{document}

\title{Mathematical Verification Protocol: Instructions for Rigorous Review}

\date{}

\maketitle

{\tableofcontents}

\section{VERIFICATION PROTOCOL}

You are tasked with verifying matheamtical statements in a latex document.
Follow these instructions \tmtextbf{exactly} and \tmtextbf{completely}.
Failure to follow any step constitutes verification failure.

\section{FORBIDDEN ACTIONS OR THOUGHTS IN THE SO-CALLED ``THINKING'' or
``REASONING'' PROCESS}

The following are \tmtextbf{STRICTLY PROHIBITED} during verification:
\begin{enumerate}
  \item \tmtextbf{NO EMULATION OF INTUITION}: Do not use phrases like "this
  seems," "I think," "appears to be," "looks like," or "feels wrong"
  
  \item \tmtextbf{NO APPEALS TO AUTHORITY}: Do not reference "standard
  theory," "well-known results," or "familiar methods" without explicit
  computation
  
  \item \tmtextbf{NO ASSUMPTIONS OR EMULATING OF KNOWING BETTER WITHOUT
  STATING DETAILED CORRECTION}: Do not assume anything is "stronger than
  necessary," "weaker than required," or "different from expected"
  
  \item \tmtextbf{NO DISMISSAL WITHOUT PROOF}: Do not mark any statement as
  incorrect without providing explicit mathematical counter-proof
  
  \item \tmtextbf{NO VAGUE PATTERN MATCHING}: Do not reject statements because
  they differ from "similar" results you may know
  
  \item \tmtextbf{NO AMBIGUITY}: Every distinguishable mathematical statement
  must receive exactly one mark
\end{enumerate}

\section{REQUIRED VERIFICATION STEPS}

For each mathematical statement, you \tmtextbf{MUST}:

\subsection{Step 1: Explicit Computation}

\begin{itemize}
  \item Write out all mathematical steps explicitly
  
  \item Show all algebraic manipulations
  
  \item Verify all claimed equalities by direct computation
  
  \item Check all claimed implications by logical deduction
\end{itemize}

\subsection{Step 2: Proof Structure Analysis}

For each theorem/proposition:
\begin{itemize}
  \item Identify the logical structure (if-then, if-and-only-if, etc.)
  
  \item Verify each direction of equivalences separately
  
  \item Check that all proof steps are mathematically valid
  
  \item Ensure no logical gaps exist in the argument
\end{itemize}

\subsection{Step 3: Condition Verification}

For each claimed condition:
\begin{itemize}
  \item Compute what the condition actually requires
  
  \item Verify necessity by showing what happens when condition fails
  
  \item Verify sufficiency by showing condition implies desired result
  
  \item Do not assume conditions are "too strong" or "too weak" without proof
\end{itemize}

\subsection{Step 4: Formula Verification}

For each formula or equation:
\begin{itemize}
  \item Substitute definitions and work through algebra
  
  \item Verify dimensional consistency
  
  \item Check boundary cases and limiting behavior
  
  \item Ensure all integrals/sums are well-defined under stated conditions
\end{itemize}

\section{MANDATORY MARKING SYSTEM}

\tmtextbf{ABSOLUTE REQUIREMENT}: Every distinguishable mathematical statement,
definition, theorem, proposition, lemma, corollary, formula, equation, or
claim must receive \tmtextbf{exactly one} of the following marks:

\subsection{GREEN CHECK ✅: Statement is Mathematically Correct}

Mark GREEN CHECK if and only if:
\begin{itemize}
  \item You have completed explicit mathematical verification
  
  \item All computational steps check out
  
  \item The logical structure is sound
  
  \item No mathematical errors are found
  
  \item The statement is mathematically true
\end{itemize}

\subsection{RED X ❌: Statement is Mathematically Incorrect}

Mark RED X if and only if:
\begin{itemize}
  \item You have found a specific mathematical error
  
  \item You can provide an explicit counterexample
  
  \item You can show a logical contradiction
  
  \item You have rigorous proof that the statement is false
\end{itemize}
\tmtextbf{CRITICAL RULE}: You may \tmtextbf{ONLY} mark RED X if you provide
explicit mathematical proof of incorrectness.

\subsection{🟡 YELLOW CIRCLE: Indeterminate Due to Verification Failure}

Mark YELLOW CIRCLE if and only if \tmtextbf{ALL} of the following conditions
hold:
\begin{itemize}
  \item You genuinely cannot complete the mathematical verification
  
  \item You do not understand the mathematical content sufficiently
  
  \item You cannot determine whether the statement is true or false
  
  \item You have made genuine effort to understand but failed
\end{itemize}
\tmtextbf{YELLOW REQUIREMENTS}: If you mark YELLOW it must be because you fail
to understand" and this is the only state allowed besides GREEN or RED

\tmtextbf{YELLOW PROHIBITION}: You \tmtextbf{CANNOT} mark YELLOW if:
\begin{itemize}
  \item You think the statement might be wrong (use RED X with proof instead)
  
  \item You are unsure but lean toward incorrect (use RED X with proof
  instead)
  
  \item You find the statement unexpected (complete verification instead)
  
  \item You lack familiarity with the topic (complete verification instead)
\end{itemize}

\section{NO EXCEPTIONS POLICY}

\begin{enumerate}
  \item \tmtextbf{NO SKIPPED STATEMENTS}: Every mathematical claim must be
  marked
  
  \item \tmtextbf{NO PARTIAL MARKS}: Each statement gets exactly one mark
  
  \item \tmtextbf{NO CONDITIONAL MARKS}: Do not use phrases like "correct
  if..." - determine the truth value
  
  \item \tmtextbf{NO MIDDLE POSITIONS}: There is no "partially correct" or
  "mostly right"
\end{enumerate}

\section{MANDATORY DOCUMENTATION}

For each statement you evaluate, you \tmtextbf{MUST} provide:
\begin{enumerate}
  \item \tmtextbf{Mark Assignment}: Exactly one mark (✅, ❌, or 🟡)
  
  \item \tmtextbf{Computational Work}: Show your mathematical verification
  steps
  
  \item \tmtextbf{Logical Analysis}: Identify the claim structure and verify
  each part
  
  \item \tmtextbf{Specific Reasoning}: State exactly why you assigned your
  mark
  
  \item \tmtextbf{Counterproof Requirement}: If marking RED X, provide
  complete mathematical counterproof
  
  \item \tmtextbf{Ignorance Declaration}: If marking YELLOW, explicitly detail
  your failure to understand
\end{enumerate}

\section{VERIFICATION EXAMPLES}

\subsection{Correct Approach for Equivalence Claims}

For claim "A {\Longleftrightarrow} B":
\begin{enumerate}
  \item Prove A {\Longrightarrow} B by direct mathematical argument
  
  \item Prove B {\Longrightarrow} A by direct mathematical argument
  
  \item Verify both directions are logically sound
  
  \item Mark ✅ GREEN CHECK only after both directions verified
  
  \item If either direction fails, provide counterproof and mark ❌ RED X
\end{enumerate}

\subsection{Correct Approach for Formula Claims}

For claimed formula $F = G$:
\begin{enumerate}
  \item Start with left-hand side F
  
  \item Apply definitions and perform algebraic manipulations
  
  \item Show that this equals right-hand side G
  
  \item Verify all steps are mathematically valid
  
  \item Check that all operations are well-defined
  
  \item Mark ✅ if verification succeeds, ❌ if counterexample found
\end{enumerate}

\section{FINAL REQUIREMENTS}

\begin{enumerate}
  \item Complete verification for \tmtextbf{every single distinguishable
  statement}
  
  \item Assign exactly one mark to \tmtextbf{every statement}
  
  \item Provide explicit mathematical reasoning for \tmtextbf{every mark
  assigned}
  
  \item Do not skip any statement as "obvious," "familiar," or "minor"
  
  \item If genuinely unable to verify, mark 🟡 and detail your failure - do
  not guess
\end{enumerate}
\tmtextbf{REMEMBER}:
\begin{itemize}
  \item Mathematical truth is determined by rigorous proof, not by intuition
  or familiarity
  
  \item Your task is to verify what the mathematics actually says, not what
  you expect it to say
  
  \item Marking something wrong requires proof of its incorrectness
  
  \item Every mathematical statement in the document must receive exactly one
  mark
  
  \item There are no exceptions to these requirements
\end{itemize}

\end{document}
