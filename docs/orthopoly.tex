\documentclass{article}
\usepackage{amsmath}
\usepackage{amssymb}
\usepackage{amsthm}

\newtheorem{theorem}{Theorem}
\newtheorem{definition}{Definition}

\title{Orthogonal Polynomials with Respect to a Measure}
\author{}
\date{}

\begin{document}

\maketitle

\section{Existence and Uniqueness of Orthogonal Polynomials}

\subsection{Conditions for Existence}

For a measure $\mu$ on the real line to have a set of orthogonal polynomials:

\begin{enumerate}
    \item $\mu$ must be a positive Borel measure.
    \item All moments of $\mu$ must exist and be finite:
    \[\int x^n d\mu(x) < \infty \quad \text{for all } n \geq 0\]
    \item $\mu$ must not be concentrated on a finite set of points.
\end{enumerate}

\subsection{Uniqueness of Orthogonal Polynomials}

If orthogonal polynomials exist for a measure, their uniqueness (up to normalization) is guaranteed by:

\begin{enumerate}
    \item Gram-Schmidt Process
    \item Three-Term Recurrence Relation:
    \[P_{n+1}(x) = (A_n x + B_n)P_n(x) - C_n P_{n-1}(x)\]
\end{enumerate}

\subsection{Determining Uniqueness}

\begin{theorem}[Carleman's Condition]
If the measure satisfies:
\[\sum_{n=1}^{\infty} (m_{2n})^{-1/2n} = \infty\]
where $m_{2n}$ is the $2n$-th moment, then the measure has a unique set of orthogonal polynomials.
\end{theorem}

\begin{theorem}[Stieltjes Moment Problem]
For measures supported on $[0,\infty)$, if:
\[\sum_{n=1}^{\infty} (m_n)^{-1/2n} = \infty\]
then the measure has a unique set of orthogonal polynomials.
\end{theorem}

\section{Generation of Orthogonal Polynomials}

\subsection{Gram-Schmidt Process}

Let $\mu$ be a measure on $\mathbb{R}$ and define the inner product:

\[\int_{\mathbb{R}} f(x)g(x)d\mu(x)\]

Starting with the monomials $\{1, x, x^2, \ldots\}$, we construct orthogonal polynomials $\{P_0, P_1, P_2, \ldots\}$ as follows:

\begin{align*}
P_0(x) &= 1 \\[10pt]
P_1(x) &= x - \frac{\int_{\mathbb{R}} x \cdot 1 \, d\mu(x)}{\int_{\mathbb{R}} 1 \cdot 1 \, d\mu(x)} \\[10pt]
P_2(x) &= x^2 - \frac{\int_{\mathbb{R}} x^2 \cdot P_1(x) \, d\mu(x)}{\int_{\mathbb{R}} P_1(x) \cdot P_1(x) \, d\mu(x)}P_1(x) - \frac{\int_{\mathbb{R}} x^2 \cdot P_0(x) \, d\mu(x)}{\int_{\mathbb{R}} P_0(x) \cdot P_0(x) \, d\mu(x)}P_0(x) \\[10pt]
P_n(x) &= x^n - \sum_{k=0}^{n-1} \frac{\int_{\mathbb{R}} x^n \cdot P_k(x) \, d\mu(x)}{\int_{\mathbb{R}} P_k(x) \cdot P_k(x) \, d\mu(x)}P_k(x)
\end{align*}

\subsection{Three-Term Recurrence Relation}

The orthogonal polynomials satisfy a three-term recurrence relation:

\[P_{n+1}(x) = (A_n x + B_n)P_n(x) - C_n P_{n-1}(x)\]

where:

\begin{align*}
A_n &= \frac{\int_{\mathbb{R}} xP_n(x) \cdot P_n(x) \, d\mu(x)}{\int_{\mathbb{R}} P_n(x) \cdot P_n(x) \, d\mu(x)} \\[10pt]
B_n &= \frac{\int_{\mathbb{R}} xP_n(x) \cdot P_{n-1}(x) \, d\mu(x)}{\int_{\mathbb{R}} P_{n-1}(x) \cdot P_{n-1}(x) \, d\mu(x)} \\[10pt]
C_n &= \frac{\int_{\mathbb{R}} P_n(x) \cdot P_n(x) \, d\mu(x)}{\int_{\mathbb{R}} P_{n-1}(x) \cdot P_{n-1}(x) \, d\mu(x)}
\end{align*}

This recurrence relation, along with initial conditions $P_0(x) = 1$ and $P_1(x) = x - B_0$, uniquely determines the set of orthogonal polynomials.

\subsection{Normalized Orthogonal Polynomials}

To obtain orthonormal polynomials, we normalize each polynomial by dividing it by its norm:

\[\hat{P}_n(x) = \frac{P_n(x)}{\sqrt{\int_{\mathbb{R}} [P_n(x)]^2 d\mu(x)}}\]

This normalization ensures that:

\[\int_{\mathbb{R}} \hat{P}_n(x) \cdot \hat{P}_m(x) \, d\mu(x) = \delta_{nm}\]

where $\delta_{nm}$ is the Kronecker delta.

\end{document}
