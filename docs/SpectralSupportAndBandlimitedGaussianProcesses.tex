\documentclass{article}
\usepackage[english]{babel}
\usepackage{amsmath,amssymb,latexsym,theorem}

%%%%%%%%%% Start TeXmacs macros
\newcommand{\nin}{\not\in}
\newenvironment{proof}{\noindent\textbf{Proof\ }}{\hspace*{\fill}$\Box$\medskip}
\newtheorem{definition}{Definition}
{\theorembodyfont{\rmfamily}\newtheorem{example}{Example}}
\newtheorem{proposition}{Proposition}
\newtheorem{theorem}{Theorem}
%%%%%%%%%% End TeXmacs macros

\begin{document}

\title{Spectral Support and Bandlimited Gaussian Processes}

\date{}

\maketitle

{\tableofcontents}

\section{Fundamental Definitions}

\begin{definition}
  [Heaviside Step Function] The Heaviside step function $H : \mathbb{R} \to
  \{0, 1\}$ is defined as
  \begin{equation}
    H (x) = \left\{\begin{array}{ll}
      1 & \text{if } x \geq 0\\
      0 & \text{if } x < 0
    \end{array}\right.
  \end{equation}
\end{definition}

\begin{definition}
  [Rectangular Function] The rectangular function $\mathrm{rect}_{[a, b]} :
  \mathbb{R} \to \{0, 1\}$ for $a < b$ is defined as
  \begin{equation}
    \mathrm{rect}_{[a, b]} (\omega) = H (\omega - a) - H (\omega - b)
  \end{equation}
  which equals 1 for $\omega \in [a, b]$ and 0 otherwise.
\end{definition}

\begin{definition}
  [Spectral Density] Let $\{X_t \}_{t \in \mathbb{R}}$ be a zero-mean,
  stationary Gaussian process with covariance function $K (\tau) =\mathbb{E}
  [X_t X_{t + \tau}]$. The spectral density $S : \mathbb{R} \to [0, \infty)$
  is the Fourier transform of the covariance function:
  \begin{equation}
    S (\omega) = \int_{- \infty}^{\infty} K (\tau) e^{- i \omega \tau} d \tau
  \end{equation}
  provided this integral exists.
\end{definition}

\begin{definition}
  [Spectral Support] The spectral support of a Gaussian process with spectral
  density $S (\omega)$ is the set
  \begin{equation}
    \mathrm{supp} (S) = \overline{\{\omega \in \mathbb{R}: S (\omega) > 0\}}
  \end{equation}
  where $\bar{A}$ denotes the closure of set $A$.
\end{definition}

\begin{definition}
  [Bandlimited Process] A stationary Gaussian process is called bandlimited if
  its spectral support is bounded, i.e., if there exist constants $a, b \in
  \mathbb{R}$ with $a < b$ such that
  \begin{equation}
    \mathrm{supp} (S) \subseteq [a, b]
  \end{equation}
\end{definition}

\section{Main Results}

\begin{theorem}
  [Sinc Kernel Spectral Density] Consider the covariance function
  \begin{equation}
    K (\tau) = \frac{\sin (2 \pi \tau)}{2 \pi \tau}
  \end{equation}
  with the convention that $K (0) = 1$. The corresponding spectral density is
  \begin{equation}
    S (\omega) = \frac{1}{2} \mathrm{rect}_{[- 1, 1]} (\omega) = \frac{1}{2} 
    [H (\omega + 1) - H (\omega - 1)]
  \end{equation}
\end{theorem}

\begin{proof}
  The Fourier transform of $K (\tau) = \frac{\sin (2 \pi \tau)}{2 \pi \tau}$
  is computed as follows. Using the identity $\sin (2 \pi \tau) = \frac{e^{i 2
  \pi \tau} - e^{- i 2 \pi \tau}}{2 i}$, one has
  
  \begin{align}
    S (\omega) & = \int_{- \infty}^{\infty} \frac{\sin (2 \pi \tau)}{2 \pi
    \tau} e^{- i \omega \tau} d \tau \\
    & = \frac{1}{2 \pi}  \int_{- \infty}^{\infty} \frac{e^{i 2 \pi \tau} -
    e^{- i 2 \pi \tau}}{2 i \tau} e^{- i \omega \tau} d \tau \\
    & = \frac{1}{4 \pi i}  \left[ \int_{- \infty}^{\infty} \frac{e^{- i
    (\omega - 2 \pi) \tau}}{\tau} d \tau - \int_{- \infty}^{\infty} \frac{e^{-
    i (\omega + 2 \pi) \tau}}{\tau} d \tau \right] 
  \end{align}
  
  By the well-known result that the Fourier transform of $\frac{\sin (a
  \tau)}{\pi \tau}$ is $\mathrm{rect}_{[- a, a]} (\omega)$, it follows that
  \begin{equation}
    S (\omega) = \frac{1}{2} \mathrm{rect}_{[- 1, 1]} (\omega) = \frac{1}{2} 
    [H (\omega + 1) - H (\omega - 1)]
  \end{equation}
\end{proof}

\begin{proposition}
  [General Bandlimited Spectral Density] A Gaussian process is bandlimited
  with spectral support $[a, b]$ if and only if its spectral density can be
  written as
  \begin{equation}
    S (\omega) = f (\omega) \cdot \mathrm{rect}_{[a, b]} (\omega) = f (\omega)
    \cdot [H (\omega - a) - H (\omega - b)]
  \end{equation}
  for some non-negative function $f : [a, b] \to [0, \infty)$.
\end{proposition}

\begin{proof}
  ($\Rightarrow$) If the process is bandlimited with spectral support $[a,
  b]$, then $S (\omega) = 0$ for $\omega \nin [a, b]$. Define $f (\omega) = S
  (\omega)$ for $\omega \in [a, b]$ and extend arbitrarily to $\mathbb{R}$.
  Then $S (\omega) = f (\omega) \cdot \mathrm{rect}_{[a, b]} (\omega)$.
  
  ($\Leftarrow$) If $S (\omega) = f (\omega) \cdot \mathrm{rect}_{[a, b]}
  (\omega)$, then $S (\omega) = 0$ for $\omega \nin [a, b]$, implying
  $\mathrm{supp} (S) \subseteq [a, b]$.
\end{proof}

\begin{example}
  [Band-pass Process] Consider a bandlimited process with spectral support $[-
  \Omega, - \omega_0] \cup [\omega_0, \Omega]$ where $0 < \omega_0 < \Omega$.
  The spectral density can be expressed as
  
  \begin{align}
    S (\omega) & = f (\omega) \cdot [\mathrm{rect}_{[- \Omega, - \omega_0]}
    (\omega) + \mathrm{rect}_{[\omega_0, \Omega]} (\omega)] \\
    & = f (\omega) \cdot [H (\omega + \Omega) - H (\omega + \omega_0) + H
    (\omega - \omega_0) - H (\omega - \Omega)] 
  \end{align}
  
  for some appropriate function $f$.
\end{example}

\begin{theorem}
  [Wiener-Khintchine Relation for Bandlimited Processes] If a Gaussian process
  has spectral density $S (\omega) = g (\omega) \cdot \mathrm{rect}_{[a, b]}
  (\omega)$, then its covariance function is given by
  \begin{equation}
    K (\tau) = \frac{1}{2 \pi}  \int_a^b g (\omega) e^{i \omega \tau} d \omega
  \end{equation}
\end{theorem}

\begin{proof}
  By the inverse Fourier transform relation,
  
  \begin{align}
    K (\tau) & = \frac{1}{2 \pi}  \int_{- \infty}^{\infty} S (\omega) e^{i
    \omega \tau} d \omega \\
    & = \frac{1}{2 \pi}  \int_{- \infty}^{\infty} g (\omega) \cdot
    \mathrm{rect}_{[a, b]} (\omega) e^{i \omega \tau} d \omega \\
    & = \frac{1}{2 \pi}  \int_a^b g (\omega) e^{i \omega \tau} d \omega 
  \end{align}
  
  since $\mathrm{rect}_{[a, b]} (\omega) = 0$ outside $[a, b]$.
\end{proof}

\section{Conclusion}

The spectral support serves as the fundamental concept for characterizing
bandlimited Gaussian processes. The Heaviside step function provides a natural
mathematical framework for expressing the boundaries of spectral support,
enabling precise characterization of the frequency domain properties of such
processes.

\end{document}
