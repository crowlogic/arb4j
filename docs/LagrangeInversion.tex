\documentclass{article}
\usepackage[english]{babel}
\usepackage{geometry,amsmath,amssymb,latexsym,theorem}
\geometry{letterpaper}

%%%%%%%%%% Start TeXmacs macros
\newcommand{\tmaffiliation}[1]{\\ #1}
\newcommand{\tmtextbf}[1]{\text{{\bfseries{#1}}}}
\newenvironment{proof}{\noindent\textbf{Proof\ }}{\hspace*{\fill}$\Box$\medskip}
\newtheorem{corollary}{Corollary}
\newtheorem{definition}{Definition}
\newtheorem{lemma}{Lemma}
\newtheorem{proposition}{Proposition}
{\theorembodyfont{\rmfamily}\newtheorem{remark}{Remark}}
\newtheorem{theorem}{Theorem}
%%%%%%%%%% End TeXmacs macros

\begin{document}

\title{The Lagrange Inversion Theorem: A Comprehensive Proof with
Measure-Theoretic Considerations}

\author{
  Stephen Crowley
  \tmaffiliation{August 11, 2025}
}

\maketitle

\begin{abstract}
  A proof of the Lagrange inversion theorem for analytic functions is
  presented, with particular attention to the case where the underlying
  function is monotonically increasing or decreasing and therefore bijective.
  Points where the derivative may vanish on sets of measure zero are carefully
  considered, establishing conditions under which the inversion formula
  remains valid.
\end{abstract}

{\tableofcontents}

\section{Introduction}

The Lagrange inversion theorem provides an explicit formula for the
coefficients of the power series expansion of the inverse of an analytic
function. This fundamental result in complex analysis has profound
applications in combinatorics, probability theory, and the study of generating
functions. The approach presented here emphasizes its measure-theoretic
foundations and the role of monotonic bijective functions.

\section{Preliminaries and Definitions}

\begin{definition}
  [Analytic Function] A function $f : \mathbb{C} \to \mathbb{C}$ is analytic
  at a point $a \in \mathbb{C}$ if there exists a neighborhood $U$ of $a$ such
  that $f$ can be represented by a convergent power series
  \begin{equation}
    f (z) = \sum_{n = 0}^{\infty} a_n  (z - a)^n \forall z \in U
  \end{equation}
\end{definition}

\begin{definition}
  [Monotonic Function] A real-valued function $g : I \to \mathbb{R}$ on an
  interval $I$ is monotonically increasing if for all $x_1, x_2 \in I$ with
  $x_1 < x_2$, we have $g (x_1) \leq g (x_2)$. It is strictly monotonic if the
  inequality is strict whenever $x_1 \neq x_2$.
\end{definition}

\begin{definition}
  [Measure Zero Set] A set $E \subset \mathbb{R}$ has Lebesgue measure zero if
  for every $\epsilon > 0$, there exists a countable collection of intervals
  $\{I_k \}_{k = 1}^{\infty}$ such that $E \subset \bigcup_{k = 1}^{\infty}
  I_k$ and $\sum_{k = 1}^{\infty} |I_k | < \epsilon$, where $|I_k |$ denotes
  the length of interval $I_k$.
\end{definition}

\section{The Main Theorem}

\begin{theorem}
  [Lagrange Inversion Theorem with Measure-Theoretic Extension] Let $f :
  \mathbb{C} \to \mathbb{C}$ be analytic in a neighborhood of the origin with
  $f (0) = 0$ and $f' (0) \neq 0$. Suppose there exists an analytic function
  $\phi : \mathbb{C} \to \mathbb{C}$ with $\phi (0) \neq 0$ such that
  \begin{equation}
    f (w) = \frac{w}{\phi (w)}
  \end{equation}
  Let $g$ be the compositional inverse of $f$, so that
  \begin{equation}
    f (g (z)) = z
  \end{equation}
  in a neighborhood of the origin.
  
  Furthermore, assume that the real part of $f$ restricted to the real axis is
  monotonically increasing or decreasing, making $f$ bijective on its domain
  of convergence, and that the derivative $f'$ vanishes only on a set of
  measure zero.
  
  Then for any analytic function $H (w)$ and for $n \geq 1$:
  \begin{equation}
    [z^n] H (g (z)) = \frac{1}{n} [w^{n - 1}]  (H' (w) \phi (w)^n)
  \end{equation}
  where $[w^k]$ denotes the coefficient of $w^k$ in the power series
  expansion.
  
  In particular, taking $H (w) = w$, we obtain:
  \begin{equation}
    [z^n] g (z) = \frac{1}{n} [w^{n - 1}] \phi (w)^n
  \end{equation}
\end{theorem}

\section{Proof}

\subsection{Step 1: Establishing the Bijective Property}

\begin{lemma}
  [Monotonic Analytic Functions are Locally Bijective] Let $f$ be analytic in
  a neighborhood $U$ of the origin with $f (0) = 0$ and $f' (0) \neq 0$. If
  Re$(f)$ is monotonic on $U \cap \mathbb{R}$, then $f$ is locally bijective.
\end{lemma}

\begin{proof}
  Since $f' (0) \neq 0$, the inverse function theorem for complex analytic
  functions guarantees that $f$ is locally invertible in some neighborhood $V$
  of the origin. The monotonicity condition on Re$(f)$ ensures that $f$ is
  injective when restricted to real values, and by the identity theorem for
  analytic functions, this injectivity extends to the complex neighborhood.
\end{proof}

\subsection{Step 2: Handling the Measure Zero Condition}

\begin{lemma}
  [Derivative Vanishing on Measure Zero Sets] Let $f$ be analytic with $f'$
  vanishing only on a set $E$ of measure zero. Then the Cauchy integral
  formula and residue calculations remain valid for the inversion process.
\end{lemma}

\begin{proof}
  Since $E$ has measure zero, the set $\{z : f' (z) = 0\}$ cannot accumulate
  at any point in the domain of analyticity (by the identity theorem).
  Therefore, $f' (z) \neq 0$ for all but a discrete set of points, and the
  standard residue-theoretic proof of the Lagrange inversion theorem applies
  without modification.
  
  The key observation is that in complex analysis, sets of measure zero (in
  the real sense) correspond to discrete or at most countable sets when
  dealing with zeros of analytic functions. Since analytic functions are
  determined by their behavior on any open set, the vanishing of $f'$ on a
  measure zero set does not affect the global analytic properties required for
  the inversion formula.
\end{proof}

\subsection{Step 3: The Core Proof via Residue Theory}

The proof proceeds using the residue theorem and contour integration. Let
\begin{equation}
  g (z) = \sum_{n = 1}^{\infty} g_n z^n
\end{equation}
be the inverse function of $f$, so that $f (g (z)) = z$. By observing
\begin{equation}
  f (w) = \frac{w}{\phi (w)}
\end{equation}


it is seen that
\begin{equation}
  z = f (g (z)) = \frac{g (z)}{\phi (g (z))}
\end{equation}
which gives
\begin{equation}
  g (z) = z \phi (g (z))
\end{equation}
For any analytic function $H (w)$, consider the coefficient $[z^n] H (g (z))$.
By the Cauchy integral formula:
\begin{equation}
  [z^n] H (g (z)) = \frac{1}{2 \pi i}  \oint_{|z| = r} \frac{H (g (z))}{z^{n +
  1}} dz
\end{equation}
for sufficiently small $r > 0$. The substitution
\begin{equation}
  z = f (w) = \frac{w}{\phi (w)}
\end{equation}
is made so that
\begin{equation}
  dz = f' (w) dw = \frac{\phi (w) - w \phi' (w)}{(\phi (w))^2} dw
\end{equation}
Since $g (z) = w$ when $z = f (w)$, it is the case that:
\begin{equation}
  [z^n] H (g (z)) = \frac{1}{2 \pi i}  \oint_C \frac{H (w)}{\left(
  \frac{w}{\phi (w)} \right)^{n + 1}} \cdot \frac{\phi (w) - w \phi'
  (w)}{(\phi (w))^2} dw
\end{equation}
where $C$ is the image of the circle $|z| = r$ under the mapping $w \mapsto f
(w)$. Simplifying the integrand:
\begin{equation}
  \begin{array}{ll}
    \frac{H (w)}{\left( \frac{w}{\phi (w)} \right)^{n + 1}} \cdot \frac{\phi
    (w) - w \phi' (w)}{(\phi (w))^2} & = \frac{H (w) \phi (w)^{n + 1}}{w^{n +
    1}} \cdot \frac{\phi (w) - w \phi' (w)}{(\phi (w))^2}\\
    & = \frac{H (w) \phi (w)^{n - 1}  (\phi (w) - w \phi' (w))}{w^{n + 1}}
  \end{array}
\end{equation}
Now, observe that:
\[ \frac{d}{dw} \left( \frac{H (w) \phi (w)^n}{n} \right) = \frac{H' (w) \phi
   (w)^n + H (w) n \phi (w)^{n - 1} \phi' (w)}{n} \]
\[ = \frac{H' (w) \phi (w)^n + H (w) w \phi (w)^{n - 1} \phi' (w)}{n} . \]
Therefore:
\begin{equation}
  \begin{array}{ll}
    H (w) \phi (w)^{n - 1}  (\phi (w) - w \phi' (w)) & = \phi (w)^n H (w) - H
    (w) w \phi (w)^{n - 1} \phi' (w)\\
    & = n \frac{d}{dw} \left( \frac{H (w) \phi (w)^n}{n} \right) - H' (w)
    \phi (w)^n
  \end{array}
\end{equation}
Substituting back:
\[ [z^n] H (g (z)) = \frac{1}{2 \pi i}  \oint_C \frac{n \frac{d}{dw} \left(
   \frac{H (w) \phi (w)^n}{n} \right) - H' (w) \phi (w)^n}{w^{n + 1}} dw \]
By the residue theorem, the integral of the exact differential vanishes,
leaving:
\begin{equation}
  \begin{array}{l}
    
  \end{array} \begin{array}{ll}
    {}[z^n] H (g (z)) & = - \frac{1}{2 \pi i}  \oint_C \frac{H' (w) \phi
    (w)^n}{w^{n + 1}} dw\\
    & = \frac{1}{n} \cdot \frac{1}{2 \pi i}  \oint_C \frac{nH' (w) \phi
    (w)^n}{w^{n + 1}} dw\\
    & = \frac{1}{n} [w^{n - 1}]  (H' (w) \phi (w)^n)
  \end{array}
\end{equation}
The measure zero condition on the vanishing of $f'$ ensures that this residue
calculation is well-defined, as the contour can be chosen to avoid the
discrete set where $f'$ vanishes.

\subsection{Step 4: Special Case and Corollary}

\begin{corollary}
  [Classical Lagrange Inversion Formula] Under the conditions of the main
  theorem, with $H (w) = w$:
  \begin{equation}
    g_n = [z^n] g (z) = \frac{1}{n} [w^{n - 1}] \phi (w)^n
  \end{equation}
\end{corollary}

\begin{proof}
  This follows immediately from the main theorem by setting $H (w) = w$, so
  that $H' (w) = 1$.
\end{proof}

\section{Applications and Remarks}

\begin{remark}
  [Monotonicity and Bijectivity] The monotonicity condition ensures global
  invertibility of the function on its real restriction, which extends to
  local bijectivity in the complex domain. This is crucial for the validity of
  the power series inversion.
\end{remark}

\begin{remark}
  [Measure Zero Sets and Analyticity] In the context of complex analytic
  functions, the condition that $f'$ vanishes only on a set of measure zero is
  automatically satisfied in most practical applications, since the zeros of a
  non-trivial analytic function form a discrete set.
\end{remark}

\begin{proposition}
  [Convergence Properties] Under the conditions of the main theorem, the
  series $g (z) = \sum_{n = 1}^{\infty} g_n z^n$ converges in a disk $|z| < R$
  where $R$ is determined by the radius of convergence of the original
  function $f$ and the behavior of $\phi$.
\end{proposition}

\section{Examples}

\tmtextbf{Example 1:} Consider
\begin{equation}
  f (w) = \frac{w}{1 + w}
\end{equation}
with
\begin{equation}
  \phi (w) = 1 + w
\end{equation}
. Then:
\begin{equation}
  g_n = \frac{1}{n} [w^{n - 1}]  (1 + w)^n = \frac{1}{n} \binom{n}{n - 1} = 1
\end{equation}
for all $n \geq 1$, giving
\begin{equation}
  g (z) = \sum_{n = 1}^{\infty} z^n = \frac{z}{1 - z}
\end{equation}
, which is indeed the inverse of $f$.

\tmtextbf{Example 2:} For
\begin{equation}
  f (w) = we^{- w}
\end{equation}
we have
\begin{equation}
  \phi (w) = e^w
\end{equation}
, leading to:
\begin{equation}
  g_n = \frac{1}{n} [w^{n - 1}] e^{nw} = \frac{n^{n - 1}}{n!}
\end{equation}
recovering the well-known series expansion for the Lambert W function.

\section{Conclusion}

We have established the Lagrange inversion theorem with explicit attention to
monotonic bijective functions and measure-theoretic considerations. The proof
demonstrates that the classical inversion formula remains valid even when the
derivative vanishes on sets of measure zero, provided the underlying function
maintains its analytic and bijective properties. This framework provides a
robust foundation for applications in combinatorics, probability theory, and
the analysis of special functions.

\end{document}
