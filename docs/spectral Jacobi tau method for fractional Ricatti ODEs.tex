\documentclass{article}
\usepackage{amsmath}
\usepackage{amssymb}
\begin{document}

For the fractional Riccati equation:

\[ D^\mu y(x) = g(x) - a(x)y(x) - b(x)y^2(x) \]

where $\mu$ is the fractional derivative order ($0 < \mu \leq 1$). Each function in the equation is expanded in the shifted Jacobi polynomial basis $P_i^{(\alpha,\beta)}(x)$ with parameters $\alpha, \beta$ defined over $[0,T]$, where the shifted polynomials are related to the standard Jacobi polynomials through:

\[ P_i^{(\alpha,\beta)}(x) = P_i^{(\alpha,\beta)}(2x/T - 1) \]

The function expansions are:

\[ g(x) = \sum_{i=0}^{n} g_i P_i^{(\alpha,\beta)}(x) \]
\[ a(x) = \sum_{i=0}^{n} a_i P_i^{(\alpha,\beta)}(x) \]
\[ b(x) = \sum_{i=0}^{n} b_i P_i^{(\alpha,\beta)}(x) \]
\[ y(x) = \sum_{i=0}^{n} c_i P_i^{(\alpha,\beta)}(x) \]

The quadratic term expands to:

\[ y^2(x) = \sum_{i=0}^{n}\sum_{j=0}^{n} c_ic_j P_i^{(\alpha,\beta)}(x)P_j^{(\alpha,\beta)}(x) \]

The fractional derivative term using the operational matrix becomes:

\[ D^\mu \sum_{i=0}^{n} c_i P_i^{(\alpha,\beta)}(x) = \sum_{i=0}^{n}\sum_{k=0}^{n} c_i p_{k,i}^{(\mu)} P_k^{(\alpha,\beta)}(x) \]

The shifted Jacobi polynomials (SJPs) are explicitly defined as:

\[ J_{T,i}^{(a,b)}(\tau) = \sum_{k=0}^{i}(-1)^{i+k} \frac{\Gamma(i+b+1)\Gamma(i+k+a+b+1)}{\Gamma(k+b+1)\Gamma(i+a+b+1)(i-k)!k!T^k}\tau^k \]

The operational matrix elements $p_{k,i}^{(\mu)}$ can be expressed in two equivalent forms:

1. Using gamma functions:
\[ p_{k,i}^{(\mu)} = \sum_{j=0}^{i} (-1)^{i+j} \frac{\Gamma(i+\beta+1)\Gamma(i+j+\alpha+\beta+1)}{\Gamma(j+\beta+1)\Gamma(i-j+1)\Gamma(j+\mu+1)} \]

2. Using Jacobi polynomials:
\[ p_{k,i}^{(\mu)} = \int_0^T D^\mu P_i^{(\alpha,\beta)}(x)P_k^{(\alpha,\beta)}(x)w^{(\alpha,\beta)}(x)dx \]

The weight function is defined as:
\[ w_T^{(a,b)}(\tau) = \tau^b(T-\tau)^a \]

with the orthogonality property:
\[ \int_0^T J_{T,i}^{(a,b)}(\tau)J_{T,j}^{(a,b)}(\tau)w_T^{(a,b)}(\tau)d\tau = h_{T,i}^{(a,b)}\delta_{ij} \]

The linear term expands as:

\[ a(x)y(x) = \sum_{i=0}^{n}\sum_{j=0}^{n} a_ic_j P_i^{(\alpha,\beta)}(x)P_j^{(\alpha,\beta)}(x) \]

\section{Projection Coefficients}

For any function $f(x)$ (including $g(x)$, $a(x)$, and $b(x)$), its expansion coefficients in the Jacobi basis are given by:

\[ k_i = \frac{1}{h_i}\int_0^T f(x)P_i^{(\alpha,\beta)}(x)w_T^{(\alpha,\beta)}(x)dx \]

where the weight function is:

\[ w_T^{(\alpha,\beta)}(x) = x^\beta(T-x)^\alpha \]

The orthogonality relation for the Jacobi polynomials is:

\[ \int_0^T P_m^{(\alpha,\beta)}(x)P_n^{(\alpha,\beta)}(x)w_T^{(\alpha,\beta)}(x)dx = h_n^{(\alpha,\beta)}\delta_{mn} \]

The normalization constant $h_n^{(\alpha,\beta)}$ is explicitly given by:

\[ h_n^{(\alpha,\beta)} = \frac{\Gamma(n+\alpha+1)\Gamma(n+\beta+1)}{(2n+\alpha+\beta+1)n!\Gamma(n+\alpha+\beta+1)} \]

Here:
\begin{itemize}
\item $\Gamma$ is the gamma function
\item $\mu$ is the fractional derivative order
\item $\alpha, \beta$ are the Jacobi polynomial parameters
\item $n$ is the degree of approximation
\item $T$ is the upper bound of the interval
\item $\delta_{ij}$ is the Kronecker delta
\item $h_i$ is the normalization constant for the i-th Jacobi polynomial
\item $k_i$ represents the projection coefficients ($g_i$, $a_i$, $b_i$, or $c_i$ depending on the function)
\end{itemize}

\end{document}