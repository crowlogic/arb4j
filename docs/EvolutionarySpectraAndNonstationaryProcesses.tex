\documentclass[12pt]{article}
\usepackage{amsmath, amssymb, amsthm, enumitem, geometry, hyperref}
\geometry{left=1in, right=1in, top=1in, bottom=1in}
\usepackage{graphicx}
\usepackage{bm}
\hypersetup{colorlinks=true, linkcolor=blue, citecolor=blue, urlcolor=blue}

\title{Evolutionary Spectra and Non-Stationary Processes}
\author{M.~B. Priestley \\ University of Manchester}
\date{Read at a Research Methods Meeting of the Society, February 3, 1965\\
\emph{(Professor D.~V. Lindley in the Chair)}}

\theoremstyle{definition}
\newtheorem{definition}{Definition}[section]
\theoremstyle{plain}
\newtheorem{theorem}{Theorem}[section]
\newtheorem{lemma}{Lemma}[section]
\newtheorem{corollary}{Corollary}[section]
\theoremstyle{remark}
\newtheorem*{remark}{Remark}
\newtheorem*{example}{Example}
\newtheorem*{discussion}{Discussion}
\newtheorem*{proofsketch}{Proof Sketch}

\begin{document}

\maketitle

\begin{abstract}
We develop an approach to the spectral analysis of non-stationary processes based on the concept of \emph{evolutionary spectra}---that is, spectral functions which are time-dependent and have a physical interpretation as local energy distributions over frequency. It is shown that the notion of evolutionary spectra generalizes the usual definition of spectra for stationary processes, and that, under certain conditions, the evolutionary spectrum at each instant of time may be estimated from a single realization of a process. By such means it is possible to study processes with continuously changing ``spectral patterns''.
\end{abstract}

\tableofcontents

\section{Introduction}

In the classical approach to statistical spectral analysis it is always assumed that the process under study, $\{X_t\}$, is stationary up to at least the second order. That is, we assume that $\mathbb{E}(X_t) = \mu$, a constant (independent of $t$), which we may take to be zero, and that, for each $s$ and $t$, the covariance
\begin{equation}
R_{s,t} = \mathbb{E}\bigl((X_s-\mu)(X_t-\mu)^*\bigr)
\label{eq:autocov_stationary}
\end{equation}
(* denoting complex conjugation) is a function of $|s-t|$ only. In this case, it is well known that $R_{s,t}$ has a spectral representation of the form
\begin{equation}
R_{s,t} = \int e^{i \omega (s-t)}\, dF(\omega),
\label{eq:spectral_stationary}
\end{equation}
where $F(\omega)$ has the properties of a distribution function, with $\omega \in (-\infty, \infty)$ for continuous parameter, or $\omega \in (-\pi, \pi)$ in the discrete case.

Corresponding to~\eqref{eq:spectral_stationary}, $\{X_t\}$ has a spectral representation
\begin{equation}
X_t = \int e^{i \omega t}\, dZ(\omega),
\label{eq:process_rep_stationary}
\end{equation}
where $Z(\omega)$ is an orthogonal process with $\mathbb{E}|dZ(\omega)|^2 = dF(\omega)$. For physical processes, the spectral density $f(\omega) = F'(\omega)$ (when it exists) describes the energy distribution over frequencies.

However, the assumption of stationarity is often questionable; e.g., atmospheric turbulence records change over time and classical spectral analysis may not be adequate. Thus, can we formulate spectral theory for non-stationary processes retaining notions such as ``energy'' and ``frequency'' and preserving physical interpretability?

If no restrictions (other than finite first and second moments) are placed, no useful inference can be drawn from a single record. But, for processes that are stationary in segments (i.e., $X_t = X^{(1)}_t$ for $t \leq t_0$, $X^{(2)}_t$ for $t > t_0$), we infer something about the spectrum. This motivates considering a \emph{continuously changing} or time-dependent spectrum.

We therefore focus on processes where non-stationary characteristics change slowly over time (akin to Jowett's ``smoothly heteromorphic'' processes~\cite{Jowett1957}).

\section{Non-Stationary Processes}
\label{sec:nonstationary}

Several attempts have been made to define a spectrum for non-stationary processes. Notably, Cramér~\cite{Cramer1960} considered harmonizable processes (Loève sense), with representations like~\eqref{eq:process_rep_stationary} but $Z(\omega)$ need not be orthogonal. The integrated spectrum is then (as a function of two variables):
\begin{equation}
dF(\omega, \nu) = \mathbb{E}[dZ(\omega) dZ^*(\nu)].
\end{equation}

Hatanaka and Suzuki define the spectrum (spectral density function) as the limiting expected value of the periodogram as sample size $\to\infty$. Page~\cite{Page1952} introduced ``instantaneous power spectra''. Specifically,
\begin{align}
f^*(\omega) &= \lim_{T \to \infty} f_T(\omega),\\
f_T(\omega) &= \mathbb{E}\left\{ \frac{1}{T} \int_0^T X_t e^{-i \omega t} dt \right\},\\
P_t(\omega) &= \frac{d}{dt}f^*(\omega).
\end{align}
The instantaneous power spectrum $P_t(\omega)$ compares the spectral content up to $t+dt$ and up to $t$.

Our approach, in contrast, is to study the spectral content \emph{within} the interval $(t, t+dt)$, which we argue is more relevant for physical interpretation.

\section{Spectral Theory for a Class of Non-Stationary Processes: Oscillatory Processes}
\label{sec:oscillatory}

Consider a complex-valued, continuous-parameter process $\{X_t\}_{t \in \mathbb{R}}$, assumed ``trend free'', i.e., $\mathbb{E}(X_t) = 0$ $\forall t$.

Define the autocovariance
\begin{equation}
R_{s,t} = \mathbb{E}(X_s X_t^*).
\label{eq:autocov_general}
\end{equation}

Assume there exists a family $\mathcal{F}$ of functions $\{\phi_t(\omega)\}$ defined on $\mathbb{R}$ and indexed by $t$, and a measure $\mu(\omega)$ on $\mathbb{R}$, such that
\begin{equation}
R_{s,t} = \int \phi_s(\omega) \phi_t^*(\omega) \, d\mu(\omega).
\label{eq:cov_oscill}
\end{equation}
(see also Parzen, unpublished). Each $t$-indexed function can also be viewed as a family $\{\phi_\omega(t)\}$ indexed by $\omega$.

For $X_t$ to have finite variance, $\phi_t(\omega)$ must be quadratically integrable in $\omega$. When $R_{s,t}$ admits~\eqref{eq:cov_oscill}, we get:
\begin{equation}
X_t = \int \phi_t(\omega)\ dZ(\omega),
\label{eq:process_general}
\end{equation}
where $Z(\omega)$ is orthogonal and
\begin{equation}
\mathbb{E}|dZ(\omega)|^2 = d\mu(\omega).
\label{eq:orthog_measure}
\end{equation}
(see Bartlett~\cite{Bartlett1955} and Grenander \& Rosenblatt~\cite{GrenanderRosenblatt1957}).

For stationary processes, a suitable $\mathcal{F}$ is
\begin{equation}
\phi_t(\omega) = e^{i \omega t},
\label{eq:phi_stationary}
\end{equation}
yielding the standard spectral decomposition.

For non-stationary processes, the family of basic elements must be ``oscillatory'' but ``non-stationary''. Suppose for each $\omega$, $\phi_t(\omega)$ has a generalized Fourier transform maximized at frequency $\theta(\omega)$, so
\begin{equation}
\phi_t(\omega) = A_t(\omega) e^{i \theta(\omega) t},
\label{eq:phi_oscillatory}
\end{equation}
where $A_t(\omega)$ is such that its Fourier transform (in $t$) is maximized at 0.

\begin{definition}
\label{def:oscillatory_function}
The function $\phi_t(\omega)$ is \emph{oscillatory} if, for some unique $\theta(\omega)$, it may be written as in~\eqref{eq:phi_oscillatory} where
\begin{equation}
A_t(\omega) = \int e^{it\theta} dH_\omega(\theta),
\label{eq:A_FT}
\end{equation}
with $|dH_\omega(\theta)|$ maximized at $\theta=0$.
\end{definition}

If $\{\phi_t(\omega)\}$ is such that $\theta(\omega)$ is single-valued, the variable in~\eqref{eq:cov_oscill} can be changed from $\omega$ to $\theta(\omega)$, yielding
\begin{align}
R_{s,t} &= \int A_s(\omega) A_t^*(\omega) e^{i\omega(s-t)} d\mu(\omega),\label{eq:cov_theta}\\
X_t &= \int A_t(\omega) e^{i\omega t}\ dZ(\omega).\label{eq:X_theta}
\end{align}

\begin{definition}
\label{def:oscillatory_process}
A process with representation~\eqref{eq:cov_oscill} in terms of a family of oscillatory functions $\mathcal{F}$ is called an \emph{oscillatory process}.
\end{definition}

Any such process can be represented as in~\eqref{eq:X_theta} with $\phi_t(\omega) = A_t(\omega) e^{i\omega t}$.

\section{Evolutionary (Power) Spectra}
\label{sec:evospectra}

For an oscillatory process as above, the variance is
\begin{equation}
\operatorname{var}(X_t) = R_{t,t} = \int |A_t(\omega)|^2 d\mu(\omega).
\label{eq:var_xt}
\end{equation}
Thus, the contribution from frequency $\omega$ is $|A_t(\omega)|^2 d\mu(\omega)$.

\begin{definition}
\label{def:evolutionary_spectrum}
Let $\mathcal{F} = \{A_t(\omega) e^{i\omega t}\}$, and $\{X_t\}$ have representation~\eqref{eq:X_theta}. The \emph{evolutionary power spectrum} at time $t$ with respect to $\mathcal{F}$ is
\begin{equation}
dF_t(\omega) = |A_t(\omega)|^2 d\mu(\omega).
\label{eq:evolutionary_spectrum}
\end{equation}
\end{definition}

For stationary processes with $\mathcal{F}$ the exponential family ($A_t(\omega) \equiv 1$), $dF_t(\omega)$ reduces to the classical spectrum.

Also,
\begin{equation}
\operatorname{var}(X_t) = \int dF_t(\omega), 
\label{eq:var_evospec}
\end{equation}
so the total energy at $t$ is independent of the family $\mathcal{F}$.

It is convenient to normalize $A_t(\omega)$ so that
\begin{equation}
\int |dH_\omega(\theta)| = 1,
\label{eq:A_norm}
\end{equation}
making $d\mu(\omega)$ the evolutionary spectrum at $t=0$.

Oscillatory processes can also be interpreted as the result of passing a stationary process through a time-varying filter. That is, with $h_t(u)$ such that
\begin{equation}
A_t(\omega) = \int e^{i u \omega} h_t(u) du,
\label{eq:A_filter}
\end{equation}
then
\begin{equation}
X_t = \int h_t(u) S_{t-u} du, \quad S_{t} = \int e^{i\omega t}\ dZ(\omega),
\label{eq:X_filter}
\end{equation}
where $S_t$ is stationary with spectrum $d\mu(\omega)$.

\section{The Uniformly Modulated Process}

A notable special case is
\begin{equation}
X_t = c(t) X^{(0)}_t,
\label{eq:uniformly_modulated}
\end{equation}
where $\{ X_t^{(0)} \}$ is stationary, $c(0)=1$, and $c(t)$'s Fourier transform has modulus maximized at the origin.

Then
\begin{equation}
X^{(0)}_t = \int e^{i\omega t}\ dZ(\omega),
\label{eq:stationary_decomp}
\end{equation}
so
\begin{equation}
X_t = \int c(t) e^{i\omega t}\ dZ(\omega).
\label{eq:um_decomp}
\end{equation}
This is an oscillatory process; its evolutionary spectrum is
\begin{equation}
dF_t(\omega) = c(t)^2\, dF(\omega).
\label{eq:um_spec}
\end{equation}

This process is special in that spectral components vary in time in the same way:
\begin{equation}
\frac{dF_{t_1}(\omega_1)}{dF_{t_2}(\omega_1)} = \frac{dF_{t_1}(\omega_2)}{dF_{t_2}(\omega_2)},~\forall \omega_1, \omega_2
\end{equation}
A process with such a property is called a \emph{uniformly modulated process}.

\section{Effect of Filters}
\label{sec:filters}

For stationary processes, the effect of a linear filter $g(u)$, applied as
\begin{equation}
Y_t = \int g(u) X_{t-u} du,
\label{eq:filter}
\end{equation}
results in a transformed spectrum
\begin{align}
dF_Y(\omega) &= |\Gamma(\omega)|^2 dF_X(\omega), \label{eq:filtered_spec}\\
\Gamma(\omega) &= \int g(u) e^{-i\omega u} du.
\end{align}

For oscillatory (non-stationary) processes, consider
\begin{equation}
Y_t = \int g(u) X_{t-u} e^{-i\omega_0 (t-u)} du,
\label{eq:nonstationary_filter}
\end{equation}
which can be expanded as
\begin{equation}
Y_t = \int \Gamma_{t,\omega_0}(\omega) A_{t}(\omega + \omega_0) e^{i\omega t} dZ(\omega + \omega_0), 
\label{eq:filtered_oscillatory}
\end{equation}
where
\begin{equation}
\Gamma_{t,\lambda}(\theta) = \int g(u) \frac{A_{t-u}(\lambda)}{A_t(\lambda)} e^{-i\theta u} du.
\label{eq:gen_transfer}
\end{equation}

If $A_{t-u}(\lambda)$ varies slowly compared to $g(u)$, then $\Gamma_{t,\lambda}(\theta) \approx \Gamma(\theta)$ and
\begin{equation}
Y_t \approx \int A_t(\omega + \omega_0) e^{i\omega t} d\widetilde{Z}(\omega),
\end{equation}
with
\begin{equation}
\mathbb{E}|d\widetilde{Z}(\omega)|^2 = |\Gamma(\omega)|^2 d\mu(\omega + \omega_0)
\end{equation}
and thus
\begin{equation}
dF_Y(\omega) \approx |\Gamma(\omega)|^2 dF_X(\omega + \omega_0).
\label{eq:filtered_evospectra}
\end{equation}

\section{Semi-Stationary Processes}

To formalize ``slowly varying'', for each family $\mathcal{F}$ define
\begin{equation}
B(\omega) = \int |\theta|\, |dH_\omega(\theta)|,
\label{eq:Bw}
\end{equation}
and
\begin{equation}
B_\mathcal{F} = [\sup_\omega B(\omega)]^{-1}.
\label{eq:B_F}
\end{equation}

\begin{definition}
\label{def:semi_stationary_family}
A family $\mathcal{F}$ is \emph{semi-stationary} if $B(\omega)$ is bounded $\forall \omega$. $B_\mathcal{F}$ is the \emph{characteristic width} of $\mathcal{F}$.
\end{definition}

\begin{definition}
\label{def:semi_stationary_process}
A process is semi-stationary if it has a representation~\eqref{eq:X_theta} with respect to a semi-stationary family.
\end{definition}

For all such admissible families $\mathcal{F}$, define
\begin{equation}
B_X = \sup_{\mathcal{F}} B_\mathcal{F}.
\label{eq:BX}
\end{equation}

Let $\mathcal{F}^*$ be a family achieving the supremum (characteristic width $B_X$).

Then
\begin{equation}
X_t = A^*_t(\omega) e^{i\omega t} dZ^*(\omega).
\label{eq:X_Fstar}
\end{equation}

The validity of the approximation~\eqref{eq:filtered_evospectra} depends on the ``width'' of $g(u)$ being much less than $B_X$.

Define the following:

\begin{definition}
We say $u(x)$ is a \emph{pseudo} $\delta$-\emph{function of order} $\epsilon$ with respect to $v(x)$ if, for every $k$, there exists $\epsilon<1$ such that
\begin{equation}
\left|\int u(x) v(x+k) dx - v(k) \int u(x) dx \right| < \epsilon.
\end{equation}
\end{definition}

Assume $g(u)$ is square integrable and normalized:
\begin{align}
\int |g(u)| du = 1,\quad \int |u|\,|g(u)| du = B_g.\label{eq:filter_width}
\end{align}

\begin{lemma}
\label{lem:pseudo_delta}
Let $\mathcal{F}$ be semi-stationary with width $B$. Then, for each $t,\omega$, $\{e^{i t \theta} dH_\omega(\theta)\}$ is a pseudo $\delta$-function of order $B_g/B$ with respect to $\Gamma(\theta)$.
\end{lemma}

\begin{proof}
For any $k$,
\[
\int e^{i t \theta} \Gamma(\theta + k) dH_\omega(\theta) = \Gamma(k) \int e^{i t \theta} dH_\omega(\theta) + R(k)
\]
with $R(k)$ bounded in terms of $B_g$ and $B$. Details follow the standard argument (see text for more).
\end{proof}

\begin{theorem}
\label{thm:approximation}
Suppose $g(u)$ satisfies~\eqref{eq:filter_width}, and $\Gamma_{t,\lambda}(\theta)$ is its generalized transfer function with respect to a semi-stationary family of width $B$. If $B_g < \epsilon B$, then
\[
|A_t(\lambda)| \cdot |\Gamma_{t,\lambda}(\theta) - \Gamma(\theta)| < \epsilon,\quad \forall t,\lambda,\theta.
\]
\end{theorem}

\begin{proof}
The proof follows by representing $A_{t-u}(\lambda)$ by its Fourier transform and applying Lemma~\ref{lem:pseudo_delta}.
\end{proof}

\section{Determination of Evolutionary Spectra}
\label{sec:determination}

Let $\{X_t\}$ be a semi-stationary process with width $B_X$. Let $g(u)$ be a filter of width $B_g < B_X$. For any frequency $\omega_0$, set
\begin{equation}
Y_t = \int g(u) X_{t-u} e^{-i\omega_0 (t-u)} du.
\label{eq:Y_sample}
\end{equation}

Expressing $X_t$ in terms of $\mathcal{F}^*$ gives, via~\eqref{eq:X_Fstar}, representation
\begin{equation}
Y_t = \int \Gamma_{t,\omega_0}(\omega) A^*_t(\omega+\omega_0) e^{i\omega t} dZ^*(\omega+\omega_0),
\end{equation}
and so
\begin{equation}
\mathbb{E}|Y_t|^2 = \int |\Gamma_{t,\omega_0}(\omega)|^2 |A^*_t(\omega+\omega_0)|^2 d\mu^*(\omega+\omega_0) + O(B_g / B_X).
\label{eq:Ey2}
\end{equation}

\section{Estimation of Evolutionary Spectra}
\label{sec:estimation}

Given a sample record $\{X_t\}$ for $0 < t \leq T$, we wish to estimate $dF_t(\omega)$ for $0 < t \leq T$. Suppose $\mu(\omega)$ is absolutely continuous, so $dF_t(\omega) = f_t(\omega) d\omega$.

Let $g(u)$ have width $B_g < B_X < T$, and for $\omega_0$ define
\begin{equation}
U_t = \int_{t-T}^t g(u) X_{t-u} e^{-i\omega_0 (t-u)} du.
\label{eq:Ut}
\end{equation}
For $t$ large enough, end effects are negligible and $U_t$ approximates $Y_t$ in~\eqref{eq:Y_sample}.

Thus,
\begin{equation}
\mathbb{E}|U_t|^2 = \int |\Gamma(\omega)|^2 f_t(\omega+\omega_0) d\omega + O(B_g/B_X),
\end{equation}
where $\Gamma(\omega)$ is the transfer function of $g(u)$.

In contrast to stationary spectral estimation, for evolutionary spectra the bandwidth of $|\Gamma(\omega)|^2$ is limited by $B_g B_X$; i.e., higher time resolution reduces frequency resolution. If $f_t(\omega)$ is smooth compared to $|\Gamma(\omega)|^2$,
\begin{equation}
\mathbb{E}|U_t|^2 \approx f_t(\omega_0),
\label{eq:Ut_f}
\end{equation}
as $\int |\Gamma(\omega)|^2 d\omega = 1$.

For a normal process,
\begin{equation}
\operatorname{var}(|U_t|^2) \approx [\int |\Gamma(\omega)|^2 f_t(\omega+\omega_0) d\omega]^2 (1 + \delta_{0,0}),
\end{equation}
which is independent of $T$: $|U_t|^2$ is a noisy estimate, as in the classical periodogram.

To improve variance, average $|U_t|^2$ over neighboring $t$ using a weight function $W_{T'}(t)$ with the following properties:
\begin{enumerate}[label=(\alph*)]
    \item $W_{T'}(t) \geq 0$ for all $t, T'$
    \item $W_{T'}(t) \to 0$ as $|t| \to \infty$
    \item $\int W_{T'}(t) dt = 1$ for all $T'$
    \item $\int [W_{T'}(t)]^2 dt < \infty$ for all $T'$
    \item For some $C > 0$, $\lim_{T' \to \infty} T' \int |W_{T'}(\lambda)|^2 d\lambda = C$ where $W_{T'}(\lambda)$ is the Fourier transform of $W_{T'}(t)$
\end{enumerate}

Define
\begin{equation}
V_t = \int W_{T'}(u) |U_{t-u}|^2 du.
\label{eq:Vt}
\end{equation}
With suitable smoothness,
\begin{equation}
\mathbb{E} V_t \approx f_t(\omega_0),
\end{equation}
and
\begin{equation}
\operatorname{var}(V_t) \sim [f_t(\omega_0)]^2 \frac{C}{T'} (1 + \delta_{0,0}),
\end{equation}
so fluctuations are $O(1/\sqrt{T'})$.

\subsection*{Practical Windows}
$g(u)$, the filter, may be any standard spectral analysis window. For example, for the Bartlett window:
\begin{align}
g(u) &= \begin{cases}
\frac{1}{2h} & |u| \leq h \\
0 & \text{otherwise}
\end{cases}
\label{eq:Bartlett}\\
|\Gamma(\omega)|^2 &= \frac{1}{2\pi} \frac{\sin^2 h\omega}{\pi^2 \omega^2}
\end{align}
For the weight function $W_{T'}(t)$, the Daniell window:
\begin{equation}
W_{T'}(t) = \begin{cases}
\frac{1}{T'} & -T' < t \leq T'\\
0 & \text{otherwise}
\end{cases}
\end{equation}
or alternatively an exponential window.

\section{Discrete Parameter Processes}
\label{sec:discrete}

For a discrete-parameter process, the (oscillatory) representation is
\begin{equation}
X_t = \int_{-\pi}^{\pi} e^{i t \omega} A_t(\omega) dZ(\omega), \quad t = 0, \pm1, \pm2, ...
\label{eq:discrete_oscillatory}
\end{equation}
where $A_t(\omega)$ (for each $\omega$) has a generalized discrete Fourier transform maximized at zero, and $Z(\omega)$ is orthogonal with $\mathbb{E}|dZ(\omega)|^2 = d\mu(\omega)$ on $(-\pi, \pi)$.

The evolutionary spectrum at $t$ (relative to $\{e^{i t \omega} A_t(\omega)\}$) is
\begin{equation}
dF_t(\omega) = |A_t(\omega)|^2 d\mu(\omega), \qquad (-\pi \leq \omega \leq \pi).
\end{equation}
Estimation proceeds as in the continuous case, but with sums instead of integrals:
\begin{align}
U_t &= \sum_u g_u X_{t-u} e^{-i\omega_0 (t-u)},\\
V_t &= \sum_v W_{v} |U_{t-v}|^2. 
\end{align}

\section{Further Problems}
\label{sec:further}

For evolutionary spectral analysis, one must choose:
\begin{enumerate}[label=(\arabic*)]
    \item Filter $g(u)$ form
    \item Weight $W_{T'}(t)$ form
    \item Filter width $B_g$
    \item Parameter $T'$ for $W_{T'}$
\end{enumerate}
The optimal choice depends on specific optimality criteria (mean square error, etc.), and may need to be addressed empirically.

Testing for stationarity can be approached via variance analysis, e.g., by comparing
\begin{equation}
f(\omega) = \frac{1}{T} \int_0^T |U_t|^2 dt
\end{equation}
with $W_{T'}(t) = \frac{1}{T'}, 0 < t \leq T'$, and testing based on
\begin{equation}
A(\omega) = \int [f_t(\omega) - f(\omega)]^2 dt.
\end{equation}

\section*{References}
\begin{thebibliography}{99}
\bibitem{Bartlett1955}
Bartlett, M.S. (1955). \emph{An Introduction to Stochastic Processes with Special Reference to Methods and Applications}. Cambridge: Cambridge University Press.

\bibitem{Cramer1960}
Cramér, H. (1960). On some classes of non-stationary processes. \emph{Proc. Fourth Berkeley Symp. Math. Statist. Prob.}, \textbf{2}, 57--78.

\bibitem{GrenanderRosenblatt1957}
Grenander, U. and Rosenblatt, M. (1957). \emph{Statistical Analysis of Stationary Time Series}. New York: Wiley.

\bibitem{Herbst1963a}
Herbst, L.J. (1963a). Periodogram analysis and variance fluctuations. \emph{J. R. Statist. Soc. B}, \textbf{25}, 442--450.

\bibitem{Herbst1963b}
Herbst, L.J. (1963b). A test for variance heterogeneity in the residuals of a Gaussian moving average. \emph{J. R. Statist. Soc. B}, \textbf{25}, 451--454.

\bibitem{Herbst1963c}
Herbst, L.J. (1963c). Almost periodic variances. \emph{Ann. Math. Statist.}, \textbf{34}, 1549--1557.

\bibitem{Jowett1957}
Jowett, G.H. (1957). Statistical analysis using local properties of smoothly heteromorphic stochastic series. \emph{Biometrika}, \textbf{44}, 454--463.

\bibitem{LomnickiZaremba1957}
Lomnicki, Z.A. and Zaremba, S.K. (1957). On estimating the spectral density function of a stochastic process. \emph{J. R. Statist. Soc. B}, \textbf{19}, 13--37.

\bibitem{Page1952}
Page, C.H. (1952). Instantaneous power spectra. \emph{J. Appl. Phys.}, \textbf{23}, 103--106.

\bibitem{Priestley1962}
Priestley, M.B. (1962). Basic consideration in the estimation of spectra. \emph{Technometrics}, \textbf{4}, 551--563.
\end{thebibliography}

\appendix
\section*{Appendix: Analysis of Artificial Processes}

To test the suggested estimation methods, artificial non-stationary processes were generated from known models. The main example considered is:
\begin{align}
X_t &= e^{-(t-500)^2 / [2 \times 200^2]} Y_t,\\
Y_{t+2} - 0.8 Y_{t+1} + 0.4 Y_t &= Z_t,
\end{align}
where $Z_t \sim \mathcal{N}(0, 100^2)$ i.i.d.

The spectral density of $Y_t$ is
\begin{equation}
f^{(1)}(\omega) = \frac{\sigma^2}{2\pi} \frac{0.792}{1 - 0.8 \cos \omega + 0.4 \cos^2 \omega}, \qquad \sigma^2 = \operatorname{var}(Y_t) = 1332.
\end{equation}
With respect to the family $F = \{ e^{-(t-500)^2 / 2 \times 200^2} e^{i t \omega}\}$, the evolutionary spectrum of $X_t$ is
\begin{equation}
f^{(x)}_t(\omega) = [e^{-(t-500)^2 / 2 \times 200^2}]^2 f^{(1)}(\omega).
\end{equation}

Estimates were computed using the discrete-time analog of equation~\eqref{eq:Vt} with $W_{T'}(u)$ from the Daniell window ($T'=200$) and $g(u)$ as in~\eqref{eq:Bartlett} with $h=7$. For this $h$ the filter's bandwidth is about half that of $f^{(x)}_t(\omega)$ at the peak.

Details on figures and further numerical analysis are omitted for brevity.

\section*{Discussion}

A detailed discussion followed the main presentation, involving remarks by M.~S.~Bartlett, D.~R.~Brillinger, C.~W.~J.~Granger, H.~H.~Robertson, J.~W.~Tukey, H.~E.~Daniels, G.~A.~Barnard, and a reply by M.~B.~Priestley. For the full text of these contributions, see the source document.

\section*{References in the Discussion}
\begin{thebibliography}{99}
\bibitem{Devinatz1953}
Devinatz, A. (1953). Integral representations of positive definite functions. \emph{Trans. Amer. Math. Soc.}, \textbf{74}, 56--77.

\bibitem{Gabor1946}
Gabor, D. (1946). Theory of communication. \emph{J. Inst. Elec. Eng.}, \textbf{93}, 429--459.

\bibitem{GrangerHatanaka1964}
Granger, C.~W.~J. and Hatanaka, M. (1964). \emph{Spectral Analysis of Economic Time Series}. Princeton: Princeton University Press.

\bibitem{Hida1960}
Hida, T. (1960). Canonical representations of Gaussian processes and their applications. \emph{Mem. Coll. Sci. Univ. Kyoto}, \textbf{33}, 109--155.

\bibitem{Levy1959}
Levy, P. (1959). Sur quelques classes de fonctions aléatoires. \emph{J. Math. Pures et App.}, \textbf{38}, 1--23.

\bibitem{Tukey1961}
Tukey, J.W. (1961). Discussion, emphasising the connection between analysis of variance and spectrum analysis. \emph{Technometrics}, \textbf{3}, 191--219.

\bibitem{Ville1948}
Ville, J. (1948). Theorie et applications de la notion de signal analytique. \emph{Cables et Transmission}, \textbf{2}, 61--74.

\bibitem{Zadeh1961}
Zadeh, L.A. (1961). Time varying networks, I. \emph{Proc. I.R.E.}, \textbf{49}, 1488--1503.
\end{thebibliography}

\end{document}
