\documentclass{article}
\usepackage{amsmath,amssymb,amsthm,amsfonts}
\usepackage{mathtools}
\usepackage{graphicx}
\usepackage{enumitem}
\usepackage{hyperref}

\newtheorem{theorem}{Theorem}
\newtheorem{definition}[theorem]{Definition}
\newtheorem{lemma}[theorem]{Lemma}
\newtheorem{corollary}[theorem]{Corollary}
\newtheorem{proposition}[theorem]{Proposition}

\theoremstyle{definition}
\newtheorem{remark}[theorem]{Remark}

\title{Explicit Definition and Properties of $h_t(u)$ in Non-Stationary Processes}
\author{Technical Report}
\date{\today}

\begin{document}

\maketitle

\section{Introduction}
We consider oscillatory processes in the framework of Priestley's evolutionary spectra for non-stationary processes. The oscillatory process $X_t$ is defined by:
\begin{equation}
X_t = \int_{-\infty}^{\infty} e^{i\omega t} A_t(\omega) \, dZ(\omega)
\end{equation}
where $dZ(\omega)$ is a process with orthogonal increments and spectrum $d\mu(\omega)$, and $A_t(\omega)$ is the \textit{gain function} that modulates the amplitude of each frequency component at time $t$.

\section{Time-Varying Filter Interpretation}

\begin{theorem}[Explicit Definition of $h_t(u)$]
For a non-stationary oscillatory process with gain function $A_t(\omega)$, the time-varying filter $h_t(u)$ is explicitly defined as:
\begin{equation}
h_t(u) = \frac{1}{2\pi}\int_{-\infty}^{\infty} A_t(\omega) e^{-i\omega u} \, d\omega
\end{equation}
That is, $h_t(u)$ is the inverse Fourier transform of the gain function $A_t(\omega)$ for each fixed time $t$.
\end{theorem}

\begin{proof}
We start with the relationship defining $A_t(\omega)$ as the Fourier transform of $h_t(u)$:
\begin{equation}
A_t(\omega) = \int_{-\infty}^{\infty} e^{i\omega u} h_t(u) \, du
\end{equation}

Apply the inverse Fourier transform to both sides:
\begin{align}
\frac{1}{2\pi}\int_{-\infty}^{\infty} A_t(\omega) e^{-i\omega v} \, d\omega &= \frac{1}{2\pi}\int_{-\infty}^{\infty} \left(\int_{-\infty}^{\infty} e^{i\omega u} h_t(u) \, du \right) e^{-i\omega v} \, d\omega \\
&= \int_{-\infty}^{\infty} h_t(u) \left(\frac{1}{2\pi}\int_{-\infty}^{\infty} e^{i\omega(u-v)} \, d\omega \right) \, du
\end{align}

The inner integral represents the Dirac delta function:
\begin{equation}
\frac{1}{2\pi}\int_{-\infty}^{\infty} e^{i\omega(u-v)} \, d\omega = \delta(u-v)
\end{equation}

Therefore:
\begin{equation}
\frac{1}{2\pi}\int_{-\infty}^{\infty} A_t(\omega) e^{-i\omega v} \, d\omega = \int_{-\infty}^{\infty} h_t(u) \delta(u-v) \, du = h_t(v)
\end{equation}

Thus, we have proven the explicit definition:
\begin{equation}
h_t(u) = \frac{1}{2\pi}\int_{-\infty}^{\infty} A_t(\omega) e^{-i\omega u} \, d\omega
\end{equation}
\end{proof}

\begin{theorem}[Representation via Time-Varying Filter]
A non-stationary oscillatory process $X_t$ can be represented as the convolution of a time-varying filter $h_t(u)$ with a stationary process $S_t$ having spectrum $d\mu(\omega)$:
\begin{equation}
X_t = \int_{-\infty}^{\infty} S_{t-u} h_t(u) \, du
\end{equation}
\end{theorem}

\begin{proof}
Starting from the oscillatory process definition:
\begin{equation}
X_t = \int_{-\infty}^{\infty} e^{i\omega t} A_t(\omega) \, dZ(\omega)
\end{equation}

Substitute the Fourier representation of $A_t(\omega)$:
\begin{align}
X_t &= \int_{-\infty}^{\infty} e^{i\omega t} \left(\int_{-\infty}^{\infty} e^{i\omega u} h_t(u) \, du\right) \, dZ(\omega) \\
&= \int_{-\infty}^{\infty} h_t(u) \left(\int_{-\infty}^{\infty} e^{i\omega(t+u)} \, dZ(\omega)\right) \, du
\end{align}

Define $S_t$ as a stationary process with the representation:
\begin{equation}
S_t = \int_{-\infty}^{\infty} e^{i\omega t} \, dZ(\omega)
\end{equation}

Then the inner integral becomes:
\begin{equation}
\int_{-\infty}^{\infty} e^{i\omega(t+u)} \, dZ(\omega) = S_{t+u}
\end{equation}

Substituting back:
\begin{equation}
X_t = \int_{-\infty}^{\infty} h_t(u) S_{t+u} \, du
\end{equation}

With the change of variable $v = -u$:
\begin{equation}
X_t = \int_{-\infty}^{\infty} h_t(-v) S_{t-v} \, dv
\end{equation}

Redefining $h_t(u) \rightarrow h_t(-u)$ for notational simplicity:
\begin{equation}
X_t = \int_{-\infty}^{\infty} h_t(u) S_{t-u} \, du
\end{equation}

Thus, $X_t$ can be represented as the output of passing a stationary process through a time-varying filter $h_t(u)$.
\end{proof}

\begin{theorem}[Evolutionary Spectrum Relationship]
The evolutionary spectrum of the process $X_t$ at time $t$ is given by:
\begin{equation}
f_t(\omega) = |A_t(\omega)|^2 d\mu(\omega)
\end{equation}
where $A_t(\omega)$ is the gain function and $d\mu(\omega)$ is the spectral measure of the underlying stationary process.
\end{theorem}

\begin{proof}
From the definition of $A_t(\omega)$ as the Fourier transform of $h_t(u)$:
\begin{equation}
A_t(\omega) = \int_{-\infty}^{\infty} e^{i\omega u} h_t(u) \, du
\end{equation}

The squared magnitude of the gain function is:
\begin{align}
|A_t(\omega)|^2 &= A_t(\omega)\overline{A_t(\omega)} \\
&= \int_{-\infty}^{\infty} \int_{-\infty}^{\infty} e^{i\omega u} e^{-i\omega v} h_t(u) \overline{h_t(v)} \, du \, dv \\
&= \int_{-\infty}^{\infty} \int_{-\infty}^{\infty} e^{i\omega(u-v)} h_t(u) \overline{h_t(v)} \, du \, dv
\end{align}

The local power spectrum at time $t$ is defined as $|A_t(\omega)|^2 d\mu(\omega)$. This represents the distribution of power across frequencies at the specific time $t$, taking into account the modulation effect of the time-varying filter on the underlying stationary process.

Therefore, the evolutionary spectrum is:
\begin{equation}
f_t(\omega) = |A_t(\omega)|^2 d\mu(\omega)
\end{equation}
which completes the proof.
\end{proof}

\section{Conclusion}
We have explicitly defined the time-varying filter $h_t(u)$ as the inverse Fourier transform of the gain function $A_t(\omega)$. This relationship provides a useful interpretation of non-stationary oscillatory processes as the output of passing a stationary process through a time-varying filter. The evolutionary spectrum directly relates to the squared magnitude of the gain function, weighted by the spectral measure of the underlying stationary process.

\end{document}
