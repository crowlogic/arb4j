\documentclass[12pt,a4paper]{article}
\usepackage{amsmath,amssymb,amsthm,amsfonts}
\usepackage{mathtools}
\usepackage{bm}
\usepackage{physics}
\usepackage{graphicx}
\usepackage{enumitem}
\usepackage{hyperref}
\usepackage{tikz}
\usepackage{xcolor}
\usepackage{thmtools}
\usepackage{geometry}
\geometry{margin=1in}

\theoremstyle{definition}
\newtheorem{definition}{Definition}
\newtheorem{theorem}{Theorem}
\newtheorem{lemma}[theorem]{Lemma}
\newtheorem{proposition}[theorem]{Proposition}
\newtheorem{corollary}[theorem]{Corollary}
\newtheorem{example}{Example}
\newtheorem{remark}{Remark}

\title{\Huge \textbf{Rigged Hilbert Spaces and Spectral Theory of Unbounded Operators}}
\author{}
\date{}

\begin{document}

\maketitle
\tableofcontents
\newpage

\section{Introduction}
Rigged Hilbert spaces, also known as Gelfand triples, provide a mathematical framework that extends traditional Hilbert space theory to accommodate operators with continuous or mixed spectra. This extension is particularly crucial for quantum mechanics, where fundamental operators like position and momentum have continuous spectra without proper eigenvectors in the conventional sense.

This document presents a comprehensive analysis of rigged Hilbert spaces, their construction, and their application to spectral theory, with particular emphasis on unbounded normal operators.

\section{Mathematical Foundations}

\subsection{Hilbert Space Limitations}

Standard Hilbert space theory provides a powerful framework for analyzing operators with discrete spectra, but it encounters significant limitations when dealing with operators possessing continuous spectra.

\begin{definition}[Spectrum of an Operator]
Let $T$ be a closed linear operator on a Hilbert space $\mathcal{H}$ with domain $\mathcal{D}(T)$. The spectrum $\sigma(T)$ of $T$ consists of all complex numbers $\lambda$ for which the operator $(T - \lambda I)$ does not have a bounded inverse defined on all of $\mathcal{H}$.
\end{definition}

\begin{definition}[Types of Spectra]
The spectrum $\sigma(T)$ of an operator $T$ is divided into:
\begin{itemize}
    \item \textbf{Point spectrum} $\sigma_p(T)$: Values $\lambda$ for which $(T - \lambda I)$ is not injective.
    \item \textbf{Continuous spectrum} $\sigma_c(T)$: Values $\lambda$ for which $(T - \lambda I)$ is injective with dense range, but not surjective.
    \item \textbf{Residual spectrum} $\sigma_r(T)$: Values $\lambda$ for which $(T - \lambda I)$ is injective but does not have dense range.
\end{itemize}
\end{definition}

The classical spectral theorem for bounded self-adjoint operators ensures the existence of a projection-valued measure, but this approach becomes unwieldy for operators with continuous spectra, particularly in quantum mechanics.

\subsection{Gelfand Triple: Definition and Structure}

\begin{definition}[Gelfand Triple]
A Gelfand triple or rigged Hilbert space consists of three spaces
\begin{equation}
    \Phi \subset \mathcal{H} \subset \Phi^*
\end{equation}
where:
\begin{itemize}
    \item $\mathcal{H}$ is a Hilbert space with inner product $\langle \cdot, \cdot \rangle$
    \item $\Phi$ is a dense subspace of $\mathcal{H}$ equipped with a stronger topology than the topology induced by $\mathcal{H}$
    \item $\Phi^*$ is the topological dual of $\Phi$, i.e., the space of continuous linear functionals on $\Phi$
\end{itemize}
\end{definition}

The inclusions in this triple are continuous embeddings. The space $\Phi$ is often taken to be a countably normed space, most commonly a nuclear space.

\begin{theorem}[Nuclear Space Characterization]
A locally convex topological vector space $\Phi$ is nuclear if and only if for any continuous seminorm $p$ on $\Phi$, there exists another continuous seminorm $q$ on $\Phi$ such that the canonical map from the completion of $\Phi$ with respect to $q$ to the completion of $\Phi$ with respect to $p$ is nuclear.
\end{theorem}

\subsection{Stronger Topology for Test Function Spaces}

The notion of "stronger topology" on $\Phi$ is central to the rigged Hilbert space construction.

\begin{definition}[Stronger Topology]
Let $\Phi$ be a subspace of a Hilbert space $\mathcal{H}$. A topology $\tau$ on $\Phi$ is said to be stronger than the topology induced by $\mathcal{H}$ if:
\begin{itemize}
    \item Every sequence $\{\phi_n\}$ that converges in $(\Phi, \tau)$ also converges in $\mathcal{H}$.
    \item The converse is not necessarily true; i.e., convergence in $\mathcal{H}$ does not imply convergence in $(\Phi, \tau)$.
\end{itemize}
\end{definition}

\begin{proposition}
If $\tau$ is a stronger topology on $\Phi$, then the identity map $\iota: (\Phi, \tau) \to \mathcal{H}$ is continuous.
\end{proposition}

\begin{proof}
Let $U$ be an open set in $\mathcal{H}$. The preimage $\iota^{-1}(U) = U \cap \Phi$ must be open in $(\Phi, \tau)$ for $\iota$ to be continuous. Since $\tau$ is stronger than the topology induced by $\mathcal{H}$, every open set in $\mathcal{H}$ induces an open set in $\Phi$ under the subspace topology. Therefore, $U \cap \Phi$ is open in $(\Phi, \tau)$, establishing the continuity of $\iota$.
\end{proof}

\section{Generalized Eigenfunctions and Spectral Theory}

\subsection{Generalized Eigenvectors in Rigged Hilbert Spaces}

\begin{definition}[Generalized Eigenvector]
Let $A$ be a linear operator with domain $D(A) \subset \mathcal{H}$, where $\Phi \subset D(A)$ and $A\Phi \subset \Phi$. A functional $F \in \Phi^*$ is called a generalized eigenvector of $A$ with eigenvalue $\lambda$ if
\begin{equation}
    \langle F, A\phi \rangle = \lambda \langle F, \phi \rangle \quad \forall \phi \in \Phi
\end{equation}
where $\langle F, \phi \rangle$ denotes the action of the functional $F$ on the vector $\phi$.
\end{definition}

This definition extends the notion of eigenvectors to include distributions and other generalized functions that are not elements of the Hilbert space.

\begin{theorem}[Existence of Generalized Eigenvectors]
Let $A$ be a self-adjoint operator on a Hilbert space $\mathcal{H}$ with domain $D(A)$. If $\Phi$ is a nuclear space satisfying:
\begin{itemize}
    \item $\Phi \subset D(A)$ and $\Phi$ is dense in $\mathcal{H}$
    \item $A\Phi \subset \Phi$
    \item $\Phi$ is equipped with a topology stronger than that induced by $\mathcal{H}$
\end{itemize}
Then for every $\lambda$ in the spectrum of $A$, there exists a generalized eigenvector $F \in \Phi^*$ with eigenvalue $\lambda$.
\end{theorem}

\begin{proof}[Sketch of Proof]
The proof relies on the Gelfand-Maurin spectral theorem and involves several steps:

1. Construct the resolvent operator $R_z = (A - zI)^{-1}$ for $z \in \mathbb{C} \setminus \sigma(A)$.

2. Show that $R_z$ can be extended to an operator from $\Phi^*$ to itself.

3. Apply analytic continuation and distributional techniques to define the limit of $R_z$ as $z$ approaches a point $\lambda$ in the spectrum from appropriate directions.

4. Extract generalized eigenvectors using the nuclear spectral theorem, which guarantees that the spectral measure can be expressed in terms of generalized eigenvectors.
\end{proof}

\subsection{Nuclear Spectral Theorem}

The nuclear spectral theorem is the cornerstone of the rigged Hilbert space approach to spectral theory.

\begin{theorem}[Nuclear Spectral Theorem]
Let $A$ be a self-adjoint operator on a Hilbert space $\mathcal{H}$ with domain $D(A)$, and let $\Phi \subset D(A)$ be a nuclear space dense in $\mathcal{H}$ such that $A\Phi \subset \Phi$. Then there exists a measure $\mu$ on the spectrum $\sigma(A)$ and a family of generalized eigenvectors $\{F_\lambda\}_{\lambda \in \sigma(A)} \subset \Phi^*$ such that:

1. For each $\lambda \in \sigma(A)$, $F_\lambda$ is a generalized eigenvector of $A$ with eigenvalue $\lambda$.

2. For all $\phi, \psi \in \Phi$,
\begin{equation}
    \langle \phi, \psi \rangle_{\mathcal{H}} = \int_{\sigma(A)} \langle F_\lambda, \phi \rangle \overline{\langle F_\lambda, \psi \rangle} \, d\mu(\lambda)
\end{equation}

3. For all $\phi \in \Phi$,
\begin{equation}
    \langle A\phi, \psi \rangle_{\mathcal{H}} = \int_{\sigma(A)} \lambda \langle F_\lambda, \phi \rangle \overline{\langle F_\lambda, \psi \rangle} \, d\mu(\lambda)
\end{equation}
\end{theorem}

This theorem establishes that every self-adjoint operator possesses a complete set of generalized eigenvectors, allowing for spectral decompositions analogous to those available for operators with purely discrete spectra.

\section{Mixed Spectra and the Rigged Hilbert Space Approach}

\subsection{Unified Treatment of Discrete and Continuous Spectra}

The rigged Hilbert space formalism provides a unified framework for handling operators with both discrete and continuous spectral components.

\begin{theorem}[Decomposition for Mixed Spectra]
Let $A$ be a self-adjoint operator on $\mathcal{H}$ with both discrete and continuous spectrum components. In the rigged Hilbert space $\Phi \subset \mathcal{H} \subset \Phi^*$, any vector $\psi \in \Phi$ can be decomposed as:
\begin{equation}
    \psi = \sum_{n} c_n \phi_n + \int_{\sigma_c(A)} c(\lambda) \psi_\lambda \, d\mu(\lambda)
\end{equation}
where $\{\phi_n\}$ are the eigenvectors corresponding to the discrete spectrum, $\{\psi_\lambda\}$ are the generalized eigenvectors corresponding to the continuous spectrum, and $c_n, c(\lambda)$ are appropriate coefficients.
\end{theorem}

\begin{proof}
The proof follows from the nuclear spectral theorem and the decomposition of the spectrum into discrete and continuous parts. The key insight is that the generalized eigenvectors $\{F_\lambda\}$ from the nuclear spectral theorem provide a "basis" for expansion, regardless of whether $\lambda$ belongs to the discrete or continuous part of the spectrum.

For the discrete spectrum, the generalized eigenvectors coincide with the actual eigenvectors in $\mathcal{H}$. For the continuous spectrum, they exist only in the larger space $\Phi^*$.
\end{proof}

\subsection{Implementation for Specific Operators}

The rigged Hilbert space approach is particularly valuable for operators in quantum mechanics.

\begin{example}[Position and Momentum Operators]
Consider the position operator $\hat{x}$ and momentum operator $\hat{p} = -i\hbar \frac{d}{dx}$ on $L^2(\mathbb{R})$.

For the rigged Hilbert space, we can choose:
\begin{itemize}
    \item $\Phi = \mathcal{S}(\mathbb{R})$, the Schwartz space of rapidly decreasing functions
    \item $\mathcal{H} = L^2(\mathbb{R})$
    \item $\Phi^* = \mathcal{S}'(\mathbb{R})$, the space of tempered distributions
\end{itemize}

The position operator $\hat{x}$ has generalized eigenvectors $\delta(x-x_0) \in \Phi^*$ with eigenvalues $x_0 \in \mathbb{R}$, satisfying:
\begin{equation}
    \langle \delta(x-x_0), \hat{x}\phi \rangle = x_0 \langle \delta(x-x_0), \phi \rangle
\end{equation}

The momentum operator $\hat{p}$ has generalized eigenvectors $e^{ipx/\hbar} \in \Phi^*$ with eigenvalues $p \in \mathbb{R}$, satisfying:
\begin{equation}
    \langle e^{ipx/\hbar}, \hat{p}\phi \rangle = p \langle e^{ipx/\hbar}, \phi \rangle
\end{equation}

Both sets of generalized eigenvectors provide a "basis" for expanding functions in $\Phi$ through the Fourier transform.
\end{example}

\section{Notable Operators Benefiting from the Rigged Hilbert Space Approach}

\subsection{Differential Operators}

Differential operators, which are typically unbounded, significantly benefit from the rigged Hilbert space approach.

\begin{proposition}
Let $D = \frac{d}{dx}$ be the differentiation operator on $L^2(\mathbb{R})$ with domain $D(D) = \{f \in L^2(\mathbb{R}) : f' \in L^2(\mathbb{R})\}$. In the rigged Hilbert space $\mathcal{S}(\mathbb{R}) \subset L^2(\mathbb{R}) \subset \mathcal{S}'(\mathbb{R})$, the operator $D$ has generalized eigenvectors $\{e^{i\lambda x}\}_{\lambda \in \mathbb{R}} \subset \mathcal{S}'(\mathbb{R})$ with eigenvalues $\{i\lambda\}_{\lambda \in \mathbb{R}}$.
\end{proposition}

\begin{proof}
For any test function $\phi \in \mathcal{S}(\mathbb{R})$, we have:
\begin{align}
    \langle e^{i\lambda x}, D\phi \rangle &= \int_{-\infty}^{\infty} e^{-i\lambda x} \frac{d\phi}{dx}(x) \, dx \\
    &= -\int_{-\infty}^{\infty} \frac{d}{dx}(e^{-i\lambda x}) \phi(x) \, dx \\
    &= i\lambda \int_{-\infty}^{\infty} e^{-i\lambda x} \phi(x) \, dx \\
    &= i\lambda \langle e^{i\lambda x}, \phi \rangle
\end{align}
Thus, $e^{i\lambda x}$ is a generalized eigenvector of $D$ with eigenvalue $i\lambda$.
\end{proof}

\subsection{Hamiltonians with Mixed Spectra}

Quantum mechanical Hamiltonians often exhibit both discrete and continuous spectral components.

\begin{example}[Hydrogen Atom Hamiltonian]
The Hamiltonian for the hydrogen atom has:
\begin{itemize}
    \item Discrete spectrum: Negative energy levels $E_n = -\frac{13.6 \text{ eV}}{n^2}$ for $n \in \mathbb{N}$, associated with bound states
    \item Continuous spectrum: Positive energies $E > 0$, associated with scattering states
\end{itemize}

Using the rigged Hilbert space approach, both types of states are treated within a unified framework: bound states as traditional eigenvectors in $\mathcal{H}$ and scattering states as generalized eigenvectors in $\Phi^*$.
\end{example}

\section{Spectral Theory Improvements for Normal Unbounded Operators}

\subsection{Extension to Normal Operators}

While most applications focus on self-adjoint operators, the rigged Hilbert space approach extends to normal operators (operators that commute with their adjoint).

\begin{theorem}
Let $T$ be a normal operator on a Hilbert space $\mathcal{H}$ with domain $D(T)$. If there exists a nuclear space $\Phi \subset D(T)$ dense in $\mathcal{H}$ such that $T\Phi \subset \Phi$ and $T^*\Phi \subset \Phi$, then there exists a measure $\mu$ on the spectrum $\sigma(T) \subset \mathbb{C}$ and a family of generalized eigenvectors $\{F_\lambda\}_{\lambda \in \sigma(T)} \subset \Phi^*$ such that:

1. For each $\lambda \in \sigma(T)$, $F_\lambda$ is a generalized eigenvector of $T$ with eigenvalue $\lambda$.

2. For all $\phi, \psi \in \Phi$,
\begin{equation}
    \langle \phi, \psi \rangle_{\mathcal{H}} = \int_{\sigma(T)} \langle F_\lambda, \phi \rangle \overline{\langle F_\lambda, \psi \rangle} \, d\mu(\lambda)
\end{equation}

3. For all $\phi \in \Phi$,
\begin{equation}
    \langle T\phi, \psi \rangle_{\mathcal{H}} = \int_{\sigma(T)} \lambda \langle F_\lambda, \phi \rangle \overline{\langle F_\lambda, \psi \rangle} \, d\mu(\lambda)
\end{equation}
\end{theorem}

\subsection{Practical Implications}

The rigged Hilbert space approach offers several practical advantages for spectral analysis:

\begin{proposition}[Diagonalization of Unbounded Operators]
In a rigged Hilbert space $\Phi \subset \mathcal{H} \subset \Phi^*$, a normal operator $T$ can be formally diagonalized using its generalized eigenvectors, even when its spectrum is purely continuous.
\end{proposition}

\begin{proof}
The nuclear spectral theorem guarantees the existence of a family of generalized eigenvectors $\{F_\lambda\}_{\lambda \in \sigma(T)}$ that provides a resolution of the identity. This allows any vector $\phi \in \Phi$ to be expanded as:
\begin{equation}
    \phi = \int_{\sigma(T)} c(\lambda) F_\lambda \, d\mu(\lambda)
\end{equation}
where $c(\lambda) = \langle F_\lambda, \phi \rangle$.

The action of $T$ on $\phi$ can then be expressed as:
\begin{equation}
    T\phi = \int_{\sigma(T)} \lambda c(\lambda) F_\lambda \, d\mu(\lambda)
\end{equation}
This constitutes a formal diagonalization of $T$ using generalized eigenvectors, even when $T$ has purely continuous spectrum.
\end{proof}

\section{Applications in Quantum Mechanics}

\subsection{Rigorous Foundation for Dirac's Formalism}

The rigged Hilbert space approach provides a mathematical foundation for Dirac's bra-ket notation.

\begin{proposition}[Bra-Ket Formalism]
In the rigged Hilbert space $\Phi \subset \mathcal{H} \subset \Phi^*$:
\begin{itemize}
    \item "Kets" $|\psi\rangle$ correspond to vectors in $\mathcal{H}$ or generalized vectors in $\Phi^*$
    \item "Bras" $\langle\phi|$ correspond to vectors in $\mathcal{H}$ or generalized vectors in $\Phi$
    \item The pairing $\langle\phi|\psi\rangle$ corresponds to the inner product in $\mathcal{H}$ when both vectors are in $\mathcal{H}$, or to the action of a functional when one vector is generalized
\end{itemize}
\end{proposition}

\begin{theorem}[Completeness Relations]
For a self-adjoint operator $A$ with spectrum $\sigma(A)$ in a rigged Hilbert space $\Phi \subset \mathcal{H} \subset \Phi^*$, the completeness relation
\begin{equation}
    \int_{\sigma(A)} |a\rangle\langle a| \, d\mu(a) = I
\end{equation}
where $|a\rangle$ are generalized eigenvectors of $A$ with eigenvalue $a$, can be given rigorous meaning as:
\begin{equation}
    \langle\phi|\psi\rangle = \int_{\sigma(A)} \langle\phi|a\rangle\langle a|\psi\rangle \, d\mu(a) \quad \forall \phi, \psi \in \Phi
\end{equation}
\end{theorem}

\subsection{Quantum Dynamics in Rigged Hilbert Spaces}

The rigged Hilbert space approach also accommodates quantum dynamics.

\begin{theorem}[Time Evolution]
Let $H$ be a self-adjoint Hamiltonian operator on $\mathcal{H}$ with corresponding rigged Hilbert space $\Phi \subset \mathcal{H} \subset \Phi^*$. If $\Phi$ is invariant under the action of $e^{-itH/\hbar}$ for all $t \in \mathbb{R}$, then the time evolution operator $U(t) = e^{-itH/\hbar}$ can be extended to act on generalized vectors in $\Phi^*$.
\end{theorem}

\begin{proof}
For any $F \in \Phi^*$ and $\phi \in \Phi$, define $U(t)F \in \Phi^*$ by:
\begin{equation}
    \langle U(t)F, \phi \rangle = \langle F, U(-t)\phi \rangle
\end{equation}
This is well-defined because $U(-t)\phi \in \Phi$ by assumption. It is straightforward to verify that this extension satisfies the properties of a group of operators:
\begin{align}
    U(t+s) &= U(t)U(s) \\
    U(0) &= I \\
    U(t)^{-1} &= U(-t)
\end{align}
Thus, time evolution is well-defined for generalized states.
\end{proof}

\section{Practical Examples and Calculations}

\subsection{Momentum Eigenstates}

The momentum operator $\hat{p} = -i\hbar\frac{d}{dx}$ on $L^2(\mathbb{R})$ illustrates the utility of the rigged Hilbert space approach.

\begin{example}[Momentum Eigenstates]
Consider the rigged Hilbert space $\mathcal{S}(\mathbb{R}) \subset L^2(\mathbb{R}) \subset \mathcal{S}'(\mathbb{R})$.

The functions $\psi_p(x) = \frac{1}{\sqrt{2\pi\hbar}}e^{ipx/\hbar}$ for $p \in \mathbb{R}$ satisfy:
\begin{equation}
    \hat{p}\psi_p(x) = p\psi_p(x)
\end{equation}

These functions are not in $L^2(\mathbb{R})$ but exist as generalized vectors in $\mathcal{S}'(\mathbb{R})$. For any $\phi \in \mathcal{S}(\mathbb{R})$, the expansion in terms of momentum eigenstates is given by the Fourier transform:
\begin{equation}
    \phi(x) = \int_{-\infty}^{\infty} \tilde{\phi}(p) \psi_p(x) \, dp
\end{equation}
where $\tilde{\phi}(p) = \frac{1}{\sqrt{2\pi\hbar}}\int_{-\infty}^{\infty} \phi(x) e^{-ipx/\hbar} \, dx$.
\end{example}

\subsection{The Hydrogen Atom}

The hydrogen atom provides a concrete example of a system with mixed spectrum.

\begin{example}[Hydrogen Atom]
The Hamiltonian for the hydrogen atom is:
\begin{equation}
    H = -\frac{\hbar^2}{2m}\nabla^2 - \frac{e^2}{r}
\end{equation}

Its spectrum consists of:
\begin{itemize}
    \item Discrete eigenvalues $E_n = -\frac{e^4m}{2\hbar^2n^2}$ for $n \in \mathbb{N}$, with corresponding eigenfunctions $\psi_{nlm}(r,\theta,\phi)$ that are square-integrable
    \item Continuous spectrum $[0,\infty)$ with generalized eigenfunctions $\psi_{E,l,m}(r,\theta,\phi)$ for $E > 0$ that are not square-integrable
\end{itemize}

In the rigged Hilbert space framework, any state $\Psi$ can be expanded as:
\begin{equation}
    \Psi = \sum_{n,l,m} c_{nlm} \psi_{nlm} + \int_0^{\infty} \sum_{l,m} c_{E,l,m} \psi_{E,l,m} \, dE
\end{equation}
where the first sum represents bound states and the integral represents scattering states.
\end{example}

\section{Conclusion}

The rigged Hilbert space approach significantly enhances spectral theory for unbounded operators by providing a unified framework for handling both discrete and continuous spectra. Its key contributions include:

\begin{itemize}
    \item Rigorous mathematical foundation for generalized eigenvectors and eigenfunction expansions
    \item Unified treatment of discrete and continuous spectral components
    \item Formal diagonalization of operators with continuous spectra
    \item Mathematical justification for physical formalisms like Dirac's bra-ket notation
    \item Extension of spectral theory to a broader class of operators and applications
\end{itemize}

This framework not only resolves theoretical challenges in spectral theory but also provides practical tools for calculations in quantum mechanics and other areas of mathematical physics.

\begin{thebibliography}{9}

\bibitem{Gelfand}
Gelfand, I. M.; Vilenkin, N. Y. (1964), \textit{Generalized Functions, Vol. 4: Applications of Harmonic Analysis}, Academic Press.

\bibitem{Maurin}
Maurin, K. (1968), \textit{General Eigenfunction Expansions and Unitary Representations of Topological Groups}, Polish Scientific Publishers.

\bibitem{Bohm}
Bohm, A.; Gadella, M. (1989), \textit{Dirac Kets, Gamow Vectors and Gelfand Triplets}, Springer-Verlag.

\bibitem{Madrid}
de la Madrid, R. (2005), ``The role of the rigged Hilbert space in Quantum Mechanics,'' \textit{European Journal of Physics}, 26, 287-312.

\bibitem{Antoine}
Antoine, J.-P. (1969), ``Dirac Formalism and Symmetry Problems in Quantum Mechanics. I. General Dirac Formalism,'' \textit{Journal of Mathematical Physics}, 10, 53-69.

\bibitem{Roberts}
Roberts, J. E. (1966), ``Rigged Hilbert spaces in quantum mechanics,'' \textit{Communications in Mathematical Physics}, 3, 98-119.

\bibitem{Melsheimer}
Melsheimer, O. (1974), ``Rigged Hilbert space formalism as an extended mathematical formalism for quantum systems. I. General theory,'' \textit{Journal of Mathematical Physics}, 15, 902-916.

\bibitem{Gadella}
Gadella, M.; Gómez, F. (2002), ``A unified mathematical formalism for the Dirac formulation of quantum mechanics,'' \textit{Foundations of Physics}, 32, 815-869.

\end{thebibliography}

\end{document}
