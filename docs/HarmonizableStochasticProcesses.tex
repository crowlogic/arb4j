\documentclass[12pt]{article}
\usepackage{amsmath, amsthm, amssymb, mathrsfs, geometry}
\geometry{a4paper, margin=1in}

\newtheorem{theorem}{Theorem}[section]
\newtheorem{definition}[theorem]{Definition}
\newtheorem{proposition}[theorem]{Proposition}
\newtheorem{lemma}[theorem]{Lemma}

\title{Comprehensive Mathematical Theory of Harmonizable Stochastic Processes}
\author{Compiled from Fundamental Works}
\date{\today}

\begin{document}

\maketitle

\section{Foundational Framework}

\begin{definition}[Harmonizable Process]
A second-order complex-valued stochastic process $\{X(t)\}_{t \in \mathbb{R}}$ is \textbf{harmonizable} if its covariance function admits:
\begin{equation}
C(s,t) = \iint_{\mathbb{R}^2} e^{i(\lambda s - \mu t)} dF(\lambda, \mu)
\end{equation}
where $F: \mathcal{B}(\mathbb{R}^2) \to \mathbb{C}$ is a positive-definite bimeasure satisfying:
\begin{enumerate}
\item $F(A,B) = \overline{F(B,A)}$
\item $\text{Var}(F) := \sup \sum_{i,j} |F(A_i,B_j)| < \infty$
\end{enumerate}
\end{definition}

\section{Spectral Representation Theory}

\subsection{Loève's Fundamental Characterization}

\begin{theorem}[Loève's Harmonizability Theorem]
For mean-square continuous $X(t)$, the following are equivalent:
\begin{enumerate}
\item $X(t)$ is harmonizable
\item $\exists$ orthogonal increment process $Z(\lambda)$ with:
\begin{equation}
X(t) = \int_{\mathbb{R}} e^{i\lambda t} dZ(\lambda)
\end{equation}
\item The covariance admits (1) with $F$ of bounded Vitali variation
\end{enumerate}
\end{theorem}

\begin{proof}
(i) $\Rightarrow$ (ii): Construct spectral measure via Bochner-Weil-Raikov theorem:
\begin{align*}
Z(\lambda) &= \text{s-}\lim_{T\to\infty} \frac{1}{2T} \int_{-T}^T e^{-i\mu t}X(t)dt \\
\mathbb{E}[Z(A)\overline{Z(B)}] &= F(A,B)
\end{align*}

(ii) $\Rightarrow$ (iii): Compute covariance:
\begin{align*}
C(s,t) &= \mathbb{E}\left[\int e^{i\lambda s}dZ(\lambda)\int e^{-i\mu t}d\overline{Z(\mu)}\right] \\
&= \iint e^{i(\lambda s - \mu t)} \mathbb{E}[dZ(\lambda)d\overline{Z(\mu)}] \\
&= \iint e^{i(\lambda s - \mu t)} dF(\lambda,\mu)
\end{align*}

(iii) $\Rightarrow$ (i): Direct from definition. $\qedhere$
\end{proof}

\subsection{Cramér's Spectral Synthesis}

\begin{theorem}[Cramér's Representation]
Every harmonizable process decomposes as:
\begin{equation}
X(t) = \int_{\mathbb{R}} e^{i\lambda t} dZ(\lambda) + \iint_{\mathbb{R}^2\setminus\Delta} e^{i(\lambda - \mu)t} dW(\lambda,\mu)
\end{equation}
where $\Delta = \{(\lambda,\lambda):\lambda \in \mathbb{R}\}$ and $W$ is a orthogonal random measure.
\end{theorem}

\section{M.M. Rao's Contributions}

\subsection{Canonical Decomposition}

\begin{theorem}[Rao's Decomposition]
Any harmonizable process uniquely splits into:
\begin{equation}
X(t) = X_s(t) + X_p(t)
\end{equation}
where:
\begin{itemize}
\item $X_s(t)$ is strongly harmonizable ($\text{supp}(F) \subseteq \Delta$)
\item $X_p(t)$ is purely nonstationary ($F_p(\Delta) = 0$)
\end{itemize}
\end{theorem}

\begin{proof}
Apply Lebesgue-Radon-Nikodym decomposition to $F$:
\begin{equation*}
F = F|_\Delta + F|_{\mathbb{R}^2\setminus\Delta}
\end{equation*}
Construct components via:
\begin{align*}
X_s(t) &= \int_\Delta e^{i\lambda t} dZ(\lambda) \\
X_p(t) &= \iint_{\mathbb{R}^2\setminus\Delta} e^{i(\lambda - \mu)t} dW(\lambda,\mu) \quad \qedhere
\end{align*}
\end{proof}

\section{Advanced Analytical Properties}

\subsection{Differentiability Criteria}

\begin{theorem}
A harmonizable process $X(t)$ is $n$-times mean-square differentiable iff:
\begin{equation}
\iint_{\mathbb{R}^2} |\lambda\mu|^n d|F|(\lambda,\mu) < \infty
\end{equation}
where $|F|$ is the total variation measure.
\end{theorem}

\subsection{Sampling Theory}

\begin{theorem}[Nonstationary Sampling]
For $X(t)$ with $\text{supp}(F) \subseteq [-\Omega,\Omega]^2$, perfect reconstruction holds:
\begin{equation}
X(t) = \sum_{n=-\infty}^\infty X\left(\frac{n\pi}{\Omega}\right) \text{sinc}\left(\Omega t - n\pi\right)
\end{equation}
\end{theorem}

\section{Operator-Theoretic Formulation}

\subsection{Nuclear Spectral Theory}

For $X(t)$ in Hilbert space $\mathcal{H}$:
\begin{equation}
X(t) = \int_{\mathbb{R}} e^{i\lambda t} dE(\lambda)\xi
\end{equation}
where $E$ is a spectral measure on $\mathcal{H}$ and $\xi \in \mathcal{H}$.

\subsection{Karhunen-Loève Expansion}

\begin{equation}
X(t) = \sum_{k=1}^\infty \sqrt{\lambda_k} \phi_k(t)\xi_k
\end{equation}
with $\{\phi_k\}$ from:
\begin{equation}
\int_{\mathbb{R}} C(s,t)\phi_k(s)ds = \lambda_k\phi_k(t)
\end{equation}

\section{Modern Applications}

\subsection{Time-Frequency Analysis}

Wigner-Ville distribution for harmonizable processes:
\begin{equation}
W_X(t,\omega) = \iint_{\mathbb{R}^2} e^{i(\lambda - \mu)t} e^{-i\omega(\lambda + \mu)/2} dF(\lambda,\mu)
\end{equation}

\subsection{Fractal Processes}

Fractional harmonizable process with Hurst index $H$:
\begin{equation}
X_H(t) = \int_{\mathbb{R}} \frac{e^{i\lambda t} - 1}{|\lambda|^{H+1/2}} dZ(\lambda)
\end{equation}

\end{document}
