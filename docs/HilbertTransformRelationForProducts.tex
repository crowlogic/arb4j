\documentclass[12pt]{article}
\usepackage{amsmath,amssymb,amsthm}
\usepackage{enumitem}
\usepackage{hyperref}
\usepackage{geometry}
\geometry{margin=1in}
\title{Hilbert Transform Relations for Products}
\author{Gerald D. Cain\\
Department of Electrical and Electronic Engineering,\\
Polytechnic of Central London, London W1, England}
\date{}

\newtheorem{theorem}{Theorem}[section]
\newtheorem{definition}{Definition}[section]
\newtheorem{remark}{Remark}[section]

\begin{document}
\maketitle

\begin{abstract}
A general formula for the Hilbert transform of a product of complex-valued functions is developed. Certain simplifications are then exhibited for products often encountered in the context of modulation and signal processing. The approach chosen is one of frequency partitioning; this permits signal definition on complementary sections of the frequency axis and leads to compact and easily manipulated expressions.
\end{abstract}

\section{Introduction}

The Hilbert transform is a useful analytical tool that has been applied extensively in signal and system theory. While this transform provides a tidy means of relating certain orthogonal time or frequency functions, the actual computation of transform pairs and the reduction of transform expressions is usually a difficult task.

One important problem is that of finding Hilbert transforms of products since most useful applications are replete with instances requiring multiplication of functions. Product relationships came under scrutiny in connection with narrow-band signal representation. Some controversy \cite{Lerner1960, Kelly1960, Urkowitz1962, Bedrosian1963, Rihaczek1966, Rubin1966, Nuttall1966} surrounded this matter before Bedrosian’s product theorem was introduced \cite{Bedrosian1963, Nuttall1966}. Bedrosian’s theorem, however, is applicable only in circumstances that are often overly restrictive. The purpose of this letter is to present a more general result and to show the relevance of work by Tricomi \cite{Tricomi1951} and Titchmarsh \cite{Titchmarsh1937}; also, some special simplifications for expressions common in modulation and signal processing analyses are demonstrated. These results can be used to advantage in a variety of specific situations (for example, \cite{Cain1965}).

\section{Hilbert Transform Definition}

\begin{definition}\label{def:hilbert}
The Hilbert transform $\mathcal{H}[g(t)]$ of a function $g(t)$ is the linear operator defined as
\begin{equation}
\mathcal{H}[g(t)] = \frac{1}{\pi} \, \mathrm{p.v.} \int_{-\infty}^{\infty} \frac{g(\tau)}{t - \tau} \, d\tau
\label{eq:hilbert_def}
\end{equation}
where the integration is taken in the Cauchy principal value sense and $\mathcal{H}[\cdot]$ denotes the Hilbert transform.
\end{definition}

If $t$ is identified as a time variable, the Hilbert transform may be viewed as a $-90^\circ$ phase-shift operator; such a viewpoint is often beneficial in understanding the usefulness of this transform. Since constants are lost in Hilbert transformation, we stipulate, in order to provide for unique inverse transformation, that no functions with DC components be allowed. It will be assumed that all indicated Hilbert transforms are being taken in a distributional sense so that we can cater for “power”-type signals (such as sinusoids) which so frequently arise in communication studies, as well as the finite energy signals originally specified in \cite{Bedrosian1963}.

\section{Bedrosian's Theorem}

\begin{theorem}[Bedrosian's Theorem]\label{thm:bedrosian}
Let $x(t)$ and $y(t)$ be (generally) complex-valued time functions. If the following sufficiency conditions hold:
\begin{enumerate}[label=(\alph*)]
    \item Denoting $f_c$ as the smallest frequency value along the positive $f$ axis at which the Fourier transform $Y(f)$ is nonzero, it is found that $X(f)$ vanishes below $-f_c$.
    \item Labeling as $-f_n$ the largest frequency value along the negative $f$ axis at which $Y(f)$ is nonzero, we find that $X(f)$ vanishes above $f_n$.
\end{enumerate}
Then,
\begin{equation}
\mathcal{H}[x(t) y(t)] = x(t) \mathcal{H}[y(t)]
\label{eq:bedrosian}
\end{equation}
\end{theorem}

\begin{remark}
If $Y(f)$ happens to be nonzero throughout an interval adjoining the origin, one is confronted with the special problem of accommodating the point $f = 0$ in any declaration of $f_c$ or $-f_n$. This problem can be overcome by the conceptual artifice of sidestepping the frequency origin; that is, for a continuous spectrum $Y(f)$ which is contiguous to the frequency origin from the positive frequency side, it is necessary to assign $f_c$ an arbitrarily small positive value, say $\epsilon_+$. Similarly, if on the negative frequency axis $Y(f)$ is nonzero on an interval adjacent to the origin, then $-f_n$ is chosen to be the small negative offset $-\epsilon_-$. This means, for instance, that if $Y(f)$ has nonzero continuous spectral content straddling the origin, a delta function at the origin (that is, a constant $x(t)$, which is not allowable due to our initial stipulation) is the only spectrum for $X(f)$ that would, according to this theorem, satisfy equation~\eqref{eq:bedrosian}.
\end{remark}

Clearly, Bedrosian’s theorem is not generally applicable for spectrally overlapping double-sided baseband signals.

\section{Generalization and Analytic Signals}

In the case of a single-sided $Y(f)$ that vanishes along the positive (or negative) frequency axis, no $f_c$ (or $-f_n$) is encountered; therefore, there is no constraint on the extent of $X(f)$ for negative (positive) frequency. It is important to note that conditions (a) and (b) are sufficiency conditions only and that equation~\eqref{eq:bedrosian} might, in some examples, hold true even if these conditions fail.

One of the simplest cases meeting the qualifications of the theorem is that of a low-pass signal $x(t)$ strictly band-limited inside $(-f_c, f_c)$ combined with a high-pass signal $y(t)$ with no spectral content inside that interval. Another obvious case is that in which both signals are analytic; this case will enable us to extend Bedrosian’s theorem.

We can form analytic signals from real-valued $r$ and $s$:
\begin{align}
z(t) &= r(t) + j \mathcal{H}[r(t)] \label{eq:analytic_z} \\
w(t) &= s(t) + j \mathcal{H}[s(t)] \label{eq:analytic_w}
\end{align}

Employing equation~\eqref{eq:bedrosian} and equating real parts, we are led to an expression that is independent of spectral considerations:
\begin{equation}
\mathcal{H}[r(t) s(t)] = r(t) \mathcal{H}[s(t)] + \mathcal{H}[r(t)] s(t) + \mathcal{H}[\mathcal{H}[r(t)] \mathcal{H}[s(t)]]
\label{eq:tricomi}
\end{equation}
Equation~\eqref{eq:tricomi} seems to have been derived first (and more rigorously than was done here) by Tricomi \cite{Tricomi1951}, although it appears to have seen little use in engineering literature.

\section{General Product Formula for Complex-Valued Functions}

We now obtain a new equation with the form of equation~\eqref{eq:tricomi} which is true for complex-valued functions. Introduce two new arbitrary real-valued functions $u(t)$ and $w(t)$. Then three additional equations like equation~\eqref{eq:tricomi} are written out, where $rs$ is replaced first by $-uw$, then by $jvs$, and finally by $jnu$. These three equations are added to equation~\eqref{eq:tricomi} and we find that, upon defining
\begin{align}
x(t) &= r(t) + j u(t) \label{eq:x_general} \\
y(t) &= s(t) + j w(t) \label{eq:y_general}
\end{align}
to be general complex-valued functions, we have
\begin{equation}
\mathcal{H}[x(t) y(t)] = x(t) \mathcal{H}[y(t)] + y(t) \mathcal{H}[x(t)] + \mathcal{H}[\mathcal{H}[x(t)] \mathcal{H}[y(t)]]
\label{eq:general_product}
\end{equation}

Thus, equation~\eqref{eq:bedrosian} is just a degenerate form of equation~\eqref{eq:general_product} that is obtained when conditions (a) and (b) hold. Or the last two terms in equation~\eqref{eq:general_product} may be looked upon as a correction factor generally needed when applying equation~\eqref{eq:bedrosian} with no heed to the spectral properties of $x$ and $y$. As a practical matter, of course, evaluation of the right-hand side of equation~\eqref{eq:general_product} is best avoided, since three different Hilbert transforms (one of which is again the transform of a product) are required.

A further simplification is interesting. In equation~\eqref{eq:general_product} choose $y$ to be identically equal to $x$ and note that two successive Hilbert transforms of a function result in multiplication of the original function by $-1$. We find for this arbitrary function $x$ that
\begin{equation}
\mathcal{H}[x(t) x(t)] = x(t) \mathcal{H}[x(t)] + x(t) \mathcal{H}[x(t)] - x(t) x(t) = 2 x(t) \mathcal{H}[x(t)] - x(t)^2
\label{eq:titchmarsh}
\end{equation}
This is a generalization of one of Titchmarsh’s real-valued results \cite{Titchmarsh1937} to complex-valued functions.

\section{Products of Spectrally Disjoint Signals}

If it happens that the two functions involved in the general formula equation~\eqref{eq:general_product} have spectra that occupy complementary segments of the frequency axis, some simplification of equation~\eqref{eq:general_product} is possible even when conditions (a) and (b) are not directly satisfied. For ease of illustration, assume an uncomplicated case where $x(t)$ is devoid of spectral content over the intervals $[-f_d, -f_c]$ and $[f_c, f_d]$ where $Y(f)$ resides. Following a frequency partitioning approach \cite{Haber1972}, we define
\begin{equation}
x(t) = x_1(t) + x_2(t)
\label{eq:partition}
\end{equation}
with $x_1$, the low-pass portion of $x$, possessing a spectrum limited to $(-f_c, f_c)$, and $x_2$ being the high-pass content of $x$, with a spectrum lying outside $[-f_d, f_d]$.

Now equation~\eqref{eq:general_product} yields the same as two separate applications of equation~\eqref{eq:bedrosian}:
\begin{align}
\mathcal{H}[x(t) y(t)] &= x_1(t) \mathcal{H}[y(t)] + x_2(t) y(t)
\label{eq:spectral1} \\
\mathcal{H}[x(t) y(t)] &= x_1(t) y(t) + x_2(t) \mathcal{H}[y(t)]
\label{eq:spectral2}
\end{align}

Alternate forms are obtained by substituting for either $x_1$ or $x_2$ via equation~\eqref{eq:partition}.

A category of problems of considerable importance is that where $y(t)$ is composed of a sum of sinusoids with frequencies falling inside the bandwidth of $x(t)$. Assuming that there are no sinusoids in $x$ which are coincident with those comprising $y$, the two spectra are effectively complementary. That is, assignment of the value zero to $X(f)$ at those isolated points where $Y(f)$ is contributing delta functions does not (for spectra of practical significance) alter the mathematical recoverability of $x(t)$. Therefore, treating only a simple monochromatic $y$,
\begin{align}
\mathcal{H}[x(t) \cos 2\pi f_c t] &= x(t) \mathcal{H}[\cos 2\pi f_c t] + \mathcal{H}[x(t)] \cos 2\pi f_c t - \mathcal{H}[x(t)] \mathcal{H}[\cos 2\pi f_c t] \label{eq:cosine_product1} \\
\mathcal{H}[x(t) \cos 2\pi f_c t] &= x(t) \sin 2\pi f_c t + \mathcal{H}[x(t)] \cos 2\pi f_c t - x(t) \sin 2\pi f_c t \label{eq:cosine_product2}
\end{align}

In equations~\eqref{eq:cosine_product1} and \eqref{eq:cosine_product2}, the last two terms may be thought of as correction terms that modify transforms taken directly in terms of the symbols on the left-hand side of the equation. If $x$ is real-valued, the correction terms in equation~\eqref{eq:cosine_product1} take the form of a lower single-sideband version of $-\mathcal{H}[x(t)]$, while equation~\eqref{eq:cosine_product2} utilizes a lower single-sideband relationship for $x(t)$.

\section{Summary}

A general Hilbert transform product theorem is given by equation~\eqref{eq:general_product}. This equation is the principal result of this letter and embraces the results of three earlier investigators. If the component functions have spectra that are disjoint, the more specific equations~\eqref{eq:spectral1}, \eqref{eq:spectral2}, or their alternates may be used; straightforward modification will be necessary if the spectral pattern is more complicated than that assumed here for demonstration purposes. Equations~\eqref{eq:cosine_product1} and \eqref{eq:cosine_product2} are particularly helpful relationships in many signal processing applications since multiplication by sinusoids is common. All of these equations accommodate either real or complex-valued functions that are free of additive constant components; equations~\eqref{eq:cosine_product1} and \eqref{eq:cosine_product2} additionally require that $x(t)$ does not contain sinusoids of frequency $f_c$.

\section*{Acknowledgment}
The author thanks the reviewers for their helpful comments.

\begin{thebibliography}{11}
\bibitem{Lerner1960}
R. M. Lerner, ``A matched filter detection system for complicated Doppler shifted signals,'' \emph{IRE Trans. Inform. Theory}, vol. IT-6, pp. 375--385, June 1960.

\bibitem{Kelly1960}
E. J. Kelly, I. S. Reed, and W. L. Root, ``The detection of radar echoes in noise,'' \emph{J. Soc. Ind. Appl. Math.}, vol. 8, pp. 309--341, June 1960.

\bibitem{Urkowitz1962}
H. Urkowitz, ``Hilbert transforms of band-pass functions,'' \emph{Proc. IRE (Corresp.)}, vol. 50, p. 2143, Oct. 1962.

\bibitem{Bedrosian1963}
E. Bedrosian, ``A product theorem for Hilbert transforms,'' \emph{Proc. IEEE (Corresp.)}, vol. 51, pp. 868--869, May 1963.

\bibitem{Rihaczek1966}
A. W. Rihaczek and E. Bedrosian, ``Hilbert transforms and the complex representation of real signals,'' \emph{Proc. IEEE (Lett.)}, vol. 54, pp. 434--435, Mar. 1966.

\bibitem{Rubin1966}
W. L. Rubin and J. V. DiFranco, ``Concerning the complex representation of narrow-band signals,'' \emph{Proc. IEEE (Lett.)}, vol. 54, p. 1088, Aug. 1966.

\bibitem{Nuttall1966}
A. H. Nuttall and E. Bedrosian, ``On the quadrature approximation to the Hilbert transform of modulated signals,'' \emph{Proc. IEEE (Lett.)}, vol. 54, pp. 1458--1459, Oct. 1966.

\bibitem{Tricomi1951}
F. G. Tricomi, ``On the finite Hilbert transformation,'' \emph{Quart. J. Math. Oxford ser. 2}, pp. 199--211, 1951.

\bibitem{Titchmarsh1937}
E. C. Titchmarsh, \emph{Introduction to the Theory of Fourier Integrals}. London, England: Oxford Univ. Press, 1937, p. 137.

\bibitem{Cain1965}
G. D. Cain, ``Hilbert transform description of linear filtering,'' \emph{Electron. Lett.}, vol. 1, pp. 232--233, June 1965.

\bibitem{Haber1972}
F. Haber, ``Signal representation,'' \emph{IEEE Trans. Commun. Technol. (Corresp.)}, vol. 8, pp. 360--382, July 27, 1972.
\end{thebibliography}

\end{document}
