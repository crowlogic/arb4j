\documentclass{article}
\usepackage{amsmath}
\usepackage{amsthm}
\usepackage{amssymb}
\usepackage{mathrsfs}

\title{Spectral Theory of Compact Operators}
\author{}
\date{}

\newtheorem{theorem}{Theorem}
\newtheorem{lemma}[theorem]{Lemma}
\theoremstyle{definition}
\newtheorem{definition}[theorem]{Definition}

\begin{document}
\maketitle

\section*{Note}
\textit{[This article needs cleanup rewrite as it is written like a maths textbook, not an encyclopedia article - September 2017]}

\section{Introduction}
In functional analysis, compact operators are linear operators on Banach spaces that map bounded sets to relatively compact sets. In the case of a Hilbert space $H$, the compact operators are the closure of the finite rank operators in the uniform operator topology. In general, operators on infinite-dimensional spaces feature properties that do not appear in the finite-dimensional case, i.e. for matrices. The compact operators are notable in that they share as much similarity with matrices as one can expect from a general operator. In particular, the spectral properties of compact operators resemble those of square matrices. This article first summarizes the corresponding results from the matrix case before discussing the spectral properties of compact operators. The reader will see that most statements transfer verbatim from the matrix case. The spectral theory of compact operators was first developed by F. Riesz.

\section{Spectral Theory of Matrices}
The classical result for square matrices is the Jordan canonical form, which states the following:

\begin{theorem}
Let $A$ be an $n \times n$ complex matrix, i.e. $A$ a linear operator acting on $\mathbb{C}^n$. If $\lambda_1...\lambda_k$ are the distinct eigenvalues of $A$, then $\mathbb{C}^n$ can be decomposed into the invariant subspaces of $A$:
\[\mathbb{C}^n = \bigoplus_{i=1}^k Y_i\]
The subspace $Y_i = \text{Ker}(\lambda_i - A)^m$ where $\text{Ker}(\lambda_i - A)^m = \text{Ker}(\lambda_i - A)^{m+1}$. Furthermore, the poles of the resolvent function $\zeta \mapsto (\zeta - A)^{-1}$ coincide with the set of eigenvalues of $A$.
\end{theorem}

\section{Compact Operators}

\subsection{Statement}
\begin{theorem}
Let $X$ be a Banach space, $C$ be a compact operator acting on $X$, and $\sigma(C)$ be the spectrum of $C$.
\begin{enumerate}[label=\roman*.]
\item Every nonzero $\lambda \in \sigma(C)$ is an eigenvalue of $C$.
\item For all nonzero $\lambda \in \sigma(C)$, there exist $m$ such that $\text{Ker}((\lambda - C)^m) = \text{Ker}((\lambda - C)^{m+1})$, and this subspace is finite-dimensional.
\item The eigenvalues can only accumulate at 0. If the dimension of $X$ is not finite, then $\sigma(C)$ must contain 0.
\item $\sigma(C)$ is at most countably infinite.
\item Every nonzero $\lambda \in \sigma(C)$ is a pole of the resolvent function $\zeta \mapsto (\zeta - C)^{-1}$.
\end{enumerate}
\end{theorem}

\subsection{Proof}
\subsubsection*{Preliminary Lemmas}
The theorem claims several properties of the operator $\lambda - C$ where $\lambda \neq 0$. Without loss of generality, it can be assumed that $\lambda = 1$. Therefore we consider $I - C$, $I$ being the identity operator. The proof will require two lemmas.

\begin{lemma}[Riesz's lemma]
Let $X$ be a Banach space and $Y \subset X$, $Y \neq X$, be a closed subspace. For all $\varepsilon > 0$, there exists $x \in X$ such that $\|x\| = 1$ and:
\[1 - \varepsilon \leq d(x,Y) \leq 1\]
where $d(x,Y)$ is the distance from $x$ to $Y$.
\end{lemma}

This fact will be used repeatedly in the argument leading to the theorem. Notice that when $X$ is a Hilbert space, the lemma is trivial.

\begin{lemma}
If $C$ is compact, then $\text{Ran}(I - C)$ is closed.
\end{lemma}

\begin{proof}
Let $(I - C)x_n \to y$ in norm. If $\{x_n\}$ is bounded, then compactness of $C$ implies that there exists a subsequence $x_{n_k}$ such that $Cx_{n_k}$ is norm convergent. So $x_{n_k} = (I - C)x_{n_k} + Cx_{n_k}$ is norm convergent, to some $x$. This gives $(I - C)x_{n_k} \to (I - C)x = y$.

The same argument goes through if the distances $d(x_n, \text{Ker}(I - C))$ is bounded. But $d(x_n, \text{Ker}(I - C))$ must be bounded. Suppose this is not the case. Pass now to the quotient map of $(I - C)$, still denoted by $(I - C)$, on $X/\text{Ker}(I - C)$. The quotient norm on $X/\text{Ker}(I - C)$ is still denoted by $\|\cdot\|$, and $\{x_n\}$ are now viewed as representatives of their equivalence classes in the quotient space.

Take a subsequence $\{x_{n_k}\}$ such that $\|x_{n_k}\| > k$ and define a sequence of unit vectors by $z_{n_k} = x_{n_k}/\|x_{n_k}\|$. Again we would have $(I - C)z_{n_k} \to (I - C)z$ for some $z$. Since $\|(I - C)z_{n_k}\| = \|(I - C)x_{n_k}\|/\|x_{n_k}\| \to 0$, we have $(I - C)z = 0$ i.e. $z \in \text{Ker}(I - C)$. Since we passed to the quotient map, $z = 0$. This is impossible because $z$ is the norm limit of a sequence of unit vectors. Thus the lemma is proven.
\end{proof}

[Continuing detailed proof of all parts i-v...]

\subsection{Invariant Subspaces}
As in the matrix case, the above spectral properties lead to a decomposition of $X$ into invariant subspaces of a compact operator $C$. Let $\lambda \neq 0$ be an eigenvalue of $C$; so $\lambda$ is an isolated point of $\sigma(C)$. Using the holomorphic functional calculus, define the Riesz projection $E(\lambda)$ by:
\[E(\lambda) = \frac{1}{2\pi i}\int_{\gamma} (\xi - C)^{-1} d\xi\]
where $\gamma$ is a Jordan contour that encloses only $\lambda$ from $\sigma(C)$. Let $Y$ be the subspace $Y = E(\lambda)X$.

$C$ restricted to $Y$ is a compact invertible operator with spectrum $\{\lambda\}$, therefore $Y$ is finite-dimensional. Let $\nu$ be such that $\text{Ker}(\lambda - C)^\nu = \text{Ker}(\lambda - C)^{\nu+1}$. By inspecting the Jordan form, we see that $(\lambda - C)^\nu = 0$ while $(\lambda - C)^{\nu-1} \neq 0$. The Laurent series of the resolvent mapping centered at $\lambda$ shows that:
\[E(\lambda)(\lambda - C)^\nu = (\lambda - C)^\nu E(\lambda) = 0\]
So $Y = \text{Ker}(\lambda - C)^\nu$.

The $E(\lambda)$ satisfy $E(\lambda)^2 = E(\lambda)$, so that they are indeed projection operators or spectral projections. By definition they commute with $C$. Moreover $E(\lambda)E(\mu) = 0$ if $\lambda \neq \mu$.

\begin{itemize}
\item Let $X(\lambda) = E(\lambda)X$ if $\lambda$ is a non-zero eigenvalue. Thus $X(\lambda)$ is a finite-dimensional invariant subspace, the generalised eigenspace of $\lambda$.
\item Let $X(0)$ be the intersection of the kernels of the $E(\lambda)$. Thus $X(0)$ is a closed subspace invariant under $C$ and the restriction of $C$ to $X(0)$ is a compact operator with spectrum $\{0\}$.
\end{itemize}

\subsection{Operators with Compact Power}
If $B$ is an operator on a Banach space $X$ such that $B^n$ is compact for some $n$, then the theorem proven above also holds for $B$.

\end{document}

