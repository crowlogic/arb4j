\documentclass{article}
\usepackage[english]{babel}
\usepackage{amsmath,amssymb,latexsym}

%%%%%%%%%% Start TeXmacs macros
\newcommand{\tmaffiliation}[1]{\\ #1}
\newenvironment{proof}{\noindent\textbf{Proof\ }}{\hspace*{\fill}$\Box$\medskip}
\newtheorem{lemma}{Lemma}
\newtheorem{theorem}{Theorem}
%%%%%%%%%% End TeXmacs macros

\begin{document}

\title{L2 Norm Preservation Under Monotonic Substitutions}

\author{
  Stephen Crowley
  \tmaffiliation{July 24, 2025}
}

\date{}

\maketitle

\begin{theorem}
  [L2 Norm Preservation via Square Root Jacobian Factor] Let $g : I \to J$ be
  a strictly monotonic and differentiable function between intervals $I, J
  \subseteq \mathbb{R}$ (possibly unbounded), with $g' (y) \neq 0$ for all $y
  \in I$. For any $f \in L^2 (J, dx)$, define the transformed function
  $\tilde{f} : I \to \mathbb{C}$ by
  \begin{equation}
    \tilde{f} (y) = f (g (y)) \sqrt{|g' (y) |}
  \end{equation}
  Then $\tilde{f} \in L^2 (I, dy)$ and
  \begin{equation}
    \| \tilde{f} \|_{L^2 (I, dy)} = \|f\|_{L^2 (J, dx)}
  \end{equation}
\end{theorem}

\begin{proof}
  Without loss of generality, assume $g$ is strictly increasing (the
  decreasing case follows by considering $- g$).
  
  First, establish the change of variables formula. For any measurable set $E
  \subseteq J$:
  \begin{equation}
    \int_E |f (x) |^2 \hspace{0.17em} dx = \int_{g^{- 1} (E)} |f (g (y)) |^2
    |g' (y) |  \hspace{0.17em} dy
  \end{equation}
  This follows from the standard change of variables theorem, since $g$ is
  strictly monotonic and differentiable with $g' (y) \neq 0$.
  
  To handle potentially unbounded intervals, consider the norm computation:
  
  \begin{align}
    \| \tilde{f} \|_{L^2 (I, dy)}^2 & = \int_I | \tilde{f} (y) |^2 
    \hspace{0.17em} dy \\
    & = \int_I |f (g (y)) \sqrt{|g' (y) |} |^2  \hspace{0.17em} dy \\
    & = \int_I |f (g (y)) |^2 |g' (y) |  \hspace{0.17em} dy. 
  \end{align}
  
  By the change of variables formula applied to $J = g (I)$:
  \begin{equation}
    \int_I |f (g (y)) |^2 |g' (y) | \hspace{0.17em} dy = \int_J |f (x) |^2 
    \hspace{0.17em} dx = \|f\|_{L^2 (J, dx)}^2
  \end{equation}
  For unbounded intervals, this equality holds by the monotone convergence
  theorem: approximate $I$ by an increasing sequence of bounded intervals $I_n
  \uparrow I$, apply the result to each $I_n$, and take the limit.
  
  Therefore:
  \begin{equation}
    \| \tilde{f} \|_{L^2 (I, dy)} = \|f\|_{L^2 (J, dx)}
  \end{equation}
  The integrability of $\tilde{f}$ follows immediately from the norm equality
  and the assumption that $f \in L^2 (J, dx)$.
\end{proof}

\begin{lemma}
  [Density of Transformed Functions] Under the conditions of Theorem 1, the
  set $\{f (g (\cdot)) : f \in L^2 (J, dx)\}$ is dense in $L^2 (I, |g' (y) |
  \hspace{0.17em} dy)$, where $L^2 (I, |g' (y) | \hspace{0.17em} dy)$ denotes
  the space of square-integrable functions with respect to the measure $|g'
  (y) |  \hspace{0.17em} dy$.
\end{lemma}

\begin{proof}
  The map $f \mapsto f \circ g$ is an isometric isomorphism from $L^2 (J, dx)$
  to $L^2 (I, |g' (y) | \hspace{0.17em} dy)$ by the change of variables
  formula. Since $L^2 (J, dx)$ is complete, its image under an isometry is
  also complete, hence dense in itself.
\end{proof}

\begin{theorem}
  [Necessity of Square Root Factor] Under the same conditions as Theorem 1,
  the factor $\sqrt{|g' (y) |}$ is necessary for L2 norm preservation. That
  is, if $h (y) = f (g (y)) \cdot \phi (y)$ for some measurable function $\phi
  : I \to \mathbb{R}^+$ satisfies $\|h\|_{L^2 (I, dy)} = \|f\|_{L^2 (J, dx)}$
  for all $f \in L^2 (J, dx)$, then $\phi (y) = \sqrt{|g' (y) |}$ almost
  everywhere.
\end{theorem}

\begin{proof}
  Suppose $\|f (g (\cdot)) \cdot \phi (\cdot)\|_{L^2 (I, dy)} = \|f\|_{L^2 (J,
  dx)}$ for all $f \in L^2 (J, dx)$.
  
  Then for any $f \in L^2 (J, dx)$:
  \begin{equation}
    \int_I |f (g (y)) |^2 | \phi (y) |^2 \hspace{0.17em} dy =\|f\|_{L^2 (J,
    dx)}^2 = \int_I |f (g (y)) |^2 |g' (y) |  \hspace{0.17em} dy
  \end{equation}
  where the second equality follows from the change of variables formula.
  
  Therefore:
  \begin{equation}
    \int_I |f (g (y)) |^2  (| \phi (y) |^2 - |g' (y) |)  \hspace{0.17em} dy =
    0
  \end{equation}
  for all $f \in L^2 (J, dx)$.
  
  By Lemma 1, functions of the form $f (g (y))$ are dense in $L^2 (I, |g' (y)
  | \hspace{0.17em} dy)$. For any $u \in L^2 (I, |g' (y) | \hspace{0.17em}
  dy)$, there exists a sequence $f_n \in L^2 (J, dx)$ such that $f_n (g (y))
  \to u (y)$ in $L^2 (I, |g' (y) | \hspace{0.17em} dy)$.
  
  Since $| \phi (y) |^2 - |g' (y) |$ is integrable with respect to $|g' (y) | 
  \hspace{0.17em} dy$ (by the boundedness of the norm-preserving property), we
  have:
  \begin{equation}
    \int_I |u (y) |^2  (| \phi (y) |^2 - |g' (y) |)  \hspace{0.17em} dy = 0
  \end{equation}
  for all $u \in L^2 (I, |g' (y) | \hspace{0.17em} dy)$.
  
  In particular, taking $u (y) = \text{sgn} (| \phi (y) |^2 - |g' (y) |) \cdot
  \textbf{1}_{\{| \phi (y) |^2 \neq |g' (y) |\}} (y)$, we obtain:
  \begin{equation}
    \int_I || \phi (y)  |^2 - |g' (y) || \hspace{0.17em} |g' (y) | 
    \hspace{0.17em} dy = 0
  \end{equation}
  Since $|g' (y) | > 0$ almost everywhere, this implies $| \phi (y) |^2 = |g'
  (y) |$ almost everywhere.
  
  Taking $\phi (y) > 0$, we conclude $\phi (y) = \sqrt{|g' (y) |}$ almost
  everywhere.
\end{proof}

\begin{theorem}
  [Extension to General Measures] Let $\mu$ and $\nu$ be $\sigma$-finite
  measures on $I$ and $J$ respectively, and let $g : I \to J$ be a measurable
  bijection. If $\nu = \mu \circ g^{- 1}$ (i.e., $\nu (E) = \mu (g^{- 1} (E))$
  for all measurable $E \subseteq J$), then for $f \in L^2 (J, d \nu)$:
  \begin{equation}
    \tilde{f} (y) = f (g (y)) \sqrt{\frac{d (\mu \circ g^{- 1})}{d \mu} (y)}
  \end{equation}
  satisfies $\| \tilde{f} \|_{L^2 (I, d \mu)} = \|f\|_{L^2 (J, d \nu)}$, where
  $\frac{d (\mu \circ g^{- 1})}{d \mu}$ is the Radon-Nikodym derivative.
\end{theorem}

\begin{proof}
  When $\mu$ and $\nu$ are both Lebesgue measure and $g$ is differentiable,
  the Radon-Nikodym derivative is $|g' (y) |$, reducing to Theorem 1. The
  general case follows by the same change of variables argument using the
  definition of the pushforward measure.
\end{proof}

\end{document}
