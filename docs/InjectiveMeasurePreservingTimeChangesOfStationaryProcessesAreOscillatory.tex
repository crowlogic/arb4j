\documentclass{article}
\usepackage[english]{babel}
\usepackage{amsmath,amssymb,textcomp,latexsym}

%%%%%%%%%% Start TeXmacs macros
\newenvironment{proof}{\noindent\textbf{Proof\ }}{\hspace*{\fill}$\Box$\medskip}
\newtheorem{corollary}{Corollary}
\newtheorem{definition}{Definition}
\newtheorem{theorem}{Theorem}
%%%%%%%%%% End TeXmacs macros

\begin{document}

\title{Injective Measure-Preserving Time-Changes of Stationary Processes are
Oscillatory}

\author{Stephen Crowley}

\date{July 24, 2025}

\maketitle

\section{\section*{Oscillatory Processes and Normalized Injective
Time-Changes}}

\begin{definition}
  [Oscillatory Process] A complex-valued second-order stochastic process
  $\{X_t \}_{t \in I}$ is said to be oscillatory if there exists a family of
  functions $\phi_t (\omega)$ and a complex orthogonal increment process $Z
  (\omega)$ with $E |dZ (\omega) |^2 = d \mu (\omega)$ such that
  \[ X_t = \int_{- \infty}^{\infty} \phi_t (\omega)  \hspace{0.17em} dZ
     (\omega) \hspace{0.17em}, \]
  where $\phi_t (\omega) = A_t (\omega) e^{i \omega t}$ and $A_t (\omega)$ is
  quadratically integrable with respect to $d \mu$.
\end{definition}

\begin{definition}
  [Stationary Process] A second-order process $\{S_t \}_{t \in J}$ is
  stationary if it admits the spectral representation
  \[ S_t = \int_{- \infty}^{\infty} e^{i \omega t}  \hspace{0.17em} dZ
     (\omega) \]
  for some orthogonal increment process $Z (\omega)$ with $E |dZ (\omega) |^2
  = d \mu (\omega)$.
\end{definition}

\begin{theorem}
  [Time-Varying Filter for Injective Time-Change] Let $S_t$ be a stationary
  process and $\theta : \mathbb{R} \to \mathbb{R}$ be smooth and strictly
  increasing with $\theta' (t) > 0$. To achieve the transformation
  \[ X_t = \sqrt{\theta' (t)} S_{\theta (t)} \]
  via convolution $X_t = \int_{- \infty}^{\infty} S_{t - u} h_t (u) du$, the
  time-varying impulse response must be
  \[ h_t (u) = \sqrt{\theta' (t)} \delta (u - (t - \theta (t))) . \]
\end{theorem}

\begin{proof}
  For the convolution to yield $X_t = \sqrt{\theta' (t)} S_{\theta (t)}$, the
  argument of $S$ in the integrand must equal $\theta (t)$ when the delta
  function is activated. This requires:
  \[ t - u = \theta (t) \]
  Solving for $u$:
  \[ u = t - \theta (t) \]
  Therefore:
  \[ h_t (u) = \sqrt{\theta' (t)} \delta (u - (t - \theta (t))) \]
  Verification by direct computation:
  
  \begin{align}
    X_t & = \int_{- \infty}^{\infty} S_{t - u} h_t (u) du \\
    & = \int_{- \infty}^{\infty} S_{t - u}  \sqrt{\theta' (t)} \delta (u - (t
    - \theta (t))) du \\
    & = \sqrt{\theta' (t)} S_{t - (t - \theta (t))} \\
    & = \sqrt{\theta' (t)} S_{\theta (t)} 
  \end{align}
\end{proof}

\begin{theorem}
  [Oscillatory Representation of Injective Time-Change] The process $X_t =
  \sqrt{\theta' (t)} S_{\theta (t)}$ admits the oscillatory representation
  \[ X_t = \int_{- \infty}^{\infty} \phi_t (\omega) dZ (\omega) \]
  where
  \[ \phi_t (\omega) = \sqrt{\theta' (t)} e^{i \omega \theta (t)} . \]
\end{theorem}

\begin{proof}
  Starting from the spectral representation of $S_t$:
  
  \begin{align}
    X_t & = \sqrt{\theta' (t)} S_{\theta (t)} \\
    & = \sqrt{\theta' (t)}  \int_{- \infty}^{\infty} e^{i \omega \theta (t)}
    dZ (\omega) \\
    & = \int_{- \infty}^{\infty} \sqrt{\theta' (t)} e^{i \omega \theta (t)}
    dZ (\omega) 
  \end{align}
  
  Thus $\phi_t (\omega) = \sqrt{\theta' (t)} e^{i \omega \theta (t)}$.
\end{proof}

\begin{corollary}
  [Envelope in Standard Form] The oscillatory functions can be written in the
  standard form $\phi_t (\omega) = A_t (\omega) e^{i \omega t}$ where
  \[ A_t (\omega) = \sqrt{\theta' (t)} e^{i \omega (\theta (t) - t)} . \]
\end{corollary}

\begin{proof}
  Factor the exponential:
  
  \begin{align}
    \phi_t (\omega) & = \sqrt{\theta' (t)} e^{i \omega \theta (t)} \\
    & = \sqrt{\theta' (t)} e^{i \omega (\theta (t) - t)} e^{i \omega t} \\
    & = A_t (\omega) e^{i \omega t} 
  \end{align}
  
  where $A_t (\omega) = \sqrt{\theta' (t)} e^{i \omega (\theta (t) - t)}$.
\end{proof}

\begin{theorem}
  [Evolutionary Power Spectrum] For the oscillatory process $X_t =
  \sqrt{\theta' (t)} S_{\theta (t)}$ with envelope $A_t (\omega) =
  \sqrt{\theta' (t)} e^{i \omega (\theta (t) - t)}$, the evolutionary power
  spectrum at time $t$ is
  \[ dF_t (\omega) = |A_t (\omega) |^2 d \mu (\omega) = \theta' (t) d \mu
     (\omega) . \]
\end{theorem}

\begin{proof}
  The evolutionary power spectrum is defined as $dF_t (\omega) = |A_t (\omega)
  |^2 d \mu (\omega)$. Computing the magnitude squared:
  
  \begin{align}
    |A_t (\omega) |^2 & = \left| \sqrt{\theta' (t)} e^{i \omega (\theta (t) -
    t)} \right|^2 \\
    & = \theta' (t) | e^{i \omega (\theta (t) - t)} |^2 \\
    & = \theta' (t) \cdot 1 \\
    & = \theta' (t) 
  \end{align}
  
  Therefore $dF_t (\omega) = \theta' (t) d \mu (\omega)$.
\end{proof}

\begin{theorem}
  [L{\texttwosuperior}-Norm Preservation] The transformation $S_t \mapsto
  \sqrt{\theta' (t)} S_{\theta (t)}$ preserves the L{\texttwosuperior}-norm in
  the sense that
  \[ \int_I E |X_t |^2 dt = \int_J E |S_s |^2 ds \]
  where $I$ is the domain of $t$ and $J = \theta (I)$.
\end{theorem}

\begin{proof}
  Using the change of variables $s = \theta (t)$, so $ds = \theta' (t) dt$:
  
  \begin{align}
    \int_I E |X_t |^2 dt & = \int_I E \left| \sqrt{\theta' (t)} S_{\theta (t)}
    \right|^2 dt \\
    & = \int_I \theta' (t) E |S_{\theta (t)} |^2 dt \\
    & = \int_J E |S_s |^2 ds 
  \end{align}
\end{proof}

\end{document}
