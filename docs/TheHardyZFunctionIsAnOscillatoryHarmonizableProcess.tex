\documentclass{article}
\usepackage{amsmath,amsthm,amssymb,amsfonts}
\usepackage{enumitem}
\usepackage[margin=0.5in]{geometry}

\newtheorem{theorem}{Theorem}
\newtheorem{lemma}{Lemma}
\newtheorem{definition}{Definition}
\newtheorem{remark}{Remark}

\title{The Hardy Z-Function as a Deterministic Oscillatory Weakly Harmonizable Process}
\author{}
\date{}

\renewcommand{\Re}{\operatorname{Re}}

\begin{document}
\maketitle

\begin{abstract}
The Hardy Z-function admits an exact deterministic oscillatory weakly harmonizable representation. The spectral bimeasure has finite Fréchet variation and infinite Vitali variation, and the Morse-Transue integral reduces to a conditionally convergent sum.
\end{abstract}

\section{Standard Definitions of Harmonizable Processes}

\subsection{Harmonizable Processes and Spectral Bimeasures}

\begin{definition}[Harmonizable Process \cite{rao1982harmonizable,chang1986bimeasures}]
A stochastic process $\{X(t):t\in\mathbb{R}\}$ is harmonizable if there exists a complex-valued vector measure $\Phi:\mathcal{B}(\mathbb{R})\to\mathcal{H}$ (taking values in a Hilbert space $\mathcal{H}$) such that
\[
X(t) = \int_{\mathbb{R}} e^{i\lambda t}\,d\Phi(\lambda).
\]
The spectral bimeasure is defined by $F(A,B) = \langle\Phi(A),\Phi(B)\rangle_{\mathcal{H}}$ for Borel sets $A,B\subset\mathbb{R}$.
\end{definition}

\subsection{Oscillatory Harmonizable Processes}

\begin{definition}[Oscillatory Harmonizable Process \cite{priestley1981spectral,rao1982harmonizable}]
An oscillatory harmonizable process is a process of the form
\[
X(t) = \int_{\mathbb{R}} A_t(\lambda) e^{i\lambda t}\,d\Phi(\lambda),
\]
where $\Phi$ is as in Definition 2.1 and $A_t(\lambda)$ is a deterministic time-varying gain function satisfying $A_t(\cdot)\in L^2(F)$ for each $t$.
\end{definition}

\subsection{Weak vs. Strong Harmonizability}

\begin{definition}[Variation Types \cite{morse1956integral,chang1986bimeasures}]
For a bimeasure $F$:
\begin{itemize}
    \item The Vitali variation is $|F|_{\text{Vitali}} = \sup\sum_{i,j} |F(A_i,A_j)|$ over finite partitions.
    \item The Fréchet variation is $|F|_{\text{Fréchet}} = \sup\left|\sum_{i,j} a_i\bar{a}_j F(A_i,A_j)\right|$ for $|a_i|\le 1$.
\end{itemize}
A process is strongly harmonizable if $|F|_{\text{Vitali}}<\infty$ and weakly harmonizable if $|F|_{\text{Fréchet}}<\infty$.
\end{definition}

\subsection{Morse-Transue Integral}

\begin{definition}[Morse-Transue Integral \cite{morse1956integral,chang1986bimeasures}]
Let $F$ be a bimeasure. For functions $f,g:\mathbb{R}\to\mathbb{C}$, the MT integral exists if there exist simple functions $f_n\to f$, $g_n\to g$ such that the iterated integrals converge:
\[
\lim_{n\to\infty} \int\left|\int f_n(\lambda)F(d\lambda,\cdot) - \int f(\lambda)F(d\lambda,\cdot)\right| = 0,
\]
with a symmetric condition for $g$. The integral is $\lim_{n\to\infty} \iint f_n(\lambda)g_n(\mu)F(d\lambda,d\mu)$.
\end{definition}

For discrete bimeasures $F(A,B) = \sum_{(m,n)\in I(A)\times I(B)} w_{mn}$, the MT integral reduces to
\[
\iint f(\lambda)g(\mu)F(d\lambda,d\mu) = \sum_{m,n=1}^\infty w_{mn} f(\lambda_m) g(\mu_n),
\]
provided the series converges conditionally via Dirichlet's test.

\section{Construction of the Hardy Z Representation}

\subsection{Spectral Gain and Spectral Measure}

Define the time-varying gain function $A_t:\mathbb{R}\to\mathbb{C}$ as a spectral Shah (Dirac comb) function:
\[
A_t(\lambda) = c_{+}(t)\sum_{n=1}^\infty \delta(\lambda+\log n) + c_{-}(t)\sum_{n=1}^\infty \delta(\lambda-\log n),
\]
with scalar gains
\[
c_{+}(t) = \frac{e^{i\theta(t)}}{2(1-2^{1/2-it})}, \qquad c_{-}(t) = \frac{e^{-i\theta(t)}}{2(1-2^{1/2+it})}.
\]

Define the deterministic orthogonal spectral measure $\Phi:\mathcal{B}(\mathbb{R})\to\ell^2(\mathbb{N})$ by
\[
\Phi(E) = \sum_{n=1}^\infty \frac{(-1)^{n-1}}{\sqrt{n}}\bigl[\mathbf{1}_E(\log n) + \mathbf{1}_E(-\log n)\bigr]\,\mathbf{e}_n,
\]
where $\{\mathbf{e}_n\}_{n\in\mathbb{N}}$ is the standard orthonormal basis of $\ell^2(\mathbb{N})$.

\subsection{Spectral Bimeasure}

The induced spectral bimeasure is
\[
F(A,B) = \langle\Phi(A),\Phi(B)\rangle_{\ell^2} = \sum_{m,n=1}^\infty \frac{(-1)^{m+n}}{\sqrt{mn}}\bigl[\mathbf{1}_A(\log m)\mathbf{1}_B(\log n) + \mathbf{1}_A(-\log m)\mathbf{1}_B(-\log n)\bigr].
\]

\section{Main Theorem and Proof}

\begin{lemma}[Hardy-Z Identity]\label{lemma:hardy-z}
For all $t\in\mathbb{R}$,
\[
Z(t) = \Re\!\left(e^{i\theta(t)}\zeta(1/2+it)\right) = \frac{e^{i\theta(t)}\eta(1/2+it)}{2(1-2^{1/2-it})} + \frac{e^{-i\theta(t)}\eta(1/2-it)}{2(1-2^{1/2+it})}.
\]
\end{lemma}

\begin{theorem}[Exact Oscillatory Weakly Harmonizable Representation]\label{thm:main}
For all real $t$, define the $\ell^2(\mathbb N)$-valued process
\[
Y(t):=\int_{\mathbb{R}} A_t(\lambda) e^{i\lambda t}\,d\Phi(\lambda),
\]
and define the associated scalar process
\[
X(t):=\sum_{n=1}^\infty \frac{(-1)^{n-1}}{\sqrt{n}}
\Bigl[c_{-}(t)n^{it} + c_{+}(t)n^{-it}\Bigr].
\]
Then $X(t)=Z(t)$ for all real $t$.
\end{theorem}

\begin{proof}
By the atomic structure of $\Phi$ and the sifting property of the Dirac delta, we evaluate the integral directly:
\begin{align*}
\int_{\mathbb{R}} A_t(\lambda)e^{i\lambda t}\,d\Phi(\lambda)
&= \int_{\mathbb{R}} \Bigl[c_{+}(t)\sum_{m=1}^\infty \delta(\lambda+\log m) + c_{-}(t)\sum_{m=1}^\infty \delta(\lambda-\log m)\Bigr] e^{i\lambda t}\,d\Phi(\lambda) \\
&= \sum_{n=1}^\infty \frac{(-1)^{n-1}}{\sqrt{n}}\Bigl[c_{+}(t)e^{-it\log n} + c_{-}(t)e^{it\log n}\Bigr]\mathbf{e}_n \\
&= c_{-}(t)\sum_{n=1}^\infty \frac{(-1)^{n-1}}{\sqrt{n}}n^{it}\mathbf{e}_n + c_{+}(t)\sum_{n=1}^\infty \frac{(-1)^{n-1}}{\sqrt{n}}n^{-it}\mathbf{e}_n.
\end{align*}

Therefore
\[
Y(t)= \sum_{n=1}^\infty \frac{(-1)^{n-1}}{\sqrt{n}}
\Bigl[c_{-}(t)n^{it} + c_{+}(t)n^{-it}\Bigr]\mathbf{e}_n.
\]

Define the scalar process $X(t)$ as the conditionally convergent Dirichlet series:
\[
X(t):=\sum_{n=1}^\infty \frac{(-1)^{n-1}}{\sqrt{n}}
\Bigl[c_{-}(t)n^{it} + c_{+}(t)n^{-it}\Bigr]
= c_{-}(t)\eta(1/2-it) + c_{+}(t)\eta(1/2+it),
\]
where convergence follows from Dirichlet's test and the value agrees with the defining Dirichlet series $\eta(s)=\sum_{n\ge1}(-1)^{n-1}n^{-s}$ at $s=1/2\mp it$.

By Lemma \ref{lemma:hardy-z}, $X(t)=Z(t)$ for all real $t$.
\end{proof}

\section{Variation and Regularity Analysis}

\subsection{Variation Analysis}

\begin{lemma}
For the spectral bimeasure $F$, we have $|F|_{\text{Vitali}} = \infty$ and $|F|_{\text{Fréchet}} < \infty$.
\end{lemma}

\begin{proof}
Vitali variation: Consider the partition $A_i = \{\log i\}$, $B_j = \{\log j\}$. Since $|F(A_i,B_j)| = \frac{1}{\sqrt{ij}}$, we have
\[
|F|_{\text{Vitali}} \ge \sum_{i,j=1}^\infty \frac{1}{\sqrt{ij}} = \left(\sum_{n=1}^\infty n^{-1/2}\right)^2 = \infty.
\]

Fréchet variation: For any sequence $|a_n|\le 1$, let $S_N = \sum_{k=1}^N (-1)^{k-1}a_k$. By Abel summation,
\[
\sum_{n=1}^N \frac{(-1)^{n-1}a_n}{\sqrt{n}} = \frac{S_N}{\sqrt{N}} + \sum_{n=1}^{N-1} S_n\bigl(n^{-1/2} - (n+1)^{-1/2}\bigr).
\]
Since $|S_n|\le 1$ and $\sum_{n=1}^\infty (n^{-1/2} - (n+1)^{-1/2}) = 1$, we have
\[
\sup_{\{a_n\}} \left|\sum_{n=1}^\infty \frac{(-1)^{n-1}a_n}{\sqrt{n}}\right|^2 \le 4 < \infty.
\]
Thus \(|F|_{\text{Fréchet}} < \infty\).
\end{proof}

\subsection{Gain Regularity}

\begin{lemma}
For each $t$, the gain $A_t$ satisfies $A_t \in L^2(F)$ with
\[
\|A_t\|_{L^2(F)}^2 = (|c_+(t)|^2 + |c_-(t)|^2)\sum_{n=1}^\infty \frac{(-1)^{n-1}}{n} < \infty.
\]
\end{lemma}

\begin{proof}
The MT-integral defining \(\|A_t\|_{L^2(F)}^2 = \int |A_t(\lambda)|^2 F(d\lambda,d\lambda)\) reduces to:
\[
\int |A_t(\lambda)|^2 F(d\lambda,d\lambda) = (|c_+(t)|^2 + |c_-(t)|^2)\sum_{n=1}^\infty \frac{(-1)^{n-1}}{n}.
\]
Since $c_{\pm}(t)$ are uniformly bounded and the alternating harmonic series converges, \(\|A_t\|_{L^2(F)}^2 < \infty\).
\end{proof}

\section{Conclusion}

The Hardy Z-function is exactly represented as a deterministic oscillatory weakly harmonizable process. The spectral bimeasure has bounded Fréchet variation and infinite Vitali variation, and the Morse-Transue integral reduces to a conditionally convergent Dirichlet series. This construction uses a Shah gain in the spectral variable and an atomic spectral measure, providing a mathematically rigorous representation.

\begin{thebibliography}{9}
\bibitem{rao1982harmonizable} M. M. Rao, \textit{Harmonizable Processes and Spectral Theory}, in Probability and Mathematical Statistics, Academic Press, 1982.
\bibitem{chang1986bimeasures} D. K. Chang, M. M. Rao, \textit{Bimeasures and Nonstationary Processes}, Lecture Notes in Math. 860, Springer, 1986.
\bibitem{morse1956integral} A. P. Morse, W. Transue, \textit{The $F\!\int$ Integral}, Ann. of Math. 64 (1956), 153--160.
\bibitem{priestley1981spectral} M. B. Priestley, \textit{Spectral Analysis of Time Series}, Academic Press, 1981.
\bibitem{hardy1916z} G. H. Hardy, J. E. Littlewood, \textit{Contributions to the Theory of the Riemann Zeta-Function}, Acta Math. 41 (1916), 119--196.
\end{thebibliography}

\end{document}
