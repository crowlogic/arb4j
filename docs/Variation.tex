\documentclass[12pt]{article}

\usepackage{amsmath}
\usepackage{amsthm}
\usepackage{amsfonts}
\usepackage{amssymb}

\title{Vitali and Fr\'echet Variations}
\author{}
\date{\today}

\begin{document}

\maketitle

\section{Vitali Variation}
\subsection{Definition}
Vitali variation is a fundamental concept in mathematical analysis, particularly in the study of functions of several variables. For a function $f: I \to \mathbb{R}$, where $I$ is a rectangle in $\mathbb{R}^n$, the Vitali variation $V(f, I)$ is given by:

\[V(f,I)=\sup_P \sum_{J\in P}|f(a_J)-f(b_J)|\]

where $P$ ranges over all partitions of $I$ into subrectangles $J$, and $a_J$ and $b_J$ are opposite vertices of $J$.

Functions with finite Vitali variation possess several important properties:
\begin{enumerate}
    \item They are bounded and continuous almost everywhere.
    \item They can be expressed as the difference of two functions with nonnegative sums.
    \item They have well-defined Riemann-Stieltjes integrals.
\end{enumerate}

\section{Fr\'echet Variation}
\subsection{Definition}
A bimeasure $F$ on a product space $\Omega_1 \times \Omega_2$ has finite Fr\'echet variation if:

\[V_F(\Omega_1,\Omega_2)=\sup_{\Pi_1,\Pi_2}\sum_{A\in \Pi_1}\sum_{B\in \Pi_2}|F(A,B)|\]

where $\Pi_1$ and $\Pi_2$ are finite partitions of $\Omega_1$ and $\Omega_2$ respectively.

\section{Mathematical Implications}

\subsection{Strongly Harmonizable Processes}
For strongly harmonizable processes, the correlation function is:

\[R(s,t)=\int_{\mathbb{R}}e^{i(s-t)\lambda}dF(\lambda)\]

where $F$ is a complex-valued measure with finite Vitali variation.

\subsection{Weakly Harmonizable Processes}
For weakly harmonizable processes, the correlation function is:

\[R(s,t)=\int_{\mathbb{R}^2}e^{i(s\lambda_1 -t\lambda_2)}dF(\lambda_1,\lambda_2)\]

where $F$ is a bimeasure with finite Fr\'echet variation.

\subsection{Key Implications}
\begin{enumerate}
    \item Spectral measures: Strongly harmonizable processes have countably additive spectral measures.
    \item Stochastic integration: More developed theory for strongly harmonizable processes.
    \item Boundedness properties: Strongly harmonizable processes are bounded in probability.
    \item Representation theory: Weakly harmonizable processes can be represented by positive definite contractive linear operators in a Hilbert space.
    \item Continuity and differentiability: Strongly harmonizable processes have stronger continuity properties.
\end{enumerate}

\end{document}

