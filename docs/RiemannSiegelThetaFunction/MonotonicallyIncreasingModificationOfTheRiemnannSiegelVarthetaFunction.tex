\documentclass{article}
\usepackage[english]{babel}
\usepackage{geometry,amsmath,latexsym}
\geometry{letterpaper}

%%%%%%%%%% Start TeXmacs macros
\catcode`\<=\active \def<{
\fontencoding{T1}\selectfont\symbol{60}\fontencoding{\encodingdefault}}
\catcode`\>=\active \def>{
\fontencoding{T1}\selectfont\symbol{62}\fontencoding{\encodingdefault}}
\newcommand{\tmaffiliation}[1]{\\ #1}
\newcommand{\tmop}[1]{\ensuremath{\operatorname{#1}}}
\newenvironment{proof}{\noindent\textbf{Proof\ }}{\hspace*{\fill}$\Box$\medskip}
\newtheorem{corollary}{Corollary}
\newtheorem{definition}{Definition}
\newtheorem{proposition}{Proposition}
\newtheorem{theorem}{Theorem}
%%%%%%%%%% End TeXmacs macros

%


\begin{document}

\title{A Bijective Modification of the Riemann-Siegel $\tmop{Theta}$ Function}

\author{
  Stephen Crowley <Stephencrowley214@gmail.com>
  \tmaffiliation{August 9, 2025}
}

\maketitle

\begin{abstract}
  A monotonically increasing version $\vartheta^+ (t)$ of the Riemann--Siegel
  theta function $\vartheta (t)$ is constructed by modifying through
  reflection about its unique nonzero critical point. {\cdot}
\end{abstract}

{\tableofcontents}

\section{The Riemann--Siegel Theta Function}

\begin{definition}
  [Riemann--Siegel Theta Function]\label{def:theta} The Riemann--Siegel theta
  function is defined as:
  \begin{equation}
    \vartheta (t) = \arg \Gamma \hspace{-0.17em} \left( \frac{1}{4} +
    \frac{it}{2} \right) - \frac{t}{2} \log \pi
  \end{equation}
  where $\Gamma$ is the gamma function and arg denotes the principal argument,
  taken continuously along the path.
\end{definition}

\begin{definition}
  [Digamma and Trigamma Functions]\label{def:psi} The digamma function
  $\psi^{(0)} (z)$ and trigamma function $\psi^{(1)} (z)$ are defined by:
  
  \begin{align}
    \psi^{(0)} (z) & = \frac{d}{dz} \log \Gamma (z) = \frac{\Gamma'
    (z)}{\Gamma (z)} \quad \text{{\cite{DLMF5.2.1}}} \\
    \psi^{(1)} (z) & = \frac{d}{dz} \psi^{(0)} (z) = \sum_{n = 0}^{\infty}
    \frac{1}{(n + z)^2} \quad \text{{\cite{DLMF5.4.2}}} 
  \end{align}
  
  for $\Re (z) > 0$.
\end{definition}

\begin{proposition}
  [Derivative Properties]\label{prop:derivative} The derivative of the
  Riemann--Siegel theta function is:
  \begin{equation}
    \vartheta' (t) = \frac{1}{2} \Im \hspace{-0.17em} \left[ \psi^{(0)} 
    \hspace{-0.17em} \left( \frac{1}{4} + \frac{it}{2} \right) \right] -
    \frac{\log \pi}{2}
  \end{equation}
\end{proposition}

\begin{proof}
  Let $w (t) = \frac{1}{4} + \frac{it}{2}$. Along the curve $t \mapsto w (t)$
  the principal argument of $\Gamma$ can be chosen continuously, so
  \begin{equation}
    \frac{d}{dt} \arg \Gamma (w (t)) = \Im \hspace{-0.17em} \left(
    \frac{d}{dt} \log \Gamma (w (t)) \right) = \Im \hspace{-0.17em} \left(
    \psi^{(0)} (w (t)) \hspace{0.17em} w' (t) \right)
  \end{equation}
  Since $w' (t) = i / 2$, this derivative equals $\frac{1}{2} \Im \psi^{(0)}
  (w (t))$. Differentiating $- \frac{t}{2} \log \pi$ gives $- \frac{\log
  \pi}{2}$.
\end{proof}

\begin{theorem}
  [Limit at the Origin]\label{thm:limit}
  \begin{equation}
    \lim_{t \to 0^+} \Im \hspace{-0.17em} \left[ \psi^{(0)}  \hspace{-0.17em}
    \left( \frac{1}{4} + \frac{it}{2} \right) \right] = 0
  \end{equation}
\end{theorem}

\begin{proof}
  Using the integral representation {\cite{AbramowitzStegun6.3.1}} and
  dominated convergence (or by analyticity and Taylor expansion in $t$), the
  imaginary part vanishes as $t \to 0^+$.
\end{proof}

\begin{theorem}
  [Monotonicity of the Digamma Imaginary Part]\label{thm:monotonicity} For
  fixed $\sigma > 0$, the function $t \mapsto \Im [\psi^{(0)} (\sigma + it)]$
  is strictly increasing for $t > 0$.
\end{theorem}

\begin{proof}
  Differentiating with respect to $t$ gives
  \begin{equation}
    \frac{\partial}{\partial t} \Im [\psi^{(0)} (\sigma + it)] = \Re
    \hspace{-0.17em} \left[ \frac{\partial}{\partial t} \psi^{(0)} (\sigma +
    it) \right] = \Re \hspace{-0.17em} \left[ i \hspace{0.17em} \psi^{(1)}
    (\sigma + it) \right] = - \hspace{0.17em} \Im \hspace{-0.17em} [\psi^{(1)}
    (\sigma + it)]
  \end{equation}
  Using the absolutely convergent series {\cite{DLMF5.4.2}}
  \begin{equation}
    \psi^{(1)} (z) = \sum_{n = 0}^{\infty} \frac{1}{(n + z)^2}, \quad \Re z >
    0
  \end{equation}
  and setting $z = \sigma + it$, we have
  \begin{equation}
    \frac{1}{(n + \sigma + it)^2} = \frac{(n + \sigma)^2 - t^2 - 2 i (n +
    \sigma) t}{((n + \sigma)^2 + t^2)^2}
  \end{equation}
  Thus
  \begin{equation}
    \Im \hspace{-0.17em} [\psi^{(1)} (\sigma + it)] = \sum_{n = 0}^{\infty}
    \frac{- \hspace{0.17em} 2 (n + \sigma) t}{((n + \sigma)^2 + t^2)^2} < 0
  \end{equation}
  for $\sigma > 0$, $t > 0$. Hence $\frac{\partial}{\partial t} \Im
  [\psi^{(0)} (\sigma + it)] > 0$.
\end{proof}

\begin{theorem}
  [Asymptotic Limit]\label{thm:growth}
  \begin{equation}
    \lim_{t \to \infty} \Im \hspace{-0.17em} \left[ \psi^{(0)} 
    \hspace{-0.17em} \left( \frac{1}{4} + \frac{it}{2} \right) \right] =
    \frac{\pi}{2}
  \end{equation}
\end{theorem}

\begin{proof}
  From the asymptotic expansion {\cite{DLMF5.11.1}}, $\psi^{(0)} (z) = \log z
  - \frac{1}{2 z} + O (|z|^{- 2})$ as $|z| \to \infty$ with $| \arg z| < \pi$.
  Writing $z = \frac{1}{4} + \frac{it}{2}$, we have $\Im \log z = \arg z \to
  \frac{\pi}{2}$, and $\Im (- 1 / (2 z)) = O (1 / t)$, so the limit is $\pi /
  2$.
\end{proof}

\begin{theorem}
  [Unique Critical Point]\label{thm:critical} There exists a unique $a > 0$
  such that $\vartheta' (a) = 0$, equivalently
  \begin{equation}
    \Im \hspace{-0.17em} \left[ \psi^{(0)}  \hspace{-0.17em} \left(
    \frac{1}{4} + \frac{ia}{2} \right) \right] = \log \pi
  \end{equation}
  Moreover:
  \begin{itemize}
    \item $\vartheta' (t) < 0$ for $t \in (0, a)$
    
    \item $\vartheta' (t) = 0$ at $t = a$
    
    \item $\vartheta' (t) > 0$ for $t > a$
  \end{itemize}
\end{theorem}

\begin{proof}
  By Theorems \ref{thm:limit}, \ref{thm:monotonicity}, and \ref{thm:growth},
  the function $t \mapsto \Im [\psi^{(0)} (1 / 4 + it / 2)]$ is continuous,
  strictly increasing from $0$ to $\pi / 2$, and therefore attains the value
  $\log \pi$ at a unique $a > 0$. The sign changes for $\vartheta' (t)$ follow
  from Proposition \ref{prop:derivative}.
\end{proof}

\section{Monotonization Construction}

\begin{definition}
  [Monotonized Theta Function]\label{def:monotonized} Define the monotonized
  Riemann--Siegel theta function
  \begin{equation}
    \vartheta^+ (t) = \left\{\begin{array}{ll}
      2 \vartheta (a) - \vartheta (t) & t \in [0, a]\\
      \vartheta (t) & t > a
    \end{array}\right.
  \end{equation}
  where $a$ is the unique critical point from Theorem \ref{thm:critical}.
\end{definition}

\begin{theorem}
  [Monotonicity of $\vartheta^+$]\label{thm:mono_construction} The function
  $\vartheta^+ (t)$ is nondecreasing on $[0, \infty)$ and strictly increasing
  on $[0, \infty) \setminus \{a\}$:
  \begin{equation}
    \frac{d}{dt} \vartheta^+ (t) = \left\{\begin{array}{ll}
      - \vartheta' (t) > 0, & t \in (0, a)\\
      0, & t = a\\
      \vartheta' (t) > 0 & t > a
    \end{array}\right.
  \end{equation}
\end{theorem}

\begin{proof}
  Immediate from the definition and the sign of $\vartheta' (t)$.
\end{proof}

\begin{proposition}
  [Continuity and Differentiability]\label{prop:continuity} The function
  $\vartheta^+ (t)$ is continuous and differentiable everywhere, including at
  $t = a$.
\end{proposition}

\begin{proof}
  Continuity at $a$:
  \begin{equation}
    \lim_{t \to a^-} \vartheta^+ (t) = 2 \vartheta (a) - \vartheta (a) =
    \vartheta (a) = \lim_{t \to a^+} \vartheta^+ (t) = \vartheta^+ (a)
  \end{equation}
  Differentiability at $a$:
  \[ \lim_{t \to a^-}  \frac{d}{dt} \vartheta^+ (t) = - \vartheta' (a) = 0 =
     \lim_{t \to a^+}  \frac{d}{dt} \vartheta^+ (t) \]
\end{proof}

\section{Phase Information Preservation}

\begin{definition}
  [Phase Representation]\label{def:phase} On the critical line,
  \begin{equation}
    \zeta \hspace{-0.17em} \left( \frac{1}{2} + it \right) = e^{- i \vartheta
    (t)}  \hspace{0.17em} Z (t)
  \end{equation}
  where $Z (t)$ is real-valued (the Hardy $Z$-function).
\end{definition}

\begin{theorem}
  [Phase Preservation]\label{thm:phase_preservation} Define
  \begin{equation}
    \tilde{Z} (t) = e^{i \vartheta^+ (t)}  \hspace{0.17em} \zeta
    \hspace{-0.17em} \left( \frac{1}{2} + it \right)
  \end{equation}
  Then
  \begin{equation}
    \tilde{Z} (t) = \left\{\begin{array}{ll}
      e^{2 i \vartheta (a)}  \hspace{0.17em} Z (t), & t \in [0, a],\\
      Z (t), & t > a.
    \end{array}\right.
  \end{equation}
\end{theorem}

\begin{proof}
  For $t > a$, $\vartheta^+ (t) = \vartheta (t)$, so $\tilde{Z} (t) = Z (t)$.
  For $t \in [0, a]$,
  \[ \tilde{Z} (t) = e^{i (2 \vartheta (a) - \vartheta (t))} \zeta
     \hspace{-0.17em} \left( \tfrac{1}{2} + it \right) = e^{2 i \vartheta (a)}
     \hspace{0.17em} e^{- i \vartheta (t)} \zeta \hspace{-0.17em} \left(
     \tfrac{1}{2} + it \right) = e^{2 i \vartheta (a)}  \hspace{0.17em} Z (t)
  \]
\end{proof}

\begin{corollary}
  [Zero Preservation]\label{cor:zeros} The zeros of $\zeta \hspace{-0.17em}
  \left( \frac{1}{2} + it \right)$ correspond exactly to the zeros of both $Z
  (t)$ and $\tilde{Z} (t)$ for $t > 0$.
\end{corollary}

\begin{proof}
  Multiplication by nonzero phase factors preserves zeros.
\end{proof}

\begin{proposition}
  [Bijectivity]\label{prop:bijective} The function $\vartheta^+ (t) : [0,
  \infty) \to [\vartheta^+ (0), \infty)$ is bijective.
\end{proposition}

\begin{proof}
  Injectivity: By Theorem \ref{thm:mono_construction}, $\vartheta^+$ is
  strictly increasing except at $t = a$ where the derivative is $0$ but the
  function increases through $a$; hence injective. Surjectivity: $\vartheta^+$
  is continuous and increases without bound as $t \to \infty$ (matching
  $\vartheta$ asymptotically), so by the Intermediate Value Theorem its range
  is $[\vartheta^+ (0), \infty)$.
\end{proof}

\begin{theorem}
  [Modulating Function Criteria]\label{thm:modulating} The function
  $\vartheta^+ (t)$ satisfies:
  \begin{enumerate}
    \item Piecewise $C^1$ with piecewise continuous first derivative and
    matching at $t = a$,
    
    \item Monotonically nondecreasing with $\frac{d}{dt} \vartheta^+ (t) \ge
    0$ and equality only at $t = a$,
    
    \item Bijective with $\lim_{t \to \infty} \vartheta^+ (t) = \infty$.
  \end{enumerate}
\end{theorem}

\begin{proof}
  Immediate from Propositions \ref{prop:continuity}, \ref{prop:bijective} and
  Theorem \ref{thm:mono_construction}.
\end{proof}

\section{Conclusion}

The monotonized Riemann--Siegel theta function $\vartheta^+ (t)$ constructed
through geometric reflection about its unique critical point provides a
bijective, monotonically increasing transformation that preserves all
essential phase information of the original theta function. This construction
maintains exact correspondence with zeros of the Riemann zeta function while
enabling applications requiring monotonic phase functions.

\begin{thebibliography}{99}
  {\bibitem{DLMF5.2.1}}NIST Digital Library of Mathematical Functions. Section
  5.2.1: Definitions.
  
  {\bibitem{DLMF5.4.2}}NIST Digital Library of Mathematical Functions. Section
  5.4.2: Series Representations.
  
  {\bibitem{DLMF5.11.1}}NIST Digital Library of Mathematical Functions.
  Section 5.11.1: Asymptotic Expansions.
  
  {\bibitem{AbramowitzStegun6.3.1}}Abramowitz, M. and Stegun, I. A. (1964).
  Handbook of Mathematical Functions. National Bureau of Standards. Chapter 6,
  Equation 6.3.1.
  
  {\bibitem{Elezovic1999}}Elezovi{\'c}, N., Giordano, C., and Pe{\v
  c}ari{\'c}, J. (1999). The best bounds in Gautschi's inequality.
  Mathematical Inequalities \& Applications, 2(2), 239--252.
\end{thebibliography}

\

\end{document}
