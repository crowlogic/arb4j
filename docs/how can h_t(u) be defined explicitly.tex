\documentclass{article}
\usepackage[english]{babel}
\usepackage{amsmath}

%%%%%%%%%% Start TeXmacs macros
\newcommand{\tmem}[1]{{\em #1\/}}
\newcommand{\tmtextbf}[1]{\text{{\bfseries{#1}}}}
\newenvironment{quoteenv}{\begin{quote} }{\end{quote}}
%%%%%%%%%% End TeXmacs macros

\providecommand{\tightlist}{}

\begin{document}

\date{}

\maketitle

\section{how can $h_t (u)$ be defined
explicitly}\label{how-can-h_tu-be-defined-explicitly}\label{how-can-h}

Sure! Let's clarify how$h_t (u)$can be \tmtextbf{explicitly defined} in the
context of M.B. Priestley's {\tmem{Evolutionary Spectra and Non-Stationary
Processes}}. This is discussed around the passage that begins:

\begin{quoteenv}
  There is an interesting alternative interpretation of oscillatory processes
  in terms of time-varying filters. ...
  
  \
  
  Let
  \[ A_t (\omega) = \int_{- \infty}^{\infty} e^{i \omega u} h_t (u) du \]
\end{quoteenv}

This is equation (filter\_fourier) in your document.

\subsection{Explicit Definition of$h_t
(u)$}\label{explicit-definition-ofh_tu}\label{explicit-definition-ofh}

\subsubsection{1. Fourier
Relationship}\label{fourier-relationship}\label{fourier-relationship}

$h_t (u)$is related to$A_t (\omega)$by the \tmtextbf{inverse Fourier
transform}:
\[ A_t (\omega) = \int_{- \infty}^{\infty} e^{i \omega u} h_t (u) du \]
Therefore, for each fixed$t$, $h_t (u)$is the inverse Fourier transform of$A_t
(\omega)$:
\[ h_t (u) = \frac{1}{2 \pi}  \int_{- \infty}^{\infty} A_t (\omega) e^{- i
   \omega u} d \omega \]
\subsubsection{2. Interpretation}\label{interpretation}\label{interpretation}
\begin{itemize}
  \item For each time$t$, $h_t (u)$is a function of ``lag''$u$.
  
  \item It represents the \tmtextbf{impulse response} of a time-varying (i.e.,
  non-stationary) filter at time$t$. -$X_t$is constructed by convolving this
  filter with a stationary process$S_{t - u}$:
\end{itemize}
\[ X_t = \int_{- \infty}^{\infty} S_{t - u} h_t (u) du \]
where$S_t$is stationary with spectrum$d \mu (\omega)$.

\subsubsection{3. Summary}\label{summary}\label{summary}
\begin{itemize}
  \item \tmtextbf{Given$A_t (\omega)$, the explicit definition is:}
\end{itemize}
\[ h_t (u) = \frac{1}{2 \pi}  \int_{- \infty}^{\infty} A_t (\omega) e^{- i
   \omega u} d \omega \]
- \tmtextbf{Practical Construction:} If you know or specify$A_t (\omega)$(the
amplitude modulation function in frequency for time$t$), then you can always
obtain the corresponding time-varying filter$h_t (u)$via its inverse Fourier
transform.

\subsection{In Context: Why is this
Useful?}\label{in-context-why-is-this-useful}\label{in-context-why-is-this-useful}
\begin{itemize}
  \item This formulation allows you to interpret a general non-stationary
  (oscillatory) process as a filtered version of a stationary process, where
  the filter itself can change over time.
  
  \item The evolutionary spectrum at time$t$, $|A_t (\omega) |^2 d \mu
  (\omega)$, is then the spectrum of the process you would get if you
  ``froze'' the filter at time$t$.
\end{itemize}
\subsection{In Summary}\label{in-summary}\label{in-summary}

\tmtextbf{$h_t (u)$is explicitly defined as the inverse Fourier transform
of$A_t (\omega)$:}
\[ h_t (u) = \frac{1}{2 \pi}  \int_{- \infty}^{\infty} A_t (\omega) e^{- i
   \omega u} d \omega \]
This allows you to construct the time-varying filter corresponding to the
amplitude modulation function$A_t (\omega)$used in the representation of your
non-stationary process.

If you need further clarification or an example with a specific$A_t (\omega)$,
let me know!

⁂

\end{document}
