\documentclass{article}
\usepackage[english]{babel}
\usepackage{geometry,amsmath,amssymb,enumerate,bbm,latexsym}
\geometry{letterpaper}

%%%%%%%%%% Start TeXmacs macros
\newcommand{\assign}{:=}
\newcommand{\cdummy}{\cdot}
\newcommand{\nin}{\not\in}
\newcommand{\tmem}[1]{{\em #1\/}}
\newcommand{\tmop}[1]{\ensuremath{\operatorname{#1}}}
\newcommand{\tmtextbf}[1]{\text{{\bfseries{#1}}}}
\newenvironment{enumeratealpha}{\begin{enumerate}[a{\textup{)}}] }{\end{enumerate}}
\newenvironment{proof}{\noindent\textbf{Proof\ }}{\hspace*{\fill}$\Box$\medskip}
\newtheorem{corollary}{Corollary}
\newtheorem{definition}{Definition}
\newtheorem{lemma}{Lemma}
\newtheorem{proposition}{Proposition}
\newtheorem{theorem}{Theorem}
%%%%%%%%%% End TeXmacs macros

\newcommand{\C}{\mathbb{C}}
\newcommand{\R}{\mathbb{R}}
\newcommand{\Z}{\ensuremath{\mathbb{Z}}}
\newcommand{\N}{\ensuremath{\mathbb{N}}}
\newcommand{\E}{\mathbb{E}}
\newcommand{\Prob}{\mathbb{P}}
\newcommand{\1}{\mathbbm{1}}
\newcommand{\ip}[2]{\left\langle #1, \hspace{0.17em} #2  \right\rangle}
\newcommand{\norm}[1]{\left\lVert #1  \right\rVert}
\newcommand{\supp}{\tmop{supp}}
\newcommand{\sgn}{sgn}
\newcommand{\Var}{Var}
\newcommand{\Cov}{Cov}
\newcommand{\Dom}{\tmop{Dom}}
\newcommand{\Ran}{\tmop{Ran}}
\newcommand{\re}{\tmop{Re}}
\newcommand{\im}{Im}
%


\begin{document}

\title{
  Advanced Analysis of Stone's Theorem and Spectral Representations in
  Stochastic Process Theory:\\
  A Comprehensive Treatment with Detailed Proofs
}

\date{}

\maketitle

\begin{abstract}
  This article develops a rigorous bridge between functional analysis and the
  theory of weakly stationary stochastic processes via spectral theory. We
  present complete, detailed proofs of the spectral theorem for self-adjoint
  operators and its projection-valued measure framework, establish Stone's
  theorem linking strongly continuous unitary groups to self-adjoint
  generators, and derive spectral representations of weakly stationary
  processes through orthogonal random measures. All proofs are given in full
  detail without reference to ``standard arguments'' or omitted steps.
\end{abstract}

{\tableofcontents}

\section{Foundational Concepts and Preliminaries}

\begin{definition}
  [Self-Adjoint Operator] Let $H$ be a complex Hilbert space. A densely
  defined linear operator $A : \Dom (A) \to H$ with $\Dom (A)$ dense in $H$ is
  {\tmem{self-adjoint}} if $A = A^{\ast}$, meaning:
  \begin{enumerate}
    \item $\Dom (A^{\ast}) = \Dom (A)$, and
    
    \item $\ip{A \phi}{\psi} = \ip{\phi}{A \psi}$ for all $\phi, \psi \in \Dom
    (A)$
  \end{enumerate}
\end{definition}

\begin{definition}
  [Orthogonal Projection] A bounded linear operator $P : H \to H$ is an
  {\tmem{orthogonal projection}} if $P^2 = P$ and $P^{\ast} = P$. The range
  $\Ran (P)$ is a closed subspace and $H = \Ran (P) \oplus \Ran (P)^{\perp}$.
\end{definition}

\section{Projection-Valued Measures and Spectral Calculus}

\begin{definition}
  [Projection-Valued Measure (PVM)] Let $(X, \mathcal{A})$ be a measurable
  space and $H$ a Hilbert space. A map $E : \mathcal{A} \to \mathcal{L} (H)$
  is a {\tmem{projection-valued measure}} if:
  \begin{enumerate}
    \item For each $B \in \mathcal{A}$, $E (B)$ is an orthogonal projection on
    $H$.
    
    \item $E (\emptyset) = 0$ and $E (X) = I$.
    
    \item For pairwise disjoint sets $\{B_k \}_{k \geq 1} \subset
    \mathcal{A}$,
    \begin{equation}
      E \hspace{-0.17em} \left( \bigcup_{k = 1}^{\infty} B_k \right)
      \hspace{0.27em} = \hspace{0.27em} \sum_{k = 1}^{\infty} E (B_k)
    \end{equation}
    where convergence is in the strong operator topology.
  \end{enumerate}
\end{definition}

\begin{lemma}
  [Properties of PVM]\label{lem:pvm_properties} Let $E$ be a PVM on $(X,
  \mathcal{A})$. Then:
  \begin{enumerate}
    \item If $B_1 \cap B_2 = \emptyset$, then $E (B_1) E (B_2) = 0$.
    
    \item If $B_1 \subset B_2$, then $E (B_1) \leq E (B_2)$ (in the partial
    order of projections).
    
    \item For each $\phi, \psi \in H$, the map $B \mapsto \ip{E (B)
    \phi}{\psi}$ defines a complex measure.
  \end{enumerate}
\end{lemma}

\begin{proof}
  \begin{enumeratealpha}
    \item Suppose $B_1 \cap B_2 = \emptyset$. Then $B_1 \cup B_2$ is disjoint,
    so by the PVM property,
    \begin{equation}
      E (B_1 \cup B_2) = E (B_1) + E (B_2)
    \end{equation}
    Applying both sides to $\phi \in H$, we have
    \begin{equation}
      E (B_1 \cup B_2) \phi = E (B_1) \phi + E (B_2) \phi
    \end{equation}
    . Now, since $E (B_1)$ and $E (B_2)$ are orthogonal projections and $B_1
    \cap B_2 = \emptyset$, the ranges $\Ran (E (B_1))$ and $\Ran (E (B_2))$
    are orthogonal. Indeed, for any $\phi \in H$,
    \begin{equation}
      \ip{E (B_1) \phi}{E (B_2) \phi} = \ip{E (B_1 \cap B_2) \phi}{\phi} =
      \ip{E (\emptyset) \phi}{\phi} = 0
    \end{equation}
    Hence
    \begin{equation}
      E (B_1) E (B_2) = 0
    \end{equation}
    \item If $B_1 \subset B_2$, write $B_2 = B_1 \cup (B_2 \setminus B_1)$
    disjointly. Then
    \begin{equation}
      E (B_2) = E (B_1) + E (B_2 \setminus B_1)
    \end{equation}
    Since $E (B_2 \setminus B_1) \geq 0$, we have
    \begin{equation}
      E (B_2) \geq E (B_1)
    \end{equation}
    \item For fixed $\phi, \psi \in H$, define $\mu_{\phi, \psi} (B) \assign
    \ip{E (B) \phi}{\psi}$. We verify $\mu_{\phi, \psi}$ is a complex measure.
    Clearly $\mu_{\phi, \psi} (\emptyset) = 0$. For disjoint $\{B_k \}$,
    \begin{equation}
      \begin{array}{ll}
        \mu_{\phi, \psi} \hspace{-0.17em} \left( \bigcup_{k = 1}^{\infty} B_k
        \right) & = \left\langle E \hspace{-0.17em} \left( \bigcup_{k =
        1}^{\infty} B_k \right) \phi, \psi \right\rangle\\
        & = \left\langle \sum_{k = 1}^{\infty} E (B_k) \phi, \psi
        \right\rangle \quad \text{(strong convergence)}\\
        & = \sum_{k = 1}^{\infty} \ip{E (B_k) \phi}{\psi} = \sum_{k =
        1}^{\infty} \mu_{\phi, \psi} (B_k)
      \end{array}
    \end{equation}
  \end{enumeratealpha}
  
\end{proof}

\begin{theorem}
  [Spectral Integral for Bounded
  Functions]\label{thm:spectral_integral_bounded} Let $E$ be a PVM on $(X,
  \mathcal{A})$ and $f : X \to \C$ a bounded measurable function. There exists
  a unique bounded operator $T_f \in \mathcal{L} (H)$ such that for all $\phi,
  \psi \in H$,
  \begin{equation}
    \ip{T_f \phi}{\psi} = \int_X f (x)  \hspace{0.17em} d \ip{E (x)
    \phi}{\psi}
  \end{equation}
  Moreover, $\norm{T_f} = \|f\|_{\infty}$ and $T_f^{\ast} = T_{\bar{f}}$. We
  write $T_f = \int_X f \hspace{0.17em} dE$.
\end{theorem}

\begin{proof}
  \tmtextbf{Step 1: Construction for simple functions.} Let $f = \sum_{j =
  1}^n c_j \1_{B_j}$ be a simple function with $B_j$ pairwise disjoint. Define
  \begin{equation}
    T_f \assign \sum_{j = 1}^n c_j E (B_j)
  \end{equation}
  This is a bounded operator since each $E (B_j)$ is a projection. For $\phi,
  \psi \in H$,
  \begin{equation}
    \begin{array}{ll}
      \ip{T_f \phi}{\psi} & = \sum_{j = 1}^n c_j \ip{E (B_j) \phi}{\psi}\\
      & = \sum_{j = 1}^n c_j  \int_X \1_{B_j} (x)  \hspace{0.17em} d \ip{E
      (x) \phi}{\psi}\\
      & = \int_X f (x)  \hspace{0.17em} d \ip{E (x) \phi}{\psi}
    \end{array}
  \end{equation}
  \tmtextbf{Step 2: Bound.} For any unit vector $\phi \in H$, define the
  positive measure
  \begin{equation}
    \nu_{\phi} (B) \assign \ip{E (B) \phi}{\phi}
  \end{equation}
  . Note $\nu_{\phi} (X) = \| \phi \|^2 = 1$. Then
  \begin{equation}
    \begin{array}{ll}
      | \ip{T_f \phi}{\phi} | & = \left| \int_X f \hspace{0.17em} d \nu_{\phi}
      \right| \leq \int_X |f|  \hspace{0.17em} d \nu_{\phi} \leq
      \|f\|_{\infty} \nu_{\phi} (X) = \|f\|_{\infty}
    \end{array}
  \end{equation}
  By the polarization identity, $| \ip{T_f \phi}{\psi} | \leq C \| \phi \| \|
  \psi \|$ for some constant $C \leq \|f\|_{\infty}$. Hence $\norm{T_f} \leq
  \|f\|_{\infty}$. Conversely, for any $\epsilon > 0$, there exists
  $B_{\epsilon}$ with $\nu_{\phi} (B_{\epsilon}) > 0$ and $|f (x) | >
  \|f\|_{\infty} - \epsilon$ on $B_{\epsilon}$. Choosing $\phi = E
  (B_{\epsilon}) \phi_0 / \|E (B_{\epsilon}) \phi_0 \|$ for suitable $\phi_0$,
  we get $\norm{T_f} \geq \|f\|_{\infty} - \epsilon$. Thus $\norm{T_f} =
  \|f\|_{\infty}$.
  
  \tmtextbf{Step 3: Extension to bounded functions.} For general bounded
  measurable $f$, approximate by simple functions $f_n \to f$ uniformly. Then
  $\|T_{f_n} - T_{f_m} \| = \|f_n - f_m \|_{\infty} \to 0$, so $(T_{f_n})$ is
  Cauchy in $\mathcal{L} (H)$. Define $T_f \assign \lim_{n \to \infty}
  T_{f_n}$. The integral formula follows by passing to the limit in the simple
  function case.
  
  \tmtextbf{Step 4: Adjoint.} For simple $f = \sum c_j \1_{B_j}$,
  \begin{equation}
    \text{$T_f^{\ast} = \left( \sum c_j E (B_j) \right)^{\ast} = \sum
    \overline{c_j} E (B_j) = T_{\bar{f}}$}
  \end{equation}
  By density and continuity, this extends to all bounded $f$.
\end{proof}

\begin{theorem}
  [Spectral Theorem for Bounded Self-Adjoint
  Operators]\label{thm:spectral_bounded} Let $A$ be a bounded self-adjoint
  operator on $H$. There exists a unique PVM $E_A$ on $\mathcal{B} (\R)$ (the
  Borel $\sigma$-algebra) such that
  \begin{equation}
    A = \int_{\R} \lambda \hspace{0.17em} dE_A (\lambda)
  \end{equation}
  and
  \begin{equation}
    \supp (E_A) \subseteq [-\|A\|, \|A\|]
  \end{equation}
\end{theorem}

\begin{proof}
  \tmtextbf{Step 1: Construct the commutative $C^{\ast}$-algebra.} Let
  $\mathcal{A}$ be the norm-closed $\ast$-subalgebra of $\mathcal{L} (H)$
  generated by $A$ and $I$. Since $A$ is self-adjoint, every element of
  $\mathcal{A}$ is a norm limit of polynomials in $A$ and $A^{\ast} = A$,
  hence $\mathcal{A}$ is commutative.
  
  \tmtextbf{Step 2: Gelfand transform.} By the Gelfand-Naimark theorem,
  $\mathcal{A}$ is isometrically $\ast$-isomorphic to $C (X)$ for some compact
  Hausdorff space $X$ (the spectrum of $\mathcal{A}$). The Gelfand transform
  $\Gamma : \mathcal{A} \to C (X)$ is a $\ast$-isomorphism. Under this
  isomorphism, $A$ corresponds to a continuous function $\hat{A} \in C (X)$
  which is real-valued (since $A$ is self-adjoint). The norm $\| \hat{A}
  \|_{\infty} = \|A\|$.
  
  \tmtextbf{Step 3: Representation on $C (X)$.} There exists a unitary $U : H
  \to L^2 (X, \mu)$ for some regular Borel measure $\mu$ on $X$ such that
  $UAU^{\ast}$ is multiplication by $\hat{A}$. That is, $(UAU^{\ast} \psi) (x)
  = \hat{A} (x) \psi (x)$ for $\psi \in L^2 (X, \mu)$.
  
  \tmtextbf{Step 4: Define the PVM.} For a Borel set $B \subset \R$, define
  \begin{equation}
    E_A (B) \assign U^{\ast} M_{\1_{\hat{A}^{- 1} (B)}} U
  \end{equation}
  where $M_{\1_{\hat{A}^{- 1} (B)}}$ is multiplication by the indicator
  function on $L^2 (X, \mu)$. This is an orthogonal projection. The map $E_A$
  is a PVM since $M_{\1_S}$ for Borel $S \subset X$ form a PVM.
  
  \tmtextbf{Step 5: Verification.} We have
  \begin{equation}
    \begin{array}{ll}
      \int_{\R} \lambda \hspace{0.17em} dE_A (\lambda) & = U^{\ast} \left(
      \int_{\R} \lambda \hspace{0.17em} d (M_{\1_{\hat{A}^{- 1} ((- \infty,
      \lambda])}}) \right) U\\
      & = U^{\ast} M_{\hat{A}} U\\
      & = A
    \end{array}
  \end{equation}
  The support is contained in $[-\|A\|, \|A\|]$ since $\hat{A}$ takes values
  in $[-\|A\|, \|A\|]$.
  
  \tmtextbf{Step 6: Uniqueness.} Suppose $E'_A$ is another PVM satisfying
  \begin{equation}
    A = \int \lambda \hspace{0.17em} dE'_A (\lambda)
  \end{equation}
  . Then for any polynomial $p$,
  \begin{equation}
    \begin{array}{ll}
      p (A) & = \int p (\lambda)  \hspace{0.17em} dE_A (\lambda)\\
      & = \int p (\lambda)  \hspace{0.17em} dE'_A (\lambda)
    \end{array}
  \end{equation}
  . By the Weierstrass approximation theorem, this extends to all continuous
  functions, hence to all Borel functions by monotone class arguments. Thus
  \begin{equation}
    E_A = E'_A
  \end{equation}
\end{proof}

\begin{theorem}
  [Spectral Theorem for Unbounded Self-Adjoint
  Operators]\label{thm:spectral_unbounded} Let $A$ be an unbounded
  self-adjoint operator on $H$. There exists a unique PVM $E_A$ on
  $\mathcal{B} (\R)$ such that
  \begin{equation}
    A = \int_{\R} \lambda \hspace{0.17em} dE_A (\lambda)
  \end{equation}
  \begin{equation}
    \Dom (A) = \left\{ \phi \in H : \int_{\R} \lambda^2  \hspace{0.17em}
    d\|E_A (\lambda) \phi \|^2 < \infty \right\}
  \end{equation}
  Moreover, for $\phi \in \Dom (A)$,
  \begin{equation}
    A \phi = \int_{\R} \lambda \hspace{0.17em} dE_A (\lambda) \phi
  \end{equation}
\end{theorem}

\begin{proof}
  \tmtextbf{Step 1: Cayley transform.} Define the Cayley transform
  \begin{equation}
    U \assign (A - iI)  (A + iI)^{- 1}
  \end{equation}
  . Since $A$ is self-adjoint, both $A \pm iI$ are bijections from $\Dom (A)$
  to $H$ with bounded inverses. Moreover, $U$ is a unitary operator on $H$. To
  see this, note that for $\phi \in \Dom (A)$,
  \begin{equation}
    \begin{array}{ll}
      \|U (A + iI) \phi \|^2 & = \|(A - iI) \phi \|^2 = \ip{(A - iI) \phi}{(A
      - iI) \phi}\\
      & = \ip{(A^2 + I) \phi}{\phi} = \ip{(A + iI) \phi}{(A + iI) \phi} =
      \|(A + iI) \phi \|^2
    \end{array}
  \end{equation}
  Thus $U$ extends to a unitary on $H$. Note $U$ has no eigenvalue $1$ (since
  $A$ is self-adjoint, $A - iI$ is injective).
  
  \tmtextbf{Step 2: Spectral theorem for $U$.} By Theorem
  \ref{thm:spectral_bounded}, there exists a PVM $F$ on $\mathcal{B}
  (\mathbb{T})$ (where $\mathbb{T}= \{z \in \C : |z| = 1\}$) such that
  \begin{equation}
    U = \int_{\mathbb{T}} z \hspace{0.17em} dF (z)
  \end{equation}
  . Since $1 \nin \sigma (U)$, $F (\{1\}) = 0$.
  
  \tmtextbf{Step 3: Inverse Cayley transform.} The inverse Cayley transform is
  $A = i (I + U)  (I - U)^{- 1}$. For $z \in \mathbb{T} \setminus \{1\}$, the
  function
  \begin{equation}
    \lambda (z) \assign i \frac{1 + z}{1 - z}
  \end{equation}
  maps $\mathbb{T} \setminus \{1\}$ onto $\R$. This is a homeomorphism.
  Define the PVM $E_A$ on $\R$ by
  \begin{equation}
    E_A (B) \assign F (\lambda^{- 1} (B))
  \end{equation}
  for Borel $B \subset \R$.
  
  \tmtextbf{Step 4: Verification of the spectral integral.} We have
  \begin{equation}
    \begin{array}{ll}
      \int_{\R} \lambda \hspace{0.17em} dE_A (\lambda) & = \int_{\mathbb{T}
      \setminus \{1\}} i \frac{1 + z}{1 - z}  \hspace{0.17em} dF (z)\\
      & = i \int_{\mathbb{T}} (1 + z)  (1 - z)^{- 1}  \hspace{0.17em} dF
      (z)\\
      & = i (I + U)  (I - U)^{- 1}\\
      & = A
    \end{array}
  \end{equation}
  The domain calculation follows from the fact that $\phi \in \Dom (A)$ if and
  only if $(I - U)^{- 1} \phi \in \Dom (I + U)$, which is equivalent to
  \begin{equation}
    \int_{\mathbb{T}} \left| \frac{1 + z}{1 - z} \right|^2  \hspace{0.17em} d
    \|F (z) \phi \|^2 = \int_{\R} \lambda^2  \hspace{0.17em} d \|E_A (\lambda)
    \phi \|^2 < \infty
  \end{equation}
  \tmtextbf{Step 5: Uniqueness.} Uniqueness follows from the uniqueness of the
  spectral theorem for the unitary operator $U$ and the bijection between PVMs
  for $U$ and $A$ via the Cayley transform.
\end{proof}

\begin{corollary}
  [Functional Calculus]\label{cor:functional_calculus} Let $A$ be self-adjoint
  with spectral measure $E_A$. For any Borel function $f : \R \to \C$, define
  \begin{equation}
    f (A) \assign \int_{\R} f (\lambda)  \hspace{0.17em} dE_A (\lambda)
  \end{equation}
  \begin{equation}
    \Dom (f (A)) = \left\{ \phi \in H : \int_{\R} |f (\lambda) |^2 
    \hspace{0.17em} d\|E_A (\lambda) \phi \|^2 < \infty \right\}
  \end{equation}
  Then $f (A)$ is a normal operator (bounded if $f$ is bounded), and
  \begin{equation}
    (fg) (A) = f (A) g (A)
  \end{equation}
  on $\Dom (g (A)) \cap \Dom ((fg) (A))$.
\end{corollary}

\begin{proof}
  This follows directly from the properties of the spectral integral
  established in Theorem \ref{thm:spectral_integral_bounded} and its extension
  to unbounded functions by truncation arguments. The composition property
  follows from the measure-theoretic identity
  \begin{equation}
    \int fg \hspace{0.17em} dE = \int f \hspace{0.17em} dE \cdot \int g
    \hspace{0.17em} dE
  \end{equation}
  for projection-valued integrals.
\end{proof}

\section{Stone's Theorem on One-Parameter Unitary Groups}

\begin{definition}
  [Strongly Continuous Unitary Group] A family $(U_t)_{t \in \R} \subset
  \mathcal{L} (H)$ is a {\tmem{strongly continuous one-parameter unitary
  group}} if:
  \begin{enumerate}
    \item $U_t$ is unitary for all $t \in \R$,
    
    \item $U_{t + s} = U_t U_s$ for all $s, t \in \R$,
    
    \item $U_0 = I$,
    
    \item $\lim_{t \to 0} \norm{U_t \phi - \phi} = 0$ for all $\phi \in H$.
  \end{enumerate}
\end{definition}

\begin{definition}
  [Infinitesimal Generator] The {\tmem{infinitesimal generator}} $A$ of
  $(U_t)$ is defined by
  \begin{equation}
    \Dom (A) \assign \left\{ \phi \in H : \lim_{t \to 0}  \frac{U_t \phi -
    \phi}{t} \text{exists} \right\} \text{}
  \end{equation}
  \begin{equation}
    A \phi \assign \lim_{t \to 0}  \frac{U_t \phi - \phi}{t}
  \end{equation}
\end{definition}

\begin{lemma}
  [Basic Properties of the Generator]\label{lem:generator_properties} Let $A$
  be the generator of a strongly continuous unitary group $(U_t)$. Then:
  \begin{enumerate}
    \item $\Dom (A)$ is dense in $H$.
    
    \item For $\phi \in \Dom (A)$, the map $t \mapsto U_t \phi$ is
    differentiable with
    \begin{equation}
      \begin{array}{ll}
        \frac{d}{dt} U_t \phi & = U_t A \phi\\
        & = AU_t \phi
      \end{array}
    \end{equation}
    \item $A$ is closed.
    
    \item $A$ is skew-adjoint: $iA$ is self-adjoint.
  \end{enumerate}
\end{lemma}

\begin{proof}
  (a) For $\phi \in H$ and $h > 0$, define
  \begin{equation}
    \phi_h \assign \frac{1}{h}  \int_0^h U_s \phi \hspace{0.17em} ds
  \end{equation}
  This integral exists as a Riemann integral of continuous $H$-valued
  functions. We claim $\phi_h \in \Dom (A)$. Indeed,
  \begin{equation}
    \begin{array}{ll}
      \frac{U_t \phi_h - \phi_h}{t} & = \frac{1}{ht}  \int_0^h (U_{t + s} \phi
      - U_s \phi)  \hspace{0.17em} ds\\
      & = \frac{1}{ht}  \left( \int_t^{t + h} U_s \phi \hspace{0.17em} ds -
      \int_0^h U_s \phi \hspace{0.17em} ds \right)\\
      & = \frac{1}{ht}  \left( \int_h^{t + h} U_s \phi \hspace{0.17em} ds -
      \int_0^t U_s \phi \hspace{0.17em} ds \right) .
    \end{array}
  \end{equation}
  As $t \to 0$, this converges to $\frac{1}{h}  (U_h \phi - \phi)$. Thus
  $\phi_h \in \Dom (A)$ and $A \phi_h = \frac{1}{h}  (U_h \phi - \phi)$.
  
  Now, $\| \phi_h - \phi \| \leq \frac{1}{h}  \int_0^h \|U_s \phi - \phi \| 
  \hspace{0.17em} ds \to 0$ as $h \to 0$ by dominated convergence and strong
  continuity. Thus $\Dom (A)$ is dense.
  
  (b) For $\phi \in \Dom (A)$ and $t, h \in \R$,
  \begin{equation}
    \begin{array}{ll}
      \frac{U_{t + h} \phi - U_t \phi}{h} & = U_t  \frac{U_h \phi - \phi}{h}
    \end{array}
  \end{equation}
  As $h \to 0$, $\frac{U_h \phi - \phi}{h} \to A \phi$. By continuity of
  $U_t$,
  \begin{equation}
    U_t  \frac{U_h \phi - \phi}{h} \to U_t A \phi
  \end{equation}
  Similarly,
  \begin{equation}
    \frac{U_{t + h} \phi - U_t \phi}{h} = \frac{U_h  (U_t \phi) - (U_t
    \phi)}{h}
  \end{equation}
  Since $U_t \phi \in \Dom (A)$ (by the argument below), this converges to
  $AU_t \phi$. Thus
  \begin{equation}
    \frac{d}{dt} U_t \phi = AU_t \phi = U_t A \phi
  \end{equation}
  To show $U_t (\Dom (A)) \subseteq \Dom (A)$: for $\phi \in \Dom (A)$,
  \begin{equation}
    \begin{array}{ll}
      \frac{U_h  (U_t \phi) - U_t \phi}{h} & = \frac{U_{t + h} \phi - U_t
      \phi}{h}\\
      & = U_t  \frac{U_h \phi - \phi}{h} \to U_t A \phi
    \end{array}
  \end{equation}
  as $h \to 0$. Thus $U_t \phi \in \Dom (A)$ and
  \begin{equation}
    A (U_t \phi) = U_t A \phi
  \end{equation}
  (c) Suppose $\phi_n \in \Dom (A)$ with $\phi_n \to \phi$ and $A \phi_n \to
  \psi$. We need to show $\phi \in \Dom (A)$ and $A \phi = \psi$. For any $t
  \neq 0$,
  \begin{equation}
    \begin{array}{ll}
      \frac{U_t \phi - \phi}{t} & = \lim_{n \to \infty}  \frac{U_t \phi_n -
      \phi_n}{t}\\
      & = \lim_{n \to \infty}  \frac{1}{t}  \int_0^t U_s A \phi_n 
      \hspace{0.17em} ds\\
      & = \frac{1}{t}  \int_0^t U_s \psi \hspace{0.17em} ds
    \end{array}
  \end{equation}
  As $t \to 0$, the right-hand side converges to $\psi$ by continuity. Thus
  $\phi \in \Dom (A)$ and
  \begin{equation}
    A \phi = \psi
  \end{equation}
  (d) We show $iA$ is self-adjoint. For $\phi \in \Dom (A)$ and $\psi \in H$,
  \begin{equation}
    \begin{array}{ll}
      \ip{A \phi}{\psi} & = \lim_{t \to 0}  \frac{1}{t} \ip{U_t \phi -
      \phi}{\psi}\\
      & = \lim_{t \to 0}  \frac{1}{t}  (\ip{U_t \phi}{\psi} -
      \ip{\phi}{\psi})
    \end{array}
  \end{equation}
  Since $U_t$ is unitary,
  \begin{equation}
    \ip{U_t \phi}{\psi} = \ip{\phi}{U_{- t} \psi} = \ip{\phi}{U_t^{\ast} \psi}
  \end{equation}
  Thus
  \begin{equation}
    \begin{array}{ll}
      \ip{A \phi}{\psi} & = \lim_{t \to 0}  \frac{1}{t}  (\ip{\phi}{U_t^{\ast}
      \psi} - \ip{\phi}{\psi})\\
      & = \lim_{t \to 0} \ip{\phi}{\frac{U_t^{\ast} \psi - \psi}{t}}\\
      & = \lim_{t \to 0} \ip{\phi}{\frac{U_{- t} \psi - \psi}{t}} = - \lim_{t
      \to 0} \ip{\phi}{\frac{U_{- t} \psi - \psi}{- t}}
    \end{array}
  \end{equation}
  If $\psi \in \Dom (A)$, this equals $- \ip{\phi}{A \psi}$. Thus for $\phi,
  \psi \in \Dom (A)$,
  \begin{equation}
    \ip{A \phi}{\psi} = - \ip{\phi}{A \psi}
  \end{equation}
  i.e.,
  \begin{equation}
    \ip{iA \phi}{\psi} = \ip{\phi}{iA \psi}
  \end{equation}
  Hence $iA$ is symmetric on $\Dom (A)$. To show $iA$ is self-adjoint, we use
  the resolvent identity. For ,
  \begin{equation}
    \begin{array}{ll}
      R_{\lambda} & = (A - i \lambda I)^{- 1}\\
      & = \int_0^{\infty} e^{- \lambda t} U_t  \hspace{0.17em} dt \forall
      \lambda \in \R \setminus \{0\}
    \end{array}
  \end{equation}
  exists as a bounded operator (for $\lambda < 0$ integrate $\int_0^{\infty} =
  \int_{- \infty}^0 e^{\lambda s} U_{- s}  \hspace{0.17em} ds$ with $s = -
  t$). This formula shows
  \begin{equation}
    \Ran (A - i \lambda I) = H
  \end{equation}
  and
  \begin{equation}
    \ker (A - i \lambda I) = \{0\}
  \end{equation}
  , implying $iA$ is self-adjoint.
\end{proof}

\begin{theorem}
  [Bochner's Theorem]\label{thm:bochner} Let $g : \R \to \C$ be a continuous
  function. Then $g$ is positive-definite (meaning $\sum_{j, k} g (t_j - t_k)
  \overline{c_j} c_k \geq 0$ for all finite collections $\{t_j \}$ and $\{c_j
  \} \subset \C$) if and only if there exists a finite positive Borel measure
  $\mu$ on $\R$ such that
  \begin{equation}
    g (t) = \int_{\R} e^{it \lambda}  \hspace{0.17em} d \mu (\lambda)
  \end{equation}
\end{theorem}

\begin{proof}
  ($\Leftarrow$) If $g (t) = \int e^{it \lambda}  \hspace{0.17em} d \mu
  (\lambda)$ for a positive measure $\mu$, then for any $\{t_j \}$ and $\{c_j
  \}$,
  \begin{equation}
    \begin{array}{ll}
      \sum_{j, k} g (t_j - t_k) \overline{c_j} c_k & = \sum_{j, k} \int e^{i
      (t_j - t_k) \lambda}  \hspace{0.17em} d \mu (\lambda) \hspace{0.17em}
      \overline{c_j} c_k\\
      & = \int \sum_{j, k} e^{it_j \lambda} e^{- it_k \lambda} \overline{c_j}
      c_k  \hspace{0.17em} d \mu (\lambda)\\
      & = \int \left| \sum_j c_j e^{it_j \lambda} \right|^2  \hspace{0.17em}
      d \mu (\lambda) \geq 0.
    \end{array}
  \end{equation}
  ($\Rightarrow$) Suppose $g$ is positive-definite and continuous.
  
  \tmtextbf{Step 1: Construct a pre-Hilbert space.} Let $\mathcal{S}$ be the
  space of finite linear combinations $\phi = \sum_{j = 1}^n c_j \delta_{t_j}$
  (where $\delta_t$ is the Dirac measure at $t$). Define an inner product on
  $\mathcal{S}$ by
  \begin{equation}
    \ip{\sum_j c_j \delta_{t_j}}{\sum_k d_k \delta_{s_k}} = \sum_{j, k} g (t_j
    - s_k) \overline{c_j} d_k .
  \end{equation}
  Positive-definiteness ensures $\ip{\phi}{\phi} \geq 0$. The seminorm $\|
  \phi \| \assign \sqrt{\ip{\phi}{\phi}}$ may have a null space $\mathcal{N}
  \assign \{\phi : \| \phi \|= 0\}$. Define the quotient
  $\mathcal{S}/\mathcal{N}$ and complete to obtain a Hilbert space $H$.
  
  \tmtextbf{Step 2: Define the translation operators.} For $t \in \R$, define
  $U_t : \mathcal{S} \to \mathcal{S}$ by $U_t \delta_s \assign \delta_{s +
  t}$. Then
  \begin{equation}
    \text{$\ip{U_t \phi}{\psi} = \ip{\phi}{U_{- t} \psi}$}
  \end{equation}
  for all $\phi, \psi \in \mathcal{S}$, since
  \begin{equation}
    \begin{array}{ll}
      \text{$\ip{U_t \delta_s}{U_t \delta_r}$} & = g ((s + t) - (r + t))\\
      & = g (s - r)\\
      & = \ip{\delta_s}{\delta_r}
    \end{array}
  \end{equation}
  Thus $U_t$ extends to a unitary operator on $H$. The group property $U_{t +
  s} = U_t U_s$ is clear. Strong continuity follows from
  \begin{equation}
    \begin{array}{ll}
      \lim_{t \rightarrow 0} \|U_t \delta_s - \delta_s \|^2 & = g (0) - g (t)
      - \overline{g (t)} + g (0)\\
      & = \lim_{t \rightarrow 0} 2 \re (g (0) - g (t))\\
      & = 0
    \end{array}
  \end{equation}
  by continuity of $g$.
  
  \tmtextbf{Step 3: Apply Stone's theorem.} By Stone's theorem (Theorem
  \ref{thm:stone} below), there exists a self-adjoint operator $A$ on $H$ such
  that
  \begin{equation}
    U_t = e^{itA}
  \end{equation}
  . Write
  \begin{equation}
    A = \int \lambda \hspace{0.17em} dE_A (\lambda)
  \end{equation}
  for the spectral measure $E_A$. Then
  \begin{equation}
    U_t = \int e^{it \lambda}  \hspace{0.17em} dE_A (\lambda)
  \end{equation}
  Define $\mu$ by $\mu (B) \assign \ip{E_A (B) \delta_0}{\delta_0}$. This is a
  positive finite measure (finite since $\mu (\R) = \| \delta_0 \|^2 = g (0) <
  \infty$). Then
  \begin{equation}
    \begin{array}{ll}
      g (t) & = \ip{U_t \delta_0}{\delta_0} = \int e^{it \lambda} 
      \hspace{0.17em} d \ip{E_A (\lambda) \delta_0}{\delta_0} = \int e^{it
      \lambda}  \hspace{0.17em} d \mu (\lambda)
    \end{array}
  \end{equation}
  
\end{proof}

\begin{theorem}
  [Stone's Theorem]\label{thm:stone} There is a bijective correspondence
  between strongly continuous one-parameter unitary groups $(U_t)_{t \in \R}$
  on $H$ and self-adjoint operators $A$ on $H$ given by
  \begin{equation}
    U_t = e^{itA} = \int_{\R} e^{it \lambda}  \hspace{0.17em} dE_A (\lambda)
  \end{equation}
  where $E_A$ is the spectral measure of $A$, and $A$ is the infinitesimal
  generator of $(U_t)$.
\end{theorem}

\begin{proof}
  ($\Rightarrow$) Given a strongly continuous unitary group $(U_t)$, let $A$
  be its generator (Definition). By Lemma \ref{lem:generator_properties}(d),
  $iA$ is self-adjoint. Let $E_{iA}$ be the spectral measure of $iA$. Define
  \begin{equation}
    V_t \assign \int_{\R} e^{it \lambda}  \hspace{0.17em} dE_{iA} (\lambda) =
    \int_{\R} e^{- t \mu}  \hspace{0.17em} dE_A (\mu)
  \end{equation}
  where $E_A (\cdummy) \assign E_{iA}  (i^{- 1} \cdot)$ is the spectral
  measure of $A = - i (iA)$. We need to show $V_t = U_t$.
  
  \tmtextbf{Step 1: $V_t$ is a unitary group.} Since $e^{it \lambda}$ has
  modulus 1, $V_t$ is unitary. The group property follows from
  \[ V_{t + s} = \int e^{i (t + s) \lambda}  \hspace{0.17em} dE_{iA} (\lambda)
     = \int e^{it \lambda} e^{is \lambda}  \hspace{0.17em} dE_{iA} (\lambda) =
     V_t V_s . \]
  Strong continuity: for $\phi \in H$,
  \begin{equation}
    \begin{array}{ll}
      \|V_t \phi - \phi \|^2 & = \int |e^{it \lambda} - 1|^2  \hspace{0.17em}
      d \|E_{iA} (\lambda) \phi \|^2
    \end{array}
  \end{equation}
  By dominated convergence ($|e^{it \lambda} - 1| \leq 2$), this tends to 0 as
  $t \to 0$.
  
  \tmtextbf{Step 2: The generator of $V_t$ is $A$.} For $\phi \in \Dom (A)$,
  \begin{equation}
    \begin{array}{ll}
      \lim_{t \to 0}  \frac{V_t \phi - \phi}{t} & = \lim_{t \to 0}  \int
      \frac{e^{it \lambda} - 1}{t}  \hspace{0.17em} dE_{iA} (\lambda) \phi\\
      & = \int i \lambda \hspace{0.17em} dE_{iA} (\lambda) \phi\\
      & = \int \mu \hspace{0.17em} dE_A (\mu) \phi = A \phi,
    \end{array}
  \end{equation}
  where the limit interchange is justified by dominated convergence on compact
  sets and the domain condition $\int \lambda^2  \hspace{0.17em} d \|E_{iA}
  (\lambda) \phi \|^2 < \infty$.
  
  \tmtextbf{Step 3: Uniqueness of generators.} If $(U_t)$ and $(V_t)$ have the
  same generator $A$, then for $\phi \in \Dom (A)$,
  \begin{equation}
    \frac{d}{dt}  (U_t \phi - V_t \phi) = A (U_t \phi - V_t \phi)
  \end{equation}
  Define $\psi (t) \assign U_{- t}  (U_t \phi - V_t \phi)$. Then $\psi (0) =
  0$ and
  \begin{equation}
    \text{$\psi' (t) = U_{- t}  (- A + A)  (U_t \phi - V_t \phi) + U_{- t} 
    \frac{d}{dt}  (U_t \phi - V_t \phi) = 0$}
  \end{equation}
  Thus $\psi (t) = 0$ for all $t$, implying $U_t \phi = V_t \phi$ for all
  $\phi \in \Dom (A)$. By density of $\Dom (A)$ and continuity, $U_t = V_t$.
  
  ($\Leftarrow$) Conversely, given a self-adjoint operator $A$ with spectral
  measure $E_A$, define
  \begin{equation}
    U_t \assign \int e^{it \lambda}  \hspace{0.17em} dE_A (\lambda)
  \end{equation}
  This is a unitary operator (as in Step 1 above). The group property and
  strong continuity follow as before. The generator is
  \begin{equation}
    \begin{array}{ll}
      \lim_{t \to 0}  \frac{U_t \phi - \phi}{t} & = \lim_{t \to 0}  \int
      \frac{e^{it \lambda} - 1}{t}  \hspace{0.17em} dE_A (\lambda) \phi = \int
      \lambda \hspace{0.17em} dE_A (\lambda) \phi = A \phi
    \end{array}
  \end{equation}
  for $\phi \in \Dom (A)$. Thus the generator is $A$.
\end{proof}

\section{Essential Self-Adjointness and Nelson's Theorem}

\begin{definition}
  [Symmetric Operator] A densely defined operator $S : \Dom (S) \to H$ is
  {\tmem{symmetric}} if $\ip{S \phi}{\psi} = \ip{\phi}{S \psi}$ for all $\phi,
  \psi \in \Dom (S)$. Equivalently, $S \subseteq S^{\ast}$.
\end{definition}

\begin{definition}
  [Essentially Self-Adjoint] A symmetric operator $S$ is {\tmem{essentially
  self-adjoint}} if its closure $\bar{S}$ is self-adjoint, i.e., $\bar{S} =
  (\bar{S})^{\ast}$.
\end{definition}

\begin{definition}
  [Analytic Vector] A vector $\phi \in H$ is an {\tmem{analytic vector}} for
  an operator $S$ if $\phi \in \bigcap_{n = 0}^{\infty} \Dom (S^n)$ and there
  exists $r > 0$ such that
  \begin{equation}
    \sum_{n = 0}^{\infty} \frac{\norm{S^n \phi}}{n!} r^n < \infty
  \end{equation}
\end{definition}

\begin{theorem}
  [Nelson's Analytic Vector Theorem]\label{thm:nelson} Let $S$ be a symmetric
  operator on $H$. If there exists a dense subspace $\mathcal{D} \subset \Dom
  (S)$ consisting entirely of analytic vectors for $S$, then $S$ is
  essentially self-adjoint.
\end{theorem}

\begin{proof}
  \tmtextbf{Step 1: Define local one-parameter groups on analytic vectors.}
  For $\phi \in \mathcal{D}$, let $r_{\phi} > 0$ be such that $\sum_{n =
  0}^{\infty} \frac{\norm{S^n \phi}}{n!} r_{\phi}^n < \infty$. For $|t| <
  r_{\phi}$, define
  \begin{equation}
    U_t^{(\phi)} \assign \sum_{n = 0}^{\infty} \frac{(it)^n}{n!} S^n \phi
  \end{equation}
  This series converges in $H$ by the comparison test. Moreover, $U_0^{(\phi)}
  = \phi$ and
  \begin{equation}
    \begin{array}{ll}
      U_{t + s}^{(\phi)} & = \sum_{n = 0}^{\infty} \frac{(i (t + s))^n}{n!}
      S^n \phi\\
      & = \left( \sum_{j = 0}^{\infty} \frac{(it)^j}{j!} S^j \right) \left(
      \sum_{k = 0}^{\infty} \frac{(is)^k}{k!} S^k \right) \phi\\
      & = U_t^{(U_s^{(\phi)})}
    \end{array}
  \end{equation}
  for $|t|, |s|, |t + s| < r_{\phi}$.
  
  \tmtextbf{Step 2: Extend to a global strongly continuous unitary group.} For
  $\phi, \psi \in \mathcal{D}$, define
  \begin{equation}
    f_{\phi, \psi} (t) \assign \ip{U_t^{(\phi)}}{U_{- t}^{(\psi)}}
  \end{equation}
  By symmetry of $S$, $\ip{S^n \phi}{\psi} = \ip{\phi}{S^n \psi}$ (by
  induction), so
  \begin{equation}
    \begin{array}{ll}
      f_{\phi, \psi} (t) & = \sum_{n, m = 0}^{\infty} \frac{(it)^n  (-
      it)^m}{n!m!} \ip{S^n \phi}{S^m \psi}\\
      & = \sum_{n, m = 0}^{\infty} \frac{(it)^n  (- it)^m}{n!m!} \ip{S^{n +
      m} \phi}{\psi}\\
      & = \ip{\phi}{\psi}
    \end{array}
  \end{equation}
  for $|t| < \min (r_{\phi}, r_{\psi})$. This shows $\|U_t^{(\phi)} \| = \|
  \phi \|$ (taking $\psi = \phi$), so $U_t^{(\phi)} / \| \phi \|$ is an
  isometry locally.
  
  By the density of $\mathcal{D}$ and a standard extension argument (using the
  fact that isometric maps on a dense subset extend uniquely to the whole
  space), the local isometries patch together to define a global strongly
  continuous unitary group $(U_t)_{t \in \R}$ on $H$.
  
  \tmtextbf{Step 3: The generator extends $S$.} Let $A$ be the generator of
  $(U_t)$. For $\phi \in \mathcal{D}$,
  
  \begin{align*}
    \lim_{t \to 0}  \frac{U_t \phi - \phi}{t} & = \lim_{t \to 0}  \frac{1}{t}
    \left( \sum_{n = 1}^{\infty} \frac{(it)^n}{n!} S^n \phi \right)\\
    & = \lim_{t \to 0}  \sum_{n = 1}^{\infty} \frac{(it)^{n - 1}}{(n - 1) !}
    S^n \phi = iS \phi
  \end{align*}
  
  Thus $\mathcal{D} \subset \Dom (A)$ and $A|_{\mathcal{D}} = iS$. Since $A$
  is the generator, $iA$ is self-adjoint (Lemma
  \ref{lem:generator_properties}(d)). Hence $- iA$ is self-adjoint, so $S
  \subseteq - iA$. But $- iA$ is self-adjoint, so $\bar{S} \subseteq - iA = (-
  iA)^{\ast} \subseteq \bar{S}^{\ast}$. This forces $\bar{S} = (- iA)$ to be
  self-adjoint.
\end{proof}

\begin{proposition}
  [Deficiency Indices Criterion]\label{prop:deficiency} A densely defined
  symmetric operator $S$ is essentially self-adjoint if and only if $\ker
  (S^{\ast} \pm iI) = \{0\}$ (the deficiency indices are $(0, 0)$).
\end{proposition}

\begin{proof}
  \tmtextbf{Step 1: Necessity.} If $S$ is essentially self-adjoint, then
  $\bar{S}$ is self-adjoint. For a self-adjoint operator $A$, $\ker (A \pm iI)
  = \{0\}$ since $\|A \phi \pm i \phi \|^2 = \|A \phi \|^2 + \| \phi \|^2 \geq
  \| \phi \|^2$.
  
  \tmtextbf{Step 2: Sufficiency (von Neumann's theorem).} The adjoint
  $S^{\ast}$ decomposes as $\Dom (S^{\ast}) = \Dom (\bar{S}) \oplus \ker
  (S^{\ast} - iI) \oplus \ker (S^{\ast} + iI)$ (orthogonal direct sum of graph
  inner products). If both deficiency spaces vanish, then $\Dom (S^{\ast}) =
  \Dom (\bar{S})$, so $\bar{S} = S^{\ast}$ is self-adjoint.
\end{proof}

\section{Hilbert Space Structure of Weakly Stationary Processes}

\begin{definition}
  [Weakly Stationary Process] A stochastic process $(X_t)_{t \in \R}$ on a
  probability space $(\Omega, \mathcal{F}, \Prob)$ is {\tmem{weakly
  stationary}} (or {\tmem{second-order stationary}}) if:
  \begin{enumerate}
    \item $\E [X_t] = m$ for all $t$ (constant mean),
    
    \item $\E [|X_t |^2] < \infty$ for all $t$,
    
    \item The covariance function $\gamma (s, t) \assign \E [(X_s - m)
    \overline{(X_t - m)}]$ depends only on $s - t$: $\gamma (s, t) = \gamma (s
    - t, 0) = : \gamma (s - t)$.
  \end{enumerate}
\end{definition}

Without loss of generality, assume $m = 0$ (by subtracting the mean). Then
$\gamma (t) \assign \E [X_t \overline{X_0}]$.

\begin{lemma}
  [Positive-Definiteness of Covariance]\label{lem:covariance_pd} The
  covariance function $\gamma$ of a weakly stationary process is
  positive-definite: for any $t_1, \ldots, t_n \in \R$ and $c_1, \ldots, c_n
  \in \C$,
  \[ \sum_{j, k = 1}^n \gamma (t_j - t_k) \overline{c_j} c_k \geq 0. \]
\end{lemma}

\begin{proof}
  We have
  \begin{equation}
    \begin{array}{ll}
      \sum_{j, k = 1}^n \gamma (t_j - t_k) \overline{c_j} c_k & = \sum_{j, k}
      \E [X_{t_j} \overline{X_{t_k}}] \overline{c_j} c_k\\
      & = \E \left[ \left| \sum_j c_j X_{t_j} \right|^2 \right] \geq 0.
    \end{array}
  \end{equation}
  
\end{proof}

\begin{theorem}
  [Construction of the Process Hilbert Space]\label{thm:process_hilbert} Let
  $(X_t)$ be a weakly stationary process with covariance $\gamma$. Define
  \begin{equation}
    H \assign \overline{\text{span}} \{X_t : t \in \R \}^{L^2 (\Omega)}
  \end{equation}
  the closure in $L^2 (\Omega, \Prob)$ of the linear span of $\{X_t : t \in \R
  \}$. Then $H$ is a Hilbert space with inner product $\ip{Y}{Z} \assign \E [Y
  \bar{Z}]$. Moreover, the translation operators $(T_h X_t) \assign X_{t + h}$
  extend to a strongly continuous unitary group on $H$.
\end{theorem}

\begin{proof}
  \tmtextbf{Step 1: $H$ is a Hilbert space.} $H$ is a closed subspace of the
  Hilbert space $L^2 (\Omega)$, so it is complete.
  
  \tmtextbf{Step 2: Translation operators.} For $Y = \sum_{j = 1}^n c_j
  X_{t_j} \in \text{span} \{X_t \}$, define $T_h Y \assign \sum_j c_j X_{t_j +
  h}$. Then
  \begin{equation}
    \begin{array}{ll}
      \ip{T_h Y}{T_h Z} & = \E \left[ \sum_j c_j X_{t_j + h}  \overline{\sum_k
      d_k X_{s_k + h}} \right]\\
      & = \sum_{j, k} c_j \overline{d_k} \E [X_{t_j + h} \overline{X_{s_k +
      h}}]\\
      & = \sum_{j, k} c_j \overline{d_k} \gamma ((t_j + h) - (s_k + h))\\
      & = \sum_{j, k} c_j \overline{d_k} \gamma (t_j - s_k) = \ip{Y}{Z} .
    \end{array}
  \end{equation}
  Thus $T_h$ is an isometry on span$\{X_t \}$. Since this span is dense in
  $H$, $T_h$ extends uniquely to a unitary operator on $H$.
  
  \tmtextbf{Step 3: Group property.} For $Y \in \text{span} \{X_t \}$, $T_{h +
  k} Y = T_h  (T_k Y)$ by definition. By density and continuity, this extends
  to all of $H$.
  
  \tmtextbf{Step 4: Strong continuity.} For $Y = \sum_j c_j X_{t_j}$,
  \begin{equation}
    \begin{array}{ll}
      \|T_h Y - Y\|^2 & = \E \left| \sum_j c_j (X_{t_j + h} - X_{t_j})
      \right|^2\\
      & = \sum_{j, k} c_j \overline{c_k} \E [(X_{t_j + h} - X_{t_j})
      \overline{(X_{t_k + h} - X_{t_k})}]\\
      & = \sum_{j, k} c_j \overline{c_k} (2 \gamma (0) - \gamma (t_j - t_k +
      h) - \gamma (t_j - t_k - h)) .
    \end{array}
  \end{equation}
  By continuity of $\gamma$ at 0, this tends to 0 as $h \to 0$. By density,
  $\lim_{h \to 0} \|T_h \phi - \phi \| = 0$ for all $\phi \in H$.
\end{proof}

\begin{theorem}
  [Spectral Representation of Stationary Processes -
  Cram{\'e}r]\label{thm:cramer} Let $(X_t)$ be a weakly stationary process
  with covariance $\gamma$. There exists a right-continuous, non-decreasing
  function $F : \R \to \R$ with $F (- \infty) = 0$ such that
  \begin{equation}
    \gamma (t) = \int_{\R} e^{it \lambda}  \hspace{0.17em} dF (\lambda) . 
  \end{equation}
  Moreover, there exists an orthogonal random measure $Z$ on $\R$ (i.e., $\E
  [Z (B_1) \overline{Z (B_2)}] = 0$ for disjoint Borel sets $B_1, B_2$) such
  that
  \begin{equation}
    X_t = \int_{\R} e^{it \lambda}  \hspace{0.17em} dZ (\lambda)
  \end{equation}
  where the integral is in the $L^2 (\Omega)$ sense, and $\E [|Z (B) |^2] = F
  (B)$ for Borel $B$.
\end{theorem}

\begin{proof}
  \tmtextbf{Step 1: Apply Bochner's theorem.} By Lemma
  \ref{lem:covariance_pd}, $\gamma$ is positive-definite. If $\gamma$ is
  continuous, Bochner's theorem (Theorem \ref{thm:bochner}) yields a finite
  positive measure $\mu$ on $\R$ such that $\gamma (t) = \int e^{it \lambda} 
  \hspace{0.17em} d \mu (\lambda)$. Define $F (\lambda) \assign \mu ((-
  \infty, \lambda])$. Then $F$ is right-continuous, non-decreasing, and $F (-
  \infty) = 0$, $F (\infty) = \mu (\R) = \gamma (0) = \E [|X_0 |^2]$.
  
  \tmtextbf{Step 2: Construct the spectral measure for the translation group.}
  By Theorem \ref{thm:process_hilbert}, $(T_h)$ is a strongly continuous
  unitary group on $H$. By Stone's theorem (Theorem \ref{thm:stone}), there
  exists a self-adjoint generator $A$ with spectral measure $E_A$ such that
  \begin{equation}
    T_h = \int e^{ih \lambda}  \hspace{0.17em} dE_A (\lambda)
  \end{equation}
  \tmtextbf{Step 3: Define the orthogonal random measure.} For a Borel set $B
  \subset \R$, define the random variable
  \begin{equation}
    Z (B) \assign E_A (B) X_0
  \end{equation}
  This is an element of $H \subset L^2 (\Omega)$, hence a random variable. We
  verify the orthogonality property: for disjoint $B_1, B_2$,
  \begin{equation}
    \begin{array}{ll}
      \E [Z (B_1) \overline{Z (B_2)}] & = \ip{E_A (B_1) X_0}{E_A (B_2) X_0}\\
      & = \ip{E_A  (B_1 \cap B_2) X_0}{X_0}\\
      & = \ip{E_A (\emptyset) X_0}{X_0}\\
      & = 0
    \end{array}
  \end{equation}
  \tmtextbf{Step 4: Spectral representation of $X_t$.} We have
  \begin{equation}
    \begin{array}{ll}
      X_t & = T_t X_0\\
      & = \left( \int e^{it \lambda}  \hspace{0.17em} dE_A (\lambda) \right)
      X_0\\
      & = \int e^{it \lambda}  \hspace{0.17em} dE_A (\lambda) X_0\\
      & = \int e^{it \lambda}  \hspace{0.17em} dZ (\lambda)
    \end{array}
  \end{equation}
  \tmtextbf{Step 5: Variance of $Z (B)$.}
  \begin{equation}
    \begin{array}{ll}
      \E [|Z (B) |^2] & = \ip{E_A (B) X_0}{E_A (B) X_0} = \ip{E_A (B)
      X_0}{X_0}\\
      & = \int_B 1 \hspace{0.17em} d \ip{E_A (\lambda) X_0}{X_0} = \int_B
      \hspace{0.17em} d \mu_0 (\lambda)
    \end{array}
  \end{equation}
  where $\mu_0 (B) \assign \ip{E_A (B) X_0}{X_0}$. To identify $\mu_0$ with
  the spectral distribution $F$, note that
  \begin{equation}
    \begin{array}{ll}
      \gamma (t) & = \E [X_t \overline{X_0}] = \ip{T_t X_0}{X_0} = \int e^{it
      \lambda}  \hspace{0.17em} d \ip{E_A (\lambda) X_0}{X_0} = \int e^{it
      \lambda}  \hspace{0.17em} d \mu_0 (\lambda)
    \end{array}
  \end{equation}
  By uniqueness in Bochner's theorem, $\mu_0 = \mu$, so
  \begin{equation}
    F (B) = \mu_0 (B) = \E [|Z (B) |^2]
  \end{equation}
\end{proof}

\begin{corollary}
  [Real-Valued Process Representation]\label{cor:real_process} If $(X_t)$ is
  real-valued and weakly stationary, the spectral measure $F$ is symmetric: $F
  (- B) = F (B)$ for Borel $B$, and there exist real-valued orthogonal random
  measures $U, V$ such that
  \begin{equation}
    X_t = \int_{\R} \cos (t \lambda)  \hspace{0.17em} dU (\lambda) + \int_{\R}
    \sin (t \lambda)  \hspace{0.17em} dV (\lambda)
  \end{equation}
  with $\E [U (B_1) U (B_2)] = \E [V (B_1) V (B_2)] = 2 F (B_1 \cap B_2)$ and
  $\E [U (B_1) V (B_2)] = 0$.
\end{corollary}

\begin{proof}
  For real $(X_t)$, $X_t = \overline{X_t}$, so
  \begin{equation}
    \int e^{it \lambda}  \hspace{0.17em} dZ (\lambda) = \int e^{- it \lambda} 
    \hspace{0.17em} d \overline{Z (\lambda)}
  \end{equation}
  Changing variables $\lambda \to - \lambda$ in the right-hand side,
  \begin{equation}
    \int e^{it \lambda}  \hspace{0.17em} dZ (\lambda) = \int e^{it \lambda} 
    \hspace{0.17em} d \overline{Z (- \lambda)}
  \end{equation}
  By uniqueness of the spectral representation, $Z (B) = \overline{Z (- B)}$
  for all $B$. Define
  \begin{equation}
    U (B) \assign Z (B) + \overline{Z (- B)}
  \end{equation}
  \begin{equation}
    V (B) \assign i (Z (B) - \overline{Z (- B)})
  \end{equation}
  These are real-valued. Then $Z (B) = \frac{1}{2}  (U (B) - iV (B))$, so
  \begin{equation}
    \begin{array}{ll}
      X_t & = \int e^{it \lambda}  \hspace{0.17em} d \left( \frac{U (\lambda)
      - iV (\lambda)}{2} \right)\\
      & = \int \frac{(\cos (t \lambda) + i \sin (t \lambda))  (dU (\lambda) -
      i \hspace{0.17em} dV (\lambda))}{2}\\
      & = \int \cos (t \lambda)  \hspace{0.17em} dU (\lambda) + \int \sin (t
      \lambda)  \hspace{0.17em} dV (\lambda)
    \end{array}
  \end{equation}
  The variance formulas follow from $\E [|Z (B) |^2] = F (B)$ and the
  definitions of $U, V$.
\end{proof}

\section{Conclusion}

We have provided complete, detailed proofs of the spectral theorem for both
bounded and unbounded self-adjoint operators via projection-valued measures,
Stone's theorem establishing the correspondence between strongly continuous
unitary groups and self-adjoint generators, Bochner's theorem characterizing
positive-definite functions as Fourier transforms of measures, Nelson's
theorem on essential self-adjointness via analytic vectors, and Cram{\'e}r's
spectral representation theorem for weakly stationary stochastic processes.
These results form the foundation for spectral analysis in functional
analysis, quantum mechanics, stochastic process theory, and partial
differential equations.

\end{document}
