\documentclass{article}
\usepackage{amsmath}
\usepackage{amsthm}
\usepackage{amssymb}

\newtheorem{theorem}{Theorem}
\newtheorem{definition}{Definition}
\newtheorem{proposition}{Proposition}
\newtheorem{corollary}{Corollary}

\title{Generation of Oscillatory Processes via Unitary Time-Changed Stationary Processes}
\author{}
\date{}

\begin{document}

\maketitle

\begin{definition}[Time Change Function]
Let $T(t)$ be a function that is strictly increasing except on a set of Lebesgue measure zero.
\end{definition}

\begin{definition}[Stationary Gaussian Process]
Let $X(t)$ be a stationary Gaussian process with spectral representation:
\[
X(t) = \int_{-\infty}^{\infty} e^{i\lambda t} \, d\Phi(\lambda)
\]
where $d\Phi(\lambda)$ is an orthogonal increment process and $S(\lambda)$ is the spectral density of $X(t)$.
\end{definition}

\begin{definition}[Unitarity-Preserving Time Change]
The unitarily time-changed process is defined as:
\[
Z(t) = \sqrt{T'(t)} \cdot X(T(t))
\]
where $T'(t)$ is the derivative of $T(t)$.
\end{definition}

\begin{theorem}[Oscillatory Function Representation]
The oscillatory function for the unitarily time-changed process is:
\[
\phi_t(\lambda) = \sqrt{T'(t)} e^{i\lambda T(t)}
\]
\end{theorem}

\begin{proof}
Substituting the spectral representation of $X(t)$ into the definition of $Z(t)$:
\begin{align}
Z(t) &= \sqrt{T'(t)} \cdot X(T(t)) \\
&= \sqrt{T'(t)} \int_{-\infty}^{\infty} e^{i\lambda T(t)} \, d\Phi(\lambda) \\
&= \int_{-\infty}^{\infty} \sqrt{T'(t)} e^{i\lambda T(t)} \, d\Phi(\lambda) \\
&= \int_{-\infty}^{\infty} \phi_t(\lambda) \, d\Phi(\lambda)
\end{align}
\end{proof}

\begin{theorem}[Priestley Gain Function]
The gain function in Priestley's oscillatory process representation is:
\[
A(t,\lambda) = \sqrt{T'(t)} e^{i\lambda(T(t)-t)}
\]
\end{theorem}

\begin{proof}
The Priestley representation requires:
\[
Z(t) = \int_{-\infty}^{\infty} A(t,\lambda) e^{i\lambda t} \, d\Phi(\lambda)
\]
Comparing with the oscillatory function representation:
\[
\int_{-\infty}^{\infty} \phi_t(\lambda) \, d\Phi(\lambda) = \int_{-\infty}^{\infty} A(t,\lambda) e^{i\lambda t} \, d\Phi(\lambda)
\]
Therefore:
\[
\phi_t(\lambda) = A(t,\lambda) e^{i\lambda t}
\]
Solving for $A(t,\lambda)$:
\begin{align}
A(t,\lambda) &= \phi_t(\lambda) e^{-i\lambda t} \\
&= \sqrt{T'(t)} e^{i\lambda T(t)} e^{-i\lambda t} \\
&= \sqrt{T'(t)} e^{i\lambda(T(t)-t)}
\end{align}
\end{proof}

\begin{theorem}[Covariance Kernel]
The covariance kernel $R(s,t)$ of the unitarily time-changed process $Z(t)$ is:
\[
R(s,t) = \sqrt{T'(s)T'(t)} \int_{-\infty}^{\infty} e^{i\lambda(T(s)-T(t))} S(\lambda) \, d\lambda
\]
\end{theorem}

\begin{proof}
The covariance kernel is defined as:
\[
R(s,t) = \text{Cov}[Z(s), Z(t)] = E[Z(s) \overline{Z(t)}]
\]
Using the oscillatory function representation:
\begin{align}
R(s,t) &= E\left[\int_{-\infty}^{\infty} \phi_s(\lambda) \, d\Phi(\lambda) \int_{-\infty}^{\infty} \overline{\phi_t(\mu)} \, \overline{d\Phi(\mu)}\right] \\
&= \int_{-\infty}^{\infty} \phi_s(\lambda) \overline{\phi_t(\lambda)} S(\lambda) \, d\lambda \\
&= \int_{-\infty}^{\infty} \sqrt{T'(s)} e^{i\lambda T(s)} \sqrt{T'(t)} e^{-i\lambda T(t)} S(\lambda) \, d\lambda \\
&= \sqrt{T'(s)T'(t)} \int_{-\infty}^{\infty} e^{i\lambda(T(s)-T(t))} S(\lambda) \, d\lambda
\end{align}
\end{proof}

\begin{theorem}[Variance Function]
The variance function of the unitarily time-changed process is:
\[
\text{Var}[Z(t)] = T'(t) \cdot \sigma_X^2
\]
where $\sigma_X^2 = \int_{-\infty}^{\infty} S(\lambda) \, d\lambda$.
\end{theorem}

\begin{proof}
Setting $s = t$ in the covariance kernel:
\begin{align}
\text{Var}[Z(t)] &= R(t,t) \\
&= \sqrt{T'(t)T'(t)} \int_{-\infty}^{\infty} e^{i\lambda(T(t)-T(t))} S(\lambda) \, d\lambda \\
&= T'(t) \int_{-\infty}^{\infty} e^{i\lambda \cdot 0} S(\lambda) \, d\lambda \\
&= T'(t) \int_{-\infty}^{\infty} S(\lambda) \, d\lambda \\
&= T'(t) \cdot \sigma_X^2
\end{align}
\end{proof}

\begin{theorem}[Time-Dependent Spectral Density]
The time-dependent spectral density (evolutionary spectral density) is:
\[
S_Z(t,\lambda) = T'(t) \cdot S(\lambda)
\]
\end{theorem}

\begin{proof}
The time-dependent spectral density is defined as:
\[
S_Z(t,\lambda) = |A(t,\lambda)|^2 \cdot S(\lambda)
\]
Computing the modulus squared of the gain function:
\begin{align}
|A(t,\lambda)|^2 &= \left|\sqrt{T'(t)} e^{i\lambda(T(t)-t)}\right|^2 \\
&= \left|\sqrt{T'(t)}\right|^2 \left|e^{i\lambda(T(t)-t)}\right|^2 \\
&= T'(t) \cdot 1 \\
&= T'(t)
\end{align}
Therefore:
\[
S_Z(t,\lambda) = T'(t) \cdot S(\lambda)
\]
\end{proof}

\begin{theorem}[Expected Zero Count]
The expected zero count of the unitarily time-changed process $Z(t)$ in interval $[a,b]$ is:
\[
E[N_Z[a,b]] = \rho_X \cdot (T(b) - T(a))
\]
where $\rho_X = \frac{1}{\pi} \sqrt{\frac{-R_X''(0)}{R_X(0)}}$ is the constant zero-crossing rate of the stationary process $X(t)$.
\end{theorem}

\begin{proof}
The zeros of $Z(t)$ occur when:
\[
Z(t) = \sqrt{T'(t)} \cdot X(T(t)) = 0
\]
Since $\sqrt{T'(t)} > 0$ almost everywhere, this occurs if and only if:
\[
X(T(t)) = 0
\]
The interval $[a,b]$ in the $Z$-process corresponds to the interval $[T(a), T(b)]$ in the $X$-process. The expected number of zeros of the stationary process $X$ in an interval of length $T(b) - T(a)$ is:
\[
E[N_X[T(a), T(b)]] = \rho_X \cdot (T(b) - T(a))
\]
Since the zeros of $Z$ in $[a,b]$ correspond exactly to the zeros of $X$ in $[T(a), T(b)]$:
\[
E[N_Z[a,b]] = E[N_X[T(a), T(b)]] = \rho_X \cdot (T(b) - T(a))
\]
\end{proof}

\begin{proposition}[Non-Stationarity]
The unitarily time-changed process $Z(t)$ is non-stationary.
\end{proposition}

\begin{proof}
For stationarity, the covariance function must depend only on the lag $s-t$. However, from the covariance kernel:
\[
R(s,t) = \sqrt{T'(s)T'(t)} \int_{-\infty}^{\infty} e^{i\lambda(T(s)-T(t))} S(\lambda) \, d\lambda
\]
The dependence on $T'(s)$ and $T'(t)$ individually, rather than through their difference, demonstrates that the process is non-stationary unless $T'(t)$ is constant (which would make $T(t)$ linear).
\end{proof}

\end{document}
