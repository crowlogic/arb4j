\documentclass{article}
\usepackage[english]{babel}
\usepackage{geometry,amsmath,latexsym}
\geometry{letterpaper}

%%%%%%%%%% Start TeXmacs macros
\newcommand{\tmaffiliation}[1]{\\ #1}
\newenvironment{proof}{\noindent\textbf{Proof\ }}{\hspace*{\fill}$\Box$\medskip}
\newtheorem{definition}{Definition}
\newtheorem{theorem}{Theorem}
%%%%%%%%%% End TeXmacs macros

\begin{document}

\title{Generation of Oscillatory Processes via Unitarily Time-Changed
Stationary Processes}

\author{
  Stephen Crowley
  \tmaffiliation{August 21, 2025}
}

\date{}

\maketitle

\begin{definition}
  [Time Change Function] Let $T (t)$ be a function that is strictly increasing
  except possibly on a set of Lebesgue measure zero.
\end{definition}

\begin{definition}
  [Stationary Gaussian Process] Let $X (t)$ be a stationary Gaussian process
  with spectral representation:
  \begin{equation}
    X (t) = \int_{- \infty}^{\infty} e^{i \lambda t}  \hspace{0.17em} d \Phi
    (\lambda)
  \end{equation}
  where $d \Phi (\lambda)$ is an orthogonal increment process and $S
  (\lambda)$ is the spectral density of $X (t)$.
\end{definition}

\begin{definition}
  [Unitarity-Preserving Time Change] The unitarily time-changed process is
  defined as:
  \begin{equation}
    Z (t) = \sqrt{T' (t)} \cdot X (T (t))
  \end{equation}
  where $T' (t)$ is the derivative of $T (t)$.
\end{definition}

\begin{theorem}
  [Oscillatory Function Representation] The oscillatory function for the
  unitarily time-changed process is:
  \begin{equation}
    \phi_t (\lambda) = \sqrt{T' (t)} e^{i \lambda T (t)}
  \end{equation}
\end{theorem}

\begin{proof}
  Substituting the spectral representation of $X (t)$ into the definition of
  $Z (t)$:
  
  \begin{align}
    Z (t) & = \sqrt{T' (t)} \cdot X (T (t)) \\
    & = \sqrt{T' (t)}  \int_{- \infty}^{\infty} e^{i \lambda T (t)} 
    \hspace{0.17em} d \Phi (\lambda) \\
    & = \int_{- \infty}^{\infty} \sqrt{T' (t)} e^{i \lambda T (t)} 
    \hspace{0.17em} d \Phi (\lambda) \\
    & = \int_{- \infty}^{\infty} \phi_t (\lambda)  \hspace{0.17em} d \Phi
    (\lambda) 
  \end{align}
\end{proof}

\begin{theorem}
  [Priestley Gain Function] The gain function in Priestley's oscillatory
  process representation is:
  \begin{equation}
    A (t, \lambda) = \sqrt{T' (t)} e^{i \lambda (T (t) - t)}
  \end{equation}
\end{theorem}

\begin{proof}
  The Priestley representation requires:
  \begin{equation}
    Z (t) = \int_{- \infty}^{\infty} A (t, \lambda) e^{i \lambda t} 
    \hspace{0.17em} d \Phi (\lambda)
  \end{equation}
  Comparing with the oscillatory function representation:
  \begin{equation}
    \int_{- \infty}^{\infty} \phi_t (\lambda)  \hspace{0.17em} d \Phi
    (\lambda) = \int_{- \infty}^{\infty} A (t, \lambda) e^{i \lambda t} 
    \hspace{0.17em} d \Phi (\lambda)
  \end{equation}
  Therefore:
  \begin{equation}
    \phi_t (\lambda) = A (t, \lambda) e^{i \lambda t}
  \end{equation}
  Solving for $A (t, \lambda)$:
  
  \begin{align}
    A (t, \lambda) & = \phi_t (\lambda) e^{- i \lambda t} \\
    & = \sqrt{T' (t)} e^{i \lambda T (t)} e^{- i \lambda t} \\
    & = \sqrt{T' (t)} e^{i \lambda (T (t) - t)} 
  \end{align}
\end{proof}

\begin{theorem}
  [Covariance Kernel] The covariance kernel $R (s, t)$ of the unitarily
  time-changed process $Z (t)$ is:
  \[ R (s, t) = \sqrt{T' (s) T' (t)}  \int_{- \infty}^{\infty} e^{i \lambda (T
     (s) - T (t))} S (\lambda)  \hspace{0.17em} d \lambda \]
\end{theorem}

\begin{proof}
  The covariance kernel is defined as:
  \begin{equation}
    R (s, t) = \text{Cov} [Z (s), Z (t)] = E [Z (s) \overline{Z (t)}]
  \end{equation}
  Using the oscillatory function representation:
  
  \begin{align}
    R (s, t) & = E \left[ \int_{- \infty}^{\infty} \phi_s (\lambda)
    \hspace{0.17em} d \Phi (\lambda) \int_{- \infty}^{\infty} \overline{\phi_t
    (\mu)} \hspace{0.17em} \overline{d \Phi (\mu)} \right] \\
    & = \int_{- \infty}^{\infty} \phi_s (\lambda) \overline{\phi_t (\lambda)}
    S (\lambda)  \hspace{0.17em} d \lambda \\
    & = \int_{- \infty}^{\infty} \sqrt{T' (s)} e^{i \lambda T (s)}  \sqrt{T'
    (t)} e^{- i \lambda T (t)} S (\lambda)  \hspace{0.17em} d \lambda \\
    & = \sqrt{T' (s) T' (t)}  \int_{- \infty}^{\infty} e^{i \lambda (T (s) -
    T (t))} S (\lambda)  \hspace{0.17em} d \lambda 
  \end{align}
\end{proof}

\begin{theorem}
  [Variance Function] The variance function of the unitarily time-changed
  process is:
  \[ \sigma_Z^2 (t) = \text{Var} [Z (t)] = T' (t) \cdot \sigma_X^2 \]
  where $\sigma_X^2 = \int_{- \infty}^{\infty} S (\lambda)  \hspace{0.17em} d
  \lambda$.
\end{theorem}

\begin{proof}
  Setting $s = t$ in the covariance kernel:
  
  \begin{align}
    \sigma_Z^2 (t) & = R (t, t) \\
    & = \sqrt{T' (t) T' (t)}  \int_{- \infty}^{\infty} e^{i \lambda (T (t) -
    T (t))} S (\lambda)  \hspace{0.17em} d \lambda \\
    & = T' (t)  \int_{- \infty}^{\infty} e^{i \lambda \cdot 0} S (\lambda) 
    \hspace{0.17em} d \lambda \\
    & = T' (t)  \int_{- \infty}^{\infty} S (\lambda)  \hspace{0.17em} d
    \lambda \\
    & = T' (t) \cdot \sigma_X^2 
  \end{align}
\end{proof}

\begin{theorem}
  [Time-Dependent Spectral Density] The time-dependent spectral density
  (evolutionary spectral density) is:
  \[ S_Z (t, \lambda) = T' (t) \cdot S (\lambda) \]
\end{theorem}

\begin{proof}
  The time-dependent spectral density is defined as:
  \begin{equation}
    S_Z (t, \lambda) = |A (t, \lambda) |^2 \cdot S (\lambda)
  \end{equation}
  Computing the modulus squared of the gain function:
  
  \begin{align}
    |A (t, \lambda) |^2 & = \left| \sqrt{T' (t)} e^{i \lambda (T (t) - t)}
    \right|^2 \\
    & = \left| \sqrt{T' (t)} \right|^2 | e^{i \lambda (T (t) - t)} |^2 \\
    & = T' (t) \cdot 1 \\
    & = T' (t) 
  \end{align}
  
  Therefore:
  \begin{equation}
    S_Z (t, \lambda) = T' (t) \cdot S (\lambda)
  \end{equation}
\end{proof}

\begin{theorem}
  [Expected Zero Count via Kac-Rice Formula] The expected zero count of the
  unitarily time-changed process $Z (t)$ in the interval $[a, b]$ is:
  \begin{equation}
    E [N_Z [a, b]] = \rho_X \cdot (T (b) - T (a))
  \end{equation}
  where
  \begin{equation}
    \rho_X = \frac{1}{\pi}  \sqrt{\frac{- R_X'' (0)}{\sigma_X^2}}
  \end{equation}
  is the constant zero-crossing rate of the stationary process $X (t)$.
\end{theorem}

\begin{proof}
  TODO: merge with other paper where I stated this most excellently 
\end{proof}

\end{document}
