\documentclass{article}
\usepackage{amsmath,amsthm,amssymb}
\newtheorem{theorem}{Theorem}
\newtheorem{lemma}[theorem]{Lemma}
\newtheorem{proposition}[theorem]{Proposition}
\newtheorem{corollary}[theorem]{Corollary}

\begin{document}

\title{Complete Theory of Spectral Factorization and RKHS Expansions}
\author{}
\maketitle

\begin{theorem}[Spectral Factorization]
Let $K(t,s)$ be a positive definite stationary kernel. Then there exists a spectral density $S(\omega)$ and spectral factor:
\begin{equation}
    h(t) = \frac{1}{2\pi} \int_{-\infty}^{\infty} \sqrt{S(\omega)} e^{i\omega t} d\omega
\end{equation}
such that:
\begin{equation}
    K(t,s) = \int_{-\infty}^{\infty} h(t+\tau)\overline{h(s+\tau)}d\tau
\end{equation}
\end{theorem}

\begin{proof}
1. By Bochner's theorem, since $K$ is positive definite and stationary:
\begin{equation}
    K(t-s) = \frac{1}{2\pi} \int_{-\infty}^{\infty} S(\omega)e^{i\omega(t-s)}d\omega
\end{equation}
where $S(\omega) \geq 0$ is the spectral density.

2. Define $h(t)$ as stated. Then:
\begin{align*}
    \int_{-\infty}^{\infty} h(t+\tau)\overline{h(s+\tau)}d\tau &= \int_{-\infty}^{\infty} \frac{1}{2\pi} \int_{-\infty}^{\infty} \sqrt{S(\omega_1)} e^{i\omega_1(t+\tau)}d\omega_1 \\
    &\quad \cdot \frac{1}{2\pi} \int_{-\infty}^{\infty} \sqrt{S(\omega_2)} e^{-i\omega_2(s+\tau)}d\omega_2 d\tau
\end{align*}

3. Rearranging integrals (justified by Fubini's theorem since $S(\omega) \geq 0$):
\begin{align*}
    &= \frac{1}{4\pi^2} \int_{-\infty}^{\infty}\int_{-\infty}^{\infty} \sqrt{S(\omega_1)S(\omega_2)} e^{i\omega_1t}e^{-i\omega_2s} \\
    &\quad \cdot \int_{-\infty}^{\infty} e^{i(\omega_1-\omega_2)\tau}d\tau d\omega_1d\omega_2
\end{align*}

4. The inner integral gives $2\pi\delta(\omega_1-\omega_2)$:
\begin{align*}
    &= \frac{1}{2\pi} \int_{-\infty}^{\infty} S(\omega)e^{i\omega(t-s)}d\omega = K(t-s)
\end{align*}
\end{proof}

\begin{lemma}[Convolution Properties]
For any orthonormal basis $\{e_i\}$ of $L^2(\mathbb{R})$, define:
\begin{equation}
    h_i(t) = \int_{-\infty}^{\infty} h(t+\tau)\overline{e_i(\tau)}d\tau
\end{equation}
Then:
\begin{equation}
    \langle h_i, h_j \rangle = \int_{-\infty}^{\infty}\int_{-\infty}^{\infty} K(\tau_1-\tau_2)\overline{e_i(\tau_1)}e_j(\tau_2)d\tau_1d\tau_2
\end{equation}
\end{lemma}

\begin{proof}
1. Expand the inner product:
\begin{align*}
    \langle h_i, h_j \rangle &= \int_{-\infty}^{\infty} h_i(t)\overline{h_j(t)}dt \\
    &= \int_{-\infty}^{\infty} \int_{-\infty}^{\infty}\int_{-\infty}^{\infty} h(t+\tau_1)\overline{h(t+\tau_2)}\overline{e_i(\tau_1)}e_j(\tau_2)d\tau_1d\tau_2dt
\end{align*}

2. Change variables $u = t+\tau_1$:
\begin{align*}
    &= \int_{-\infty}^{\infty}\int_{-\infty}^{\infty}\int_{-\infty}^{\infty} h(u)\overline{h(u-\tau_1+\tau_2)}\overline{e_i(\tau_1)}e_j(\tau_2)d\tau_1d\tau_2du
\end{align*}

3. The inner integral gives $K(\tau_1-\tau_2)$ by definition of $h$:
\begin{equation*}
    = \int_{-\infty}^{\infty}\int_{-\infty}^{\infty} K(\tau_1-\tau_2)\overline{e_i(\tau_1)}e_j(\tau_2)d\tau_1d\tau_2
\end{equation*}
\end{proof}

\begin{proposition}[Orthonormal Basis Construction]
The normalized functions:
\begin{equation}
    \psi_i(t) = \frac{h_i(t)}{\sqrt{|h_i|^2}}
\end{equation}
form an orthonormal basis for the RKHS $\mathcal{H}$.
\end{proposition}

\begin{proof}
1. First verify orthonormality. For any $i,j$:
\begin{align*}
    \langle \psi_i, \psi_j \rangle &= \left\langle \frac{h_i}{\sqrt{|h_i|^2}}, \frac{h_j}{\sqrt{|h_j|^2}} \right\rangle \\
    &= \frac{1}{\sqrt{|h_i|^2|h_j|^2}} \int_{-\infty}^{\infty} h_i(t)\overline{h_j(t)}dt \\
    &= \frac{\langle h_i, h_j \rangle}{\sqrt{|h_i|^2|h_j|^2}} = \delta_{ij}
\end{align*}

2. For completeness, show any $f \in \mathcal{H}$ can be expanded:
\begin{equation*}
    f(t) = \sum_i \langle f, \psi_i \rangle \psi_i(t)
\end{equation*}

3. This follows from completeness of $\{e_i\}$ in $L^2$ and the fact that $h$ maps $L^2$ to $\mathcal{H}$.
\end{proof}

\begin{theorem}[Kernel Expansion]
The kernel has the expansion:
\begin{equation}
    K(t,s) = \sum_i h_i(t)h_i(s)
\end{equation}
which converges in the RKHS norm.
\end{theorem}

\begin{proof}
1. Start with normalized basis expansion:
\begin{equation*}
    K(t,s) = \sum_i |h_i|^2 \psi_i(t)\psi_i(s)
\end{equation*}

2. Substitute normalized basis functions:
\begin{align*}
    &= \sum_i |h_i|^2 \frac{h_i(t)}{\sqrt{|h_i|^2}} \frac{h_i(s)}{\sqrt{|h_i|^2}} \\
    &= \sum_i h_i(t)h_i(s)
\end{align*}

3. Expand using convolution definition:
\begin{align*}
    &= \sum_i \int_{-\infty}^{\infty}\int_{-\infty}^{\infty} h(t+\tau_1)h(s+\tau_2)\overline{e_i(\tau_1)}e_i(\tau_2)d\tau_1d\tau_2
\end{align*}

4. By completeness of $\{e_i\}$:
\begin{equation*}
    = \int_{-\infty}^{\infty} h(t+\tau)\overline{h(s+\tau)}d\tau = K(t,s)
\end{equation*}
\end{proof}

\begin{corollary}[Basis Independence]
The kernel expansion is independent of the choice of orthonormal basis $\{e_i\}$.
\end{corollary}

\begin{proof}
Let $\{e_i\}$ and $\{f_i\}$ be two orthonormal bases of $L^2(\mathbb{R})$. The expansions:
\begin{align*}
    K(t,s) &= \sum_i h_i^e(t)h_i^e(s) \\
    K(t,s) &= \sum_i h_i^f(t)h_i^f(s)
\end{align*}
where superscripts indicate the basis used, must be equal as they both equal:
\begin{equation*}
    \int_{-\infty}^{\infty} h(t+\tau)\overline{h(s+\tau)}d\tau
\end{equation*}
by the previous theorem.
\end{proof}

\end{document}
