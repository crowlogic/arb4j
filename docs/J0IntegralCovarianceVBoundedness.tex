\documentclass{article}
\usepackage{amsmath}
\usepackage{amssymb}
\usepackage{amsthm}

\newtheorem{theorem}{Theorem}
\newtheorem{lemma}[theorem]{Lemma}
\newtheorem{definition}[theorem]{Definition}
\newtheorem{remark}[theorem]{Remark}

\title{Compactness of the $J_0$ Integral Covariance Operator}
\author{}
\date{}

\begin{document}

\maketitle

\section{Introduction}

We consider the integral operator $T$ on $L^2[0,\infty)$ defined by:

\[(Tf)(x) = \int_0^\infty J_0(|x-y|)f(y)dy\]

where $J_0$ is the Bessel function of the first kind of order zero. We aim to prove that $T$ is compact using the concept of Bochner V-boundedness.

\section{Preliminaries}

\begin{lemma}
For $x \neq 0$, $|J_0(x)| \leq \min(1, \sqrt{2/(\pi|x|)})$.
\end{lemma}

\begin{proof}
This follows from the asymptotic behavior of $J_0(x)$ and its maximum value of 1 at $x = 0$.
\end{proof}

\begin{lemma}
\label{lem:integral}
For any $\epsilon > 0$, the integral $\int_0^\infty (J_0(x)/(\epsilon + x))^2 dx$ converges and is equal to:
\[\frac{1}{\pi^{3/2}\epsilon^2} G_{2,4}^{2,1}\left({\epsilon^2}\left|
\begin{array}{c}
 1,1 \\
 \frac{3}{2},1,\frac{1}{2},\frac{1}{2} \\
\end{array}
\right.\right)\]
where $G$ is the Meijer G-function. Moreover, $\epsilon = 0$ is the abscissa of convergence for this integral.
\end{lemma}

\begin{proof}
The convergence and exact value can be derived using complex analysis techniques and properties of special functions. The fact that $\epsilon = 0$ is the abscissa of convergence follows from the divergence of the integral when $\epsilon = 0$, due to the behavior of $J_0(x)$ near $x = 0$ and its slow decay as $x \to \infty$.
\end{proof}

\section{Bochner V-boundedness}

\begin{definition}
An integral operator $T$ with kernel $K(x,y)$ is Bochner V-bounded if there exists a positive function $V(x)$ such that:
\[\int_0^\infty \sup_{y\geq0} |K(x,y)/V(y)|^2 V(x)^2 dx < \infty\]
\end{definition}

\begin{theorem}
If $T$ is Bochner V-bounded on $L^2[0,\infty)$, then $T$ is compact.
\end{theorem}

\begin{proof}
Let $\{e_n\}$ be an orthonormal basis for $L^2[0,\infty)$. Define the finite rank operators:

\[T_N f = \sum_{n=1}^N \langle Tf, e_n \rangle e_n\]

We will show that $T_N \to T$ in operator norm. Let $f \in L^2[0,\infty)$ with $\|f\| \leq 1$. Then:

\begin{align*}
\|(T-T_N)f\|^2 &= \sum_{n>N} |\langle Tf, e_n \rangle|^2 \\
&= \sum_{n>N} \left|\int_0^\infty \int_0^\infty K(x,y)f(y)e_n(x) dy dx\right|^2 \\
&\leq \sum_{n>N} \left(\int_0^\infty \int_0^\infty |K(x,y)/V(y)| |V(y)f(y)| |e_n(x)| dy dx\right)^2 \\
&\leq \sum_{n>N} \left(\int_0^\infty \sup_{y\geq0} |K(x,y)/V(y)| \|V f\| |e_n(x)| dx\right)^2 \\
&\leq \|V f\|^2 \sum_{n>N} \int_0^\infty \sup_{y\geq0} |K(x,y)/V(y)|^2 |e_n(x)|^2 dx \\
&= \|V f\|^2 \int_0^\infty \sup_{y\geq0} |K(x,y)/V(y)|^2 \sum_{n>N} |e_n(x)|^2 dx
\end{align*}

By Parseval's identity, for any fixed $x$, $\sum_{n=1}^\infty |e_n(x)|^2 = 1$ almost everywhere. Therefore, $\sum_{n>N} |e_n(x)|^2$ represents the tail of this series and converges to zero pointwise as $N \to \infty$ for almost every $x$. This sum is also bounded by 1 for all $N$ and $x$.

By the dominated convergence theorem and the Bochner V-boundedness condition, $\|(T-T_N)f\|^2 \to 0$ as $N \to \infty$, uniformly for $\|f\| \leq 1$. Thus, $T$ is the limit of finite rank operators and is therefore compact.
\end{proof}

\section{Proof of Compactness}

We will show that $T$ is Bochner V-bounded with $V(x) = \epsilon + x$ for any $\epsilon > 0$.

\begin{theorem}
The operator $T$ defined by $(Tf)(x) = \int_0^\infty J_0(|x-y|)f(y)dy$ is compact on $L^2[0,\infty)$.
\end{theorem}

\begin{proof}
We need to show:
\[\int_0^\infty \sup_{y\geq0} |J_0(|x-y|)/(\epsilon+y)|^2 (\epsilon+x)^2 dx < \infty\]

Using the result from Lemma \ref{lem:integral}, we have for any $\epsilon > 0$:

\[\int_0^\infty |J_0(x)/(\epsilon+x)|^2 dx = \frac{1}{\pi^{3/2}\epsilon^2} G_{2,4}^{2,1}\left({\epsilon^2}\left|
\begin{array}{c}
 1,1 \\
 \frac{3}{2},1,\frac{1}{2},\frac{1}{2} \\
\end{array}
\right.\right) < \infty\]

Now, observe that:
\[\sup_{y\geq0} |J_0(|x-y|)/(\epsilon+y)|^2 \leq |J_0(x)/(\epsilon+x)|^2\]

Therefore,

\begin{align*}
\int_0^\infty \sup_{y\geq0} |J_0(|x-y|)/(\epsilon+y)|^2 (\epsilon+x)^2 dx 
&\leq \int_0^\infty |J_0(x)/(\epsilon+x)|^2 (\epsilon+x)^2 dx \\
&= \int_0^\infty |J_0(x)/(\epsilon+x)|^2 dx + \int_0^\infty |J_0(x)/(\epsilon+x)|^2 (2\epsilon x+x^2) dx \\
&< \infty + C\int_0^\infty (2\epsilon x+x^2)/(\epsilon+x)^2 dx \\
&< \infty
\end{align*}

where $C$ is a constant. The last integral converges because the integrand behaves as $O(1)$ for large $x$.

This proves that $T$ is Bochner V-bounded with $V(x) = \epsilon + x$ for any $\epsilon > 0$, and therefore compact.
\end{proof}

\begin{remark}
The choice of $V(x) = \epsilon + x$ for any $\epsilon > 0$ is sufficient to prove compactness. The fact that $\epsilon = 0$ is the abscissa of convergence for the integral in Lemma \ref{lem:integral} shows that this is the optimal family of functions for establishing the Bochner V-boundedness of $T$.
\end{remark}

\end{document}
