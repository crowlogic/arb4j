\documentclass{article}
\usepackage{amsmath}
\usepackage{amssymb}
\usepackage{amsthm}

\newtheorem{theorem}{Theorem}
\newtheorem{lemma}[theorem]{Lemma}
\newtheorem{definition}[theorem]{Definition}
\newtheorem{remark}[theorem]{Remark}

\title{Compactness of the $J_0$ Integral Covariance Operator}
\author{}
\date{}

\begin{document}

\maketitle

\section{Introduction}

We consider the integral operator $T$ on $L^2[0,\infty)$ defined by:

\[(Tf)(x) = \int_0^\infty J_0(|x-y|)f(y)dy\]

where $J_0$ is the Bessel function of the first kind of order zero. We aim to prove that $T$ is compact using the concept of Bochner V-boundedness.

\section{Preliminaries}

\begin{lemma}
For $x \neq 0$, $|J_0(x)| \leq \min(1, \sqrt{2/(\pi|x|)})$.
\end{lemma}

\begin{proof}
This follows from the asymptotic behavior of $J_0(x)$ and its maximum value of 1 at $x = 0$.
\end{proof}

\begin{lemma}
\label{lem:integral}
The integral $\int_0^\infty (J_0(x)/(1+x))^2 dx$ converges.
\end{lemma}

\begin{proof}
We split the integral into two parts:

\[\int_0^\infty \left(\frac{J_0(x)}{1+x}\right)^2 dx = \int_0^1 \left(\frac{J_0(x)}{1+x}\right)^2 dx + \int_1^\infty \left(\frac{J_0(x)}{1+x}\right)^2 dx\]

For the first part, since $|J_0(x)| \leq 1$:

\[\int_0^1 \left(\frac{J_0(x)}{1+x}\right)^2 dx \leq \int_0^1 \frac{1}{(1+x)^2} dx = \frac{1}{2} < \infty\]

For the second part, we use the asymptotic behavior $|J_0(x)| \leq \sqrt{2/(\pi x)}$ for $x > 1$:

\[\int_1^\infty \left(\frac{J_0(x)}{1+x}\right)^2 dx \leq \int_1^\infty \frac{2}{\pi x(1+x)^2} dx\]

We can directly evaluate this integral:

\begin{align*}
\int_1^\infty \frac{2}{\pi x(1+x)^2} dx &= \frac{2}{\pi} \left[ -\frac{1}{1+x} + \log\left(\frac{x}{1+x}\right) \right]_1^\infty \\
&= \frac{2}{\pi} \left[ \frac{1}{2} + \log\left(\frac{2}{1}\right) \right] < \infty
\end{align*}

Therefore, the entire integral converges.
\end{proof}

\section{Bochner V-boundedness}

\begin{definition}
An integral operator $T$ with kernel $K(x,y)$ is Bochner V-bounded if there exists a positive function $V(x)$ such that:
\[\int_0^\infty \sup_{y\geq0} |K(x,y)/V(y)|^2 V(x)^2 dx < \infty\]
\end{definition}

\begin{theorem}
If $T$ is Bochner V-bounded on $L^2[0,\infty)$, then $T$ is compact.
\end{theorem}

\begin{proof}
Let $\{e_n\}$ be an orthonormal basis for $L^2[0,\infty)$. Define the finite rank operators:

\[T_N f = \sum_{n=1}^N \langle Tf, e_n \rangle e_n\]

We will show that $T_N \to T$ in operator norm. Let $f \in L^2[0,\infty)$ with $\|f\| \leq 1$. Then:

\begin{align*}
\|(T-T_N)f\|^2 &= \sum_{n>N} |\langle Tf, e_n \rangle|^2 \\
&= \sum_{n>N} \left|\int_0^\infty \int_0^\infty K(x,y)f(y)e_n(x) dy dx\right|^2 \\
&\leq \sum_{n>N} \left(\int_0^\infty \int_0^\infty |K(x,y)/V(y)| |V(y)f(y)| |e_n(x)| dy dx\right)^2 \\
&\leq \sum_{n>N} \left(\int_0^\infty \sup_{y\geq0} |K(x,y)/V(y)| \|V f\| |e_n(x)| dx\right)^2 \\
&\leq \|V f\|^2 \sum_{n>N} \int_0^\infty \sup_{y\geq0} |K(x,y)/V(y)|^2 |e_n(x)|^2 dx \\
&= \|V f\|^2 \int_0^\infty \sup_{y\geq0} |K(x,y)/V(y)|^2 \sum_{n>N} |e_n(x)|^2 dx
\end{align*}

By Parseval's identity, for any fixed $x$, $\sum_{n=1}^\infty |e_n(x)|^2 = 1$ almost everywhere. Therefore, $\sum_{n>N} |e_n(x)|^2$ represents the tail of this series and converges to zero pointwise as $N \to \infty$ for almost every $x$. This sum is also bounded by 1 for all $N$ and $x$.

By the dominated convergence theorem and the Bochner V-boundedness condition, $\|(T-T_N)f\|^2 \to 0$ as $N \to \infty$, uniformly for $\|f\| \leq 1$. Thus, $T$ is the limit of finite rank operators and is therefore compact.
\end{proof}

\section{Proof of Compactness}

We will show that $T$ is Bochner V-bounded with $V(x) = 1 + x$.

\begin{theorem}
The operator $T$ defined by $(Tf)(x) = \int_0^\infty J_0(|x-y|)f(y)dy$ is compact on $L^2[0,\infty)$.
\end{theorem}

\begin{proof}
We need to show:
\[\int_0^\infty \sup_{y\geq0} |J_0(|x-y|)/(1+y)|^2 (1+x)^2 dx < \infty\]

First, note that for any $x, y \geq 0$:
\[|J_0(|x-y|)| \leq \min(1, \sqrt{2/(\pi|x-y|)})\]

Now, let's consider two cases:

1) For $|x-y| \leq 1$:
   \[|J_0(|x-y|)|/(1+y) \leq 1/(1+y) \leq 1/(1+|x|-1)^+\]
   where $(\cdot)^+$ denotes the positive part.

2) For $|x-y| > 1$:
   \[|J_0(|x-y|)|/(1+y) \leq \sqrt{2/(\pi|x-y|)}/(1+y)\]

To take the supremum over $y$, we consider:

a) When $x \leq 1$, the supremum is achieved in case 1, giving $1$.

b) When $x > 1$:
   - For $y \in [0, x-1] \cup [x+1, \infty)$, we use case 2.
   - For $y \in (x-1, x+1)$, we use case 1.

Thus, for $x > 1$:

\[\sup_{y\geq0} |J_0(|x-y|)/(1+y)| \leq \max\left(\frac{1}{x}, \sup_{y \in [0,x-1] \cup [x+1,\infty)} \frac{\sqrt{2/(\pi|x-y|)}}{1+y}\right)\]

For $y \in [0,x-1]$, $|x-y| \leq x$ and $1+y \geq 1$, so:

\[\frac{\sqrt{2/(\pi|x-y|)}}{1+y} \leq \sqrt{\frac{2}{\pi x}}\]

For $y \in [x+1,\infty)$, $|x-y| = y-x$ and $1+y \geq y$, so:

\[\frac{\sqrt{2/(\pi|x-y|)}}{1+y} \leq \frac{\sqrt{2/(\pi(y-x))}}{y} \leq \frac{\sqrt{2/\pi}}{x^{3/2}}\]

Therefore, for all $x > 0$:

\[\sup_{y\geq0} |J_0(|x-y|)/(1+y)| \leq \max\left(1, \frac{1}{x}, \sqrt{\frac{2}{\pi x}}, \frac{\sqrt{2/\pi}}{x^{3/2}}\right)\]

Now, we can bound our integral:

\begin{align*}
&\int_0^\infty \sup_{y\geq0} |J_0(|x-y|)/(1+y)|^2 (1+x)^2 dx \\
&\leq \int_0^1 (1+x)^2 dx + \int_1^\infty \max\left(\frac{1}{x^2}, \frac{2}{\pi x}, \frac{2/\pi}{x^3}\right) (1+x)^2 dx \\
&= \frac{7}{3} + \int_1^\infty \left(\frac{1}{x^2} + \frac{2}{\pi x} + \frac{2/\pi}{x^3}\right) (1+2x+x^2) dx \\
&= \frac{7}{3} + \int_1^\infty \left(\frac{1}{x^2} + \frac{2}{x} + 1 + \frac{2}{\pi x} + \frac{4}{\pi} + \frac{2}{\pi x^2} + \frac{2/\pi}{x^3} + \frac{4/\pi}{x^2} + \frac{2/\pi}{x}\right) dx \\
&= \frac{7}{3} + \left[-\frac{1}{x} + 2\log x + x + \frac{2}{\pi}\log x + \frac{4}{\pi}x - \frac{1}{\pi x} - \frac{1/\pi}{x^2} - \frac{2/\pi}{x} + \frac{2}{\pi}\log x\right]_1^\infty \\
&< \infty
\end{align*}

This proves that $T$ is Bochner V-bounded with $V(x) = 1 + x$, and therefore compact.
\end{proof}

\begin{remark}
The choice of $V(x) = 1 + x$ is optimal. If we chose $V(x) = 1$, the integral would diverge due to the slow decay of $J_0$. If we chose $V(x) = (1+x)^{1+\epsilon}$ for any $\epsilon > 0$, the proof would be easier as the integral would converge faster, but this would provide a weaker result.
\end{remark}

\end{document}
