\documentclass{article}
\usepackage{amsmath,amssymb,amsthm}
\usepackage{mathtools}

\newtheorem{theorem}{Theorem}
\newtheorem{lemma}[theorem]{Lemma}
\newtheorem{corollary}[theorem]{Corollary}
\newtheorem{definition}[theorem]{Definition}
\newtheorem{proposition}[theorem]{Proposition}

\begin{document}

\title{Defect Indices of Covariance Operators with Bessel Kernels under Coordinate Transformations}
\author{}
\date{}
\maketitle

\section{Definitions}

\begin{definition}[Bessel Kernel]
Let $J_0$ be the Bessel function of the first kind of order zero. The standard Bessel kernel is defined as $B(s,t) = J_0(2\pi|s-t|)$ for $s,t \in \mathbb{R}$.
\end{definition}

\begin{definition}[Transformed Bessel Kernel]
Given a function $\theta: \mathbb{R} \to \mathbb{R}$, the transformed Bessel kernel is defined as $K_\theta(s,t) = J_0(2\pi|\theta(s) - \theta(t)|)$ for $s,t \in \mathbb{R}$.
\end{definition}

\begin{definition}[Covariance Operator]
The integral operator $T_\theta$ associated with kernel $K_\theta$ acts on functions $f \in L^2(\mathbb{R})$ as:
\begin{equation}
    (T_\theta f)(s) = \int_{\mathbb{R}} J_0(2\pi|\theta(s) - \theta(t)|)f(t)dt
\end{equation}
\end{definition}

\begin{definition}[Defect Indices]
For a densely defined symmetric operator $T$ on a Hilbert space $\mathcal{H}$ with adjoint $T^*$, the defect indices $(n_+, n_-)$ are:
\begin{equation}
    n_+ = \dim \ker(T^* - i\cdot I), \quad n_- = \dim \ker(T^* + i\cdot I)
\end{equation}
where $I$ denotes the identity operator.
\end{definition}

\begin{definition}[Self-Adjoint Operator]
A symmetric operator $T$ is self-adjoint if and only if $T = T^*$, which is equivalent to having defect indices $n_+ = n_- = 0$.
\end{definition}

\section{Main Results}

\begin{theorem}\label{thm:main}
The covariance operator $T_\theta$ with kernel $K_\theta(s,t) = J_0(2\pi|\theta(s) - \theta(t)|)$ has zero defect indices $(n_+ = n_- = 0)$ if and only if $\theta$ is strictly monotonic.
\end{theorem}

To prove this theorem, several preliminary results are needed.

\begin{lemma}\label{lemma:bessel-pd}
The Bessel kernel $B(s,t) = J_0(2\pi|s-t|)$ defines a positive definite operator.
\end{lemma}

\begin{proof}
By Bochner's theorem, a continuous function $\phi(s-t)$ is positive definite if and only if it is the Fourier transform of a non-negative measure. The Fourier transform of $J_0(2\pi|x|)$ is:
\begin{equation}
    \mathcal{F}[J_0(2\pi|x|)](\omega) = \frac{1}{2\pi\sqrt{1-\omega^2/(4\pi^2)}}\mathbf{1}_{[-2\pi,2\pi]}(\omega)
\end{equation}
where $\mathbf{1}_{[-2\pi,2\pi]}$ is the indicator function of the interval $[-2\pi,2\pi]$.

Since this is a non-negative function, $J_0(2\pi|x|)$ is positive definite, and hence $B(s,t)$ defines a positive definite operator.
\end{proof}

\begin{lemma}\label{lemma:standard-sa}
The operator $S$ associated with the standard Bessel kernel $B(s,t) = J_0(2\pi|s-t|)$ is self-adjoint.
\end{lemma}

\begin{proof}
The operator $S$ with kernel $B(s,t)$ is unitarily equivalent to multiplication by the function $\frac{1}{2\pi\sqrt{1-\omega^2/(4\pi^2)}}\mathbf{1}_{[-2\pi,2\pi]}(\omega)$ in the Fourier domain. Since this is a bounded, real-valued multiplication operator, it is self-adjoint, and thus $S$ has defect indices $(0,0)$.
\end{proof}

\begin{proposition}\label{prop:monotonic-implies-sa}
If $\theta: \mathbb{R} \to \mathbb{R}$ is strictly monotonic, then the covariance operator $T_\theta$ is self-adjoint.
\end{proposition}

\begin{proof}
When $\theta$ is strictly monotonic, it is invertible. Consider the change of variables:
\begin{equation}
    u = \theta(s), \quad v = \theta(t)
\end{equation}

Define the unitary transformation $U: L^2(\mathbb{R}, ds) \to L^2(\mathbb{R}, du)$ by:
\begin{equation}
    (Uf)(u) = f(\theta^{-1}(u))\sqrt{\left|\frac{d\theta^{-1}}{du}(u)\right|}
\end{equation}

Under this transformation, the operator $T_\theta$ becomes:
\begin{equation}
    (UT_\theta U^{-1}g)(u) = \int_{\mathbb{R}} J_0(2\pi|u-v|)g(v)dv
\end{equation}
which is precisely the operator $S$ with the standard Bessel kernel.

Since $S$ is self-adjoint by Lemma \ref{lemma:standard-sa}, and unitary equivalence preserves self-adjointness, $T_\theta = U^{-1}SU$ is also self-adjoint. Thus, its defect indices are $(0,0)$.
\end{proof}

\begin{proposition}\label{prop:nonmonotonic-implies-defect}
If $\theta$ is not strictly monotonic, then $T_\theta$ has non-zero defect indices.
\end{proposition}

\begin{proof}
If $\theta$ is not strictly monotonic, there exist points $s_1 \neq s_2$ such that $\theta(s_1) = \theta(s_2)$.

Let $\mathcal{E} = \{(s_1,s_2) \in \mathbb{R}^2 : s_1 \neq s_2, \theta(s_1) = \theta(s_2)\}$. This set is non-empty by assumption.

For any pair $(s_1,s_2) \in \mathcal{E}$, the kernel satisfies:
\begin{equation}
    K_\theta(s_1,t) = J_0(2\pi|\theta(s_1) - \theta(t)|) = J_0(2\pi|\theta(s_2) - \theta(t)|) = K_\theta(s_2,t)
\end{equation}

This introduces a linear dependence in the kernel, violating the strict positive definiteness needed for self-adjointness.

To formalize this, consider the distribution:
\begin{equation}
    f_{s_1,s_2}(t) = \delta(t-s_1) - \delta(t-s_2)
\end{equation}

While $f_{s_1,s_2}$ itself is not in $L^2(\mathbb{R})$, it can be approximated by $L^2$ functions. Using the symmetry property $K_\theta(s_1,t) = K_\theta(s_2,t)$:
\begin{equation}
    (T_\theta f_{s_1,s_2})(s) = \int_{\mathbb{R}} K_\theta(s,t)f_{s_1,s_2}(t)dt = K_\theta(s,s_1) - K_\theta(s,s_2) = 0
\end{equation}

This implies that $T_\theta$ has a non-trivial null space, and consequently, there exist non-zero solutions to the equations $(T_\theta^* \pm i\cdot I)g = 0$. Therefore, both defect indices $n_+$ and $n_-$ are at least 1.
\end{proof}

\begin{lemma}\label{lemma:nonmonotonic-not-pd}
If $\theta$ is not strictly monotonic, then the kernel $K_\theta(s,t) = J_0(2\pi|\theta(s) - \theta(t)|)$ is not positive definite.
\end{lemma}

\begin{proof}
Let $s_1 \neq s_2$ with $\theta(s_1) = \theta(s_2)$. Consider the matrix:
\begin{equation}
    M = 
    \begin{pmatrix}
        K_\theta(s_1,s_1) & K_\theta(s_1,s_2) \\
        K_\theta(s_2,s_1) & K_\theta(s_2,s_2)
    \end{pmatrix}
\end{equation}

Since $\theta(s_1) = \theta(s_2)$, we have:
\begin{equation}
    K_\theta(s_1,s_1) = K_\theta(s_2,s_2) = J_0(0) = 1
\end{equation}
\begin{equation}
    K_\theta(s_1,s_2) = K_\theta(s_2,s_1) = J_0(2\pi|\theta(s_1) - \theta(s_2)|) = J_0(0) = 1
\end{equation}

Thus, $M = \begin{pmatrix} 1 & 1 \\ 1 & 1 \end{pmatrix}$, which has eigenvalues 2 and 0. The presence of the zero eigenvalue means $M$ is not strictly positive definite. Therefore, $K_\theta$ is not a positive definite kernel.
\end{proof}

\begin{proof}[Proof of Theorem \ref{thm:main}]
Combining Proposition \ref{prop:monotonic-implies-sa} and Proposition \ref{prop:nonmonotonic-implies-defect}, the covariance operator $T_\theta$ has defect indices $(0,0)$ if and only if $\theta$ is strictly monotonic.
\end{proof}

\begin{corollary}
The Gaussian process with covariance function $K_\theta(s,t) = J_0(2\pi|\theta(s) - \theta(t)|)$ is well-defined if and only if $\theta$ is strictly monotonic.
\end{corollary}

\begin{proof}
A Gaussian process is well-defined if and only if its covariance function is positive definite. By Lemma \ref{lemma:nonmonotonic-not-pd} and Lemma \ref{lemma:bessel-pd}, $K_\theta$ is positive definite if and only if $\theta$ is strictly monotonic. Furthermore, the self-adjointness of $T_\theta$ (which occurs if and only if $\theta$ is strictly monotonic by Theorem \ref{thm:main}) ensures the existence of a spectral decomposition, which is necessary for the proper definition of the process.
\end{proof}

\end{document}
