\documentclass{article}
\usepackage[english]{babel}
\usepackage{geometry,amsmath,amssymb,latexsym}
\geometry{letterpaper}

%%%%%%%%%% Start TeXmacs macros
\newcommand{\assign}{:=}
\newcommand{\tmem}[1]{{\em #1\/}}
\newcommand{\tmmisc}[1]{\thanks{\textit{Misc:} #1}}
\newcommand{\tmrsup}[1]{\textsuperscript{#1}}
\newenvironment{proof}{\noindent\textbf{Proof\ }}{\hspace*{\fill}$\Box$\medskip}
\newtheorem{definition}{Definition}
\newtheorem{lemma}{Lemma}
\newtheorem{theorem}{Theorem}
%%%%%%%%%% End TeXmacs macros

\begin{document}

\title{L\tmrsup{2} Norm Preservation Under Smooth Bijective Unbounded
Substitutions}

\author{
  Stephen Crowley
  \tmmisc{July 30, 2025}
}

\date{}

\maketitle

{\tableofcontents}

\section{Introduction}

This document concerns the structure of $L^2$-norm-preserving operators
induced on $L^2$ spaces by smooth, bijective, orientation-preserving
substitutions $g : I \to J$ on (possibly unbounded) intervals $I, J \subseteq
\mathbb{R}$. The topic is fundamental in ergodic theory and operator theory,
as it precisely characterizes when a substitution operator corresponds to a
unitary operator, and relates directly to the behavior of measures under
measure-preserving bijections. The classical result is also crucial for
understanding the behavior of the $L^2$ norm under change of variables.
Canonicity and necessity of the Jacobian factor is established, and the role
of unboundedness is treated from the start.

\section{Smooth Bijective Transformations and $L^2$ Norm Preservation}

\begin{definition}
  \label{def:bijectiveC1}Let $I, J \subseteq \mathbb{R}$ be (possibly
  unbounded) open intervals. A map $g : I \to J$ is called a {\tmem{smooth
  bijection}} if $g$ is:
  \begin{enumerate}
    \item Bijection between $I$ and $J$,
    
    \item Differentiable on $I$ with $g' (y) > 0$ for almost every $y \in I$
    (i.e., $g$ is strictly increasing except possibly on a set of Lebesgue
    measure zero).
  \end{enumerate}
\end{definition}

\begin{lemma}
  [Bijectivity of Strictly Increasing Unbounded $C^1$
  Maps]\label{lem:bijective_unbounded} Let $I, J \subseteq \mathbb{R}$ be
  (possibly unbounded) open intervals. Suppose $g : I \to J$ is a $C^1$
  function with $g' (y) > 0$ for all $y \in I$ except possibly a Lebesgue null
  set, and $g$ is unbounded above and below on $I$. Then $g$ is bijective onto
  $J = g (I)$, $g^{- 1}$ exists and is also strictly increasing and
  differentiable a.e.
\end{lemma}

\begin{proof}
  The function $g$ is strictly increasing on every subset of $I$ where $g' (y)
  > 0$; on the (at most measure-zero) set where $g' (y) = 0$, $g$ remains
  monotonic and continuous by $C^1$ regularity. Since $I$ is an interval and
  $g$ is continuous and strictly increasing almost everywhere, $g$ is
  injective by the intermediate value property of continuous strictly
  increasing functions.
  
  Unboundedness of $g$ on $I$ implies that $g (I)$ is also an open interval in
  $\mathbb{R}$ (possibly the whole real line), so $g : I \to J$ is surjective.
  Thus, $g$ is bijective from $I$ onto $J = g (I)$. Its inverse $g^{- 1} : J
  \to I$ is again continuous, strictly increasing (except possibly on a null
  set), and differentiable almost everywhere by the inverse function theorem.
\end{proof}

\section{Norm-Preserving Substitution Operators: Measure-Preservation and
Unitarity}

\begin{theorem}
  [L\tmrsup{2} Norm Preservation via Jacobian Factor]\label{thm:L2_jacobian}
  Let $g : I \to J$ be a smooth bijection in the sense of
  Definition~\ref{def:bijectiveC1}. For any $f \in L^2 (J, dx)$, define
  \begin{equation}
    \label{eq:transformationdef} \tilde{f} (y) \assign f (g (y)) \sqrt{g' (y)}
    .
  \end{equation}
  Then $\tilde{f} \in L^2 (I, dy)$ and
  \begin{equation}
    \label{eq:L2normpres} \| \tilde{f} \|_{L^2 (I, dy)} = \|f\|_{L^2 (J, dx)}
    .
  \end{equation}
\end{theorem}

\begin{proof}
  Since $g : I \to J$ is bijective, strictly increasing and differentiable
  almost everywhere with $g' (y) > 0$ a.e., the change of variables theorem
  applies (see e.g., {\cite{RoydenFitzpatrick}}, {\cite{Folland}}).
  
  For any $f \in L^2 (J, dx)$,
  
  \begin{align}
    \| \tilde{f} \|_{L^2 (I, dy)}^2 & = \int_I |f (g (y)) \sqrt{g' (y)} |^2 
    \hspace{0.17em} dy  \label{eq:normexpansion1}\\
    & = \int_I |f (g (y)) |^2 g' (y)  \hspace{0.17em} dy 
    \label{eq:normexpansion2}
  \end{align}
  
  By the change of variables formula for Lebesgue integrals, for any
  measurable function $\varphi$ and bijective, strictly increasing $g$ as in
  Lemma~\ref{lem:bijective_unbounded}:
  \begin{equation}
    \label{eq:cov} \int_I \varphi (g (y)) g' (y)  \hspace{0.17em} dy = \int_J
    \varphi (x)  \hspace{0.17em} dx.
  \end{equation}
  Setting $\varphi (x) = |f (x) |^2$, one obtains
  \begin{equation}
    \int_I |f (g (y)) |^2 g' (y) \hspace{0.17em} dy = \int_J |f (x) |^2 
    \hspace{0.17em} dx = \|f\|_{L^2 (J, dx)}^2 \label{eq:aftercov}
  \end{equation}
  Thus, $\| \tilde{f} \|_{L^2 (I, dy)} = \|f\|_{L^2 (J, dx)}$ as claimed.
\end{proof}

\section{Necessity and Canonicality of the Jacobian Weight}

\begin{lemma}
  [Density of Substitution Images]\label{lem:L2density} Let $g : I \rightarrow
  J$ be as in Theorem~\ref{thm:L2_jacobian}. Then the collection $\{f \circ g
  : f \in L^2 (J, dx)\}$ is dense in $L^2 (I, g' (y) \hspace{0.17em} dy)$.
\end{lemma}

\begin{proof}
  The transformation $T : L^2 (J, dx) \to L^2 (I, g' (y) \hspace{0.17em} dy)$
  defined by $T (f) = f \circ g$ is an isometric isomorphism by the change of
  variables~\eqref{eq:cov}. The image of an isomorphism from a complete space
  is itself complete and thus dense.
\end{proof}

\begin{theorem}
  [Necessity of the Square Root Jacobian Factor]\label{thm:necessity} Let $g :
  I \to J$ be as above. Suppose $\psi : I \to \mathbb{R}^+$ is measurable and
  for every $f \in L^2 (J, dx)$,
  \begin{equation}
    \label{eq:generalweight} | f (g (\cdot)) \cdot \psi (\cdot) |_{L^2 (I,
    dy)} = \|f\|_{L^2 (J, dx)} .
  \end{equation}
  Then $\psi (y) = \sqrt{g' (y)}$ for almost every $y \in I$.
\end{theorem}

\begin{proof}
  Suppose \eqref{eq:generalweight} holds for all $f \in L^2 (J, dx)$. Compute:
  
  \begin{align}
    \int_I |f (g (y)) |^2 | \psi (y) |^2  \hspace{0.17em} dy & = \|f\|_{L^2
    (J, dx)}^2 \\
    & = \int_I |f (g (y)) |^2 g' (y)  \hspace{0.17em} dy  \label{eq:proofcov}
  \end{align}
  
  Subtracting, for every $f$,
  \begin{equation}
    \int_I |f (g (y)) |^2  (| \psi (y) |^2 - g' (y))  \hspace{0.17em} dy = 0
  \end{equation}
  By Lemma~\ref{lem:L2density}, the set $\{f (g (y))\}$ is dense in $L^2 (I,
  g' (y) dy)$. Thus, for every $u \in L^2 (I, g' (y) dy)$,
  \begin{equation}
    \int_I |u (y) |^2  (| \psi (y) |^2 - g' (y))  \hspace{0.17em} dy = 0
  \end{equation}
  By standard measure-theoretic arguments (cf. {\cite{Folland}}, p. 70), the
  only way for this to be true for all $u$ is for $| \psi (y) |^2 = g' (y)$
  almost everywhere. Since $\psi$ is taken as non-negative, $\psi (y) =
  \sqrt{g' (y)}$ a.e.
\end{proof}

\section{Unitary Operators, Invariant Measures, and Measure-Preservation}

\begin{definition}
  [Koopman Operator]\label{def:koopman} Let $(X, \mathcal{B}, \mu)$ be a
  probability measure space, $T : X \rightarrow X$ a measurable bijection, and
  $\mu$ a $T$-invariant measure: for all $A \in \mathcal{B}$, $\mu (T^{- 1} A)
  = \mu (A)$. The {\tmem{Koopman operator}} $U_T$ is defined for measurable $f
  : X \to \mathbb{C}$ by
  \begin{equation}
    \label{eq:koopman} (U_T f) (x) = f (Tx) .
  \end{equation}
\end{definition}

\begin{theorem}
  [Unitarity Corresponds to
  Measure-Preservation]\label{thm:measurepreserving_unitary} The Koopman
  operator $U_T$ on $L^2 (X, \mu)$ is unitary if and only if $T$ is invertible
  and both $T$ and $T^{- 1}$ preserve the measure $\mu$.
\end{theorem}

\begin{proof}
  If $T$ is invertible and $\mu$ is $T$-invariant,
  \[ \|U_T f\|_{L^2 (X, \mu)}^2 = \int_X |f (Tx) |^2  \hspace{0.17em} d \mu
     (x) = \int_X |f (x) |^2  \hspace{0.17em} d \mu (x) \]
  where the last equality is by change of variables $x = T^{- 1} (y)$ and
  measure-preservation, so $U_T$ is an isometry. Surjectivity follows from
  invertibility of $T$ and surjectivity of $L^2$ composition. Conversely, if
  $U_T$ is unitary, then the above identity must hold for all $f$. Choosing
  indicator functions of sets $A$, it follows that $\mu (T^{- 1} (A)) = \mu
  (A)$, so $T$ preserves the measure.
\end{proof}

\section{Bibliography}

\begin{thebibliography}{99}
  {\bibitem{RoydenFitzpatrick}}H. L. Royden and P. M. Fitzpatrick, {\tmem{Real
  Analysis}}, Fourth Edition, Pearson, 2010.
  
  {\bibitem{Folland}}G. B. Folland, {\tmem{Real Analysis: Modern Techniques
  and Their Applications}}, Second Edition, Wiley, 1999.
  
  {\bibitem{Walters}}P. Walters, {\tmem{An Introduction to Ergodic Theory}},
  Springer, 1982.
  
  {\bibitem{Halmos}}P. R. Halmos, {\tmem{Measure Theory}}, Springer, 1974.
  
  {\bibitem{EinsiedlerWard}}M. Einsiedler and T. Ward, {\tmem{Ergodic Theory
  with a View Towards Number Theory}}, Springer, 2011.
\end{thebibliography}

\end{document}
