\documentclass{article}
\usepackage[english]{babel}

%%%%%%%%%% Start TeXmacs macros
\newcommand{\mathd}{\mathrm{d}}
%%%%%%%%%% End TeXmacs macros

\begin{document}

\section{Convolution and Fourier Transform Discussion}

The convolution of two functions $f (x)$ and $g (x)$ can be expressed using
the Fourier transform:
\[ (f \ast g) (x) = \int_{- \infty}^{\infty} f (x - t) g (t) dt
   =\mathcal{F}^{- 1}  \{\mathcal{F}\{f\} \cdot \mathcal{F}\{g\}\} \]
Where:
\begin{itemize}
  \item $\ast$ denotes convolution
  
  \item $\mathcal{F}$ denotes the Fourier transform
  
  \item $\mathcal{F}^{- 1}$ denotes the inverse Fourier transform
  
  \item $\cdot$ denotes multiplication in the frequency domain
\end{itemize}
This formulation represents the standard convolution and its relationship to
Fourier transforms.

The Convolution Theorem states that the convolution of two functions in the
time (or spatial) domain is equivalent to the multiplication of their Fourier
transforms in the frequency domain, followed by an inverse Fourier transform.

For the convolution over a positive domain:
\[ \int_0^{\infty} f (x - t) g (t) dt \]
The Fourier transform relationship is:
\[ \mathcal{F} \{f (x - t) \ast g (x)\} =\mathcal{F} \{f (x)\} \cdot
   \mathcal{F} \{g (x)\} \]
Additionally, we have:
\[ (f \ast g) (x) = (g \ast f) (x) \]
And for the Fourier transform of a convolution:
\[ \mathcal{F} \{f \ast g\} =\mathcal{F} \{f\} \cdot \mathcal{F} \{g\} \]

\section{Covariance Operator and Eigenfunction Expansion}

For a stationary Gaussian process, we consider the integral covariance
operator $T$ with kernel $K$. The eigenfunctions $\phi_n (x)$ of this operator
satisfy:
\begin{equation}
  (T \phi_n) (y) = \int_0^{\infty} K (x, y) \phi_n (x) \mathd x = \lambda_n
  \phi_n (y)
\end{equation}
where $\lambda_n$ are the corresponding eigenvalues.

We can expand these eigenfunctions in terms of a uniformly convergent
orthonormal basis $\{\psi_k (x)\}$ for $L^2 (0, \infty)$:
\begin{equation}
  \phi_n (x) = \sum_{k = 0}^{\infty} c_{n, k} \psi_k (x)
\end{equation}
The expansion coefficients $c_{n, k}$ can be expressed as:
\begin{equation}
  c_{n, k} = \frac{\int_{- \infty}^{\infty} \psi_k (x)  (T \phi_n) (x)
  dx}{\lambda_n} 
\end{equation}

\subsection{Proof of the Expansion Coefficient Formula}

Let's prove this formula by substitution and expansion:

1) Start with the eigenvalue equation:
\[ (K \phi_n) (x) = \lambda_n \phi_n (x) \]
2) Multiply both sides by $\psi_k (x)$ and integrate over the entire domain:
\[ \int_{- \infty}^{\infty} \psi_k (x)  (K \phi_n) (x) dx = \lambda_n  \int_{-
   \infty}^{\infty} \psi_k (x) \phi_n (x) dx \]
3) The right-hand side integral is the definition of $c_{n, k}$ due to the
orthonormality of $\{\psi_k (x)\}$:
\[ \int_{- \infty}^{\infty} \psi_k (x)  (K \phi_n) (x) dx = \lambda_n c_{n, k}
\]
4) Rearranging this equation gives us the formula for $c_{n, k}$:
\[ c_{n, k} = \frac{1}{\lambda_n}  \int_{- \infty}^{\infty} \psi_k (x)  (K
   \phi_n) (x) dx \]
5) To verify, let's substitute the expansion of $\phi_n (x)$ into the
eigenvalue equation:
\[ K \left( \sum_{k = 1}^{\infty} c_{n, k} \psi_k (x) \right) = \lambda_n 
   \sum_{k = 1}^{\infty} c_{n, k} \psi_k (x) \]
6) By linearity of $K$:
\[ \sum_{k = 1}^{\infty} c_{n, k}  (K \psi_k) (x) = \lambda_n  \sum_{k =
   1}^{\infty} c_{n, k} \psi_k (x) \]
7) Multiply both sides by $\psi_j (x)$ and integrate:
\[ \sum_{k = 1}^{\infty} c_{n, k}  \int_{- \infty}^{\infty} \psi_j (x)  (K
   \psi_k) (x) dx = \lambda_n c_{n, j} \]
8) The left-hand side integral is our formula for $c_{n, k}$ multiplied by
$\lambda_n$:
\[ \sum_{k = 1}^{\infty} c_{n, k}  (\lambda_n c_{j, k}) = \lambda_n c_{n, j}
\]
9) This reduces to an identity, proving that our formula for $c_{n, k}$
satisfies the eigenvalue equation.

Thus, we have proven that the formula for $c_{n, k}$ is correct and consistent
with the eigenvalue equation.

\end{document}
