\documentclass{article}
\usepackage[english]{babel}
\usepackage{geometry,amsmath,amssymb,latexsym}
\geometry{letterpaper}

%%%%%%%%%% Start TeXmacs macros
\newcommand{\tmaffiliation}[1]{\\ #1}
\newcommand{\tmem}[1]{{\em #1\/}}
\newenvironment{proof}{\noindent\textbf{Proof\ }}{\hspace*{\fill}$\Box$\medskip}
\newtheorem{corollary}{Corollary}
\newtheorem{definition}{Definition}
\newtheorem{theorem}{Theorem}
%%%%%%%%%% End TeXmacs macros

\begin{document}

\title{Unitary Bijections From Strictly Increasing Functions On The Real Line
}

\author{
  Stephen Crowley
  \tmaffiliation{August 11, 2025}
}

\date{July 30, 2025}

\maketitle

{\tableofcontents}

\section{Introduction}

This document establishes the fundamental relationship between unitary
bijections in $L^2$ spaces and measure-preserving transformations in ergodic
theory. The central result demonstrates that $L^2$ norm preservation under
bijective transformations of unbounded domains necessarily involves specific
scaling factors derived from the transformation's differential structure.

\section{Bijective Transformations on Unbounded Domains}

\begin{theorem}
  [Bijectivity of Strictly Increasing Functions on Unbounded
  Domains]\label{thm:bijective_unbounded}Let $g : I \to \mathbb{R}$ be a
  strictly increasing function where $I \subseteq \mathbb{R}$ is an unbounded
  interval. Then $g$ is bijective onto its range $J = g (I)$, and $J$ is also
  an unbounded interval.
\end{theorem}

\begin{proof}
  Since $g$ is strictly increasing, injectivity is immediate. For any $x_1,
  x_2 \in I$ with $x_1 < x_2$, one has $g (x_1) < g (x_2)$.
  
  For surjectivity onto $J = g (I)$, let $y \in J$. By definition, there
  exists $x \in I$ such that $g (x) = y$. The uniqueness of such $x$ follows
  from injectivity.
  
  To establish that $J$ is unbounded, consider two cases:
  \begin{enumerate}
    \item If $I = (a, \infty)$ or $I = [a, \infty)$ for some $a \in
    \mathbb{R}$, then as $x \to \infty$, since $g$ is strictly increasing,
    either $g (x) \to \infty$ or $g (x)$ approaches some finite supremum. If
    the latter held, then by the intermediate value theorem and strict
    monotonicity, $g$ would map $(a, \infty)$ to some bounded interval,
    contradicting the strict increase property over an unbounded domain.
    
    \item If $I = (- \infty, b)$ or $I = (- \infty, b]$, a similar argument
    shows $J$ extends to $- \infty$.
    
    \item If $I =\mathbb{R}$, then $J$ must be unbounded in both directions.
  \end{enumerate}
  Therefore, $g : I \to J$ is bijective with both $I$ and $J$ unbounded
  intervals.
\end{proof}

\begin{theorem}
  [Differentiable Bijections with Positive
  Derivative]\label{thm:diff_bijective}Let $g : I \to J$ be a $C^1$ bijection
  between unbounded intervals $I, J \subseteq \mathbb{R}$ such that $g' (y) >
  0$ for all $y \in I$ except possibly on a set of measure zero. Then $g$ is a
  well-defined change of variables for Lebesgue integration.
\end{theorem}

\begin{proof}
  The condition $g' (y) > 0$ almost everywhere ensures that $g$ is locally
  invertible almost everywhere. Since $g$ is already assumed bijective and
  $C^1$, the standard change of variables formula applies:
  \begin{equation}
    \label{eq:change_vars} \int_J f (x)  \hspace{0.17em} dx = \int_I f (g (y))
    |g' (y) |  \hspace{0.17em} dy = \int_I f (g (y)) g' (y)  \hspace{0.17em}
    dy
  \end{equation}
  where the last equality uses $g' (y) > 0$ almost everywhere. The points
  where $g' (y) = 0$ form a set of measure zero and do not affect the
  integral.
\end{proof}

\section{$L^2$ Norm Preservation}

\begin{definition}
  [Scaled Transformation Operator]\label{def:scaled_transform}Let $g : I \to
  J$ be a $C^1$ bijection between unbounded intervals with $g' (y) > 0$ almost
  everywhere. For $f \in L^2 (J, dx)$, define the scaled transformation
  operator $T_g$ by:
  \begin{equation}
    \label{eq:scaled_transform} (T_g f) (y) = f (g (y)) \sqrt{g' (y)}
  \end{equation}
\end{definition}

\begin{theorem}
  [$L^2$ Norm Preservation for Unbounded
  Domains]\label{thm:l2_preservation}Under the conditions of
  Definition~\ref{def:scaled_transform}, the operator $T_g : L^2 (J, dx) \to
  L^2 (I, dy)$ is an isometric isomorphism. Specifically:
  \begin{equation}
    \label{eq:norm_equality} \|T_g f\|_{L^2 (I, dy)} = \|f\|_{L^2 (J, dx)}
  \end{equation}
\end{theorem}

\begin{proof}
  For $f \in L^2 (J, dx)$, compute directly:
  
  \begin{align}
    \|T_g f\|_{L^2 (I, dy)}^2 & = \int_I |f (g (y)) \sqrt{g' (y)} |^2 
    \hspace{0.17em} dy  \label{eq:norm_calc1}\\
    & = \int_I |f (g (y)) |^2 g' (y)  \hspace{0.17em} dy 
    \label{eq:norm_calc2}
  \end{align}
  
  By the change of variables formula from Theorem~\ref{thm:diff_bijective}
  with $x = g (y)$:
  \begin{equation}
    \label{eq:change_vars_apply} \int_I |f (g (y)) |^2 g' (y) \hspace{0.17em}
    dy = \int_J |f (x) |^2  \hspace{0.17em} dx = \|f\|_{L^2 (J, dx)}^2
  \end{equation}
  Since both $I$ and $J$ are unbounded, the change of variables is justified
  by approximating with bounded subintervals and applying the monotone
  convergence theorem.
  
  Therefore:
  \begin{equation}
    \label{eq:final_norm} \|T_g f\|_{L^2 (I, dy)} = \|f\|_{L^2 (J, dx)}
  \end{equation}
  The fact that $T_g f \in L^2 (I, dy)$ follows immediately from
  equation~\eqref{eq:final_norm} and the assumption $f \in L^2 (J, dx)$.
\end{proof}

\begin{theorem}
  [Necessity of Square Root Scaling]\label{thm:necessity}Let $g : I \to J$ be
  as in Theorem~\ref{thm:l2_preservation}. If $\phi : I \to \mathbb{R}^+$ is
  any measurable function such that $f (g (y)) \phi (y) \in L^2 (I, dy)$ and
  \begin{equation}
    \label{eq:general_norm} \|f (g (\cdot)) \phi (\cdot)\|_{L^2 (I, dy)} =
    \|f\|_{L^2 (J, dx)}
  \end{equation}
  for all $f \in L^2 (J, dx)$, then $\phi (y) = \sqrt{g' (y)}$ almost
  everywhere.
\end{theorem}

\begin{proof}
  From the norm condition in equation~\eqref{eq:general_norm}:
  \begin{equation}
    \label{eq:phi_condition} \int_I |f (g (y)) |^2 \phi (y)^2  \hspace{0.17em}
    dy = \int_J |f (x) |^2  \hspace{0.17em} dx
  \end{equation}
  Using the change of variables $x = g (y)$ on the right side:
  \begin{equation}
    \label{eq:phi_comparison} \int_I |f (g (y)) |^2 \phi (y)^2 
    \hspace{0.17em} dy = \int_I |f (g (y)) |^2 g' (y)  \hspace{0.17em} dy
  \end{equation}
  This gives:
  \begin{equation}
    \label{eq:phi_difference} \int_I |f (g (y)) |^2  (\phi (y)^2 - g' (y)) 
    \hspace{0.17em} dy = 0
  \end{equation}
  Since this holds for all $f \in L^2 (J, dx)$ and the composition $f (g
  (\cdot))$ generates a dense subspace of $L^2 (I, g' (y) \hspace{0.17em}
  dy)$, the fundamental lemma of calculus of variations implies:
  \begin{equation}
    \label{eq:phi_ae_equal} \phi (y)^2 = g' (y) \text{almost everywhere}
  \end{equation}
  Taking $\phi (y) > 0$, one obtains $\phi (y) = \sqrt{g' (y)}$ almost
  everywhere.
\end{proof}

\section{Unitary Operators and Measure Preservation}

\begin{definition}
  [Koopman Operator]\label{def:koopman}Let $(X, \mathcal{B}, \mu)$ be a
  probability space and $T : X \to X$ be a measure-preserving bijection. The
  Koopman operator $U_T : L^2 (X, \mu) \to L^2 (X, \mu)$ is defined by:
  \begin{equation}
    \label{eq:koopman} (U_T f) (x) = f (T (x))
  \end{equation}
\end{definition}

\begin{theorem}
  [Unitarity of Koopman Operator]\label{thm:koopman_unitary}The Koopman
  operator $U_T$ defined in Definition~\ref{def:koopman} is unitary on $L^2
  (X, \mu)$.
\end{theorem}

\begin{proof}
  For $f, h \in L^2 (X, \mu)$:
  
  \begin{align}
    \langle U_T f, U_T h \rangle & = \int_X f (T (x)) \overline{h (T (x))}
    \hspace{0.17em} d \mu (x)  \label{eq:inner_product1}\\
    & = \int_X f (y) \overline{h (y)} \hspace{0.17em} d \mu (T^{- 1} (y)) 
    \label{eq:inner_product2}\\
    & = \int_X f (y) \overline{h (y)} \hspace{0.17em} d \mu (y) 
    \label{eq:inner_product3}\\
    & = \langle f, h \rangle  \label{eq:inner_product4}
  \end{align}
  
  where equation~\eqref{eq:inner_product2} uses the change of variables $y = T
  (x)$, and equation~\eqref{eq:inner_product3} follows from the
  measure-preserving property of $T$.
  
  Since $T$ is bijective and measure-preserving, $U_T$ is surjective,
  completing the proof of unitarity.
\end{proof}

\begin{corollary}
  [Equivalence of Unitary Bijection and Measure
  Preservation]\label{cor:unitary_equiv}A bijective transformation $T$ on a
  probability space induces a unitary operator on $L^2$ if and only if $T$ is
  measure-preserving.
\end{corollary}

\begin{proof}
  This follows directly from Theorem~\ref{thm:koopman_unitary} and the fact
  that the Koopman operator construction is reversible.
\end{proof}

\section{Invariant Measures}

\begin{definition}
  [Invariant Measure]\label{def:invariant}A measure $\mu$ on a measurable
  space $(X, \mathcal{B})$ is invariant under a transformation $T : X \to X$
  if $\mu (T^{- 1} (A)) = \mu (A)$ for all $A \in \mathcal{B}$.
\end{definition}

\begin{theorem}
  [Uniqueness of Finite Invariant Measures for Ergodic
  Systems]\label{thm:invariant_unique}Let $T : X \to X$ be an ergodic
  transformation on a measurable space. If finite invariant measures $\mu_1$
  and $\mu_2$ exist for $T$, then $\mu_1 = c \mu_2$ for some constant $c > 0$.
\end{theorem}

\begin{proof}
  The proof follows from the ergodic theorem and the fact that ergodic systems
  admit at most one invariant probability measure up to
  scaling~{\cite{petersen1989ergodic}}.
\end{proof}

\section{Conclusion}

The results establish that unitary bijections in $L^2$ spaces correspond
precisely to measure-preserving transformations. The scaling factor $\sqrt{g'
(y)}$ in Theorem~\ref{thm:l2_preservation} is both necessary and sufficient
for norm preservation, providing the connection between differential geometry
and functional analysis in the context of ergodic theory.

\begin{thebibliography}{9}
  {\bibitem{petersen1989ergodic}}K. Petersen, {\tmem{Ergodic Theory}},
  Cambridge Studies in Advanced Mathematics, Cambridge University Press, 1989.
  
  {\bibitem{halmos1956lectures}}P. R. Halmos, {\tmem{Lectures on Ergodic
  Theory}}, Chelsea Publishing Company, 1956.
  
  {\bibitem{walters1982introduction}}P. Walters, {\tmem{An Introduction to
  Ergodic Theory}}, Graduate Texts in Mathematics, Springer-Verlag, 1982.
  
  {\bibitem{reed1980functional}}M. Reed and B. Simon, {\tmem{Methods of Modern
  Mathematical Physics I: Functional Analysis}}, Academic Press, 1980.
\end{thebibliography}

\end{document}
