\documentclass{article}
\usepackage[english]{babel}
\usepackage{geometry,amsmath,amssymb,textcomp,latexsym}
\geometry{letterpaper}

%%%%%%%%%% Start TeXmacs macros
\newcommand{\tmaffiliation}[1]{\\ #1}
\newcommand{\tmem}[1]{{\em #1\/}}
\newcommand{\tmtextbf}[1]{\text{{\bfseries{#1}}}}
\newenvironment{proof}{\noindent\textbf{Proof\ }}{\hspace*{\fill}$\Box$\medskip}
\newtheorem{definition}{Definition}
\newtheorem{lemma}{Lemma}
\newtheorem{theorem}{Theorem}
%%%%%%%%%% End TeXmacs macros

\begin{document}

{\cdot}\title{Unitary Bijections From Strictly Increasing Functions On The
Real Line}

\author{
  Stephen Crowley
  \tmaffiliation{August 11, 2025}
}

\maketitle

\begin{abstract}
  This paper establishes a comprehensive theory of unitary change-of-variables
  operators on $L^2$ spaces, encompassing both the general framework for $L^2
  (\mathbb{R})$ and specialized results for measure-preserving transformations
  on unbounded domains. The investigation begins with the characterization of
  when weighted composition operators $(Uf) (x) = f (T (x)) \cdot w (x)$
  achieve unitarity, requiring measurable bijections modulo null sets, mutual
  absolute continuity of measures, and specific weight functions involving
  Radon-Nikodym derivatives. For differentiable transformations, this reduces
  to the condition $|w (x) |^2 = |T' (x) |$. The analysis then specializes to
  $C^1$ bijective transformations $g : I \to J$ between unbounded intervals
  with positive derivative almost everywhere, where $L^2$ norm preservation
  under Lebesgue measure is achieved through the unitary change of variables
  operator $T_g f = f (g (y)) \sqrt{g' (y)}$. The framework is further
  extended to arbitrary {\sigma}-finite measures {\textmu} and {\nu}, where
  the scaling factor becomes the square root of the Radon-Nikodym derivative
  $\sqrt{\frac{d (\mu \circ g^{- 1})}{d \mu} (y)}$. The necessity of these
  specific scaling factors is rigorously established through variational
  arguments in all settings. These findings provide a unified theoretical
  foundation bridging the change-of-variables formula in real analysis with
  the unitary structure of $L^2$ spaces over general measure spaces, with
  applications in ergodic theory, functional analysis, and measure theory.
\end{abstract}

{\tableofcontents}

\section{Introduction}

This paper presents a comprehensive theory of unitary change-of-variables
operators on $L^2$ spaces, establishing the fundamental relationship between
unitary bijections and measure-preserving transformations in both general and
specialized settings. The investigation begins with the general framework for
weighted composition operators on $L^2 (\mathbb{R})$, then specializes to
measure-preserving transformations on unbounded domains, extending from
classical Lebesgue measure to general {\sigma}-finite measures.

\section{General Framework: Unitary Change-of-Variables Operators}

\begin{definition}
  \label{def:cov-operator}A {\tmem{change-of-variables operator}} on $L^2
  (\mathbb{R}, \mu)$ where $\mu$ is Lebesgue measure is a bounded linear
  operator $U : L^2 (\mathbb{R}) \to L^2 (\mathbb{R})$ of the form
  \begin{equation}
    (Uf) (x) = f (T (x)) \cdot w (x)
  \end{equation}
  for some measurable map $T : \mathbb{R} \to \mathbb{R}$ and measurable
  weight function $w : \mathbb{R} \to \mathbb{C}$ with $|w (x) | > 0$ almost
  everywhere.
\end{definition}

\begin{theorem}
  \label{thm:main}Let $U$ be a change-of-variables operator as in
  Definition~\ref{def:cov-operator}. Then $U$ is unitary if and only if the
  following conditions hold:
  \begin{enumerate}
    \item $T : \mathbb{R} \to \mathbb{R}$ is a measurable bijection modulo
    null sets;
    
    \item $\mu \circ T^{- 1} \ll \mu$ and $\mu \ll \mu \circ T^{- 1}$ (mutual
    absolute continuity);
    
    \item $|w (x) |^2 = \frac{d (\mu \circ T)}{d \mu} (x)$ almost everywhere;
    
    \item $w (x) = \sqrt{\frac{d (\mu \circ T)}{d \mu} (x)} \cdot e^{i \theta
    (x)}$ for some measurable phase function $\theta : \mathbb{R} \to
    \mathbb{R}$.
  \end{enumerate}
  Furthermore, if $T$ is differentiable almost everywhere with $T' (x) \neq 0$
  a.e., then condition (3) becomes
  \begin{equation}
    |w (x) |^2 = |T' (x) |
  \end{equation}
\end{theorem}

\begin{proof}
  The proof proceeds by establishing necessity and sufficiency separately.
  
  \tmtextbf{Necessity:} Assume $U$ is unitary. Since $U$ is an isometry, for
  any $f \in L^2 (\mathbb{R})$,
  \begin{equation}
    \label{eq:isometry} \|Uf\|_2^2 = \|f\|_2^2
  \end{equation}
  Computing the left side:
  \begin{equation}
    \|Uf\|_2^2 = \int_{\mathbb{R}} |f (T (x)) |^2 |w (x) |^2  \hspace{0.17em}
    d \mu (x)
  \end{equation}
  Define the measure $\nu$ by $d \nu = |w|^2 d \mu$. By the
  change-of-variables formula for the pushforward measure,
  \begin{equation}
    \int_{\mathbb{R}} |f (T (x)) |^2 |w (x) |^2 \hspace{0.17em} d \mu (x) =
    \int_{\mathbb{R}} |f (y) |^2  \hspace{0.17em} d (T_{\ast} \nu) (y)
  \end{equation}
  where $(T_{\ast} \nu) (A) = \nu (T^{- 1} (A))$ for measurable sets $A$.
  
  From equation~\eqref{eq:isometry}, we require
  \begin{equation}
    \label{eq:measure-condition} \int_{\mathbb{R}} |f (y) |^2  \hspace{0.17em}
    d (T_{\ast} \nu) (y) = \int_{\mathbb{R}} |f (y) |^2  \hspace{0.17em} d \mu
    (y)
  \end{equation}
  for all $f \in L^2 (\mathbb{R})$.
  
  This implies $T_{\ast} \nu = \mu$ as measures. Therefore, for any measurable
  set $A$,
  \begin{equation}
    \mu (A) = \nu (T^{- 1} (A)) = \int_{T^{- 1} (A)} |w (x) |^2 
    \hspace{0.17em} d \mu (x)
  \end{equation}
  For $U$ to be surjective (hence unitary rather than merely isometric), $T$
  must be invertible modulo null sets. This requires both directions of
  absolute continuity in condition (2).
  
  By the Radon-Nikodym theorem, since $\mu \circ T^{- 1} \ll \mu$, there
  exists $\rho \geq 0$ such that
  \begin{equation}
    \rho (y) = \frac{d (\mu \circ T^{- 1})}{d \mu} (y)
  \end{equation}
  The standard change-of-variables identity gives, for nonnegative measurable
  $g$,
  \begin{equation}
    \int_{\mathbb{R}} g (T (x))  \hspace{0.17em} d \mu (x) = \int_{\mathbb{R}}
    g (y) \rho (y)  \hspace{0.17em} d \mu (y)
  \end{equation}
  Comparing with the isometry requirement from
  equation~\eqref{eq:measure-condition}, we obtain
  \begin{equation}
    \int_{\mathbb{R}} g (T (x)) |w (x) |^2  \hspace{0.17em} d \mu (x) =
    \int_{\mathbb{R}} g (y)  \hspace{0.17em} d \mu (y)
  \end{equation}
  This requires
  \begin{equation}
    |w (x) |^2 = \rho (T (x))^{- 1}
  \end{equation}
  almost everywhere. By the chain rule for Radon-Nikodym derivatives,
  \begin{equation}
    |w (x) |^2 = \frac{d (\mu \circ T)}{d \mu} (x)
  \end{equation}
  The phase freedom in condition (4) follows from the fact that only $|w|^2$
  is determined by the isometry condition.
  
  \tmtextbf{Sufficiency:} Conversely, assume conditions (1)-(4) hold. Define
  $U$ as in Definition~\ref{def:cov-operator} with the specified $T$ and $w$.
  The computation above shows that $U$ is isometric. Since $T$ is bijective
  modulo null sets with mutual absolute continuity, the operator $U^{\ast}$
  exists and is given by
  \begin{equation}
    (U^{\ast} g) (x) = g (T^{- 1} (x)) \cdot \overline{w (T^{- 1} (x))} \cdot
    \sqrt{\frac{d (\mu \circ T^{- 1})}{d \mu} (x)} \cdot e^{- i \theta (T^{-
    1} (x))}
  \end{equation}
  Direct computation verifies $UU^{\ast} = U^{\ast} U = I$, establishing
  unitarity.
  
  The final statement regarding differentiable $T$ follows from the fact that
  for such maps,
  \begin{equation}
    \frac{d (\mu \circ T)}{d \mu} (x) = |T' (x) |
  \end{equation}
  by the classical change-of-variables theorem.
\end{proof}

\begin{lemma}
  \label{lem:monotone}If $T : \mathbb{R} \to \mathbb{R}$ is a measurable
  bijection that is differentiable almost everywhere, then $T$ is either
  almost everywhere monotone increasing or almost everywhere monotone
  decreasing.
\end{lemma}

\begin{proof}
  Since $T$ is a bijection of $\mathbb{R}$, the intermediate value theorem and
  injectivity require that $T$ cannot change monotonicity on any interval
  where it is continuous. As $T$ is differentiable almost everywhere, it is
  continuous almost everywhere, and the set where $T'$ exists has full
  measure. On this set, $T'$ cannot change sign without violating the
  bijection property, hence $T' (x) \geq 0$ almost everywhere or $T' (x) \leq
  0$ almost everywhere.
\end{proof}

\section{Bijective Transformations on Unbounded Domains}

\begin{theorem}[Bijectivity of Strictly Increasing Functions on Unbounded
Domains]
  \label{thm:bijective_unbounded}Let $g : I \to \mathbb{R}$ be a strictly
  increasing function where $I \subseteq \mathbb{R}$ is an unbounded interval.
  Then $g$ is bijective onto its range $J = g (I)$, and $J$ is also an
  unbounded interval.
\end{theorem}

\begin{proof}
  Since $g$ is strictly increasing, injectivity is immediate. For any $x_1,
  x_2 \in I$ with $x_1 < x_2$, one has $g (x_1) < g (x_2)$.
  
  For surjectivity onto $J = g (I)$, let $y \in J$. By definition, there
  exists $x \in I$ such that $g (x) = y$. The uniqueness of such $x$ follows
  from injectivity.
  
  To establish that $J$ is unbounded, consider two cases:
  \begin{enumerate}
    \item If $I = (a, \infty)$ or $I = [a, \infty)$ for some $a \in
    \mathbb{R}$, then as $x \to \infty$, since $g$ is strictly increasing,
    either $g (x) \to \infty$ or $g (x)$ approaches some finite supremum. If
    the latter held, then by the intermediate value theorem and strict
    monotonicity, $g$ would map $(a, \infty)$ to some bounded interval,
    contradicting the strict increase property over an unbounded domain.
    
    \item If $I = (- \infty, b)$ or $I = (- \infty, b]$, a similar argument
    shows $J$ extends to $- \infty$.
    
    \item If $I =\mathbb{R}$, then $J$ must be unbounded in both directions.
  \end{enumerate}
  Therefore, $g : I \to J$ is bijective with both $I$ and $J$ unbounded
  intervals.
\end{proof}

\begin{theorem}[Differentiable Bijections with Positive Derivative]
  \label{thm:diff_bijective}Let $g : I \to J$ be a $C^1$ bijection between
  unbounded intervals $I, J \subseteq \mathbb{R}$ such that $g' (y) > 0$ for
  all $y \in I$ except possibly on a set of measure zero. Then $g$ is a
  well-defined change of variables for Lebesgue integration.
\end{theorem}

\begin{proof}
  The condition $g' (y) > 0$ almost everywhere ensures that $g$ is locally
  invertible almost everywhere. Since $g$ is already assumed bijective and
  $C^1$, the standard change of variables formula applies:
  \begin{equation}
    \label{eq:change_vars} \int_J f (x)  \hspace{0.17em} dx = \int_I f (g (y))
    |g' (y) |  \hspace{0.17em} dy = \int_I f (g (y)) g' (y)  \hspace{0.17em}
    dy
  \end{equation}
  where the last equality uses $g' (y) > 0$ almost everywhere. The points
  where $g' (y) = 0$ form a set of measure zero and do not affect the
  integral.
\end{proof}

\section{$L^2$ Norm Preservation Under Lebesgue Measure}

\begin{definition}[Unitary Change of Variables Operator]
  \label{def:unitary_transform}Let $g : I \to J$ be a $C^1$ bijection between
  unbounded intervals with $g' (y) > 0$ almost everywhere. For $f \in L^2 (J,
  dx)$, define the unitary change of variables operator $T_g$ by:
  \begin{equation}
    \label{eq:unitary_transform} (T_g f) (y) = f (g (y)) \sqrt{g' (y)}
  \end{equation}
\end{definition}

\begin{theorem}[$L^2$ Norm Preservation for Unbounded Domains]
  \label{thm:l2_preservation}Under the conditions of
  Definition~\ref{def:unitary_transform}, the operator $T_g : L^2 (J, dx) \to
  L^2 (I, dy)$ is an isometric isomorphism. Specifically:
  \begin{equation}
    \label{eq:norm_equality} \|T_g f\|_{L^2 (I, dy)} = \|f\|_{L^2 (J, dx)}
  \end{equation}
\end{theorem}

\begin{proof}
  For $f \in L^2 (J, dx)$, compute directly:
  
  \begin{align}
    \|T_g f\|_{L^2 (I, dy)}^2 & = \int_I |f (g (y)) \sqrt{g' (y)} |^2 
    \hspace{0.17em} dy  \label{eq:norm_calc1}\\
    & = \int_I |f (g (y)) |^2 g' (y)  \hspace{0.17em} dy 
    \label{eq:norm_calc2}
  \end{align}
  
  By the change of variables formula from Theorem~\ref{thm:diff_bijective}
  with $x = g (y)$:
  \begin{equation}
    \label{eq:change_vars_apply} \int_I |f (g (y)) |^2 g' (y) \hspace{0.17em}
    dy = \int_J |f (x) |^2  \hspace{0.17em} dx = \|f\|_{L^2 (J, dx)}^2
  \end{equation}
  Since both $I$ and $J$ are unbounded, the change of variables is justified
  by approximating with bounded subintervals and applying the monotone
  convergence theorem.
  
  Therefore:
  \begin{equation}
    \label{eq:final_norm} \|T_g f\|_{L^2 (I, dy)} = \|f\|_{L^2 (J, dx)}
  \end{equation}
  The fact that $T_g f \in L^2 (I, dy)$ follows immediately from
  equation~\eqref{eq:final_norm} and the assumption $f \in L^2 (J, dx)$.
\end{proof}

\begin{theorem}[Necessity of Square Root Unitary Transformation]
  \label{thm:necessity}Let $g : I \to J$ be as in
  Theorem~\ref{thm:l2_preservation}. If $\phi : I \to \mathbb{R}^+$ is any
  measurable function such that $f (g (y)) \phi (y) \in L^2 (I, dy)$ and
  \begin{equation}
    \label{eq:general_norm} \|f (g (\cdot)) \phi (\cdot)\|_{L^2 (I, dy)} =
    \|f\|_{L^2 (J, dx)}
  \end{equation}
  for all $f \in L^2 (J, dx)$, then $\phi (y) = \sqrt{g' (y)}$ almost
  everywhere.
\end{theorem}

\begin{proof}
  From the norm condition in equation~\eqref{eq:general_norm}:
  \begin{equation}
    \label{eq:phi_condition} \int_I |f (g (y)) |^2 \phi (y)^2  \hspace{0.17em}
    dy = \int_J |f (x) |^2  \hspace{0.17em} dx
  \end{equation}
  Using the change of variables $x = g (y)$ on the right side:
  \begin{equation}
    \label{eq:phi_comparison} \int_I |f (g (y)) |^2 \phi (y)^2 
    \hspace{0.17em} dy = \int_I |f (g (y)) |^2 g' (y)  \hspace{0.17em} dy
  \end{equation}
  This gives:
  \begin{equation}
    \label{eq:phi_difference} \int_I |f (g (y)) |^2  (\phi (y)^2 - g' (y)) 
    \hspace{0.17em} dy = 0
  \end{equation}
  Since this holds for all $f \in L^2 (J, dx)$ and the composition $f (g
  (\cdot))$ generates a dense subspace of $L^2 (I, g' (y) \hspace{0.17em}
  dy)$, the fundamental lemma of calculus of variations implies:
  \begin{equation}
    \label{eq:phi_ae_equal} \phi (y)^2 = g' (y) \text{almost everywhere}
  \end{equation}
  Taking $\phi (y) > 0$, one obtains $\phi (y) = \sqrt{g' (y)}$ almost
  everywhere.
\end{proof}

\section{Extension to General {\sigma}-Finite Measures}

\begin{theorem}[Extension to General Measures]
  \label{thm:general_measures}Let $\mu$ and $\nu$ be {\sigma}-finite measures
  on $I$ and $J$ respectively, and let $g : I \to J$ be a measurable
  bijection. If $\nu = \mu \circ g^{- 1}$ (i.e., $\nu (E) = \mu (g^{- 1} (E))$
  for all measurable $E \subseteq J$), then for $f \in L^2 (J, d \nu)$:
  \begin{equation}
    \label{eq:general_transform} \tilde{f} (y) = f (g (y)) \sqrt{\frac{d (\mu
    \circ g^{- 1})}{d \mu} (y)}
  \end{equation}
  satisfies $\| \tilde{f} \|_{L^2 (I, d \mu)} = \|f\|_{L^2 (J, d \nu)}$, where
  $\frac{d (\mu \circ g^{- 1})}{d \mu}$ is the Radon-Nikodym derivative.
\end{theorem}

\begin{proof}
  When $\mu$ and $\nu$ are both Lebesgue measure and $g$ is differentiable,
  the Radon-Nikodym derivative is $|g' (y) |$, reducing to
  Theorem~\ref{thm:l2_preservation}.
  
  For the general case, compute:
  
  \begin{align}
    \| \tilde{f} \|_{L^2 (I, d \mu)}^2 & = \int_I |f (g (y)) |^2 \frac{d (\mu
    \circ g^{- 1})}{d \mu} (y)  \hspace{0.17em} d \mu (y) 
    \label{eq:general_norm_calc1}\\
    & = \int_I |f (g (y)) |^2  \hspace{0.17em} d (\mu \circ g^{- 1}) (y) 
    \label{eq:general_norm_calc2}
  \end{align}
  
  By the definition of the pushforward measure $\mu \circ g^{- 1}$ and since
  $\nu = \mu \circ g^{- 1}$:
  \begin{equation}
    \label{eq:pushforward_change} \int_I |f (g (y)) |^2 \hspace{0.17em} d (\mu
    \circ g^{- 1}) (y) = \int_J |f (x) |^2  \hspace{0.17em} d \nu (x) =
    \|f\|_{L^2 (J, d \nu)}^2
  \end{equation}
  The change of variables follows from the same argument using the definition
  of the pushforward measure. Therefore:
  \begin{equation}
    \label{eq:general_final_norm} \| \tilde{f} \|_{L^2 (I, d \mu)} =
    \|f\|_{L^2 (J, d \nu)}
  \end{equation}
\end{proof}

\section{Conclusion}

The results establish a comprehensive theory of unitary change-of-variables
operators on $L^2$ spaces. The general framework shows that unitarity requires
measurable bijections modulo null sets, mutual absolute continuity, and weight
functions given by square roots of Radon-Nikodym derivatives. For $L^2$ norm
preservation under measurable bijections, the scaling factor $\sqrt{g'}$ for
Lebesgue measure generalizes to $\sqrt{\frac{d (\mu \circ g^{- 1})}{d \mu}}$
for arbitrary {\sigma}-finite measures. These factors are both necessary and
sufficient for isometry, linking the change-of-variables formula to unitary
structure on $L^2$ spaces over arbitrary measure spaces.

\begin{thebibliography}{9}
  {\bibitem{petersen1989ergodic}}K. Petersen, {\tmem{Ergodic Theory}},
  Cambridge Studies in Advanced Mathematics, Cambridge University Press, 1989.
  
  {\bibitem{halmos1956lectures}}P. R. Halmos, {\tmem{Lectures on Ergodic
  Theory}}, Chelsea Publishing Company, 1956.
  
  {\bibitem{walters1982introduction}}P. Walters, {\tmem{An Introduction to
  Ergodic Theory}}, Graduate Texts in Mathematics, Springer-Verlag, 1982.
  
  {\bibitem{reed1980functional}}M. Reed and B. Simon, {\tmem{Methods of Modern
  Mathematical Physics I: Functional Analysis}}, Academic Press, 1980.
\end{thebibliography}

\

\end{document}
