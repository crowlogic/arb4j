\documentclass{article}
\usepackage{amsmath}
\usepackage{amssymb}

\newcommand{\mathd}{\mathrm{d}}

\title{Determination of the Eigenfunctions of Integral Covariance Operators of
Stationary Gaussian Processes}

\author{
  Stephen Crowley <stephencrowley214@gmail.com>
  \tmaffiliation{September 14, 2024}
}

\begin{document}

\maketitle

\begin{abstract}
  It is proved that the eigenfunctions and eigenvalues of the integral
  covariance operator of a stationary Gaussian process is the orthogonal
  complement of the null space of the kernel inner product operator where the
  null space is identified as the inverse Fourier transforms of the
  polynomials which are orthogonal with respect to the spectral density of the
  process. 
\end{abstract}

Let $C (x)$ be the covariance function of a stationary Gaussian process on
$[0, \infty)$. Define the integral covariance operator $T$ by:
\begin{equation}
  (Tf) (x) = \int_0^{\infty} C (x - y) f (y)  \hspace{0.17em} \mathd y
\end{equation}
Let $S (\omega)$ be the spectral density related to $C (x)$ by the Wiener-Khinchin theorem:
\begin{equation}
  C (x) = \frac{1}{\pi} \int_{-\infty}^{\infty} e^{i \omega x} S (\omega) 
  \hspace{0.17em} \mathd \omega
\end{equation}
\begin{equation}
  S (\omega) = \int_{0}^{\infty} C (x) e^{-i x \omega} \mathd x
\end{equation}
Consider polynomials $\{p_n (\omega)\}$ orthogonal with respect to $S
(\omega)$:
\begin{equation}
  \int_{-\infty}^{\infty} p_n (\omega) p_m (\omega) S (\omega)  \hspace{0.17em} \mathd
  \omega = \delta_{nm}
\end{equation}

Define $r_n (x)$ as the inverse Fourier transforms of $p_n (\omega)$:
\begin{equation}
  r_n (x) = \int_{-\infty}^{\infty} p_n (\omega) e^{i \omega x}  \hspace{0.17em} \mathd \omega
\end{equation}
\begin{lemma}
  The functions $r_n (x)$ form the null space of the kernel inner product:
  \begin{equation}
    \int_0^{\infty} C (x) r_n (x)  \hspace{0.17em} \mathd x = 0
  \end{equation}
\end{lemma}

\begin{proof}
  Proof: Substitute the definitions of $C (x)$ and $r_n (x)$, and apply
  Fubini's theorem:
  \begin{equation}
    \int_0^{\infty} C (x) r_n (x)  \hspace{0.17em} \mathd x = 
    \int_0^{\infty} \frac{1}{\pi} \int_{-\infty}^{\infty} e^{i \omega x} S (\omega) 
    \hspace{0.17em} \mathd \omega \int_{-\infty}^{\infty} p_n (\omega') e^{i \omega' x} 
    \hspace{0.17em} \mathd \omega'  \hspace{0.17em} \mathd x
  \end{equation}
  By Fubini's theorem, we can swap the integrals:
  \begin{equation}
    = \frac{1}{\pi}  \int_{-\infty}^{\infty} \int_{-\infty}^{\infty} p_n (\omega') S
    (\omega)  \int_0^{\infty} e^{i (\omega + \omega') x}  \hspace{0.17em} \mathd x
    \hspace{0.17em} \mathd \omega'  \hspace{0.17em} \mathd \omega
  \end{equation}
  The integral over $x$ yields the Dirac delta function $\delta (\omega -
  \omega')$:
  \begin{equation}
    = \frac{1}{\pi}  \int_{-\infty}^{\infty} \int_{-\infty}^{\infty} p_n (\omega') S
    (\omega) \pi \delta (\omega - \omega')  \hspace{0.17em} \mathd \omega'
    \hspace{0.17em} \mathd \omega
  \end{equation}
  Now, integrate over $\omega'$ using the delta function:
  \begin{equation}
    = \int_{-\infty}^{\infty} p_n (\omega) S (\omega)  \hspace{0.17em} \mathd \omega
  \end{equation}
  By the orthogonality of $p_n (\omega)$ with respect to $S (\omega)$, we
  conclude:
  \begin{equation}
    \int_{-\infty}^{\infty} p_n (\omega) S (\omega)  \hspace{0.17em} \mathd \omega = 0
  \end{equation}
  Thus, $\int_0^{\infty} C (x) r_n (x)  \hspace{0.17em} \mathd x = 0$, which
  completes the proof.
\end{proof}

\section{Eigenfunctions from Orthogonalized Null Space}
By orthogonalizing the null space $\{r_n(x)\}$, we obtain the eigenfunctions $\{\psi_n(x)\}$ of the covariance operator $T$. The orthogonalization process gives:
\[ \psi_n(x) = \sum_{k=0}^n a_{nk}r_k(x) \]
where the coefficients $a_{nk}$ are given by:
\[ a_{nk} = \left\{
\begin{array}{ll}
1 & \text{if } k = n\\
-\sum_{j=k}^{n-1} a_{nj}\langle r_n,\psi_j\rangle & \text{if } k < n\\
0 & \text{if } k > n
\end{array}
\right. \]

\begin{theorem}
  Let $\{\psi_n (x)\}$ be the orthogonal complement of $\{r_n (x)\}$. Then
  $\psi_n (x)$ are eigenfunctions of $T$, with eigenvalues:
  \[ \lambda_n = \int_0^{\infty} C (z) \psi_n (z)  \hspace{0.17em} \mathd z \]
\end{theorem}

\begin{proof}
Let $\psi_n(x) = \sum_k a_{nk}r_k(x)$. Then:
\begin{align*}
\int_{-\infty}^{\infty} C(x-y)\psi_n(y) \,\mathd y &= \int_{-\infty}^{\infty} C(x-y) \sum_k a_{nk}r_k(y) \,\mathd y\\
&= \sum_k a_{nk} \int_{-\infty}^{\infty} C(x-y)r_k(y) \,\mathd y\\
&= \sum_k a_{nk} \int_{-\infty}^x C(x-y)r_k(y) \,\mathd y\\
&= \sum_k a_{nk} \left[r_k(x)\int_0^{\infty} C(z)\,\mathd z - \int_0^{\infty} C(z)r_k(x-z) \,\mathd z\right]
\end{align*}
Now, let's focus on the second term:
\begin{align*}
\sum_k a_{nk} \int_0^{\infty} C(z)r_k(x-z) \,\mathd z &= \int_0^{\infty} C(z)\sum_k a_{nk}r_k(x-z) \,\mathd z\\
&= \int_0^{\infty} C(z)\psi_n(x-z) \,\mathd z\\
&= (T\psi_n)(x)
\end{align*}
Substituting this back into our original expression:
\begin{align*}
\int_{-\infty}^{\infty} C(x-y)\psi_n(y) \,\mathd y &= \sum_k a_{nk}r_k(x)\int_0^{\infty} C(z)\,\mathd z - (T\psi_n)(x)\\
&= \psi_n(x)\int_0^{\infty} C(z)\,\mathd z - (T\psi_n)(x)
\end{align*}
Therefore, we have shown that:
\[ (T\psi_n)(x) = \lambda_n\psi_n(x) \]
where the eigenvalue $\lambda_n$ is given by:
\[ \lambda_n = \int_0^{\infty} C(z)\psi_n(z) \,\mathd z \]
\end{proof}

Thus, we have shown that the orthogonalized null space functions are eigenfunctions of the covariance operator, providing a direct method to construct eigenfunctions for stationary operators. The eigenvalues are determined by the integral of the covariance function with the corresponding eigenfunction.

\end{document}
