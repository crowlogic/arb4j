\documentclass[11pt]{article}
\usepackage{amsmath,amssymb,amsthm}
\usepackage{enumitem}
\usepackage{geometry}
\geometry{margin=1in}

\newtheorem{theorem}{Theorem}[section]
\newtheorem{lemma}[theorem]{Lemma}
\newtheorem{corollary}[theorem]{Corollary}
\newtheorem{definition}[theorem]{Definition}

\title{Spectral Recovery and Pre-Envelope Theory: Formal Theorems}
\author{Mathematical Analysis}
\date{\today}

\begin{document}

\maketitle

\section{Spectral Recovery Theory}

\begin{theorem}[Spectral Recovery for Real Stationary Processes]
\label{thm:spectral_recovery}
Let $X(t)$ be a real-valued, zero-mean, stationary Gaussian process with spectral representation:
\begin{equation}
\label{eq:spectral_rep_real}
X(t) = \int_0^{\infty} \cos(\lambda t) \, dU(\lambda) + \sin(\lambda t) \, dV(\lambda)
\end{equation}

Then the orthogonal random measures $U(\lambda)$ and $V(\lambda)$ are recovered from the sample path by:
\begin{equation}
\label{eq:u_recovery}
U(\lambda) = \frac{2}{\pi} \int_0^{\infty} X(t) \cos(\lambda t) dt
\end{equation}
\begin{equation}
\label{eq:v_recovery}
V(\lambda) = \frac{2}{\pi} \int_0^{\infty} X(t) \sin(\lambda t) dt
\end{equation}
\end{theorem}

\begin{proof}
Apply the orthogonality relations of the trigonometric system. For $U(\lambda)$:
\begin{equation}
\label{eq:u_proof_step1}
\int_0^{\infty} X(t) \cos(\mu t) dt = \int_0^{\infty} \left[\int_0^{\infty} \cos(\lambda t) \, dU(\lambda) + \sin(\lambda t) \, dV(\lambda)\right] \cos(\mu t) dt
\end{equation}

Interchanging integration order:
\begin{equation}
\label{eq:u_proof_step2}
= \int_0^{\infty} dU(\lambda) \int_0^{\infty} \cos(\lambda t) \cos(\mu t) dt + \int_0^{\infty} dV(\lambda) \int_0^{\infty} \sin(\lambda t) \cos(\mu t) dt
\end{equation}

Using orthogonality:
\begin{equation}
\label{eq:orthogonality_cos}
\int_0^{\infty} \cos(\lambda t) \cos(\mu t) dt = \frac{\pi}{2} \delta(\lambda - \mu)
\end{equation}
\begin{equation}
\label{eq:orthogonality_sin_cos}
\int_0^{\infty} \sin(\lambda t) \cos(\mu t) dt = 0
\end{equation}

Therefore:
\begin{equation}
\label{eq:u_proof_final}
\int_0^{\infty} X(t) \cos(\mu t) dt = \frac{\pi}{2} U(\mu)
\end{equation}

Hence: $U(\mu) = \frac{2}{\pi} \int_0^{\infty} X(t) \cos(\mu t) dt$. The proof for $V(\lambda)$ is analogous.
\end{proof}

\begin{theorem}[Analytic Signal Representation]
\label{thm:analytic_signal}
For any real-valued stationary process $X(t)$, the analytic signal $Z(t)$ is defined as:
\begin{equation}
\label{eq:analytic_signal_def}
Z(t) = X(t) + i\hat{X}(t)
\end{equation}

where the quadrature process is:
\begin{equation}
\label{eq:quadrature_process}
\hat{X}(t) = \int_0^{\infty} \sin(\lambda t) \, dU(\lambda) - \cos(\lambda t) \, dV(\lambda)
\end{equation}

Then $Z(t)$ admits the complex exponential representation:
\begin{equation}
\label{eq:complex_exponential_rep}
Z(t) = \int_0^{\infty} e^{i\lambda t} d\zeta(\lambda)
\end{equation}

where $d\zeta(\lambda) = dU(\lambda) - i \, dV(\lambda)$ is the pre-envelope spectral measure.
\end{theorem}

\begin{proof}
\begin{equation}
\label{eq:analytic_proof_step1}
Z(t) = X(t) + i\hat{X}(t)
\end{equation}
\begin{align}
\label{eq:analytic_proof_step2}
&= \int_0^{\infty} [\cos(\lambda t) \, dU(\lambda) + \sin(\lambda t) \, dV(\lambda)] \nonumber \\
&\quad + i\int_0^{\infty} [\sin(\lambda t) \, dU(\lambda) - \cos(\lambda t) \, dV(\lambda)]
\end{align}

Collecting terms:
\begin{align}
\label{eq:analytic_proof_step3}
&= \int_0^{\infty} [(\cos(\lambda t) + i\sin(\lambda t)) \, dU(\lambda) + (\sin(\lambda t) - i\cos(\lambda t)) \, dV(\lambda)] \nonumber \\
&= \int_0^{\infty} [e^{i\lambda t} \, dU(\lambda) - ie^{i\lambda t} \, dV(\lambda)] \nonumber \\
&= \int_0^{\infty} e^{i\lambda t} [dU(\lambda) - i \, dV(\lambda)]
\end{align}

Setting $d\zeta(\lambda) = dU(\lambda) - i \, dV(\lambda)$ completes the proof.
\end{proof}

\begin{theorem}[Pre-Envelope Spectral Measure Properties]
\label{thm:pre_envelope_properties}
The pre-envelope spectral measure $\zeta(\lambda)$ satisfies:
\begin{enumerate}
\item $E[d\zeta(\lambda)] = 0$
\item $E[|d\zeta(\lambda)|^2] = dG(\lambda)$ where $G(\lambda)$ is the spectral distribution function
\item $E[d\zeta(\lambda_1) \overline{d\zeta(\lambda_2)}] = 0$ for $\lambda_1 \neq \lambda_2$
\end{enumerate}
\end{theorem}

\begin{proof}
\begin{enumerate}
\item 
\begin{equation}
\label{eq:pre_envelope_mean}
E[d\zeta(\lambda)] = E[dU(\lambda) - i \, dV(\lambda)] = 0 - i \cdot 0 = 0
\end{equation}

\item 
\begin{align}
\label{eq:pre_envelope_variance}
E[|d\zeta(\lambda)|^2] &= E[(dU(\lambda) - i \, dV(\lambda))(\overline{dU(\lambda) - i \, dV(\lambda)})] \nonumber \\
&= E[(dU(\lambda) - i \, dV(\lambda))(dU(\lambda) + i \, dV(\lambda))] \nonumber \\
&= E[|dU(\lambda)|^2] + E[|dV(\lambda)|^2] = dG(\lambda)
\end{align}

Since $E[|dU(\lambda)|^2] = E[|dV(\lambda)|^2] = \frac{1}{2}dG(\lambda)$ by symmetry.

\item Orthogonality follows from the orthogonality of $U$ and $V$ increments.
\end{enumerate}
\end{proof}

\section{Envelope Theory}

\begin{theorem}[Envelope as Absolute Value of Analytic Signal]
\label{thm:envelope_absolute_value}
The envelope $R(t)$ of the real process $X(t)$ is given by:
\begin{equation}
\label{eq:envelope_def}
R(t) = |Z(t)| = \sqrt{X^2(t) + \hat{X}^2(t)}
\end{equation}

where $Z(t)$ is the analytic signal.
\end{theorem}

\begin{proof}
By definition:
\begin{equation}
\label{eq:envelope_proof}
R(t) = |Z(t)| = |X(t) + i\hat{X}(t)| = \sqrt{X^2(t) + \hat{X}^2(t)}
\end{equation}

This establishes that the envelope is the modulus of the pre-envelope process $Z(t)$.
\end{proof}

\begin{theorem}[Polar Representation of Complex Process]
\label{thm:polar_representation}
The analytic signal $Z(t)$ admits the polar representation:
\begin{equation}
\label{eq:polar_rep}
Z(t) = R(t) e^{i\Theta(t)}
\end{equation}

where:
\begin{enumerate}
\item $R(t) = |Z(t)| = \sqrt{X^2(t) + \hat{X}^2(t)}$ is the envelope
\item $\Theta(t) = \arctan\left(\frac{\hat{X}(t)}{X(t)}\right)$ is the instantaneous phase
\end{enumerate}
\end{theorem}

\begin{proof}
For any complex number $z = a + ib$, the polar form is $z = |z|e^{i\arg(z)}$ where:
\begin{equation}
\label{eq:complex_modulus}
|z| = \sqrt{a^2 + b^2}
\end{equation}
\begin{equation}
\label{eq:complex_argument}
\arg(z) = \arctan(b/a)
\end{equation}

Applying this to $Z(t) = X(t) + i\hat{X}(t)$:
\begin{equation}
\label{eq:envelope_polar}
R(t) = |Z(t)| = \sqrt{X^2(t) + \hat{X}^2(t)}
\end{equation}
\begin{equation}
\label{eq:phase_polar}
\Theta(t) = \arg(Z(t)) = \arctan\left(\frac{\hat{X}(t)}{X(t)}\right)
\end{equation}

Therefore: $Z(t) = R(t) e^{i\Theta(t)}$.
\end{proof}

\begin{theorem}[Instantaneous Frequency]
\label{thm:instantaneous_frequency}
The instantaneous frequency $\omega(t)$ of the process $X(t)$ is defined as:
\begin{equation}
\label{eq:instantaneous_frequency_def}
\omega(t) = \frac{d\Theta(t)}{dt}
\end{equation}

where $\Theta(t)$ is the instantaneous phase from Theorem~\ref{thm:polar_representation}.
\end{theorem}

\begin{proof}
By definition of instantaneous frequency as the time derivative of phase:
\begin{equation}
\label{eq:freq_derivative}
\omega(t) = \frac{d}{dt}\left[\arctan\left(\frac{\hat{X}(t)}{X(t)}\right)\right]
\end{equation}

Using the chain rule:
\begin{equation}
\label{eq:freq_chain_rule}
\omega(t) = \frac{1}{1 + \left(\frac{\hat{X}(t)}{X(t)}\right)^2} \cdot \frac{d}{dt}\left(\frac{\hat{X}(t)}{X(t)}\right)
\end{equation}
\begin{equation}
\label{eq:freq_quotient_rule}
= \frac{X^2(t)}{X^2(t) + \hat{X}^2(t)} \cdot \frac{\hat{X}'(t)X(t) - \hat{X}(t)X'(t)}{X^2(t)}
\end{equation}
\begin{equation}
\label{eq:freq_final}
= \frac{\hat{X}'(t)X(t) - \hat{X}(t)X'(t)}{X^2(t) + \hat{X}^2(t)} = \frac{\hat{X}'(t)X(t) - \hat{X}(t)X'(t)}{R^2(t)}
\end{equation}

This expresses the instantaneous frequency in terms of the original process and its quadrature.
\end{proof}

\section{Spectral Inversion}

\begin{corollary}[Spectral Recovery from Analytic Signal]
\label{cor:spectral_recovery_analytic}
The pre-envelope spectral measure can be recovered from the analytic signal by:
\begin{equation}
\label{eq:spectral_recovery_analytic}
\zeta(\lambda) = \frac{1}{2\pi} \int_{-\infty}^{\infty} Z(t) e^{-i\lambda t} dt
\end{equation}
\end{corollary}

\begin{proof}
This follows directly from the inverse Fourier transform applied to the complex exponential representation in Theorem~\ref{thm:analytic_signal}.
\end{proof}

\end{document}
