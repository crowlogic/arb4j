\documentclass{article}
\usepackage[english]{babel}
\usepackage{amsmath}

%%%%%%%%%% Start TeXmacs macros
\newcommand{\tmop}[1]{\ensuremath{\operatorname{#1}}}
%%%%%%%%%% End TeXmacs macros

\begin{document}

\title{Romanovski Polynomials}

\date{}

\maketitle

\section*{Short Description}

Mathematics concept.

In mathematics, the Romanovski polynomials are one of three finite subsets of
real orthogonal polynomials discovered by Vsevolod
Romanovsky{\cite{Romanovski1929}} (Romanovski in French transcription) within
the context of probability distribution functions in statistics. They form an
orthogonal subset of a more general family of little-known Routh polynomials
introduced by Edward John Routh{\cite{Routh1884}} in 1884. The term Romanovski
polynomials was put forward by Raposo,{\cite{Raposo2007}} with reference to
the so-called 'pseudo-Jacobi polynomials in Lesky's classification
scheme.{\cite{Lesky1996}} It seems more consistent to refer to them as
Romanovski--Routh polynomials, by analogy with the terms Romanovski--Bessel
and Romanovski--Jacobi used by Lesky for two other sets of orthogonal
polynomials.

In some contrast to the standard classical orthogonal polynomials, the
polynomials under consideration differ, in so far as for arbitrary parameters
only a finite number of them are orthogonal, as discussed in more detail
below.

\section*{The Differential Equation for the Romanovski Polynomials}

The Romanovski polynomials solve the following version of the hypergeometric
differential equation:
\begin{equation}
  s (x) {R^{(\alpha, \beta)}_n}'' (x) + t^{(\alpha, \beta)}_1 (x) {R^{(\alpha,
  \beta)}_n}' (x) + \lambda_n R^{(\alpha, \beta)}_n (x) = 0,
\end{equation}
where $x \in (- \infty, + \infty)$, $s (x) = (1 + x^2)$, $t^{(\alpha,
\beta)}_1 (x) = 2 \beta x + \alpha$, and $\lambda_n = - n (2 \beta + n - 1)$.

Curiously, they have been omitted from the standard textbooks on special
functions in mathematical physics and in
mathematics{\cite{Abramowitz1972,Nikiforov1988,Szego1939,Ismail2005}} and have
only a relatively scarce presence elsewhere in the mathematical
literature.{\cite{Askey1987,Zarzo1995}}

The weight functions are:
\begin{equation}
  w^{(\alpha, \beta)} (x) = (1 + x^2)^{\beta - 1} \exp \left( - \alpha \arccot
  x \right) ;
\end{equation}
they solve Pearson's differential equation:
\begin{equation}
  [s (x) w (x)]' = t (x) w (x), \quad s (x) = 1 + x^2,
\end{equation}
that assures the self-adjointness of the differential operator of the
hypergeometric ordinary differential equation.

For $\alpha = 0$ and $\beta < 0$, the weight function of the Romanovski
polynomials takes the shape of the Cauchy distribution, whence the associated
polynomials are also denoted as Cauchy polynomials{\cite{Witte2000}} in their
applications in random matrix theory.{\cite{Forrester2010}}

The Rodrigues formula specifies the polynomial $R^{(\alpha, \beta)}_n (x)$ as:
\begin{equation}
  R^{(\alpha, \beta)}_n (x) = N_n  \frac{1}{w^{(\alpha, \beta)} (x)} 
  \frac{\mathrm{d}^n}{\mathrm{d} x^n}  (w^{(\alpha, \beta)} (x) s (x)^n),
  \quad 0 \leq n,
\end{equation}
where $N_n$ is a normalization constant. This constant is related to the
coefficient $c_n$ of the term of degree $n$ in the polynomial $R^{(\alpha,
\beta)}_n (x)$ by the expression:
\begin{equation}
  N_n = \frac{(- 1)^n n! \hspace{0.17em} c_n}{\prod_{k = 0}^{n - 1}
  \lambda_n^{(k)}}, \quad \lambda_n = - n \left( {t^{(\alpha, \beta)}_n}' +
  \tfrac{1}{2} (n - 1) s'' (x) \right),
\end{equation}
which holds for $n \geq 1$.

\section*{Relationship Between the Polynomials of Romanovski and Jacobi}

As shown by Askey, this finite sequence of real orthogonal polynomials can be
expressed in terms of Jacobi polynomials of imaginary argument and thereby is
frequently referred to as complexified Jacobi polynomials.{\cite{Cotfas2004}}
Namely, the Romanovski equation can be formally obtained from the Jacobi
equation,{\cite{MathWorldJacobiDifferentialEquation}}
\begin{equation}
  \begin{aligned}
    & (1 - x^2) {P_n^{(\gamma, \delta)}}'' (x) + t^{(\gamma, \delta)}_1 (x)
    {P_n^{(\gamma, \delta)}}' (x) + \lambda_n P^{(\gamma, \delta)}_n (x) =
    0,\\
    & \qquad t^{(\gamma, \delta)}_1 (x) = \delta - \gamma - (\gamma + \delta
    + 2) x, \quad \lambda_n = n (n + \gamma + \delta + 1), \quad x \in [- 1,
    1],
  \end{aligned}
\end{equation}
via the replacements, for real $x$,
\begin{equation}
  x \to ix, \quad \frac{\mathrm{d}}{\mathrm{d} x} \to - i
  \frac{\mathrm{d}}{\mathrm{d} x}, \quad \gamma = \delta^{\ast} = \beta - 1 +
  \frac{\alpha i}{2},
\end{equation}
in which case one finds:
\begin{equation}
  R^{(\alpha, \beta)}_n (x) = i^n P^{\left( \beta - 1 + \frac{i}{2} \alpha,
  \beta - 1 - \frac{i}{2} \alpha \right)}_n  (ix),
\end{equation}
(with suitably chosen normalization constants for the Jacobi polynomials). The
complex Jacobi polynomials on the right are defined via Kuijlaars et al.
(2003),{\cite{Kuijlaars2005}} which assures that these are real polynomials in
$x$.

Since the cited authors discuss the non-hermitian (complex) orthogonality
conditions only for real Jacobi indexes, the overlap between their analysis
and definition of Romanovski polynomials exists only if $\alpha = 0$. However,
examination of this peculiar case requires more scrutiny beyond the limits of
this article. Notice the invertibility of:
\begin{equation}
  P^{(\alpha, \beta)}_n (x) = (- i)^n R^{\left( i (\alpha - \beta),
  \frac{1}{2} (\alpha + \beta) + 1 \right)}_n  (- ix),
\end{equation}
where, now, $P^{(\alpha, \beta)}_n (x)$ is a real Jacobi polynomial and:
\[ R^{\left. \left( i (\alpha - \beta), \frac{1}{2} (\alpha + \beta) + 1
   \right) \right)}_n  (- ix) \]
would be a complex Romanovski polynomial.

\section*{Properties of Romanovski Polynomials}

\subsection*{Explicit Construction}

For real $\alpha, \beta$ and $n = 0, 1, 2$,..., a function $R^{(\alpha,
\beta)}_n (x)$ can be defined by the Rodrigues formula as:
\begin{equation}
  R_n^{(\alpha, \beta)} (x) \equiv \frac{1}{w^{(\alpha, \beta)} (x)} 
  \frac{\mathrm{d}^n}{\mathrm{d} x^n}  (w^{(\alpha, \beta)} (x) s (x)^n),
\end{equation}
where $w^{(\alpha, \beta)} (x)$ is the same weight function as in (2), and $s
(x) = 1 + x^2$ is the coefficient of the second derivative of the
hypergeometric differential equation.

Note that we have chosen the normalization constants $N_n = 1$, which is
equivalent to making a choice of the coefficient of highest degree in the
polynomial, as given by the equation:
\begin{equation}
  c_n = \frac{1}{n!}  \prod_{k = 0}^{n - 1} (2 \beta (n - k) + n (n - 1) - k
  (k - 1)), \quad n \geq 1.
\end{equation}
Also, note that the coefficient $c_n$ does not depend on the parameter
$\alpha$, but only on $\beta$ and, for particular values of $\beta$, $c_n$
vanishes (i.e., for all the values:
\[ \beta = \frac{k (k - 1) - n (n - 1)}{2 (n - k)} \]
where $k = 0, ..., n - 1$). This observation poses a problem addressed below.

For later reference, we write explicitly the polynomials of degree 0, 1, and
2,

\begin{align*}
  R_0^{(\alpha, \beta)} (x) & = 1,\\
  R^{(\alpha, \beta)}_1 (x) & = \frac{1}{w^{(\alpha, \beta)} (x)}  \left(
  {w'}^{(\alpha, \beta)} (x) s (x) + s' (x) w^{(\alpha, \beta)} (x) \right)\\
  & = t^{(\alpha, \beta)} (x) = 2 \beta x + \alpha,\\
  R^{(\alpha, \beta)}_2 (x) & = \frac{1}{w^{(\alpha, \beta)} (x)} 
  \frac{\mathrm{d}}{\mathrm{d} x}  \left( s^2 {(x) w'}^{(\alpha, \beta)} (x) +
  2 s (x) s' (x) w^{(\alpha, \beta)} (x) \right)\\
  & = \frac{1}{w^{(\alpha, \beta)} (x)}  \frac{\mathrm{d}}{\mathrm{d} x}  (s
  (x) w^{(\alpha, \beta)} (x) (t^{(\alpha, \beta)} (x) + s' (x)))\\
  & = (2 x + t^{(\alpha, \beta)} (x)) t^{(\alpha, \beta)} (x) + \left( {2 +
  t'}^{(\alpha, \beta)} (x) \right) s (x)\\
  & = (2 \beta + 1)  (2 \beta + 2) x^2 + 2 (2 \beta + 1) \alpha x + (2 \beta
  + \alpha^2 + 2) .
\end{align*}

\subsection*{Orthogonality}

The two polynomials, $R_m^{(\alpha, \beta)} (x)$ and $R_n^{(\alpha, \beta)}
(x)$ with $m \neq n$, are orthogonal,{\cite{Raposo2007}}
\begin{equation}
  \int_{- \infty}^{+ \infty} w^{(\alpha, \beta)} (x) R_m^{(\alpha, \beta)} (x)
  R_n^{(\alpha, \beta)} (x) = 0,
\end{equation}
if and only if,
\begin{equation}
  m + n < 1 - 2 \beta .
\end{equation}
In other words, for arbitrary parameters, only a finite number of Romanovski
polynomials are orthogonal. This property is referred to as finite
orthogonality. However, for some special cases in which the parameters depend
in a particular way on the polynomial degree, infinite orthogonality can be
achieved.

This is the case of a version of the differential equation that has been
independently encountered anew within the context of the exact solubility of
the quantum mechanical problem of the trigonometric Rosen--Morse potential and
reported in Compean Kirchbach (2006).{\cite{Compean2006}} There, the
polynomial parameters $\alpha$ and $\beta$ are no longer arbitrary but are
expressed in terms of the potential parameters, $a$ and $b$, and the degree
$n$ of the polynomial according to the relations,
\begin{equation}
  \alpha \to \alpha_n = \frac{2 b}{n + 1 + a}, \quad \beta \to \beta_n = - (a
  + n + 1) + 1, \quad n = 0, 1, 2, \ldots, \infty .
\end{equation}
Correspondingly, $\lambda_n$ emerges as $\lambda_n = - n (2 a + n - 1)$, while
the weight function takes the shape:
\begin{equation}
  (1 + x^2)^{- (a + n + 1)} \exp \left( - \frac{2 b}{n + a + 1} \tmop{arccot}
  (x) \right) \text{}
\end{equation}
Finally, the one-dimensional variable, $x$, in Compean Kirchbach
(2006){\cite{Compean2006}} has been taken as


\[ x = \cot \left( \frac{r}{d} \right), \]
where $r$ is the radial distance, while $d$ is an appropriate length
parameter. In Compean Kirchbach,{\cite{Compean2006}} it has been shown that
the family of Romanovski polynomials corresponding to the infinite sequence of
parameter pairs,
\begin{equation}
  (\alpha_1, \beta_1), (\alpha_2 \beta_2), \ldots, (\alpha_n \beta_n), \ldots,
  \quad n \longrightarrow \infty,
\end{equation}
is orthogonal.

\subsection*{Generating Function}

In Weber (2007),{\cite{Weber2007}} polynomials $Q_{\nu}^{(\alpha_n, \beta_n +
n)} (x)$, with $\beta_n + n = - a$, and complementary to $R^{(\alpha_n,
\beta_n)}_n (x)$ have been studied, generated in the following way:
\begin{equation}
  Q_{\nu}^{(\alpha_n, \beta_n + n)} (x) = \frac{1}{w^{(\alpha_n, \beta_n + n -
  \nu)}}  \frac{\mathrm{d}^{\nu}}{\mathrm{d} x^{\nu}} w^{(\alpha_n, \beta_n)}
  (x)  (1 + x^2)^n .
\end{equation}
In taking into account the relation,
\begin{equation}
  w^{(\alpha_n, \beta_n)} (x)  (1 + x^2)^{\delta} = w^{(\alpha_n, \beta_n +
  \delta)} (x),
\end{equation}
Equation becomes equivalent to:
\begin{equation}
  Q_{\nu}^{(\alpha_n, \beta_n + n)} (x) = \frac{1}{w^{(\alpha_n, \beta_n + n -
  \nu)}}  \frac{\mathrm{d}^{\nu}}{\mathrm{d} x^{\nu}} w^{(\alpha_n, \beta_n +
  n - \nu)} (x)  (1 + x^2)^{\nu} = R_{\nu}^{(\alpha_n, \beta_n + n - \nu)}
  (x),
\end{equation}
and thus links the complementary to the principal Romanovski polynomials.

The main attraction of the complementary polynomials is that their generating
function can be calculated in closed form.{\cite{Weber2007}} Such a generating
function, written for the Romanovski polynomials based on Equation with the
parameters and therefore referring to infinite orthogonality, has been
introduced as:
\begin{equation}
  G^{(\alpha_n, \beta_n)} (x, y) = \sum_{\nu = 0}^{\infty} R_{\nu}^{(\alpha_n,
  \beta_n + n - \nu)} (x) \frac{y^{\nu}}{\nu !} .
\end{equation}
The notational differences between Weber{\cite{Weber2007}} and those used here
are summarized as follows:
\begin{itemize}
  \item $G^{(\alpha_n, \beta_n)} (x, y)$ here versus $Q (x, y ; \alpha, - a)$
  there, $\alpha$ there in place of $\alpha_n$ here,
  
  \item $a = - \beta_n - n$, and
  
  \item $Q^{(\alpha, - a)}_{\nu} (x)$ in Equation (15) in
  Weber{\cite{Weber2007}} corresponding to $R^{(\alpha_n, \beta_n + n -
  \nu)}_{\nu} (x)$ here.
\end{itemize}
The generating function under discussion obtained in Weber{\cite{Weber2007}}
now reads:
\begin{equation}
  G^{(\alpha_n, \beta_n)} (x, y) = (1 + x^2)^{- \beta_n - n + 1} \exp \left(
  \alpha_n \arccot x \right)  (1 + (x + y (1 + x^2))^2)^{- (- \beta_n - n +
  1)} \exp \left( - \alpha_n \arccot (x + y (1 + x^2)) \right) .
\end{equation}

\subsection*{Recurrence Relations}

Recurrence relations between the infinite orthogonal series of Romanovski
polynomials with the parameters in the above equations follow from the
generating function,{\cite{Weber2007}}
\begin{equation}
  \nu (\nu + 1 - 2 (\beta_n + n)) R_{\nu - 1}^{(\alpha_n, \beta_n + n - \nu +
  1)} (x) + \frac{\mathrm{d}}{\mathrm{d} x} R_{\nu}^{(\alpha_n, \beta_n + n -
  \nu)} (x) = 0,
\end{equation}
and
\begin{equation}
  R^{(\alpha_n, \beta_n + n - \nu - 1)}_{\nu + 1} (x) = (\alpha_n - 2 x (-
  \beta_n - n + \nu + 1)) R_{\nu}^{(\alpha_n, \beta_n + n - \nu)} - \nu (1 +
  x^2)  (2 (- \beta_n - n) + \nu + 1) R_{\nu - 1}^{(\alpha_n, \beta_n + n -
  \nu + 1)},
\end{equation}
as Equations (10) and (23) of Weber (2007){\cite{Weber2007}} respectively.

\section*{See Also}

\begin{itemize}
  \item Associated Legendre functions
  
  \item Gaussian quadrature
  
  \item Gegenbauer polynomials
  
  \item Legendre rational functions
  
  \item Tur{\'a}n's inequalities
  
  \item Legendre wavelet
  
  \item Jacobi polynomials
  
  \item Legendre polynomials
  
  \item Spherical harmonics
  
  \item Trigonometric Rosen--Morse potential
\end{itemize}
\begin{thebibliography}{}
  \ 
\end{thebibliography}

\end{document}
