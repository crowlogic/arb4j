\documentclass{article}
\usepackage{amsmath, amsthm, amssymb}
\usepackage{xcolor}
\usepackage{geometry}
\geometry{margin=1in}
\title{Measure-Preserving Bijective Time Changes of Stationary Gaussian Processes on $\sigma$-Compact Domains: A Subclass of Oscillatory Gaussian Processes}
\author{Anonymous}
\date{\today}

\newtheorem{theorem}{Theorem}[section]
\newtheorem{lemma}[theorem]{Lemma}
\newtheorem{corollary}[theorem]{Corollary}
\newtheorem{definition}[theorem]{Definition}
\newtheorem{remark}[theorem]{Remark}

\begin{document}
\maketitle

\begin{abstract}
This article establishes that Gaussian processes obtained through measure-preserving bijective unitary time transformations of stationary processes constitute a subclass of oscillatory processes in the sense of Priestley. The transformation $Z(t) = \sqrt{\dot{\theta}(t)} X(\theta(t))$, where $X(t)$ represents a realization of a stationary Gaussian process and $\theta$ denotes a strictly increasing $C^1$ differentiable monotonic function, yields an oscillatory process with evolutionary power spectrum $dF_t(\omega) = \dot{\theta}(t)\, d\mu(\omega)$. An explicit unitary transformation between the input stationary process and the transformed oscillatory process is established on the space of functions square-integrable on $\sigma$-compact sets, preserving local $L^2$-norms and providing a complete spectral characterization.
\end{abstract}

\tableofcontents

\section{Introduction}
The theory of non-stationary stochastic processes has applications in signal processing, time series analysis, and mathematical physics. Among the various classes of non-stationary processes, oscillatory processes as defined by Priestley provide an elegant framework for understanding time-varying spectral characteristics. It is shown that a subclass of oscillatory processes can be constructed through measure-preserving bijective time transformations of stationary Gaussian processes, within the natural framework of functions square-integrable on $\sigma$-compact sets.

The main contribution establishes a connection between stationary and oscillatory processes through unitary operators that preserve geometric properties while introducing controlled non-stationarity, providing theoretical insights and methods for generating oscillatory processes with prescribed spectral evolution.

\section{Mathematical Background}

\subsection{Function spaces on $\sigma$-compact sets}

\begin{definition}[$\sigma$-compact sets]\label{def:sigma_compact}
A subset $U \subseteq \mathbb{R}$ is $\sigma$-compact if $U = \bigcup_{n=1}^{\infty} K_n$ with each $K_n$ compact.
\end{definition}

\begin{definition}[Square-integrability on $\sigma$-compact sets]\label{def:L2_sigma_compact}
The space $L^2_{\sigma\text{-comp}}(\mathbb{R})$ consists of measurable $f:\mathbb{R}\to\mathbb{C}$ such that
\[
\int_U |f(x)|^2\,dx < \infty
\]
for every $\sigma$-compact $U \subseteq \mathbb{R}$.
\end{definition}

\begin{remark}\label{rem:sigma_compact_properties}
Every bounded measurable set in $\mathbb{R}$ is $\sigma$-compact. Hence $L^2_{\sigma\text{-comp}}(\mathbb{R})$ contains all functions square-integrable on every bounded interval, including those with polynomial growth at infinity.
\end{remark}

\subsection{Stationary Gaussian processes}

\begin{definition}[Stationary Gaussian process]\label{def:stationary}
A real-valued process $\{X(t)\}_{t\in\mathbb{R}}$ is stationary Gaussian if it admits the spectral representation
\begin{equation}\label{eq:stationary_rep}
X(t) = \int_{-\infty}^{\infty} e^{i\omega t}\, d\Phi(\omega),
\end{equation}
where $\Phi$ is a complex orthogonal-increment random measure with
\begin{equation}
E|d\Phi(\omega)|^2 = d\mu(\omega) = S(\omega)\, d\omega = \frac{1}{2\pi}\int_{-\infty}^{\infty} K(u)\, e^{-i\omega u}\, du,
\end{equation}
$\mu$ is absolutely continuous, and $K$ is the autocovariance. Each realization $X(\cdot)$ belongs to $L^2_{\sigma\text{-comp}}(\mathbb{R})$.
\end{definition}

\subsection{Oscillatory processes}

\begin{definition}[Oscillatory process]\label{def:oscillatory}
A second-order process $\{Z(t)\}_{t\in\mathbb{R}}$ is oscillatory if there exist functions $\phi_t(\omega)=A_t(\omega)e^{i\omega t}$ with $A_t(\cdot)\in L^2(\mu)$ and a complex orthogonal random measure $\Phi$ with $E|d\Phi(\omega)|^2=d\mu(\omega)$, such that
\begin{equation}\label{eq:oscillatory_rep}
Z(t)=\int_{-\infty}^{\infty} \phi_t(\omega)\, d\Phi(\omega)
= \int_{-\infty}^{\infty} A_t(\omega)e^{i\omega t}\, d\Phi(\omega)
= \int_{-\infty}^{\infty} h(t,u)\, X(u)\, du,
\end{equation}
with $h(t,u) = \frac{1}{2\pi}\int_{-\infty}^{\infty} \phi_t(\lambda)\, e^{-i\lambda u}\, d\lambda$ and $X$ as in \eqref{eq:stationary_rep}.
\end{definition}

\subsection{Time scaling functions}

\begin{definition}[Scaling functions]\label{def:scaling}
Let $\mathcal{F}$ be the set of functions $\theta:\mathbb{R}\to\mathbb{R}$ such that:
\begin{enumerate}
\item $\theta$ is absolutely continuous with $\dot{\theta}(t)\ge 0$ almost everywhere and $\dot{\theta}(t)=0$ only on sets of Lebesgue measure zero,
\item $\theta$ is strictly increasing and bijective,
\item $\theta$ maps $\sigma$-compact sets to $\sigma$-compact sets.
\end{enumerate}
\end{definition}

\begin{remark}\label{rem:inverse_properties}
Under Definition~\ref{def:scaling}, $\theta^{-1}$ exists, is absolutely continuous, and satisfies
\[
\frac{d}{ds}(\theta^{-1})(s) = \frac{1}{\dot{\theta}(\theta^{-1}(s))}
\]
for almost all $s$. The reciprocal is well-defined almost everywhere because $\dot{\theta}>0$ almost everywhere.
\end{remark}

\section{The unitary time-change transformation}

\subsection{Operator and inverse}

\begin{definition}[Unitary time-change operator]\label{def:unitary_op}
For $\theta\in\mathcal{F}$, define $M_\theta: L^2_{\sigma\text{-comp}}(\mathbb{R})\to L^2_{\sigma\text{-comp}}(\mathbb{R})$ by
\[
(M_\theta f)(t) = \sqrt{\dot{\theta}(t)}\, f(\theta(t)).
\]
\end{definition}

\begin{definition}[Time-changed stationary process]\label{def:time_changed_proc}
For $\theta\in\mathcal{F}$ and a realization $X$ of a stationary process, define
\[
Z(t) = \sqrt{\dot{\theta}(t)}\, X(\theta(t)), \quad t\in\mathbb{R}.
\]
\end{definition}

\begin{definition}[Inverse operator]\label{def:inverse_unitary_op}
For $g\in L^2_{\sigma\text{-comp}}(\mathbb{R})$, define
\[
(M_\theta^{-1} g)(s) = \frac{g(\theta^{-1}(s))}{\sqrt{\dot{\theta}(\theta^{-1}(s))}}.
\]
\end{definition}

\subsection{Well-definedness and local unitarity}

\begin{lemma}[Well-definedness]\label{lem:inverse_well_defined}
The operator $M_\theta^{-1}$ is well-defined on $L^2_{\sigma\text{-comp}}(\mathbb{R})$.
\end{lemma}

\begin{proof}
Since $\dot{\theta}>0$ almost everywhere and $\theta^{-1}$ maps null sets to null sets, the denominator is positive almost everywhere, hence the expression defines an $L^2_{\sigma\text{-comp}}$ function modulo null sets.
\end{proof}

\begin{theorem}[Local unitarity]\label{thm:local_unitary}
For every $\sigma$-compact $U\subseteq\mathbb{R}$ and $f\in L^2_{\sigma\text{-comp}}(\mathbb{R})$,
\[
\int_U |(M_\theta f)(t)|^2\, dt = \int_{\theta(U)} |f(s)|^2\, ds.
\]
\end{theorem}

\begin{proof}
Compute
\[
\int_U |(M_\theta f)(t)|^2\, dt
= \int_U \dot{\theta}(t)\, |f(\theta(t))|^2\, dt.
\]
With the change of variables $s=\theta(t)$, $ds=\dot{\theta}(t)\,dt$, and $\theta(U)$ $\sigma$-compact, the identity follows.
\end{proof}

\section{Sample path square-integrability}

\begin{theorem}[Sample paths in $L^2_{\sigma\text{-comp}}(\mathbb{R})$]\label{thm:sample_paths_in_L2sigma}
Let $\{X(t)\}_{t\in\mathbb{R}}$ be a second-order stationary Gaussian process with finite second moment $\sigma^2=E[X(t)^2]<\infty$. Then almost surely $X(\omega,\cdot)\in L^2_{\sigma\text{-comp}}(\mathbb{R})$.
\end{theorem}

\begin{proof}
Fix $[a,b]$. Let $Y_{[a,b]}=\int_a^b X(t)^2\,dt$. Then
\[
E[Y_{[a,b]}]=\int_a^b E[X(t)^2]\,dt=\sigma^2(b-a)<\infty.
\]
By Markov's inequality, for any $M>0$,
\[
P\left(Y_{[a,b]}>M\right)\le \frac{E[Y_{[a,b]}]}{M}.
\]
Letting $M\to\infty$ yields $P\left(\int_a^b X(t)^2\,dt<\infty\right)=1$. Since $\mathbb{R}=\bigcup_{n=1}^\infty [-n,n]$ and the above holds for each $n$, countable subadditivity implies
\[
P\left(\bigcap_{n=1}^\infty \left\{\int_{-n}^n X(t)^2\,dt<\infty\right\}\right)=1,
\]
hence almost surely $X(\cdot)\in L^2_{\sigma\text{-comp}}(\mathbb{R})$.
\end{proof}

\section{Main results}

\subsection{Oscillatory representation}

\begin{theorem}[Oscillatory form]\label{thm:osc_rep}
The process $Z(t)=\sqrt{\dot{\theta}(t)}\,X(\theta(t))$ is oscillatory with
\[
\phi_t(\omega)=\sqrt{\dot{\theta}(t)}\, e^{i\omega \theta(t)},\qquad
A_t(\omega)=\sqrt{\dot{\theta}(t)}\, e^{i\omega(\theta(t)-t)}.
\]
\end{theorem}

\begin{proof}
From \eqref{eq:stationary_rep},
\[
Z(t)=\sqrt{\dot{\theta}(t)}\int_{-\infty}^{\infty} e^{i\omega \theta(t)}\, d\Phi(\omega)
=\int_{-\infty}^{\infty} \phi_t(\omega)\, d\Phi(\omega),
\]
with $\phi_t(\omega)$ as stated. Writing $\phi_t(\omega)=A_t(\omega)e^{i\omega t}$ gives the asserted $A_t(\omega)$.
\end{proof}

\subsection{Evolutionary power spectrum}

\begin{corollary}[Evolutionary spectrum]\label{cor:evolving_spec}
The evolutionary power spectrum is
\[
dF_t(\omega)=|A_t(\omega)|^2\, d\mu(\omega)=\dot{\theta}(t)\, d\mu(\omega).
\]
\end{corollary}

\begin{proof}
Since $|e^{i\alpha}|=1$ for all $\alpha\in\mathbb{R}$,
\[
|A_t(\omega)|^2=\left|\sqrt{\dot{\theta}(t)}\, e^{i\omega(\theta(t)-t)}\right|^2=\dot{\theta}(t).
\]
\end{proof}

\subsection{Operator conjugation}

\begin{theorem}[Operator conjugation]\label{thm:operator_conjugation}
Let $(T_K f)(t)=\int_{-\infty}^{\infty} K(|t-s|)\, f(s)\, ds$ with stationary kernel $K(h)=\int_{-\infty}^{\infty} S(\lambda) e^{i\lambda h}\, d\lambda$. Define
\[
(T_{K_\theta} f)(t)=\int_{-\infty}^{\infty} K(|\theta(t)-\theta(s)|)\, \sqrt{\dot{\theta}(t)\dot{\theta}(s)}\, f(s)\, ds.
\]
Then $T_{K_\theta}=M_\theta T_K M_\theta^{-1}$ on $L^2_{\sigma\text{-comp}}(\mathbb{R})$.
\end{theorem}

\begin{proof}
Compute $(M_\theta T_K M_\theta^{-1}g)(t)=\sqrt{\dot{\theta}(t)}\int K(|\theta(t)-\theta(u)|)\, g(u)\sqrt{\dot{\theta}(u)}\, du$, then change variables $s=\theta(u)$ to obtain $T_{K_\theta}$.
\end{proof}

\subsection{Zero-crossing analysis}

\begin{theorem}[Expected zero-count]\label{thm:zero_count}
Let $\theta\in\mathcal{F}$ and $K(\tau)=\mathrm{cov}(X(t),X(t+\tau))$ be twice differentiable at $\tau=0$. For any bounded interval $[a,b]$,
\[
\mathbb{E}[N_{[a,b]}]=\sqrt{-\ddot{K}(0)}\; (\theta(b)-\theta(a)).
\]
\end{theorem}

\begin{proof}
The covariance of $Z$ is $K_\theta(s,t)=\sqrt{\dot{\theta}(s)\dot{\theta}(t)}\, K(|\theta(t)-\theta(s)|)$. Using the Kac–Rice formula on $[a,b]$ and $\dot{K}(0)=0$ for stationary processes gives the stated result.
\end{proof}

\section{Conclusion}
Gaussian processes generated by measure-preserving bijective time changes of stationary processes form a subclass of oscillatory processes on $\sigma$-compact domains. The unitary operator $M_\theta$ preserves local $L^2$ norms, yields an explicit oscillatory representation with envelope $A_t(\omega)=\sqrt{\dot{\theta}(t)} e^{i\omega(\theta(t)-t)}$, produces the evolutionary spectrum $dF_t(\omega)=\dot{\theta}(t)\, d\mu(\omega)$, intertwines covariance operators via $T_{K_\theta}=M_\theta T_K M_\theta^{-1}$, and admits a closed-form expected zero count on bounded intervals.

\begin{thebibliography}{99}
\bibitem{priestley1965} M.B. Priestley. Evolutionary spectra and non-stationary processes. \emph{J. Roy. Stat. Soc. Ser. B}, 27(2):204--237, 1965.
\bibitem{cramer1967} H. Cramér and M.R. Leadbetter. \emph{Stationary and Related Stochastic Processes}. Wiley, 1967.
\bibitem{kac1943} M. Kac. On the average number of real roots of a random algebraic equation. \emph{Bull. Amer. Math. Soc.}, 49(4):314--320, 1943.
\bibitem{rice1945} S.O. Rice. Mathematical analysis of random noise. \emph{Bell Syst. Tech. J.}, 24(1):46--156, 1945.
\end{thebibliography}

\end{document}
