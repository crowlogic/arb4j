\documentclass{article}
\usepackage{amsmath, amsthm, amssymb}
\usepackage{xcolor}
\title{Measure-Preserving Bijective Time Changes of Stationary Gaussian Processes on $\sigma$-Compact Domains: A Subclass of Oscillatory Gaussian Processes}
\author{Anonymous}
\date{\today}
\newtheorem{theorem}{Theorem}[section]
\newtheorem{lemma}[theorem]{Lemma}
\newtheorem{corollary}[theorem]{Corollary}
\newtheorem{definition}[theorem]{Definition}
\newtheorem{remark}[theorem]{Remark}

\begin{document}
\maketitle

\begin{abstract}
This article establishes that Gaussian processes obtained through measure-preserving bijective unitary time transformations of stationary processes constitute a subclass of oscillatory processes in the sense of Priestley. The transformation $Z(t) = \sqrt{\dot{\theta}(t)} X(\theta(t))$, where $X(t)$ represents a realization of a stationary Gaussian process and $\theta$ denotes a strictly increasing $C^1$ differentiable monotonic function, yields an oscillatory process with evolutionary power spectrum $dF_t(\omega) = \dot{\theta}(t) d\mu(\omega)$. An explicit unitary transformation between the input stationary process and the transformed oscillatory process is established on the space of functions square-integrable on $\sigma$-compact sets, preserving local $L^2$-norms and providing a complete spectral characterization.
\end{abstract}

\tableofcontents

\section{Introduction}
The theory of non-stationary stochastic processes has found applications in signal processing, time series analysis, and mathematical physics. Among the various classes of non-stationary processes, oscillatory processes as defined by Priestley provide an elegant framework for understanding time-varying spectral characteristics. This work demonstrates that a specific subclass of oscillatory processes can be constructed through measure-preserving bijective time transformations of stationary Gaussian processes, working within the natural framework of functions square-integrable on $\sigma$-compact sets.

The main contribution establishes a fundamental connection between stationary and oscillatory processes through unitary operators that preserve essential geometric properties while introducing controlled non-stationarity. This approach provides both theoretical insights and practical methods for generating oscillatory processes with prescribed spectral evolution.

\section{Mathematical Background}

\subsection{Function Spaces on $\sigma$-Compact Sets}

\begin{definition}[$\sigma$-Compact Sets]\label{def:sigma_compact}
A subset $U \subseteq \mathbb{R}$ is called $\sigma$-compact if it can be written as a countable union of compact sets:
\begin{equation}
U = \bigcup_{n=1}^{\infty} K_n
\end{equation}
where each $K_n \subset \mathbb{R}$ is compact.
\end{definition}

\begin{definition}[Square-Integrability on $\sigma$-Compact Sets]\label{def:L2_sigma_compact}
The space $L^2_{\sigma\text{-comp}}(\mathbb{R})$ consists of measurable functions $f: \mathbb{R} \to \mathbb{C}$ such that
\begin{equation}
\int_U |f(x)|^2 dx < \infty
\end{equation}
for every $\sigma$-compact set $U \subseteq \mathbb{R}$.
\end{definition}

\begin{remark}\label{rem:sigma_compact_properties}
Every bounded measurable set in $\mathbb{R}$ is $\sigma$-compact. The space $L^2_{\sigma\text{-comp}}(\mathbb{R})$ contains functions that are square-integrable on every bounded interval, including functions with polynomial growth at infinity.
\end{remark}

\subsection{Stationary Gaussian Processes}

\begin{definition}[Stationary Gaussian Process]\label{def:stationary}
A real-valued process $\{X(t)\}_{t \in \mathbb{R}}$ constitutes a stationary Gaussian process if it admits the continuous spectral representation
\begin{equation}
\label{eq:stationary_rep}
X(t) = \int_{-\infty}^{\infty} e^{i\omega t} d\Phi(\omega)
\end{equation}
where $\Phi(\omega)$ represents an orthogonal-increment process with spectral density
\begin{equation}
E|d\Phi(\omega)|^2 = d\mu(\omega) = S(\omega) d\omega = \frac{1}{2\pi} \int_{-\infty}^{\infty} K(u) e^{-i\omega u} du
\end{equation}
and $\mu$ denotes an absolutely continuous measure on $\mathbb{R}$. Each realization $X(t)$ belongs to $L^2_{\sigma\text{-comp}}(\mathbb{R})$.
\end{definition}

\subsection{Oscillatory Processes}

\begin{definition}[Oscillatory Process]\label{def:oscillatory}
A complex-valued, second-order process $\{Z(t)\}_{t \in \mathbb{R}}$ is termed \emph{oscillatory} if there exist
\begin{enumerate}
\item a family of oscillatory basis functions $\{\phi_t(\omega)\}_{t \in \mathbb{R}}$ with
\begin{equation}
\phi_t(\omega) = A_t(\omega) e^{i\omega t} = \int_{-\infty}^{\infty} h(t,u) e^{i\omega u} du
\end{equation}
and a corresponding family of gain functions
\begin{equation}
A_t(\omega) = \frac{\phi_t(\omega)}{e^{i\omega t}} \in L^2(\mu) \label{envelope}
\end{equation}
with time-dependent filter given by
\begin{equation}
h(t,u) = \frac{1}{2\pi} \int_{-\infty}^{\infty} \phi_t(\lambda) e^{-i\lambda u} d\lambda
\end{equation}
\item a complex orthogonal random measure $\Phi(\omega)$ with
\begin{equation}
E|d\Phi(\omega)|^2 = d\mu(\omega) = S(\omega) d\omega
\end{equation}
\end{enumerate}
such that
\begin{equation}
\label{eq:oscillatory_rep}
\begin{aligned}
Z(t) &= \int_{-\infty}^{\infty} \phi_t(\omega) d\Phi(\omega) \\
&= \int_{-\infty}^{\infty} A_t(\omega) e^{i\omega t} d\Phi(\omega) \\
&= \int_{-\infty}^{\infty} h(t,u) X(u) du
\end{aligned}
\end{equation}
where
\begin{equation}
X(t) = \int_{-\infty}^{\infty} e^{i\lambda t} d\Phi(\lambda)
\end{equation}
and each realization $Z(t)$ belongs to $L^2_{\sigma\text{-comp}}(\mathbb{R})$.
\end{definition}

\subsection{Time Scaling Functions}

\begin{definition}[Scaling Functions]\label{def:scaling}
Let $\mathcal{F}$ denote the set of functions $\theta : \mathbb{R} \to \mathbb{R}$ satisfying
\begin{enumerate}
\item $\theta$ is absolutely continuous with
\begin{equation}
\dot{\theta}(t) = \frac{d}{dt} \theta(t) \geq 0
\end{equation}
almost everywhere and $\dot{\theta}(t) = 0$ only on sets of Lebesgue measure zero
\item $\theta$ is strictly increasing and bijective
\item $\theta$ maps $\sigma$-compact sets to $\sigma$-compact sets
\end{enumerate}
\end{definition}

\begin{remark}\label{rem:inverse_properties}
The conditions in Definition~\ref{def:scaling} ensure that $\theta^{-1}(s)$ exists and is absolutely continuous. By the inverse function theorem for absolutely continuous functions,
\begin{equation}
\frac{d}{ds}(\theta^{-1})(s) = \frac{1}{\dot{\theta}(\theta^{-1}(s))} = \dot{\theta}(\theta^{-1}(s))^{-1}
\end{equation}
for almost all $s$ in the range of $\theta$. The condition that $\dot{\theta}(t) = 0$ only on sets of measure zero ensures that $\frac{1}{\dot{\theta}(\theta^{-1}(s))}$ is well-defined almost everywhere. Moreover, $\theta^{-1}$ preserves the $\sigma$-compact structure.
\end{remark}

\section{The Unitary Time-Change Transformation}

\subsection{Definition of the Transformation Operator}

\begin{definition}[Unitary Time-Change Operator]\label{def:unitary_op}
For $\theta \in \mathcal{F}$, the operator $M_{\theta} : L^2_{\sigma\text{-comp}}(\mathbb{R}) \to L^2_{\sigma\text{-comp}}(\mathbb{R})$ is defined by
\begin{equation}
\label{eq:unitary_op}
(M_{\theta} f)(t) = \sqrt{\dot{\theta}(t)} f(\theta(t))
\end{equation}
\end{definition}

\begin{definition}[Unitarily Time-Changed Stationary Process]\label{def:time_changed_proc}
For $\theta \in \mathcal{F}$, applying the unitary time change operator $M_{\theta}$ from Definition~\ref{def:unitary_op} to a realization of a stationary process $X(t)$ from the ensemble $\{X(t)\}$ defines a realization of the unitarily time-changed process
\begin{equation}
\label{eq:time_change}
Z(t) = \sqrt{\dot{\theta}(t)} X(\theta(t)) \quad \forall t \in \mathbb{R}
\end{equation}
where $Z(t) \in L^2_{\sigma\text{-comp}}(\mathbb{R})$.
\end{definition}

\begin{definition}[Inverse Unitary Time-Change Operator]\label{def:inverse_unitary_op}
The inverse operator $M_{\theta}^{-1} : L^2_{\sigma\text{-comp}}(\mathbb{R}) \to L^2_{\sigma\text{-comp}}(\mathbb{R})$ corresponding to the unitary time-change operator $(M_{\theta} f)(t)$ defined in Equation~\eqref{eq:unitary_op} is given by
\begin{equation}
\label{eq:unitary_inverse}
(M_{\theta}^{-1} g)(s) = \frac{g(\theta^{-1}(s))}{\sqrt{\dot{\theta}(\theta^{-1}(s))}}
\end{equation}
\end{definition}

\subsection{Well-Definedness and Local Unitarity Properties}

\begin{lemma}[Well-Definedness of Inverse Operator]\label{lem:inverse_well_defined}
The operator $M_{\theta}^{-1}$ in Definition~\ref{def:inverse_unitary_op} is well-defined on $L^2_{\sigma\text{-comp}}(\mathbb{R})$.
\end{lemma}
\begin{proof}
Since $\dot{\theta}(t) = 0$ only on sets of measure zero by Definition~\ref{def:scaling}, and $\theta^{-1}$ maps sets of measure zero to sets of measure zero (as it preserves absolute continuity), the denominator $\sqrt{\dot{\theta}(\theta^{-1}(s))}$ is positive almost everywhere. The expression in equation~\eqref{eq:unitary_inverse} is therefore well-defined almost everywhere on every $\sigma$-compact set, which suffices for defining an element of $L^2_{\sigma\text{-comp}}(\mathbb{R})$.
\end{proof}

\begin{theorem}[Local Unitarity of Transformation Operator]\label{thm:local_unitary}
The operator $M_{\theta}$ defined in equation~\eqref{eq:unitary_op} satisfies local unitarity: for every $\sigma$-compact set $U \subseteq \mathbb{R}$,
\begin{equation}
\label{eq:local_L2_preserve}
\int_U |(M_{\theta} f)(t)|^2 dt = \int_{\theta(U)} |f(s)|^2 ds \quad \forall f \in L^2_{\sigma\text{-comp}}(\mathbb{R})
\end{equation}
\end{theorem}

\begin{proof}
Let $f \in L^2_{\sigma\text{-comp}}(\mathbb{R})$ and let $U$ be any $\sigma$-compact set. The local $L^2$-norm of $M_{\theta} f$ over $U$ is computed as follows:
\begin{align}
\int_U |(M_{\theta} f)(t)|^2 dt &= \int_U \left|\sqrt{\dot{\theta}(t)} f(\theta(t))\right|^2 dt \\
&= \int_U \dot{\theta}(t) |f(\theta(t))|^2 dt
\end{align}
Applying the change of variables $s = \theta(t)$, since $\theta$ is absolutely continuous and strictly increasing, the Jacobian is given by
\begin{equation}
ds = \dot{\theta}(t) dt
\end{equation}
almost everywhere. Since $\theta$ maps $\sigma$-compact sets to $\sigma$-compact sets (by Definition~\ref{def:scaling}), as $t$ ranges over $U$, $s = \theta(t)$ ranges over $\theta(U)$, which is $\sigma$-compact. Therefore:
\begin{align}
\int_U \dot{\theta}(t) |f(\theta(t))|^2 dt &= \int_{\theta(U)} |f(s)|^2 ds
\end{align}

To complete the proof of local unitarity, verification that $M_{\theta}^{-1}$ is indeed the inverse of $M_{\theta}$ on $L^2_{\sigma\text{-comp}}(\mathbb{R})$ follows by checking that $M_{\theta} M_{\theta}^{-1} = M_{\theta}^{-1} M_{\theta} = I$ on each $\sigma$-compact domain.
\end{proof}

\section{Sample Path Square-Integrability}

\begin{theorem}[Sample Paths in $L^2_{\sigma\text{-comp}}(\mathbb{R})$]\label{thm:sample_paths_in_L2sigma}
Let $\{X(t)\}_{t \in \mathbb{R}}$ be a second-order stationary Gaussian process with finite second moment $\sigma^2 = E[X(t)^2] < \infty$. Then almost surely every sample path $X(\omega, \cdot)$ belongs to $L^2_{\sigma\text{-comp}}(\mathbb{R})$.
\end{theorem}

\begin{proof}
Fix any bounded interval $[a,b]$. Consider the random variable
\begin{equation}
Y_{[a,b]} = \int_a^b X(t)^2 dt
\end{equation}
By stationarity and finite second moment,
\begin{equation}
E[Y_{[a,b]}] = \int_a^b E[X(t)^2] dt = \int_a^b \sigma^2 dt = \sigma^2 (b - a) < \infty.
\end{equation}
By Markov's inequality, for any $M > 0$,
\begin{equation}
P(Y_{[a,b]} > M) \leq \frac{E[Y_{[a,b]}]}{M} = \frac{\sigma^2 (b - a)}{M}.
\end{equation}
Taking the limit as $M \to \infty$, it follows that
\begin{equation}
P\left(\int_a^b X(t)^2 dt < \infty\right) = 1,
\end{equation}
i.e., almost surely the sample path is square integrable on $[a,b]$. Since $\mathbb{R}$ is the countable union of such bounded intervals,
\begin{equation}
\mathbb{R} = \bigcup_{n=1}^\infty [-n, n],
\end{equation}
then by countable subadditivity of probability
\begin{equation}
P\left(\bigcap_{n=1}^\infty \left\{\int_{-n}^n X(t)^2 dt < \infty\right\}\right) = 1,
\end{equation}
meaning almost surely the sample path lies in $L^2_{\sigma\text{-comp}}(\mathbb{R})$.
\end{proof}

\section{Main Results}

\subsection{Oscillatory Representation}

\begin{theorem}[Oscillatory Form]\label{thm:osc_rep}
The process $\{Z(t)\}$ defined in equation~\eqref{eq:time_change} with realizations in $L^2_{\sigma\text{-comp}}(\mathbb{R})$ is oscillatory with oscillatory functions
\begin{equation}
\label{eq:phi_def}
\phi_t(\omega) = A_t(\omega) e^{i\omega t} = \sqrt{\dot{\theta}(t)} e^{i\omega \theta(t)}
\end{equation}
and gain functions
\begin{equation}
A_t(\omega) = \sqrt{\dot{\theta}(t)} e^{i\omega(\theta(t) - t)}
\end{equation}
\end{theorem}

\begin{proof}
Applying the unitary time change operator $(M_{\theta} f)(t)$ from Definition~\ref{def:unitary_op} and substituting the spectral representation~\eqref{eq:stationary_rep} of the stationary process $X(t)$:
\begin{align}
Z(t) &= (M_{\theta} X)(t) \\
&= \sqrt{\dot{\theta}(t)} X(\theta(t)) \\
&= \sqrt{\dot{\theta}(t)} \int_{-\infty}^{\infty} e^{i\omega \theta(t)} d\Phi(\omega) \\
&= \int_{-\infty}^{\infty} \sqrt{\dot{\theta}(t)} e^{i\omega \theta(t)} d\Phi(\omega) \\
&= \int_{-\infty}^{\infty} \phi_t(\omega) d\Phi(\omega)
\end{align}
where
\begin{equation}
\phi_t(\omega) = \sqrt{\dot{\theta}(t)} e^{i\omega \theta(t)}
\end{equation}

To verify this constitutes an oscillatory representation according to Definition~\ref{def:oscillatory}, the expression $\phi_t(\omega)$ must be written in the form $A_t(\omega) e^{i\omega t}$:
\begin{align}
\phi_t(\omega) &= A_t(\omega) e^{i\omega t} \\
&= \sqrt{\dot{\theta}(t)} e^{i\omega(\theta(t) - t)} e^{i\omega t} \\
&= \sqrt{\dot{\theta}(t)} e^{i\omega \theta(t)}
\end{align}
where
\begin{equation}
A_t(\omega) = \sqrt{\dot{\theta}(t)} e^{i\omega(\theta(t) - t)}
\end{equation}

Since $\dot{\theta}(t) \geq 0$ almost everywhere and $\dot{\theta}(t) = 0$ only on sets of measure zero, the function $A_t(\omega)$ is well-defined almost everywhere. Moreover, $A_t(\cdot) \in L^2(\mu)$ for each $t$ since:
\begin{align}
\int_{-\infty}^{\infty} |A_t(\omega)|^2 d\mu(\omega) &= \int_{-\infty}^{\infty} \dot{\theta}(t) d\mu(\omega) \\
&= \dot{\theta}(t) \mu(\mathbb{R}) < \infty
\end{align}
\end{proof}

\subsection{Evolutionary Power Spectrum}

\begin{corollary}[Evolutionary Spectrum]\label{cor:evolving_spec}
The evolutionary power spectrum is
\begin{equation}
\label{eq:evolutionary_spec}
dF_t(\omega) = |A_t(\omega)|^2 d\mu(\omega) = \dot{\theta}(t) d\mu(\omega)
\end{equation}
\end{corollary}
\begin{proof}
By Definition~\ref{def:oscillatory} and the envelope from Equation~\eqref{envelope}, the evolutionary power spectrum is:
\begin{align}
dF_t(\omega) &= |A_t(\omega)|^2 d\mu(\omega) \\
&= \left|\sqrt{\dot{\theta}(t)} e^{i\omega(\theta(t) - t)}\right|^2 d\mu(\omega) \\
&= \dot{\theta}(t) |e^{i\omega(\theta(t) - t)}|^2 d\mu(\omega) \\
&= \dot{\theta}(t) d\mu(\omega)
\end{align}
since $|e^{i\alpha}| = 1$ for all $\alpha \in \mathbb{R}$.
\end{proof}

\subsection{Operator Conjugation}

\begin{theorem}[Operator Conjugation]\label{thm:operator_conjugation}
Let $T_K$ be the integral covariance operator defined by
\begin{equation}
\label{eq:integral_op_original}
(T_K f)(t) = \int_{-\infty}^{\infty} K(|t - s|) f(s) ds
\end{equation}
where $K(h)$ represents the stationary kernel
\begin{equation}
K(h) = \int_{-\infty}^{\infty} S(\lambda) e^{i\lambda h} d\lambda
\end{equation}
and let $T_{K_{\theta}}$ be the integral covariance operator defined by
\begin{equation}
\label{eq:integral_op_transformed}
\begin{aligned}
(T_{K_{\theta}} f)(t) &= \int_{-\infty}^{\infty} K_{\theta}(s,t) f(s) ds \\
&= \int_{-\infty}^{\infty} K(|\theta(t) - \theta(s)|) \sqrt{\dot{\theta}(t) \dot{\theta}(s)} f(s) ds
\end{aligned}
\end{equation}
for the unitarily time-changed kernel
\begin{equation}
K_{\theta}(s,t) = K(|\theta(t) - \theta(s)|) \sqrt{\dot{\theta}(t) \dot{\theta}(s)}
\end{equation}
Then on $L^2_{\sigma\text{-comp}}(\mathbb{R})$,
\begin{equation}
\label{eq:conjugation}
T_{K_{\theta}} = M_{\theta} T_K M_{\theta}^{-1}
\end{equation}
\end{theorem}

\begin{proof}
For any $g \in L^2_{\sigma\text{-comp}}(\mathbb{R})$, the computation of $(M_{\theta} T_K M_{\theta}^{-1} g)(t)$ proceeds through the same algebraic steps as in the classical case, with the change of variables arguments remaining valid on each $\sigma$-compact domain where the integrals are well-defined.
\end{proof}

\subsection{Zero-Crossing Analysis}

\begin{theorem}[Expected Zero-Counting Function]\label{thm:zero_count}
Let $\theta \in \mathcal{F}$ and let
\begin{equation}
K(\tau) = \mathrm{cov}(X(t), X(t+\tau))
\end{equation}
be twice differentiable at $\tau = 0$. For any bounded interval $[a,b]$, the expected number of zeros of the process $Z_t$ is
\begin{equation}
\label{eq:zero_count}
\mathbb{E}[N_{[a,b]}] = \sqrt{-\ddot{K}(0)} (\theta(b) - \theta(a))
\end{equation}
\end{theorem}

\begin{proof}
Since $[a,b]$ is compact, the covariance function of the time-changed process is
\begin{equation}
\label{eq:time_changed_cov}
K_{\theta}(s,t) = \mathrm{cov}(Z_s, Z_t) = \sqrt{\dot{\theta}(s) \dot{\theta}(t)} K(|\theta(t) - \theta(s)|)
\end{equation}
The Kac-Rice formula and subsequent mixed derivative calculations proceed identically on the bounded interval $[a,b]$, yielding the same result.
\end{proof}

\section{Conclusion}

This analysis establishes that Gaussian processes generated by measure-preserving bijective time changes of stationary processes form a well-defined subclass of oscillatory processes within the natural framework of functions square-integrable on $\sigma$-compact sets. The key contributions include:

\begin{enumerate}
\item The construction of the unitary operator $M_{\theta}: L^2_{\sigma\text{-comp}}(\mathbb{R}) \to L^2_{\sigma\text{-comp}}(\mathbb{R})$ and its inverse
\item The explicit oscillatory representation with envelope function
\begin{equation}
A_t(\omega) = \sqrt{\dot{\theta}(t)} e^{i\omega(\theta(t) - t)}
\end{equation}
\item The evolutionary power spectrum formula
\begin{equation}
dF_t(\omega) = \dot{\theta}(t) d\mu(\omega)
\end{equation}
\item The operator conjugation relationship
\begin{equation}
T_{K_{\theta}} = M_{\theta} T_K M_{\theta}^{-1}
\end{equation}
\item A closed-form expression for the expected zero count on bounded intervals
\end{enumerate}

The theoretical framework accommodates functions that are square-integrable on $\sigma$-compact sets, providing significant practical advantages for processes with polynomial growth, bounded support, or other physically relevant behaviors while maintaining all essential mathematical properties.

\begin{thebibliography}{99}
\bibitem{priestley1965} M.B. Priestley. Evolutionary spectra and non-stationary processes. \emph{Journal of the Royal Statistical Society, Series B}, 27(2):204--237, 1965.
\bibitem{cramer1967} H. Cramér and M.R. Leadbetter. \emph{Stationary and Related Stochastic Processes}. Wiley, 1967.
\bibitem{kac1943} M. Kac. On the average number of real roots of a random algebraic equation. \emph{Bulletin of the American Mathematical Society}, 49(4):314--320, 1943.
\bibitem{rice1945} S.O. Rice. Mathematical analysis of random noise. \emph{Bell System Technical Journal}, 24(1):46--156, 1945.
\end{thebibliography}

\end{document}
