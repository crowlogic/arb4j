\documentclass{article}
\usepackage[english]{babel}
\usepackage{geometry,amsmath,amssymb,latexsym,theorem}
\geometry{letterpaper}

%%%%%%%%%% Start TeXmacs macros
\newcommand{\cdummy}{\cdot}
\newcommand{\tmemail}[1]{\\ \textit{Email:} \texttt{#1}}
\newcommand{\tmtextit}[1]{\text{{\itshape{#1}}}}
\newenvironment{proof}{\noindent\textbf{Proof\ }}{\hspace*{\fill}$\Box$\medskip}
\newtheorem{corollary}{Corollary}
\newtheorem{definition}{Definition}
\newtheorem{lemma}{Lemma}
{\theorembodyfont{\rmfamily}\newtheorem{remark}{Remark}}
\newtheorem{theorem}{Theorem}
%%%%%%%%%% End TeXmacs macros

\begin{document}

\title{Measure-Preserving Bijective Time Changes of Stationary Gaussian
Processes Constitute a Subclass of Oscillatory Gaussian Processes}

\author{
  Stephen Crowley
  \tmemail{stephencrowley214@gmail.com}
}

\date{August 1, 2025}

\maketitle

\begin{abstract}
  This article establishes that Gaussian processes obtained through
  measure-preserving bijective unitary time transformations of stationary
  processes constitute a subclass of oscillatory processes in the sense of
  Priestley. The transformation $Z (t) = \sqrt{\dot{\theta} (t)} X (\theta
  (t))$, where $X (t)$ is a realization of stationary Gaussian process and
  $\theta$ is a strictly increasing $C^1$ differentiable monotonic function,
  yields an oscillatory process with evolutionary power spectrum $dF_t
  (\omega) = \dot{\theta} (t) d \mu (\omega)$. An explicit unitary
  transformation between the input stationary process and the transformed
  oscillatory process is established, preserving the $L^2$-norm and providing
  a complete spectral characterization.
\end{abstract}

{\tableofcontents}

\section{Preliminaries}

\subsection{Scaling functions}\label{sec:scaling}

\begin{definition}[Scaling Functions]
  \label{def:scaling}Let $\mathcal{F}$ denote the set of functions $\theta :
  \mathbb{R} \to \mathbb{R}$ satisfying
  \begin{enumerate}
    \item $\theta$ is absolutely continuous with
    \begin{equation}
      \dot{\theta} (t) = \frac{\mathrm{d}}{\mathrm{d} t} \theta (t) \geq 0
    \end{equation}
    almost everywhere and $\dot{\theta} (t) = 0$ only on sets of Lebesgue
    measure zero.
    
    \item $\theta$ is strictly increasing and bijective.
  \end{enumerate}
\end{definition}

\begin{remark}
  \label{rem:inverse_properties}The conditions in Definition~\ref{def:scaling}
  ensure that $\theta^{- 1} (s)$ exists and is absolutely continuous. By the
  inverse function theorem for absolutely continuous functions,
  \begin{equation}
    \frac{\mathrm{d}}{\mathrm{d} s} (\theta^{- 1}) (s) = \frac{1}{\dot{\theta}
    (\theta^{- 1} (s))} = \dot{\theta} (\theta^{- 1} (s))^{- 1}
  \end{equation}
  for almost all $s$ in the range of $\theta$. The condition that
  $\dot{\theta} (t) = 0$ only on sets of measure zero ensures that
  $\frac{1}{\dot{\theta} (\theta^{- 1} (s))}$ is well-defined almost
  everywhere.
\end{remark}

\subsection{Stationary Gaussian reference process}\label{sec:stationary}

\begin{definition}[Stationary Process]
  A real-valued process $\{X (t)\}_{t \in \mathbb{R}}$ is a stationary
  Gaussian process if it can be represented by the continuous spectral
  representation
  \begin{equation}
    \label{eq:stationary_rep} X (t) = \int_{- \infty}^{\infty} e^{i \omega t} 
    \hspace{0.17em} d \Phi (\omega)
  \end{equation}
  where $\Phi (\omega)$ is an orthogonal-increment process with spectral
  density
  \begin{equation}
    E \lvert d \Phi (\omega) \rvert^2 = d \mu (\omega) = S (\omega) =
    \frac{1}{2 \pi}  \int_{- \infty}^{\infty} K (u) e^{- i \omega u}
    \hspace{0.17em} \mathrm{d} u = \dot{\mu} (\omega) =
    \frac{\mathrm{d}}{\mathrm{d} \omega} \mu (\omega)
  \end{equation}
  and $\mu$ is an absolutely continuous Lebesgue measure on $\mathbb{R}$.
\end{definition}

\section{Unitary time change and basic properties}\label{sec:time_change}

\subsection{Definition and inverse}

\begin{definition}[Unitary Time-Change Operator]
  \label{M}For $\theta \in \mathcal{F}$, define the operator $M_{\theta} : L^2
  (\mathbb{R}) \to L^2 (\mathbb{R})$ by
  \begin{equation}
    \label{eq:unitary_op} (M_{\theta} f) (t) = \sqrt{\dot{\theta} (t)} 
    \hspace{0.17em} f (\theta (t))
  \end{equation}
\end{definition}

\begin{definition}[Unitarily Time-Changed Stationary Process]
  \label{def:time_changed_proc}For $\theta \in \mathcal{F}$, apply
  $M_{\theta}$ from Definition~\ref{M} to a realization of a stationary
  process $X (t)$ from the ensemble $\{X (t)\}$ to define a realization of the
  unitarily time-changed process
  \begin{equation}
    \label{eq:time_change} Z (t) = \sqrt{\dot{\theta} (t)}  \hspace{0.17em} X
    (\theta (t)) \quad \forall t \in \mathbb{R}
  \end{equation}
\end{definition}

\begin{definition}[Inverse Unitary Time-Change Operator]
  \label{def:inverse_unitary_op}The inverse operator $M_{\theta}^{- 1} : L^2
  (\mathbb{R}) \to L^2 (\mathbb{R})$ corresponding to $M_{\theta}$ in
  \eqref{eq:unitary_op} is
  \begin{equation}
    \label{eq:unitary_inverse} (M_{\theta}^{- 1} g) (s) = \frac{g (\theta^{-
    1} (s))}{\sqrt{\dot{\theta} (\theta^{- 1} (s))}}
  \end{equation}
\end{definition}

\begin{lemma}[Well-Definedness of Inverse Operator]
  \label{lem:inverse_well_defined}The operator $M_{\theta}^{- 1}$ in
  Definition~\ref{def:inverse_unitary_op} is well-defined $\forall \theta \in
  \mathcal{F}$.
\end{lemma}

\begin{proof}
  Since $\dot{\theta} (t) = 0$ only on sets of measure zero by
  Definition~\ref{def:scaling}, and $\theta^{- 1}$ maps sets of measure zero
  to sets of measure zero (as it preserves absolute continuity), the
  denominator $\sqrt{\dot{\theta} (\theta^{- 1} (s))}$ is positive almost
  everywhere. The expression in equation~\eqref{eq:unitary_inverse} is
  therefore well-defined almost everywhere, which is sufficient for defining
  an element of $L^2 (\mathbb{R})$.
\end{proof}

\subsection{Unitarity and measure preservation}

\begin{theorem}[Unitarity of Transformation Operator]
  \label{thm:unitary}The operator $M_{\theta}$ defined in
  \eqref{eq:unitary_op} is unitary, i.e.,
  \begin{equation}
    \label{eq:L2_preserve} \int_{\mathbb{R}} \lvert (M_{\theta} f) (t)
    \rvert^2 \hspace{0.17em} \mathrm{d} t = \int_{\mathbb{R}} \lvert f (s)
    \rvert^2 \hspace{0.17em} \mathrm{d} s \quad \forall f \in L^2 (\mathbb{R})
  \end{equation}
\end{theorem}

\begin{proof}
  Let $f \in L^2 (\mathbb{R})$. Then
  
  \begin{align}
    \int_{\mathbb{R}} \lvert (M_{\theta} f) (t) \rvert^2 \hspace{0.17em}
    \mathrm{d} t & = \int_{\mathbb{R}} \left| \sqrt{\dot{\theta} (t)} 
    \hspace{0.17em} f (\theta (t)) \right|^2 \hspace{0.17em} \mathrm{d} t \\
    & = \int_{\mathbb{R}} \dot{\theta} (t) \hspace{0.17em} \lvert f (\theta
    (t)) \rvert^2 \hspace{0.17em} \mathrm{d} t 
  \end{align}
  
  Apply the change of variables $s = \theta (t)$. Since $\theta$ is absolutely
  continuous and strictly increasing, $ds = \dot{\theta} (t)  \hspace{0.17em}
  dt$ almost everywhere. As $t$ ranges over $\mathbb{R}$, $s = \theta (t)$
  ranges over $\mathbb{R}$ due to bijectivity. Therefore:
  
  \begin{align}
    \int_{\mathbb{R}} \dot{\theta} (t) \hspace{0.17em} \lvert f (\theta (t))
    \rvert^2 \hspace{0.17em} \mathrm{d} t & = \int_{\mathbb{R}} \lvert f (s)
    \rvert^2 \hspace{0.17em} \mathrm{d} s 
  \end{align}
  
  To show that $M_{\theta}^{- 1}$ is the inverse of $M_{\theta}$, for any $f
  \in L^2 (\mathbb{R})$:
  
  \begin{align}
    (M_{\theta}^{- 1} M_{\theta} f) (s) & = \frac{\sqrt{\dot{\theta}
    (\theta^{- 1} (s))}  \hspace{0.17em} f (\theta (\theta^{- 1}
    (s)))}{\sqrt{\dot{\theta} (\theta^{- 1} (s))}} = f (s) 
  \end{align}
  
  Similarly, for any $g \in L^2 (\mathbb{R})$:
  
  \begin{align}
    (M_{\theta} M_{\theta}^{- 1} g) (t) & = \sqrt{\dot{\theta} (t)}  \frac{g
    (\theta^{- 1} (\theta (t)))}{\sqrt{\dot{\theta} (\theta^{- 1} (\theta
    (t)))}} = g (t) 
  \end{align}
  
  Hence $M_{\theta} M_{\theta}^{- 1} = M_{\theta}^{- 1} M_{\theta} = I$.
\end{proof}

\begin{corollary}[Measure Preservation on Sets]
  \label{cor:measure_preserve}The transformation $M_{\theta}$ preserves the
  $L^2$-measure in the sense that for any measurable set $A \subseteq
  \mathbb{R}$,
  \begin{equation}
    \label{eq:measure_preserve_sets} \int_A \lvert (M_{\theta} f) (t) \rvert^2
    \hspace{0.17em} \mathrm{d} t = \int_{\theta (A)} \lvert f (s) \rvert^2
    \hspace{0.17em} \mathrm{d} s
  \end{equation}
\end{corollary}

\begin{proof}
  This follows by the same change-of-variables argument as in
  Theorem~\ref{thm:unitary}, applied to the characteristic function of $A$.
\end{proof}

\subsection{$L^2$-norm preservation for
processes}\label{sec:norm_preservation}

\begin{theorem}[Measure Preservation]
  \label{thm:measure_preserve}The transformation defined in
  \eqref{eq:time_change} preserves the $L^2$-norm in the sense that
  \begin{equation}
    \label{eq:measure_preserve} \int_I \mathrm{var} (Z (t)) \hspace{0.17em}
    \mathrm{d} t = \int_{\theta (I)} \mathrm{var} (X (s)) \hspace{0.17em}
    \mathrm{d} s
  \end{equation}
  for any measurable set $I \subseteq \mathbb{R}$.
\end{theorem}

\begin{proof}
  Using the change of variables $s = \theta (t)$ with $ds = \dot{\theta} (t) 
  \hspace{0.17em} dt$:
  
  \begin{align}
    \int_I \mathrm{var} \hspace{-0.17em} \left( \sqrt{\dot{\theta} (t)} 
    \hspace{0.17em} X (\theta (t)) \right) \mathrm{d} t & = \int_I
    \dot{\theta} (t) \hspace{0.17em} \mathrm{var} (X (\theta (t)))
    \hspace{0.17em} \mathrm{d} t \\
    & = \int_{\theta (I)} \mathrm{var} (X (s)) \hspace{0.17em} \mathrm{d} s 
  \end{align}
\end{proof}

\section{Oscillatory processes and representation}

An oscillatory process can be represented as a time-dependent filter applied
to a stationary process.

\begin{definition}[Oscillatory Process]
  \label{def:oscillatory}A complex-valued, second-order process $\{Z (t)\}_{t
  \in \mathbb{R}}$ is called oscillatory if there exist:
  \begin{enumerate}
    \item A family of oscillatory basis functions $\{\phi_t (\omega)\}_{t \in
    \mathbb{R}}$ with
    \begin{equation}
      \begin{aligned}
        \phi_t (\omega) & = A_t (\omega) e^{i \omega t}\\
        & = \int_{- \infty}^{\infty} h (t, u)  \hspace{0.17em} e^{i \omega u}
        \hspace{0.17em} \mathrm{d} u,
      \end{aligned}
    \end{equation}
    and a given family of gain functions
    \begin{equation}
      A_t (\omega) = \frac{\phi_t (\omega)}{e^{i \omega t}} \in L^2 (\mu)
      \label{envelope}
    \end{equation}
    with time-dependent filter given by the inverse transform
    \begin{equation}
      h (t, u) = \frac{1}{2 \pi}  \int_{- \infty}^{\infty} \phi_t (\lambda) 
      \hspace{0.17em} e^{- i \lambda u} \hspace{0.17em} \mathrm{d} \lambda .
    \end{equation}
    \item A complex orthogonal random measure $\Phi (\omega)$ with
    \begin{equation}
      E \lvert d \Phi (\omega) \rvert^2 = d \mu (\omega) = S (\omega)
    \end{equation}
  \end{enumerate}
  such that
  \begin{equation}
    \label{eq:oscillatory_rep}
    
    \begin{aligned}
      Z (t) & = \int_{- \infty}^{\infty} \phi_t (\omega)  \hspace{0.17em} d
      \Phi (\omega)\\
      & = \int_{- \infty}^{\infty} A_t (\omega) e^{i \omega t} 
      \hspace{0.17em} d \Phi (\omega)\\
      & = \int_{- \infty}^{\infty} h (t, u)  \hspace{0.17em} X (u)
      \hspace{0.17em} \mathrm{d} u,
    \end{aligned}
  \end{equation}
  where
  \begin{equation}
    X (t) = \int_{- \infty}^{\infty} e^{i \lambda t} \hspace{0.17em}
    \mathrm{d} \Phi (\lambda)
  \end{equation}
\end{definition}

\subsection{Oscillatory representation of the time-changed process}

\begin{theorem}[Oscillatory Form]
  \label{thm:osc_rep}The process $\{Z (t)\}$ defined in \eqref{eq:time_change}
  is oscillatory with oscillatory functions
  \begin{equation}
    \label{eq:phi_def} \phi_t (\omega) = A_t (\omega) e^{i \omega t} =
    \sqrt{\dot{\theta} (t)}  \hspace{0.17em} e^{i \omega \theta (t)}
  \end{equation}
  and gain functions
  \begin{equation}
    A_t (\omega) = \sqrt{\dot{\theta} (t)}  \hspace{0.17em} e^{i \omega
    (\theta (t) - t)}
  \end{equation}
\end{theorem}

\begin{proof}
  Apply the unitary time change operator $M_{\theta}$ in Definition~\ref{M}
  and substitute the spectral representation~\eqref{eq:stationary_rep} of the
  stationary process $X (t)$:
  
  \begin{align}
    Z (t) & = (M_{\theta} X) (t) \nonumber\\
    & = \sqrt{\dot{\theta} (t)}  \hspace{0.17em} X (\theta (t)) \\
    & = \sqrt{\dot{\theta} (t)}  \int_{- \infty}^{\infty} e^{i \omega \theta
    (t)}  \hspace{0.17em} d \Phi (\omega) \\
    & = \int_{- \infty}^{\infty} \sqrt{\dot{\theta} (t)}  \hspace{0.17em}
    e^{i \omega \theta (t)}  \hspace{0.17em} d \Phi (\omega) \\
    & = \int_{- \infty}^{\infty} \phi_t (\omega)  \hspace{0.17em} d \Phi
    (\omega) 
  \end{align}
  
  where $\phi_t (\omega) = \sqrt{\dot{\theta} (t)}  \hspace{0.17em} e^{i
  \omega \theta (t)}$. To verify this is an oscillatory representation
  according to Definition~\ref{def:oscillatory}, express $\phi_t (\omega)$ in
  the form $A_t (\omega) e^{i \omega t}$:
  \begin{equation}
    \begin{aligned}
      \phi_t (\omega) & = A_t (\omega)  \hspace{0.17em} e^{i \omega t}\\
      & = \sqrt{\dot{\theta} (t)}  \hspace{0.17em} e^{i \omega (\theta (t) -
      t)}  \hspace{0.17em} e^{i \omega t}\\
      & = \sqrt{\dot{\theta} (t)}  \hspace{0.17em} e^{i \omega \theta (t)}
    \end{aligned}
  \end{equation}
  Since $\dot{\theta} (t) \geq 0$ almost everywhere and $\dot{\theta} (t) = 0$
  only on sets of measure zero, the function $A_t (\omega)$ is well-defined
  almost everywhere. Moreover, $A_t (\cdummy) \in L^2 (\mu)$ for each $t$
  since:
  
  \begin{align}
    \int_{- \infty}^{\infty} \lvert A_t (\omega) \rvert^2  \hspace{0.17em} d
    \mu (\omega) & = \int_{- \infty}^{\infty} \dot{\theta} (t) 
    \hspace{0.17em} d \mu (\omega) \\
    & = \dot{\theta} (t)  \int_{- \infty}^{\infty} d \mu (\omega) \\
    & = \dot{\theta} (t)  \hspace{0.17em} \mu (\mathbb{R}) < \infty 
  \end{align}
  
  where finiteness follows from $\mu$ being a finite measure and $\dot{\theta}
  (t)$ being finite almost everywhere.
\end{proof}

\subsection{Envelope and evolutionary spectrum}

\begin{corollary}[Evolutionary Spectrum]
  \label{cor:evolving_spec}The evolutionary power spectrum is
  \begin{equation}
    \label{eq:evolutionary_spec} dF_t (\omega) = \lvert A_t (\omega) \rvert^2 
    \hspace{0.17em} d \mu (\omega) = \dot{\theta} (t)  \hspace{0.17em} d \mu
    (\omega) .
  \end{equation}
\end{corollary}

\begin{proof}
  By Definition~\ref{def:oscillatory} and \eqref{envelope},
  
  \begin{align}
    dF_t (\omega) & = \lvert A_t (\omega) \rvert^2  \hspace{0.17em} d \mu
    (\omega) \\
    & = \left| \sqrt{\dot{\theta} (t)}  \hspace{0.17em} e^{i \omega (\theta
    (t) - t)} \right|^2  \hspace{0.17em} d \mu (\omega) \\
    & = \dot{\theta} (t) \hspace{0.17em} |e^{i \omega (\theta (t) - t)} |^2 
    \hspace{0.17em} d \mu (\omega) \\
    & = \dot{\theta} (t)  \hspace{0.17em} d \mu (\omega) 
  \end{align}
  
  since $|e^{i \alpha} | = 1$ for all $\alpha \in \mathbb{R}$.
\end{proof}

\section{Operator conjugation}\label{sec:conjugation}

\begin{theorem}[Operator Conjugation]
  \label{thm:operator_conjugation}Let $T_K$ be the integral covariance
  operator defined by
  \begin{equation}
    \label{eq:integral_op_original} (T_K f) (t) = \int_{- \infty}^{\infty} K
    (|t - s|)  \hspace{0.17em} f (s) \hspace{0.17em} \mathrm{d} s
  \end{equation}
  where $K (h) = \int_{- \infty}^{\infty} S (\lambda) e^{i \lambda h}
  \hspace{0.17em} \mathrm{d} \lambda$. Let $T_{K_{\theta}}$ be the integral
  covariance operator defined by
  \begin{equation}
    \label{eq:integral_op_transformed} (T_{K_{\theta}} f) (t) = \int_{-
    \infty}^{\infty} K_{\theta} (s, t)  \hspace{0.17em} f (s) \hspace{0.17em}
    \mathrm{d} s = \int_{- \infty}^{\infty} K (| \theta (t) - \theta (s) |)
    \sqrt{\dot{\theta} (t)  \hspace{0.17em} \dot{\theta} (s)}  \hspace{0.17em}
    f (s) \hspace{0.17em} \mathrm{d} s
  \end{equation}
  for the unitarily time-changed kernel
  \begin{equation}
    K_{\theta} (s, t) = K (| \theta (t) - \theta (s) |) \sqrt{\dot{\theta} (t)
    \hspace{0.17em} \dot{\theta} (s)}
  \end{equation}
  Then
  \begin{equation}
    \label{eq:conjugation} T_{K_{\theta}} = M_{\theta}  \hspace{0.17em} T_K 
    \hspace{0.17em} M_{\theta}^{- 1}
  \end{equation}
\end{theorem}

\begin{proof}
  For any $g \in L^2 (\mathbb{R})$, compute $(M_{\theta} T_K M_{\theta}^{- 1}
  g) (t)$:
  
  \begin{align}
    (M_{\theta}^{- 1} g) (s) & = \frac{g (\theta^{- 1}
    (s))}{\sqrt{\dot{\theta} (\theta^{- 1} (s))}}, \\
    (T_K M_{\theta}^{- 1} g) (t) & = \int_{- \infty}^{\infty} K (|t - s|)
    \frac{g (\theta^{- 1} (s))}{\sqrt{\dot{\theta} (\theta^{- 1} (s))}}
    \hspace{0.17em} \mathrm{d} s. 
  \end{align}
  
  Apply the change of variables $u = \theta^{- 1} (s)$, so $s = \theta (u)$
  and $\mathrm{d} s = \dot{\theta} (u) \hspace{0.17em} \mathrm{d} u$:
  
  \begin{align}
    (T_K M_{\theta}^{- 1} g) (t) & = \int_{- \infty}^{\infty} K (|t - \theta
    (u) |) \frac{g (u)}{\sqrt{\dot{\theta} (u)}}  \hspace{0.17em} \dot{\theta}
    (u) \hspace{0.17em} \mathrm{d} u \\
    & = \int_{- \infty}^{\infty} K (|t - \theta (u) |)  \hspace{0.17em} g (u)
    \sqrt{\dot{\theta} (u)} \hspace{0.17em} \mathrm{d} u 
  \end{align}
  
  Now apply $M_{\theta}$:
  
  \begin{align}
    (M_{\theta} T_K M_{\theta}^{- 1} g) (t) & = \sqrt{\dot{\theta} (t)} 
    \hspace{0.17em} (T_K M_{\theta}^{- 1} g) (\theta (t)) \\
    & = \sqrt{\dot{\theta} (t)}  \int_{- \infty}^{\infty} K (| \theta (t) -
    \theta (u) |)  \hspace{0.17em} g (u) \sqrt{\dot{\theta} (u)}
    \hspace{0.17em} \mathrm{d} u 
  \end{align}
  
  Apply the change of variables $s = \theta (u)$ in the reverse direction:
  
  \begin{align}
    (M_{\theta} T_K M_{\theta}^{- 1} g) (t) & = \int_{- \infty}^{\infty} K (|
    \theta (t) - \theta (s) |)  \hspace{0.17em} g (s) \hspace{0.17em}
    \mathrm{d} s \\
    & = (T_{K_{\theta}} g) (t) . 
  \end{align}
  
  This establishes the conjugation relation~\eqref{eq:conjugation}.
\end{proof}

\section{Expected zero count}\label{sec:zero_count}

\begin{theorem}[Expected Zero-Counting Function]
  \label{thm:zero_count}Let $\theta \in \mathcal{F}$ and let
  \begin{equation}
    K (\tau) = \mathrm{cov} (X (t), X (\tau))
  \end{equation}
  be twice differentiable at $\tau = 0$. The expected number of zeros of the
  process $X_t$ in $[a, b]$ is
  \begin{equation}
    \label{eq:zero_count} \mathbb{E} [N_{[a, b]}] = \sqrt{- \ddot{K} (0)} 
    \hspace{0.17em} (\theta (b) - \theta (a))
  \end{equation}
\end{theorem}

\begin{proof}
  The covariance function of the time-changed process is
  \begin{equation}
    \label{eq:time_changed_cov} K_{\theta} (s, t) = \mathrm{cov} (X_s, X_t) =
    \sqrt{\dot{\theta} (s)  \hspace{0.17em} \dot{\theta} (t)}  \hspace{0.17em}
    K (| \theta (t) - \theta (s) |)
  \end{equation}
  For the zero-crossing analysis, consider the normalized process. By the
  Kac-Rice formula:
  \begin{equation}
    \label{eq:kac_rice} \mathbb{E} [N_{[a, b]}] = \int_a^b \sqrt{- \lim_{s \to
    t}  \frac{\partial^2}{\partial s \hspace{0.17em} \partial t} K_{\theta}
    (s, t)} \hspace{0.17em} \mathrm{d} t
  \end{equation}
  Computing the mixed partial derivative:
  
  \begin{align}
    \frac{\partial}{\partial t} K_{\theta} (s, t) & = \frac{1}{2} 
    \frac{\ddot{\theta} (t)}{\sqrt{\dot{\theta} (t)}}  \sqrt{\dot{\theta} (s)}
    \hspace{0.17em} K (| \theta (t) - \theta (s) |) \\
    & \quad + \sqrt{\dot{\theta} (s)  \hspace{0.17em} \dot{\theta} (t)} 
    \hspace{0.17em} \dot{K} (| \theta (t) - \theta (s) |) \hspace{0.17em}
    \mathrm{sgn} (\theta (t) - \theta (s))  \hspace{0.17em} \dot{\theta} (t) .
    
  \end{align}
  
  Taking the limit as $s \to t$ and using $\dot{K} (0) = 0$ for stationary
  processes:
  
  \begin{align}
    \lim_{s \to t}  \frac{\partial^2}{\partial s \hspace{0.17em} \partial t}
    K_{\theta} (s, t) & = \dot{\theta} (s)  \hspace{0.17em} \dot{\theta} (t) 
    \hspace{0.17em} \ddot{K} (0) \\
    & = \dot{\theta} (t)^2  \hspace{0.17em} \ddot{K} (0) 
  \end{align}
  
  Substituting into the Kac-Rice formula:
  
  \begin{align}
    \mathbb{E} [N_{[a, b]}] & = \int_a^b \sqrt{- \dot{\theta} (t)^2 
    \hspace{0.17em} \ddot{K} (0)} \hspace{0.17em} \mathrm{d} t \\
    & = \sqrt{- \ddot{K} (0)}  \int_a^b \dot{\theta} (t) \hspace{0.17em}
    \mathrm{d} t \\
    & = \sqrt{- \ddot{K} (0)}  \hspace{0.17em} (\theta (b) - \theta (a)) 
  \end{align}
  
  where $\dot{\theta} (t) \ge 0$ almost everywhere was used.
\end{proof}

\section{Conclusion}\label{sec:conclusion}

This analysis establishes that Gaussian processes generated by
measure-preserving bijective time changes of stationary processes form a
well-defined subclass of oscillatory processes. The key contributions include:
\begin{enumerate}
  \item The construction of the unitary operator $M_{\theta}$ and its inverse,
  with proper treatment of the case where $\dot{\theta} (t) = 0$ on sets of
  measure zero.
  
  \item The explicit oscillatory representation with envelope function $A_t
  (\omega) = \sqrt{\dot{\theta} (t)}  \hspace{0.17em} e^{i \omega (\theta (t)
  - t)}$.
  
  \item The evolutionary power spectrum formula $dF_t (\omega) = \dot{\theta}
  (t)  \hspace{0.17em} d \mu (\omega)$.
  
  \item The operator conjugation relationship $T_{K_{\theta}} = M_{\theta} 
  \hspace{0.17em} T_K  \hspace{0.17em} M_{\theta}^{- 1}$.
  
  \item A closed-form expression for the expected zero count in terms of the
  range of the time scaling function.
\end{enumerate}
\begin{thebibliography}{99}
  {\bibitem{priestley1965}}M.{\hspace{0.17em}}B. Priestley. Evolutionary
  spectra and non-stationary processes. \tmtextit{Journal of the Royal
  Statistical Society, Series B}, 27(2):204--237, 1965.
  
  {\bibitem{cramer1967}}H. Cram{\'e}r and M.{\hspace{0.17em}}R. Leadbetter.
  \tmtextit{Stationary and Related Stochastic Processes}. Wiley, 1967.
  
  {\bibitem{kac1943}}M. Kac. On the average number of real roots of a random
  algebraic equation. \tmtextit{Bulletin of the American Mathematical
  Society}, 49(4):314--320, 1943.
  
  {\bibitem{rice1945}}S.{\hspace{0.17em}}O. Rice. Mathematical analysis of
  random noise. \tmtextit{Bell System Technical Journal}, 24(1):46--156, 1945.
\end{thebibliography}

\

\end{document}
