\documentclass{article}
\usepackage{amsmath,amssymb,amsthm}
\usepackage[utf8]{inputenc}
\usepackage[T1]{fontenc}

% Theorem environments
\newtheorem{theorem}{Theorem}
\newtheorem{proposition}{Proposition}
\theoremstyle{definition}
\newtheorem{remark}{Remark}

\title{STRONG LAW OF LARGE NUMBERS FOR WEAKLY HARMONIZABLE PROCESSES}
\author{Dominique DEHAY\\
Laboratoire de Statistique et Probabilités (M.2), Université des Sciences et Techniques de Lille Flandres Artois, Cité Scientifique, 59655, Villeneuve D'Ascq, France}
\date{Received 5 September 1985\\
Revised 4 February 1987}

\begin{document}
\maketitle

\begin{abstract}
If $X: \mathbb{R} \rightarrow L_{\mathbb{C}}^{2}(\Omega, \mathscr{A}, P)$ is a weakly harmonizable process with spectral stochastic measure $\mu: \mathscr{B}_{\mathbf{R}} \rightarrow L_{\mathbb{C}}^{2}(\Omega, \mathscr{A}, P)$, we first prove that
\begin{equation}\label{eq:1}
\lim_{t \rightarrow+\infty} \frac{1}{2 t} \int_{-t}^{t} X(s) \mathrm{d} s=\mu(0) \quad \text{a.s.}
\end{equation}
if and only if there exists some integer $p \geqslant 2$ such that
\begin{equation}\label{eq:2}
\lim_{n \rightarrow+\infty} \mu\left(|u|<p^{-n}\right)=\mu(0) \quad \text{a.s.}
\end{equation}

As a consequence we then get criteria for the strong law of large numbers for the process $X$ to hold, i.e.
\begin{equation}\label{eq:3}
\lim_{t \rightarrow+\infty} \frac{1}{2 t} \int_{-t}^{t} X(s) \mathrm{d} s=0 \quad \text{a.s.}
\end{equation}

These are extensions to the weakly harmonizable case of results previously obtained by several authors and specially by Gaposhkin in the strongly harmonizable case.

\noindent\textit{harmonizable processes * stochastic measures * bimeasures * strong laws of large numbers}
\end{abstract}

\section{Introduction}
\subsection{}

Let $(\Omega, \mathscr{A}, P)$ be a probability space. A weakly harmonizable process $X: \mathbb{R} \rightarrow L_{\mathbb{C}}^{2}(\Omega, \mathscr{A}, P)$ is the Fourier transform of a stochastic measure i.e. a $\sigma$-additive set function $\mu: \mathscr{B}_{\mathbb{R}} \rightarrow L_{\mathbb{C}}^{2}(\Omega, \mathscr{A}, P)$, which is called its spectral stochastic measure.

The spectral bimeasure of $X$ is the complex function defined on $\mathscr{B}_{\mathbb{R}} \times \mathscr{B}_{\mathbb{R}}$ by
\begin{equation}\label{eq:4}
M(A \times B)=E(\mu(A) \cdot \overline{\mu(B)}), \quad A, B \in \mathscr{B}_{\mathrm{R}}
\end{equation}

The process $X$ is called strongly harmonizable if its spectral bimeasure $M$ is extendable to a measure (known as its spectral measure) on $\mathscr{B}_{\mathbb{R}} \otimes \mathscr{B}_{\mathbb{R}}$. More particularly, if $M$ concentrates on the diagonal $\Delta$ of $\mathbb{R} \times \mathbb{R}$, i.e. if
\begin{equation}\label{eq:5}
M(B)=M(B \cap \Delta), \quad B \in \mathscr{B}_{\mathbb{R}} \otimes \mathscr{B}_{\mathbf{R}},
\end{equation}
$X$ is a continuous (in q.m.) (wide sense) stationary process, and conversely.

It is well known that there exist non extendable spectral bimeasures (e.g., [6, Example 1]). We will overcome the technical problems generated by this difficulty with the help of the following Miamee and Salehi's domination lemma [7]: for every spectral bimeasure $M: \mathscr{B}_{\mathbb{R}} \times \mathscr{B}_{\mathbb{R}} \rightarrow \mathbb{C}$ there exists a bounded non-negative measure $m$ on $(\mathbb{R}, \mathscr{B}_{\mathbb{R}})$ such that for any bounded measurable function $f: \mathbb{R} \rightarrow \mathbb{C}$, one has
\begin{equation}\label{eq:6}
0 \leqslant \iint f(t) \cdot \overline{f(s)} M(\mathrm{~d} t, \mathrm{~d} s) \leqslant \int|f(t)|^{2} m(\mathrm{~d} t)
\end{equation}
where we use the concept of integration w.r.t. the bimeasure $M$ as introduced by Moché [8, Chapter IV] (see also [11]).

\subsection{The mean process}

Let $X$ be a weakly harmonizable process. Since it is continuous, we can suppose that it is measurable and has locally integrable sample paths (if $X$ is separable, it actually has such a measurable modification). Therefore we can define a new second order process $\left.\sigma_{X}:\right] 0,+\infty\left[\rightarrow L_{\mathbb{C}}^{2}(\Omega, \mathscr{A}, P)\right.$ called the (time averaged) mean process of $X$ such that
\begin{equation}\label{eq:7}
\sigma_{X}(t)=\frac{1}{2 t} \int_{-t}^{t} X(s) \mathrm{d} s, \quad t>0 \text{ (strong } L^{2} \text{-integral), }
\end{equation}
\begin{equation}\label{eq:8}
\sigma_{X}(t, \omega)=\frac{1}{2 t} \int_{-t}^{t} X(s, \omega) \mathrm{d} s, \quad t>0, \omega \in \Omega
\end{equation}

\subsection{Convergence of the mean process-previous results}

We recall that one has [11, inversion formulae]
\begin{equation}\label{eq:9}
\sigma_{X}(t)=\int_{\mathbb{R}} \frac{\sin (t u)}{t u} \cdot \mu(\mathrm{d} u) \underset{t\rightarrow+\infty}{\longrightarrow} \mu(0) \quad \text{(in q.m.).}
\end{equation}

Does this result remain true for the a.s. convergence?
If we put
\begin{equation}\label{eq:10}
\sigma_{X}(t)=\Psi_{X}(t)+\mu\left(|u|<2^{-n}\right), \quad 2^{n}+1<t \leqslant 2^{n+1}+1,
\end{equation}

Gaposhkin proved [5] that in the strongly harmonizable case, one has
\begin{equation}\label{eq:11}
\Psi_{X}(t) \underset{t \rightarrow+\infty}{\longrightarrow} 0 \quad \text{a.s}
\end{equation}

So he obtained that $X$ obeys the strong law of large numbers (SLLN) if and only if one has
\begin{equation}\label{eq:12}
\mu\left(|u|<2^{-n}\right) \underset{n \rightarrow+\infty}{\longrightarrow} 0 \quad \text{a.s.}
\end{equation}

He also deduced [4] that in the continuous stationary case, if $M$ denotes the spectral bimeasure of $X$ as well as its trace on $\Delta$, $X$ obeys the SLLN if the following conditions are both fulfilled:
\begin{equation}\label{eq:13}
M(0)=0 \text{ and there exists a real number } u_{0}>0 \text{ such that }
\end{equation}
\begin{equation}\label{eq:14}
\int_{\left(0<|u|<u_{0}\right)}\left(\log \log \frac{1}{|u|}\right)^{2} M(\mathrm{~d} u)<+\infty.
\end{equation}

Various other criteria for the SLLN in the strongly harmonizable case had been previously settled [1,2,10]. All of them use the total variation measure of $M$ and consequently are not applicable if $M$ is not extendable to a spectral measure [9, 2. Harmonizability].

\subsection{The new results}

In Section 2 we extend (\ref{eq:2}) and (\ref{eq:3}) to the weakly harmonizable case while Section 3 is devoted to the SLLN. More particularly, theorem 3.2 is an extension of the criterion (\ref{eq:13}) from the continuous stationary case to the weakly harmonizable one.

\section{Asymptotic behaviour of the mean process}
\subsection{}

Let $X$ be a weakly harmonizable process, $p$ an integer $\geqslant 2$, and let us put
\begin{equation}\label{eq:15}
\begin{aligned}
\sigma_{X}(t)&=\Psi_{X}(p, t)+\mu\left(|u|<p^{-q}\right) \\
&\quad t>p+1, q \in \mathbb{N} \backslash(0), p^{q}+1<t \leqslant p^{q+1}+1,
\end{aligned}
\end{equation}
\begin{equation}\label{eq:16}
\begin{aligned}
\Psi_{X}(p, t)&=\left(\sigma_{X}(t)-\sigma_{X}(n)\right)+\left(\sigma_{X}(n)-\sigma_{X}\left(p^{q}\right)\right)+\left(\sigma_{X}\left(p^{q}\right)-\mu\left(|u|<p^{-q}\right)\right) \\
&\quad n, q \in \mathbb{N} \backslash(0), n<t \leqslant n+1, \quad p^{q}<n \leqslant p^{q+1}.
\end{aligned}
\end{equation}

We are going to prove that each term of the right-hand side of (\ref{eq:16}) converges almost surely to 0 as $t$ tends to infinity. This is already done for the first term since Rousseau has proved [10, Prop. 1] that
\begin{equation}\label{eq:17}
\operatorname{Sup}\left(\left|\sigma_{X}(t)-\sigma_{X}(n)\right| ; n<t \leqslant n+1\right) \underset{n \rightarrow+\infty}{\longrightarrow} 0 \quad \text{a.s.}
\end{equation}

\subsection{}

For the second term, we have
\begin{proposition}\label{prop:1}
$\operatorname{Lim}_{q \rightarrow+\infty} \operatorname{Max}\left(\left|\sigma_{X}(n)-\sigma_{X}\left(p^{q}\right)\right| ; p^{q}<n \leqslant p^{q+1}\right)=0$ a.s.
\end{proposition}

\begin{proof}
In order to simplify the proof, let us suppose that $p=2$ (for $p>2$, see [3, Chapter 3]).

For every integer $q \geqslant 1, k$ such that $1 \leqslant k \leqslant q$ and every $\boldsymbol{e} \in E(k)=\{0,1\}^{k}$, we put
\begin{equation}\label{eq:18}
\begin{aligned}
a(q, k, e)&=2^{q}+1+\sum_{j=1}^{k} e_{j} 2^{q-j}, \\
b(q, k, e)&= 
\begin{cases}
2^{q}+1+\sum_{j=1}^{k-1} e_{j} 2^{q-j} & \text{if } k \geqslant 2, \\
2^{q} & \text{if } k=1,
\end{cases}
\end{aligned}
\end{equation}
and let $(\alpha_{k}, k \geqslant 1)$ be a sequence of strictly positive numbers. Utilizing Rousseau's majorization lemma [10], we can deduce that
\begin{equation}\label{eq:19}
\begin{aligned}
&\operatorname{Max}\left(\left|\sigma_{X}(n)-\sigma_{X}\left(2^{q}\right)\right|^{2} ; 2^{q}<n \leqslant 2^{q+1}\right) \\
&\quad \leqslant\left(\sum_{k=1}^{q} \alpha_{k}^{-1}\right)\left(\sum_{k=1}^{q} \alpha_{k}\left(\sum_{e \in E(k)}\left|\sum_{j=b(q, q, k e)+1}^{a(q, k, e)}\left(\sigma_{X}(j)-\sigma_{X}(j-1)\right)\right|^{2}\right)\right) \\
&\quad=\left(\sum_{k=1}^{q} \alpha_{k}^{-1}\right)\left(\sum_{k=1}^{q} \alpha_{k}\left(\sum_{e \in E(k)} \mid \sigma_{X}\left(a(q, k, e)-\left.\sigma_{X}(b(q, k, e))\right|^{2}\right)\right) \right.
\end{aligned}
\end{equation}

Then we obtain, by integration,
\begin{equation}\label{eq:20}
\begin{aligned}
E&\left(\operatorname{Max}\left(\left|\sigma_{X}(n)-\sigma_{X}\left(2^{q}\right)\right|^{2} ; 2^{q}<n \leqslant 2^{q+1}\right)\right) \\
&\leqslant\left(\sum_{k=1}^{q} \alpha_{k}^{-1}\right)\left(\sum_{k=1}^{q} 2^{k} \alpha_{k} \cdot \operatorname{Max}\left(E\left(\left|\sigma_{X}(a(q, k, e))-\sigma_{X}(b(q, k, e))\right|^{2}\right) ; e \in E(k)\right)\right) \\
&=\left(\sum_{k=1}^{q} \alpha_{k}^{-1}\right)\left(\sum_{k=1}^{q} 2^{k} \alpha_{k} \operatorname{Max}\left(\iint f_{q, k, e}(u) \cdot f_{q, k, e}(v) M(\mathrm{~d} u, \mathrm{~d} v) ; e \in E(k)\right)\right)
\end{aligned}
\end{equation}
from (\ref{eq:4}), where
\begin{equation}\label{eq:21}
f_{q, k, e}(u)=\frac{\sin (a(q, k, e) \cdot u)}{a(q, k, e) \cdot u}-\frac{\sin (b(q, k, e) \cdot u)}{b(q, k, e) \cdot u}, \quad u \in \mathbb{R}.
\end{equation}

Now we use the key idea of the proof i.e. we reduce the problem to the classical stationary case through the domination lemma: there exists a bounded non-negative measure $m$ on $(\mathbb{R}, \mathscr{B}_{\mathbf{R}})$ associated to the spectral bimeasure $M$ such that
\begin{equation}\label{eq:22}
0 \leqslant \iint f_{q, k, e}(u) \cdot f_{q, k, e}(v) M(\mathrm{~d} u, \mathrm{~d} v) \leqslant \int f_{q, k, e}^{2}(u) m(\mathrm{~d} u)
\end{equation}

So we have
\begin{equation}\label{eq:23}
\begin{aligned}
&\sum_{q=1}^{+\infty} E\left(\operatorname{Max}\left|\sigma_{X}(n)-\sigma_{X}\left(2^{q}\right)\right|^{2} ; 2^{q}<n \leqslant 2^{q+1}\right) \\
&\quad \leqslant \sum_{q=1}^{+\infty}\left(\sum_{k=1}^{q} \alpha_{k}^{-1}\right)\left(\sum_{k=1}^{q} 2^{k} \alpha_{k} \operatorname{Max}\left(\int f_{q, k, e}^{2}(u) m(\mathrm{~d} u) ; e \in E(k)\right)\right)
\end{aligned}
\end{equation}

The end of the proof is not new: if we divide the integration domain of the last integral into the following four parts:
\begin{equation}\label{eq:24}
\left(|u|<2^{-q-1}\right), \quad\left(2^{-q-1} \leqslant|u|<2^{-q+k}\right), \quad\left(2^{-q+k} \leqslant|u|<1\right), \quad(1 \leqslant|u|),
\end{equation}
and if $\alpha_{k}=\alpha^{k}, k \geqslant 1,1<\alpha<2$, it appears four convergent series [4, Theorem $1 ; 10$, Prop. 4] so that the series (\ref{eq:23}) is also convergent. We can obviously conclude that
\begin{equation}\label{eq:25}
\max \left(\left|\sigma_{X}(n)-\sigma_{X}\left(2^{q}\right)\right| ; 2^{q}<n \leqslant 2^{q+1}\right) \underset{q \rightarrow+\infty}{\longrightarrow} 0 \text{ a.s.}
\end{equation}
\end{proof}

\subsection{}

It is easy to prove that
\begin{equation}\label{eq:26}
\sigma_{X}\left(p^{q}\right)-\mu\left(|u|<p^{-q}\right) \underset{q \rightarrow+\infty}{\longrightarrow} 0 \quad \text{a.s.}
\end{equation}
but one will need once again the domination lemma:
\begin{equation}\label{eq:27}
\begin{aligned}
&\sigma_{X}\left(p^{q}\right)-\mu\left(|u|<p^{-q}\right) \\
&\quad=\int_{\left(p^{-q} \leqslant|u|\right)} \frac{\sin \left(p^{q} u\right)}{p^{q} u} \mu(\mathrm{d} u)+\int_{\left(|u|<p^{-q}\right)}\left(\frac{\sin \left(p^{q} u\right)}{p^{q} u}-1\right) \mu(\mathrm{d} u);
\end{aligned}
\end{equation}
\begin{equation}\label{eq:28}
\begin{aligned}
&E\left(\left|\sigma_{X}\left(p^{q}\right)-\mu\left(|u|<p^{-q}\right)\right|^{2}\right) \\
&\quad \leqslant 2 \iint_{\left(p^{-q} \leqslant|u|,|v|\right)} \frac{\sin \left(p^{q} u\right)}{p^{q} u} \cdot \frac{\sin \left(p^{q} v\right)}{p^{q} v} M(\mathrm{~d} u, \mathrm{~d} v) \\
&\quad+2 \iint_{\left(|u|,|v|<p^{-q}\right)}\left(\frac{\sin \left(p^{q} u\right)}{p^{q} u}-1\right)\left(\frac{\sin \left(p^{q} v\right)}{p^{q} v}-1\right) M(\mathrm{~d} u, \mathrm{~d} v) \\
&\quad\leqslant 2 \int_{\left(p^{-q} \leqslant|u|\right)}\left(\frac{\sin \left(p^{q} u\right)}{p^{q} u}\right)^{2} m(\mathrm{~d} u)+2 \int_{\left(|u|<p^{-q}\right)}\left(\frac{\sin \left(p^{q} u\right)}{p^{q} u}-1\right)^{2} m(\mathrm{~d} u).
\end{aligned}
\end{equation}

So we are now in the stationary case from which [4, Theorem 1] we can prove that
\begin{equation}\label{eq:29}
\sum_{q=1}^{+\infty} E\left(\left|\sigma_{X}\left(p^{q}\right)-\mu\left(|u|<p^{-q}\right)\right|^{2}\right)<+\infty
\end{equation}
so that (\ref{eq:26}) is true.

At last we can summarize (\ref{eq:15}), (\ref{eq:16}), (\ref{eq:17}), (\ref{eq:25}) and (\ref{eq:26}) by the following theorem.

\subsection{}

\begin{theorem}\label{thm:1}
For every weakly harmonizable process $X$ and every integer $p \geqslant 2$, one has
\begin{equation}\label{eq:30}
\Psi_{X}(p, t) \underset{t \rightarrow+\infty}{\longrightarrow} 0 \quad \text{a.s.}
\end{equation}
so that the following two conditions are equivalent:

(i) $\sigma_{X}(t)$ converges a.s. as tends to infinity.

(ii) there exists an integer $p \geqslant 2$ such that $\mu\left(|u|<p^{-q}\right)$ converges a.s. when $q$ tends to infinity.

Moreover, one then has, for every integer $p \geqslant 2$,
\begin{equation}\label{eq:31}
\lim_{t \rightarrow+\infty} \sigma_{X}(t)=\lim_{q \rightarrow+\infty} \mu\left(|u|<p^{-q}\right)=\mu(0) \quad \text{a.s.}
\end{equation}
\end{theorem}

\section{Criteria for the SLLN}
\subsection{}

The next statement is an obvious corollary of the theorem in 2.4.

\begin{theorem}\label{thm:2}
Let $X$ be a weakly harmonizable process: it obeys the SLLN if and only if there exists an integer $p \geqslant 2$ such that:
\begin{equation}\label{eq:32}
\lim_{q \rightarrow+\infty} \mu\left(|u|<p^{-q}\right)=0 \quad \text{a.s.}
\end{equation}
\end{theorem}

\subsection{}

We can now give an extension of Gaposhkin's criterion (\ref{eq:13}):

\begin{theorem}\label{thm:3}
Let $X$ be a weakly harmonizable process. If there exists a bounded nonnegative measure $M_{0}$ on $(\mathbb{R}^{2}, \mathscr{B}_{\mathbb{R}^{2}})$ such that

(i) for every event $A$ of the ring generated by the intervals, one has
\begin{equation}\label{eq:33}
M(A \times A) \leqslant M_{0}(A \times A)
\end{equation}

(ii) there exists a real number $u_{0}>0$ such that
\begin{equation}\label{eq:34}
\iint_{\left(0<|u|,|v|<u_{0}\right)}\left(\log \log \frac{1}{|u|}\right)\left(\log \log \frac{1}{|v|}\right) M_{0}(\mathrm{~d} u, \mathrm{~d} v)<+\infty
\end{equation}
then one has
\begin{equation}\label{eq:35}
\sigma_{X}(t) \underset{t \rightarrow+\infty}{\longrightarrow} \mu(0) \quad \text{a.s.,}
\end{equation}
and $X$ obeys the $SLLN$ if and only if $M(0,0)=0$.
\end{theorem}

\begin{proof}
(a) The theorem in 2.4 shows that we have only to prove that
\begin{equation}\label{eq:36}
\lim_{n \rightarrow+\infty} \mu\left(|u|<2^{-n}\right)=\mu(0) \quad \text{a.s.}
\end{equation}

More particularly, since we have
\begin{equation}\label{eq:37}
\mu\left(|u|<2^{-n}\right)=\mu(0)+\mu\left(0<|u|<2^{-2^{q}}\right)-\mu\left(2^{-n} \leqslant|u|<2^{-2^{q}}\right), \quad 2^{q}<n \leqslant 2^{q+1}
\end{equation}
we have only to prove that the last two terms converge to 0 a.s. as $n$ tends to infinity.

(b) Let $q_{0}$ be an integer such that $2^{-2^{q_{0}}}<u_{0}$. Putting $B_{q}=\left(0<|u|<2^{-2^{q}}\right)$ and utilizing (i) and (ii), we obtain
\begin{equation}\label{eq:38}
\begin{aligned}
&\sum_{q=q_{0}}^{+\infty} E\left(\left|\mu\left(B_{q}\right)\right|^{2}\right) \leqslant \sum_{q=q_{0}}^{+\infty} M_{0}\left(B_{q} \times B_{q}\right) \\
&\quad \leqslant \sum_{q=q_{0}}^{+\infty} q^{-2} \iint_{B_{q} \times B_{q}}\left(\log_{2} \log_{2} \frac{1}{|u|}\right)\left(\log_{2} \log_{2} \frac{1}{|v|}\right) M_{0}(\mathrm{~d} u, \mathrm{~d} v) \\
&\quad \leqslant\left(\sum_{q=q_{0}}^{+\infty} q^{-2}\right) \iint_{\left(0<|u|,|v|<u_{0}\right)}\left(\log_{2} \log_{2} \frac{1}{|u|}\right)\left(\log_{2} \log_{2} \frac{1}{|v|}\right) M_{0}(\mathrm{~d} u, \mathrm{~d} v) \\
&\quad<+\infty
\end{aligned}
\end{equation}
where $\log_{2}$ is the $\log$ function to the base 2. Therefore we have
\begin{equation}\label{eq:39}
\mu\left(0<|u|<2^{-2^{q}}\right) \underset{q \rightarrow+\infty}{\longrightarrow} 0 \text{ a.s.}
\end{equation}

(c) Using the previous notations, we put
\begin{equation}\label{eq:40}
A(q, k, e)=\left(2^{-a(q, k, e)} \leqslant|u|<2^{-b(q, k, e)}\right)
\end{equation}
and
\begin{equation}\label{eq:41}
C_{q}=\left(2^{-2^{q+1}} \leqslant|u|<2^{-2^{q}}\right).
\end{equation}

By means of (i), (ii) and the Rousseau's majorization lemma, one has
\begin{equation}\label{eq:42}
\begin{aligned}
&\sum_{q=q_{0}}^{+\infty} E\left(\operatorname{Max}\left(\left|\mu\left(2^{-n} \leqslant|u|<2^{-2^{q}}\right)\right|^{2} ; 2^{q}<n \leqslant 2^{q+1}\right)\right) \\
&\quad \leqslant \sum_{q=q_{0}}^{+\infty} q\left(\sum_{k=1}^{q}\left(\sum_{e \in E(k)} E\left(|\mu(A(q, k, e))|^{2}\right)\right)\right) \\
&\quad \leqslant \sum_{q=q_{0}}^{+\infty} q\left(\sum_{k=1}^{q}\left(\sum_{e \in E(k)} M_{0}(A(q, k, e) \times A(q, k, e))\right)\right) \\
&\quad \leqslant \sum_{q=q_{0}}^{+\infty} q\left(\sum_{k=1}^{q} M_{0}\left(C_{q} \times C_{q}\right)\right)=\sum_{q=q_{0}}^{+\infty} q^{2} M_{0}\left(C_{q} \times C_{q}\right) \\
&\quad \leqslant \sum_{q=q_{0}}^{+\infty} \iint_{C_{q} \times C_{q}}\left(\log_{2} \log_{2} \frac{1}{|u|}\right)\left(\log_{2} \log_{2} \frac{1}{|v|}\right) M_{0}(\mathrm{~d} u, \mathrm{~d} v) \\
&\quad \leqslant \iint_{\left(0<|u|,|v|<u_{0}\right)}\left(\log_{2} \log_{2} \frac{1}{|u|}\right)\left(\log_{2} \log_{2} \frac{1}{|v|}\right) M_{0}(\mathrm{~d} u, \mathrm{~d} v) \\
&\quad<+\infty
\end{aligned}
\end{equation}

Therefore we have
\begin{equation}\label{eq:43}
\operatorname{Max}\left(\left|\mu\left(2^{-n} \leqslant|u|<2^{-2^{q}}\right)\right| ; 2^{q}<n \leqslant 2^{q+1}\right) \underset{q \rightarrow+\infty}{\longrightarrow} 0 \quad \text{a.s.,}
\end{equation}
and this completes the proof.
\end{proof}

\subsection{Remarks}

(a) In the strongly harmonizable case, the both conditions (i) and (ii) of the theorem in 3.2 can be replaced by the following single one: there exists a real number $u_{0}$ such that
\begin{equation}\label{eq:44}
\iint_{\left(0<|u|,|v|<u_{0}\right)}\left(\log \log \frac{1}{|u|}\right)\left(\log \log \frac{1}{|v|}\right)|M|(\mathrm{d} u, \mathrm{~d} v)<+\infty,
\end{equation}
where $|M|$ is the total variation measure of $M$. Moreover, if $X$ is continuous and stationary, it reduces exactly to the Gaposhkin's criterion (\ref{eq:13}).

In the general weakly harmonizable case, the condition (ii) can be replaced by more practical ones [3, Lemma 4.2.6 and Remark 4.2.7] which are also extensions of the corresponding result of Gaposhkin [4, Corollary 3].

(b) At last, once again as Gaposhkin [4] we have got some information about the rate of convergence of $\sigma_{X}(t)$ towards $\mu(0)$ :

\begin{theorem}\label{thm:4}
Let $X$ be a weakly harmonizable process. Suppose that there exists a non-decreasing function $g: \mathbb{R}^{+} \rightarrow \mathbb{R}^{+}$ such that

(i) there exist integers $p \geqslant 2, q_{0}$ and a real $A, 1<A<\sqrt{p}$ for which we have
\begin{equation}\label{eq:45}
g^{2}\left(p^{p^{q+1}}\right) \leqslant A g^{2}\left(p^{p^{q}}\right), \quad q \text{ integer}, q \geqslant q_{0}
\end{equation}

(ii) there exist a bounded non-negative measure $m$ on $(\mathbb{R}, \mathscr{B}_{\mathbb{R}})$ which dominates $M$ (in the sense of the domination lemma) and a real number $u_{0}$ such that
\begin{equation}\label{eq:46}
\int_{\left(0<|u|<u_{0}\right)}\left(\left(\log \log \frac{1}{|u|}\right) g\left(\frac{1}{|u|}\right)\right)^{2} m(\mathrm{~d} u)<+\infty
\end{equation}

Then one has
\begin{equation}\label{eq:47}
\lim_{t \rightarrow+\infty} g(t) \cdot\left(\sigma_{X}(t)-\mu(0)\right)=0 \quad \text{a.s.}
\end{equation}
\end{theorem}

\section*{Acknowledgement}

I would like to thank the referees for their precious advice.

\begin{thebibliography}{99}

\bibitem{1} A. Arimoto, On the strong law of large numbers for harmonizable stochastic processes, Keio Engineering Reports 25 (1972) 101-111.

\bibitem{2} A. Blanc Lapierre, Problèmes liés à la détermination des spectres de puissance en théorie des fonctions aléatoires, 8th Prague Conference, Information Theory, (Prague, 1978) vol. A 11-25, Reidel, Dordrecht 1978.

\bibitem{3} D. Dehay, Quelques lois des grands nombres pour les processus harmonisables, thèse, U.E.R. Math. Univ. Sc. Tech. Lille, 1985.

\bibitem{4} V.F. Gaposhkin, Criteria for the strong law of large numbers of some classes of second order stationary processes and homogeneous random fields, Th. Probability Appl. 22 (1977) 286-310.

\bibitem{5} V.F. Gaposhkin, A theorem on the convergence almost everywhere of measurable functions, and its applications to sequence of stochastic integrals, Math USSR Sbornik 33 (1977) 1-17.

\bibitem{6} E. Gladyshev, Periodically and almost-periodically correlated random processes with continuous time parameter, Th. Probability Appl. 8 (1963) 173-177.

\bibitem{7} A.G. Miamee and H. Salehi, Harmonizability, V-boundness and stationarity dilation of stochastic processes, Indiana Univ. Math. J. 27 (1978) 37-50.

\bibitem{8} R. Moché, Introduction aux processus harmonisables, U.E.R. Math. Univ. Sc. Tech. Lille, 1985.

\bibitem{9} M.M. Rao, Harmonizable processes: Structure theory, Enseign. Math. 28 (1982) 295-351.

\bibitem{10} J. Rousseau-Egelé, La loi forte des grands nombres pour les processus harmonisables, Ann. Inst. H. Poincaré B 15, (1979) 175-185.

\bibitem{11} Yu.A. Rozanov, Spectral analysis of abstract function, Th. Probability. Appl. 4 (1959) 271-287.

\end{thebibliography}
