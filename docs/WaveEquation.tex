
\documentclass{article}
\usepackage{amsmath}

\begin{document}

\section{Solutions of the equation of wave motions which involve Bessel functions.}

We shall now obtain a class of particular solutions of the equation of wave motions, useful for the solution of certain special problems.

In physical investigations, it is desirable to have the time occurring by means of a factor $\sin ckt$ or $\cos ckt$, where $k$ is constant. This suggests that we should consider solutions of the type

\[V = \int_{-\pi}^{\pi} \int_0^{\pi} e^{ik(x\sin u \cos v + y \sin u \sin v + z \cos u + ct)} f(u, v) dudv.\]

Physically this means that we consider motions in which all the elementary waves have the same period.

Now let the polar coordinates of $(x, y, z)$ be $(r, \theta, \phi)$ and let $(\omega, \psi)$ be the polar coordinates of the direction $(u, v)$ referred to new axes such that the polar axis is the direction $(\theta, \phi)$, and the plane $\psi = 0$ passes through $OZ$; so that

\[\cos \omega = \cos \theta \cos u + \sin \theta \sin u \cos(\phi - v),\]
\[\sin u \sin(\phi - v) = \sin \omega \sin \psi.\]

Also, take the arbitrary function $f(u, v)$ to be $S_n(u, v)\sin u$, where $S_n$ denotes a surface harmonic in $u$, $v$ of degree $n$; so that we may write

\[S_n(u, v) = \bar{S}_n(\theta, \phi; \omega, \psi),\]

where ($\S$ 18·31) $\bar{S}_n$ is a surface harmonic in $\omega$, $\psi$ of degree $n$.

We thus get
\[V = e^{ikct} \int_{-\pi}^{\pi} \int_0^{\pi} e^{ikr \cos \omega} \bar{S}_n (\theta, \phi; \omega, \psi) \sin \omega d\omega d\psi.\]

Now we may write ($\S$ 18·31)
\[\bar{S}_n (\theta, \phi; \omega, \psi) = A_n (\theta, \phi) \cdot P_n (\cos \omega)
+ \sum_{m=1}^n \{A_n^{(m)}(\theta, \phi)\cos m\psi + B_n^{(m)}(\theta, \phi)\sin m\psi\} P_n^m (\cos \omega),\]

where $A_n (\theta, \phi), A_n^{(m)} (\theta, \phi)$ and $B_n^{(m)} (\theta, \phi)$ are independent of $\psi$ and $\omega$.

Performing the integration with respect to $\psi$, we get
\[V = 2\pi e^{ikct} A_n (\theta, \phi) \int_0^{\pi} e^{ikr \cos \omega} P_n (\cos \omega) \sin \omega d\omega\]
\[= 2\pi e^{ikct} A_n (\theta, \phi) \int_{-1}^1 e^{ikr\mu} P_n (\mu) d\mu\]
\[= 2\pi e^{ikct} A_n (\theta, \phi) \int_{-1}^1 e^{ikr\mu} \frac{1}{2^n \cdot n!} \frac{d^n}{d\mu^n} (\mu^2 - 1)^n d\mu,\]

by Rodrigues' formula ($\S$ 15·11); on integrating by parts $n$ times and using Hankel's integral ($\S$ 17·3 corollary), we obtain the equation

\[V = \frac{2\pi}{2^n \cdot n!} e^{ikct} A_n (\theta, \phi) (ikr)^n \int_{-1}^1 e^{ikr\mu} (1 - \mu^2)^n d\mu\]
\[= (2\pi)^{\frac{1}{2}} i^n e^{ikct} (kr)^{-\frac{1}{2}} J_{n+\frac{1}{2}} (kr) A_n (\theta, \phi),\]

and so $V$ is a constant multiple of $e^{ikct} r^{-\frac{1}{2}} J_{n+\frac{1}{2}} (kr) A_n (\theta, \phi)$.

Now the equation of wave motions is unaffected if we multiply $x, y, z$ and $t$ by the same constant factor, i.e. if we multiply $r$ and $t$ by the same constant factor leaving $\theta$ and $\phi$ unaltered; so that $A_n(\theta, \phi)$ may be taken to be independent of the arbitrary constant $k$ which multiplies $r$ and $t$.

Hence $\lim_{k \to 0} e^{ikct} r^{-\frac{1}{2}} k^{-n-\frac{1}{2}} J_{n+\frac{1}{2}} (kr) A_n (\theta, \phi)$ is a solution of the equation of wave motions; and therefore $r^n A_n (\theta, \phi)$ is a solution (independent of $t$) of the equation of wave motions, and is consequently a solution of Laplace's equation; it is, accordingly, permissible to take $A_n (\theta, \phi)$ to be any surface harmonic of degree $n$; and so we obtain the result that

\[r^{-\frac{1}{2}} J_{n+\frac{1}{2}} (kr) P_n^m (\cos \theta) \begin{array}{c} \cos \\ \sin \end{array} m\phi \begin{array}{c} \cos \\ \sin \end{array} ckt\]

is a particular solution of the equation of wave motions.

\end{document}
