\documentclass{article}
\usepackage[english]{babel}
\usepackage{geometry,amsmath,amssymb,xcolor,latexsym}
\geometry{letterpaper}

%%%%%%%%%% Start TeXmacs macros
\newcommand{\tmtextbf}[1]{\text{{\bfseries{#1}}}}
\newenvironment{proof}{\noindent\textbf{Proof\ }}{\hspace*{\fill}$\Box$\medskip}
\newtheorem{definition}{Definition}
\newtheorem{theorem}{Theorem}
%%%%%%%%%% End TeXmacs macros

\begin{document}

{\color[HTML]{000000}\begin{center}
  {\huge \tmtextbf{{\color[HTML]{FFD700}Measure-Theoretic Derivation of
  Spectral Inversion Formulas}}}\\
  {\large Complete mathematical exposition of orthogonal random measures and
  spectral inversion} 
\end{center}{\vspace{2em}}

\section{1. Orthogonal Random Measures}

\begin{definition}
  Let $(\Omega, \mathcal{F}, P)$ be a complete probability space and
  $(\mathbb{R}, \mathcal{B}(\mathbb{R}))$ be the real line with its Borel
  $\sigma$-algebra. A {\color[HTML]{FFD700}\tmtextbf{complex orthogonal random
  measure}} $\Phi$ is a mapping $\Phi : \mathcal{B} (\mathbb{R}) \to L^2
  (\Omega, \mathcal{F}, P)$ satisfying:
  \begin{enumerate}
    \item \tmtextbf{Additivity:} For disjoint sets $A_1, A_2, \ldots \in
    \mathcal{B} (\mathbb{R})$:
    \begin{equation}
      \Phi \left( \bigcup_{n = 1}^{\infty} A_n \right) = \sum_{n = 1}^{\infty}
      \Phi (A_n)
    \end{equation}
    where convergence is in $L^2$ norm
    
    \item \tmtextbf{Orthogonality:} For any $A, B \in \mathcal{B}
    (\mathbb{R})$:
    \begin{equation}
      \mathbb{E} [\Phi (A) \overline{\Phi (B)}] = \mu (A \cap B)
    \end{equation}
    for a unique positive measure $\mu$ called the
    {\color[HTML]{FFD700}\tmtextbf{structural measure}}
  \end{enumerate}
\end{definition}

\subsection{Integration with Respect to Orthogonal Random Measures}

For $f \in L^2 (\mu)$, the stochastic integral $\int f (\lambda) 
\hspace{0.17em} d \Phi (\lambda)$ is defined as the $L^2$-limit of simple
function approximations.
\begin{itemize}
  \item \tmtextbf{Isometry:} $\mathbb{E} \left[ \left| \int f (\lambda)
  \hspace{0.17em} d \Phi (\lambda) \right|^2 \right] = \int |f (\lambda) |^2 
  \hspace{0.17em} d \mu (\lambda)$
  
  \item \tmtextbf{Linearity:} $\int (af + bg)  \hspace{0.17em} d \Phi = a \int
  f \hspace{0.17em} d \Phi + b \int g \hspace{0.17em} d \Phi$
  
  \item \tmtextbf{Continuity:} The mapping $f \mapsto \int f \hspace{0.17em} d
  \Phi$ is an isometry from $L^2 (\mu)$ to $L^2 (\Omega, \mathcal{F}, P)$
\end{itemize}

\section{2. Spectral Representation Theorem}

\begin{theorem}
  [Spectral Representation] Let $\{\xi (t) : t \in \mathbb{R}\}$ be a
  mean-square continuous second-order stationary process with covariance
  function $r (t)$. Then there exists a unique complex orthogonal random
  measure $\Phi$ such that:
  \begin{equation}
    \xi (t) = \int_{\mathbb{R}} e^{it \lambda}  \hspace{0.17em} d \Phi
    (\lambda)
  \end{equation}
  The covariance function has the representation:
  \begin{equation}
    r (t) = \int_{\mathbb{R}} e^{it \lambda}  \hspace{0.17em} d \mu (\lambda)
  \end{equation}
  where $\mu$ is the structural measure of $\Phi$ with $\mu (\mathbb{R}) = r
  (0)$.
\end{theorem}

\begin{proof}
  The proof relies on the fact that $r (t)$ is positive definite, so by
  Bochner's theorem, there exists a unique finite positive measure $\mu$ such
  that $r (t) = \int e^{it \lambda} d \mu (\lambda)$. The orthogonal random
  measure $\Phi$ is constructed via the isometric isomorphism between $L^2
  (\mu)$ and the Hilbert space $H (\xi) = \overline{\text{span}} \{\xi (t) : t
  \in \mathbb{R}\}$.
\end{proof}

\section{3. Cumulative Spectral Distribution and Measure}

\begin{definition}
  The {\color[HTML]{FFD700}\tmtextbf{cumulative spectral distribution
  function}} $F : \mathbb{R} \to \mathbb{R}$ is defined as:
  \begin{equation}
    F (\lambda) = \mu ((- \infty, \lambda]) =\mathbb{E} [| \Phi ((- \infty,
    \lambda]) |^2]
  \end{equation}
  The {\color[HTML]{FFD700}\tmtextbf{cumulative orthogonal random measure}}
  is:
  \begin{equation}
    \Phi (\lambda) = \Phi ((- \infty, \lambda])
  \end{equation}
\end{definition}

\subsection{Properties}

\begin{enumerate}
  \item $F$ is non-decreasing and right-continuous
  
  \item $\lim_{\lambda \to - \infty} F (\lambda) = 0$ and $\lim_{\lambda \to
  \infty} F (\lambda) = r (0)$
  
  \item For $\lambda_1 < \lambda_2$: $F (\lambda_2) - F (\lambda_1) = \mu
  ((\lambda_1, \lambda_2])$
  
  \item $\Phi (\lambda_2) - \Phi (\lambda_1) = \Phi ((\lambda_1, \lambda_2])$
\end{enumerate}

\section{4. Derivation of Spectral Inversion Formulas}

\subsection{The Fourier-Stieltjes Inversion Theorem}

Since $r (t) = \int_{\mathbb{R}} e^{it \lambda} dF (\lambda)$ and $F$ is of
bounded variation, we apply the Fourier-Stieltjes inversion:

For continuity points $a < b$ of $F$:
\begin{equation}
  F (b) - F (a) = \frac{1}{2 \pi} \lim_{T \to \infty}  \int_{- T}^T \frac{e^{-
  ita} - e^{- itb}}{it} r (t)  \hspace{0.17em} dt
\end{equation}

\subsection{Single Frequency Increment from $- \infty$}

To obtain $F (\lambda)$ from the base frequency $- \infty$, set $a = - \infty$
and $b = \lambda$:
\begin{equation}
  F (\lambda) - F (- \infty) = \frac{1}{2 \pi} \lim_{T \to \infty}  \int_{-
  T}^T \frac{e^{- it (- \infty)} - e^{- it \lambda}}{it} r (t) 
  \hspace{0.17em} dt
\end{equation}
Since $F (- \infty) = 0$ and the boundary term $e^{it \infty}$ contributes 1
(representing the starting point of the cumulative measure):
\begin{equation}
  F (\lambda) = \frac{1}{2 \pi} \lim_{T \to \infty}  \int_{- T}^T \frac{1 -
  e^{- it \lambda}}{it} r (t)  \hspace{0.17em} dt
\end{equation}
For the orthogonal random measure:
\begin{equation}
  \Phi (\lambda) = \lim_{T \to \infty}  \frac{1}{2 \pi}  \int_{- T}^T \frac{1
  - e^{- it \lambda}}{it} \xi (t)  \hspace{0.17em} dt
\end{equation}

\subsection{Derivation of Real Trigonometric Form}

Substitute Euler's formula $e^{- it \lambda} = \cos (t \lambda) - i \sin (t
\lambda)$:
\begin{equation}
  \frac{1 - e^{- it \lambda}}{it} = \frac{1 - \cos (t \lambda) + i \sin (t
  \lambda)}{it} = \frac{1 - \cos (t \lambda)}{it} + \frac{\sin (t \lambda)}{t}
\end{equation}}

\tmtextbf{{\color[HTML]{000000}Symmetry Analysis for $\int_{- T}^T$:}}

{\color[HTML]{000000}\begin{enumerate}
  \item \tmtextbf{First term}: $\frac{1 - \cos (t \lambda)}{t} r (t)$
  \begin{itemize}
    \item $1 - \cos (t \lambda)$ is even
    
    \item $t$ is odd
    
    \item $r (t)$ is even (covariance function)
    
    \item Product is {\color[HTML]{FFD700}\tmtextbf{odd}} {\rightarrow}
    integral over $[- T, T]$ equals {\color[HTML]{FFD700}\tmtextbf{0}}
  \end{itemize}
  \item \tmtextbf{Second term}: $\frac{\sin (t \lambda)}{t} r (t)$
  \begin{itemize}
    \item $\sin (t \lambda)$ is odd
    
    \item $t$ is odd
    
    \item $r (t)$ is even
    
    \item Product is {\color[HTML]{FFD700}\tmtextbf{even}} {\rightarrow}
    $\int_{- T}^T = 2 \int_0^T$
  \end{itemize}
\end{enumerate}

\subsection{Final Real Form}

Combining these results:
\begin{equation}
  F (\lambda) = \frac{1}{\pi} \lim_{T \to \infty}  \int_0^T \frac{\sin (t
  \lambda)}{t} r (t)  \hspace{0.17em} dt
\end{equation}
\begin{equation}
  \Phi (\lambda) = \lim_{T \to \infty}  \frac{1}{\pi}  \int_0^T \frac{\sin (t
  \lambda)}{t} \xi (t)  \hspace{0.17em} dt
\end{equation}

\section{5. Alternative Representations}

\subsection{Dirichlet Kernel Form}

Using the identity $\frac{\sin (t \lambda)}{t} = \int_0^{\lambda} \cos (t
\omega)  \hspace{0.17em} d \omega$:
\begin{equation}
  F (\lambda) = \frac{2}{\pi}  \int_0^{\lambda} \left( \int_0^{\infty} \cos (t
  \omega) r (t) \hspace{0.17em} dt \right) d \omega
\end{equation}

\subsection{Spectral Density Representation}

When $F$ is absolutely continuous with spectral density $f$:
\begin{equation}
  F (\lambda) = \int_{- \infty}^{\lambda} f (\omega)  \hspace{0.17em} d \omega
\end{equation}
\begin{equation}
  f (\lambda) = \frac{1}{2 \pi}  \int_{- \infty}^{\infty} e^{- it \lambda} r
  (t)  \hspace{0.17em} dt
\end{equation}

\section{6. Mathematical Interpretation}

\begin{itemize}
  \item \tmtextbf{$F (\lambda)$}: Cumulative spectral distribution - total
  spectral power up to frequency $\lambda$
  
  \item \tmtextbf{$\Phi (\lambda)$}: Cumulative orthogonal random measure -
  random spectral content up to $\lambda$
  
  \item Both are cumulative measures starting from 0 at $- \infty$
  
  \item The measure-theoretic framework ensures rigorous treatment of
  infinite-dimensional stochastic integrals
\end{itemize}}

\end{document}
