\documentclass{article}
\usepackage{amsmath, amssymb, amsthm}
\usepackage{mathrsfs}

\title{The Caputo Fractional Derivative of the Sine Function}
\author{}
\date{}

\newtheorem{theorem}{Theorem}
\newtheorem{definition}[theorem]{Definition}
\newtheorem{lemma}[theorem]{Lemma}

\begin{document}

\maketitle

\begin{definition}[Caputo Fractional Derivative]
For $\alpha > 0$, the Caputo fractional derivative of order $\alpha$ for a function $f(t)$ is defined as:
\begin{equation}
_0^CD_t^\alpha f(t) = \frac{1}{\Gamma(\lceil \alpha \rceil - \alpha)} \int_0^t \frac{f^{(\lceil \alpha \rceil)}(\tau)}{(t-\tau)^{\alpha - \lceil \alpha \rceil + 1}} d\tau
\end{equation}
where $\lceil \alpha \rceil$ is the ceiling function, denoting the smallest integer greater than or equal to $\alpha$.
\end{definition}

\begin{definition}[Mittag-Leffler Function]
The two-parameter Mittag-Leffler function is defined for $\alpha, \beta > 0$ as:
\begin{equation}
E_{\alpha,\beta}(z) = \sum_{k=0}^{\infty} \frac{z^k}{\Gamma(\alpha k + \beta)}
\end{equation}
\end{definition}

\begin{lemma}
For non-negative integers $m$, the $m$-th derivative of $\sin(t)$ is given by:
\begin{equation}
\sin^{(m)}(t) = \sin\left(t + \frac{m\pi}{2}\right)
\end{equation}
\end{lemma}

\begin{lemma}
For $a > 0$ and $b > 0$, the Beta function satisfies:
\begin{equation}
\int_0^1 u^{a-1}(1-u)^{b-1} du = B(a,b) = \frac{\Gamma(a)\Gamma(b)}{\Gamma(a+b)}
\end{equation}
\end{lemma}

\begin{theorem}[Caputo Fractional Derivative of Sine]
For any $\alpha > 0$, the Caputo fractional derivative of $\sin(t)$ is given by:
\begin{equation}
_0^CD_t^\alpha \sin(t) = t^{\lceil \alpha \rceil - \alpha} E_{2,\lceil \alpha \rceil - \alpha + 1}(-t^2)
\end{equation}
where $E_{2,\lceil \alpha \rceil - \alpha + 1}(-t^2)$ is the Mittag-Leffler function.
\end{theorem}

\begin{proof}
We begin with the Caputo fractional derivative definition for $\alpha > 0$:
\begin{equation}
_0^CD_t^\alpha \sin(t) = \frac{1}{\Gamma(\lceil \alpha \rceil - \alpha)} \int_0^t \frac{\sin^{(\lceil \alpha \rceil)}(\tau)}{(t-\tau)^{\alpha - \lceil \alpha \rceil + 1}} d\tau
\end{equation}

From the lemma, we know that:
\begin{equation}
\sin^{(\lceil \alpha \rceil)}(\tau) = \sin\left(\tau + \frac{\lceil \alpha \rceil\pi}{2}\right)
\end{equation}

The $\lceil \alpha \rceil$-th derivative of sine can be expanded in Taylor series:
\begin{equation}
\sin\left(\tau + \frac{\lceil \alpha \rceil\pi}{2}\right) = \sum_{k=0}^{\infty} \frac{(-1)^k \tau^{2k+p}}{(2k+p)!}
\end{equation}
where $p = 0$ if $\lceil \alpha \rceil$ is even, and $p = 1$ if $\lceil \alpha \rceil$ is odd.

Substituting into the Caputo definition:
\begin{align}
_0^CD_t^\alpha \sin(t) &= \frac{1}{\Gamma(\lceil \alpha \rceil - \alpha)} \int_0^t \frac{1}{(t-\tau)^{\alpha - \lceil \alpha \rceil + 1}} \sum_{k=0}^{\infty} \frac{(-1)^k \tau^{2k+p}}{(2k+p)!} d\tau\\
&= \frac{1}{\Gamma(\lceil \alpha \rceil - \alpha)} \sum_{k=0}^{\infty} \frac{(-1)^k}{(2k+p)!} \int_0^t \frac{\tau^{2k+p}}{(t-\tau)^{\alpha - \lceil \alpha \rceil + 1}} d\tau
\end{align}

Making the substitution $\tau = tu$, we get:
\begin{align}
\int_0^t \frac{\tau^{2k+p}}{(t-\tau)^{\alpha - \lceil \alpha \rceil + 1}} d\tau &= \int_0^1 \frac{(tu)^{2k+p}}{(t-tu)^{\alpha - \lceil \alpha \rceil + 1}} \cdot t \, du\\
&= t^{2k+p+\lceil \alpha \rceil - \alpha} \int_0^1 \frac{u^{2k+p}}{(1-u)^{\alpha - \lceil \alpha \rceil + 1}} du
\end{align}

The integral can be expressed in terms of the Beta function:
\begin{align}
\int_0^1 \frac{u^{2k+p}}{(1-u)^{\alpha - \lceil \alpha \rceil + 1}} du &= \int_0^1 u^{2k+p} (1-u)^{(\lceil \alpha \rceil - \alpha) - 1} du\\
&= B(2k+p+1, \lceil \alpha \rceil - \alpha)\\
&= \frac{\Gamma(2k+p+1)\Gamma(\lceil \alpha \rceil - \alpha)}{\Gamma(2k+p+1+\lceil \alpha \rceil - \alpha)}
\end{align}

Substituting back:
\begin{align}
_0^CD_t^\alpha \sin(t) &= \frac{1}{\Gamma(\lceil \alpha \rceil - \alpha)} \sum_{k=0}^{\infty} \frac{(-1)^k}{(2k+p)!} t^{2k+p+\lceil \alpha \rceil - \alpha} \frac{\Gamma(2k+p+1)\Gamma(\lceil \alpha \rceil - \alpha)}{\Gamma(2k+p+1+\lceil \alpha \rceil - \alpha)}\\
&= \sum_{k=0}^{\infty} \frac{(-1)^k \Gamma(2k+p+1)}{(2k+p)! \Gamma(2k+p+1+\lceil \alpha \rceil - \alpha)} t^{2k+p+\lceil \alpha \rceil - \alpha}
\end{align}

Since $(2k+p)! = \Gamma(2k+p+1)$, we simplify to:
\begin{align}
_0^CD_t^\alpha \sin(t) &= \sum_{k=0}^{\infty} \frac{(-1)^k}{\Gamma(2k+p+1+\lceil \alpha \rceil - \alpha)} t^{2k+p+\lceil \alpha \rceil - \alpha}
\end{align}

Now we analyze the cases based on parity of $\lceil \alpha \rceil$:

\textbf{Case 1:} When $\lceil \alpha \rceil$ is even, $p = 0$:
\begin{align}
_0^CD_t^\alpha \sin(t) &= \sum_{k=0}^{\infty} \frac{(-1)^k}{\Gamma(2k+1+\lceil \alpha \rceil - \alpha)} t^{2k+\lceil \alpha \rceil - \alpha}\\
&= t^{\lceil \alpha \rceil - \alpha} \sum_{k=0}^{\infty} \frac{(-1)^k t^{2k}}{\Gamma(2k+1+\lceil \alpha \rceil - \alpha)}\\
&= t^{\lceil \alpha \rceil - \alpha} \sum_{k=0}^{\infty} \frac{(-t^2)^k}{\Gamma(2k+1+\lceil \alpha \rceil - \alpha)}
\end{align}

This matches the form of the Mittag-Leffler function with $\alpha=2$ and $\beta=1+\lceil \alpha \rceil - \alpha$:
\begin{align}
_0^CD_t^\alpha \sin(t) = t^{\lceil \alpha \rceil - \alpha} E_{2,1+\lceil \alpha \rceil - \alpha}(-t^2)
\end{align}

\textbf{Case 2:} When $\lceil \alpha \rceil$ is odd, $p = 1$:
\begin{align}
_0^CD_t^\alpha \sin(t) &= \sum_{k=0}^{\infty} \frac{(-1)^k}{\Gamma(2k+2+\lceil \alpha \rceil - \alpha)} t^{2k+1+\lceil \alpha \rceil - \alpha}\\
&= t^{1+\lceil \alpha \rceil - \alpha} \sum_{k=0}^{\infty} \frac{(-1)^k t^{2k}}{\Gamma(2k+2+\lceil \alpha \rceil - \alpha)}\\
&= t^{1+\lceil \alpha \rceil - \alpha} \sum_{k=0}^{\infty} \frac{(-t^2)^k}{\Gamma(2k+2+\lceil \alpha \rceil - \alpha)}\\
&= t^{1+\lceil \alpha \rceil - \alpha} E_{2,2+\lceil \alpha \rceil - \alpha}(-t^2)
\end{align}

But notice for odd $\lceil \alpha \rceil$, we can write $1+\lceil \alpha \rceil - \alpha = \lceil \alpha \rceil - \alpha + 1$, and $2+\lceil \alpha \rceil - \alpha = \lceil \alpha \rceil - \alpha + 2$. 

For even $\lceil \alpha \rceil$, we have $\lceil \alpha \rceil - \alpha$ and $1+\lceil \alpha \rceil - \alpha = \lceil \alpha \rceil - \alpha + 1$.

Therefore, we can unify both cases as:
\begin{align}
_0^CD_t^\alpha \sin(t) = t^{\lceil \alpha \rceil - \alpha} E_{2,\lceil \alpha \rceil - \alpha + 1}(-t^2)
\end{align}
This completes the proof.
\end{proof}

\end{document}
