\documentclass{article}
\usepackage[english]{babel}
\usepackage{amsmath,amssymb,latexsym}

%%%%%%%%%% Start TeXmacs macros
\newcommand{\tmaffiliation}[1]{\\ #1}
\newenvironment{proof}{\noindent\textbf{Proof\ }}{\hspace*{\fill}$\Box$\medskip}
\newtheorem{definition}{Definition}
\newtheorem{theorem}{Theorem}
%%%%%%%%%% End TeXmacs macros

\begin{document}

\title{The Spectrum of the Zero Crossing Number Operator}

\author{
  Stephen Crowley
  \tmaffiliation{August 11, 2025}
}

\maketitle

\begin{definition}
  [Zero-Crossing Measure] Let $X (t, \omega)$ be a sample path of a Gaussian
  process with derivative $X' (t, \omega)$ existing almost surely where
  $\omega \in \Omega$ represents a specific sample path element from the
  ensemble of possible sample paths denoted by $\Omega$. Define the
  zero-crossing measure $\mu_{\omega}$ on $\mathbb{R}$ by:
  \begin{equation}
    d \mu_{\omega} (s) = \delta (X (s, \omega)) |X' (s, \omega) | ds
  \end{equation}
  where $\delta$ is the Dirac delta function.
\end{definition}

\begin{definition}
  [Zero-Crossing Spectral Operator] Define the zero-crossing spectral operator
  $T : L^2 (\mathbb{R}, \mu_{\omega}) \to L^2 (\mathbb{R}, \mu_{\omega})$ by:
  \begin{equation}
    (Tf) (s) = s \cdot f (s)
  \end{equation}
\end{definition}

\begin{theorem}
  [Zero-Crossing Spectrum] The operator $T$ has spectrum given by:
  \begin{equation}
    \sigma (T) = \overline{\{t \in \mathbb{R}: X (t, \omega) = 0\}}
  \end{equation}
  where the overline denotes topological closure.
\end{theorem}

\begin{proof}
  The operator $T$ is multiplication by the function $m (s) = s$ on the
  measure space $(\mathbb{R}, \mu_{\omega})$.
  
  For multiplication operators on measure spaces, the spectrum is given by:
  \begin{equation}
    \sigma (T) = \text{essential range of } m \text{with respect to }
    \mu_{\omega}
  \end{equation}
  The essential range of $m (s) = s$ with respect to measure $\mu_{\omega}$
  is:
  \begin{equation}
    \text{ess ran}_{\mu_{\omega}} (s) = \{\lambda \in \mathbb{R}: \mu_{\omega}
    (\{s : |s - \lambda | < \epsilon\}) > 0 \text{for all } \epsilon > 0\}
  \end{equation}
  Since $\mu_{\omega}$ is supported precisely on $Z_{\omega} = \{t \in
  \mathbb{R}: X (t, \omega) = 0\}$, we have:
  \begin{equation}
    \mu_{\omega} (\{s : |s - \lambda | < \epsilon\}) > 0 \text{if and only if
    } (\lambda - \epsilon, \lambda + \epsilon) \cap Z_{\omega} \neq \emptyset
  \end{equation}
  This occurs if and only if $\lambda$ is in the closure of $Z_{\omega}$.
  
  Therefore:
  \begin{equation}
    \sigma (T) = \overline{Z_{\omega}} = \overline{\{t \in \mathbb{R}: X (t,
    \omega) = 0\}}
  \end{equation}
  Equivalently, $\lambda \in \sigma (T)$ if and only if $(T - \lambda I)$ is
  not invertible, which occurs precisely when the multiplier $(s - \lambda)$
  is not bounded away from zero on the support of $\mu_{\omega}$.
\end{proof}

\end{document}
