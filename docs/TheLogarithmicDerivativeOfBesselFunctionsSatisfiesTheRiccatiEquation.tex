\documentclass{article}
\usepackage{amsmath}

\title{The Logarithmic Derivative of Bessel Functions}
\author{}
\date{}

\begin{document}
\maketitle

The Bessel function J n ( t ) satisfies:
[
t^2J_n''(t) + tJ_n'(t) + (t^2 - n^2)J_n(t) = 0
]

Its logarithmic derivative is:
[
w(t) = \frac{J_n'(t)}{J_n(t)}
]

The logarithmic derivative satisfies a Riccati equation, as follows:

\begin{enumerate}
\item The definition directly implies:
[J_n'(t) = w(t)J_n(t)]

\item The derivative of this equation is:
[J_n''(t) = w'(t)J_n(t) + w(t)^2J_n(t)]

\item These expressions in the Bessel equation yield:
[t^2(w'(t)J_n(t) + w(t)^2J_n(t)) + t(w(t)J_n(t)) + (t^2 - n^2)J_n(t) = 0]

\item Division by J n ( t ) produces:
[t^2w'(t) + t^2w(t)^2 + tw(t) + (t^2 - n^2) = 0]

\item The equation rearranges to:
[w'(t) = -w(t)^2 - \frac{w(t)}{t} - (1 - \frac{n^2}{t^2})]
\end{enumerate}

This Riccati equation has the standard form:
[w'(t) = p(t) + q(t)w(t) + r(t)w(t)^2]

where:
[p(t) = -(1 - \frac{n^2}{t^2})]
[q(t) = -\frac{1}{t}]
[r(t) = -1]

\end{document}
