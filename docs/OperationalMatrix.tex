\documentclass[12pt]{article}
\usepackage{amsmath, amssymb, amsthm, geometry}
\geometry{margin=1in}

\newtheorem{theorem}{Theorem}
\newtheorem{lemma}[theorem]{Lemma}
\newtheorem{corollary}[theorem]{Corollary}

\begin{document}

\title{The Operational Matrix of The Random Wave Process}
\author{}
\date{}
\maketitle

\section*{Introduction}

Let $I_{m,n}(y)$ be defined as the integral
\[
I_{m,n}(y) = \int_{-1}^1 {}_2F_1\left(-m, m+1; 1; \frac{1}{2} - \frac{x}{2}\right) 
{}_2F_1\left(-n, n+1; 1; \frac{1}{2} - \frac{x}{2}\right) e^{i x y} \, dx,
\]
where ${}_2F_1(a, b; c; z)$ is the Gauss hypergeometric function, and $m, n$ are non-negative integers.

\section*{Hypergeometric Series Expansion}

\begin{lemma}
For any non-negative integer $p$ and complex numbers $b,c$ with $c \notin \{0,-1,-2,\ldots\}$:
\[
{}_2F_1(-p, b; c; z) = \sum_{k=0}^p \frac{(-p)_k (b)_k}{(c)_k k!} z^k
\]
where $(a)_k = a(a+1)\cdots(a+k-1)$ is the Pochhammer symbol.
\end{lemma}

\begin{proof}
By definition, ${}_2F_1(a,b;c;z) = \sum_{k=0}^\infty \frac{(a)_k(b)_k}{(c)_k k!}z^k$. When $a=-p$ for non-negative integer $p$:
\[
(-p)_k = (-p)(-p+1)\cdots(-p+k-1) = (-1)^k p(p-1)\cdots(p-k+1)
\]
Therefore $(-p)_k = 0$ for all $k > p$ since one of the factors becomes zero. Thus the infinite series terminates at $k=p$.
\end{proof}

\begin{lemma}
For the given integral, applying the series expansion:
\[
I_{m,n}(y) = \sum_{k=0}^m \sum_{l=0}^n \frac{(-m)_k (m+1)_k}{k!} \frac{(-n)_l (n+1)_l}{l!} 
\int_{-1}^1 \left(\frac{1}{2} - \frac{x}{2}\right)^{k+l} e^{i x y} \, dx
\]
\end{lemma}

\begin{proof}
Substituting the series expansions for both hypergeometric functions:
\begin{align*}
I_{m,n}(y) &= \int_{-1}^1 \left(\sum_{k=0}^m \frac{(-m)_k (m+1)_k}{k!} \left(\frac{1}{2} - \frac{x}{2}\right)^k\right) \\
&\quad \times \left(\sum_{l=0}^n \frac{(-n)_l (n+1)_l}{l!} \left(\frac{1}{2} - \frac{x}{2}\right)^l\right) e^{ixy} \, dx
\end{align*}
The series are finite, so we can interchange summation and integration by Fubini's theorem.
\end{proof}

\section*{Integral Evaluation}

\begin{lemma}
For non-negative integer $j$:
\[
\int_{-1}^1 \left(\frac{1}{2} - \frac{x}{2}\right)^j e^{ixy} \, dx = 
\frac{e^{iy}}{2^j} \left[\frac{\Gamma(j+1,-2iy)}{(-iy)^{j+1}} - \frac{\Gamma(j+1)}{(-iy)^{j+1}}\right]
\]
where $\Gamma(a,z)$ is the upper incomplete gamma function and $\Gamma(a)$ is the complete gamma function.
\end{lemma}

\begin{proof}
Make the substitution $u = 1-x$. Then $dx = -du$ and when $x = -1$, $u = 2$; when $x = 1$, $u = 0$. Thus:
\begin{align*}
\int_{-1}^1 \left(\frac{1}{2} - \frac{x}{2}\right)^j e^{ixy} \, dx 
&= \frac{1}{2^j} \int_{-1}^1 (1-x)^j e^{ixy} \, dx \\
&= \frac{1}{2^j} \int_0^2 u^j e^{iy(1-u)} \, du \\
&= \frac{e^{iy}}{2^j} \int_0^2 u^j e^{-iuy} \, du
\end{align*}

Using integration by parts and the relationship between incomplete gamma functions:
\[
\Gamma(a,z) = \Gamma(a) - \gamma(a,z)
\]
where $\gamma(a,z)$ is the lower incomplete gamma function, we obtain:
\begin{align*}
\int_0^2 u^j e^{-iuy} \, du &= \frac{\Gamma(j+1)}{(-iy)^{j+1}} - \frac{\Gamma(j+1,2iy)}{(-iy)^{j+1}}
\end{align*}

Therefore:
\[
\frac{e^{iy}}{2^j} \int_0^2 u^j e^{-iuy} \, du = 
\frac{e^{iy}}{2^j} \left[\frac{\Gamma(j+1,-2iy)}{(-iy)^{j+1}} - \frac{\Gamma(j+1)}{(-iy)^{j+1}}\right]
\]
\end{proof}

\section*{Double Sum Transformation and Hypergeometric Representation}

\begin{theorem}[Double Sum Transformation]
The double sum can be rewritten as:
\[
\sum_{k=0}^m \sum_{l=0}^n \frac{(-m)_k (m+1)_k}{k!} \frac{(-n)_l (n+1)_l}{l!} z^{k+l}
= \sum_{j=0}^{m+n} \Psi_j(m,n) z^j
\]
where
\[
\Psi_j(m,n) = \sum_{k=\max(0,j-n)}^{\min(j,m)} 
\frac{(-m)_k (m+1)_k}{k!} 
\frac{(-n)_{j-k} (n+1)_{j-k}}{(j-k)!}
\]
\end{theorem}

\begin{proof}
Let $j = k + l$. For each power $z^j$, we sum over all valid combinations of $k$ and $l$ that sum to $j$. The limits on $k$ are determined by:
\begin{itemize}
\item $k \geq 0$ and $k \leq m$ from the first sum
\item $j-k \geq 0$ and $j-k \leq n$ from the second sum
\end{itemize}
These constraints give us $k \geq \max(0,j-n)$ and $k \leq \min(j,m)$. The result follows by collecting terms with the same power of $z$.
\end{proof}

\begin{theorem}[Hypergeometric Formulation of $\Psi_j(m,n)$]
\[
\Psi_j(m,n) = \frac{(-1)^j}{j!} \cdot {}_4F_3\left(
\begin{matrix}
- m, \, m+1, \, -n, \, n+1 \\
1, \, 1, \, j+1
\end{matrix}; 1 \right)
\]
\end{theorem}

\begin{proof}
Starting with the sum representation of $\Psi_j(m,n)$:
\[
\Psi_j(m,n) = \sum_{k=0}^j \frac{(-m)_k (m+1)_k}{k!} \frac{(-n)_{j-k} (n+1)_{j-k}}{(j-k)!}
\]

The ${}_4F_3$ hypergeometric function has the series representation:
\[
{}_4F_3\left(\begin{matrix}
a_1, a_2, a_3, a_4 \\
b_1, b_2, b_3
\end{matrix}; z\right) = \sum_{k=0}^\infty 
\frac{(a_1)_k(a_2)_k(a_3)_k(a_4)_k}{(b_1)_k(b_2)_k(b_3)_k} \frac{z^k}{k!}
\]

Substituting the parameters and comparing coefficients:
\begin{align*}
\frac{(-m)_k(m+1)_k(-n)_{j-k}(n+1)_{j-k}}{k!(j-k)!} &= 
\frac{(-1)^j}{j!} \cdot \frac{(-m)_k(m+1)_k(-n)_k(n+1)_k}{(1)_k(1)_k(j+1)_k}
\end{align*}

The equality holds term by term, verifying the hypergeometric representation.
\end{proof}

\section*{Final Result}

\begin{theorem}
The integral evaluates to:
\[
I_{m,n}(y) = e^{iy} \sum_{j=0}^{m+n} \frac{\Psi_j(m,n)}{2^j} 
\left[\frac{\Gamma(j+1,-2iy)-\Gamma(j+1)}{(-iy)^{j+1}}\right]
\]
where $\Psi_j(m,n)$ is given equivalently by either sum or hypergeometric representations as shown above.
\end{theorem}

\begin{proof}
The result follows by combining the series expansion from Lemma 2, the integral evaluation from Lemma 3, and both sum transformations from Theorems 1 and 2. The convergence of all sums is guaranteed as they are finite sums over non-negative integers $m$ and $n$.
\end{proof}

\end{document}
