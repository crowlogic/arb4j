\documentclass{article}
\usepackage{amsmath}
\usepackage{amssymb}
\usepackage{amsthm}
\usepackage{hyperref}

\newcommand{\tmaffiliation}[1]{#1}
\newcommand{\tmtextbf}[1]{\textbf{#1}}

\title{The Operational Matrix of the Random Wave Process}

\author{
  Stephen Crowley \\
  \tmaffiliation{Department of Intuition \\
  The Royal Citadel of Pure Being\href{mailto:your.email@example.com}{}}
}

\date{February 11, 2025}

\begin{document}

\maketitle

\begin{abstract}
  An expression for the Fourier transform of Legendre polynomial products is derived.
\end{abstract}

\section{Preliminaries}

\subsection{Incomplete Gamma Function}
\begin{definition}[Incomplete Gamma Function]
The lower incomplete gamma function is defined as
\[
\gamma(s,x) = \int_0^x t^{s-1} e^{-t} \, dt,
\]
where $s$ is a complex number with $\Re(s) > 0$, and $x$ is a non-negative real number.
\end{definition}

\section{Lemmas}

\subsection{Terminating Hypergeometric Series}
\begin{lemma}[Terminating Hypergeometric Series]
  \label{lem:HyperExpansions}
  For any $p \in \mathbb{Z}_{\geq 0}$, the Gauss hypergeometric function terminates:
  \begin{equation*}
    {}_2 F_1 (- p, b ; c ; z) = \sum_{k = 0}^p \frac{(- p)_k (b)_k}{(c)_k k!}
    z^k
  \end{equation*}
  where $(a)_k = \prod_{i = 0}^{k - 1} (a + i)$.
\end{lemma}

\begin{proof}
  By definition, the Gauss hypergeometric series is:
  \begin{equation*}
    {}_2 F_1 (a, b ; c ; z) = \sum_{k = 0}^{\infty} \frac{(a)_k (b)_k}{(c)_k k!}
    z^k
  \end{equation*}
  Setting $a = - p$ with $p \in \mathbb{Z}_{\geq 0}$, the Pochhammer symbol
  $(- p)_k$ becomes zero for all $k > p$. Explicitly:
  \begin{equation*}
    (- p)_k = \prod_{i = 0}^{k - 1} (- p + i) = \begin{cases}
      (- p) (- p + 1) \cdots (- p + k - 1), & k \leq p \\
      0 & k > p
    \end{cases}
  \end{equation*}
  Thus, the series terminates at $k = p$, yielding:
  \begin{equation*}
    {}_2 F_1 (- p, b ; c ; z) = \sum_{k = 0}^p \frac{(- p)_k (b)_k}{(c)_k k!}
    z^k
  \end{equation*}
\end{proof}

\subsection{Integral with Incomplete Gamma Function}
\begin{lemma}[Integral with Incomplete Gamma Function]
  \label{lem:IntegralGamma}
  For $j \geq 0$,
  \begin{equation*}
    \int_{- 1}^1 \left( \frac{1 - x}{2} \right)^j e^{ixy} dx =
    \frac{e^{iy}}{2^j}  \frac{\gamma (j + 1, 2 iy)}{(iy)^{j + 1}}
  \end{equation*}
  where $\gamma (s, x)$ denotes the lower incomplete gamma function.
\end{lemma}

\begin{proof}
  Substitute $t = \frac{1 - x}{2} \Longrightarrow x = 1 - 2 t$, $dx = - 2 dt$,
  adjusting limits:
  \begin{equation*}
    \int_1^0 t^j e^{i (1 - 2 t) y}  (- 2 dt) = 2 e^{iy}  \int_0^1 t^j e^{- 2
    iyt} dt
  \end{equation*}
  Let $u = 2 iyt \Longrightarrow t = \frac{u}{2 iy}$, $dt = \frac{d u}{2 i
  y}$:
  \begin{equation*}
     \frac{2 e^{iy}}{(2 iy)^{j + 1}}  \int_0^{2 iy} u^j e^{- u} du =
     \frac{e^{iy}}{2^j}  \frac{\gamma (j + 1, 2 iy)}{(iy)^{j + 1}} 
  \end{equation*}
\end{proof}

\subsection{Legendre Polynomial Representation}
\begin{lemma}[Legendre Polynomial Representation]
  \label{lem:Legendre}
  The Legendre polynomials are hypergeometric functions
  \begin{equation*}
    P_m (x) = {}_2 F_1 (- m, m + 1 ; 1 ; \tfrac{1 - x}{2})
  \end{equation*}
\end{lemma}

\begin{proof}
  From the Rodrigues formula $P_m (x) = \frac{1}{2^m m!}  \frac{d^m}{dx^m}
  (x^2 - 1)^m$ expand using the binomial theorem:
  \begin{equation*}
    (x^2 - 1)^m = \sum_{k = 0}^m (- 1)^{m - k} \binom{m}{k} x^{2 k}
  \end{equation*}
  Differentiating $m$ times yields terms proportional to $x^k$, matching the
  hypergeometric series:
  \begin{equation*}
    P_m (x) = {}_2 F_1 \left( - m, m + 1 ; 1 ; \tfrac{1 - x}{2} \right)
  \end{equation*}
\end{proof}

\section{Main Theorem}

\begin{theorem}[Fourier Transform of Legendre Polynomial Products]
  \label{thm:MainResult}
  \begin{equation*}
    \begin{aligned}
      I_{m, n} (y) & = \int_{- 1}^1 P_m (x) P_n (x) e^{ixy} dx \\
      & = e^{iy}  \sum_{j = 0}^{m + n} \frac{\Psi_j (m, n)}{2^j} \left[
      \frac{\gamma (j + 1, 2 iy)}{(iy)^{j + 1}} \right] \\
      & = e^{iy}  \sum_{j = 0}^{m + n} \frac{\frac{{}_4 F_3
      \left( \begin{array}{c}
        - m, m + 1, - n, n + 1 \\
        1, 1, j + 1
      \end{array} ; 1 \right)}{j!}}{2^j} \left[ \frac{\gamma (j + 1, 2
      iy)}{(iy)^{j + 1}} \right] \\
      & = e^{iy}  \sum_{j = 0}^{m + n} \frac{{}_4 F_3 \left(
      \begin{array}{c}
        - m, m + 1, - n, n + 1 \\
        1, 1, j + 1
      \end{array} ; 1 \right)}{j! 2^j} \frac{\gamma (j + 1, 2 iy)}{(iy)^{j +
      1}} \\
      & = e^{iy}  \sum_{j = 0}^{m + n} \frac{\gamma (j + 1, 2 iy)}{j! 2^j
      (iy)^{j + 1}} {}_4 F_3 \left( \begin{array}{c}
        - m, m + 1, - n, n + 1 \\
        1, 1, j + 1
      \end{array} ; 1 \right)
    \end{aligned}
  \end{equation*}
  where
  \begin{equation*}
    \Psi_j (m, n) = \frac{{}_4 F_3 \left( \begin{array}{c}
      - m, m + 1, - n, n + 1 \\
      1, 1, j + 1
    \end{array} ; 1 \right)}{j!}
  \end{equation*}
\end{theorem}

\begin{proof}
  \textbf{Part 1: Integral Reduction}

  Expand $P_m (x) P_n (x)$ using Lemma \ref{lem:HyperExpansions}:
  \begin{equation*}
    P_m (x) P_n (x) = \sum_{k = 0}^m \sum_{\ell = 0}^n \frac{(- m)_k (m +
    1)_k (- n)_{\ell} (n + 1)_{\ell}}{(1)_k (1)_{\ell} k! \ell !} \left(
    \tfrac{1 - x}{2} \right)^{k + \ell}
  \end{equation*}
  Let $j = k + \ell$, valid for $0 \leq k \leq m$, $0 \leq \ell \leq n$. Then:
  \begin{equation*}
    I_{m, n} (y) = \sum_{j = 0}^{m + n} \underbrace{\sum_{k = \max (0, j -
    n)}^{\min (j, m)} \frac{(- m)_k (m + 1)_k (- n)_{j - k} (n + 1)_{j -
    k}}{(1)_k (1)_{j - k} k! (j - k) !}}_{\Psi_j (m, n)} \int_{- 1}^1 \left(
    \tfrac{1 - x}{2} \right)^j e^{ixy} dx
  \end{equation*}
  Apply Lemma \ref{lem:IntegralGamma} to obtain the result.

  \textbf{Part 2: $\Psi_j (m, n)$ as a $_4 F_3$ Function}

  Expand the $_4 F_3$ series:
  \begin{equation*}
    \begin{aligned}
      {}_4 F_3 \left( \begin{array}{c}
        - m, m + 1, - n, n + 1 \\
        1, 1, j + 1
      \end{array} ; 1 \right) & = \sum_{k = 0}^j \frac{(- m)_k (m + 1)_k (-
      n)_k (n + 1)_k}{(1)_k (1)_k (j + 1)_k k!} \\
      & = \sum_{k = 0}^j \frac{(- m)_k (m + 1)_k (- n)_k (n + 1)_k}{k! k!
      (j + 1)_k k!} \\
      & = \sum_{k = 0}^j \frac{(0 - m)_k (1 + m)_k (0 - n)_k (1 + n)_k}{(1
      + j)_k (k!)^3 }
    \end{aligned}
  \end{equation*}
  The main result follows.
\end{proof}

\end{document}
