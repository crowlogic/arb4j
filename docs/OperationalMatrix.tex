\documentclass[12pt]{article}
\usepackage{amsmath, amssymb, amsthm, geometry, mathrsfs}
\geometry{margin=1in}

\newtheorem{theorem}{Theorem}
\newtheorem{lemma}[theorem]{Lemma}
\newtheorem{corollary}[theorem]{Corollary}

\title{The Operational Matrix of the Random Wave Process}
\author{Your Name \\ Department of Mathematics \\ Your Institution \\ Email: your.email@example.com}
\date{\today}

% Custom commands
\DeclareMathOperator{\F}{F}
\newcommand{\pFq}[2]{{}_{#1}\F_{#2}}

\begin{document}

\maketitle

\begin{lemma}\label{lem:HyperExpansions}
For any \( p \in \mathbb{Z}_{\geq 0} \), the Gauss hypergeometric function terminates:
\[
{}_2F_1(-p, b; c; z) = \sum_{k=0}^p \frac{(-p)_k (b)_k}{(c)_k k!} z^k,
\]
where \((a)_k = \prod_{i=0}^{k-1}(a+i)\).
\end{lemma}

\begin{lemma}\label{lem:IntegralGamma}
For \( j \geq 0 \), 
\[
\int_{-1}^1 \left(\frac{1-x}{2}\right)^j e^{ixy}dx = \frac{e^{iy}}{2^j}\frac{\gamma(j+1,2iy)}{(iy)^{j+1}},
\]
where \(\gamma(s, x)\) denotes the lower incomplete gamma function.
\end{lemma}

\begin{theorem}\label{thm:MainResult}
For \( m,n \geq 0 \),
\[
I_{m,n}(y) = \int_{-1}^1 {}_2F_1\left(-m,m+1;1;\tfrac{1-x}{2}\right){}_2F_1\left(-n,n+1;1;\tfrac{1-x}{2}\right)e^{ixy}dx
\]
satisfies:
\[
I_{m,n}(y) = e^{iy}\sum_{j=0}^{m+n}\frac{\Psi_j(m,n)}{2^j}\left[\frac{\gamma(j+1,2iy)}{(iy)^{j+1}}\right],
\]
where \( \Psi_j(m,n) \) is defined as:
\[
\Psi_j(m,n) = \sum_{k=\max(0,j-n)}^{\min(j,m)} \frac{(-m)_k(m+1)_k}{k!}\frac{(-n)_{j-k}(n+1)_{j-k}}{(j-k)!},
\]
and equivalently:
\[
\Psi_j(m,n) = \frac{1}{j!}\,{}_4F_3\left(\begin{array}{c} -m, m+1, -n, n+1 \\ 1, 1, j+1 \end{array};1\right).
\]
\end{theorem}

\begin{proof}
\textbf{Part 1: Integral Reduction to Finite Sums}

Expand both hypergeometric series using Lemma \ref{lem:HyperExpansions}:
\[
{}_2F_1\left(-m,m+1;1;\tfrac{1-x}{2}\right){}_2F_1\left(-n,n+1;1;\tfrac{1-x}{2}\right) = \sum_{k=0}^m \sum_{\ell=0}^n \frac{(-m)_k(m+1)_k}{k!} \frac{(-n)_\ell(n+1)_\ell}{\ell!} \left(\tfrac{1-x}{2}\right)^{k+\ell}.
\]
Let \( j = k + \ell \). For fixed \( j \), \( k \) must satisfy \( \max(0,j-n) \leq k \leq \min(j,m) \). Thus:
\[
I_{m,n}(y) = \sum_{j=0}^{m+n} \sum_{k=\max(0,j-n)}^{\min(j,m)} \frac{(-m)_k(m+1)_k}{k!} \frac{(-n)_{j-k}(n+1)_{j-k}}{(j-k)!} \int_{-1}^1 \left(\frac{1-x}{2}\right)^j e^{ixy}dx.
\]
Apply Lemma \ref{lem:IntegralGamma} to evaluate the integral:
\[
I_{m,n}(y) = e^{iy}\sum_{j=0}^{m+n}\frac{\Psi_j(m,n)}{2^j}\left[\frac{\gamma(j+1,2iy)}{(iy)^{j+1}}\right].
\]

\textbf{Part 2: Equivalence of \( \Psi_j(m,n) \) and \( {}_4F_3 \)}

Start from the hypergeometric representation:
\[
\Psi_j(m,n) = \frac{1}{j!}\,{}_4F_3\left(\begin{array}{c} -m, m+1, -n, n+1 \\ 1, 1, j+1 \end{array};1\right).
\]
Expand the \( {}_4F_3 \) series:
\[
{}_4F_3\left(\begin{array}{c} -m, m+1, -n, n+1 \\ 1, 1, j+1 \end{array};1\right) = \sum_{k=0}^\infty \frac{(-m)_k(m+1)_k(-n)_k(n+1)_k}{(1)_k(1)_k(j+1)_k k!}.
\]
The series terminates at \( k = \min(m,n) \) due to \((-m)_k = 0\) for \( k > m \) and \((-n)_k = 0\) for \( k > n \). Perform the substitution \( \ell = j - k \):
\[
\Psi_j(m,n) = \sum_{k=\max(0,j-n)}^{\min(j,m)} \frac{(-m)_k(m+1)_k}{k!}\frac{(-n)_{j-k}(n+1)_{j-k}}{(j-k)!}.
\]
Using the identity \((j+1)_{j-\ell} = \frac{(2j - \ell)!}{j!}\), we simplify:
\[
\frac{1}{(j+1)_{j-\ell}} = \frac{j!}{(2j - \ell)!}.
\]
Substituting this into the expression for \( \Psi_j(m,n) \) yields:
\[
\Psi_j(m,n) = \sum_{\ell=0}^j \frac{(-m)_{j-\ell}(m+1)_{j-\ell}(-n)_{j-\ell}(n+1)_{j-\ell}}{(1)_{j-\ell}(1)_{j-\ell}(2j - \ell)!(j - \ell)!}.
\]
Reverse the substitution (\( k = j - \ell \)) to obtain:
\[
\Psi_j(m,n) = \sum_{k=\max(0,j-n)}^{\min(j,m)} \frac{(-m)_k(m+1)_k}{k!}\frac{(-n)_{j-k}(n+1)_{j-k}}{(j-k)!}.
\]
This establishes term-by-term equality, confirming the hypergeometric representation:
\[
\Psi_j(m,n) = \frac{1}{j!}\,{}_4F_3\left(\begin{array}{c} -m, m+1, -n, n+1 \\ 1, 1, j+1 \end{array};1\right). \quad \blacksquare
\]
\end{proof}
\end{document}
