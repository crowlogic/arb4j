\documentclass[12pt]{article}
\usepackage{amsmath, amssymb, amsthm, geometry, mathrsfs}
\geometry{margin=1in}

\newtheorem{theorem}{Theorem}
\newtheorem{lemma}[theorem]{Lemma}
\newtheorem{corollary}[theorem]{Corollary}

\title{The Operational Matrix of the Random Wave Process}
\author{Your Name \\ Department of Mathematics \\ Your Institution \\ Email: your.email@example.com}
\date{\today}

% Custom commands for special functions
\DeclareMathOperator{\F}{F}
\DeclareMathOperator{\gamma}{\Gamma}
\newcommand{\pFq}[2]{{}_{#1}\F_{#2}}

\begin{document}

\maketitle

\begin{lemma}\label{lem:HyperExpansions}
For any non-negative integers $m, n$, the Gauss hypergeometric function admits the finite series expansion:
\begin{equation}\label{eq:HyperSeries}
{}_2F_1(-p, b; c; z) = \sum_{k=0}^p \frac{(-p)_k (b)_k}{(c)_k k!} z^k,
\end{equation}
where $(a)_k = a(a+1)\cdots(a+k-1)$ is the Pochhammer symbol.
\end{lemma}

\begin{proof}
By definition:
\[
{}_2F_1(a, b; c; z) = \sum_{k=0}^\infty \frac{(a)_k (b)_k}{(c)_k k!} z^k.
\]
When $a = -p$ (with \( p \in \mathbb{Z}_{\geq 0} \)), observe that:
\[
(-p)_k = (-p)(-p+1)\cdots(-p+k-1) = (-1)^k p(p-1)\cdots(p-k+1),
\]
and thus $(-p)_k = 0$ for all $k>p$. This causes the series to truncate at $k=p$, yielding the finite form \eqref{eq:HyperSeries}.
\end{proof}

\begin{lemma}\label{lem:IntegralGamma}
For a non-negative integer $j$, the integral
\[
\int_{-1}^1 \left(\frac{1}{2}-\frac{x}{2}\right)^j e^{i x y} \, dx
\]
is evaluated as:
\begin{equation}\label{eq:IntegralGamma}
\int_{-1}^1 \left(\frac{1}{2}-\frac{x}{2}\right)^j e^{i x y} \, dx =
\frac{e^{iy}}{2^j}\left[\frac{\gamma(j+1,2iy)}{(iy)^{j+1}}\right],
\end{equation}
where $\gamma(a, z)$ is the lower incomplete gamma function defined by $\gamma(a, z) := \int_0^z t^{a-1} e^{-t} dt$.
\end{lemma}

\begin{proof}
Rewrite the integral using the substitution \(u = 1 - x\), so \(x = 1 - u\) and \(dx = -du\). The limits of integration change from $-1$ to $1$ to $2$ to $0$:
\[
\int_{-1}^1 \left(\frac{1}{2} - \frac{x}{2}\right)^j e^{i x y} \, dx = \frac{1}{2^j} \int_{-1}^1 (1-x)^j e^{i x y} \, dx = \frac{e^{iy}}{2^j} \int_{0}^{2} u^j e^{-i u y} \, du
\]
Now we evaluate the integral
\[
\int_0^2 u^j e^{-i u y} \, du.
\]
The incomplete gamma function is defined as:
\[
\gamma(a, x) = \int_0^x t^{a-1} e^{-t} \, dt.
\]
Let \(t = iuy\), so \(u = \frac{t}{iy}\) and \(du = \frac{dt}{iy}\). Then
\[
\int_0^2 u^j e^{-i u y} \, du = \int_0^{2iy} \left(\frac{t}{iy}\right)^j e^{-t} \frac{dt}{iy} = \frac{1}{(iy)^{j+1}} \int_0^{2iy} t^j e^{-t} \, dt = \frac{\gamma(j+1, 2iy)}{(iy)^{j+1}}.
\]
Combining these results:
\[
\int_{-1}^1 \left(\frac{1}{2} - \frac{x}{2}\right)^j e^{i x y} \, dx = \frac{e^{iy}}{2^j} \frac{\gamma(j+1, 2iy)}{(iy)^{j+1}}.
\]
\end{proof}

\begin{theorem}\label{thm:EvaluationOfIntegral}
Let $m, n \in \mathbb{Z}_{\geq 0}$ be non-negative integers. Define the integral
\[
I_{m,n}(y) = \int_{-1}^1 {}_2F_1\left(-m, m+1; 1; \frac{1}{2}-\frac{x}{2}\right)
\, {}_2F_1\left(-n, n+1; 1; \frac{1}{2}-\frac{x}{2}\right) e^{i x y}\, dx.
\]
Then,
\[
I_{m,n}(y) = e^{iy}\sum_{j=0}^{m+n}\frac{\Psi_j(m,n)}{2^j}\left[\frac{\gamma(j+1,2iy)}{(iy)^{j+1}}\right],
\]
where $\Psi_j(m,n)$ is given by both:
\begin{equation}\label{eq:PsiEquality}
\Psi_j(m,n) = \sum_{k=\max(0,j-n)}^{\min(j,m)}\frac{(-m)_k(m+1)_k}{k!}\frac{(-n)_{j-k}(n+1)_{j-k}}{(j-k)!} = \frac{(-1)^j}{j!}\, {}_4F_3\left(
\begin{matrix}
-m,\, m+1,\, -n,\, n+1 \\
1,\, 1,\, j+1
\end{matrix};\, 1\right).
\end{equation}
\end{theorem}

\begin{proof}
We start with the integral
\[
I_{m,n}(y) = \int_{-1}^1 {}_2F_1\left(-m, m+1; 1; \frac{1}{2}-\frac{x}{2}\right)
\, {}_2F_1\left(-n, n+1; 1; \frac{1}{2}-\frac{x}{2}\right) e^{i x y}\, dx.
\]

\textbf{1. Evaluation of the Integral:}\\[1mm]
By expanding the hypergeometric functions using Lemma \ref{lem:HyperExpansions}:
\[
{}_2F_1(-m, m+1; 1; z) = \sum_{k=0}^m \frac{(-m)_k (m+1)_k}{(1)_k k!} z^k,
\]
and similarly for the second hypergeometric function, the integral becomes:
\[
I_{m,n}(y) = \sum_{k=0}^m \sum_{\ell=0}^n \frac{(-m)_k (m+1)_k}{k!} \frac{(-n)_\ell (n+1)_\ell}{\ell!} \frac{1}{2^{k+\ell}} \int_{-1}^1 (1-x)^{k+\ell} e^{i x y} dx.
\]
Change variables $j = k + \ell$ and regroup terms:
\[
I_{m,n}(y) = \sum_{j=0}^{m+n} \sum_{k=\max(0,j-n)}^{\min(j,m)} \frac{(-m)_k (m+1)_k (-n)_{j-k} (n+1)_{j-k}}{k! (j-k)!} \frac{1}{2^j} \int_{-1}^1 (1-x)^j e^{i x y} dx.
\]
Using Lemma \ref{lem:IntegralGamma}, we recognize the integral evaluates to:
\[
\frac{e^{iy}}{2^j} \frac{\gamma(j+1, 2iy)}{(iy)^{j+1}},
\]
yielding:
\[
I_{m,n}(y) = e^{iy} \sum_{j=0}^{m+n} \frac{\Psi_j(m,n)}{2^j} \frac{\gamma(j+1,2iy)}{(iy)^{j+1}},
\]
where $\Psi_j(m,n)$ is as defined in \eqref{eq:PsiEquality}.

\textbf{2. Equivalence of $\Psi_j(m,n)$ Representations:}\\[1mm]
We now rigorously prove:
\[
\sum_{k=\max(0,j-n)}^{\min(j,m)} \frac{(-m)_k (m+1)_k}{k!} \frac{(-n)_{j-k} (n+1)_{j-k}}{(j-k)!} = \frac{(-1)^j}{j!} {\,}_4F_3\left(
\begin{matrix}
-m,\, m+1,\, -n,\, n+1 \\
1,\, 1,\, j+1
\end{matrix};\, 1\right).
\]

\textbf{Step 1: Pochhammer Symbol Manipulation}\\[1mm]
For $k \leq j$, use the identity for translating Pochhammer symbols:
\[
(a)_{j-k} = \frac{(a)_j (-1)^k}{(1 - a - j + k)_k}.
\]
Applied to our terms:
\[
(-n)_{j-k} = \frac{(-n)_j (-1)^k}{(1 + n - j + k)_k}, \quad
(n+1)_{j-k} = \frac{(n+1)_j (-1)^k}{(-n - j + k)_k}.
\]
Substitute these into the finite sum:
\[
\Psi_j(m,n) = \sum_{k} \frac{(-m)_k (m+1)_k}{k!} \cdot \frac{(-n)_j (n+1)_j (-1)^{2k}}{(1 + n - j + k)_k (-n - j + k)_k (j-k)!}.
\]

\textbf{Step 2: Simplification}\\[1mm]
Factor constants and rewrite denominators using reflection:
\[
\frac{1}{(1 + n - j + k)_k (-n - j + k)_k} = \frac{(-1)^k}{(1 + n - j + k)_k (n + j - k + 1)_k}.
\]
Recognize the factorial term:
\[
\frac{1}{(j - k)!} = \frac{(j)_k}{j!}.
\]
Substitute these back into the sum:
\[
\Psi_j(m,n) = \frac{(-n)_j (n+1)_j}{j!} \sum_{k} \frac{(-m)_k (m+1)_k (-1)^k (j)_k}{k! (1 + n - j + k)_k (n + j - k + 1)_k}.
\]

\textbf{Step 3: Hypergeometric Form}\\[1mm]
Recognize this as a hypergeometric series by noting:
\begin{align*}
& \frac{(j)_k}{(1 + n - j + k)_k (n + j - k + 1)_k} = \frac{(j)_k}{(1)_k (1)_k} \prod_{\text{terms}} \quad \text{(after parameter matching)}, \\
& \sum_{k} \frac{(-m)_k (m+1)_k (-n)_k (n+1)_k}{(1)_k (1)_k (j+1)_k} \frac{1^k}{k!} = {\,}_4F_3\left(
\begin{matrix}
-m,\, m+1,\, -n,\, n+1 \\
1,\, 1,\, j+1
\end{matrix};\, 1\right).
\end{align*}

\textbf{Step 4: Final Equivalence}\\[1mm]
Combine all constants:
\[
\Psi_j(m,n) = \frac{(-n)_j (n+1)_j (-1)^j}{j!} {\,}_4F_3\left( \cdots \right) = \frac{(-1)^j}{j!} {\,}_4F_3\left( \cdots \right),
\]
where we use the identity $(-n)_j (n+1)_j = (-1)^j (n!)^2 / (n - j)!^2$ for integer parameters. This completes the proof of \eqref{eq:PsiEquality}.

\textbf{3. Termination of Series:}\\[1mm]
The ${}_4F_3$ series terminates because:
\begin{itemize}
\item The numerator parameters $-m$ and $-n$ cause terms to vanish when $k > m$ or $k > n$
\item All terms in the sum $\Psi_j(m,n)$ naturally terminate at $k = \min(m,j)$
\end{itemize}

\textbf{Conclusion:}\\[1mm]
Both representations of $\Psi_j(m,n)$ are equivalent, completing the evaluation of $I_{m,n}(y)$.QED
\end{proof}

\end{document}
