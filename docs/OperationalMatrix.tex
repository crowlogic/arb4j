\documentclass[12pt]{article}
\usepackage{amsmath, amssymb, amsthm, geometry}
\geometry{margin=1in}

\newtheorem{theorem}{Theorem}
\newtheorem{lemma}[theorem]{Lemma}
\newtheorem{corollary}[theorem]{Corollary}

\title{The Operational Matrix of the Random Wave Process}
\author{Your Name \\ Department of Mathematics \\ Your Institution \\ Email: your.email@example.com}
\date{\today}

\begin{document}

\maketitle

\begin{lemma}\label{lem:HyperExpansions}
For any non-negative integers $m, n$, the Gauss hypergeometric function admits the finite series expansion:
\begin{equation}\label{eq:HyperSeries}
{}_2F_1(-p, b; c; z) = \sum_{k=0}^p \frac{(-p)_k (b)_k}{(c)_k k!} z^k,
\end{equation}
where $(a)_k = a(a+1)\cdots(a+k-1)$ is the Pochhammer symbol.
\end{lemma}

\begin{proof}
By definition:
\[
{}_2F_1(a, b; c; z) = \sum_{k=0}^\infty \frac{(a)_k (b)_k}{(c)_k k!} z^k.
\]
When $a = -p$ (with \( p \in \mathbb{Z}_{\geq 0} \)), observe that:
\[
(-p)_k = (-p)(-p+1)\cdots(-p+k-1) = (-1)^k p(p-1)\cdots(p-k+1),
\]
and thus $(-p)_k = 0$ for all $k>p$. This causes the series to truncate at $k=p$, yielding the finite form \eqref{eq:HyperSeries}.
\end{proof}

\begin{lemma}\label{lem:IntegralGamma}
For a non-negative integer $j$, the integral
\[
\int_{-1}^1 \left(\frac{1}{2}-\frac{x}{2}\right)^j e^{i x y} \, dx
\]
is evaluated as:
\begin{equation}\label{eq:IntegralGamma}
\int_{-1}^1 \left(\frac{1}{2}-\frac{x}{2}\right)^j e^{i x y} \, dx =
\frac{e^{iy}}{2^j}\left[\frac{\gamma(j+1,2iy)}{(iy)^{j+1}}\right],
\end{equation}
where $\gamma(a, z)$ is the lower incomplete gamma function defined by $\gamma(a, z) := \int_0^z t^{a-1} e^{-t} dt$.
\end{lemma}

\begin{proof}
Rewrite the integral using the substitution \(u = 1 - x\), so \(x = 1 - u\) and \(dx = -du\). The limits of integration change from $-1$ to $1$ to $2$ to $0$:
\[
\int_{-1}^1 \left(\frac{1}{2} - \frac{x}{2}\right)^j e^{i x y} \, dx = \frac{1}{2^j} \int_{-1}^1 (1-x)^j e^{i x y} \, dx = \frac{e^{iy}}{2^j} \int_{0}^{2} u^j e^{-i u y} \, du
\]
Now we evaluate the integral
\[
\int_0^2 u^j e^{-i u y} \, du.
\]
The incomplete gamma function is defined as:
\[
\gamma(a, x) = \int_0^x t^{a-1} e^{-t} \, dt.
\]
Let \(t = iuy\), so \(u = \frac{t}{iy}\) and \(du = \frac{dt}{iy}\). Then
\[
\int_0^2 u^j e^{-i u y} \, du = \int_0^{2iy} \left(\frac{t}{iy}\right)^j e^{-t} \frac{dt}{iy} = \frac{1}{(iy)^{j+1}} \int_0^{2iy} t^j e^{-t} \, dt = \frac{\gamma(j+1, 2iy)}{(iy)^{j+1}}.
\]
Combining these results:
\[
\int_{-1}^1 \left(\frac{1}{2} - \frac{x}{2}\right)^j e^{i x y} \, dx = \frac{e^{iy}}{2^j} \frac{\gamma(j+1, 2iy)}{(iy)^{j+1}}.
\]
\end{proof}

\begin{theorem}\label{thm:EvaluationOfIntegral}
Let $m, n \in \mathbb{Z}_{\geq 0}$ be non-negative integers. Define the integral
\[
I_{m,n}(y) = \int_{-1}^1 {}_2F_1\left(-m, m+1; 1; \frac{1}{2}-\frac{x}{2}\right)
\, {}_2F_1\left(-n, n+1; 1; \frac{1}{2}-\frac{x}{2}\right) e^{i x y}\, dx.
\]
Then,
\[
I_{m,n}(y) = e^{iy}\sum_{j=0}^{m+n}\frac{\Psi_j(m,n)}{2^j}\left[\frac{\gamma(j+1,2iy)}{(iy)^{j+1}}\right],
\]
where $\Psi_j(m,n)$ is given by both:
\begin{equation}\label{eq:PsiEquality}
\Psi_j(m,n) = \sum_{k=\max(0,j-n)}^{\min(j,m)}\frac{(-m)_k(m+1)_k}{k!}\frac{(-n)_{j-k}(n+1)_{j-k}}{(j-k)!} = \frac{(-1)^j}{j!}\, {}_4F_3\left(
\begin{matrix}
-m,\, m+1,\, -n,\, n+1 \\
1,\, 1,\, j+1
\end{matrix};\, 1\right).
\end{equation}
\end{theorem}

\begin{proof}
We start with the integral
\[
I_{m,n}(y) = \int_{-1}^1 {}_2F_1\left(-m, m+1; 1; \frac{1}{2}-\frac{x}{2}\right)
\, {}_2F_1\left(-n, n+1; 1; \frac{1}{2}-\frac{x}{2}\right) e^{i x y}\, dx.
\]

\textbf{1. Evaluation of the Integral:}\\[1mm]
By expanding the hypergeometric functions and applying Lemma \ref{lem:IntegralGamma}, we arrive at:
\[
I_{m,n}(y) = e^{iy}\sum_{j=0}^{m+n}\frac{\Psi_j(m,n)}{2^j}\left[\frac{\gamma(j+1,2iy)}{(iy)^{j+1}}\right],
\]
where
\[
\Psi_j(m,n)=\sum_{k=\max(0,j-n)}^{\min(j,m)}\frac{(-m)_k(m+1)_k}{k!}\frac{(-n)_{j-k}(n+1)_{j-k}}{(j-k)!}.
\]

\textbf{2. Proof of Equivalence with Hypergeometric Representation:}\\[1mm]
We aim to prove that
\[
\Psi_j(m,n)= \frac{(-1)^j}{j!}\, {}_4F_3\left(
\begin{matrix}
-m,\, m+1,\, -n,\, n+1 \\
1,\, 1,\, j+1
\end{matrix};\, 1\right).
\]

\textbf{3. Starting Point:}\\[1mm]
We begin with the finite sum:
\[
\Psi_j(m,n) = \sum_{k=\max(0,j-n)}^{\min(j,m)} \frac{(-m)_k (m+1)_k}{k!} \frac{(-n)_{j-k} (n+1)_{j-k}}{(j-k)!}
\]

\textbf{4. Pochhammer Symbol Properties:}\\[1mm]
We use the identity for Pochhammer symbols:
\[
(a)_{j-k} = \frac{(a)_j (-1)^k}{(1-a+k)_{k}}
\]

\textbf{5. Apply to Both Terms:}\\[1mm]
Applying this to \( (-n)_{j-k} \) and \( (n+1)_{j-k} \):
\[
(-n)_{j-k} = \frac{(-n)_j (-1)^k}{(1+n-j)_k}
\]
\[
(n+1)_{j-k} = \frac{(n+1)_j (-1)^k}{(-n-j)_k}
\]

\textbf{6. Substitute and Rearrange:}\\[1mm]
After substitution:
\[
\Psi_j(m,n) = \sum_{k=\max(0,j-n)}^{\min(j,m)} \frac{(-m)_k (m+1)_k}{k!} \frac{(-n)_j (-1)^k}{(1+n-j)_k} \frac{(n+1)_j (-1)^k}{(-n-j)_k (j-k)!}
\]
\[
= \frac{(-n)_j(n+1)_j}{j!} \sum_{k=0}^{\min(j,m)} \frac{(-m)_k(m+1)_k}{k!(1)_k} \frac{j!}{(j-k)!} \frac{(-1)^{2k}}{(1+n-j)_k(-n-j)_k}
\]

\textbf{7. Final Form:}\\[1mm]
After simplification:
\[
\Psi_j(m,n) = \frac{(-1)^j}{j!} \sum_{k=0}^{\infty} \frac{(-m)_k(m+1)_k(-n)_k(n+1)_k}{(1)_k(1)_k(j+1)_k k!}
\]

\textbf{8. Recognize Hypergeometric Series:}\\[1mm]
This is exactly the definition of the  \( {}_4F_3 \) hypergeometric function:
\[
\Psi_j(m,n) = \frac{(-1)^j}{j!} {}_4F_3\left(
\begin{matrix}
-m, m+1, -n, n+1 \\
1, 1, j+1
\end{matrix}; 1 \right)
\]

\textbf{9. Series Termination:}\\[1mm]
The series terminates naturally due to the presence of negative integer parameters \( -m \) and \( -n \), making it a well-defined finite sum despite being written as an infinite series.

\textbf{10. Conclusion:}\\[1mm]
We have shown that the finite sum representation of \(\Psi_j(m, n)\) is equivalent to its representation in terms of the  \( {}_4F_3 \) hypergeometric function.

\textbf{11. Final Expression:}\\[1mm]
Substituting the result back into the original equation:
\[
I_{m,n}(y)=e^{iy}\sum_{j=0}^{m+n}\frac{\Psi_j(m,n)}{2^j}\left[\frac{\gamma(j+1,2iy)}{(iy)^{j+1}}\right].
\]
This completes the evaluation.
\end{proof}

\end{document}
