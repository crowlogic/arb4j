\documentclass[12pt]{article}
\usepackage{amsmath, amssymb, amsthm, geometry, mathrsfs}
\geometry{margin=1in}

\newtheorem{theorem}{Theorem}
\newtheorem{lemma}[theorem]{Lemma}
\newtheorem{corollary}[theorem]{Corollary}

\title{The Operational Matrix of the Random Wave Process}
\author{Your Name \\ Department of Mathematics \\ Your Institution \\ Email: your.email@example.com}
\date{\today}

% Custom commands for special functions
\DeclareMathOperator{\F}{F}
\DeclareMathOperator{\gamma}{\Gamma}
\newcommand{\pFq}[2]{{}_{#1}\F_{#2}}

\begin{document}

\maketitle

\begin{lemma}\label{lem:HyperExpansions}
For any non-negative integers $m, n$, the Gauss hypergeometric function admits the finite series expansion:
\begin{equation}\label{eq:HyperSeries}
{}_2F_1(-p, b; c; z) = \sum_{k=0}^p \frac{(-p)_k (b)_k}{(c)_k k!} z^k,
\end{equation}
where $(a)_k = a(a+1)\cdots(a+k-1)$ is the Pochhammer symbol.
\end{lemma}

\begin{proof}
By definition:
\[
{}_2F_1(a, b; c; z) = \sum_{k=0}^\infty \frac{(a)_k (b)_k}{(c)_k k!} z^k.
\]
For $a = -p$ with $p \in \mathbb{Z}_{\geq 0}$, the term $(-p)_k$ satisfies:
\[
(-p)_k = (-1)^k p(p-1)\cdots(p-k+1),
\]
which becomes zero for $k > p$. This reduces the infinite series to the finite sum in \eqref{eq:HyperSeries}.
\end{proof}

\begin{lemma}\label{lem:IntegralGamma}
For $j \in \mathbb{Z}_{\geq 0}$, the integral evaluation holds:
\begin{equation}\label{eq:IntegralGamma}
\int_{-1}^1 \left(\frac{1}{2}-\frac{x}{2}\right)^j e^{i x y} dx = \frac{e^{iy}}{2^j}\frac{\gamma(j+1,2iy)}{(iy)^{j+1}},
\end{equation}
where $\gamma(a, z) := \int_0^z t^{a-1} e^{-t} dt$.
\end{lemma}

\begin{proof}
Using substitution $u = 1 - x$, the integral becomes:
\[
\int_{-1}^1 \left(\frac{1}{2}-\frac{x}{2}\right)^j e^{i x y} dx = \frac{1}{2^j} \int_0^{2} u^j e^{-iuy} du.
\]
Applying the substitution $t = iuy$ gives $u = \frac{t}{iy}$ and $du = \frac{dt}{iy}$, so:
\[
\int_0^2 u^j e^{-i u y} du = \int_0^{2iy} \left(\frac{t}{iy}\right)^j e^{-t} \frac{dt}{iy} = \frac{1}{(iy)^{j+1}} \int_0^{2iy} t^j e^{-t} dt.
\]
The integral $\int_0^{2iy} t^j e^{-t} dt$ corresponds to the incomplete gamma function $\gamma(j+1, 2iy)$, yielding:
\[
\int_0^2 u^j e^{-i u y} du = \frac{\gamma(j+1, 2iy)}{(iy)^{j+1}}.
\]
Substituting back, the integral evaluates to:
\[
\int_{-1}^1 \left(\frac{1}{2}-\frac{x}{2}\right)^j e^{i x y} dx = \frac{e^{iy}}{2^j} \frac{\gamma(j+1, 2iy)}{(iy)^{j+1}}.
\]
\end{proof}

\begin{theorem}\label{thm:EvaluationOfIntegral}
For $m, n \in \mathbb{Z}_{\geq 0}$, define:
\[
I_{m,n}(y) = \int_{-1}^1 {}_2F_1\left(-m, m+1; 1; \frac{1-x}{2}\right) {}_2F_1\left(-n, n+1; 1; \frac{1-x}{2}\right) e^{i x y} dx.
\]
Then:
\[
I_{m,n}(y) = e^{iy}\sum_{j=0}^{m+n}\frac{\Psi_j(m,n)}{2^j}\left[\frac{\gamma(j+1,2iy)}{(iy)^{j+1}}\right],
\]
where $\Psi_j(m,n)$ satisfies:
\begin{equation}\label{eq:PsiEquality}
\Psi_j(m,n) = \sum_{k=\max(0,j-n)}^{\min(j,m)}\frac{(-m)_k(m+1)_k}{k!}\frac{(-n)_{j-k}(n+1)_{j-k}}{(j-k)!}.
\end{equation}
\end{theorem}

\begin{proof}
Expanding both hypergeometric functions using Lemma \ref{lem:HyperExpansions} yields:
\[
{}_2F_1(-m, m+1; 1; z) = \sum_{k=0}^m \frac{(-m)_k (m+1)_k}{(1)_k k!} z^k,
\]
and similarly for the second hypergeometric function. Substituting these into the integral gives:
\[
I_{m,n}(y) = \sum_{k=0}^m \sum_{\ell=0}^n \frac{(-m)_k (m+1)_k}{k!} \frac{(-n)_\ell (n+1)_\ell}{\ell!} \frac{1}{2^{k+\ell}} \int_{-1}^1 (1-x)^{k+\ell} e^{i x y} dx.
\]
Letting $j = k + \ell$, the terms reorganize as:
\[
I_{m,n}(y) = \sum_{j=0}^{m+n} \sum_{k=\max(0,j-n)}^{\min(j,m)} \frac{(-m)_k(m+1)_k}{k!} \frac{(-n)_{j-k}(n+1)_{j-k}}{(j-k)!} \frac{1}{2^j} \int_{-1}^1 (1-x)^j e^{i x y} dx.
\]
Lemma \ref{lem:IntegralGamma} evaluates the integral:
\[
\int_{-1}^1 (1-x)^j e^{ixy} dx = \frac{e^{iy}}{2^j} \frac{\gamma(j+1, 2iy)}{(iy)^{j+1}}.
\]
Substituting this result gives:
\[
I_{m,n}(y) = e^{iy} \sum_{j=0}^{m+n} \frac{\Psi_j(m, n)}{2^j} \frac{\gamma(j+1, 2iy)}{(iy)^{j+1}},
\]
where
\[
\Psi_j(m, n) = \sum_{k=\max(0, j-n)}^{\min(j, m)} \frac{(-m)_k(m+1)_k}{k!} \frac{(-n)_{j-k}(n+1)_{j-k}}{(j-k)!}.
\]

Using the explicit definition of Pochhammer symbols:
\[
(-m)_k = (-1)^k \frac{\Gamma(m+1)}{\Gamma(m-k+1)}, \quad (-n)_{j-k} = (-1)^{j-k} \frac{\Gamma(n+1)}{\Gamma(n-j+k+1)},
\]
and writing out the factorials:
\[
\Psi_j(m,n) = \frac{(-1)^j \Gamma(m+1) \Gamma(n+1)}{j!} \sum_{k=\max(0,j-n)}^{\min(j,m)} \frac{(m+1)_k (n+1)_{j-k}}{\Gamma(m-k+1)\Gamma(n-j+k+1)k!(j-k)!}.
\]
Expressing $(m+1)_k$ and $(n+1)_{j-k}$ explicitly:
\[
(m+1)_k = \frac{\Gamma(m+k+1)}{\Gamma(m+1)}, \quad (n+1)_{j-k} = \frac{\Gamma(n+j-k+1)}{\Gamma(n+1)},
\]
leads to:
\[
\Psi_j(m, n) = \frac{1}{j!} \sum_{k=\max(0,j-n)}^{\min(j,m)} \frac{\Gamma(m+k+1)}{\Gamma(m-k+1)} \frac{\Gamma(n+j-k+1)}{\Gamma(n-j+k+1)} \frac{1}{k! (j-k)!}.
\]
Rewriting in terms of hypergeometric notation:
\[
\Psi_j(m, n) = \frac{1}{j!} {}_4F_3\left(\begin{smallmatrix}-m,\, m+1,\, -n,\, n+1 \\ 1,\, 1,\, j+1\end{smallmatrix}; 1\right).
\]
The hypergeometric series terminates because $(-m)_k = 0$ for $k > m$ and $(-n)_{j-k} = 0$ for $j-k > n$, enforcing the limits of summation.
\end{proof}

\end{document}
