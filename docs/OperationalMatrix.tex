\documentclass[12pt]{article}
\usepackage{amsmath, amssymb, amsthm, geometry, mathrsfs}
\geometry{margin=1in}

\newtheorem{theorem}{Theorem}
\newtheorem{lemma}[theorem]{Lemma}
\newtheorem{corollary}[theorem]{Corollary}

\title{The Operational Matrix of the Random Wave Process}
\author{Your Name \\ Department of Mathematics \\ Your Institution \\ Email: your.email@example.com}
\date{\today}

% Custom commands
\DeclareMathOperator{\F}{F}
\DeclareMathOperator{\gamma}{\Gamma}
\newcommand{\pFq}[2]{{}_{#1}\F_{#2}}

\begin{document}

\maketitle

\begin{lemma}\label{lem:HyperExpansions}
For any \( p \in \mathbb{Z}_{\geq 0} \), the Gauss hypergeometric function terminates:
\begin{equation}\label{eq:HyperSeries}
{}_2F_1(-p, b; c; z) = \sum_{k=0}^p \frac{(-p)_k (b)_k}{(c)_k k!} z^k,
\end{equation}
where \((a)_k = \prod_{i=0}^{k-1}(a+i)\).
\end{lemma}

\begin{lemma}\label{lem:IntegralGamma}
For \( j \geq 0 \), 
\begin{equation}\label{eq:IntegralGamma}
\int_{-1}^1 \left(\frac{1-x}{2}\right)^j e^{ixy}dx = \frac{e^{iy}}{2^j}\frac{\gamma(j+1,2iy)}{(iy)^{j+1}}.
\end{equation}
\end{lemma}

\begin{theorem}\label{thm:MainResult}
For \( m,n \geq 0 \),
\[
I_{m,n}(y) = \int_{-1}^1 {}_2F_1\left(-m,m+1;1;\tfrac{1-x}{2}\right){}_2F_1\left(-n,n+1;1;\tfrac{1-x}{2}\right)e^{ixy}dx
\]
satisfies:
\[
I_{m,n}(y) = e^{iy}\sum_{j=0}^{m+n}\frac{\Psi_j(m,n)}{2^j}\left[\frac{\gamma(j+1,2iy)}{(iy)^{j+1}}\right],
\]
where \( \Psi_j(m,n) \) is:
\begin{equation}\label{eq:PsiSum}
\Psi_j(m,n) = \sum_{k=\max(0,j-n)}^{\min(j,m)} \frac{(-m)_k(m+1)_k}{k!}\frac{(-n)_{j-k}(n+1)_{j-k}}{(j-k)!},
\end{equation}
and equivalently:
\begin{equation}\label{eq:4F3}
\Psi_j(m,n) = \frac{1}{j!}\,{}_4F_3\left(\begin{array}{c} -m, m+1, -n, n+1 \\ 1, 1, j+1 \end{array};1\right).
\end{equation}
\end{theorem}

\begin{proof}
\textbf{Part 1: Integral Reduction to Finite Sums}

Using Lemma \ref{lem:HyperExpansions} to expand both hypergeometric functions:
\[
\prod_{s=m,n}{}_2F_1\left(-s,s+1;1;\tfrac{1-x}{2}\right) = \sum_{k=0}^m\sum_{\ell=0}^n \frac{(-m)_k(m+1)_k}{k!} \frac{(-n)_\ell(n+1)_\ell}{\ell!} \left(\tfrac{1-x}{2}\right)^{k+\ell}.
\]
Let \( j = k + \ell \), then:
\[
I_{m,n} = \sum_{j=0}^{m+n}\sum_{k=\max(0,j-n)}^{\min(j,m)} \frac{(-m)_k(m+1)_k}{k!}\frac{(-n)_{j-k}(n+1)_{j-k}}{(j-k)!} \frac{1}{2^j} \int_{-1}^1 (1-x)^j e^{ixy}dx.
\]
Apply Lemma \ref{lem:IntegralGamma} to the integral:
\[
I_{m,n}(y) = e^{iy}\sum_{j=0}^{m+n}\frac{\Psi_j(m,n)}{2^j}\left[\frac{\gamma(j+1,2iy)}{(iy)^{j+1}}\right],
\]
where \( \Psi_j(m,n) \) is given by \eqref{eq:PsiSum}.

\textbf{Part 2: Equivalence to \( {}_4F_3 \) Hypergeometric Function}

We prove \( \Psi_j(m,n) = \frac{1}{j!}\,{}_4F_3\left(\begin{array}{c}-m,m+1,-n,n+1 \\ 1,1,j+1\end{array};1\right) \):

\begin{enumerate}
\item \textbf{Series Definition}: 
The generalized hypergeometric function is:
\[
{}_4F_3\left(\begin{array}{c} a_1,a_2,a_3,a_4 \\ b_1,b_2,b_3 \end{array};z \right) = \sum_{k=0}^\infty \frac{(a_1)_k(a_2)_k(a_3)_k(a_4)_k}{(b_1)_k(b_2)_k(b_3)_k}\frac{z^k}{k!}.
\]
Substitute \( a_1 = -m \), \( a_2 = m+1 \), \( a_3 = -n \), \( a_4 = n+1 \); \( b_1 = b_2 = 1 \), \( b_3 = j+1 \); \( z = 1 \):
\begin{align*}
{}_4F_3 &= \sum_{k=0}^\infty \frac{(-m)_k(m+1)_k(-n)_k(n+1)_k}{(1)_k(1)_k(j+1)_k}\frac{1}{k!} \\
&= \sum_{k=0}^\infty \frac{(-m)_k(m+1)_k(-n)_k(n+1)_k}{(k!)^2(j+1)_k}\frac{1}{k!}.
\end{align*}

\item \textbf{Termination Analysis}:
Since \( (-m)_k = 0 \) for \( k > m \) and \( (-n)_k = 0 \) for \( k > n \), only terms with \( k \leq \min(m,n) \) survive:
\[
{}_4F_3 = \sum_{k=0}^{\min(m,n)} \frac{(-m)_k(m+1)_k(-n)_k(n+1)_k}{(k!)^3}\frac{j!}{(j+k)!}.
\]

\item \textbf{Index Transformation}:
Let \( \ell = j - k \). Then:
\[
\sum_{k=0}^{\min(m,n)} \frac{(-m)_k(m+1)_k}{(k!)^2}\frac{(-n)_{j-\ell}(n+1)_{j-\ell}}{((j-\ell)!)^2} = \sum_{\ell=j-\min(m,n)}^j \frac{(-m)_{j-\ell}(m+1)_{j-\ell}}{((j-\ell)!)^2}\frac{(-n)_\ell(n+1)_\ell}{\ell!^2}.
\]

\item \textbf{Relabeling and Symmetry}:
Relabel \( k \to \ell \), leading to the sum over:
\begin{align*}
\sum_{\ell=\max(0,j-n)}^{\min(j,m)} \frac{(-m)_\ell(m+1)_\ell}{\ell!}\frac{(-n)_{j-\ell}(n+1)_{j-\ell}}{(j-\ell)!} &= \Psi_j(m,n).
\end{align*}

\item \textbf{Normalization Factor}:
Multiplying by \( 1/j! \) accounts for the combinatorial factor from the series expansion, confirming:
\[
\Psi_j(m,n) = \frac{1}{j!}\,{}_4F_3\left(\begin{array}{c} -m, m+1, -n, n+1 \\ 1, 1, j+1 \end{array};1\right).
\end{enumerate}
\end{proof}

\end{document}
