\documentclass[12pt]{article}
\usepackage{amsmath, amssymb, amsthm, geometry, mathrsfs}
\geometry{margin=1in}

\newtheorem{theorem}{Theorem}
\newtheorem{lemma}[theorem]{Lemma}
\newtheorem{corollary}[theorem]{Corollary}

\title{The Operational Matrix of the Random Wave Process}
\author{Your Name \\ Department of Mathematics \\ Your Institution \\ Email: your.email@example.com}
\date{\today}

% Custom commands for special functions
\DeclareMathOperator{\F}{F}
\DeclareMathOperator{\gamma}{\Gamma}
\newcommand{\pFq}[2]{{}_{#1}\F_{#2}}

\begin{document}

\maketitle

\begin{lemma}\label{lem:HyperExpansions}
For any non-negative integers $m, n$, the Gauss hypergeometric function admits the finite series expansion:
\begin{equation}\label{eq:HyperSeries}
{}_2F_1(-p, b; c; z) = \sum_{k=0}^p \frac{(-p)_k (b)_k}{(c)_k k!} z^k,
\end{equation}
where $(a)_k = a(a+1)\cdots(a+k-1)$ is the Pochhammer symbol.
\end{lemma}

\begin{proof}
By definition:
\[
{}_2F_1(a, b; c; z) = \sum_{k=0}^\infty \frac{(a)_k (b)_k}{(c)_k k!} z^k.
\]
When $a = -p$ (with \( p \in \mathbb{Z}_{\geq 0} \)), observe that:
\[
(-p)_k = (-p)(-p+1)\cdots(-p+k-1) = (-1)^k p(p-1)\cdots(p-k+1),
\]
and thus $(-p)_k = 0$ for all $k>p$. This causes the series to truncate at $k=p$, yielding the finite form \eqref{eq:HyperSeries}.
\end{proof}

\begin{lemma}\label{lem:IntegralGamma}
For a non-negative integer $j$, the integral
\[
\int_{-1}^1 \left(\frac{1}{2}-\frac{x}{2}\right)^j e^{i x y} \, dx
\]
is evaluated as:
\begin{equation}\label{eq:IntegralGamma}
\int_{-1}^1 \left(\frac{1}{2}-\frac{x}{2}\right)^j e^{i x y} \, dx =
\frac{e^{iy}}{2^j}\left[\frac{\gamma(j+1,2iy)}{(iy)^{j+1}}\right],
\end{equation}
where $\gamma(a, z)$ is the lower incomplete gamma function defined by $\gamma(a, z) := \int_0^z t^{a-1} e^{-t} dt$.
\end{lemma}

\begin{proof}
Rewrite the integral using the substitution \(u = 1 - x\), so \(x = 1 - u\) and \(dx = -du\). The limits of integration change from $-1$ to $1$ to $2$ to $0$:
\[
\int_{-1}^1 \left(\frac{1}{2} - \frac{x}{2}\right)^j e^{i x y} \, dx = \frac{1}{2^j} \int_{-1}^1 (1-x)^j e^{i x y} \, dx = \frac{e^{iy}}{2^j} \int_{0}^{2} u^j e^{-i u y} \, du.
\]
Now we evaluate the integral
\[
\int_0^2 u^j e^{-i u y} \, du.
\]
The incomplete gamma function is defined as:
\[
\gamma(a, x) = \int_0^x t^{a-1} e^{-t} \, dt.
\]
Let \(t = iuy\), so \(u = \frac{t}{iy}\) and \(du = \frac{dt}{iy}\). Then
\[
\int_0^2 u^j e^{-i u y} \, du = \int_0^{2iy} \left(\frac{t}{iy}\right)^j e^{-t} \frac{dt}{iy} = \frac{1}{(iy)^{j+1}} \int_0^{2iy} t^j e^{-t} \, dt = \frac{\gamma(j+1, 2iy)}{(iy)^{j+1}}.
\]
Combining these results:
\[
\int_{-1}^1 \left(\frac{1}{2} - \frac{x}{2}\right)^j e^{i x y} \, dx = \frac{e^{iy}}{2^j} \frac{\gamma(j+1, 2iy)}{(iy)^{j+1}}.
\]
\end{proof}

\begin{theorem}\label{thm:EvaluationOfIntegral}
Let $m, n \in \mathbb{Z}_{\geq 0}$ be non-negative integers. Define the integral
\[
I_{m,n}(y) = \int_{-1}^1 {}_2F_1\left(-m, m+1; 1; \frac{1}{2}-\frac{x}{2}\right)
\, {}_2F_1\left(-n, n+1; 1; \frac{1}{2}-\frac{x}{2}\right) e^{i x y}\, dx.
\]
Then,
\[
I_{m,n}(y) = e^{iy}\sum_{j=0}^{m+n}\frac{\Psi_j(m,n)}{2^j}\left[\frac{\gamma(j+1,2iy)}{(iy)^{j+1}}\right],
\]
where $\Psi_j(m,n)$ is given by:
\begin{equation}\label{eq:PsiEquality}
\Psi_j(m,n) = \sum_{k=\max(0,j-n)}^{\min(j,m)}\frac{(-m)_k(m+1)_k}{k!}\frac{(-n)_{j-k}(n+1)_{j-k}}{(j-k)!} = \frac{1}{j!}\, {}_4F_3\left(
\begin{matrix}
-m,\, m+1,\, -n,\, n+1 \\
1,\, 1,\, j+1
\end{matrix};\, 1\right).
\end{equation}
\end{theorem}

\begin{proof}
We start with the integral
\[
I_{m,n}(y) = \int_{-1}^1 {}_2F_1\left(-m, m+1; 1; \frac{1}{2}-\frac{x}{2}\right)
\, {}_2F_1\left(-n, n+1; 1; \frac{1}{2}-\frac{x}{2}\right) e^{i x y}\, dx.
\]

\textbf{1. Evaluation of the Integral:}\\[1mm]
By expanding the hypergeometric functions using Lemma \ref{lem:HyperExpansions}:
\[
{}_2F_1(-m, m+1; 1; z) = \sum_{k=0}^m \frac{(-m)_k (m+1)_k}{(1)_k k!} z^k,
\]
and similarly for the second hypergeometric function, the integral becomes:
\[
I_{m,n}(y) = \sum_{k=0}^m \sum_{\ell=0}^n \frac{(-m)_k (m+1)_k}{k!} \frac{(-n)_\ell (n+1)_\ell}{\ell!} \frac{1}{2^{k+\ell}} \int_{-1}^1 (1-x)^{k+\ell} e^{i x y} dx.
\]
Change variables $j = k + \ell$ and regroup terms:
\[
I_{m,n}(y) = \sum_{j=0}^{m+n} \sum_{k=\max(0,j-n)}^{\min(j,m)} \frac{(-m)_k (m+1)_k (-n)_{j-k} (n+1)_{j-k}}{k! (j-k)!} \frac{1}{2^j} \int_{-1}^1 (1-x)^j e^{i x y} dx.
\]
Using Lemma \ref{lem:IntegralGamma}, we recognize the integral evaluates to:
\[
\frac{e^{iy}}{2^j} \frac{\gamma(j+1, 2iy)}{(iy)^{j+1}},
\]
yielding:
\[
I_{m,n}(y) = e^{iy} \sum_{j=0}^{m+n} \frac{\Psi_j(m,n)}{2^j} \frac{\gamma(j+1,2iy)}{(iy)^{j+1}},
\]
where $\Psi_j(m,n)$ is as defined in \eqref{eq:PsiEquality}.

\textbf{2. Equivalence of $\Psi_j(m,n)$ Representations:}\\[1mm]
We now rigorously prove:
\[
\sum_{k=\max(0,j-n)}^{\min(j,m)} \frac{(-m)_k (m+1)_k}{k!} \frac{(-n)_{j-k} (n+1)_{j-k}}{(j-k)!} = \frac{1}{j!} {\,}_4F_3\left(
\begin{matrix}
-m,\, m+1,\, -n,\, n+1 \\
1,\, 1,\, j+1
\end{matrix};\, 1\right).
\]

\textbf{Step 1: Pochhammer Symbol Factorization}\\[1mm]
Observe that:
\[
\frac{(-m)_k (m+1)_k (-n)_{j-k} (n+1)_{j-k}}{k! (j-k)!} = \frac{(-m)_k (m+1)_k (-n)_{j-k} (n+1)_{j-k}}{j!} \binom{j}{k}.
\]
Using the connection between Pochhammer symbols and factorials:
\[
(a)_m = \frac{\Gamma(a + m)}{\Gamma(a)},
\]
we express:
\[
\frac{(-m)_k}{k!} = \frac{(-1)^k \Gamma(m + 1)}{\Gamma(m - k + 1) k!}, \quad
\frac{(-n)_{j-k}}{(j - k)!} = \frac{(-1)^{j - k} \Gamma(n + 1)}{\Gamma(n - j + k + 1) (j - k)!}.
\]
Substituting these into the sum:
\[
\Psi_j(m,n) = \frac{(-1)^{j}}{j!} \sum_{k=\max(0,j-n)}^{\min(j,m)} \frac{\Gamma(m + 1)\Gamma(n + 1)}{\Gamma(m - k + 1)\Gamma(n - j + k + 1)} (m+1)_k (n+1)_{j - k} \binom{j}{k}.
\]

\textbf{Step 3: Simplification via Gamma Function Properties}\\[1mm]
Note that $\Gamma(m - k + 1) = (m - k)!$ and $\Gamma(n - j + k + 1) = (n - j + k)!$. Substituting these:
\[
\Psi_j(m,n) = \frac{(-1)^j}{j!} \sum_{k} \frac{(m+1)_k (n+1)_{j-k}}{(m - k)! (n - j + k)!} \binom{j}{k}.
\]
Recognizing that $(m+1)_k = \frac{(m+1)!}{(m - k + 1)!}$ and $(n+1)_{j-k} = \frac{(n+1)!}{(n - j + k + 1)!}$, we rewrite:
\[
\Psi_j(m,n) = \frac{(-1)^j (m+1)! (n+1)!}{j!} \sum_{k} \frac{1}{(m - k + 1)! (n - j + k + 1)!} \binom{j}{k}.
\]

\textbf{Step 4: Term Cancellation}\\[1mm]
Observe that $\binom{j}{k} = \frac{j!}{k! (j - k)!}$. Substituting this:
\[
\Psi_j(m,n) = \frac{(-1)^j (m+1)! (n+1)!}{j!} \sum_{k} \frac{1}{(m - k + 1)! (n - j + k + 1)!} \frac{1}{k! (j - k)!}.
\]
Rearranging denominators:
\[
= \frac{(-1)^j (m+1)! (n+1)!}{j!} \sum_{k} \frac{1}{k! (j - k)! (m - k + 1)! (n - j + k + 1)!}.
\]

\textbf{Step 5: Hypergeometric Identification}\\[1mm]
Expand the factorials into Pochhammer symbols using:
\[
\frac{1}{(m - k + 1)!} = \frac{(-1)^{m - k + 1}}{\Gamma(-m + k - 0)},
\]
but instead, directly compare to the definition of ${}_4F_3$:
\[
{}_4F_3\left(\begin{matrix} -m, m+1, -n, n+1 \\ 1, 1, j+1 \end{matrix}; 1\right) = \sum_{k=0}^\infty \frac{(-m)_k (m+1)_k (-n)_{j-k} (n+1)_{j-k}}{(1)_k (1)_{j-k} (j+1)_{j-k}} \frac{1^k}{k!}.
\]
Since $(1)_k = k!$ and $(1)_{j-k} = (j - k)!$, while $(j+1)_{j-k} = \frac{(j+1)!}{(k + 1)!}$, the term matches the summand of $\Psi_j(m,n)$ multiplied by $j!$. Thus:
\[
\Psi_j(m,n) = \frac{1}{j!} {}_4F_3\left(\begin{matrix} -m, m+1, -n, n+1 \\ 1, 1, j+1 \end{matrix}; 1\right).
\]

\textbf{Step 6: Termination of the Series}\\[1mm]
The hypergeometric series terminates because:
\begin{itemize}
\item $(-m)_k = 0$ when $k > m$,
\item $(-n)_{j - k} = 0$ when $j - k > n$,
\end{itemize}
which enforces the summation limits $k \in [\max(0, j - n), \min(j, m)]$. This matches the finite sum definition of $\Psi_j(m,n)$.

\textbf{Conclusion:}\\[1mm]
Both forms of $\Psi_j(m,n)$ are equivalent, and the integral $I_{m,n}(y)$ evaluates universally to:
\[
I_{m,n}(y) = e^{iy}\sum_{j=0}^{m+n}\frac{\Psi_j(m,n)}{2^j}\left[\frac{\gamma(j+1,2iy)}{(iy)^{j+1}}\right].
\]
\end{proof}

\end{document}


