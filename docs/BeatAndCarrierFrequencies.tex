\documentclass[12pt]{article}
\usepackage{amsmath}
\usepackage{amssymb}
\usepackage{amsthm}
\usepackage{geometry}
\geometry{margin=1in}

\newtheorem{theorem}{Theorem}
\newtheorem{lemma}{Lemma}
\newtheorem{corollary}{Corollary}
\newtheorem{definition}{Definition}

\title{Spectral Relations and Beat Frequency Analysis\\
       in Complex Fourier Transforms}
\author{Analysis of Formulas 7.5.10 and 7.5.11}
\date{}

\begin{document}

\maketitle

\section{Fundamental Definitions}

\begin{definition}[Beat Frequency]
When two waves of slightly different frequencies $f_1$ and $f_2$ interfere, they produce a periodic variation in amplitude known as beats. The beat frequency is defined as:
\begin{equation}
f_{\text{beat}} = |f_2 - f_1|
\end{equation}
This represents the number of amplitude modulations (beats) per unit time observed in the interference pattern.
\end{definition}

\begin{definition}[Carrier Frequency]
In modulated wave systems, the carrier frequency is the base frequency of the unmodulated wave that serves as the medium for transmitting information. For two interfering waves with frequencies $f_1$ and $f_2$, the carrier frequency is defined as:
\begin{equation}
f_{\text{carrier}} = \frac{f_1 + f_2}{2}
\end{equation}
This represents the average frequency around which the beat modulation occurs.
\end{definition}

\section{Original Spectral Relations}

From the theory of Fourier transforms for real-valued functions, we have the following fundamental relations:

\begin{align}
F(\lambda_2) - F(\lambda_1) &= \frac{1}{2\pi} \lim_{T \to \infty} \int_{-T}^{T} \frac{e^{-i\lambda_2 t} - e^{-i\lambda_1 t}}{-it} \tau(t) \, dt \tag{7.5.10}\\
\xi(\lambda_2) - \xi(\lambda_1) &= \frac{1}{2\pi} \lim_{T \to \infty} \int_{-T}^{T} \frac{e^{-i\lambda_2 t} - e^{-i\lambda_1 t}}{-it} \xi(t) \, dt \tag{7.5.11}
\end{align}

These relations differ only by the functions being transformed: $\tau(t)$ in (7.5.10) and $\xi(t)$ in (7.5.11).

\section{Trigonometric Expansions}

Using Euler's formula $e^{-i\lambda t} = \cos(\lambda t) - i\sin(\lambda t)$, we can expand the complex exponentials.

\begin{lemma}[Trigonometric Form of Complex Exponential Difference]
The difference of complex exponentials can be written as:
\begin{equation}
e^{-i\lambda_2 t} - e^{-i\lambda_1 t} = [\cos(\lambda_2 t) - \cos(\lambda_1 t)] - i[\sin(\lambda_2 t) - \sin(\lambda_1 t)]
\end{equation}
\end{lemma}

\begin{proof}
Direct application of Euler's formula:
\begin{align}
e^{-i\lambda_2 t} - e^{-i\lambda_1 t} &= [\cos(\lambda_2 t) - i\sin(\lambda_2 t)] - [\cos(\lambda_1 t) - i\sin(\lambda_1 t)]\\
&= [\cos(\lambda_2 t) - \cos(\lambda_1 t)] - i[\sin(\lambda_2 t) - \sin(\lambda_1 t)]
\end{align}
\end{proof}

\section{Sum-to-Product Transformations}

The trigonometric differences can be simplified using sum-to-product identities.

\begin{lemma}[Sum-to-Product Identities]
For any real numbers $A$ and $B$:
\begin{align}
\cos A - \cos B &= -2\sin\left(\frac{A+B}{2}\right)\sin\left(\frac{A-B}{2}\right)\\
\sin A - \sin B &= 2\cos\left(\frac{A+B}{2}\right)\sin\left(\frac{A-B}{2}\right)
\end{align}
\end{lemma}

\begin{proof}
Using the angle addition formulas:
\begin{align}
\cos A &= \cos\left(\frac{A+B}{2} + \frac{A-B}{2}\right) = \cos\left(\frac{A+B}{2}\right)\cos\left(\frac{A-B}{2}\right) - \sin\left(\frac{A+B}{2}\right)\sin\left(\frac{A-B}{2}\right)\\
\cos B &= \cos\left(\frac{A+B}{2} - \frac{A-B}{2}\right) = \cos\left(\frac{A+B}{2}\right)\cos\left(\frac{A-B}{2}\right) + \sin\left(\frac{A+B}{2}\right)\sin\left(\frac{A-B}{2}\right)
\end{align}
Subtracting: $\cos A - \cos B = -2\sin\left(\frac{A+B}{2}\right)\sin\left(\frac{A-B}{2}\right)$.

Similarly for sine:
\begin{align}
\sin A &= \sin\left(\frac{A+B}{2} + \frac{A-B}{2}\right) = \sin\left(\frac{A+B}{2}\right)\cos\left(\frac{A-B}{2}\right) + \cos\left(\frac{A+B}{2}\right)\sin\left(\frac{A-B}{2}\right)\\
\sin B &= \sin\left(\frac{A+B}{2} - \frac{A-B}{2}\right) = \sin\left(\frac{A+B}{2}\right)\cos\left(\frac{A-B}{2}\right) - \cos\left(\frac{A+B}{2}\right)\sin\left(\frac{A-B}{2}\right)
\end{align}
Subtracting: $\sin A - \sin B = 2\cos\left(\frac{A+B}{2}\right)\sin\left(\frac{A-B}{2}\right)$.
\end{proof}

\section{Beat Frequency and Carrier Frequency Analysis}

\begin{theorem}[Beat Frequency Decomposition]
The complex exponential difference can be expressed in terms of beat and carrier frequencies:
\begin{equation}
e^{-i\lambda_2 t} - e^{-i\lambda_1 t} = -2i \sin\left(\frac{(\lambda_2 - \lambda_1)t}{2}\right) e^{-i\frac{(\lambda_2 + \lambda_1)t}{2}}
\end{equation}
where:
\begin{itemize}
\item The beat angular frequency is $\omega_{\text{beat}} = \frac{\lambda_2 - \lambda_1}{2}$ (corresponding to beat frequency $f_{\text{beat}} = \frac{|\lambda_2 - \lambda_1|}{4\pi}$)
\item The carrier angular frequency is $\omega_{\text{carrier}} = \frac{\lambda_2 + \lambda_1}{2}$ (corresponding to carrier frequency $f_{\text{carrier}} = \frac{\lambda_2 + \lambda_1}{4\pi}$)
\end{itemize}
\end{theorem}

\begin{proof}
Applying the sum-to-product identities to the trigonometric form:
\begin{align}
\cos(\lambda_2 t) - \cos(\lambda_1 t) &= -2\sin\left(\frac{(\lambda_2 + \lambda_1)t}{2}\right)\sin\left(\frac{(\lambda_2 - \lambda_1)t}{2}\right)\\
\sin(\lambda_2 t) - \sin(\lambda_1 t) &= 2\cos\left(\frac{(\lambda_2 + \lambda_1)t}{2}\right)\sin\left(\frac{(\lambda_2 - \lambda_1)t}{2}\right)
\end{align}

Therefore:
\begin{align}
&[\cos(\lambda_2 t) - \cos(\lambda_1 t)] - i[\sin(\lambda_2 t) - \sin(\lambda_1 t)]\\
&= -2\sin\left(\frac{(\lambda_2 + \lambda_1)t}{2}\right)\sin\left(\frac{(\lambda_2 - \lambda_1)t}{2}\right) - i \cdot 2\cos\left(\frac{(\lambda_2 + \lambda_1)t}{2}\right)\sin\left(\frac{(\lambda_2 - \lambda_1)t}{2}\right)\\
&= -2\sin\left(\frac{(\lambda_2 - \lambda_1)t}{2}\right)\left[\sin\left(\frac{(\lambda_2 + \lambda_1)t}{2}\right) + i\cos\left(\frac{(\lambda_2 + \lambda_1)t}{2}\right)\right]
\end{align}

Using the identity $\sin\theta + i\cos\theta = i(\cos\theta - i\sin\theta) = ie^{-i\theta}$:
\begin{equation}
\sin\left(\frac{(\lambda_2 + \lambda_1)t}{2}\right) + i\cos\left(\frac{(\lambda_2 + \lambda_1)t}{2}\right) = ie^{-i\frac{(\lambda_2 + \lambda_1)t}{2}}
\end{equation}

Therefore:
\begin{equation}
e^{-i\lambda_2 t} - e^{-i\lambda_1 t} = -2i \sin\left(\frac{(\lambda_2 - \lambda_1)t}{2}\right) e^{-i\frac{(\lambda_2 + \lambda_1)t}{2}}
\end{equation}

The beat frequency arises from the $\sin\left(\frac{(\lambda_2 - \lambda_1)t}{2}\right)$ term, which modulates the amplitude with angular frequency $\frac{|\lambda_2 - \lambda_1|}{2}$, corresponding to beat frequency $f_{\text{beat}} = \frac{|\lambda_2 - \lambda_1|}{4\pi}$.

The carrier frequency comes from the $e^{-i\frac{(\lambda_2 + \lambda_1)t}{2}}$ term, which oscillates with angular frequency $\frac{\lambda_2 + \lambda_1}{2}$, corresponding to carrier frequency $f_{\text{carrier}} = \frac{\lambda_2 + \lambda_1}{4\pi}$.
\end{proof}

\begin{corollary}[Beat Frequency Interpretation of Spectral Relations]
The spectral relations (7.5.10) and (7.5.11) can be rewritten as:
\begin{align}
F(\lambda_2) - F(\lambda_1) &= \frac{-i}{\pi} \lim_{T \to \infty} \int_{-T}^{T} \frac{\sin\left(\frac{(\lambda_2 - \lambda_1)t}{2}\right)}{t} e^{-i\frac{(\lambda_2 + \lambda_1)t}{2}} \tau(t) \, dt\\
\xi(\lambda_2) - \xi(\lambda_1) &= \frac{-i}{\pi} \lim_{T \to \infty} \int_{-T}^{T} \frac{\sin\left(\frac{(\lambda_2 - \lambda_1)t}{2}\right)}{t} e^{-i\frac{(\lambda_2 + \lambda_1)t}{2}} \xi(t) \, dt
\end{align}
This form explicitly shows how the spectral difference depends on:
\begin{itemize}
\item The beat envelope function $\frac{\sin\left(\frac{(\lambda_2 - \lambda_1)t}{2}\right)}{t}$
\item The carrier oscillation $e^{-i\frac{(\lambda_2 + \lambda_1)t}{2}}$
\end{itemize}
\end{corollary}

\begin{proof}
Substituting the beat frequency decomposition into the original formulas:
\begin{align}
F(\lambda_2) - F(\lambda_1) &= \frac{1}{2\pi} \lim_{T \to \infty} \int_{-T}^{T} \frac{-2i \sin\left(\frac{(\lambda_2 - \lambda_1)t}{2}\right) e^{-i\frac{(\lambda_2 + \lambda_1)t}{2}}}{-it} \tau(t) \, dt\\
&= \frac{1}{2\pi} \lim_{T \to \infty} \int_{-T}^{T} \frac{2 \sin\left(\frac{(\lambda_2 - \lambda_1)t}{2}\right) e^{-i\frac{(\lambda_2 + \lambda_1)t}{2}}}{t} \tau(t) \, dt \cdot (-i)\\
&= \frac{-i}{\pi} \lim_{T \to \infty} \int_{-T}^{T} \frac{\sin\left(\frac{(\lambda_2 - \lambda_1)t}{2}\right)}{t} e^{-i\frac{(\lambda_2 + \lambda_1)t}{2}} \tau(t) \, dt
\end{align}
The same derivation applies to (7.5.11).
\end{proof}

\section{Physical Interpretation}

The beat frequency decomposition reveals crucial physical insights about the spectral relations:

\begin{itemize}
\item \textbf{Beat Envelope}: The factor $\sin\left(\frac{(\lambda_2 - \lambda_1)t}{2}\right)$ creates an amplitude modulation envelope with beat frequency $f_{\text{beat}} = \frac{|\lambda_2 - \lambda_1|}{4\pi}$. This envelope determines how rapidly the interference pattern oscillates between constructive and destructive interference.

\item \textbf{Carrier Wave}: The factor $e^{-i\frac{(\lambda_2 + \lambda_1)t}{2}}$ represents the carrier wave oscillating at the average frequency $f_{\text{carrier}} = \frac{\lambda_2 + \lambda_1}{4\pi}$. This carrier provides the fundamental oscillation that is modulated by the beat envelope.

\item \textbf{Spectral Resolution}: The $\frac{\sin\left(\frac{(\lambda_2 - \lambda_1)t}{2}\right)}{t}$ term in the integral acts as a frequency resolution kernel. As $|\lambda_2 - \lambda_1| \to 0$, this kernel approaches a delta function, providing perfect frequency resolution.

\item \textbf{Time-Frequency Uncertainty}: The beat structure demonstrates the fundamental time-frequency uncertainty principle in Fourier analysis - better frequency resolution (smaller $|\lambda_2 - \lambda_1|$) requires longer integration times $T$.
\end{itemize}

This decomposition is particularly valuable in signal processing, spectroscopy, and quantum mechanics where understanding the interference between close frequencies is essential for proper interpretation of measured spectra.

\end{document}
