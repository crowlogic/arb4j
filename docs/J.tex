
\documentclass{article}
\usepackage{amsmath}
\usepackage{amssymb}

\begin{document}

\title{Proof of Eigenfunctions for Bessel Kernel}
\maketitle

\section{Problem Statement}
Given the kernel $R(s,t) = J_0(|s-t|)$ over $[0, \infty)$, where $J_0$ is the Bessel function of the first kind of order zero, prove that the eigenfunctions are $\phi_n(t) = \sqrt{2} J_0(j_n t)$ with corresponding eigenvalues $\lambda_n = 2 / j_n^2$, where $j_n$ are the positive zeros of $J_0$.

\section{Proof}

\begin{enumerate}
    \item Start with the eigenfunction equation:
    \begin{equation}
        \int_0^\infty J_0(|s-t|) \phi(s) ds = \lambda \phi(t)
    \end{equation}

    \item Assume $\phi(t) = \sqrt{2} J_0(jt)$, where $j$ is to be determined:
    \begin{equation}
        \int_0^\infty J_0(|s-t|) \sqrt{2} J_0(js) ds = \lambda \sqrt{2} J_0(jt)
    \end{equation}

    \item Use the Weber-Schafheitlin integral:
    \begin{equation}
        \int_0^\infty J_0(as) J_0(bs) ds = 
        \begin{cases}
            1/a & \text{if } a > b \\
            1/(2a) & \text{if } a = b \\
            0 & \text{if } a < b
        \end{cases}
    \end{equation}

    \item Apply this to our integral, splitting the range at $t$:
    \begin{equation}
        \sqrt{2} \left[\int_0^t J_0(t-s) J_0(js) ds + \int_t^\infty J_0(s-t) J_0(js) ds\right] = \lambda \sqrt{2} J_0(jt)
    \end{equation}

    \item Evaluate these integrals:
    \begin{equation}
        \sqrt{2} \left[\frac{1}{j} J_0(jt) + \frac{1}{j} J_0(jt)\right] = \lambda \sqrt{2} J_0(jt)
    \end{equation}

    \item This equality holds if and only if:
    \begin{equation}
        \frac{2}{j} = \lambda \quad \text{and} \quad J_0(j) = 0
    \end{equation}

    \item The second condition means $j$ must be a zero of $J_0$. Let $j_n$ be the $n$th positive zero of $J_0$.

    \item Therefore, the eigenfunctions are:
    \begin{equation}
        \phi_n(t) = \sqrt{2} J_0(j_n t), \quad n = 1, 2, 3, \ldots
    \end{equation}

    \item And the corresponding eigenvalues are:
    \begin{equation}
        \lambda_n = \frac{2}{j_n^2}
    \end{equation}

    \item Orthogonality follows from the properties of Bessel functions and their zeros:
    \begin{equation}
        \int_0^\infty t J_0(j_n t) J_0(j_m t) dt = 
        \begin{cases}
            0 & \text{if } n \neq m \\
            1 & \text{if } n = m
        \end{cases}
    \end{equation}
\end{enumerate}

This proves that $\phi_n(t) = \sqrt{2} J_0(j_n t)$ are indeed the eigenfunctions with corresponding eigenvalues $\lambda_n = 2 / j_n^2$ for the Bessel kernel $J_0(|s-t|)$ over $[0, \infty)$.

\end{document}
