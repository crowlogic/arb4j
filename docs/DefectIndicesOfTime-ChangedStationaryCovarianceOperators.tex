\documentclass[12pt]{article}
\usepackage{amsmath,amsthm,amssymb,mathrsfs,bm}

\title{An Operator-Theoretic Formulation of the Invertibility Theorem for Oscillatory Gaussian Processes}
\author{Stephen Crowley}
\date{}

\theoremstyle{plain}
\newtheorem{theorem}{Theorem}
\newtheorem{lemma}{Lemma}
\newtheorem{definition}{Definition}

\begin{document}

\maketitle

\section{Abstract}
In the Hilbert space of square-integrable sample paths, oscillatory processes of Priestley's type can be expressed through a deterministic amplitude kernel and an orthogonal increment spectral measure. The inversion theorem for such processes can be reformulated in operator-theoretic terms: recovery of the spectral measure is equivalent to the unitarity of a synthesis map defined by the amplitude function. This formulation makes the necessity and sufficiency of the orthonormality condition immediate.

\section{Hilbert Space Setting}
\begin{definition}[Oscillatory process]
Let $(\Omega,\mathcal{F},\mathbb{P})$ be a probability space and $\mu$ a finite positive Borel measure on $\mathbb{R}$. An \emph{oscillatory Gaussian process} is a mapping
\[
X:\mathbb{R} \times \Omega \to \mathbb{C}, \quad
X(t) = \int_{\mathbb{R}} A(t,\lambda) e^{i\lambda t} \, dZ(\lambda),
\]
where $A:\mathbb{R}^2 \to \mathbb{C}$ is measurable and $dZ(\lambda)$ is a complex Gaussian orthogonal increment process satisfying
\[
\mathbb{E}\bigl[dZ(\lambda_1) \overline{dZ(\lambda_2)}\bigr] = \delta(\lambda_1 - \lambda_2)\,\mu(d\lambda_1).
\]
\end{definition}

\section{Operator Formulation}
Let $L^2(\mu)$ denote the complex Hilbert space of square-integrable functions with respect to $\mu$, and let $L^2(\mathbb{R},dt)$ be the space of complex square-integrable functions on $\mathbb{R}$ with respect to Lebesgue measure $t$.

\subsection{Synthesis Operator}
Define the \emph{synthesis operator} $S_A : L^2(\mu) \to L^2(\mathbb{R},dt)$ by
\[
(S_A g)(t) = \int_{\mathbb{R}} A(t,\lambda) e^{i\lambda t} g(\lambda) \,\mu^{1/2}(d\lambda),
\]
where $\mu^{1/2}(d\lambda)$ denotes integration against $\mu$ with square-root density viewed as measure factor.

\subsection{Analysis Operator}
The formal $L^2$-adjoint $S_A^* : L^2(\mathbb{R},dt) \to L^2(\mu)$ is given by
\[
(S_A^* f)(\lambda) = \frac{1}{2\pi} \int_{\mathbb{R}} \overline{A(t,\lambda)} e^{-i\lambda t} f(t)\, dt.
\]
This operator is exactly the inversion operator $\mathcal{I}$ from the original theorem.

\section{Invertibility Theorem in Operator Terms}
\begin{theorem}
Let $S_A$ be as above, with adjoint $S_A^*$. The following are equivalent:
\begin{enumerate}
\item $S_A$ is a unitary isomorphism from $L^2(\mu)$ onto its range in $L^2(\mathbb{R},dt)$.
\item The inversion $\mathcal{I} = S_A^*$ satisfies $\mathcal{I}[X] = dZ$ for every oscillatory Gaussian process with amplitude $A$.
\item The kernel family $\{t \mapsto A(t,\lambda)e^{i\lambda t}:\lambda \in \mathbb{R}\}$ is orthonormal:
\[
\frac{1}{2\pi} \int_{\mathbb{R}} \overline{A(t,\lambda_1)} A(t,\lambda_2) e^{i(\lambda_2 - \lambda_1)t}\, dt
= \delta(\lambda_1 - \lambda_2).
\]
\end{enumerate}
In this situation, $S_A$ and $S_A^*$ are mutual inverses and the inversion operator is unique.
\end{theorem}

\begin{proof}
$(1) \Rightarrow (2)$: If $S_A$ is unitary, then $S_A^* S_A = I_{L^2(\mu)}$. For $g(\lambda) = dZ(\lambda)$, $X = S_A g$ yields $S_A^* X = g = dZ$.

$(2) \Rightarrow (3)$: Fix $\lambda_0$ and test $\mathcal{I}$ on $X(t) = A(t,\lambda) e^{i\lambda t}$; the computation forces the inner product relation of (3).

$(3) \Rightarrow (1)$: The relation in (3) shows that the collection $\{\lambda \mapsto A(\cdot,\lambda)e^{i\lambda \cdot}\}$ is orthonormal in $L^2(\mathbb{R},dt)$, making $S_A$ an isometry onto its range. Nonvanishing of $A$ ensures surjectivity onto that range, completing unitarity.
\end{proof}

\section{Uniqueness}
\begin{lemma}
If $L_1$ and $L_2$ are linear operators from the range of $S_A$ to $L^2(\mu)$ satisfying $L_j[X] = dZ$ for all Gaussian oscillatory $X$ with amplitude $A$, then $L_1 = L_2$.
\end{lemma}

\begin{proof}
Let $L = L_1 - L_2$. Since $L[X] = 0$ for all $X$ in the range of $S_A$, and the set $\{A(\cdot,\lambda)e^{i\lambda \cdot}\}$ spans a dense subset of that range, $L = 0$ by continuity.
\end{proof}

\section{Concluding Remarks}
In operator language, the original kernel orthonormality condition corresponds exactly to the requirement that the synthesis operator be an isometry, while the nonvanishing condition guarantees no loss of spectral components. The inversion operator is then simply the Hilbert space adjoint, unique by standard functional analytic arguments.
\end{document}
