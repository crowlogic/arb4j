\documentclass[11pt]{article}
\usepackage{amsmath,amsthm,amssymb,amsfonts}
\usepackage{enumitem}
\usepackage{hyperref}

\title{Testing for Harmonizability}
\author{H. L. Hurd}
\date{IEEE Transactions on Information Theory, Vol. IT-19, No. 3, May 1973}

\theoremstyle{plain}
\newtheorem{theorem}{Theorem}[section]
\newtheorem{lemma}[theorem]{Lemma}
\theoremstyle{definition}
\newtheorem{definition}[theorem]{Definition}
\newtheorem{example}[theorem]{Example}
\theoremstyle{remark}
\newtheorem*{remark}{Remark}

\begin{document}

\maketitle

\begin{abstract}
Let $R(s,t)$ be a covariance function having the representation
\begin{equation}
R(s,t) = \int_{-\infty}^{\infty} \int_{-\infty}^{\infty} \exp(isx - ity) \, d^2G(x,y)
\label{eq:covariance-harmonizable}
\end{equation}
where $G(x,y)$ is continuous to the right in both variables and is of bounded variation in the plane; then $R(s,t)$ is harmonizable in that $G(x,y)$ is also a covariance. Examples are given in which this result is used to determine the harmonizability of new processes and covariances that are formed by operations on old processes and covariances. Specifically, if $X(t)$ is a real Gaussian harmonizable process, then $X^n(t)$ is harmonizable. If $X(t)$ is harmonizable, $G(x,y)$ has bounded support and $g(t)$ is a Fourier-Stieltjes transform, then $X(g(t))$ and $X(t + g(t))$ are harmonizable. If
\begin{equation}
X(t) = \int_{-\infty}^{\infty} f(t,u) \, dZ(u)
\label{eq:moving-average}
\end{equation}
where $f(t,u) = f(t-u)$ is a Fourier-Stieltjes transform and $G(u,v) = \mathbb{E}\{Z(u)Z^*(v)\}$ has finite total variation, then $X(t)$ is harmonizable. A sufficient condition for Priestley's oscillatory processes to be harmonizable is also obtained. The Bochner-Eberlein characterization of Fourier-Stieltjes transforms is particularly convenient for determining harmonizability in these cases.
\end{abstract}

\section{Introduction}

Let $\{X(t,\omega), -\infty < t < \infty, \omega \in \Omega\}$ be a second-order continuous-parameter stochastic process defined on some probability space $(\Omega, \mathcal{F}, \mathbb{P})$. The process $X(t,\omega)$ is said to be harmonizable~\cite[p.~474]{loeve1955probability} if it has the quadratic mean representation
\begin{equation}
X(t,\omega) = \int_{-\infty}^{\infty} \exp(itx) \, dZ(x,\omega)
\label{eq:process-harmonizable}
\end{equation}
where $\{Z(x,\omega), -\infty < x < \infty\}$ is a process whose covariance is of bounded variation (BV) in the plane. Harmonizable processes are of engineering interest because decomposition relative to $\exp(itx)$ admits the usual frequency interpretation of linear filtering. If $H(x)$ is the frequency response of a stable, linear time-invariant system, then the system output process $Y(t,\omega)$ is given by the quadratic mean integral
\begin{equation}
Y(t,\omega) = \int_{-\infty}^{\infty} \exp(itx) H(x) \, dZ(x,\omega)
\label{eq:lti-filter}
\end{equation}
A detailed account may be found in~\cite[Ch.~8]{blanc-lapierre1968random}. For recent results on harmonizable processes in engineering, see~\cite{cambanis1970harmonizable,donati1971spectra,ogura1971spectral}.

The covariance functions for $X(t)$ and $Y(t)$ are
\begin{align}
R(s,t) &= \int_{-\infty}^{\infty} \int_{-\infty}^{\infty} \exp(isx - ity)\, d^2G(x,y) \label{eq:cov-fn-harm} \\
R_Y(s,t) &= \int_{-\infty}^{\infty} \int_{-\infty}^{\infty} \exp(isx - ity) H(x)H^*(y)\, d^2G(x,y) \label{eq:cov-fn-filtered}
\end{align}
where $d^2G(x,y) = \mathbb{E}\{dZ(x) dZ^*(y)\}$. Any covariance with representation~\eqref{eq:cov-fn-harm}, with $G(x,y)$ a covariance of bounded variation, is called harmonizable; harmonizable processes have harmonizable covariances. Conversely, processes with harmonizable covariances are themselves harmonizable~\cite[p.~476]{loeve1955probability}. For brevity, we call those corresponding to $G$ with finite total variation simply ``harmonizable.''

This paper addresses the determination of harmonizability for new processes or covariances constructed from old ones. The main results are as follows:
\begin{itemize}
    \item If $X(t)$ is a real Gaussian harmonizable process, then $X^n(t)$ is harmonizable.
    \item If $X(t)$ is harmonizable with spectral decomposition of bounded support and $g(t)$ is a Fourier-Stieltjes transform, then $X(t+g(t))$ and $X(g(t))$ are harmonizable.
    \item If $R_1$ and $R_2$ are harmonizable covariances, then for $T$ of finite Lebesgue measure,
    \begin{equation}
    R_3(s,t) = \int_{T} R_1(s,u)R_2(u,t) \, du
    \label{eq:integral-composed}
    \end{equation}
    is harmonizable.
    \item If $X(t)$ is a moving average as in~\eqref{eq:moving-average} with $f(t,u) = f(t-u)$ a Fourier-Stieltjes transform and $G(u,v)$ of bounded variation, then $X(t)$ is harmonizable.
    \item If $X(t)$ is as above, $Z(u)$ has orthogonal increments with $dF(u) = \mathbb{E}|dZ(u)|^2$ and
    \begin{equation}
    f(t,u) = \exp(iut) \int_{-\infty}^{\infty} \exp(itx) \, dH_u(x)
    \label{eq:oscillatory-fn}
    \end{equation}
    with $H_u(x)$ of bounded variation for every $u$, the resulting processes include Priestley's oscillatory processes~\cite{priestley1965evolutionary}, which are harmonizable under suitable conditions.
\end{itemize}

The method is to use the following result.

\begin{theorem}
\label{thm:covariance-fs}
If $R(s,t)$ is simultaneously a covariance and a Fourier-Stieltjes (FS) transform with respect to some $G(x,y)$ of bounded variation, then $R(s,t)$ is harmonizable in that $G$ is necessarily a covariance.
\end{theorem}

\begin{proof}
Sufficiency is immediate: one can find a process $Z(x,\omega)$ whose covariance is $G$ and whose FS transform $X(t,\omega)$ as in~\eqref{eq:process-harmonizable} has covariance $R(s,t)$. Conversely, suppose $R$ is both a covariance and an FS transform with respect to some $G$ that is BV. Define
\begin{equation}
G_a(x,y) = G(x,y) - G(a,y) - G(x,a) + G(a, a)
\label{eq:nnd-Ga}
\end{equation}
For any $n$ and sequence $\{x_j \ge a,\, j=1,\ldots,n\}$ and complex $\{c_j\}$,
\[
\sum_{j,k=1}^n c_j c_k^* G_a(x_j, x_k) \ge 0
\]
This follows by constructing $g_a(s) = \sum_{j=1}^n c_j [1 - \exp(-isx_j)] \exp(-isa)$ and applying the inversion theorem~\cite[p.~475]{loeve1955probability}. Letting $a \to -\infty$, $G_a(x,y) \to G(x,y)$. Thus $G(x,y)$ is non-negative definite.
\end{proof}

Thus, any characterization of FS transforms, such as the Bochner-Eberlein theorem, also provides a characterization for harmonizable covariances~\cite{bochner1934fst,eberlein1955fst,rudin1962groups}. Cramér~\cite{cramer1939representation} and Dominguez~\cite{dominguez1940fst} provide alternative characterizations.

\section{Mathematical Preliminaries}

Assume $G(x,y)$ is normalized, e.g.,
\begin{equation}
G(x,y) = \frac{1}{4}\left[G(x+0, y+0) + G(x+0, y-0) + G(x-0, y+0) + G(x-0, y-0)\right]
\label{eq:G-normalization}
\end{equation}
and satisfies 
\[
\lim_{x\to -\infty} G(x, y) = \lim_{y\to -\infty} G(x, y) = 0.
\]
We now state the key characterizations.

\begin{theorem}[Bochner]
\label{thm:bochner}
A necessary and sufficient condition that $f(t)$, $-\infty < t < \infty$, has the representation
\begin{equation}
f(t) = \int_{-\infty}^{\infty} \exp(ixt) \, dG(x)
\label{eq:bochner}
\end{equation}
for a complex measure $G$ of bounded variation is that, for any $n$, any sequence $\{t_j\}_{j=1}^n$ and any complex $\{c_j\}_{j=1}^n$,
\begin{equation}
\left|\sum_{j=1}^n c_j f(t_j)\right| \leq M \left[\sum_{j=1}^n \sum_{k=1}^n c_j c_k^* \exp(ix(t_j-t_k))\right]^{1/2}
\label{eq:bochner-nnd}
\end{equation}
for some $M>0$.
\end{theorem}

\begin{theorem}[Bochner-Eberlein]
\label{thm:bochner-eberlein}
A necessary and sufficient condition for a function $R(s,t)$ to have representation~\eqref{eq:covariance-harmonizable} with
\begin{equation}
\int_{-\infty}^\infty \int_{-\infty}^\infty |d^2G(x,y)| \leq M
\label{eq:bounded-var-G}
\end{equation}
is that for any $n$, any sequence of pairs $\{(s_j,t_j)\}_{j=1}^n$ and any complex $\{c_j\}_{j=1}^n$,
\begin{equation}
\left|\sum_{j=1}^n c_j R(s_j, t_j)\right| \leq M \left[\sum_{j=1}^n \sum_{k=1}^n c_j c_k^* \exp(i(x s_j - y t_j - x s_k + y t_k))\right]^{1/2}
\label{eq:BE-nnd}
\end{equation}
for some $M > 0$.
\end{theorem}

\section{Examples and Application}

\begin{example}
Let $X(t)$ be a zero-mean real Gaussian harmonizable process with covariance $R(s,t)$. Then, for any $n \geq 1$, $X^n(t)$ is harmonizable.
\end{example}

\begin{proof}
An exercise with characteristic functions shows that
\begin{equation}
\mathbb{E}[X^n(s)X^n(t)] = \sum_{p, q, r \geq 0} c_z(p, q, r, n) R^p(s, s) R^q(s, t) R^r(t, s) R^p(t, t)
\label{eq:gaussian-moments}
\end{equation}
where the sum is over all $p, q, r \geq 0$ with $n = 2p + q + r$ and $c_z(p, q, r, n)$ are combinatorial coefficients.

Since both FS transforms~\cite[p.151]{bochner1934fst} and covariances~\cite[p.468]{loeve1955probability} are closed under products, $R^q(s,t) R^r(t,s) = R^{q+r}(s,t)$ is an FS transform and a covariance. The product $R^p(s,s) R^p(t,t)$ is also an FS transform, as for $f(s) = R(s,s)$
\[
f^p(s) f^p(t)
\]
is NND and an FS transform by Theorem~\ref{thm:bochner}. The sum in~\eqref{eq:gaussian-moments} is harmonizable since FS transforms and covariances are closed under positive sums.
\end{proof}

\begin{example}
\label{ex:composition}
Suppose $X(t)$ is harmonizable with spectral decomposition supported by a bounded set $A$ and $g(t)$ is the FS transform of some $G(x)$ of finite variation. Then $X(t+g(t))$ and $X(g(t))$ are harmonizable.
\end{example}

\begin{proof}
Set $Y(t) = X(t + g(t))$, so
\[
R_Y(s, t) = R_X(s + g(s), t + g(t)) = \iint_A \exp\left(ix(s+g(s)) - iy(t+g(t))\right) d^2G(x, y).
\]
Let $M_A$ denote the variation over $A \times A$ of $G(x, y)$. For any complex $\{c_j\}$ and parameter pairs,
\[
Q = \left|\sum_{j=1}^n c_j R_Y(s_j, t_j)\right| \leq M_A \left|\sum_{j=1}^n c_j \exp\left(ix's_j + ix' g(s_j) - iy' t_j - iy' g(t_j)\right)\right|
\]
where $x',y'$ in closure of $A$. The mappings $t \mapsto \exp[ix' g(t)]$ and $t \mapsto \exp[-iy' g(t)]$ are FS transforms, so by repeated application of Bochner's condition this is bounded, and $R_Y(s, t)$ is an FS transform.
\end{proof}

\begin{example}
\label{ex:integral-composed}
Suppose $R_1$ and $R_2$ are harmonizable covariances, and for $T$ of finite Lebesgue measure define $R_3(s,t)$ as in~\eqref{eq:integral-composed}. Then $R_3$ is harmonizable.
\end{example}

\begin{proof}
By the Bochner-Eberlein condition, for any $\{c_j\}$,
\[
Q = \left|\sum_{j=1}^n c_j \int_T R_1(s_j, u) R_2(u, t_j)\, du \right| \leq M_1 \sup_{x, y} \left|\sum_{j=1}^n c_j \int_T R_2(u, t_j) \exp(is_j x - iuy) du \right|.
\]
With $M_2$ the variation bound for $R_2$,
\[
Q \leq M_1 M_2 m(T) \sup_{x, y} \left|\sum_{j=1}^n c_j \exp(is_j x - i t_j y)\right|
\]
where $m(T)$ is Lebesgue measure. Thus $R_3$ is an FS transform.
\end{proof}

\begin{example}
Suppose $X(t)$ is a moving average as in~\eqref{eq:moving-average} with $f(t,u) = f(t-u)$ a Fourier-Stieltjes transform and $G(u,v)$ of bounded variation. Then $X(t)$ is harmonizable.
\end{example}

\begin{proof}
From~\eqref{eq:moving-average} and the bounded variation of $H(x)$ and $G(u,v)$,
\[
R(s,t) = \iiiint \exp[ix(s-u) - iy(t-v)] \, d^2G(u,v) \, dH(x) \, dH^*(y).
\]
This is an FS transform with variation bound $M_G M_H^2$, where $M_G$ bounds $G(u,v)$ and $M_H$ bounds $H(x)$.
\end{proof}

\begin{example}
Suppose $X(t)$ is as above, $Z(u)$ has orthogonal increments with $dF(u) = \mathbb{E}[|dZ(u)|^2]$ and $f(t,u)$ as in~\eqref{eq:oscillatory-fn}. Then $X(t)$ is a Priestley oscillatory process, and is harmonizable provided
\begin{equation}
\int_{-\infty}^{\infty} \int_{-\infty}^{\infty} |dH_u(x)|\,|dH_u^*(y)|\, dF(u) < \infty.
\label{eq:osc-harmonizable}
\end{equation}
\end{example}

\begin{proof}
The covariance is
\[
R(s,t) = \int_{-\infty}^\infty \exp[iu(s-t)] f(u, s) f^*(u, t) dF(u).
\]
By repeated application of Bochner's and Fubini's theorems, provided~\eqref{eq:osc-harmonizable} holds, $R(s,t)$ is an FS transform and thus harmonizable.
\end{proof}

\section*{Acknowledgment}

The author is indebted to H. J. Landau for remarks on an early version of this paper and to a referee for several helpful suggestions.

\begin{thebibliography}{12}
\bibitem{loeve1955probability}
M.~Loève, \emph{Probability Theory}. New York: Van Nostrand, 1955.

\bibitem{blanc-lapierre1968random}
A.~Blanc-Lapierre and R.~Fortet, \emph{Theory of Random Functions}. New York: Gordon and Breach, 1968.

\bibitem{cambanis1970harmonizable}
S.~Cambanis and B.~Liu, ``On harmonizable stochastic processes,'' \emph{Inform. Contr.}, vol.~17, pp.~183--202, 1970.

\bibitem{donati1971spectra}
F.~Donati, ``Finite-time averaged power spectra'', \emph{IEEE Trans. Inform. Theory}, vol. IT-17, pp. 7--16, Jan. 1971.

\bibitem{ogura1971spectral}
H.~Ogura, ``Spectral representation of a periodic nonstationary random process,'' \emph{IEEE Trans. Inform. Theory}, vol. IT-17, pp. 143--149, Mar. 1971.

\bibitem{gladyshev1963almost}
E.~G.~Gladyshev, ``Periodically and almost-periodically correlated random processes with continuous time parameter,'' \emph{Theory Prob. Appl.}, vol.~8, pp.~173--177, 1963.

\bibitem{priestley1965evolutionary}
M.~B.~Priestley, ``Evolutionary spectra and nonstationary processes,'' \emph{J. Roy. Statist. Soc. B}, vol.~27, pp.~204--237, 1965.

\bibitem{bochner1934fst}
S.~Bochner, ``A theorem on Fourier-Stieltjes integrals,'' \emph{Bull. Amer. Math. Soc.}, vol.~40, pp.~271--276, 1934.

\bibitem{eberlein1955fst}
W.~F.~Eberlein, ``Characterization of Fourier-Stieltjes transforms,'' \emph{Duke Math. J.}, vol.~22, pp.~465--468, 1955.

\bibitem{rudin1962groups}
W.~Rudin, \emph{Fourier Analysis on Groups}. New York: Interscience, 1962.

\bibitem{cramer1939representation}
H.~Cramér, ``On the representation of a function by certain Fourier integrals,'' \emph{Trans. Amer. Math. Soc.}, vol.~46, pp.~191--201, 1939.

\bibitem{dominguez1940fst}
A.~G.~Dominguez, ``The representation of functions by Fourier integrals,'' \emph{Duke Math. J.}, vol.~6, pp.~246--255, 1940.
\end{thebibliography}

\end{document}
