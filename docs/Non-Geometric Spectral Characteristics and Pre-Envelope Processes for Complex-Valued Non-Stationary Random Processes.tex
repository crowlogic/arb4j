\documentclass[12pt,a4paper]{article}
\usepackage{amsmath,amssymb,amsthm}
\usepackage{enumitem}
\usepackage{cite}

% Define theorem environments
\newtheorem{theorem}{Theorem}[section]
\newtheorem{lemma}[theorem]{Lemma}
\newtheorem{definition}[theorem]{Definition}
\newtheorem{corollary}[theorem]{Corollary}

% Define proof environment
\renewenvironment{proof}{\noindent\textbf{Proof.}}{\qed}

\title{Non-Geometric Spectral Characteristics and Pre-Envelope Processes for Complex-Valued Non-Stationary Random Processes}
\author{Anonymous}
\date{\today}

\begin{document}

\maketitle

\section{Pre-Envelope Process Theory}

\begin{definition}[Pre-Envelope Process]\label{def:pre_envelope}
For an oscillatory process $X(t)$, the pre-envelope process $Z(t)$ is defined as the complex analytical signal given by
\begin{equation}\label{eq:pre_envelope_def}
Z(t) = X(t) + jY(t)
\end{equation}
where $Y(t)$ is the Hilbert transform of $X(t)$, expressed as
\begin{equation}\label{eq:hilbert_transform}
Y(t) = -j\int_{-\infty}^{\infty} \text{sign}(\omega) A_X(\omega, t) e^{j\omega t} dZ(\omega)
\end{equation}
\end{definition}

\begin{proof}
The pre-envelope formulation provides a complex representation that captures both the instantaneous amplitude and phase information of the oscillatory process. The Hilbert transform in equation~\eqref{eq:hilbert_transform} ensures that the imaginary component $Y(t)$ is orthogonal to the real component $X(t)$, creating the analytic signal representation.
\end{proof}

\section{Spectral Representation}

\begin{theorem}[General Spectral Representation]\label{thm:spectral_rep}
The oscillatory process $X(t)$ has the general spectral representation
\begin{equation}\label{eq:spectral_rep}
X(t) = \int_{-\infty}^{\infty} A_X(\omega, t) e^{j\omega t} dZ(\omega)
\end{equation}
where $A_X(\omega, t)$ is the complex-valued time-frequency modulating function and $dZ(\omega)$ is the orthogonal increment process with $\mathbb{E}[|dZ(\omega)|^2] = d\mu(\omega)$.
\end{theorem}

\begin{proof}
This representation follows from the general theory of stochastic processes and extends the classical spectral decomposition to non-stationary cases. The modulating function $A_X(\omega, t)$ captures the time-varying spectral characteristics, while the orthogonal increment process $dZ(\omega)$ provides the stochastic component with spectral measure $d\mu(\omega)$.
\end{proof}

\section{Non-Geometric Spectral Characteristics}

\begin{definition}[Non-Geometric Spectral Characteristics]\label{def:ngsc}
For complex-valued non-stationary processes, two sets of Non-Geometric Spectral Characteristics (NGSCs) are defined as follows:

\begin{enumerate}[label=(\roman*)]
\item Type 1 - Auto-covariances:
\begin{equation}\label{eq:ngsc_type1}
c_{ik,XX}(t) = \int_{-\infty}^{\infty} \Phi_{X^{(i)}X^{(k)}}(\omega, t) d\omega = \sigma_{X^{(i)}X^{(k)}}(t)
\end{equation}

\item Type 2 - Cross-covariances with Hilbert transform:
\begin{equation}\label{eq:ngsc_type2}
c_{ik,XY}(t) = \int_{-\infty}^{\infty} \Phi_{X^{(i)}Y^{(k)}}(\omega, t) d\omega = \sigma_{X^{(i)}Y^{(k)}}(t)
\end{equation}
\end{enumerate}

where $i, k = 0, 1, 2, \ldots$ and $X^{(i)}(t) = d^i X(t)/dt^i$.
\end{definition}

\begin{proof}
These NGSCs extend the classical spectral moment definitions to non-stationary processes while maintaining physical interpretability. The Type 1 characteristics capture auto-covariance relationships, while Type 2 characteristics incorporate the phase information through the Hilbert transform relationships.
\end{proof}

\section{Evolutionary Cross-Power Spectral Density Functions}

\begin{theorem}[Evolutionary Cross-PSD]\label{thm:cross_psd}
The evolutionary cross-power spectral density functions are given by
\begin{equation}\label{eq:cross_psd}
\Phi_{X^{(i)}W^{(k)}}(\omega, t) = A_{X^{(i)}}^*(\omega, t) \cdot \Phi(\omega) \cdot A_{W^{(k)}}(\omega, t)
\end{equation}
where $W = X, Y$ and the modulating functions for derivatives are computed as
\begin{equation}\label{eq:modulating_derivatives}
A_{W^{(i)}}(\omega, t) = e^{-j\omega t} \frac{\partial^i}{\partial t^i}[A_W(\omega, t) \cdot e^{j\omega t}]
\end{equation}
\end{theorem}

\begin{proof}
The evolutionary cross-PSD captures the time-varying spectral relationships between different derivatives of the processes. Equation~\eqref{eq:modulating_derivatives} provides the derivative relationship through the chain rule applied to the time-frequency modulating functions, accounting for the complex exponential factors.
\end{proof}

\section{Pre-Envelope Covariance Structure}

\begin{lemma}[Pre-Envelope Covariance Matrix]\label{lem:covariance_matrix}
The pre-envelope covariance relationships are expressed through
\begin{equation}\label{eq:covariance_matrix}
\mathbb{E}[S^*(t) S^T(t)] = \text{spectral moment matrix}
\end{equation}
where $S(t)$ represents the complex modal response vector in the pre-envelope formulation.
\end{lemma}

\begin{proof}
This relationship follows from the complex modal decomposition of the system response. The spectral moment matrix captures all second-order statistical properties of the complex modal responses, providing a complete characterization for Gaussian processes.
\end{proof}

\section{Time-Variant Spectral Parameters}

\begin{theorem}[Central Frequency and Bandwidth]\label{thm:freq_bandwidth}
Using the pre-envelope formulation, the central frequency and bandwidth parameter are defined as:

\begin{enumerate}[label=(\roman*)]
\item Central frequency:
\begin{equation}\label{eq:central_freq}
\omega_c(t) = \frac{c_{01,XY}(t)}{c_{00,XX}(t)} = \frac{\sigma_{X\dot{Y}}(t)}{\sigma_X^2(t)}
\end{equation}

\item Bandwidth parameter:
\begin{equation}\label{eq:bandwidth}
q(t) = \sqrt{1 - \frac{[c_{01,XY}(t)]^2}{\sigma_X^2(t)\sigma_{\dot{X}}^2(t)}} = \sqrt{1 - \frac{\sigma_{X\dot{Y}}^2(t)}{\sigma_X^2(t)\sigma_{\dot{X}}^2(t)}}
\end{equation}
\end{enumerate}
\end{theorem}

\begin{proof}
The central frequency in equation~\eqref{eq:central_freq} provides the instantaneous predominant frequency through the ratio of the first-order cross-covariance to the variance. The bandwidth parameter in equation~\eqref{eq:bandwidth} measures the spectral concentration, with values approaching zero for narrowband processes and approaching unity for broadband processes.
\end{proof}

\section{Closed-Form Solution for Unit-Step Modulation}

\begin{theorem}[Unit-Step Solution]\label{thm:unit_step}
For a unit-step modulated white noise input, the cross-covariance $\sigma_{X\dot{Y}}(t)$ has the exact closed-form solution
\begin{align}\label{eq:unit_step_solution}
\sigma_{X\dot{Y}}(t) &= \frac{j\phi_0}{2\xi\omega_0\omega_d} \left[ E_1(-\lambda_1 t) - E_1(-\lambda_2 t) - 2j \arctan\left(\sqrt{\frac{1-\xi^2}{\xi^2}}\right) \right] \nonumber \\
&\quad + \frac{j\phi_0}{2\xi\omega_0\omega_d} e^{-2\xi\omega_0 t} \left[ E_1(\lambda_1 t) - E_1(\lambda_2 t) + 2j\left(\pi - \arctan\left(\sqrt{\frac{1-\xi^2}{\xi^2}}\right)\right) \right]
\end{align}
where $E_1(x)$ is the exponential integral function, $\lambda_{1,2} = -\xi\omega_0 \pm j\omega_d$, $\omega_d = \omega_0\sqrt{1-\xi^2}$, and $\phi_0$ is the white noise spectral density.
\end{theorem}

\begin{proof}
This closed-form solution is derived using Cauchy's residue theorem applied to the complex modal decomposition. The exponential integral functions $E_1(x)$ arise naturally from the integration of the complex exponential functions over the frequency domain. The solution captures both the transient and steady-state behavior of the cross-covariance function for the unit-step modulated excitation.

The derivation involves:
\begin{enumerate}[label=(\arabic*)]
\item Application of complex modal analysis to decompose the system response
\item Use of the orthogonal increment process properties
\item Integration using Cauchy's residue theorem in the complex plane
\item Evaluation of the resulting integrals in terms of exponential integral functions
\end{enumerate}

The two terms in equation~\eqref{eq:unit_step_solution} represent the growing component (first term) and the decaying transient component (second term) of the cross-covariance function.
\end{proof}

\begin{corollary}[Asymptotic Behavior]\label{cor:asymptotic}
As $t \to \infty$, the solution in Theorem~\ref{thm:unit_step} approaches the stationary value
\begin{equation}\label{eq:stationary_value}
\sigma_{X\dot{Y},\infty} = \frac{\phi_0}{\xi\omega_0\omega_d} \arctan\left(\sqrt{\frac{1-\xi^2}{\xi^2}}\right)
\end{equation}
\end{corollary}

\begin{proof}
This follows from the asymptotic properties of the exponential integral function and the fact that the exponentially decaying term in equation~\eqref{eq:unit_step_solution} vanishes as $t \to \infty$.
\end{proof}

\section{Applications and Implications}

The results presented in this work have significant implications for:

\begin{enumerate}[label=(\arabic*)]
\item Structural dynamics analysis of non-classically damped systems
\item Time-variant reliability assessment of engineering structures
\item Signal processing applications involving complex-valued processes
\item Random vibration theory for non-stationary excitations
\end{enumerate}

The closed-form solutions enable efficient computation of time-variant spectral characteristics without numerical integration, providing both computational advantages and theoretical insights into the evolution of spectral properties for linear systems under non-stationary excitation.

\begin{thebibliography}{9}
\bibitem{barbato2008} M. Barbato and J.P. Conte. Spectral characteristics of non-stationary random processes: Theory and applications to linear structural models. \textit{Probabilistic Engineering Mechanics}, 23(4):416--426, 2008.

\bibitem{kuleuven2011} Author Name. On the integral and differential evaluation of non-geometric spectral characteristics. In \textit{Proceedings of EURODYN 2011}, pages MS16--294, 2011.
\end{thebibliography}

\end{document}
