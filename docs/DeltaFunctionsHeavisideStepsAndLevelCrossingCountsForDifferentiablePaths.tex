\documentclass{article}
\usepackage[english]{babel}
\usepackage{geometry,amsmath,amssymb,latexsym}
\geometry{letterpaper}

%%%%%%%%%% Start TeXmacs macros
\newcommand{\assign}{:=}
\newcommand{\tmaffiliation}[1]{\\ #1}
\newenvironment{proof}{\noindent\textbf{Proof\ }}{\hspace*{\fill}$\Box$\medskip}
\newtheorem{definition}{Definition}
\newtheorem{theorem}{Theorem}
%%%%%%%%%% End TeXmacs macros

\begin{document}

\title{Delta Functions, Heaviside Steps, and Level Crossing Counts for
Differentiable Paths}

\author{
  Stephen Crowley
  \tmaffiliation{September 6, 2025}
}

\date{}

\maketitle

{\tableofcontents}

\section{Foundations of Distributions on Real Line}

\begin{definition}
  [Schwartz Test Function Space] The Schwartz space $\mathcal{S} (\mathbb{R})$
  is the space of all infinitely differentiable functions $\phi : \mathbb{R}
  \to \mathbb{R}$ such that for every pair of nonnegative integers $m, n$,
  \[ \sup_{x \in \mathbb{R}} |x^m \phi^{(n)} (x) | < \infty . \]
  Functions in $\mathcal{S} (\mathbb{R})$ are called rapidly decreasing smooth
  test functions.
\end{definition}

\begin{definition}
  [Tempered Distribution] A tempered distribution is a continuous linear
  functional
  \[ T : \mathcal{S} (\mathbb{R}) \to \mathbb{R}. \]
\end{definition}

\begin{definition}
  [Dirac Delta Distribution] The Dirac delta distribution $\delta_a \in
  \mathcal{S}' (\mathbb{R})$ centered at $a \in \mathbb{R}$ is defined by
  \[ \langle \delta_a, \phi \rangle = \phi (a) \]
  for all $\phi \in \mathcal{S} (\mathbb{R})$. When $a = 0$, one writes
  $\delta = \delta_0$.
\end{definition}

\begin{definition}
  [Heaviside Step Function] The Heaviside step function $H : \mathbb{R} \to
  \{0, 1\}$ is defined by
  \[ H (x) = \left\{\begin{array}{ll}
       1 & \text{if } x > 0\\
       0 & \text{if } x \leq 0
     \end{array}\right. \]
\end{definition}

\begin{definition}
  [Distributional Derivative] For a tempered distribution $T \in \mathcal{S}'
  (\mathbb{R})$, its distributional derivative $T' \in \mathcal{S}'
  (\mathbb{R})$ is defined by
  \[ \langle T', \phi \rangle = - \langle T, \phi' \rangle \]
  for all $\phi \in \mathcal{S} (\mathbb{R})$.
\end{definition}

\section{Basic Identities}

\begin{theorem}
  [Heaviside Derivative] The Heaviside step function $H$ satisfies
  \[ H' = \delta \]
  as distributions on $\mathcal{S}' (\mathbb{R})$.
\end{theorem}

\begin{proof}
  For all $\phi \in \mathcal{S} (\mathbb{R})$,
  
  \begin{align}
    \langle H', \phi \rangle & = - \langle H, \phi' \rangle \\
    & = - \int_{- \infty}^{\infty} H (x) \phi' (x)  \hspace{0.17em} dx \\
    & = - \int_0^{\infty} \phi' (x)  \hspace{0.17em} dx \\
    & = - [\phi (x)]_0^{\infty} \\
    & = - (\lim_{x \to \infty} \phi (x) - \phi (0)) \\
    & = \phi (0) 
  \end{align}
  
  where the limit vanishes since $\phi \in \mathcal{S} (\mathbb{R})$ decays
  rapidly at infinity. Thus $\langle H', \phi \rangle = \phi (0) = \langle
  \delta, \phi \rangle$.
\end{proof}

\begin{theorem}
  [Integral of Delta] For any $a \in \mathbb{R}$ and $T \in \mathbb{R}$,
  \[ \int_{- \infty}^T \delta (t - a)  \hspace{0.17em} dt = H (T - a) . \]
\end{theorem}

\begin{proof}
  Define $F (T) = \int_{- \infty}^T \delta (t - a)  \hspace{0.17em} dt$.
  Taking the distributional derivative with respect to $T$:
  \[ F' (T) = \frac{d}{dT}  \int_{- \infty}^T \delta (t - a)  \hspace{0.17em}
     dt = \delta (T - a) \]
  Since $F (- \infty) = 0$ and $F' (T) = \delta (T - a) = H'  (T - a)$ from
  the previous theorem, one has $F (T) = H (T - a) + C$ for some constant $C$.
  The boundary condition $F (- \infty) = 0 = H (- \infty) + C$ implies $C =
  0$, thus $F (T) = H (T - a)$.
\end{proof}

\section{Delta of a Smooth Function}

\begin{theorem}
  [Delta under Change of Variables] Let $g : \mathbb{R} \to \mathbb{R}$ be
  continuously differentiable with isolated, simple zeros $\{x_i \}$ such that
  $g (x_i) = 0$ and $g' (x_i) \neq 0$. Then the identity
  \[ \delta (g (x)) = \sum_i \frac{\delta (x - x_i)}{|g' (x_i) |} \]
  holds in $\mathcal{S}' (\mathbb{R})$.
\end{theorem}

\begin{proof}
  For $\phi \in \mathcal{S} (\mathbb{R})$,
  
  \begin{align*}
    \langle \delta (g (x)), \phi \rangle & = \int_{- \infty}^{\infty} \phi (x)
    \delta (g (x))  \hspace{0.17em} dx.
  \end{align*}
  
  Near each zero $x_i$, where $g$ is locally monotone by the implicit function
  theorem, the change of variables $u = g (x)$ gives
  
  \begin{align*}
    \int_{I_i} \phi (x) \delta (g (x))  \hspace{0.17em} dx & = \int_{g (I_i)}
    \phi (g^{- 1} (u)) \delta (u) \hspace{0.17em} \frac{1}{|g' (g^{- 1} (u))
    |}  \hspace{0.17em} du\\
    & = \frac{\phi (x_i)}{|g' (x_i) |}
  \end{align*}
  
  by the sifting property of $\delta$. Summing over all zeros yields
  \[ \langle \delta (g (x)), \phi \rangle = \sum_i \frac{\phi (x_i)}{|g' (x_i)
     |} = \left\langle \sum_i \frac{\delta (x - x_i)}{|g' (x_i) |}, \phi
     \right\rangle . \]
  Since this holds for all $\phi \in \mathcal{S} (\mathbb{R})$, the
  distributional equality follows.
\end{proof}

\section{Counting Function for Level Crossings}

Let $x : \mathbb{R} \to \mathbb{R}$ be continuously differentiable, and fix $u
\in \mathbb{R}$. Assume the zeros of $g (t) \assign x (t) - u$ are isolated
and simple; that is, for every zero $t_i$, $g' (t_i) = x' (t_i) \neq 0$.

\begin{definition}
  [Level Crossing Counting Function] Define the counting function
  \[ N (T) \assign \text{the number of zeros } t_i  \text{of } x (t) - u
     \text{with } t_i \leq T. \]
\end{definition}

\begin{theorem}
  [Counting Function as Integral Over Delta] For every $T \in \mathbb{R}$,
  \[ N (T) = \int_{- \infty}^T |x' (t) | \delta (x (t) - u)  \hspace{0.17em}
     dt. \]
\end{theorem}

\begin{proof}
  Using the delta change of variables theorem with $g (t) = x (t) - u$,
  
  \begin{align}
    |x' (t) | \delta (x (t) - u) & = |x' (t) |  \sum_i \frac{\delta (t -
    t_i)}{|x' (t_i) |} \\
    & = \sum_i |x' (t) | \frac{\delta (t - t_i)}{|x' (t_i) |} 
  \end{align}
  
  Since $x' (t_i) \neq 0$ by assumption, and $\delta (t - t_i)$ picks out the
  value at $t = t_i$,
  \[ |x' (t) | \delta (x (t) - u) = \sum_i \frac{|x' (t_i) |}{|x' (t_i) |}
     \delta (t - t_i) = \sum_i \delta (t - t_i) . \]
  Therefore,
  \[ \int_{- \infty}^T |x' (t) | \delta (x (t) - u)  \hspace{0.17em} dt =
     \sum_i \int_{- \infty}^T \delta (t - t_i)  \hspace{0.17em} dt = \sum_{t_i
     \leq T} 1 = N (T) . \]
\end{proof}

\begin{theorem}
  [Counting Function as Sum of Heaviside Steps] For every $T \in \mathbb{R}$,
  \[ N (T) = \sum_i H (T - t_i) \]
  where the sum runs over all level crossing times $t_i$.
\end{theorem}

\begin{proof}
  By definition of the Heaviside function, $H (T - t_i) = 1$ if and only if $T
  \geq t_i$, and $H (T - t_i) = 0$ otherwise. Therefore,
  \[ \sum_i H (T - t_i) = \sum_{t_i \leq T} 1 = N (T) . \]
\end{proof}

\begin{theorem}
  [Equivalence of Representations] The delta integral representation and the
  Heaviside step sum representation are equivalent:
  \[ \int_{- \infty}^T |x' (t) | \delta (x (t) - u)  \hspace{0.17em} dt =
     \sum_i H (T - t_i) . \]
\end{theorem}

\begin{proof}
  This follows immediately from the two previous theorems, since both
  expressions equal $N (T)$.
\end{proof}

\end{document}
