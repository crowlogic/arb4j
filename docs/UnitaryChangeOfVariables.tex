\documentclass{article}
\usepackage{amsmath}
\usepackage{amssymb}
\usepackage{amsthm}

\newtheorem{theorem}{Theorem}
\newtheorem{lemma}{Lemma}

\title{L2 Norm Preservation Under Monotonic Substitutions}
\author{}
\date{}

\begin{document}

\maketitle

\begin{theorem}[L2 Norm Preservation via Square Root Jacobian Factor]
Let $g: I \to J$ be a strictly monotonic and differentiable function between intervals $I, J \subseteq \mathbb{R}$ (possibly unbounded), with $g'(y) \neq 0$ for all $y \in I$. For any $f \in L^2(J, dx)$, define the transformed function $\tilde{f}: I \to \mathbb{C}$ by
$$\tilde{f}(y) = f(g(y)) \sqrt{|g'(y)|}.$$
Then $\tilde{f} \in L^2(I, dy)$ and
$$\|\tilde{f}\|_{L^2(I, dy)} = \|f\|_{L^2(J, dx)}.$$
\end{theorem}

\begin{proof}
Without loss of generality, assume $g$ is strictly increasing (the decreasing case follows by considering $-g$).

First, establish the change of variables formula. For any measurable set $E \subseteq J$:
$$\int_E |f(x)|^2 \, dx = \int_{g^{-1}(E)} |f(g(y))|^2 |g'(y)| \, dy.$$

This follows from the standard change of variables theorem, since $g$ is strictly monotonic and differentiable with $g'(y) \neq 0$.

To handle potentially unbounded intervals, consider the norm computation:
\begin{align}
\|\tilde{f}\|_{L^2(I, dy)}^2 &= \int_I |\tilde{f}(y)|^2 \, dy \\
&= \int_I |f(g(y)) \sqrt{|g'(y)|}|^2 \, dy \\
&= \int_I |f(g(y))|^2 |g'(y)| \, dy.
\end{align}

By the change of variables formula applied to $J = g(I)$:
$$\int_I |f(g(y))|^2 |g'(y)| \, dy = \int_J |f(x)|^2 \, dx = \|f\|_{L^2(J, dx)}^2.$$

For unbounded intervals, this equality holds by the monotone convergence theorem: approximate $I$ by an increasing sequence of bounded intervals $I_n \uparrow I$, apply the result to each $I_n$, and take the limit.

Therefore:
$$\|\tilde{f}\|_{L^2(I, dy)} = \|f\|_{L^2(J, dx)}.$$

The integrability of $\tilde{f}$ follows immediately from the norm equality and the assumption that $f \in L^2(J, dx)$.
\end{proof}

\begin{lemma}[Density of Transformed Functions]
Under the conditions of Theorem 1, the set $\{f(g(\cdot)) : f \in L^2(J, dx)\}$ is dense in $L^2(I, |g'(y)| \, dy)$, where $L^2(I, |g'(y)| \, dy)$ denotes the space of square-integrable functions with respect to the measure $|g'(y)| \, dy$.
\end{lemma}

\begin{proof}
The map $f \mapsto f \circ g$ is an isometric isomorphism from $L^2(J, dx)$ to $L^2(I, |g'(y)| \, dy)$ by the change of variables formula. Since $L^2(J, dx)$ is complete, its image under an isometry is also complete, hence dense in itself.
\end{proof}

\begin{theorem}[Necessity of Square Root Factor]
Under the same conditions as Theorem 1, the factor $\sqrt{|g'(y)|}$ is necessary for L2 norm preservation. That is, if $h(y) = f(g(y)) \cdot \phi(y)$ for some measurable function $\phi: I \to \mathbb{R}^+$ satisfies $\|h\|_{L^2(I, dy)} = \|f\|_{L^2(J, dx)}$ for all $f \in L^2(J, dx)$, then $\phi(y) = \sqrt{|g'(y)|}$ almost everywhere.
\end{theorem}

\begin{proof}
Suppose $\|f(g(\cdot)) \cdot \phi(\cdot)\|_{L^2(I, dy)} = \|f\|_{L^2(J, dx)}$ for all $f \in L^2(J, dx)$.

Then for any $f \in L^2(J, dx)$:
$$\int_I |f(g(y))|^2 |\phi(y)|^2 \, dy = \|f\|_{L^2(J, dx)}^2 = \int_I |f(g(y))|^2 |g'(y)| \, dy,$$
where the second equality follows from the change of variables formula.

Therefore:
$$\int_I |f(g(y))|^2 (|\phi(y)|^2 - |g'(y)|) \, dy = 0$$
for all $f \in L^2(J, dx)$.

By Lemma 1, functions of the form $f(g(y))$ are dense in $L^2(I, |g'(y)| \, dy)$. For any $u \in L^2(I, |g'(y)| \, dy)$, there exists a sequence $f_n \in L^2(J, dx)$ such that $f_n(g(y)) \to u(y)$ in $L^2(I, |g'(y)| \, dy)$.

Since $|\phi(y)|^2 - |g'(y)|$ is integrable with respect to $|g'(y)| \, dy$ (by the boundedness of the norm-preserving property), we have:
$$\int_I |u(y)|^2 (|\phi(y)|^2 - |g'(y)|) \, dy = 0$$
for all $u \in L^2(I, |g'(y)| \, dy)$.

In particular, taking $u(y) = \text{sgn}(|\phi(y)|^2 - |g'(y)|) \cdot \mathbf{1}_{\{|\phi(y)|^2 \neq |g'(y)|\}}(y)$, we obtain:
$$\int_I ||\phi(y)|^2 - |g'(y)|| \, |g'(y)| \, dy = 0.$$

Since $|g'(y)| > 0$ almost everywhere, this implies $|\phi(y)|^2 = |g'(y)|$ almost everywhere.

Taking $\phi(y) > 0$, we conclude $\phi(y) = \sqrt{|g'(y)|}$ almost everywhere.
\end{proof}

\begin{theorem}[Extension to General Measures]
Let $\mu$ and $\nu$ be $\sigma$-finite measures on $I$ and $J$ respectively, and let $g: I \to J$ be a measurable bijection. If $\nu = \mu \circ g^{-1}$ (i.e., $\nu(E) = \mu(g^{-1}(E))$ for all measurable $E \subseteq J$), then for $f \in L^2(J, d\nu)$:
$$\tilde{f}(y) = f(g(y)) \sqrt{\frac{d(\mu \circ g^{-1})}{d\mu}(y)}$$
satisfies $\|\tilde{f}\|_{L^2(I, d\mu)} = \|f\|_{L^2(J, d\nu)}$, where $\frac{d(\mu \circ g^{-1})}{d\mu}$ is the Radon-Nikodym derivative.
\end{theorem}

\begin{proof}
When $\mu$ and $\nu$ are both Lebesgue measure and $g$ is differentiable, the Radon-Nikodym derivative is $|g'(y)|$, reducing to Theorem 1. The general case follows by the same change of variables argument using the definition of the pushforward measure.
\end{proof}

\end{document}
