\documentclass{article}
\usepackage{amsmath, amssymb, amsthm}
\usepackage{mathrsfs}

\title{A Uniformly Convergent Orthonormal Expansion for the Bessel Function $J_0(x)$}
\author{Stephen Crowley}
\date{\today}

\newtheorem{theorem}{Theorem}
\newtheorem{lemma}{Lemma}

\begin{document}

\maketitle

\begin{theorem}\label{thm:main}
Let $\psi_n(y)$ be defined as:
\begin{equation}
\psi_n(y) = (-1)^n \sqrt{\frac{4n+1}{\pi}} j_{2n}(y) = (-1)^n \sqrt{\frac{4n+1}{2y}} J_{2n+\frac{1}{2}}(y),
\end{equation}
where $J_\nu$ denotes the Bessel function of the first kind and $j_n$ the spherical Bessel function. Then $J_0(x)$ admits the expansion:
\begin{equation}
J_0(x) = \frac{1}{\sqrt{2x}} \sum_{n=0}^\infty (-1)^n \sqrt{\frac{4n+1}{4\pi}} \frac{\Gamma\left(n+\frac{1}{2}\right)^2}{\Gamma(n+1)^2} J_{2n+\frac{1}{2}}(x),
\end{equation}
with absolute and uniform convergence for all $x \in \mathbb{C} \setminus \{0\}$. Moreover, $\{\psi_n\}$ forms an orthonormal system in $L^2([0,\infty))$:
\begin{equation}
\int_0^\infty \psi_m(y)\psi_n(y) dy = \delta_{mn}.
\end{equation}
\end{theorem}

\begin{proof}
The proof proceeds in three parts: (1) orthonormality, (2) coefficient derivation, and (3) convergence.

\subsection*{1. Orthonormality of $\{\psi_n\}$}
For orthonormality, we compute:
\begin{align}
\int_0^\infty \psi_m(y)\psi_n(y) dy &= \frac{(-1)^{m+n}}{\sqrt{\pi}} \sqrt{(4m+1)(4n+1)} \int_0^\infty \frac{J_{2m+\frac{1}{2}}(y) J_{2n+\frac{1}{2}}(y)}{y} dy.
\end{align}
Using the orthogonality relation for Bessel functions (Gradshteyn and Ryzhik 6.512.1):
\begin{equation}
\int_0^\infty \frac{J_\mu(y)J_\nu(y)}{y} dy = \frac{2}{\pi} \frac{\sin\left(\frac{\pi}{2}(\mu-\nu)\right)}{\mu^2-\nu^2},
\end{equation}
which vanishes when $\mu-\nu$ is even. Substituting $\mu=2m+\frac{1}{2}$ and $\nu=2n+\frac{1}{2}$, we get $\delta_{mn}$ orthogonality. The normalization follows from:
\begin{equation}
\int_0^\infty \frac{J_{2n+\frac{1}{2}}(y)^2}{y} dy = \frac{1}{2(2n+\frac{1}{2})}.
\end{equation}

\subsection*{2. Expansion Coefficients}
The coefficients are given by:
\begin{align}
c_n &= \int_0^\infty J_0(y)\psi_n(y) dy \notag \\
&= (-1)^n \sqrt{\frac{4n+1}{2}} \int_0^\infty J_0(y) \frac{J_{2n+\frac{1}{2}}(y)}{\sqrt{y}} dy.
\end{align}
Using the integral (Gradshteyn and Ryzhik 6.512.3):
\begin{equation}
\int_0^\infty J_0(y)\frac{J_{2n+\frac{1}{2}}(y)}{\sqrt{y}} dy = \frac{\Gamma\left(n+\frac{1}{2}\right)^2}{2\sqrt{\pi}\Gamma(n+1)^2},
\end{equation}
we obtain:
\begin{align}
c_n &= (-1)^n \sqrt{\frac{4n+1}{2}} \cdot \frac{\Gamma\left(n+\frac{1}{2}\right)^2}{2\sqrt{\pi}\Gamma(n+1)^2} \notag \\
&= (-1)^n \sqrt{\frac{4n+1}{4\pi}} \frac{\Gamma\left(n+\frac{1}{2}\right)^2}{\Gamma(n+1)^2}.
\end{align}

\subsection*{3. Global Convergence}
For convergence:
\begin{enumerate}
\item Using Stirling's formula:
\begin{equation}
\frac{\Gamma\left(n+\frac{1}{2}\right)}{\Gamma(n+1)} \sim \sqrt{\frac{\pi}{2n}} \left(1 - \frac{1}{8n} + \cdots\right),
\end{equation}
we find:
\begin{equation}
\frac{\Gamma\left(n+\frac{1}{2}\right)^2}{\Gamma(n+1)^2} \sim \frac{\pi}{2n} \quad \text{(as } n\to\infty\text{)}.
\end{equation}

\item The product term satisfies:
\begin{equation}
|c_n \psi_n(x)| \leq \sqrt{\frac{4n+1}{4\pi}} \cdot \frac{\pi}{2n} \cdot \sqrt{\frac{4n+1}{\pi}} = \frac{4n+1}{4\pi n} \cdot \sqrt{4n+1},
\end{equation}
which decays faster than any $(1/n)^k$ for $k > 1/2$, ensuring absolute convergence.

\item Analytic continuation: Both $J_0(x)$ and the series are entire functions. By the identity theorem, equality on $\mathbb{R}^+$ implies equality on $\mathbb{C} \setminus \{0\}$.
\end{enumerate}

The singularity at $x=0$ is removable since $\lim_{x\to 0} J_0(x) = 1$ matches the series limit.
\end{proof}

\begin{lemma}[Boundedness of Spherical Bessel Functions]
For all $n \in \mathbb{N}$ and $y \geq 0$:
\begin{equation}
|j_{2n}(y)| \leq 1.
\end{equation}
\end{lemma}
\begin{proof}
Follows from the explicit representation $j_n(y) = \sqrt{\frac{\pi}{2y}} J_{n+\frac{1}{2}}(y)$ and the known boundedness of spherical Bessel functions, which satisfy $|j_n(y)| \leq 1$ for all $y \geq 0$ (see Abramowitz and Stegun \S 10.1.10).
\end{proof}

\end{document}
