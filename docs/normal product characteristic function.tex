\documentclass{article}
\usepackage[english]{babel}
\usepackage{geometry,amsmath,latexsym}
\geometry{letterpaper}

%%%%%%%%%% Start TeXmacs macros
\newcommand{\tmtextbf}[1]{\text{{\bfseries{#1}}}}
\newenvironment{proof}{\noindent\textbf{Proof\ }}{\hspace*{\fill}$\Box$\medskip}
\newtheorem{theorem}{Theorem}
%%%%%%%%%% End TeXmacs macros

\begin{document}

\title{The Characteristic Function of the Product of Independent Standard
Normal Variables}

\author{Stephen Crowley}

\date{November 19, 2024}

\maketitle

\

\begin{abstract}
  This paper demonstrates that the characteristic function of the product of
  two independent standard normal random variables involves the Bessel
  function of the first kind of order 0. Polar coordinate transformations and
  properties of Bessel functions are utilized to obtain the closed form
  expression.
\end{abstract}

\

{\tableofcontents}

\section{Introduction}

The characteristic function of the product of two independent standard normal
random variables serves as a fundamental result in probability theory and
statistical analysis. This paper presents a rigorous derivation of its closed
form.

\section{Main Result}

\begin{theorem}
  For independent standard normal random variables X and Y, the characteristic
  function of their product XY is given by:
  \begin{equation}
    \phi_{XY} (t) = \frac{J_0 \left( \frac{t}{\sqrt{1 + t^2}} \right)}{\sqrt{1
    + t^2}}
  \end{equation}
  where $J_0$ denotes the Bessel function of the first kind of order zero.
\end{theorem}

\section{Proof}

\begin{proof}
  The derivation begins with the characteristic function definition:
  \begin{equation}
    \phi_{XY} (t) = E [e^{itXY}] = \frac{1}{2 \pi}  \int_{- \infty}^{\infty}
    \int_{- \infty}^{\infty} e^{itxy} e^{- (x^2 + y^2) / 2} dxdy
  \end{equation}
  \tmtextbf{Polar Coordinate Transformation}
  
  The introduction of polar coordinates where $x = r \cos \theta$, $y = r \sin
  \theta$, and $dxdy = rdrd \theta$ transforms the integral to:
  \begin{equation}
    \frac{1}{2 \pi}  \int_0^{\infty} \int_0^{2 \pi} e^{itr^2 \cos \theta \sin
    \theta} re^{- r^2 / 2} d \theta dr
  \end{equation}
  \tmtextbf{Variable Substitution}
  
  The substitution $u = r^2 / 2$, with $du = rdr$, yields:
  \begin{equation}
    \frac{1}{2 \pi}  \int_0^{\infty} \int_0^{2 \pi} e^{2 itu \cos \theta \sin
    \theta} e^{- u} d \theta du
  \end{equation}
  \tmtextbf{Double Angle Formula}
  
  Application of the identity $\cos \theta \sin \theta = \frac{1}{2} \sin (2
  \theta)$ gives:
  \begin{equation}
    \frac{1}{2 \pi}  \int_0^{\infty} \int_0^{2 \pi} e^{itu \sin (2 \theta)}
    e^{- u} d \theta du
  \end{equation}
  \tmtextbf{Bessel Function Connection}
  
  The inner integral relates to the Bessel function through the identity:
  \begin{equation}
    \int_0^{2 \pi} e^{itu \sin (2 \theta)} d \theta = 2 \pi J_0  (tu)
  \end{equation}
  This follows from the integral representation of the Bessel function of the
  first kind:
  \begin{equation}
    J_0 (z) = \frac{1}{2 \pi}  \int_0^{2 \pi} e^{iz \sin (\theta)} d \theta
  \end{equation}
  The integral thus reduces to:
  \begin{equation}
    \int_0^{\infty} J_0  (tu) e^{- u} du
  \end{equation}
  \tmtextbf{Final Evaluation}
  
  The evaluation proceeds through the known Laplace transform of Bessel
  functions:
  \begin{equation}
    \int_0^{\infty} J_0  (at) e^{- ut} dt = \frac{1}{\sqrt{1 + a^2}} J_0
    \left( \frac{a}{\sqrt{1 + a^2}} \right)
  \end{equation}
  This leads to the final result:
  \begin{equation}
    \phi_{XY} (t) = \frac{J_0 \left( \frac{t}{\sqrt{1 + t^2}} \right)}{\sqrt{1
    + t^2}}
  \end{equation}
\end{proof}

\section{Conclusion}

The derivation establishes that the characteristic function of the product of
two independent standard normal random variables takes the form $J_0  (t /
\sqrt{1 + t^2}) / \sqrt{1 + t^2}$. The proof relies on coordinate
transformation, properties of Bessel functions, and integral transforms. This
result holds significance in various applications of probability theory and
statistical analysis.

\end{document}
