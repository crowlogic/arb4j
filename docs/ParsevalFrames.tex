\documentclass{article}
\usepackage[english]{babel}
\usepackage{geometry}
\geometry{letterpaper}

%%%%%%%%%% Start TeXmacs macros
\newcommand{\tmaffiliation}[1]{\\ #1}
%%%%%%%%%% End TeXmacs macros

\begin{document}

\title{Parseval Frames}

\author{
  $S \tau \Sigma v \varepsilon$
  \tmaffiliation{December 7, 2024}
}

\maketitle

Parseval frames are a specialized type of frame in Hilbert spaces that extend
the concept of orthonormal bases. They maintain the preservation of vector
norms while introducing redundancy into the system. This redundancy allows for
greater flexibility in representation and analysis, distinguishing Parseval
frames as a valuable mathematical tool in the study of Hilbert spaces.

\section{Definition of Parseval Frames}

A Parseval frame is a specific type of frame in linear algebra and functional
analysis that generalizes the concept of orthonormal bases while maintaining
certain desirable properties. Formally, a sequence of vectors $\{f_j \}$ in a
Hilbert space $H$ is called a Parseval frame if it satisfies the following
condition for all vectors $x$ in $H$:
\begin{equation}
  \sum_j | \langle x, f_j \rangle |^2 = \|x\|^2
\end{equation}
This equation, known as the frame condition, is a generalization of Parseval's
identity for orthonormal bases. It ensures that the norm of any vector $x$ is
preserved when expressed in terms of its inner products with the frame
elements.

Parseval frames can be characterized by their frame operator $S$, defined as:
\begin{equation}
  Sx = \sum_j \langle x, f_j \rangle f_j
\end{equation}
For a Parseval frame, the frame operator is equal to the identity operator,
i.e., $S = I$. This property distinguishes Parseval frames from general frames
and tight frames.

An equivalent definition of a Parseval frame can be given in terms of the
analysis operator $T$ and its adjoint $T^{\ast}$:
\[ T^{\ast} T = I \]
where $T$ is the operator that maps a vector $x$ to its sequence of frame
coefficients $\{\langle x, f_j \rangle\}$.

In finite-dimensional spaces, Parseval frames have an additional
characterization: a set of vectors $\{f_j \}$ forms a Parseval frame if and
only if the matrix $F$ whose columns are the frame vectors satisfies
$FF^{\ast} = I$, where $F^{\ast}$ is the conjugate transpose of $F$.

\section{Key Properties of Parseval Frames}

\begin{enumerate}
  \item Norm Equivalence: One of the fundamental properties of Parseval frames
  is their ability to maintain norm equivalence between a vector in the
  Hilbert space and its sequence of coefficients.
  
  \item Redundancy: Unlike orthonormal bases, Parseval frames can have more
  vectors than the dimension of the space they span.
  
  \item Tight Frame Property: Parseval frames are a special case of tight
  frames, where the frame bounds $A$ and $B$ are equal to $1$.
  
  \item Reconstruction Formula: For any vector $x$ in the Hilbert space, a
  Parseval frame $\{f_j \}$ satisfies the reconstruction formula:
  \begin{equation}
    x = \sum_j \langle x, f_j \rangle f_j
  \end{equation}
  \item Parseval's Identity: Parseval frames satisfy a generalized version of
  Parseval's identity:
  \begin{equation}
    \|x\|^2 = \sum_j | \langle x, f_j \rangle |^2
  \end{equation}
  \item Duality: Every Parseval frame is self-dual, meaning that the frame
  itself serves as its own dual frame.
  
  \item Invariance Under Unitary Transformations: If $\{f_j \}$ is a Parseval
  frame and $U$ is a unitary operator, then $\{Uf_j \}$ is also a Parseval
  frame.
  
  \item Finite-Dimensional Characterization: In finite-dimensional Hilbert
  spaces, a set of vectors forms a Parseval frame if and only if the matrix
  whose columns are the frame vectors has orthonormal rows.
\end{enumerate}

\section{Construction via Orthogonal Projection}

Let $H$ be infinite-dimensional Hilbert space\\
Let $W \subseteq H$ be finite-dimensional subspace\\
Let $\{e_1, e_2, ..., e_n \}$ be orthonormal basis for $W$\\
Let $\{f_k \}_{k = 1}^{\infty}$ be orthonormal basis for $H$

Orthogonal projection $P_W$ onto $W$:\\
$P_W x = \sum \langle x, e_i \rangle e_i$

Construction of Parseval frame:
\begin{equation}
  \tilde{f}_k = \frac{P_W f_k}{\sqrt{\sum | \langle f_k, e_i \rangle |^2}}
\end{equation}
Verification:
\begin{enumerate}
  \item For any $x \in W$:
  \begin{equation}
    \langle x, \tilde{f}_k \rangle = \frac{\langle x, P_W f_k
    \rangle}{\sqrt{\sum | \langle f_k, e_i \rangle |^2}} = \frac{\langle P_W
    x, f_k \rangle}{\sqrt{\sum | \langle f_k, e_i \rangle |^2}} =
    \frac{\langle x, f_k \rangle}{\sqrt{\sum | \langle f_k, e_i \rangle |^2}}
  \end{equation}
  \item Parseval frame condition:
  \begin{equation}
    \sum | \langle x, \tilde{f}_k \rangle |^2 = \sum \frac{| \langle x, f_k
    \rangle |^2}{\sum | \langle f_k, e_i \rangle |^2}
  \end{equation}
  \item Interchanging sums:
  \begin{equation}
    \sum_i \sum_k \frac{| \langle x, f_k \rangle |^2 | \langle f_k, e_i
    \rangle |^2}{(\sum | \langle f_k, e_j \rangle |^2)^2}
  \end{equation}
  \item Using orthonormal basis property:
  \begin{equation}
    \sum | \langle x, e_i \rangle |^2 = \|P_W x\|^2 = \|x\|^2
  \end{equation}
\end{enumerate}

\end{document}
