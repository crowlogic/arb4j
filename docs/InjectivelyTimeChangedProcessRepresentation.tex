\documentclass[11pt]{article}
\usepackage{amsmath,amsthm,amssymb,amsfonts}
\usepackage{geometry}
\geometry{margin=1in}

\newtheorem{theorem}{Theorem}
\newtheorem{lemma}{Lemma}
\newtheorem{proposition}{Proposition}
\newtheorem{definition}{Definition}
\newtheorem{corollary}{Corollary}

\title{Injectively Time-Changed Stationary Processes:\\ A Spectral Analysis}
\author{}
\date{}

\begin{document}
\maketitle

\section{Introduction}

This analysis concerns injectively time-changed stationary processes, which arise from spectral representations involving a warping function $\theta(t)$ applied to the oscillatory kernel.

\begin{definition}
An \emph{injectively time-changed stationary process} is a stochastic process $\{X(t)\}_{t \in \mathbb{R}}$ admitting the spectral representation
\begin{equation}
X(t) = \int_{-1}^1 e^{i\lambda\theta(t)} \, dZ(\lambda)
\end{equation}
where $\theta: \mathbb{R} \to \mathbb{R}$ is strictly increasing, $\theta \in C^1(\mathbb{R})$, and $\{Z(\lambda)\}_{\lambda \in [-1,1]}$ is an orthogonal increment process with $E[|dZ(\lambda)|^2] = F(d\lambda)$ for some finite measure $F$ on $[-1,1]$.
\end{definition}

\section{Gain Function Representation}

\begin{proposition}[Evolutionary Spectral Form]
The process $X(t)$ admits the evolutionary spectral representation
\begin{equation}
X(t) = \int_{-1}^1 A(t,\lambda) e^{i\lambda t} \, dZ(\lambda)
\end{equation}
where the gain function is
\begin{equation}
A(t,\lambda) = e^{i\lambda(\theta(t)-t)}
\end{equation}
\end{proposition}

\begin{proof}
Direct substitution yields:
\begin{align}
X(t) &= \int_{-1}^1 e^{i\lambda\theta(t)} \, dZ(\lambda)\\
&= \int_{-1}^1 e^{i\lambda(\theta(t)-t)} e^{i\lambda t} \, dZ(\lambda)\\
&= \int_{-1}^1 A(t,\lambda) e^{i\lambda t} \, dZ(\lambda)
\end{align}
\end{proof}

\section{Fundamental Properties}

\begin{theorem}[Spectral Characteristics]
Let $X(t)$ be an injectively time-changed stationary process. Then:
\begin{enumerate}
\item $X(t)$ is well-defined for all $t \in \mathbb{R}$
\item $E[|X(t)|^2] = \int_{-1}^1 F(d\lambda) < \infty$
\item The covariance function satisfies
\begin{equation}
\text{Cov}(X(s),X(t)) = \int_{-1}^1 e^{i\lambda(\theta(s)-\theta(t))} F(d\lambda)
\end{equation}
\end{enumerate}
\end{theorem}

\begin{proof}
(1) Since $\theta$ is strictly increasing and continuous, $\theta(t)$ is well-defined for all $t$. The stochastic integral converges in $L^2$ by the isometry property:
\begin{align}
E[|X(t)|^2] &= E\left[\left|\int_{-1}^1 e^{i\lambda\theta(t)} dZ(\lambda)\right|^2\right]\\
&= \int_{-1}^1 |e^{i\lambda\theta(t)}|^2 F(d\lambda) = \int_{-1}^1 F(d\lambda) < \infty
\end{align}

(2) Follows immediately from (1).

(3) By orthogonality of the random measure increments:
\begin{align}
\text{Cov}(X(s),X(t)) &= E\left[\int_{-1}^1 e^{i\lambda\theta(s)} dZ(\lambda) \cdot \overline{\int_{-1}^1 e^{i\mu\theta(t)} dZ(\mu)}\right]\\
&= \int_{-1}^1 e^{i\lambda\theta(s)} \overline{e^{i\lambda\theta(t)}} F(d\lambda)\\
&= \int_{-1}^1 e^{i\lambda(\theta(s)-\theta(t))} F(d\lambda)
\end{align}
\end{proof}

\begin{theorem}[Non-Stationarity Condition]
An injectively time-changed stationary process $X(t)$ is stationary if and only if $\theta(t) = t + c$ for some constant $c \in \mathbb{R}$.
\end{theorem}

\begin{proof}
($\Leftarrow$) If $\theta(t) = t + c$, then
\begin{equation}
\text{Cov}(X(s),X(t)) = \int_{-1}^1 e^{i\lambda c} e^{-i\lambda c} F(d\lambda) = \int_{-1}^1 F(d\lambda)
\end{equation}
which is independent of $s$ and $t$, establishing stationarity.

($\Rightarrow$) Suppose $X(t)$ is stationary. Then $\text{Cov}(X(s),X(t))$ depends only on $t-s$. From the covariance formula, this requires
\begin{equation}
\theta(s) - \theta(t) = g(s-t)
\end{equation}
for some function $g$. Differentiating with respect to $s$:
\begin{equation}
\theta'(s) = g'(s-t)
\end{equation}
Since the left side depends only on $s$ and the right side on $s-t$, both must be constant. Thus $\theta'(t) = k$ for some constant $k$, implying $\theta(t) = kt + c$. For the covariance to depend only on the difference $s-t$, one requires $k = 1$, yielding $\theta(t) = t + c$.
\end{proof}

\section{Warping Function Analysis}

\begin{definition}
The \emph{warping deviation function} is $\Delta(t) := \theta(t) - t$.
\end{definition}

\begin{proposition}[Deviation Dynamics]
Let $\Delta(t) = \theta(t) - t$ where $\theta$ is strictly increasing. Then:
\begin{enumerate}
\item $\Delta'(t) = \theta'(t) - 1$
\item The gain function becomes $A(t,\lambda) = e^{i\lambda\Delta(t)}$
\item The instantaneous frequency modulation is $\lambda\Delta'(t)$
\end{enumerate}
\end{proposition}

\begin{proof}
(1) and (2) are immediate. For (3), the phase of the spectral component at frequency $\lambda$ is $\lambda\theta(t)$. The instantaneous frequency is
\begin{equation}
\frac{d}{dt}[\lambda\theta(t)] = \lambda\theta'(t) = \lambda(1 + \Delta'(t)) = \lambda + \lambda\Delta'(t)
\end{equation}
The modulation relative to the base frequency $\lambda$ is $\lambda\Delta'(t)$.
\end{proof}

\section{Inversion Theory}

\begin{theorem}[Spectral Inversion]
Let $X(t)$ be an injectively time-changed stationary process with absolutely continuous spectral measure $F(d\lambda) = f(\lambda)d\lambda$. If $\theta$ is invertible with inverse $\psi$, then
\begin{equation}
f(\lambda) = \lim_{T \to \infty} \frac{1}{2T} \int_{-T}^T X(\psi(u)) e^{-i\lambda u} \frac{du}{\psi'(u)}
\end{equation}
\end{theorem}

\begin{proof}
Making the substitution $u = \theta(t)$, so $t = \psi(u)$ and $dt = \psi'(u)du$:
\begin{align}
X(\psi(u)) &= \int_{-1}^1 e^{i\mu u} dZ(\mu)
\end{align}
This is the spectral representation of a stationary process in the $u$-domain. The standard inversion formula for stationary processes gives:
\begin{equation}
f(\lambda) = \lim_{T \to \infty} \frac{1}{2T} \int_{-T}^T X(\psi(u)) e^{-i\lambda u} \frac{du}{\psi'(u)}
\end{equation}
where the factor $\frac{1}{\psi'(u)}$ accounts for the change of measure.
\end{proof}

\section{Band-Limited Structure}

\begin{theorem}[Oscillatory Characterization]
An injectively time-changed stationary process $X(t)$ with spectral support in $[-1,1]$ exhibits oscillatory behavior in the sense of Priestley if and only if the spectral measure $F$ concentrates mass away from $\lambda = 0$.
\end{theorem}

\begin{proof}
The process has the representation
\begin{equation}
X(t) = \int_{-1}^1 e^{i\lambda\theta(t)} dZ(\lambda)
\end{equation}
Oscillatory behavior requires sustained periodic components. If $F$ has significant mass at $\lambda = 0$, the corresponding component $e^{i \cdot 0 \cdot \theta(t)} = 1$ contributes a non-oscillatory constant term. Conversely, if $F$ concentrates away from zero, all spectral components $e^{i\lambda\theta(t)}$ with $\lambda \neq 0$ exhibit oscillatory behavior modulated by the time-change $\theta(t)$.
\end{proof}

\begin{corollary}[Band-Limited Narrow-Band Property]
If $F$ is concentrated in an interval $[\lambda_0 - \epsilon, \lambda_0 + \epsilon]$ with $\lambda_0 \neq 0$ and small $\epsilon > 0$, then $X(t)$ exhibits narrow-band oscillatory behavior around the carrier frequency $\lambda_0$.
\end{corollary}

\section{Asymptotic Analysis}

\begin{theorem}[Large Deviation Asymptotics]
If $\Delta(t) = \theta(t) - t$ grows without bound as $|t| \to \infty$, then the process $X(t)$ exhibits asymptotic phase decorrelation:
\begin{equation}
\lim_{|t-s| \to \infty} \text{Cov}(X(s),X(t)) = 0
\end{equation}
provided $F$ has no point masses.
\end{theorem}

\begin{proof}
The covariance is
\begin{equation}
\text{Cov}(X(s),X(t)) = \int_{-1}^1 e^{i\lambda(\theta(s)-\theta(t))} F(d\lambda)
\end{equation}
If $|\theta(s) - \theta(t)| \to \infty$ as $|t-s| \to \infty$, then by the Riemann-Lebesgue lemma, the oscillatory integral converges to zero when $F$ is absolutely continuous with respect to Lebesgue measure.
\end{proof}

\section{Conclusion}

Injectively time-changed stationary processes provide a mathematically rigorous framework for analyzing non-stationary oscillatory phenomena through spectral methods. The warping function $\theta(t)$ induces a time-dependent modulation while preserving the fundamental spectral structure inherited from the underlying orthogonal increment process.

\end{document}
