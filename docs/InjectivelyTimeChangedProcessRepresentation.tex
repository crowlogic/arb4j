\documentclass[11pt]{article}
\usepackage{amsmath,amsthm,amssymb,amsfonts}
\usepackage{geometry}
\geometry{margin=1in}

\newtheorem{theorem}{Theorem}
\newtheorem{lemma}{Lemma}
\newtheorem{proposition}{Proposition}
\newtheorem{definition}{Definition}
\newtheorem{corollary}{Corollary}

\title{Injectively Time-Changed Stationary Processes:\\ A Spectral Analysis}
\author{}
\date{}

\begin{document}
\maketitle

\section{Introduction}

We develop the theory of injectively time-changed stationary processes, which arise from spectral representations of the form
\begin{equation}
X(t) = \int_{-1}^1 f(\lambda) e^{i\lambda\theta(t)} \, d\lambda
\end{equation}
where $\theta: \mathbb{R} \to \mathbb{R}$ is strictly increasing and $f \in L^2([-1,1])$.

\begin{definition}
An \emph{injectively time-changed stationary process} is a stochastic process $\{X(t)\}_{t \in \mathbb{R}}$ admitting the spectral representation
\begin{equation}
X(t) = \int_{-1}^1 e^{i\lambda\theta(t)} \, dZ(\lambda)
\end{equation}
where $\theta: \mathbb{R} \to \mathbb{R}$ is strictly increasing, $\theta \in C^1(\mathbb{R})$, and $\{Z(\lambda)\}_{\lambda \in [-1,1]}$ is an orthogonal increment process with $E[|dZ(\lambda)|^2] = F(d\lambda)$ for some finite measure $F$ on $[-1,1]$.
\end{definition}

\section{Gain Function Decomposition}

\begin{proposition}[Evolutionary Spectral Representation]
The process $X(t)$ can be written in evolutionary form as
\begin{equation}
X(t) = \int_{-1}^1 A(t,\lambda) e^{i\lambda t} \, dZ(\lambda)
\end{equation}
where the gain function is
\begin{equation}
A(t,\lambda) = e^{i\lambda(\theta(t)-t)}
\end{equation}
\end{proposition}

\begin{proof}
Direct substitution gives
\begin{align}
A(t,\lambda) e^{i\lambda t} &= e^{i\lambda(\theta(t)-t)} e^{i\lambda t}\\
&= e^{i\lambda(\theta(t)-t+t)}\\
&= e^{i\lambda\theta(t)}
\end{align}
\end{proof}

\section{Fundamental Properties}

\begin{theorem}[Spectral Representation]
Let $X(t)$ be an injectively time-changed stationary process. Then:
\begin{enumerate}
\item $X(t)$ is well-defined for all $t \in \mathbb{R}$
\item $E[|X(t)|^2] = \int_{-1}^1 F(d\lambda) < \infty$
\item The covariance function satisfies
\begin{equation}
\text{Cov}(X(s),X(t)) = \int_{-1}^1 e^{i\lambda(\theta(t)-\theta(s))} F(d\lambda)
\end{equation}
\end{enumerate}
\end{theorem}

\begin{proof}
(1) Since $\theta$ is strictly increasing and continuous, $\theta(t)$ is well-defined for all $t$. The integral converges by the Cauchy-Schwarz inequality.

(2) By orthogonality:
\begin{align}
E[|X(t)|^2] &= E\left[\left|\int_{-1}^1 e^{i\lambda\theta(t)} dZ(\lambda)\right|^2\right]\\
&= \int_{-1}^1 F(d\lambda) < \infty
\end{align}

(3) By orthogonality of increments:
\begin{align}
\text{Cov}(X(s),X(t)) &= E\left[\int_{-1}^1 e^{i\lambda\theta(s)} dZ(\lambda) \cdot \overline{\int_{-1}^1 e^{i\mu\theta(t)} dZ(\mu)}\right]\\
&= \int_{-1}^1 e^{i\lambda(\theta(s)-\theta(t))} F(d\lambda)
\end{align}
\end{proof}

\begin{theorem}[Non-Stationarity Characterization]
An injectively time-changed stationary process $X(t)$ is stationary if and only if $\theta(t) = t + c$ for some constant $c \in \mathbb{R}$.
\end{theorem}

\begin{proof}
($\Leftarrow$) If $\theta(t) = t + c$, then
\begin{equation}
\text{Cov}(X(s),X(t)) = \int_{-1}^1 e^{i\lambda(s+c-t-c)} F(d\lambda) = \int_{-1}^1 e^{i\lambda(s-t)} F(d\lambda)
\end{equation}
which depends only on $s-t$, so $X(t)$ is stationary.

($\Rightarrow$) Suppose $X(t)$ is stationary. Then $\text{Cov}(X(s),X(t))$ depends only on $t-s$. This requires $\theta(t)-\theta(s) = g(t-s)$ for some function $g$. Differentiating with respect to $t$: $\theta'(t) = g'(t-s)$. Since the left side depends only on $t$, $\theta'(t)$ must be constant, so $\theta(t) = at + b$. For stationarity, we need $a = 1$.
\end{proof}

\section{Time-Warping Analysis}

\begin{definition}
The \emph{instantaneous frequency modulation} at time $t$ and frequency $\lambda$ is
\begin{equation}
\omega(t,\lambda) = \lambda\theta'(t)
\end{equation}
\end{definition}

\begin{proposition}[Frequency Modulation Properties]
Let $X(t)$ be an injectively time-changed stationary process. Then:
\begin{enumerate}
\item The phase function is $\Phi(t,\lambda) = \lambda\theta(t)$
\item The instantaneous frequency is $\frac{\partial\Phi}{\partial t} = \lambda\theta'(t)$
\item When $\theta'(t) > 1$, frequencies are compressed; when $\theta'(t) < 1$, frequencies are stretched
\end{enumerate}
\end{proposition}

\begin{proof}
(1) and (2) are immediate from the definition. For (3), the rate of phase evolution $\lambda\theta'(t)$ compared to the baseline rate $\lambda$ determines frequency scaling.
\end{proof}

\section{Inversion Theory}

\begin{theorem}[Warping Inversion]
Let $X(t)$ be an injectively time-changed stationary process with $\theta$ strictly increasing. Define the inverse process
\begin{equation}
Y(s) = X(\theta^{-1}(s))
\end{equation}
Then $Y(s)$ has the spectral representation
\begin{equation}
Y(s) = \int_{-1}^1 e^{i\lambda s} \, dZ(\lambda)
\end{equation}
and is stationary.
\end{theorem}

\begin{proof}
Since $\theta$ is strictly increasing, $\theta^{-1}$ exists. Substituting $t = \theta^{-1}(s)$:
\begin{align}
Y(s) &= X(\theta^{-1}(s))\\
&= \int_{-1}^1 e^{i\lambda\theta(\theta^{-1}(s))} dZ(\lambda)\\
&= \int_{-1}^1 e^{i\lambda s} dZ(\lambda)
\end{align}
This is the standard stationary representation with covariance
\begin{equation}
\text{Cov}(Y(s_1),Y(s_2)) = \int_{-1}^1 e^{i\lambda(s_1-s_2)} F(d\lambda)
\end{equation}
depending only on $s_1-s_2$.
\end{proof}

\section{Band-Limited Structure}

\begin{theorem}[Oscillatory Character]
An injectively time-changed stationary process $X(t)$ with spectral support in $[-1,1]$ exhibits oscillatory behavior with time-varying instantaneous frequency
\begin{equation}
\omega_{\text{inst}}(t) = \int_{-1}^1 \lambda\theta'(t) \frac{|dZ(\lambda)|^2}{\int_{-1}^1 |dZ(\mu)|^2}
\end{equation}
\end{theorem}

\begin{proof}
The instantaneous frequency is the weighted average of component frequencies $\lambda\theta'(t)$, weighted by spectral power $|dZ(\lambda)|^2$.
\end{proof}

\begin{corollary}[Priestley Oscillatory Process]
When the spectral measure $F$ is concentrated away from $\lambda = 0$ and $\theta'(t) > 0$, the process $X(t)$ is oscillatory in Priestley's sense with time-varying carrier frequency.
\end{corollary}

\section{Reconstruction and Synthesis}

\begin{theorem}[Spectral Synthesis]
Given a strictly increasing warping function $\theta(t)$ and spectral density $f(\lambda)$ supported on $[-1,1]$, the process
\begin{equation}
X(t) = \int_{-1}^1 f(\lambda) e^{i\lambda\theta(t)} d\lambda
\end{equation}
uniquely determines an injectively time-changed stationary process.
\end{theorem}

\begin{proof}
The integral converges since $f \in L^2([-1,1])$ and $|e^{i\lambda\theta(t)}| = 1$. The covariance structure follows from Theorem 1, and injectivity of $\theta$ ensures the process is well-defined and non-degenerate.
\end{proof}

\section{Conclusion}

Injectively time-changed stationary processes provide a natural framework for analyzing non-stationary oscillatory phenomena while preserving spectral interpretability. The warping function $\theta(t)$ serves as the fundamental mechanism for temporal modulation, with the gain function $A(t,\lambda) = e^{i\lambda(\theta(t)-t)}$ encoding the deviation from stationary evolution.

\end{document}
