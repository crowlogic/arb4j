\documentclass{article}
\usepackage{amsmath,amsthm,amssymb}
\usepackage{bbm}

\newtheorem{theorem}{Theorem}
\newtheorem{definition}{Definition}
\newtheorem{lemma}{Lemma}
\newtheorem{corollary}{Corollary}
\newtheorem{remark}{Remark}

\begin{document}

\title{Detailed Analysis: Harmonizable Representation and Evolutionary Spectrum\\of Monotonically Modulated Stationary Processes}
\author{}
\date{}
\maketitle

\begin{definition}[Harmonizable Process]
A stochastic process $\{X_t, t \in \mathbb{R}\}$ is \textit{harmonizable} if it admits the representation:
$$X_t = \int_{\mathbb{R}} e^{i\lambda t} dZ(\lambda)$$
where $dZ$ is a complex-valued random measure with bounded variation, not necessarily having orthogonal increments. The correlation structure is given by:
$$\mathbb{E}[dZ(\lambda)d\overline{Z(\mu)}] = F(d\lambda, d\mu)$$
where $F$ is a measure on $\mathbb{R}^2$ of bounded variation.
\end{definition}

\begin{definition}[Projection Operator for Time-Modulated Processes]
Let $\{Y_{(t,\tau)}\}$ be a stochastic process defined on $\mathbb{R}^2$ and $\theta: \mathbb{R} \to \mathbb{R}$ be a monotonically increasing function. The \textit{projection operator} $P_\theta$ is defined as:
$$(P_\theta Y)_t = Y_{(t,\theta(t))}$$
for all $t \in \mathbb{R}$. This operator projects from the space of processes on $\mathbb{R}^2$ to the space of processes on $\mathbb{R}$ by restricting to the curve $\{(t,\theta(t)): t \in \mathbb{R}\}$.

The projection operator $P_\theta$ satisfies:
\begin{enumerate}
\item $P_\theta^2 = P_\theta$ (idempotent): 
$$(P_\theta^2 Y)_t = (P_\theta(P_\theta Y))_t = P_\theta(Y_{(\cdot,\theta(\cdot))})_t = Y_{(t,\theta(t))} = (P_\theta Y)_t$$
\item $P_\theta^* = P_\theta$ (self-adjoint): If $\langle \cdot, \cdot \rangle$ denotes the inner product in the appropriate Hilbert space, then $\langle P_\theta Y, Z \rangle = \langle Y, P_\theta Z \rangle$
\end{enumerate}
\end{definition}

\begin{definition}[Evolutionary Spectrum]
A non-stationary process $\{X_t, t \in \mathbb{R}\}$ has an evolutionary spectral representation if:
$$X_t = \int_{\mathbb{R}} A_t(\lambda)e^{i\lambda t}dZ(\lambda)$$
where:
\begin{itemize}
\item $dZ(\lambda)$ is an orthogonal increment process with $\mathbb{E}|dZ(\lambda)|^2 = d\lambda$
\item $A_t(\lambda)$ is a time-varying amplitude function
\item The evolutionary spectral density is $h_t(\lambda) = |A_t(\lambda)|^2$
\end{itemize}
\end{definition}

\begin{definition}[Monotonically Modulated Process]
Let $X_0(t)$ be a stationary process with kernel $K_0(t-s)$. A monotonically modulated process is defined as:
$$X_t = X_0(\theta(t))$$
where $\theta: \mathbb{R} \to \mathbb{R}$ is a monotonically increasing function, yielding the kernel:
$$K(t,s) = K_0(\theta(t) - \theta(s))$$
\end{definition}

\begin{theorem}[Harmonizable Structure of Modulated Processes]
The monotonically modulated process $X_t = X_0(\theta(t))$ is a harmonizable process with spectral representation:
$$X_t = \int_{\mathbb{R}} e^{i\lambda\theta(t)} dZ_0(\lambda)$$
where $dZ_0$ is the spectral measure of the original stationary process $X_0$.
\end{theorem}

\begin{proof}
\textbf{Step 1:} By Cramér's representation theorem, the stationary process $X_0(t)$ has representation:
$$X_0(t) = \int_{\mathbb{R}} e^{i\lambda t}dZ_0(\lambda)$$
where $dZ_0$ has orthogonal increments with $\mathbb{E}[dZ_0(\lambda)d\overline{Z_0(\mu)}] = \delta(\lambda-\mu)f_0(\lambda)d\lambda d\mu$.

\textbf{Step 2:} For any fixed time point $u \in \mathbb{R}$, we have:
$$X_0(u) = \int_{\mathbb{R}} e^{i\lambda u}dZ_0(\lambda)$$

\textbf{Step 3:} Setting $u = \theta(t)$ specifically, we get:
$$X_0(\theta(t)) = \int_{\mathbb{R}} e^{i\lambda\theta(t)}dZ_0(\lambda)$$

\textbf{Step 4:} By definition of the modulated process $X_t = X_0(\theta(t))$, we have:
$$X_t = \int_{\mathbb{R}} e^{i\lambda\theta(t)}dZ_0(\lambda)$$

\textbf{Step 5:} The covariance function is directly calculated:
\begin{align*}
K(t,s) &= \mathbb{E}[X_t\overline{X_s}]\\
&= \mathbb{E}\left[\int_{\mathbb{R}} e^{i\lambda\theta(t)}dZ_0(\lambda) \overline{\int_{\mathbb{R}} e^{i\mu\theta(s)}dZ_0(\mu)}\right]\\
&= \mathbb{E}\left[\int_{\mathbb{R}}\int_{\mathbb{R}} e^{i\lambda\theta(t)}e^{-i\mu\theta(s)}dZ_0(\lambda)d\overline{Z_0(\mu)}\right]\\
&= \int_{\mathbb{R}}\int_{\mathbb{R}} e^{i\lambda\theta(t)}e^{-i\mu\theta(s)}\mathbb{E}[dZ_0(\lambda)d\overline{Z_0(\mu)}]\\
&= \int_{\mathbb{R}}\int_{\mathbb{R}} e^{i\lambda\theta(t)}e^{-i\mu\theta(s)}\delta(\lambda-\mu)f_0(\lambda)d\lambda d\mu\\
&= \int_{\mathbb{R}} e^{i\lambda\theta(t)}e^{-i\lambda\theta(s)}f_0(\lambda)d\lambda\\
&= \int_{\mathbb{R}} e^{i\lambda(\theta(t)-\theta(s))}f_0(\lambda)d\lambda\\
&= K_0(\theta(t)-\theta(s))
\end{align*}
Thus, $X_t$ is harmonizable with the specified covariance structure.
\end{proof}

\begin{theorem}[Evolutionary Spectral Representation]
The harmonizable process $X_t = X_0(\theta(t))$ has an exact evolutionary spectral representation:
$$X_t = \int_{\mathbb{R}} A_t(\lambda)e^{i\lambda t}dZ_0(\lambda)$$
where $A_t(\lambda) = e^{i\lambda(\theta(t)-t)}$ is the time-varying amplitude function.
\end{theorem}

\begin{proof}
\textbf{Step 1:} Starting from the harmonizable representation:
$$X_t = \int_{\mathbb{R}} e^{i\lambda\theta(t)}dZ_0(\lambda)$$

\textbf{Step 2:} We perform exact algebraic manipulation of the complex exponential term:
\begin{align*}
e^{i\lambda\theta(t)} &= e^{i\lambda\theta(t)} \cdot \frac{e^{i\lambda t}}{e^{i\lambda t}}\\
&= e^{i\lambda t} \cdot e^{i\lambda\theta(t) - i\lambda t}\\
&= e^{i\lambda t} \cdot e^{i\lambda(\theta(t) - t)}
\end{align*}

\textbf{Step 3:} Substituting this factorization back:
\begin{align*}
X_t &= \int_{\mathbb{R}} e^{i\lambda\theta(t)}dZ_0(\lambda)\\
&= \int_{\mathbb{R}} e^{i\lambda t} \cdot e^{i\lambda(\theta(t) - t)}dZ_0(\lambda)
\end{align*}

\textbf{Step 4:} Define the time-varying amplitude function:
$$A_t(\lambda) = e^{i\lambda(\theta(t) - t)}$$

\textbf{Step 5:} This gives us the evolutionary spectral representation:
$$X_t = \int_{\mathbb{R}} A_t(\lambda)e^{i\lambda t}dZ_0(\lambda)$$

\textbf{Step 6:} The evolutionary spectral density is:
\begin{align*}
h_t(\lambda) &= |A_t(\lambda)|^2 \cdot f_0(\lambda)\\
&= |e^{i\lambda(\theta(t) - t)}|^2 \cdot f_0(\lambda)\\
&= 1 \cdot f_0(\lambda)\\
&= f_0(\lambda)
\end{align*}
where we used the fact that $|e^{ix}|^2 = 1$ for any real $x$.

\textbf{Step 7:} Note that while the evolutionary spectral density equals the original spectral density, the phase information in $A_t(\lambda)$ captures the non-stationarity introduced by the time-transformation $\theta(t)$.
\end{proof}

\begin{theorem}[Stationary Dilation via Naimark's Theorem]
The harmonizable process $X_t = X_0(\theta(t))$ admits a stationary dilation $Y_{(t,\tau)}$ in an expanded space:
$$Y_{(t,\tau)} = \int_{\mathbb{R}} e^{i\lambda\tau}dZ_0(\lambda)$$

The original harmonizable process is recovered via the projection operator $P_\theta$:
$$X_t = (P_\theta Y)_t = Y_{(t,\theta(t))}$$
\end{theorem}

\begin{proof}
\textbf{Step 1:} We construct the stationary dilation:
$$Y_{(t,\tau)} = \int_{\mathbb{R}} e^{i\lambda\tau}dZ_0(\lambda)$$

\textbf{Step 2:} This process is stationary in the parameter $\tau$ as shown by its covariance:
\begin{align*}
\tilde{K}((t,\tau),(s,\sigma)) &= \mathbb{E}[Y_{(t,\tau)}\overline{Y_{(s,\sigma)}}]\\
&= \mathbb{E}\left[\int_{\mathbb{R}} e^{i\lambda\tau}dZ_0(\lambda) \overline{\int_{\mathbb{R}} e^{i\mu\sigma}dZ_0(\mu)}\right]\\
&= \mathbb{E}\left[\int_{\mathbb{R}}\int_{\mathbb{R}} e^{i\lambda\tau}e^{-i\mu\sigma}dZ_0(\lambda)d\overline{Z_0(\mu)}\right]\\
&= \int_{\mathbb{R}}\int_{\mathbb{R}} e^{i\lambda\tau}e^{-i\mu\sigma}\mathbb{E}[dZ_0(\lambda)d\overline{Z_0(\mu)}]\\
&= \int_{\mathbb{R}}\int_{\mathbb{R}} e^{i\lambda\tau}e^{-i\mu\sigma}\delta(\lambda-\mu)f_0(\lambda)d\lambda d\mu\\
&= \int_{\mathbb{R}} e^{i\lambda\tau}e^{-i\lambda\sigma}f_0(\lambda)d\lambda\\
&= \int_{\mathbb{R}} e^{i\lambda(\tau-\sigma)}f_0(\lambda)d\lambda\\
&= K_0(\tau-\sigma)
\end{align*}
The covariance depends only on $\tau-\sigma$, confirming stationarity.

\textbf{Step 3:} Apply the projection operator $P_\theta$ defined earlier:
\begin{align*}
(P_\theta Y)_t &= Y_{(t,\theta(t))}\\
&= \int_{\mathbb{R}} e^{i\lambda\theta(t)}dZ_0(\lambda)\\
&= X_t
\end{align*}

\textbf{Step 4:} Verify that $P_\theta$ is idempotent (already established in the definition):
\begin{align*}
(P_\theta^2 Y)_t &= (P_\theta(P_\theta Y))_t\\
&= P_\theta(Y_{(\cdot,\theta(\cdot))})_t\\
&= Y_{(t,\theta(t))}\\
&= (P_\theta Y)_t
\end{align*}

\textbf{Step 5:} This confirms that $Y_{(t,\tau)}$ is the stationary dilation of $X_t$, and the original process is precisely the projection of this stationary process via the projection operator $P_\theta$.
\end{proof}

\begin{corollary}[Complete Characterization]
For a monotonically modulated process $X_t = X_0(\theta(t))$:
\begin{enumerate}
\item It is harmonizable with representation $X_t = \int_{\mathbb{R}} e^{i\lambda\theta(t)} dZ_0(\lambda)$
\item It has evolutionary spectral representation $X_t = \int_{\mathbb{R}} e^{i\lambda(\theta(t)-t)}e^{i\lambda t}dZ_0(\lambda)$
\item It is the projection of a stationary process $Y_{(t,\tau)} = \int_{\mathbb{R}} e^{i\lambda\tau}dZ_0(\lambda)$ via $X_t = (P_\theta Y)_t = Y_{(t,\theta(t))}$
\item Its kernel $K(t,s) = K_0(\theta(t)-\theta(s))$ maintains positive definiteness from the original process
\end{enumerate}
\end{corollary}

\end{document}
