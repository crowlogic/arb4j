\documentclass[12pt]{article}

\usepackage{amsmath,amsthm,amssymb,amsfonts}
\usepackage{mathtools}
\usepackage{mathrsfs}
\usepackage[T1]{fontenc}
\usepackage[utf8]{inputenc}
\usepackage{microtype}
\usepackage{geometry}
\geometry{margin=1in}

% Environments
\newtheorem{theorem}{Theorem}
\newtheorem{lemma}{Lemma}
\newtheorem{claim}{Claim}
\newtheorem{proposition}{Proposition}
\newtheorem{corollary}{Corollary}
\theoremstyle{definition}
\newtheorem{definition}{Definition}
\theoremstyle{remark}
\newtheorem{remark}{Remark}

% Notation helpers
\DeclareMathOperator{\supp}{supp}
\DeclareMathOperator{\Var}{Var}
\DeclareMathOperator{\Cov}{Cov}
\DeclareMathOperator{\sgn}{sgn}

\title{Unitary Time Changes of Stationary Processes Yield Oscillatory Processes\\
and a Functional Framework Toward a Hilbert--P\'olya Construction}
\author{Stephen Crowley}
\date{\today}

\begin{document}
\maketitle

\section{Unitary Time Change on $L^2(\mathbb{R})$}

\begin{definition}[Unitary time change operator on $L^2(\mathbb{R})$]
Let $\theta:\mathbb{R}\to\mathbb{R}$ be absolutely continuous with $\theta'(t)\neq 0$ almost everywhere.
Define $U_\theta:L^2(\mathbb{R})\to L^2(\mathbb{R})$ by
\begin{equation}
  (U_\theta f)(t)\coloneqq \sqrt{|\theta'(t)|}\, f(\theta(t))\qquad (f\in L^2(\mathbb{R})).
\end{equation}
\end{definition}

\begin{theorem}[Unitarity of $U_\theta$]
$U_\theta$ is unitary on $L^2(\mathbb{R})$.
\end{theorem}

\begin{proof}
By absolute continuity and $\theta'(t)\neq 0$ a.e., the change-of-variables formula gives
\[
\int_{\mathbb{R}} |(U_\theta f)(t)|^2\,dt
= \int_{\mathbb{R}} |\theta'(t)|\,|f(\theta(t))|^2\,dt
= \int_{\mathbb{R}} |f(u)|^2\,du,
\]
so $U_\theta$ is an isometry. Since $\theta$ admits an a.e.\ inverse $\theta^{-1}$ with the same regularity and nonvanishing derivative a.e., one has $U_{\theta^{-1}}U_\theta=\mathrm{Id}$ and $U_\theta U_{\theta^{-1}}=\mathrm{Id}$ a.e., hence $U_\theta$ is unitary.
\end{proof}

\section{Oscillatory Processes in the Sense of Priestley}

\begin{definition}[Oscillatory process, gain and oscillatory function]
Let $F$ be a finite nonnegative Borel measure on $\mathbb{R}$. For each $t\in\mathbb{R}$ let $A_t:\mathbb{R}\to\mathbb{C}$ be measurable and square-integrable with respect to $F$.
Define
\[
\varphi_t(\lambda)\coloneqq A_t(\lambda)\,e^{i\lambda t}.
\]
An \emph{oscillatory process} $Z$ is a stochastic process with spectral representation
\begin{equation}
  Z(t) \coloneqq \int_{\mathbb{R}} \varphi_t(\lambda)\,\Phi(d\lambda)
  = \int_{\mathbb{R}} A_t(\lambda)\,e^{i\lambda t}\,\Phi(d\lambda),
\end{equation}
where $\Phi$ is a complex orthogonal random measure with spectral measure $F$ satisfying the orthogonality of infinitesimal increments
\begin{equation}
  \mathbb{E}\!\left[\Phi(d\lambda)\,\overline{\Phi(d\mu)}\right] = \delta(\lambda-\mu)\,dF(\lambda).
\end{equation}
The covariance kernel is
\begin{equation}
  R_Z(t,s)\coloneqq \mathbb{E}\big[Z(t)\,\overline{Z(s)}\big]
  = \int_{\mathbb{R}} A_t(\lambda)\,\overline{A_s(\lambda)}\,e^{i\lambda(t-s)}\,dF(\lambda).
\end{equation}
\end{definition}

\begin{remark}[Real-valuedness]
$Z$ is real-valued if and only if, for each fixed $t$, $A_t(-\lambda)=\overline{A_t(\lambda)}$ for $F$-a.e.\ $\lambda$, equivalently $\varphi_t(-\lambda)=\overline{\varphi_t(\lambda)}$ for $F$-a.e.\ $\lambda$.
\end{remark}

\begin{theorem}[Existence of oscillatory processes with prescribed $(A_t)_t$]
Let $F$ be finite and $(A_t)_t$ measurable with $\int |A_t(\lambda)|^2\,dF(\lambda)<\infty$ for each $t$. There exists a complex orthogonal random measure $\Phi$ on $\mathbb{R}$ with spectral measure $F$ such that $Z(t)=\int \varphi_t(\lambda)\,\Phi(d\lambda)$ is well-defined in $L^2(\Omega)$ and has covariance
\[
R_Z(t,s)=\int_{\mathbb{R}} \varphi_t(\lambda)\,\overline{\varphi_s(\lambda)}\,dF(\lambda)
=\int_{\mathbb{R}} A_t(\lambda)\,\overline{A_s(\lambda)}\,e^{i\lambda(t-s)}\,dF(\lambda).
\]
\end{theorem}

\begin{proof}
Construct the stochastic integral first for simple functions in $L^2(\mathbb{R},F)$ and extend by isometry using $\mathbb{E}\big[\big|\int g(\lambda)\,\Phi(d\lambda)\big|^2\big]=\int |g(\lambda)|^2\,dF(\lambda)$. Apply with $g=\varphi_t$ to obtain $Z(t)$ and the stated covariance.
\end{proof}

\section{Unitary Time Changes Map Stationary to Oscillatory}

\begin{definition}[Stationary process via Cram\'er representation]
A zero-mean stationary process $X$ with spectral measure $F$ admits
\begin{equation}
  X(t) = \int_{\mathbb{R}} e^{i\lambda t}\,\Phi(d\lambda),
\end{equation}
with covariance
\begin{equation}
  R_X(t-s)=\int_{\mathbb{R}} e^{i\lambda(t-s)}\,dF(\lambda).
\end{equation}
\end{definition}

\begin{theorem}[Unitary time change yields an oscillatory process]
Let $X$ be zero-mean stationary with
\[
X(t)=\int_{\mathbb{R}} e^{i\lambda t}\,\Phi(d\lambda).
\]
Let $\theta$ satisfy the hypotheses of the unitary time change and set
\begin{equation}
  Z(t)\coloneqq (U_\theta X)(t)=\sqrt{|\theta'(t)|}\,X(\theta(t)).
\end{equation}
Then $Z$ is an oscillatory process with oscillatory function
\begin{equation}
  \varphi_t(\lambda)=\sqrt{|\theta'(t)|}\,e^{i\lambda \theta(t)},
\end{equation}
and gain
\begin{equation}
  A_t(\lambda)=\sqrt{|\theta'(t)|}\,e^{i\lambda(\theta(t)-t)}.
\end{equation}
The covariance is
\begin{equation}
  R_Z(t,s)=\int_{\mathbb{R}} A_t(\lambda)\,\overline{A_s(\lambda)}\,e^{i\lambda(t-s)}\,dF(\lambda)
  = \int_{\mathbb{R}} \sqrt{|\theta'(t)\theta'(s)|}\,e^{i\lambda(\theta(t)-\theta(s))}\,dF(\lambda).
\end{equation}
\end{theorem}

\begin{proof}
Compute
\[
Z(t)=\sqrt{|\theta'(t)|}\,X(\theta(t))
=\sqrt{|\theta'(t)|}\int_{\mathbb{R}} e^{i\lambda \theta(t)}\,\Phi(d\lambda)
=\int_{\mathbb{R}} \sqrt{|\theta'(t)|}\,e^{i\lambda \theta(t)}\,\Phi(d\lambda).
\]
Thus $\varphi_t(\lambda)=\sqrt{|\theta'(t)|}\,e^{i\lambda \theta(t)}$ and $A_t(\lambda)=\varphi_t(\lambda)e^{-i\lambda t}$. The covariance follows from orthogonality of $\Phi$.
\end{proof}

\begin{remark}[Real-valuedness under time change]
If $X$ is real-valued and $\theta$ is real with $\theta'(t)>0$ a.e., then $Z$ is real-valued by the Hermitian symmetry of $A_t$.
\end{remark}

\section{Zero Localization by a Functional Measure}

\begin{definition}[Zero localization measure]
Let $Z$ be real-valued, with sample paths in $C^1(\mathbb{R})$, and such that every zero of $Z$ is simple (i.e.\ $Z(t_0)=0\implies Z'(t_0)\neq 0$). Define the measure on Borel $B\subset\mathbb{R}$ by
\begin{equation}
  \mu(B)\coloneqq \int_{\mathbb{R}} \mathbf{1}_B(t)\,\delta(Z(t))\,|Z'(t)|\,dt.
\end{equation}
\end{definition}

\begin{theorem}[Support and mass on the zero set]
For any test function $\phi\in C_c^\infty(\mathbb{R})$,
\begin{equation}
  \int_{\mathbb{R}} \phi(t)\,\delta(Z(t))\,|Z'(t)|\,dt
  = \sum_{t_0: Z(t_0)=0} \phi(t_0),
\end{equation}
and hence $\mu=\sum_{t_0: Z(t_0)=0} \delta_{t_0}$ is a discrete measure assigning unit mass to each simple zero of $Z$.
\end{theorem}

\begin{proof}
At a simple zero $t_0$, the distributional identity holds:
\[
\delta(Z(t))=\frac{\delta(t-t_0)}{|Z'(t_0)|}+\sum_{t_1\neq t_0: Z(t_1)=0}\frac{\delta(t-t_1)}{|Z'(t_1)|}.
\]
Multiplying by $|Z'(t)|$ and integrating against $\phi$ yields the stated identity and the atomic form of $\mu$.
\end{proof}

\section{Hilbert Space on the Zero Set and Multiplication Operator}

\begin{definition}[Hilbert space on the zero set via $\mu$]
Define
\[
\mathcal{H}\coloneqq L^2(\mu)
=\Big\{ f:\mathbb{R}\to\mathbb{C}\,:\, \|f\|_{\mathcal{H}}^2=\int |f(t)|^2\,\delta(Z(t))\,|Z'(t)|\,dt<\infty\Big\}.
\]
The inner product is $\langle f,g\rangle=\int f(t)\overline{g(t)}\,\delta(Z(t))\,|Z'(t)|\,dt$.
\end{definition}

\begin{proposition}[Atomic structure]
With $\mu=\sum_{t_0: Z(t_0)=0}\delta_{t_0}$, one has
\[
\mathcal{H}=\Big\{ f:\{t_0: Z(t_0)=0\}\to\mathbb{C}\,:\, \sum_{Z(t_0)=0} |f(t_0)|^2<\infty\Big\}\cong \ell^2,
\]
and the functions $e_{t_0}$ defined by $e_{t_0}(t_1)=\delta_{t_0t_1}$ form an orthonormal basis.
\end{proposition}

\begin{proof}
Substitute the atomic form of $\mu$ into the $L^2$-definition to obtain the $\ell^2$-structure; the canonical coordinate functions form an ONB.
\end{proof}

\begin{definition}[Multiplication operator]
Define $L:\mathcal{D}(L)\subset\mathcal{H}\to\mathcal{H}$ by $(Lf)(t)=t\,f(t)$ on $\supp(\mu)$, with
\[
\mathcal{D}(L)=\Big\{ f\in\mathcal{H}\,:\, \int |t\,f(t)|^2\,\delta(Z(t))\,|Z'(t)|\,dt<\infty\Big\}.
\]
\end{definition}

\begin{theorem}[Self-adjointness and spectrum]
$L$ is self-adjoint on $\mathcal{H}$, and its spectrum is exactly
\[
\sigma(L)=\{\, t\in\mathbb{R}\,:\, Z(t)=0 \,\},
\]
with pure point spectrum consisting of simple eigenvalues $\lambda=t_0$ (for each zero $t_0$) and eigenvectors $e_{t_0}$.
\end{theorem}

\begin{proof}
For $f,g\in\mathcal{D}(L)$,
\[
\langle Lf,g\rangle = \int t\,f(t)\,\overline{g(t)}\,\delta(Z(t))\,|Z'(t)|\,dt
= \int f(t)\,\overline{t\,g(t)}\,\delta(Z(t))\,|Z'(t)|\,dt = \langle f,Lg\rangle,
\]
so $L$ is symmetric. On the atomic space, $L$ is unitarily equivalent to the diagonal operator $(c_{t_0})\mapsto (t_0 c_{t_0})$ on $\ell^2$, which is self-adjoint with spectrum equal to the set of diagonal entries $\{t_0: Z(t_0)=0\}$, each simple, with eigenvectors the coordinate basis identified with $e_{t_0}$.
\end{proof}

\section{Time-Changed Stationary Processes and a Hilbert--P\'olya Scaffold}

\begin{definition}[Arithmetic phase time change]
Let $\theta:\mathbb{R}\to\mathbb{R}$ be an absolutely continuous phase with $\theta'(t)>0$ a.e.\ encoding the target arithmetic structure (e.g.\ a Riemann--Siegel-type phase). Let $X$ be zero-mean stationary with spectral measure $F$ and orthogonal random measure $\Phi$. Define the time-changed oscillatory process
\[
Z(t)=\int_{\mathbb{R}} \sqrt{|\theta'(t)|}\,e^{i\lambda \theta(t)}\,\Phi(d\lambda).
\]
\end{definition}

\begin{proposition}[Covariance under time change]
\[
R_Z(t,s)=\int_{\mathbb{R}} \sqrt{|\theta'(t)\theta'(s)|}\,e^{i\lambda(\theta(t)-\theta(s))}\,dF(\lambda).
\]
In particular, if $F$ is chosen so that $R_Z$ concentrates along $\theta(t)=\theta(s)$, then the correlation structure of $Z$ is phase-aligned with $\theta$.
\end{proposition}

\begin{proof}
Insert the oscillatory function into the covariance integral and use the orthogonality relation for $\Phi$.
\end{proof}

\begin{definition}[Zero-localized Hilbert space and operator]
With the zero localization measure $\mu(dt)=\delta(Z(t))\,|Z'(t)|\,dt$, define $\mathcal{H}=L^2(\mu)$ and $L$ as multiplication by $t$ on $\mathcal{H}$.
\end{definition}

\begin{theorem}[Spectral encoding of zero set]
The spectrum of $L$ is the zero set of $Z$:
\[
\sigma(L)=\{t: Z(t)=0\},
\]
and $L$ has simple pure point spectrum with eigenvectors supported at individual zeros.
\end{theorem}

\begin{proof}
Follows from the established atomic structure of $\mu$ and the diagonal form of $L$ on $L^2(\mu)$.
\end{proof}

\begin{remark}[Operator scaffold]
The sequence
\[
\text{stationary }X \xrightarrow{\,U_\theta\,} \text{oscillatory }Z \xrightarrow{\,\delta(Z)\,|Z'|\,dt\,} \mu \xrightarrow{\,L^2(\mu)\,} \mathcal{H} \xrightarrow{\,t\cdot\,} L
\]
produces a concrete self-adjoint operator whose spectrum equals the (constructed) zero set governed by the choice of $\theta$ and $F$. Aligning $\theta$ and $F$ to a prescribed arithmetic target sets the stage for a Hilbert--P\'olya-type identification.
\end{remark}

\section{Appendix: Regularity and Simple Zeros}

\begin{definition}[Regularity and simplicity]
Assume $Z\in C^1(\mathbb{R})$ and every zero of $Z$ is simple: $Z(t_0)=0\implies Z'(t_0)\neq 0$.
\end{definition}

\begin{lemma}[Local finiteness and decomposition]
Under the above condition, zeros are locally finite and the distributional identity
\[
\delta(Z(t))=\sum_{t_0: Z(t_0)=0}\frac{\delta(t-t_0)}{|Z'(t_0)|}
\]
holds, yielding $\mu=\sum_{t_0}\delta_{t_0}$.
\end{lemma}

\begin{proof}
Continuity and $Z'(t_0)\neq 0$ imply isolated zeros by the inverse function theorem; the distributional identity is standard from the one-dimensional change-of-variables formula for the Dirac delta under monotone $C^1$ maps near each zero.
\end{proof}

\end{document}
