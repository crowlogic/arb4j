\documentclass{article}
\usepackage[english]{babel}
\usepackage{geometry,amssymb,theorem}
\geometry{letterpaper}

%%%%%%%%%% Start TeXmacs macros
\newtheorem{corollary}{Corollary}
\newtheorem{definition}{Definition}
\newtheorem{proposition}{Proposition}
{\theorembodyfont{\rmfamily}\newtheorem{remark}{Remark}}
\newtheorem{theorem}{Theorem}
%%%%%%%%%% End TeXmacs macros

\begin{document}

\section{Stationary Dilations}

\begin{definition}
  Let $(\Omega, \mathcal{F}, P)$ and $(\tilde{\Omega}, \tilde{\mathcal{F}},
  \tilde{P})$ be probability spaces. We say that $(\Omega, \mathcal{F}, P)$ is
  a factor of $(\tilde{\Omega}, \tilde{\mathcal{F}}, \tilde{P})$ if there
  exists a measurable surjective map $\phi : \tilde{\Omega} \to \Omega$ such
  that:
  \begin{enumerate}
    \item For all $A \in \mathcal{F}$, $\phi^{- 1} (A) \in
    \tilde{\mathcal{F}}$
    
    \item For all $A \in \mathcal{F}$, $P (A) = \tilde{P} (\phi^{- 1} (A))$
  \end{enumerate}
  In other words, $(\Omega, \mathcal{F}, P)$ can be obtained from
  $(\tilde{\Omega}, \tilde{\mathcal{F}}, \tilde{P})$ by projecting the larger
  space onto the smaller one while preserving the probability measure
  structure.
\end{definition}

\begin{remark}
  In the context of stationary dilations, this means that the original
  nonstationary process $\{X_t \}$ can be recovered from the stationary
  dilation $\{Y_t \}$ through a measurable projection that preserves the
  probabilistic structure of the original process.
\end{remark}

\begin{definition}[Stationary Dilation]
  Let $(\Omega, \mathcal{F}, P)$ be a probability space and let $\{X_t \}_{t
  \in \mathbb{R}_+}$ be a nonstationary stochastic process. A stationary
  dilation of $\{X_t \}$ is a stationary process $\{Y_t \}_{t \in
  \mathbb{R}_+}$ defined on a larger probability space $(\tilde{\Omega},
  \tilde{\mathcal{F}}, \tilde{P})$ such that:
  \begin{enumerate}
    \item $(\Omega, \mathcal{F}, P)$ is a factor of $(\tilde{\Omega},
    \tilde{\mathcal{F}}, \tilde{P})$
    
    \item There exists a measurable projection operator $\Pi$ such that:
    \begin{equation}
      X_t = \Pi Y_t \quad \forall t \in \mathbb{R}_+
    \end{equation}
  \end{enumerate}
\end{definition}

\begin{theorem}[Representation of Nonstationary Processes]
  For a continuous-time nonstationary process $\{X_t \}_{t \in \mathbb{R}_+}$,
  its stationary dilation exists which has sample paths $t \mapsto X_t
  (\omega)$ which are continuous with probability one when $X_t$:
  \begin{itemize}
    \item is uniformly continuous in probability over compact intervals:
    \begin{equation}
      \lim_{s \to t} P (|X_s - X_t | > \epsilon) = 0 \quad \forall \epsilon >
      0, t \in [0, T], T > 0
    \end{equation}
    \item has finite second moments:
    \begin{equation}
      \mathbb{E} [|X_t |^2] < \infty \quad \forall t \in \mathbb{R}_+
    \end{equation}
    \item has an integral representation of the form:
    \begin{equation}
      X_t = \int_0^t \eta (s) ds
    \end{equation}
    where $\eta (t)$ is a measurable random function that is stationary in the
    wide sense (with $\int_0^t \mathbb{E} [| \eta (s) |^2]  \hspace{0.17em} ds
    < \infty$ for all $t$)
    
    \item and has a covariance operator
    \begin{equation}
      R (t, s) =\mathbb{E} [X_t X_s]
    \end{equation}
    which is symmetric $(R (t, s) = R (s, t))$, positive definite and
    continuous
  \end{itemize}
  Under these conditions, there exists a representation:
  \begin{equation}
    X_t = M (t) \cdot S_t
  \end{equation}
  where:
  \begin{itemize}
    \item $M (t)$ is a continuous deterministic modulation function
    
    \item $\{S_t \}_{t \in \mathbb{R}_+}$ is a stationary process
  \end{itemize}
  This representation can be obtained through the stationary dilation by
  choosing:
  \begin{equation}
    Y_t = \left( \begin{array}{c}
      M (t)\\
      S_t
    \end{array} \right)
  \end{equation}
  with the projection operator $\Pi$ defined as:
  \begin{equation}
    \Pi Y_t = M (t) \cdot S_t
  \end{equation}
\end{theorem}

\begin{proposition}[Properties of Dilation]
  The stationary dilation satisfies:
  \begin{enumerate}
    \item Preservation of moments:
    \begin{equation}
      \mathbb{E} [|X_t |^p] \leq \mathbb{E} [|Y_t |^p] \quad \forall p \geq 1
    \end{equation}
    \item Minimal extension: Among all stationary processes that dilate $X_t$,
    there exists a minimal one (unique up to isomorphism) in terms of the
    probability space dimension
  \end{enumerate}
\end{proposition}

\begin{corollary}
  For any nonstationary process satisfying the above conditions, the
  stationary dilation provides a canonical factorization into deterministic
  time-varying components and stationary stochastic components.
\end{corollary}

\end{document}
