\documentclass{article}
\usepackage{amsmath}
\usepackage{amssymb}
\usepackage{amsthm}
\usepackage{amsfonts}

\newtheorem{theorem}{Theorem}
\newtheorem{lemma}{Lemma}
\newtheorem{proposition}{Proposition}
\newtheorem{definition}{Definition}

\title{Periodization, Theta Functions, and Fundamental Solutions of the Heat Equation: Connections to Discrete Fourier Transforms and Method of Images}
\author{}
\date{}

\begin{document}

\maketitle

\begin{abstract}
This document establishes the mathematical connections between the Poisson summation formula, discrete Fourier transforms, and the method of images for solving the heat equation with boundary conditions. The analysis demonstrates how periodization serves as the unifying principle, leading to theta function representations of heat kernels on bounded domains.
\end{abstract}

\section{The Poisson Summation Formula}

\begin{definition}
For a function $f \in L^1(\mathbb{R})$ with Fourier transform $\hat{f}(\xi) = \int_{-\infty}^{\infty} f(x) e^{-2\pi i x \xi} dx$, the Poisson summation formula states
\begin{equation}
\label{eq:poisson}
\sum_{n \in \mathbb{Z}} f(n) = \sum_{k \in \mathbb{Z}} \hat{f}(k)
\end{equation}
\end{definition}

\begin{theorem}
Let $f$ be a Schwartz function on $\mathbb{R}$. Then equation \eqref{eq:poisson} holds.
\end{theorem}

\begin{proof}
Define the periodized function
\begin{equation}
\label{eq:periodized}
F(x) = \sum_{n \in \mathbb{Z}} f(x + n)
\end{equation}

Since $f$ is Schwartz, the series converges uniformly and $F$ is periodic with period 1. The Fourier series expansion of $F$ is
\begin{equation}
\label{eq:fourier_series}
F(x) = \sum_{k \in \mathbb{Z}} c_k e^{2\pi i k x}
\end{equation}
where the Fourier coefficients are
\begin{align}
c_k &= \int_0^1 F(x) e^{-2\pi i k x} dx \label{eq:coeff_def}\\
&= \int_0^1 \sum_{n \in \mathbb{Z}} f(x + n) e^{-2\pi i k x} dx \label{eq:coeff_expand}\\
&= \sum_{n \in \mathbb{Z}} \int_0^1 f(x + n) e^{-2\pi i k x} dx \label{eq:interchange}
\end{align}

Substituting $u = x + n$ in each integral:
\begin{align}
c_k &= \sum_{n \in \mathbb{Z}} \int_n^{n+1} f(u) e^{-2\pi i k (u-n)} du \label{eq:substitution}\\
&= \sum_{n \in \mathbb{Z}} e^{2\pi i k n} \int_n^{n+1} f(u) e^{-2\pi i k u} du \label{eq:factor_out}\\
&= \sum_{n \in \mathbb{Z}} \int_n^{n+1} f(u) e^{-2\pi i k u} du \label{eq:periodic_sum}\\
&= \int_{-\infty}^{\infty} f(u) e^{-2\pi i k u} du \label{eq:union_intervals}\\
&= \hat{f}(k) \label{eq:fourier_transform}
\end{align}

Step \eqref{eq:periodic_sum} uses the fact that $\sum_{n \in \mathbb{Z}} e^{2\pi i k n} = 1$ for integer $k$.

Setting $x = 0$ in equation \eqref{eq:fourier_series}:
\begin{equation}
\label{eq:x_zero}
F(0) = \sum_{k \in \mathbb{Z}} c_k = \sum_{k \in \mathbb{Z}} \hat{f}(k)
\end{equation}

From definition \eqref{eq:periodized}:
\begin{equation}
\label{eq:f_zero}
F(0) = \sum_{n \in \mathbb{Z}} f(n)
\end{equation}

Combining equations \eqref{eq:x_zero} and \eqref{eq:f_zero} yields \eqref{eq:poisson}.
\end{proof}

\section{Connection to Discrete Fourier Transform}

\begin{definition}
For a sequence $\{x_n\}_{n=0}^{N-1}$, the $N$-point discrete Fourier transform is
\begin{equation}
\label{eq:dft}
X_k = \sum_{n=0}^{N-1} x_n e^{-2\pi i k n / N}, \quad k = 0, 1, \ldots, N-1
\end{equation}
with inverse transform
\begin{equation}
\label{eq:idft}
x_n = \frac{1}{N} \sum_{k=0}^{N-1} X_k e^{2\pi i k n / N}, \quad n = 0, 1, \ldots, N-1
\end{equation}
\end{definition}

\begin{theorem}
Let $f$ be a function on $\mathbb{R}$ and define $x_n = f(n/N)$ for $n = 0, 1, \ldots, N-1$. Then
\begin{equation}
\label{eq:dft_poisson_relation}
X_k = N \sum_{m \in \mathbb{Z}} \hat{f}(k + mN)
\end{equation}
where $\hat{f}$ is the continuous Fourier transform of $f$.
\end{theorem}

\begin{proof}
From definition \eqref{eq:dft}:
\begin{equation}
\label{eq:dft_expanded}
X_k = \sum_{n=0}^{N-1} f(n/N) e^{-2\pi i k n / N}
\end{equation}

Consider the scaled function $g(x) = f(x/N)$. Its Fourier transform is
\begin{equation}
\label{eq:scaled_fourier}
\hat{g}(\xi) = N \hat{f}(N\xi)
\end{equation}

Define the impulse train
\begin{equation}
\label{eq:impulse_train}
h(x) = \sum_{n=0}^{N-1} f(n/N) \delta(x - n/N)
\end{equation}

The Fourier transform of $h$ is
\begin{equation}
\label{eq:impulse_fourier}
\hat{h}(\xi) = \sum_{n=0}^{N-1} f(n/N) e^{-2\pi i n \xi / N}
\end{equation}

Setting $\xi = k$ gives $\hat{h}(k) = X_k$.

The function $h$ can be written as
\begin{equation}
\label{eq:h_periodic}
h(x) = \frac{1}{N} \sum_{n=0}^{N-1} g(x) \text{III}_N(x - n/N)
\end{equation}
where $\text{III}_N$ is the Shah function with period $1/N$.

By the Poisson summation formula applied to the periodization:
\begin{align}
\hat{h}(k) &= N \sum_{m \in \mathbb{Z}} \hat{g}(k + mN) \label{eq:poisson_applied}\\
&= N \sum_{m \in \mathbb{Z}} N \hat{f}(N(k + mN)/N) \label{eq:scaling_applied}\\
&= N^2 \sum_{m \in \mathbb{Z}} \hat{f}(k + mN) \label{eq:simplified}
\end{align}

Since $X_k = \hat{h}(k)/N$, we obtain equation \eqref{eq:dft_poisson_relation}.
\end{proof}

\section{Heat Equation and Fundamental Solutions}

\begin{definition}
The heat equation in $\mathbb{R}^d$ is
\begin{equation}
\label{eq:heat}
\frac{\partial u}{\partial t} = \Delta u
\end{equation}
where $\Delta = \sum_{i=1}^d \frac{\partial^2}{\partial x_i^2}$ is the Laplacian.
\end{definition}

\begin{theorem}
The fundamental solution of equation \eqref{eq:heat} is
\begin{equation}
\label{eq:heat_kernel}
G(x,t) = \frac{1}{(4\pi t)^{d/2}} e^{-|x|^2/(4t)}
\end{equation}
for $t > 0$ and $x \in \mathbb{R}^d$.
\end{theorem}

\begin{proof}
Take the Fourier transform with respect to $x$:
\begin{equation}
\label{eq:fourier_heat}
\frac{\partial \hat{u}}{\partial t} = -4\pi^2 |\xi|^2 \hat{u}
\end{equation}

The solution is
\begin{equation}
\label{eq:fourier_solution}
\hat{u}(\xi, t) = \hat{u}(\xi, 0) e^{-4\pi^2 |\xi|^2 t}
\end{equation}

For the fundamental solution with initial condition $u(x,0) = \delta(x)$, we have $\hat{u}(\xi,0) = 1$, so
\begin{equation}
\label{eq:fundamental_fourier}
\hat{G}(\xi, t) = e^{-4\pi^2 |\xi|^2 t}
\end{equation}

Taking the inverse Fourier transform:
\begin{align}
G(x,t) &= \int_{\mathbb{R}^d} e^{-4\pi^2 |\xi|^2 t} e^{2\pi i x \cdot \xi} d\xi \label{eq:inverse_transform}\\
&= \prod_{j=1}^d \int_{-\infty}^{\infty} e^{-4\pi^2 \xi_j^2 t} e^{2\pi i x_j \xi_j} d\xi_j \label{eq:product_form}
\end{align}

For each one-dimensional integral:
\begin{align}
\int_{-\infty}^{\infty} e^{-4\pi^2 \xi^2 t} e^{2\pi i x \xi} d\xi &= \int_{-\infty}^{\infty} e^{-4\pi^2 t(\xi^2 - \frac{ix}{2\pi t})} d\xi \label{eq:completing_square}\\
&= e^{-x^2/(4t)} \int_{-\infty}^{\infty} e^{-4\pi^2 t(\xi - \frac{ix}{4\pi t})^2} d\xi \label{eq:completed_square}\\
&= \frac{e^{-x^2/(4t)}}{\sqrt{4\pi t}} \label{eq:gaussian_integral}
\end{align}

Therefore:
\begin{equation}
\label{eq:final_kernel}
G(x,t) = \frac{1}{(4\pi t)^{d/2}} e^{-|x|^2/(4t)}
\end{equation}
\end{proof}

\section{Method of Images and Absorbing Boundaries}

\begin{theorem}
For the heat equation on the half-line $x > 0$ with absorbing boundary condition $u(0,t) = 0$, the fundamental solution is
\begin{equation}
\label{eq:half_line_solution}
G_+(x,y,t) = G(x-y,t) - G(x+y,t)
\end{equation}
where $G$ is given by equation \eqref{eq:heat_kernel} in one dimension.
\end{theorem}

\begin{proof}
The method constructs the solution using the original source at $y > 0$ and an image source at $-y < 0$ with opposite sign.

For the source term $\delta(x-y)$, place an image source $-\delta(x+y)$ at $-y$. The combined solution is
\begin{align}
u(x,t) &= \int_0^{\infty} G(x-y,t) \delta(y-y_0) dy - \int_0^{\infty} G(x+y,t) \delta(y-y_0) dy \label{eq:superposition}\\
&= G(x-y_0,t) - G(x+y_0,t) \label{eq:evaluation}
\end{align}

Verification of boundary condition:
\begin{align}
u(0,t) &= G(-y_0,t) - G(y_0,t) \label{eq:boundary_check}\\
&= \frac{1}{\sqrt{4\pi t}} e^{-y_0^2/(4t)} - \frac{1}{\sqrt{4\pi t}} e^{-y_0^2/(4t)} \label{eq:symmetry}\\
&= 0 \label{eq:boundary_satisfied}
\end{align}

The solution satisfies the heat equation since both $G(x-y_0,t)$ and $G(x+y_0,t)$ are solutions, and linear combinations of solutions are solutions.
\end{proof}

\section{Periodization and Rectangular Domains}

\begin{theorem}
For the heat equation on the rectangle $(0,a) \times (0,b)$ with absorbing boundary conditions, the fundamental solution is
\begin{equation}
\label{eq:rectangle_solution}
G_R(x,y,\xi,\eta,t) = \sum_{m,n \in \mathbb{Z}} (-1)^{m+n} G(x-\xi+2ma, y-\eta+2nb, t)
\end{equation}
\end{theorem}

\begin{proof}
Apply the method of images successively in both directions. For absorbing boundaries at $x = 0, a$ and $y = 0, b$, place image sources at:
\begin{itemize}
\item $(2ma \pm \xi, 2nb \pm \eta)$ for all integers $m,n$
\item Signs alternate: $(-1)^{m+n}$ for the source at $(2ma + (-1)^m \xi, 2nb + (-1)^n \eta)$
\end{itemize}

The infinite sum in equation \eqref{eq:rectangle_solution} represents the superposition of all image sources. Each term satisfies the heat equation, so their sum does as well.

Boundary condition verification at $x = 0$:
\begin{align}
&G_R(0,y,\xi,\eta,t) \label{eq:boundary_x_zero}\\
&= \sum_{m,n \in \mathbb{Z}} (-1)^{m+n} G(-\xi+2ma, y-\eta+2nb, t) \label{eq:x_zero_expanded}\\
&= \sum_{m,n \in \mathbb{Z}} (-1)^{m+n} G(\xi-2ma, y-\eta+2nb, t) \label{eq:symmetry_applied}\\
&= \sum_{m,n \in \mathbb{Z}} (-1)^{m+n+1} G(\xi+2ma, y-\eta+2nb, t) \label{eq:index_shift}\\
&= -G_R(0,y,\xi,\eta,t) \label{eq:sign_flip}
\end{align}

This implies $G_R(0,y,\xi,\eta,t) = 0$. Similar calculations verify the boundary conditions at $x = a$, $y = 0$, and $y = b$.
\end{proof}

\section{Theta Functions and Periodic Heat Kernels}

\begin{definition}
The Jacobi theta function is defined as
\begin{equation}
\label{eq:theta_def}
\vartheta_3(z,\tau) = \sum_{n=-\infty}^{\infty} e^{\pi i n^2 \tau + 2\pi i n z}
\end{equation}
where $\Im(\tau) > 0$.
\end{definition}

\begin{theorem}
The solution of the heat equation on the circle $S^1 = \mathbb{R}/\mathbb{Z}$ with initial condition $u(x,0) = \delta(x-x_0)$ is
\begin{equation}
\label{eq:circle_solution}
u(x,t) = \sum_{n \in \mathbb{Z}} G(x - x_0 + n, t) = \frac{1}{\sqrt{4\pi t}} \vartheta_3\left(\frac{x-x_0}{2}, \frac{it}{\pi}\right)
\end{equation}
\end{theorem}

\begin{proof}
The periodization of the fundamental solution gives:
\begin{align}
u(x,t) &= \sum_{n \in \mathbb{Z}} \frac{1}{\sqrt{4\pi t}} e^{-(x-x_0+n)^2/(4t)} \label{eq:periodized_kernel}\\
&= \frac{1}{\sqrt{4\pi t}} e^{-(x-x_0)^2/(4t)} \sum_{n \in \mathbb{Z}} e^{-n(x-x_0)/(2t)} e^{-n^2/(4t)} \label{eq:factored}\\
&= \frac{1}{\sqrt{4\pi t}} e^{-(x-x_0)^2/(4t)} \sum_{n \in \mathbb{Z}} e^{\pi i n^2 (it/\pi) + 2\pi i n \frac{x-x_0}{2}} \label{eq:theta_form}\\
&= \frac{1}{\sqrt{4\pi t}} \vartheta_3\left(\frac{x-x_0}{2}, \frac{it}{\pi}\right) \label{eq:theta_identified}
\end{align}

Step \eqref{eq:theta_form} uses the identity $e^{-n^2/(4t) - n(x-x_0)/(2t)} = e^{\pi i n^2 (it/\pi) + 2\pi i n (x-x_0)/2}$.

The connection to Poisson summation follows from:
\begin{equation}
\label{eq:poisson_theta}
\sum_{n \in \mathbb{Z}} e^{-(x+n)^2/4t} = \sqrt{\pi t} \sum_{k \in \mathbb{Z}} e^{-\pi^2 k^2 t} e^{2\pi i k x}
\end{equation}

This establishes the theta function as the periodized heat kernel on the circle.
\end{proof}

\section{Unified Framework}

The connections established demonstrate that:

\begin{enumerate}
\item The Poisson summation formula provides the mathematical foundation for relating continuous and discrete Fourier transforms through periodization.

\item The discrete Fourier transform emerges as a sampled and periodized version of the continuous transform, with aliasing effects captured by equation \eqref{eq:dft_poisson_relation}.

\item The method of images for boundary value problems creates solutions through systematic periodization and reflection of fundamental solutions.

\item Theta functions arise naturally as periodized heat kernels, representing the exact solutions for heat flow on compact domains.

\item The inverse discrete Fourier transform reconstructs signals from their frequency components using the same periodization principles that govern heat kernel construction on bounded domains.
\end{enumerate}

The mathematical structure reveals periodization as the fundamental operation connecting discrete sampling, boundary conditions, and exact solutions of the heat equation through theta function representations.

\end{document}

