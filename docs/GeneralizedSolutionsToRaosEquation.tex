\documentclass[11pt]{article}
\usepackage{amsmath}
\usepackage{amsfonts}
\usepackage{amssymb}
\usepackage{amsthm}

\newtheorem{theorem}{Theorem}
\newtheorem{corollary}{Corollary}

\title{Generalized Solutions to the Zeta Function Integral Equation}
\author{}
\date{}

\begin{document}

\maketitle

\section{Introduction}

Following the work of Rao~\cite{rao2025}, this note establishes the complete characterization of generalized solutions to the integral equation arising from the equivalence between zeros of the Riemann zeta function and solutions to a convolution equation. Rao showed that the existence of nontrivial zeros $\zeta(\sigma + it) = 0$ for $\sigma \in (0,1)$ is equivalent to the existence of nontrivial solutions to the integral equation
\begin{equation}
\int_{-\infty}^{\infty} K_\sigma(x-y) \phi(y) \, dy = 0
\end{equation}
where the kernel $K_\sigma$ arises from the Fourier representation of the zeta function.

The kernel $K_\sigma$ is defined explicitly through the integral representation
\begin{equation}
\zeta(s)(1-2^{1-s}) = \int_{\mathbb{R}} K_\sigma(u) e^{itu} \, du
\end{equation}
for $s = \sigma + it$, where
\begin{equation}
K_\sigma(u) = \frac{e^{\sigma u}}{e^{e^u} + 1}, \quad u \in \mathbb{R}.
\end{equation}

This kernel is obtained by the change of variables $u = \log x$ applied to the integral $\int_0^\infty \frac{x^{\sigma-1}}{e^x + 1} e^{it \log x} \, dx$ appearing in Rao's derivation.

\begin{theorem}[Complete Space of Generalized Solutions]
\label{thm:generalized_solutions}
Let $K_\sigma(u) = \frac{e^{\sigma u}}{e^{e^u} + 1}$ for $\sigma \in (0,1)$. Since $K_\sigma \in L^1(\mathbb{R})$, its Fourier transform
\begin{equation}
\widehat{K_\sigma}(t) = \int_{\mathbb{R}} K_\sigma(u) e^{-itu} \, du
\end{equation}
exists and is continuous. Define the zero set
\begin{equation}
Z_\sigma := \{t \in \mathbb{R} : \widehat{K_\sigma}(t) = 0\}.
\end{equation}

The complete space of generalized solutions to the convolution equation
\begin{equation}
\int_{-\infty}^{\infty} K_\sigma(x-y) \phi(y) \, dy = 0, \quad \forall x \in \mathbb{R}
\end{equation}
in the space of tempered distributions $\mathcal{S}'(\mathbb{R})$ is
\begin{equation}
\mathcal{N}_\sigma = \left\{ \phi \in \mathcal{S}'(\mathbb{R}) : \text{supp}(\widehat{\phi}) \subseteq Z_\sigma \right\}.
\end{equation}

Moreover, every solution $\phi \in \mathcal{N}_\sigma$ admits the integral representation
\begin{equation}
\phi(x) = \int_{Z_\sigma} e^{itx} \, d\mu(t)
\end{equation}
where $\mu$ is a complex tempered measure on $Z_\sigma$.
\end{theorem}

\begin{proof}
The convolution operator $T: \mathcal{S}'(\mathbb{R}) \to \mathcal{S}'(\mathbb{R})$ defined by $T\phi = K_\sigma * \phi$ satisfies
\begin{equation}
\widehat{T\phi} = \widehat{K_\sigma} \cdot \widehat{\phi}
\end{equation}
in the sense of tempered distributions.

The equation $T\phi = 0$ is equivalent to $\widehat{K_\sigma}(t) \widehat{\phi}(t) = 0$ as an identity of distributions. Since $\widehat{K_\sigma}$ is a continuous function, this occurs if and only if $\text{supp}(\widehat{\phi}) \subseteq Z_\sigma$.

For the integral representation, any tempered distribution $\phi$ with $\text{supp}(\widehat{\phi}) \subseteq Z_\sigma$ can be written as
\begin{equation}
\phi(x) = \int_{Z_\sigma} e^{itx} \, d\mu(t)
\end{equation}
by the Bochner-Schwartz theorem, where $\mu$ is a tempered measure on $Z_\sigma$. The integral converges in $\mathcal{S}'(\mathbb{R})$ since for any test function $\psi \in \mathcal{S}(\mathbb{R})$,
\begin{equation}
\langle \phi, \psi \rangle = \int_{Z_\sigma} \widehat{\psi}(t) \, d\mu(t)
\end{equation}
is well-defined due to the rapid decay of $\widehat{\psi}$.

Conversely, any such integral representation yields a solution since $\widehat{\phi} = \mu$ as measures, and $\text{supp}(\mu) \subseteq Z_\sigma$ implies $\widehat{K_\sigma} \cdot \widehat{\phi} = 0$.
\end{proof}

\begin{corollary}[Application to Zeta Function Zeros]
For the kernel $K_\sigma$ defined above, the Fourier transform satisfies
\begin{equation}
\widehat{K_\sigma}(t) = \frac{\zeta(\sigma + it)(1-2^{1-(\sigma+it)})}{C_\sigma}
\end{equation}
for some nonzero constant $C_\sigma$. Since $1-2^{1-s} \neq 0$ for $s$ with $\text{Re}(s) \in (0,1)$, the zero set is
\begin{equation}
Z_\sigma = \{t \in \mathbb{R} : \zeta(\sigma + it) = 0\}.
\end{equation}

Therefore, the complete space of generalized solutions is
\begin{equation}
\mathcal{N}_\sigma = \left\{ \phi(x) = \int_{\{t: \zeta(\sigma+it)=0\}} e^{itx} \, d\mu(t) : \mu \text{ is a tempered measure} \right\}.
\end{equation}
\end{corollary}

\begin{thebibliography}{9}
\bibitem{rao2025} Rao, M.M. Harmonic and Probabilistic Approaches to Zeros of Riemann's Zeta Function. Department of Mathematics, University of California, Riverside.
\end{thebibliography}

\end{document}
