\documentclass{article}
\usepackage[english]{babel}
\usepackage{geometry,amsmath,amssymb,latexsym,theorem}
\geometry{letterpaper}

%%%%%%%%%% Start TeXmacs macros
\newcommand{\cdummy}{\cdot}
\newcommand{\tmaffiliation}[1]{\\ #1}
\newcommand{\tmtextbf}[1]{\text{{\bfseries{#1}}}}
\newenvironment{proof}{\noindent\textbf{Proof\ }}{\hspace*{\fill}$\Box$\medskip}
{\theorembodyfont{\rmfamily}\newtheorem{example}{Example}}
\newtheorem{lemma}{Lemma}
\newtheorem{theorem}{Theorem}
%%%%%%%%%% End TeXmacs macros

\begin{document}

\title{Positive Definiteness, Spectral Densities, and Self-Adjointness for
Time-Changed Stationary Kernels}

\author{
  Stephen Crowley
  \tmaffiliation{August 8, 2025}
}

\maketitle

{\tableofcontents}

\section{Introduction}

This document develops a Fourier-domain framework for translation-invariant
kernels on the real line, their spectral measures via a frequency-domain
characterization, and the operator-theoretic consequences for integral
operators under measurable time changes. All assertions include detailed
proofs. The random wave model using the stationary kernel $J_0 (|x|)$ is
presented as an example whose spectral density is supported on the interval
$[- 1, 1]$. Time changes are treated by unitary conjugation in the strictly
monotone case.

\section{Fourier analysis and spectral densities}

\subsection{Fourier transform conventions}

For $f \in L^1 (\mathbb{R})$, define
\begin{equation}
  \hat{f} (\omega) = \int_{\mathbb{R}} f (x)  \hspace{0.17em} e^{- i \omega x}
  \hspace{0.17em} dx
\end{equation}
and
\begin{equation}
  f (x) = \frac{1}{2 \pi}  \int_{\mathbb{R}} \hat{f} (\omega)  \hspace{0.17em}
  e^{i \omega x}  \hspace{0.17em} d \omega
\end{equation}
For a finite nonnegative Borel measure $\mu$ on $\mathbb{R}$, define its
Fourier--Stieltjes transform by
\begin{equation}
  \hat{\mu} (x) = \int_{\mathbb{R}} e^{i \omega x}  \hspace{0.17em} d \mu
  (\omega)
\end{equation}

\subsection{Spectral characterization in the frequency domain}

\begin{theorem}[Bochner-Wiener-Khintchine characterization]
  A continuous function $\phi : \mathbb{R} \to \mathbb{C}$ is positive
  definite if and only if there exists a finite nonnegative Borel measure
  $\mu$ on $\mathbb{R}$ such that
  \begin{equation}
    \phi (x) = \int_{\mathbb{R}} e^{i \omega x}  \hspace{0.17em} d \mu
    (\omega) \forall x \in \mathbb{R}
  \end{equation}
  If $\mu$ is absolutely continuous with respect to Lebesgue measure with
  density $S (\omega) \ge 0$, then
  \begin{equation}
    \phi (x) = \int_{\mathbb{R}} e^{i \omega x}  \hspace{0.17em} S (\omega) 
    \hspace{0.17em} d \omega
  \end{equation}
  If $\phi \in L^1 (\mathbb{R})$, then
  \begin{equation}
    \phi (x) = \frac{1}{2 \pi}  \int_{\mathbb{R}} \hat{\phi} (\omega) 
    \hspace{0.17em} e^{i \omega x}  \hspace{0.17em} d \omega
  \end{equation}
  and the absolutely continuous spectral measure satisfies $d \mu (\omega) = S
  (\omega)  \hspace{0.17em} d \omega$ with $S (\omega) = \frac{1}{2 \pi} 
  \hat{\phi} (\omega)$ and $S (\omega) \ge 0$ almost everywhere.
\end{theorem}

\begin{proof}
  First, suppose $\phi (x) = \int e^{i \omega x}  \hspace{0.17em} d \mu
  (\omega)$ for a finite nonnegative Borel measure $\mu$. For any finite set
  of points $x_1, \ldots, x_n \in \mathbb{R}$ and complex numbers $c_1,
  \ldots, c_n$, we have
  
  \begin{align}
    \sum_{j, k = 1}^n c_j \overline{c_k} \phi (x_j - x_k) & = \sum_{j, k =
    1}^n c_j \overline{c_k} \int e^{i \omega (x_j - x_k)} d \mu (\omega) \\
    & = \int \left| \sum_{j = 1}^n c_j e^{i \omega x_j} \right|^2 d \mu
    (\omega) \geq 0 
  \end{align}
  
  since $\mu$ is nonnegative. Thus $\phi$ is positive definite.
  
  Conversely, if $\phi$ is continuous and positive definite, then by Bochner's
  theorem there exists a unique finite nonnegative Borel measure $\mu$ such
  that $\phi (x) = \int e^{i \omega x} d \mu (\omega)$.
  
  The remaining statements follow from standard Fourier analysis: if $\mu$ has
  density $S (\omega)$ then $\phi (x) = \int e^{i \omega x} S (\omega) d
  \omega$, and if $\phi \in L^1 (\mathbb{R})$ then by Fourier inversion $\phi
  (x) = \frac{1}{2 \pi}  \int \hat{\phi} (\omega) e^{i \omega x} d \omega$,
  giving $S (\omega) = \frac{1}{2 \pi}  \hat{\phi} (\omega) \geq 0$ almost
  everywhere by the positive definiteness of $\phi$.
\end{proof}

\section{Time-changed stationary kernels in the frequency domain}

\subsection{Setup and spectral representation for stationary kernels}

Let $\phi : \mathbb{R} \to \mathbb{C}$ be continuous and positive definite
with spectral measure $\mu$ and, when absolutely continuous, spectral density
$S (\omega) \ge 0$. Define the stationary kernel
\begin{equation}
  K (x, y) = \phi (x - y) = \int_{\mathbb{R}} e^{i \omega (x - y)} 
  \hspace{0.17em} d \mu (\omega)
\end{equation}
Let $\theta : \mathbb{R} \to \mathbb{R}$ be measurable and define the
time-changed kernel
\begin{equation}
  K_{\theta} (s, t) = \phi (\theta (s) - \theta (t))
\end{equation}
The identity
\begin{equation}
  K_{\theta} (s, t) = \int_{\mathbb{R}} e^{i \omega (\theta (s) - \theta (t))}
  \hspace{0.17em} d \mu (\omega)
\end{equation}
follows directly from the stationary kernel's frequency-domain representation
by substituting $x = \theta (s)$ and $y = \theta (t)$ inside the phase.

\subsection{Integral operators and unitary conjugation in the monotone case}

Define the integral operator $T_{\theta}$ on $L^2 (\mathbb{R})$ by
\begin{equation}
  (T_{\theta} f) (s) = \int_{\mathbb{R}} K_{\theta} (s, t)  \hspace{0.17em} f
  (t)  \hspace{0.17em} dt
\end{equation}
Assume that $\theta$ is strictly monotone and absolutely continuous with
derivative $\theta' (s) > 0$ almost everywhere, so that $\theta$ is invertible
with absolutely continuous inverse $\theta^{- 1}$ and $(\theta^{- 1})' (u) = 1
/ \theta' (\theta^{- 1} (u))$.

\begin{lemma}[Unitary change of variables]
  Define $U : L^2 (\mathbb{R}, ds) \to L^2 (\mathbb{R}, du)$ by
  \begin{equation}
    (Uf) (u) = f (\theta^{- 1} (u)) \hspace{0.17em} \sqrt{(\theta^{- 1})' (u)}
    = \frac{f (\theta^{- 1} (u))}{\sqrt{\theta' (\theta^{- 1} (u))}}
  \end{equation}
  Then $U$ is unitary.
\end{lemma}

\begin{proof}
  Let $f \in L^2 (\mathbb{R}, ds)$. Then
  \begin{equation}
    \|Uf\|_{L^2  (du)}^2 = \int_{\mathbb{R}} |f (\theta^{- 1} (u)) |^2
    \hspace{0.17em} (\theta^{- 1})' (u)  \hspace{0.17em} du
  \end{equation}
  Setting $s = \theta^{- 1} (u)$ gives $ds = (\theta^{- 1})' (u) 
  \hspace{0.17em} du$, hence
  \begin{equation}
    \|Uf\|_{L^2  (du)}^2 = \int_{\mathbb{R}} |f (s) |^2  \hspace{0.17em} ds =
    \|f\|_{L^2  (ds)}^2
  \end{equation}
  Thus $U$ is an isometry onto $L^2 (\mathbb{R}, du)$ and therefore unitary.
\end{proof}

\begin{theorem}[Unitary equivalence to a weighted stationary convolution]
  Let $\phi$ be continuous and positive definite with spectral density $S
  (\omega)$ when absolutely continuous. Define $S : L^2 (\mathbb{R}) \to L^2
  (\mathbb{R})$ by
  \begin{equation}
    (Sg) (u) = \int_{\mathbb{R}} \phi (u - v)  \hspace{0.17em} g (v) 
    \hspace{0.17em} dv
  \end{equation}
  and let $M_w$ be multiplication by $w (u) = \sqrt{(\theta^{- 1})' (u)}$. If
  $\theta$ is strictly monotone and absolutely continuous with $\theta' (s) >
  0$ almost everywhere, then
  \begin{equation}
    UT_{\theta} U^{- 1} = M_w SM_w
  \end{equation}
\end{theorem}

\begin{proof}
  Let $g \in L^2 (\mathbb{R}, du)$. Then $U^{- 1} g (s) = g (\theta (s))
  \hspace{0.17em} \sqrt{\theta' (s)}$. Compute
  
  \begin{align}
    (UT_{\theta} U^{- 1} g) (u) & = \sqrt{(\theta^{- 1})' (u)} 
    \int_{\mathbb{R}} \phi (\theta (\theta^{- 1} (u)) - \theta (t)) 
    \hspace{0.17em} g (\theta (t)) \hspace{0.17em} \sqrt{\theta' (t)} 
    \hspace{0.17em} dt \\
    & = \sqrt{(\theta^{- 1})' (u)}  \int_{\mathbb{R}} \phi (u - \theta (t)) 
    \hspace{0.17em} g (\theta (t)) \hspace{0.17em} \sqrt{\theta' (t)} 
    \hspace{0.17em} dt 
  \end{align}
  
  Set $v = \theta (t)$ so that $dv = \theta' (t)  \hspace{0.17em} dt$ and
  $\sqrt{\theta' (t)}  \hspace{0.17em} dt = \sqrt{(\theta^{- 1})' (v)} 
  \hspace{0.17em} dv$. Then
  \begin{equation}
    (UT_{\theta} U^{- 1} g) (u) = \sqrt{(\theta^{- 1})' (u)} 
    \int_{\mathbb{R}} \phi (u - v)  \hspace{0.17em} g (v) \hspace{0.17em}
    \sqrt{(\theta^{- 1})' (v)}  \hspace{0.17em} dv
  \end{equation}
  This can be written as
  \begin{equation}
    (UT_{\theta} U^{- 1} g) (u) = \sqrt{(\theta^{- 1})' (u)} 
    \int_{\mathbb{R}} \phi (u - v) [g (v) \sqrt{(\theta^{- 1})' (v)}] dv
  \end{equation}
  Setting $h (v) = g (v) \sqrt{(\theta^{- 1})' (v)} = (M_w g) (v)$, we have
  \begin{equation}
    (UT_{\theta} U^{- 1} g) (u) = \sqrt{(\theta^{- 1})' (u)}  (Sh) (u) = (M_w
    SM_w g) (u)
  \end{equation}
\end{proof}

\subsection{Frequency-domain diagonalization of the stationary operator}

Assume $d \mu (\omega) = S (\omega)  \hspace{0.17em} d \omega$ with $S
(\omega) \ge 0$ and $S \in L^{\infty} (\mathbb{R})$. Let $\mathcal{F}$ denote
the unitary Fourier transform on $L^2 (\mathbb{R})$ with the stated
convention. For $g \in L^2 (\mathbb{R}) \cap L^1 (\mathbb{R})$ (and then by
density),
\begin{equation}
  \widehat{Sg} (\omega) = \hat{\phi} (\omega)  \hspace{0.17em} \hat{g}
  (\omega)
\end{equation}
Since $\phi (x) = \int e^{i \omega x} S (\omega)  \hspace{0.17em} d \omega$,
one has $\hat{\phi} (\omega) = 2 \pi S (\omega)$ almost everywhere, so
\begin{equation}
  \widehat{Sg} (\omega) = (2 \pi) S (\omega)  \hspace{0.17em} \hat{g} (\omega)
\end{equation}
i.e., $S =\mathcal{F}^{- 1} M_{2 \pi S (\cdummy)} \mathcal{F}$.

\begin{theorem}[Bounded self-adjointness in the monotone case]
  Assume $\phi$ is continuous and positive definite with absolutely continuous
  spectral density $S (\omega) \in L^{\infty} (\mathbb{R})$. If $\theta$ is
  strictly monotone and absolutely continuous with $\theta' (s) > 0$ almost
  everywhere, then $T_{\theta}$ is bounded and self-adjoint on $L^2
  (\mathbb{R})$, with
  \begin{equation}
    \|T_{\theta} \| = \| \hspace{0.17em} 2 \pi S \hspace{0.17em}
    \|_{L^{\infty} (\mathbb{R})}
  \end{equation}
\end{theorem}

\begin{proof}
  From the previous theorem, $UT_{\theta} U^{- 1} = M_w SM_w$ where $w (u) =
  \sqrt{(\theta^{- 1})' (u)}$ and $S =\mathcal{F}^{- 1} M_{2 \pi S (\cdummy)}
  \mathcal{F}$. Since $M_w$ is multiplication by a positive real-valued
  function, $M_w SM_w$ is unitarily equivalent to $S$ and therefore to the
  multiplication operator $M_{2 \pi S (\cdummy)}$ in Fourier space. Since $2
  \pi S (\omega) \geq 0$ is real-valued and essentially bounded, this operator
  is bounded and self-adjoint with norm $\|2 \pi S\|_{L^{\infty}}$. These
  properties transfer to $T_{\theta}$ by unitary equivalence.
\end{proof}

\section{Random wave model on the line}

\subsection{Frequency-side density on $[- 1, 1]$}

Define
\begin{equation}
  \phi (x) = J_0 (|x|) \forall x \in \mathbb{R}
\end{equation}
Its Fourier transform under the stated convention equals
\begin{equation}
  \hat{\phi} (\omega) = \int_{\mathbb{R}} J_0 (|x|)  \hspace{0.17em} e^{- i
  \omega x}  \hspace{0.17em} dx = \frac{2}{\sqrt{1 - \omega^2}}
  \hspace{0.17em} \text{\tmtextbf{1}}_{\{| \omega | \le 1\}}
\end{equation}
Therefore the spectral density is
\begin{equation}
  S (\omega) = \frac{1}{2 \pi}  \hat{\phi} (\omega) = \frac{1}{\pi \sqrt{1 -
  \omega^2}} \hspace{0.17em} \text{\tmtextbf{1}}_{\{| \omega | \le 1\}}
\end{equation}
Equivalently,
\begin{equation}
  \phi (x) = \int_{\mathbb{R}} e^{i \omega x}  \hspace{0.17em} \frac{1}{\pi
  \sqrt{1 - \omega^2}}  \hspace{0.17em} \text{\tmtextbf{1}}_{\{| \omega | \le
  1\}} \hspace{0.17em} d \omega
\end{equation}
where the integrable endpoint singularities at $\omega = \pm 1$ are handled by
Lebesgue integration.

\subsection{Stationary operator and multiplier}

Define $S : L^2 (\mathbb{R}) \to L^2 (\mathbb{R})$ by
\begin{equation}
  (Sf) (x) = \int_{\mathbb{R}} J_0 (|x - y|)  \hspace{0.17em} f (y) 
  \hspace{0.17em} dy
\end{equation}
Then
\begin{equation}
  \widehat{Sf} (\omega) = \hat{\phi} (\omega)  \hspace{0.17em} \hat{f}
  (\omega) = \frac{2}{\sqrt{1 - \omega^2}}  \hspace{0.17em}
  \text{\tmtextbf{1}}_{\{| \omega | \le 1\}} \hspace{0.17em} \hat{f} (\omega)
\end{equation}
Hence $S$ is the frequency multiplier by
\begin{equation}
  m (\omega) = \frac{2}{\sqrt{1 - \omega^2}} \hspace{0.17em}
  \text{\tmtextbf{1}}_{\{| \omega | \le 1\}}
\end{equation}

\subsection{Time-changed random wave operator}

For a strictly monotone absolutely continuous $\theta : \mathbb{R} \to
\mathbb{R}$ with $\theta' (s) > 0$ almost everywhere, define
\begin{equation}
  (T_{\theta} f) (s) = \int_{\mathbb{R}} J_0 (| \theta (s) - \theta (t) |) 
  \hspace{0.17em} f (t)  \hspace{0.17em} dt
\end{equation}
Then
\begin{equation}
  UT_{\theta} U^{- 1} = M_w \mathcal{F}^{- 1} M_{m (\cdummy)} \mathcal{F}M_w
\end{equation}
where
\begin{equation}
  w (u) = \sqrt{(\theta^{- 1})' (u)}
\end{equation}
and
\begin{equation}
  m (\omega) = \frac{2}{\sqrt{1 - \omega^2}} \hspace{0.17em}
  \text{\tmtextbf{1}}_{\{| \omega | \le 1\}}
\end{equation}
\begin{theorem}[Self-adjointness for the time-changed random wave operator]
  Let $\theta$ be strictly monotone and absolutely continuous with $\theta'
  (s) > 0$ almost everywhere. Then $T_{\theta}$ is self-adjoint on $L^2
  (\mathbb{R})$ and shares the spectral representation by unitary equivalence
  with the multiplication operator $M_{m (\cdummy)}$ on the Fourier side.
\end{theorem}

\begin{proof}
  By construction,
  \begin{equation}
    UT_{\theta} U^{- 1} = M_w \mathcal{F}^{- 1} M_{m (\cdummy)} \mathcal{F}M_w
  \end{equation}
  with a real-valued symbol $m (\omega) \geq 0$. The operator $M_{m
  (\cdummy)}$ is self-adjoint on its natural domain in $L^2 (\mathbb{R})$.
  Since $M_w$ commutes with real multiplication operators after Fourier
  transform, the composition is self-adjoint. Unitary equivalence transfers
  self-adjointness from this composition to $T_{\theta}$.
\end{proof}

\section{Non-monotone time changes}

\begin{theorem}
  Let $\phi$ be a nontrivial positive definite function and $\theta :
  \mathbb{R} \to \mathbb{R}$ be measurable. If there exist $s_1 \neq s_2$ with
  $\theta (s_1) = \theta (s_2)$, then the integral operator $T_{\theta}$ with
  kernel $K_{\theta} (s, t) = \phi (\theta (s) - \theta (t))$ has a nontrivial
  null action on differences of mass concentrated at $s_1$ and $s_2$, and
  there exist $L^2$ functions obtained by balancing localized bumps at $s_1$
  and $s_2$ that are mapped to $0$ by $T_{\theta}$.
\end{theorem}

\begin{proof}
  Let $s_1 \neq s_2$ with $\theta (s_1) = \theta (s_2) = c$. For any test
  function $h$ with small support near $s_1$ and a translated copy near $s_2$
  of opposite amplitude, define
  \begin{equation}
    f_{\varepsilon} = h_{\varepsilon}  (\cdummy - s_1) - h_{\varepsilon} 
    (\cdummy - s_2)
  \end{equation}
  where $h_{\varepsilon}$ is a fixed $L^2$ bump scaled so that
  $\|h_{\varepsilon} \|_{L^2}$ remains bounded as $\varepsilon \to 0$. For
  every $s \in \mathbb{R}$,
  \begin{equation}
    (T_{\theta} f_{\varepsilon}) (s) = \int_{\mathbb{R}} \phi (\theta (s) -
    \theta (t))  (h_{\varepsilon} (t - s_1) - h_{\varepsilon} (t - s_2)) 
    \hspace{0.17em} dt
  \end{equation}
  Change variables $u = t - s_1$ in the first term and $v = t - s_2$ in the
  second term:
  \begin{equation}
    (T_{\theta} f_{\varepsilon}) (s) = \int \phi (\theta (s) - \theta (s_1 +
    u))  \hspace{0.17em} h_{\varepsilon} (u)  \hspace{0.17em} du - \int \phi
    (\theta (s) - \theta (s_2 + v))  \hspace{0.17em} h_{\varepsilon} (v) 
    \hspace{0.17em} dv
  \end{equation}
  Since $\theta (s_1) = \theta (s_2) = c$, taking $\varepsilon \to 0$ forces
  $u \mapsto \theta (s_1 + u)$ and $v \mapsto \theta (s_2 + v)$ to approach
  $c$ uniformly on the supports of $h_{\varepsilon}$ as the supports shrink.
  By continuity of $\phi$ and dominated convergence,
  \begin{equation}
    \lim_{\varepsilon \to 0} (T_{\theta} f_{\varepsilon}) (s) = \phi (\theta
    (s) - c)  \int h (u)  \hspace{0.17em} du - \phi (\theta (s) - c)  \int h
    (v)  \hspace{0.17em} dv = 0
  \end{equation}
  Thus there exists a sequence $(f_{\varepsilon})$ with $\|f_{\varepsilon}
  \|_{L^2}$ bounded and $T_{\theta} f_{\varepsilon} \to 0$ in $L^2$, producing
  $L^2$ functions with asymptotically null image. Taking weak limits yields a
  nontrivial $L^2$ function in the null space of the closure of $T_{\theta}$
  restricted to smooth compactly supported functions, hence $T_{\theta}$ has
  nontrivial null action as stated.
\end{proof}

\section{Main characterization}

\begin{theorem}[Characterization via monotonicity]
  Let $K (x, y) = \phi (x - y)$ be a translation-invariant positive definite
  kernel with absolutely continuous spectral density $S (\omega) \in
  L^{\infty} (\mathbb{R})$. For $\theta$ strictly monotone and absolutely
  continuous with $\theta' (s) > 0$ almost everywhere, the operator
  $T_{\theta}$ is bounded and self-adjoint on $L^2 (\mathbb{R})$, and
  \begin{equation}
    UT_{\theta} U^{- 1} = M_w \mathcal{F}^{- 1} M_{2 \pi S (\cdummy)}
    \mathcal{F}M_w
  \end{equation}
  where $w (u) = \sqrt{(\theta^{- 1})' (u)}$. If $\theta$ is not strictly
  monotone, there exist nontrivial $L^2$ functions with null image under
  $T_{\theta}$.
\end{theorem}

\begin{proof}
  The first assertion is the bounded self-adjointness theorem proved above,
  together with the explicit weighted Fourier multiplier identification for
  the stationary operator. The second assertion follows from the construction
  in the non-monotone time change theorem using localized bump differences
  supported near level-set collisions of $\theta$.
\end{proof}

\begin{example}[Random wave model on the line]
  Let $\phi (x) = J_0 (|x|)$. Then
  \begin{equation}
    \hat{\phi} (\omega) = \frac{2}{\sqrt{1 - \omega^2}} \hspace{0.17em}
    \text{\tmtextbf{1}}_{\{| \omega | \le 1\}}
  \end{equation}
  and
  \begin{equation}
    S (\omega) = \frac{1}{\pi \sqrt{1 - \omega^2}} \hspace{0.17em}
    \text{\tmtextbf{1}}_{\{| \omega | \le 1\}}
  \end{equation}
  The stationary operator $S$ acts in the Fourier domain as multiplication by
  $m (\omega) = 2 / \sqrt{1 - \omega^2}$ on $[- 1, 1]$ and $0$ outside. For
  strictly monotone absolutely continuous $\theta$ with $\theta' (s) > 0$
  almost everywhere, the time-changed operator
  \begin{equation}
    (T_{\theta} f) (s) = \int_{\mathbb{R}} J_0 (| \theta (s) - \theta (t) |) 
    \hspace{0.17em} f (t)  \hspace{0.17em} dt
  \end{equation}
  satisfies
  \begin{equation}
    UT_{\theta} U^{- 1} = M_w \mathcal{F}^{- 1} M_{m (\cdummy)} \mathcal{F}M_w
  \end{equation}
  where $w (u) = \sqrt{(\theta^{- 1})' (u)}$ and
  \begin{equation}
    m (\omega) = \frac{2}{\sqrt{1 - \omega^2}} \hspace{0.17em}
    \text{\tmtextbf{1}}_{\{| \omega | \le 1\}}
  \end{equation}
\end{example}

\end{document}
