\documentclass{article}
\usepackage{amsmath}
\usepackage{amsfonts}

\begin{document}

\section*{Wheeler-DeWitt Equation}

\[
\left( - \frac{16 \pi G}{c^3} G_{ijkl} \frac{\delta^2}{\delta g_{ij}(x) \delta g_{kl}(x)} + \frac{c^3}{16 \pi G} \sqrt{g(x)} \left( R(x) - 2 \Lambda \right) \right) \Psi[g_{ij}(x)] = 0
\]

\section*{Definitions of Variables and Arguments}

\begin{itemize}
    \item G_{ijkl}: DeWitt metric, defines the geometry of the space of 3-metrics g_{ij}(x):
\[
    G_{ijkl} = \frac{1}{2} \sqrt{g(x)} \left( g_{ik}(x) g_{jl}(x) + g_{il}(x) g_{jk}(x) - g_{ij}(x) g_{kl}(x) \right)
\]
    
    \item g_{ij}(x): The 3-metric on the spatial hypersurface at spatial coordinate x.
    
    \item \frac{\delta^2}{\delta g_{ij}(x) \delta g_{kl}(x)}: Functional second derivative with respect to the 3-metric g_{ij}(x).
    
    \item \sqrt{g(x)}: Square root of the determinant of the 3-metric g_{ij}(x).
    
    \item R(x): Scalar curvature of the spatial hypersurface at point x.
    
    \item \Lambda: Cosmological constant.
    
    \item \Psi[g_{ij}(x)]: Wave function of the universe, depends on the 3-metric g_{ij}(x).
    
    \item G: Newton's gravitational constant.
    
    \item c: Speed of light.
\end{itemize}

\section*{Purpose}

The task is to find the wave function \Psi[g_{ij}(x)] that satisfies the Wheeler-DeWitt equation. This wave function encapsulates the quantum state of the entire universe and describes the geometry of the universe in terms of quantum mechanics.

\end{document}
