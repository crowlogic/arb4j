\documentclass{article}
\usepackage[english]{babel}
\usepackage{amsmath,latexsym}

%%%%%%%%%% Start TeXmacs macros
\newcommand{\mathd}{\mathrm{d}}
\newcommand{\tmaffiliation}[1]{\\ #1}
\newcommand{\tmtextbf}[1]{\text{{\bfseries{#1}}}}
\newenvironment{proof}{\noindent\textbf{Proof\ }}{\hspace*{\fill}$\Box$\medskip}
\newtheorem{theorem}{Theorem}
%%%%%%%%%% End TeXmacs macros

\begin{document}

\title{The Integral $\int_{- 1}^{\omega} \frac{1}{^4 \sqrt{1 - \lambda^2}}
\mathd \lambda$}

\author{
  Stephen Crowley
  \tmaffiliation{July 28, 2025}
}

\maketitle

\

\begin{theorem}
  [Integral of $(1 - y^2)^{- 1 / 4}$] For $\lambda \in [- 1, 1]$, the definite
  integral
  \begin{equation}
    \int_{- 1}^{\lambda} \sqrt{\frac{1}{\sqrt{1 - y^2}}}  \hspace{0.17em} dy
  \end{equation}
  can be expressed in terms of incomplete elliptic integrals of the second
  kind as:
  \begin{equation}
    \int_{- 1}^{\lambda} \sqrt{\frac{1}{\sqrt{1 - y^2}}}  \hspace{0.17em} dy =
    2 \left[ \mathrm{EllipticE} \left( \frac{\arcsin \lambda}{2} | 2 \right) -
    \mathrm{EllipticE} \left( - \frac{\pi}{4} | 2 \right) \right]
  \end{equation}
  where $\mathrm{EllipticE} (\phi |m)$ denotes the incomplete elliptic
  integral of the second kind.
\end{theorem}

\begin{proof}
  We proceed in several steps.
  
  \tmtextbf{Step 1:} Simplify the integrand.
  \begin{equation}
    \sqrt{\frac{1}{\sqrt{1 - y^2}}} = \left( \frac{1}{\sqrt{1 - y^2}}
    \right)^{1 / 2} = (1 - y^2)^{- 1 / 4}
  \end{equation}
  Therefore, our integral becomes:
  \begin{equation}
    I = \int_{- 1}^{\lambda} (1 - y^2)^{- 1 / 4}  \hspace{0.17em} dy
  \end{equation}
  \tmtextbf{Step 2:} Apply the trigonometric substitution $y = \sin \theta$.
  
  Under this substitution:
  
  \begin{align}
    dy & = \cos \theta \hspace{0.17em} d \theta \\
    1 - y^2 & = 1 - \sin^2 \theta = \cos^2 \theta \\
    (1 - y^2)^{- 1 / 4} & = (\cos^2 \theta)^{- 1 / 4} = | \cos \theta |^{- 1 /
    2} 
  \end{align}
  
  For $\theta \in [- \pi / 2, \pi / 2]$, we have $\cos \theta \geq 0$, so $|
  \cos \theta | = \cos \theta$.
  
  The limits of integration transform as:
  
  \begin{align}
    y = - 1 & \Rightarrow \theta = \arcsin (- 1) = - \frac{\pi}{2} \\
    y = \lambda & \Rightarrow \theta = \arcsin (\lambda) 
  \end{align}
  
  \tmtextbf{Step 3:} Transform the integral.
  \begin{equation}
    I = \int_{- \pi / 2}^{\arcsin \lambda} (\cos \theta)^{- 1 / 2} \cos \theta
    \hspace{0.17em} d \theta = \int_{- \pi / 2}^{\arcsin \lambda} (\cos
    \theta)^{1 / 2}  \hspace{0.17em} d \theta
  \end{equation}
  \tmtextbf{Step 4:} Express in terms of elliptic integrals.
  
  It is a known result that:
  \begin{equation}
    \int \sqrt{\cos \theta}  \hspace{0.17em} d \theta = 2 \hspace{0.17em}
    \mathrm{EllipticE} \left( \frac{\theta}{2} | 2 \right) + C
  \end{equation}
  where $\mathrm{EllipticE} (\phi |m)$ is the incomplete elliptic integral of
  the second kind.
  
  \tmtextbf{Step 5:} Evaluate the definite integral.
  
  \begin{align}
    I & = \left[ 2 \hspace{0.17em} \mathrm{EllipticE} \left( \frac{\theta}{2}
    | 2 \right) \right]_{- \pi / 2}^{\arcsin \lambda} \\
    & = 2 \hspace{0.17em} \mathrm{EllipticE} \left( \frac{\arcsin \lambda}{2}
    | 2 \right) - 2 \hspace{0.17em} \mathrm{EllipticE} \left( \frac{- \pi /
    2}{2} | 2 \right) \\
    & = 2 \left[ \mathrm{EllipticE} \left( \frac{\arcsin \lambda}{2} | 2
    \right) - \mathrm{EllipticE} \left( - \frac{\pi}{4} | 2 \right) \right] 
  \end{align}
  
  This completes the proof.
\end{proof}

\end{document}
