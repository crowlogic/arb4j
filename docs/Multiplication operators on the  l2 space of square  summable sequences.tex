\documentclass[11pt]{article}
\usepackage{amsmath,amsfonts,amssymb,amsthm}
\usepackage[utf8]{inputenc}
\usepackage{mathtools}
\usepackage{tcolorbox}

% Theorem environments
\newtheorem{theorem}{Theorem}[section]
\newtheorem{lemma}[theorem]{Lemma}
\newtheorem{proposition}[theorem]{Proposition}
\newtheorem{corollary}[theorem]{Corollary}
\theoremstyle{definition}
\newtheorem{definition}[theorem]{Definition}
\newtheorem{example}[theorem]{Example}
\theoremstyle{remark}
\newtheorem{remark}[theorem]{Remark}

% Commands
\newcommand{\ell}{\ell}
\newcommand{\norm}[1]{\left\|#1\right\|}
\newcommand{\inner}[2]{\langle #1, #2 \rangle}
\newcommand{\R}{\mathbb{R}}
\newcommand{\C}{\mathbb{C}}
\newcommand{\N}{\mathbb{N}}
\DeclareMathOperator{\sup}{sup}
\DeclareMathOperator{\inf}{inf}

\title{Multiplication Operators on $\ell^2$ Space}
\author{}
\date{}

\begin{document}

\maketitle

\section{Introduction}

This document presents a comprehensive analysis of multiplication operators on the Hilbert space $\ell^2$ of square-summable sequences. The focus lies on establishing the fundamental properties of these operators through rigorous proofs.

\section{The $\ell^2$ Space}

\begin{definition}[The $\ell^2$ Space]
The space $\ell^2$ consists of all sequences $x = (x_0, x_1, x_2, \ldots)$ of complex numbers such that
\[
\sum_{n=0}^{\infty} |x_n|^2 < \infty.
\]
\end{definition}

\begin{theorem}[$\ell^2$ is a Hilbert Space]
The space $\ell^2$ equipped with the inner product
\[
\inner{x}{y} = \sum_{j=0}^{\infty} x_j \overline{y_j}
\]
and induced norm $\norm{x}_2 = \sqrt{\inner{x}{x}}$ forms a complete Hilbert space.
\end{theorem}

\begin{proof}
The proof consists of several steps:

\textbf{Step 1:} Verify that the inner product is well-defined.
For $x, y \in \ell^2$, the Cauchy-Schwarz inequality gives
\[
\left|\sum_{j=0}^{n} x_j \overline{y_j}\right| \leq \left(\sum_{j=0}^{n} |x_j|^2\right)^{1/2} \left(\sum_{j=0}^{n} |y_j|^2\right)^{1/2} \leq \norm{x}_2 \norm{y}_2.
\]
Since the right side is finite, the series $\sum_{j=0}^{\infty} x_j \overline{y_j}$ converges absolutely.

\textbf{Step 2:} Verify inner product axioms.
The inner product satisfies linearity, conjugate symmetry, and positive definiteness by direct verification using properties of infinite series.

\textbf{Step 3:} Prove completeness.
Let $(x^{(k)})_{k=1}^{\infty}$ be a Cauchy sequence in $\ell^2$. For each fixed $n \in \N$, the sequence $(x_n^{(k)})_{k=1}^{\infty}$ is Cauchy in $\C$ and hence converges to some limit $x_n$. 

One can show that $x = (x_0, x_1, x_2, \ldots) \in \ell^2$ and $x^{(k)} \to x$ in $\ell^2$ norm, establishing completeness.
\end{proof}

\section{Multiplication Operators}

\begin{definition}[Multiplication Operator]
Let $a = (a_0, a_1, a_2, \ldots) \in \ell^{\infty}$ be a bounded sequence. The multiplication operator $M_a: \ell^2 \to \ell^2$ is defined by
\[
M_a x = (a_0 x_0, a_1 x_1, a_2 x_2, \ldots)
\]
for all $x = (x_0, x_1, x_2, \ldots) \in \ell^2$.
\end{definition}

\begin{lemma}[Well-definedness of $M_a$]
If $a \in \ell^{\infty}$ and $x \in \ell^2$, then $M_a x \in \ell^2$.
\end{lemma}

\begin{proof}
For $x \in \ell^2$ and $a \in \ell^{\infty}$, one has
\[
\sum_{j=0}^{\infty} |a_j x_j|^2 \leq \norm{a}_{\infty}^2 \sum_{j=0}^{\infty} |x_j|^2 = \norm{a}_{\infty}^2 \norm{x}_2^2 < \infty.
\]
Therefore, $M_a x \in \ell^2$.
\end{proof}

\begin{theorem}[Operator Norm of Multiplication Operators]
Let $a \in \ell^{\infty}$ and let $M_a: \ell^2 \to \ell^2$ be the corresponding multiplication operator. Then
\[
\norm{M_a} = \norm{a}_{\infty} = \sup_{n \geq 0} |a_n|.
\]
\end{theorem}

\begin{proof}
The proof proceeds in two parts to establish both inequalities.

\textbf{Step 1:} Show $\norm{M_a} \leq \norm{a}_{\infty}$.
For any $x \in \ell^2$ with $\norm{x}_2 = 1$, one has
\begin{align}
\norm{M_a x}_2^2 &= \sum_{j=0}^{\infty} |a_j x_j|^2 \\
&\leq \sum_{j=0}^{\infty} |a_j|^2 |x_j|^2 \\
&\leq \norm{a}_{\infty}^2 \sum_{j=0}^{\infty} |x_j|^2 \\
&= \norm{a}_{\infty}^2 \norm{x}_2^2 \\
&= \norm{a}_{\infty}^2.
\end{align}
Taking the supremum over all unit vectors $x$, this gives $\norm{M_a} \leq \norm{a}_{\infty}$.

\textbf{Step 2:} Show $\norm{M_a} \geq \norm{a}_{\infty}$.
Let $\epsilon > 0$ be given. Since $\norm{a}_{\infty} = \sup_{n \geq 0} |a_n|$, there exists an index $n_0$ such that $|a_{n_0}| > \norm{a}_{\infty} - \epsilon$.

Consider the unit vector $e_{n_0} = (\delta_{n_0,0}, \delta_{n_0,1}, \delta_{n_0,2}, \ldots)$ where $\delta$ is the Kronecker delta. Then
\[
\norm{M_a e_{n_0}}_2 = \norm{(0, 0, \ldots, a_{n_0}, 0, \ldots)}_2 = |a_{n_0}| > \norm{a}_{\infty} - \epsilon.
\]
Since $\norm{e_{n_0}}_2 = 1$, this shows $\norm{M_a} \geq |a_{n_0}| > \norm{a}_{\infty} - \epsilon$.

Since $\epsilon > 0$ was arbitrary, one obtains $\norm{M_a} \geq \norm{a}_{\infty}$.

Combining both inequalities yields $\norm{M_a} = \norm{a}_{\infty}$.
\end{proof}

\begin{proposition}[Action on Canonical Basis]
Let $\{e_n\}_{n=0}^{\infty}$ denote the canonical orthonormal basis for $\ell^2$. Then
\[
M_a e_n = a_n e_n
\]
for all $n \geq 0$.
\end{proposition}

\begin{proof}
By definition, $e_n = (\delta_{n,0}, \delta_{n,1}, \delta_{n,2}, \ldots)$. Therefore,
\[
M_a e_n = (a_0 \delta_{n,0}, a_1 \delta_{n,1}, a_2 \delta_{n,2}, \ldots) = a_n e_n.
\]
This shows that multiplication operators are diagonal with respect to the canonical basis.
\end{proof}

\section{Algebraic Properties}

\begin{theorem}[Algebraic Structure of Multiplication Operators]
The set of multiplication operators on $\ell^2$ forms a commutative algebra under the operations:
\begin{enumerate}
\item $M_a + M_b = M_{a+b}$
\item $\lambda M_a = M_{\lambda a}$ for scalars $\lambda \in \C$
\item $M_a M_b = M_{ab}$ (componentwise product)
\end{enumerate}
\end{theorem}

\begin{proof}
Each property is verified by direct computation:

\textbf{Property 1:} For $x \in \ell^2$,
\[
(M_a + M_b)x = M_a x + M_b x = (a_0 x_0 + b_0 x_0, a_1 x_1 + b_1 x_1, \ldots) = M_{a+b}x.
\]

\textbf{Property 2:} For $x \in \ell^2$ and $\lambda \in \C$,
\[
(\lambda M_a)x = \lambda(M_a x) = (\lambda a_0 x_0, \lambda a_1 x_1, \ldots) = M_{\lambda a}x.
\]

\textbf{Property 3:} For $x \in \ell^2$,
\begin{align}
(M_a M_b)x &= M_a(M_b x) = M_a(b_0 x_0, b_1 x_1, \ldots) \\
&= (a_0 b_0 x_0, a_1 b_1 x_1, \ldots) = M_{ab}x.
\end{align}

Commutativity follows since $M_a M_b = M_{ab} = M_{ba} = M_b M_a$.
\end{proof}

\begin{theorem}[Adjoint of Multiplication Operators]
The adjoint of the multiplication operator $M_a$ is given by
\[
M_a^* = M_{\overline{a}},
\]
where $\overline{a} = (\overline{a_0}, \overline{a_1}, \overline{a_2}, \ldots)$ is the componentwise complex conjugate.
\end{theorem}

\begin{proof}
For $x, y \in \ell^2$, one computes
\begin{align}
\inner{M_a x}{y} &= \sum_{j=0}^{\infty} (a_j x_j) \overline{y_j} \\
&= \sum_{j=0}^{\infty} x_j \overline{a_j} \overline{y_j} \\
&= \sum_{j=0}^{\infty} x_j \overline{(\overline{a_j} y_j)} \\
&= \inner{x}{M_{\overline{a}} y}.
\end{align}
By the definition of the adjoint operator, this establishes $M_a^* = M_{\overline{a}}$.
\end{proof}

\section{Spectral Properties}

\begin{theorem}[Invertibility of Multiplication Operators]
The multiplication operator $M_a$ is invertible if and only if
\[
\inf_{n \geq 0} |a_n| > 0.
\]
When invertible, the inverse is given by $M_a^{-1} = M_{1/a}$ where $1/a = (1/a_0, 1/a_1, 1/a_2, \ldots)$.
\end{theorem}

\begin{proof}
\textbf{Necessity:} Suppose $M_a$ is invertible. If $\inf_{n \geq 0} |a_n| = 0$, then there exists a subsequence $(a_{n_k})$ such that $|a_{n_k}| \to 0$ as $k \to \infty$. 

For each $k$, consider the unit vector $e_{n_k}$. Then $M_a e_{n_k} = a_{n_k} e_{n_k}$, so $\norm{M_a e_{n_k}}_2 = |a_{n_k}| \to 0$. This contradicts the existence of $M_a^{-1}$ since $M_a$ would not be bounded below.

\textbf{Sufficiency:} Suppose $\delta := \inf_{n \geq 0} |a_n| > 0$. Define the sequence $1/a = (1/a_0, 1/a_1, 1/a_2, \ldots)$. Since $|1/a_n| = 1/|a_n| \leq 1/\delta$ for all $n$, one has $1/a \in \ell^{\infty}$.

For any $x \in \ell^2$,
\[
M_{1/a}(M_a x) = M_{1/a}(a_0 x_0, a_1 x_1, \ldots) = (x_0, x_1, x_2, \ldots) = x.
\]
Similarly, $M_a(M_{1/a} x) = x$. Therefore, $M_a^{-1} = M_{1/a}$.
\end{proof}

\begin{theorem}[Spectrum of Multiplication Operators]
The spectrum of the multiplication operator $M_a$ is given by
\[
\sigma(M_a) = \overline{\{a_n : n \geq 0\}},
\]
the closure of the range of the sequence $a$.
\end{theorem}

\begin{proof}
\textbf{Step 1:} Show $\{a_n : n \geq 0\} \subseteq \sigma(M_a)$.
For any $n \geq 0$, consider $\lambda = a_n$. Then $(M_a - \lambda I)e_n = a_n e_n - a_n e_n = 0$. Since $e_n \neq 0$, the operator $M_a - \lambda I$ is not injective, hence not invertible. Therefore, $a_n \in \sigma(M_a)$.

\textbf{Step 2:} Show $\sigma(M_a) \subseteq \overline{\{a_n : n \geq 0\}}$.
Let $\lambda \notin \overline{\{a_n : n \geq 0\}}$. Then there exists $\epsilon > 0$ such that $|\lambda - a_n| \geq \epsilon$ for all $n \geq 0$. This means the sequence $b = ((\lambda - a_0)^{-1}, (\lambda - a_1)^{-1}, \ldots)$ is bounded, so $M_b$ exists and satisfies $M_b(M_a - \lambda I) = (M_a - \lambda I)M_b = I$. Hence $\lambda \notin \sigma(M_a)$.

Since the spectrum is closed, one obtains $\sigma(M_a) = \overline{\{a_n : n \geq 0\}}$.
\end{proof}

\begin{remark}[Significance of $\ell^2$ Space]
The space $\ell^2$ serves as the prototypical separable infinite-dimensional Hilbert space. Every separable Hilbert space is isometrically isomorphic to $\ell^2$, making it the canonical model for quantum mechanical state spaces and numerous applications in functional analysis. The completeness of $\ell^2$ enables the development of spectral theory, projection theory, and the rich geometric structure that characterizes Hilbert space analysis.
\end{remark}

\end{document}
