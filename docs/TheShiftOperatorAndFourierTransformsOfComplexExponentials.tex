\documentclass{article}
\usepackage[english]{babel}
\usepackage{amsmath,amssymb,latexsym}

%%%%%%%%%% Start TeXmacs macros
\newcommand{\assign}{:=}
\newcommand{\tmtextbf}[1]{\text{{\bfseries{#1}}}}
\newenvironment{proof}{\noindent\textbf{Proof\ }}{\hspace*{\fill}$\Box$\medskip}
\newtheorem{corollary}{Corollary}
\newtheorem{theorem}{Theorem}
%%%%%%%%%% End TeXmacs macros

\begin{document}

\title{Stone's Theorem, Shift Group, and Fourier Transform}

\maketitle

\section*{Definitions}

\tmtextbf{Shift Group:} For $f \in L^2 (\mathbb{R})$, define the family of
unitary operators $(S_t)_{t \in \mathbb{R}}$ by
\[ (S_t f) (x) = f (x + t) . \]
\tmtextbf{Generator of Shift Group:} Define $A = \frac{d}{dx}$ on the domain
\[ D (A) = \{f \in L^2 (\mathbb{R}) \hspace{0.27em} : \hspace{0.27em} f' \in
   L^2 (\mathbb{R})\}, \]
where $f'$ is in the distributional sense.

\tmtextbf{Momentum Operator:} Define $P = - iA = - i \frac{d}{dx}$ on the same
domain $D (P) = D (A)$.

\tmtextbf{Fourier Transform:}
\[ \mathcal{F} [f] (\omega) = \hat{f} (\omega) \assign \int_{-
   \infty}^{\infty} f (x) e^{- i \omega x} dx. \]

\section*{Theorems and Proofs}

\begin{theorem}
  [Stone's Theorem Applied to Shift Group] The strongly continuous unitary
  group $(S_t)$ on $L^2 (\mathbb{R})$ has a densely defined skew-adjoint
  generator $A = \frac{d}{dx}$ such that $S_t = e^{tA}$. The generator
  satisfies
  \[ Af = \lim_{h \to 0}  \frac{S_h f - f}{h} \]
  in the $L^2$ topology on the domain $D (A)$.
\end{theorem}

\begin{proof}
  Let $f \in D (A)$. Then
  \[ \frac{S_h f (x) - f (x)}{h} = \frac{f (x + h) - f (x)}{h} \to f' (x) \]
  as $h \to 0$ in $L^2$ norm. Thus, the infinitesimal generator of $S_t$ is $A
  = \frac{d}{dx}$.
  
  To verify $A$ is skew-adjoint, for $f, g \in D (A)$:
  
  \begin{align}
    \langle Af, g \rangle & = \int_{- \infty}^{\infty} f' (x) \overline{g (x)}
    dx \\
    & = f (x) \overline{g (x)} |_{- \infty}^{\infty} - \int_{-
    \infty}^{\infty} f (x) \overline{g' (x)} dx \\
    & = 0 - \int_{- \infty}^{\infty} f (x) \overline{g' (x)} dx \\
    & = - \langle f, Ag \rangle 
  \end{align}
  
  Therefore $A^{\ast} = - A$, confirming $A$ is skew-adjoint.
\end{proof}

\begin{theorem}
  [Relation Between Generators] The shift group is generated by both the
  skew-adjoint operator $A = \frac{d}{dx}$ and the self-adjoint momentum
  operator $P = - iA$:
  \[ S_t = e^{tA} = e^{- itP} \]
\end{theorem}

\begin{proof}
  Since $P = - iA$, we have $- itP = - it (- iA) = - i^2 tA = tA$. Therefore:
  \[ e^{- itP} = e^{tA} \]
  For $f \in D (A)$, using the Taylor expansion:
  \[ e^{tA} f (x) = \sum_{n = 0}^{\infty} \frac{t^n}{n!} A^n f (x) = \sum_{n =
     0}^{\infty} \frac{t^n}{n!} f^{(n)} (x) = f (x + t) = S_t f (x) \]
\end{proof}

\begin{theorem}
  [Complex Exponentials Are Eigenfunctions] For any $\omega \in \mathbb{R}$:
  \begin{enumerate}
    \item $Ae^{i \omega x} = i \omega e^{i \omega x}$ (eigenvalue $i \omega$
    for skew-adjoint $A$)
    
    \item $Pe^{i \omega x} = \omega e^{i \omega x}$ (eigenvalue $\omega$ for
    self-adjoint $P$)
  \end{enumerate}
\end{theorem}

\begin{proof}
  Direct calculations:
  \begin{enumerate}
    \item $Ae^{i \omega x} = \frac{d}{dx} e^{i \omega x} = i \omega e^{i
    \omega x}$
    
    \item $Pe^{i \omega x} = - i \frac{d}{dx} e^{i \omega x} = - i (i \omega)
    e^{i \omega x} = \omega e^{i \omega x}$
  \end{enumerate}
\end{proof}

\begin{theorem}
  [Spectral Decomposition via Fourier Transform] Under the Fourier transform
  $\mathcal{F}$:
  \begin{enumerate}
    \item The self-adjoint momentum operator becomes multiplication by
    $\omega$: $\mathcal{F} [Pf] (\omega) = \omega \hat{f} (\omega)$
    
    \item The shift group becomes multiplication by a phase: $\mathcal{F} [S_t
    f] (\omega) = e^{i \omega t}  \hat{f} (\omega)$
  \end{enumerate}
\end{theorem}

\begin{proof}
  For part 1, if $f \in D (P)$:
  
  \begin{align}
    \mathcal{F} [Pf] (\omega) & = \int_{- \infty}^{\infty} (- if' (x)) e^{- i
    \omega x}  \hspace{0.17em} dx \\
    & = - i \int_{- \infty}^{\infty} f' (x) e^{- i \omega x}  \hspace{0.17em}
    dx 
  \end{align}
  
  Integration by parts (boundary terms vanish):
  
  \begin{align}
    & = - i [0 + i \omega \hat{f} (\omega)] = \omega \hat{f} (\omega) 
  \end{align}
  
  For part 2:
  
  \begin{align}
    \mathcal{F} [S_t f] (\omega) & = \int_{- \infty}^{\infty} f (x + t) e^{- i
    \omega x} dx 
  \end{align}
  
  Let $u = x + t$, so $x = u - t$, $dx = du$:
  
  \begin{align}
    & = \int_{- \infty}^{\infty} f (u) e^{- i \omega (u - t)} du \\
    & = e^{i \omega t}  \int_{- \infty}^{\infty} f (u) e^{- i \omega u} du =
    e^{i \omega t}  \hat{f} (\omega) 
  \end{align}
\end{proof}

\begin{theorem}
  [Eigenfunction Property of Shift Group] Complex exponentials are
  eigenfunctions of the shift group:
  \[ S_t e^{i \omega x} = e^{i \omega t} e^{i \omega x} \]
  with eigenvalue $e^{i \omega t}$.
\end{theorem}

\begin{proof}
  \[ S_t e^{i \omega x} = e^{i \omega (x + t)} = e^{i \omega x} e^{i \omega t}
     = e^{i \omega t} e^{i \omega x} \]
\end{proof}

\begin{corollary}
  [Consistency Check] The eigenvalue relationships are consistent:
  \[ S_t e^{i \omega x} = e^{tA} e^{i \omega x} = e^{t (i \omega)} e^{i \omega
     x} = e^{i \omega t} e^{i \omega x} \]
  since $A$ has eigenvalue $i \omega$ on $e^{i \omega x}$.
\end{corollary}

\section*{Conclusion}

Stone's theorem ensures that the shift group $(S_t)$ has a **skew-adjoint
generator** $A = \frac{d}{dx}$, whose eigenfunctions are the complex
exponentials $e^{i \omega x}$ with purely imaginary eigenvalues $i \omega$.
The related **self-adjoint momentum operator** $P = - iA$ has the same
eigenfunctions but with real eigenvalues $\omega$.

The Fourier transform provides the spectral decomposition that diagonalizes
both operators:
\begin{itemize}
  \item $P$ becomes multiplication by $\omega$ (real eigenvalues)
  
  \item $S_t$ becomes multiplication by $e^{i \omega t}$ (unitary eigenvalues
  on the unit circle)
\end{itemize}
This mathematical structure underlies all of Fourier analysis: complex
exponentials are the fundamental building blocks because they are precisely
the functions that transform simply under shifts, making them the natural
basis for analyzing translation-invariant systems. The distinction between the
skew-adjoint generator $A$ (with imaginary eigenvalues) and the self-adjoint
momentum operator $P$ (with real eigenvalues) is crucial for understanding why
unitary groups arise from self-adjoint operators via Stone's theorem.

\end{document}
