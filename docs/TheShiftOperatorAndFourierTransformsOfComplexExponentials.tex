\documentclass{article}
\usepackage{amsmath, amsthm, amssymb}
\usepackage[margin=1in]{geometry}

\theoremstyle{plain}
\newtheorem{theorem}{Theorem}
\newtheorem{lemma}[theorem]{Lemma}
\newtheorem{corollary}[theorem]{Corollary}

\theoremstyle{definition}
\newtheorem{definition}[theorem]{Definition}

\title{The Shift Group and Fourier Transform of Complex Exponentials}
\author{}
\date{}

\begin{document}

\maketitle

\begin{definition}[Shift Operator]
For $t_0 \in \mathbb{R}$, the shift operator $S_{t_0}$ is defined by
\[(S_{t_0}f)(t) = f(t - t_0)\]
for any function $f$ in the appropriate function space.
\end{definition}

\begin{definition}[Fourier Transform]
For $f \in L^1(\mathbb{R})$, the Fourier transform is defined by
\[\hat{f}(\omega) = \mathcal{F}[f](\omega) = \int_{-\infty}^{\infty} f(t)e^{-i\omega t} dt\]
\end{definition}

\begin{theorem}[Complex Exponentials as Eigenfunctions of Shift Operators]
Let $\omega \in \mathbb{R}$ and consider the complex exponential function $e_\omega(t) = e^{i\omega t}$. Then for any $t_0 \in \mathbb{R}$,
\[S_{t_0}[e_\omega] = e^{-i\omega t_0} \cdot e_\omega\]
That is, $e_\omega$ is an eigenfunction of $S_{t_0}$ with eigenvalue $e^{-i\omega t_0}$.
\end{theorem}

\begin{proof}
By definition of the shift operator:
\begin{align}
(S_{t_0}e_\omega)(t) &= e_\omega(t - t_0) \\
&= e^{i\omega(t - t_0)} \\
&= e^{i\omega t - i\omega t_0} \\
&= e^{-i\omega t_0} \cdot e^{i\omega t} \\
&= e^{-i\omega t_0} \cdot e_\omega(t)
\end{align}
Therefore, $S_{t_0}[e_\omega] = e^{-i\omega t_0} \cdot e_\omega$.
\end{proof}

\begin{theorem}[Time Shift Property of Fourier Transform]
Let $f \in L^1(\mathbb{R})$ and $t_0 \in \mathbb{R}$. Then
\[\mathcal{F}[S_{t_0}f](\omega) = e^{-i\omega t_0} \hat{f}(\omega)\]
\end{theorem}

\begin{proof}
Let $g(t) = f(t - t_0) = (S_{t_0}f)(t)$. Then:
\begin{align}
\mathcal{F}[g](\omega) &= \int_{-\infty}^{\infty} g(t)e^{-i\omega t} dt \\
&= \int_{-\infty}^{\infty} f(t - t_0)e^{-i\omega t} dt
\end{align}
Substituting $u = t - t_0$, so $t = u + t_0$ and $dt = du$:
\begin{align}
\mathcal{F}[g](\omega) &= \int_{-\infty}^{\infty} f(u)e^{-i\omega(u + t_0)} du \\
&= e^{-i\omega t_0} \int_{-\infty}^{\infty} f(u)e^{-i\omega u} du \\
&= e^{-i\omega t_0} \hat{f}(\omega)
\end{align}
\end{proof}

\begin{theorem}[Frequency Shift Property of Fourier Transform]
Let $f \in L^1(\mathbb{R})$ and $\omega_0 \in \mathbb{R}$. Then
\[\mathcal{F}[e^{i\omega_0 t}f(t)](\omega) = \hat{f}(\omega - \omega_0)\]
\end{theorem}

\begin{proof}
\begin{align}
\mathcal{F}[e^{i\omega_0 t}f(t)](\omega) &= \int_{-\infty}^{\infty} e^{i\omega_0 t}f(t)e^{-i\omega t} dt \\
&= \int_{-\infty}^{\infty} f(t)e^{-i(\omega - \omega_0)t} dt \\
&= \hat{f}(\omega - \omega_0)
\end{align}
\end{proof}

\begin{theorem}[Fourier Transform of Complex Exponential]
Let $\omega_0 \in \mathbb{R}$. In the distributional sense,
\[\mathcal{F}[e^{i\omega_0 t}](\omega) = 2\pi\delta(\omega - \omega_0)\]
where $\delta$ is the Dirac delta distribution.
\end{theorem}

\begin{proof}
We prove this by showing that for any test function $\phi \in \mathcal{S}(\mathbb{R})$ (Schwartz space):
\[\langle \mathcal{F}[e^{i\omega_0 t}], \phi \rangle = 2\pi\phi(\omega_0)\]

By definition of the Fourier transform of distributions:
\begin{align}
\langle \mathcal{F}[e^{i\omega_0 t}], \phi \rangle &= \langle e^{i\omega_0 t}, \mathcal{F}[\phi] \rangle \\
&= \int_{-\infty}^{\infty} e^{i\omega_0 t} \hat{\phi}(t) dt \\
&= \int_{-\infty}^{\infty} e^{i\omega_0 t} \int_{-\infty}^{\infty} \phi(\omega)e^{-i\omega t} d\omega dt
\end{align}

By Fubini's theorem (valid for $\phi \in \mathcal{S}(\mathbb{R})$):
\begin{align}
&= \int_{-\infty}^{\infty} \phi(\omega) \int_{-\infty}^{\infty} e^{i(\omega_0 - \omega)t} dt d\omega \\
&= \int_{-\infty}^{\infty} \phi(\omega) \cdot 2\pi\delta(\omega_0 - \omega) d\omega \\
&= 2\pi\phi(\omega_0) \\
&= \langle 2\pi\delta(\omega - \omega_0), \phi \rangle
\end{align}

Therefore, $\mathcal{F}[e^{i\omega_0 t}] = 2\pi\delta(\omega - \omega_0)$.
\end{proof}

\begin{theorem}[Diagonalization Property]
Let $T$ be a bounded linear operator on $L^2(\mathbb{R})$ that commutes with all shift operators, i.e., $TS_{t_0} = S_{t_0}T$ for all $t_0 \in \mathbb{R}$. Then $T$ is diagonalized by the Fourier transform in the sense that there exists a function $m(\omega)$ such that
\[\mathcal{F}[Tf] = m \cdot \mathcal{F}[f]\]
for all $f$ in the domain of $T$.
\end{theorem}

\begin{proof}
Since $T$ commutes with all shift operators, by Theorem 1, the complex exponentials $e^{i\omega t}$ are eigenfunctions of $T$. Let $Te^{i\omega t} = m(\omega)e^{i\omega t}$ for some function $m(\omega)$.

For any $f \in L^2(\mathbb{R})$ with $\hat{f} \in L^2(\mathbb{R})$, we can write (by the inverse Fourier transform):
\[f(t) = \frac{1}{2\pi} \int_{-\infty}^{\infty} \hat{f}(\omega)e^{i\omega t} d\omega\]

Applying $T$ and using linearity:
\begin{align}
Tf(t) &= T\left[\frac{1}{2\pi} \int_{-\infty}^{\infty} \hat{f}(\omega)e^{i\omega t} d\omega\right] \\
&= \frac{1}{2\pi} \int_{-\infty}^{\infty} \hat{f}(\omega)T[e^{i\omega t}] d\omega \\
&= \frac{1}{2\pi} \int_{-\infty}^{\infty} \hat{f}(\omega)m(\omega)e^{i\omega t} d\omega
\end{align}

Taking the Fourier transform:
\[\mathcal{F}[Tf](\omega) = m(\omega)\hat{f}(\omega) = m(\omega) \cdot \mathcal{F}[f](\omega)\]
\end{proof}

\begin{corollary}
The shift operators and multiplication by complex exponentials are the fundamental operations that generate all translation-invariant linear systems, and the Fourier transform provides the natural basis that simultaneously diagonalizes all such systems.
\end{corollary}

\end{document}
