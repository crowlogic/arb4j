\documentclass{article}
\usepackage[english]{babel}
\usepackage{geometry,amsmath,amssymb,latexsym}
\geometry{letterpaper}

%%%%%%%%%% Start TeXmacs macros
\newcommand{\assign}{:=}
\newcommand{\tmaffiliation}[1]{\\ #1}
\newcommand{\tmtextbf}[1]{\text{{\bfseries{#1}}}}
\newenvironment{proof}{\noindent\textbf{Proof\ }}{\hspace*{\fill}$\Box$\medskip}
\newtheorem{corollary}{Corollary}
\newtheorem{theorem}{Theorem}
%%%%%%%%%% End TeXmacs macros

\begin{document}

\title{Stone's Theorem, The Shift Group, and The Fourier Transform}

\author{
  Stephen Crowley
  \tmaffiliation{August 12, 2025}
}

\maketitle

{\tableofcontents}

\section*{Definitions}

\tmtextbf{Shift Group:} For $f \in L^2 (\mathbb{R})$, define the family of
unitary operators $(S_t)_{t \in \mathbb{R}}$ by
\[ (S_t f) (x) = f (x + t) . \]
\tmtextbf{Generator of Shift Group:} Define $A = \frac{d}{dx}$ on the domain
\[ D (A) = \{f \in L^2 (\mathbb{R}) \hspace{0.27em} : \hspace{0.27em} f' \in
   L^2 (\mathbb{R})\}, \]
where $f'$ is in the distributional sense.

\tmtextbf{Momentum Operator:} Define $P = - iA = - i \frac{d}{dx}$ on the same
domain $D (P) = D (A)$.

\tmtextbf{Fourier Transform:}
\[ \mathcal{F} [f] (\omega) = \hat{f} (\omega) \assign \int_{-
   \infty}^{\infty} f (x) e^{- i \omega x} dx. \]

\section*{Theorems and Proofs}

\begin{theorem}
  [Stone's Theorem Applied to Shift Group] The strongly continuous unitary
  group $(S_t)$ on $L^2 (\mathbb{R})$ has a densely defined skew-adjoint
  generator $A = \frac{d}{dx}$ such that $S_t = e^{tA}$. The generator
  satisfies
  \[ Af = \lim_{h \to 0}  \frac{S_h f - f}{h} \]
  in the $L^2$ topology on the domain $D (A)$.
\end{theorem}

\begin{proof}
  Let $f \in D (A)$. Then
  \[ \frac{S_h f (x) - f (x)}{h} = \frac{f (x + h) - f (x)}{h} \to f' (x) \]
  as $h \to 0$ in $L^2$ norm. Thus, the infinitesimal generator of $S_t$ is $A
  = \frac{d}{dx}$.
  
  To verify $A$ is skew-adjoint, for $f, g \in D (A)$:
  
  \begin{align}
    \langle Af, g \rangle & = \int_{- \infty}^{\infty} f' (x) \overline{g (x)}
    dx \\
    & = f (x) \overline{g (x)} |_{- \infty}^{\infty} - \int_{-
    \infty}^{\infty} f (x) \overline{g' (x)} dx \\
    & = 0 - \int_{- \infty}^{\infty} f (x) \overline{g' (x)} dx \\
    & = - \langle f, Ag \rangle 
  \end{align}
  
  Therefore $A^{\ast} = - A$, confirming $A$ is skew-adjoint.
\end{proof}

\begin{theorem}
  [Relation Between Generators] The shift group is generated by both the
  skew-adjoint operator $A = \frac{d}{dx}$ and the self-adjoint momentum
  operator $P = - iA$:
  \[ S_t = e^{tA} = e^{- itP} \]
\end{theorem}

\begin{proof}
  Since $P = - iA$, we have $- itP = - it (- iA) = - i^2 tA = tA$. Therefore:
  \[ e^{- itP} = e^{tA} \]
  For $f \in D (A)$, using the Taylor expansion:
  \[ e^{tA} f (x) = \sum_{n = 0}^{\infty} \frac{t^n}{n!} A^n f (x) = \sum_{n =
     0}^{\infty} \frac{t^n}{n!} f^{(n)} (x) = f (x + t) = S_t f (x) \]
\end{proof}

\begin{theorem}
  [Complex Exponentials Are Eigenfunctions] For any $\omega \in \mathbb{R}$:
  \begin{enumerate}
    \item $Ae^{i \omega x} = i \omega e^{i \omega x}$ (eigenvalue $i \omega$
    for skew-adjoint $A$)
    
    \item $Pe^{i \omega x} = \omega e^{i \omega x}$ (eigenvalue $\omega$ for
    self-adjoint $P$)
  \end{enumerate}
\end{theorem}

\begin{proof}
  Direct calculations:
  \begin{enumerate}
    \item $Ae^{i \omega x} = \frac{d}{dx} e^{i \omega x} = i \omega e^{i
    \omega x}$
    
    \item $Pe^{i \omega x} = - i \frac{d}{dx} e^{i \omega x} = - i (i \omega)
    e^{i \omega x} = \omega e^{i \omega x}$
  \end{enumerate}
\end{proof}

\begin{theorem}
  [Fourier Transform of Complex Exponential] Let $\omega_0 \in \mathbb{R}$. In
  the distributional sense,
  \[ \mathcal{F} [e^{i \omega_0 x}] (\omega) = 2 \pi \delta (\omega -
     \omega_0) \]
  where $\delta$ is the Dirac delta distribution.
\end{theorem}

\begin{proof}
  We prove this by showing that for any test function $\phi \in \mathcal{S}
  (\mathbb{R})$ (Schwartz space):
  \[ \langle \mathcal{F}[e^{i \omega_0 x}], \phi \rangle = 2 \pi \phi
     (\omega_0) \]
  By definition of the Fourier transform of distributions:
  
  \begin{align}
    \langle \mathcal{F}[e^{i \omega_0 x}], \phi \rangle & = \langle e^{i
    \omega_0 x}, \mathcal{F}[\phi] \rangle \\
    & = \int_{- \infty}^{\infty} e^{i \omega_0 x}  \hat{\phi} (x) dx \\
    & = \int_{- \infty}^{\infty} e^{i \omega_0 x}  \int_{- \infty}^{\infty}
    \phi (\omega) e^{- i \omega x} d \omega dx 
  \end{align}
  
  By Fubini's theorem (valid for $\phi \in \mathcal{S} (\mathbb{R})$):
  
  \begin{align}
    & = \int_{- \infty}^{\infty} \phi (\omega)  \int_{- \infty}^{\infty} e^{i
    (\omega_0 - \omega) x} dxd \omega \\
    & = \int_{- \infty}^{\infty} \phi (\omega) \cdot 2 \pi \delta (\omega_0 -
    \omega) d \omega \\
    & = 2 \pi \phi (\omega_0) \\
    & = \langle 2 \pi \delta (\omega - \omega_0), \phi \rangle 
  \end{align}
  
  Therefore, $\mathcal{F} [e^{i \omega_0 x}] = 2 \pi \delta (\omega -
  \omega_0)$.
\end{proof}

\begin{theorem}
  [Inverse Fourier Transform of Dirac Delta] In the distributional sense,
  \[ \mathcal{F}^{- 1}  [\delta (\omega - \omega_0)] (x) = \frac{1}{2 \pi}
     e^{i \omega_0 x} \]
\end{theorem}

\begin{proof}
  From the previous theorem, we have $\mathcal{F} [e^{i \omega_0 x}] = 2 \pi
  \delta (\omega - \omega_0)$. Applying $\mathcal{F}^{- 1}$ to both sides:
  \[ e^{i \omega_0 x} =\mathcal{F}^{- 1}  [2 \pi \delta (\omega - \omega_0)]
  \]
  Therefore:
  \[ \mathcal{F}^{- 1}  [\delta (\omega - \omega_0)] = \frac{1}{2 \pi} e^{i
     \omega_0 x} \]
\end{proof}

\begin{theorem}
  [Spectral Decomposition via Fourier Transform] Under the Fourier transform
  $\mathcal{F}$:
  \begin{enumerate}
    \item The self-adjoint momentum operator becomes multiplication by
    $\omega$: $\mathcal{F} [Pf] (\omega) = \omega \hat{f} (\omega)$
    
    \item The shift group becomes multiplication by a phase: $\mathcal{F} [S_t
    f] (\omega) = e^{i \omega t}  \hat{f} (\omega)$
  \end{enumerate}
\end{theorem}

\begin{proof}
  For part 1, if $f \in D (P)$:
  
  \begin{align}
    \mathcal{F} [Pf] (\omega) & = \int_{- \infty}^{\infty} (- if' (x)) e^{- i
    \omega x}  \hspace{0.17em} dx \\
    & = - i \int_{- \infty}^{\infty} f' (x) e^{- i \omega x}  \hspace{0.17em}
    dx 
  \end{align}
  
  Integration by parts (boundary terms vanish):
  
  \begin{align}
    & = - i [0 + i \omega \hat{f} (\omega)] = \omega \hat{f} (\omega) 
  \end{align}
  
  For part 2:
  
  \begin{align}
    \mathcal{F} [S_t f] (\omega) & = \int_{- \infty}^{\infty} f (x + t) e^{- i
    \omega x} dx 
  \end{align}
  
  Let $u = x + t$, so $x = u - t$, $dx = du$:
  
  \begin{align}
    & = \int_{- \infty}^{\infty} f (u) e^{- i \omega (u - t)} du \\
    & = e^{i \omega t}  \int_{- \infty}^{\infty} f (u) e^{- i \omega u} du =
    e^{i \omega t}  \hat{f} (\omega) 
  \end{align}
\end{proof}

\begin{theorem}
  [Eigenfunction Property of Shift Group] Complex exponentials are
  eigenfunctions of the shift group:
  \[ S_t e^{i \omega x} = e^{i \omega t} e^{i \omega x} \]
  with eigenvalue $e^{i \omega t}$.
\end{theorem}

\begin{proof}
  \[ S_t e^{i \omega x} = e^{i \omega (x + t)} = e^{i \omega x} e^{i \omega t}
     = e^{i \omega t} e^{i \omega x} \]
\end{proof}

\begin{theorem}
  [Spectral Representation of Identity] The identity operator can be
  represented using the Dirac delta:
  \[ I = \frac{1}{2 \pi}  \int_{- \infty}^{\infty} |e^{i \omega \cdot} \rangle
     \langle e^{i \omega \cdot} |d \omega \]
  where in distributional form:
  \[ \delta (x - y) = \frac{1}{2 \pi}  \int_{- \infty}^{\infty} e^{i \omega (x
     - y)} d \omega \]
\end{theorem}

\begin{proof}
  For any test function $f$:
  
  \begin{align}
    \frac{1}{2 \pi}  \int_{- \infty}^{\infty} \int_{- \infty}^{\infty} f (y)
    e^{i \omega (x - y)} dyd \omega & = \int_{- \infty}^{\infty} f (y)  \left[
    \frac{1}{2 \pi}  \int_{- \infty}^{\infty} e^{i \omega (x - y)} d \omega
    \right] dy \\
    & = \int_{- \infty}^{\infty} f (y) \delta (x - y) dy \\
    & = f (x) 
  \end{align}
  
  This shows the spectral representation of the identity using the continuous
  spectrum of the momentum operator.
\end{proof}

\begin{corollary}
  [Consistency Check] The eigenvalue relationships are consistent:
  \[ S_t e^{i \omega x} = e^{tA} e^{i \omega x} = e^{t (i \omega)} e^{i \omega
     x} = e^{i \omega t} e^{i \omega x} \]
  since $A$ has eigenvalue $i \omega$ on $e^{i \omega x}$.
\end{corollary}

\section*{Conclusion}

Stone's theorem ensures that the shift group $(S_t)$ has a skew-adjoint
generator $A = \frac{d}{dx}$, whose eigenfunctions are the complex
exponentials $e^{i \omega x}$ with purely imaginary eigenvalues $i \omega$.
The related **self-adjoint momentum operator** $P = - iA$ has the same
eigenfunctions but with real eigenvalues $\omega$.

The Dirac delta function emerges naturally as the Fourier transform of complex
exponentials, providing the spectral measure for the continuous spectrum of
the momentum operator. This gives us the fundamental relationships:
\begin{itemize}
  \item Complex exponentials $e^{i \omega_0 x} \leftrightarrow 2 \pi \delta
  (\omega - \omega_0)$ under Fourier transform
  
  \item The identity operator has the spectral representation $I = \frac{1}{2
  \pi}  \int_{- \infty}^{\infty} |e^{i \omega \cdot} \rangle \langle e^{i
  \omega \cdot} |d \omega$
  
  \item The delta function provides the orthogonality relation for the
  continuous eigenfunction basis
\end{itemize}
The Fourier transform provides the spectral decomposition that diagonalizes
both operators, with the Dirac delta serving as the key distributional tool
that makes the continuous spectrum rigorous. This mathematical structure
underlies all of Fourier analysis and quantum mechanics, where complex
exponentials are the fundamental building blocks precisely because they
diagonalize translation-invariant systems.

\

\end{document}
