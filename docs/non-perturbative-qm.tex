\documentclass{article}
\usepackage[english]{babel}

\providecommand{\tightlist}{}

\begin{document}

\date{}

\maketitle

In a non-perturbative approach, we aim to solve the time-dependent
Schr{\"o}dinger equation exactly, without relying on perturbative
approximations. The overlap integral plays a crucial role in this context as
well.

Consider the time-dependent Schr{\"o}dinger equation:
\begin{equation}
  i \hbar \frac{\partial}{\partial t} | \psi (t) \rangle = \hat{H} | \psi (t)
  \rangle
\end{equation}
where $\hat{H}$ is the Hamiltonian operator. If $\hat{H}$ is time-independent,
we can formally solve this equation as:
\begin{equation}
  | \psi (t) \rangle = e^{- i \hat{H} t / \hbar} | \psi (0) \rangle
\end{equation}
Here, $| \psi (0) \rangle$ is the initial state at time $t = 0$. The
exponential operator $e^{- i \hat{H} t / \hbar}$ is the time-evolution
operator, which governs the unitary evolution of the quantum state.

To calculate the probability amplitude of finding the system in an energy
eigenstate $| \lambda \rangle$ at time $t$, we compute the overlap integral:
\[ \langle \lambda | \psi (t) \rangle = \langle \lambda |e^{- i \hat{H} t /
   \hbar} | \psi (0) \rangle \]
Using the spectral decomposition of the Hamiltonian,
\begin{equation}
  \hat{H} = \sum_{\lambda} \lambda | \lambda \rangle \langle \lambda |
\end{equation}
, we can express this as:
\[ \langle \lambda | \psi (t) \rangle = e^{- i \lambda t / \hbar} \langle
   \lambda | \psi (0) \rangle \]
The probability of measuring the system in the eigenstate $| \lambda \rangle$
at time $t$ is then given by:
\begin{equation}
  P (\lambda, t) = | \langle \lambda | \psi (t) \rangle |^2 = | \langle
  \lambda | \psi (0) \rangle |^2
\end{equation}
This result shows that the probability is time-independent, a consequence of
the stationary nature of energy eigenstates.

To obtain the average value of an observable $\hat{A}$ at time $t$, we use the
expectation value:
\begin{equation}
  \langle \hat{A} \rangle_t = \langle \psi (t) | \hat{A} | \psi (t) \rangle =
  \sum_{\lambda} \langle \psi (t) | \lambda \rangle  \langle \lambda | \hat{A}
  | \psi (t) \rangle
\end{equation}
By evaluating these overlap integrals and summing over the energy eigenstates,
we can compute the time-dependent expectation values of observables without
resorting to perturbation theory.

This non-perturbative approach provides exact solutions, but it relies on our
ability to diagonalize the Hamiltonian and find its eigenstates and
eigenvalues.

\end{document}
