\documentclass{article}
\usepackage[english]{babel}
\usepackage{amssymb,latexsym}

%%%%%%%%%% Start TeXmacs macros
\newcommand{\assign}{:=}
\newcommand{\nin}{\not\in}
\newcommand{\tmaffiliation}[1]{\\ #1}
\newenvironment{proof}{\noindent\textbf{Proof\ }}{\hspace*{\fill}$\Box$\medskip}
\newtheorem{corollary}{Corollary}
\newtheorem{theorem}{Theorem}
%%%%%%%%%% End TeXmacs macros

\begin{document}

{\cdot}\title{Branch-Cut Discontinuities of $\frac{1}{2} \log \xi \left(
\frac{1}{2} + it \right)$}

\author{
  Stephen Crowley
  \tmaffiliation{August 28, 2025}
}

\maketitle

\begin{theorem}
  \label{thm:branch_cut_discontinuities}Let $\xi (s)$ denote Riemann's
  $\xi$-function, which is entire, and fix a single-valued branch
  $\mathrm{Log}$ of the complex logarithm on $\mathbb{C} \setminus (- \infty,
  0]$ (the principal branch). Consider the function
  \begin{equation}
    F (t) = \frac{1}{2} \hspace{0.17em} \mathrm{Log} \left( \xi \left(
    \frac{1}{2} + it \right) \right) \forall t \in \mathbb{R}
  \end{equation}
  defined wherever $\xi \left( \frac{1}{2} + it \right) \nin (- \infty, 0]$.
  Then the following hold:
  \begin{enumerate}
    \item \label{item:continuity} $F$ is continuous on any maximal open
    interval of $t$ for which $\xi \left( \frac{1}{2} + it \right)$ avoids $(-
    \infty, 0]$.
    
    \item \label{item:jump_discontinuity} At any $t_0$ with $\xi \left(
    \frac{1}{2} + it_0 \right) \in (- \infty, 0]$ and $\xi \left( \frac{1}{2}
    + it_0 \right) \neq 0$, the one-sided limits of $F$ exist and satisfy
    \begin{equation}
      \lim_{t \to t_0^+} F (t) - \lim_{t_0^-} F (t) = \pi i
    \end{equation}
    i.e., $F$ exhibits a jump discontinuity of size $\pi i$ (equivalently,
    $\mathrm{Log}$ jumps by $2 \pi i$ and the prefactor $\frac{1}{2}$ halves
    the jump).
    
    \item \label{item:characterization} The set of $t$ at which these
    discontinuities occur is precisely the preimage of the negative real axis
    under the map $t \mapsto \xi \left( \frac{1}{2} + it \right)$, excluding
    zeros of $\xi$; consequently, the observed discontinuities are branch-cut
    crossings of $\mathrm{Log} \circ \xi$ and not singularities of $\xi$.
  \end{enumerate}
\end{theorem}

\begin{proof}
  The proof proceeds in three steps corresponding to parts
  \eqref{item:continuity}, \eqref{item:jump_discontinuity}, and
  \eqref{item:characterization} of
  Theorem~\ref{thm:branch_cut_discontinuities}.
  
  \paragraph{Step 1: Proof of \eqref{item:continuity}.} Since $\xi (s)$ is
  entire by construction, $\xi$ has no poles or branch points in $s$ and $\xi
  \left( \frac{1}{2} + it \right)$ is a continuous function of $t$ into
  $\mathbb{C}$. Since the only multivalued object in $F$ is $\mathrm{Log}$,
  all discontinuities of $F$ must arise from the branch structure of
  $\mathrm{Log}$ applied to the continuous path $t \mapsto \xi \left(
  \frac{1}{2} + it \right)$. This establishes the reduction to the logarithm.
  
  The principal branch $\mathrm{Log}$ with branch cut along $(- \infty, 0]$ is
  analytic and thus continuous on $\mathbb{C} \setminus (- \infty, 0]$.
  Therefore $F$ is continuous at any $t_0$ for which $\xi \left( \frac{1}{2} +
  it_0 \right) \nin (- \infty, 0]$.
  
  \paragraph{Step 2: Proof of \eqref{item:jump_discontinuity}.} Let $w (t)
  \assign \xi \left( \frac{1}{2} + it \right)$ and suppose $w (t_0) \in (-
  \infty, 0]$ with $w (t_0) \neq 0$. Because $w$ is continuous and nonzero at
  $t_0$, there exists $\delta > 0$ such that:
  \begin{enumerate}
    \item $w (t) \neq 0$ for $|t - t_0 | < \delta$, and
    
    \item the image $w ((t_0 - \delta, t_0 + \delta))$ crosses the branch cut
    transversely at $w (t_0)$.
  \end{enumerate}
  Approaching $w (t_0)$ from the upper half-plane corresponds to arguments
  $\arg w (t) \to \pi^-$, while from the lower half-plane corresponds to $\arg
  w (t) \to (- \pi)^+$ (principal values). Hence
  \begin{equation}
    \lim_{t \to t_0^+} \mathrm{Log} w (t) - \lim_{t \to t_0^-} \mathrm{Log} w
    (t) = 2 \pi i
  \end{equation}
  the standard $2 \pi i$ jump across the negative real axis for the principal
  logarithm. Multiplying by $\frac{1}{2}$ yields the stated jump $\pi i$ for
  $F$.
  
  \paragraph{Step 3: Proof of \eqref{item:characterization}.} By definition,
  the principal branch is continuous precisely off its branch cut. Therefore
  one-sided jumps can only occur when the continuous path $w (t)$ intersects
  the branch cut, i.e., when $w (t) \in (- \infty, 0]$.
  
  If $w (t) = 0$, then $\mathrm{Log} w (t)$ is undefined and a different
  analysis is required (zeros are branch points of $\mathrm{Log} \circ w$),
  but by hypothesis these are excluded from consideration in
  part~\eqref{item:jump_discontinuity}.
  
  Conversely, every transverse crossing of $(- \infty, 0]$ yields the jump
  quantified in part~\eqref{item:jump_discontinuity}. Thus the discontinuity
  set is exactly $w^{- 1} ((- \infty, 0]) \setminus w^{- 1} (\{0\})$.
\end{proof}

\begin{corollary}
  \label{cor:specific_locations}Any specific numerical locations of
  discontinuities (e.g., $t = \pm (e - 1)$ in a given plot) are exactly the
  real parameters for which $\xi \left( \frac{1}{2} + it \right)$ lands on the
  negative real axis under the chosen branch. These values therefore solve
  \begin{equation}
    \arg \xi \left( \frac{1}{2} + it \right) \equiv \pi \pmod{2 \pi}
  \end{equation}
\end{corollary}

\begin{corollary}
  \label{cor:nature_of_discontinuities}Since $\xi$ is entire, the
  discontinuities described in Theorem~\ref{thm:branch_cut_discontinuities}
  are not singularities of $\xi$; they arise solely from composing the entire
  function $\xi$ with a single-valued branch of $\mathrm{Log}$ along the
  critical line.
\end{corollary}

\end{document}
