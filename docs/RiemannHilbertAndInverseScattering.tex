\documentclass{article}
\usepackage[english]{babel}
\usepackage{geometry,amsmath,amssymb}
\geometry{letterpaper}

%%%%%%%%%% Start TeXmacs macros
\newcommand{\tmaffiliation}[1]{\\ #1}
\newcommand{\tmtextbf}[1]{\text{{\bfseries{#1}}}}
%%%%%%%%%% End TeXmacs macros

\begin{document}

\title{Answering The Riemann-Hilbert Question With Methods From The Theory of
Inverse Scattering}

\author{
  Stephen Crowley
  \tmaffiliation{April 15, 2024}
}

\maketitle

\

\

The connection between the Riemann-Hilbert problem and the
Gelfand-Levitan-Marchenko (GLM) inverse scattering method is foundational in
linking analytical and algebraic structures within integrable systems. Here we
explore this relationship:
\begin{enumerate}
  \item \tmtextbf{Inverse Scattering and Integrable Systems:} The GLM inverse
  scattering method is crucial for solving integrable differential equations
  such as the Korteweg-de Vries (KdV) equation, which is a significant
  nonlinear partial differential equation in mathematical physics. This
  equation is given by:
  \begin{equation}
    \frac{\partial u}{\partial t} + 6 u \frac{\partial u}{\partial x} +
    \frac{\partial^3 u}{\partial x^3} = 0
  \end{equation}
  This method reconstructs potentials from the provided scattering data,
  characterizing the evolution of solitons or wave functions.
  
  \item \tmtextbf{Riemann-Hilbert Problem Formulation:} This involves finding
  a matrix-valued function $Y (z)$ that is analytic off a real line
  $\mathbb{R}$ and satisfies the boundary condition:
  \begin{equation}
    Y_+ (z) = Y_- (z) \cdot v (z) \forall z \in \mathbb{R}
  \end{equation}
  where $v (z)$ represents the jump matrix derived from scattering data.
  
  \item \tmtextbf{Connection through Jump Conditions:} The GLM method's kernel
  $K (x, t)$ related to the system's potential, can be derived from the
  scattering data $S (k)$. The integral equation in GLM form is given by:
  \begin{equation}
    K (x, y) + F (x + y) + \int_x^{\infty} K (x, z) F (z + y)  \hspace{0.17em}
    dz = 0 \forall x, y \geq 0
  \end{equation}
  where $F$ is related to the inverse Fourier transform of $S (k)$.
  
  \item \tmtextbf{Analytic to Algebraic Translation:} Solving the analytic
  Riemann-Hilbert problem translates into solving algebraic GLM integral
  equations, providing a framework for understanding how changes in the
  contour or boundary conditions impact the solutions.
  
  \item \tmtextbf{Applications in Soliton Theory:} The solution to the
  Riemann-Hilbert problem facilitates the explicit expression of solitons,
  which are solutions to the KdV and other nonlinear wave equations. The
  soliton solutions $q (x, t)$ are often expressed as:
  \begin{equation}
    q (x, t) = - 2 \frac{d^2}{dx^2} \log \det (I + K)
  \end{equation}
  \item \tmtextbf{Theoretical Implications:} Beyond understanding soliton
  dynamics, solving the Riemann-Hilbert problem has broader implications for
  the stability analysis of solitons and the asymptotic behavior of solutions
  in integrable systems.
\end{enumerate}

\end{document}
