\documentclass[11pt]{article}
\usepackage{amsmath,amssymb,amsthm,amsfonts}
\usepackage{hyperref}
\usepackage{geometry}
\geometry{margin=1in}

\newtheorem{theorem}{Theorem}[section]
\newtheorem{lemma}[theorem]{Lemma}
\newtheorem{proposition}[theorem]{Proposition}
\newtheorem{corollary}[theorem]{Corollary}
\theoremstyle{definition}
\newtheorem{definition}[theorem]{Definition}
\newtheorem{remark}[theorem]{Remark}

\newcommand{\assign}{\coloneqq}

\title{Unitarily Time-Changed Stationary Processes:\\ A Subclass of Oscillatory Processes}
\author{Stephen Crowley}
\date{December 13, 2025}

\begin{document}
\maketitle

\begin{abstract}
This article provides a complete and rigorous exposition of the Ces{\`a}ro stationarity result for the Hardy Z-function viewed as a unitarily time-changed stationary process. We provide explicit verification of foundational asymptotic expansions and give detailed theoretical justification for each calculation. The key result establishes that the inverse unitary transform of the Hardy Z-function possesses a well-defined stationary covariance structure in the Ces{\`a}ro sense, confirming that the Hardy Z-function is a concrete instance of a unitarily time-changed oscillatory process.
\end{abstract}

\tableofcontents

\section{Introduction}

The Hardy Z-function, defined as
\[
Z(t)=e^{i\theta(t)}\zeta(\tfrac12+it),
\]
where $\theta(t)$ is the Riemann-Siegel theta function, has been the subject of intense study in analytic number theory. The purpose of this paper is to demonstrate that $Z(t)$ is a unitarily time-changed stationary process which is identified as a proper subclass of oscillatory processes. Specifically, there exists a strictly increasing, absolutely continuous function $\Theta(t)$ such that the inverse unitary operator $U_{\Theta}^{-1}$ applied to $Z(t)$ yields an underlying stationary process $X(u)$ with well-defined Ces{\`a}ro covariance.

This article develops the complete theoretical foundation for this claim, with all proofs rigorously justified.

\section{Preliminary Theory}

\subsection{The Unitary Time-Change Operator}

\begin{definition}
[Unitary Time-Change Operator] Let $\Theta:\mathbb{R}\to\mathbb{R}$ be absolutely continuous, strictly increasing, and bijective with $\dot{\Theta}(t)>0$ almost everywhere. For measurable $f$, define:
\[
(U_{\Theta}f)(t)=\sqrt{\dot{\Theta}(t)}\,f(\Theta(t))
\]
The inverse operator is:
\[
(U_{\Theta}^{-1}g)(w)=\frac{g(\Theta^{-1}(w))}{\sqrt{\dot{\Theta}(\Theta^{-1}(w))}}
\]
\end{definition}

\begin{theorem}
[Local Isometry and Inverse Properties] For every compact $K\subseteq\mathbb{R}$ and $f\in L^2_{\mathrm{loc}}(\mathbb{R})$:
\[
\int_K|(U_{\Theta}f)(t)|^2\,dt=\int_{\Theta(K)}|f(s)|^2\,ds
\]
Moreover, for $f,g\in L^2_{\mathrm{loc}}(\mathbb{R})$:
\[
U_{\Theta}^{-1}(U_{\Theta}f)=f,\qquad U_{\Theta}(U_{\Theta}^{-1}g)=g
\]
\end{theorem}

\begin{proof}
\textbf{Part 1: Local Isometry.} Using change of variables $s=\Theta(t)$, $ds=\dot{\Theta}(t)\,dt$:
\[
\int_K|(U_{\Theta}f)(t)|^2\,dt=\int_K\dot{\Theta}(t)|f(\Theta(t))|^2\,dt=\int_{\Theta(K)}|f(s)|^2\,ds
\]

\textbf{Part 2: Inverse Identity $U_{\Theta}^{-1}(U_{\Theta}f)=f$.} For any $w$ in the range of $\Theta$, let $t=\Theta^{-1}(w)$. Then:
\[
(U_{\Theta}^{-1}(U_{\Theta}f))(w)=\frac{(U_{\Theta}f)(\Theta^{-1}(w))}{\sqrt{\dot{\Theta}(\Theta^{-1}(w))}}
=\frac{\sqrt{\dot{\Theta}(\Theta^{-1}(w))}f(\Theta(\Theta^{-1}(w)))}{\sqrt{\dot{\Theta}(\Theta^{-1}(w))}}=f(w)
\]

\textbf{Part 3: Inverse Identity $U_{\Theta}(U_{\Theta}^{-1}g)=g$.} For any $t\in\mathbb{R}$:
\[
U_{\Theta}(U_{\Theta}^{-1}g)(t)=\sqrt{\dot{\Theta}(t)}\,(U_{\Theta}^{-1}g)(\Theta(t))
=\sqrt{\dot{\Theta}(t)}\cdot\frac{g(\Theta^{-1}(\Theta(t)))}{\sqrt{\dot{\Theta}(\Theta^{-1}(\Theta(t)))}}=g(t)
\]
where $\Theta^{-1}(\Theta(t))=t$.
\end{proof}

\subsection{Oscillatory Processes}

\begin{definition}
[Oscillatory Process] Let $F$ be a finite nonnegative Borel measure on $\mathbb{R}$. An oscillatory process is a stochastic process of the form:
\[
Z(t)=\int_{\mathbb{R}}A_t(\lambda)\,e^{i\lambda t}\,d\Phi(\lambda)
\]
where $A_t(\lambda)\in L^2(F)$ for all $t$, and $\Phi$ is a complex orthogonal random measure with spectral measure $F$.
\end{definition}

\begin{theorem}
[Unitary Time Change Produces Oscillatory Process] Let $X$ be a zero-mean stationary process with Cram\'er spectral representation $X(t)=\int_{\mathbb{R}}e^{i\lambda t}\,d\Phi(\lambda)$. Let $\Theta$ satisfy Definition 2.1. Then:
\[
Z(t)=(U_{\Theta}X)(t)=\sqrt{\dot{\Theta}(t)}\,X(\Theta(t))
\]
is an oscillatory process with $\phi_t(\lambda)=\sqrt{\dot{\Theta}(t)}\,e^{i\lambda\Theta(t)}$ and gain function $A_t(\lambda)=\sqrt{\dot{\Theta}(t)}\,e^{i\lambda(\Theta(t)-t)}$.
\end{theorem}

\begin{proof}
Substituting $t\mapsto\Theta(t)$ in the Cram\'er representation:
\[
Z(t)=\sqrt{\dot{\Theta}(t)}\int_{\mathbb{R}}e^{i\lambda\Theta(t)}\,d\Phi(\lambda)=\int_{\mathbb{R}}\sqrt{\dot{\Theta}(t)}\,e^{i\lambda\Theta(t)}\,d\Phi(\lambda)
\]
Thus $\phi_t(\lambda)=\sqrt{\dot{\Theta}(t)}\,e^{i\lambda\Theta(t)}$. Since $\phi_t(\lambda)=A_t(\lambda)e^{i\lambda t}$:
\[
A_t(\lambda)=\phi_t(\lambda)e^{-i\lambda t}=\sqrt{\dot{\Theta}(t)}\,e^{i\lambda(\Theta(t)-t)}
\]
\end{proof}

\subsection{Zero Localization}

\subsubsection{Kac-Rice Formula for Unitarily Time-Changed Processes}

\begin{corollary}
[Kac-Rice Formula for Expected Zero Crossings] Let $Z(t)=\sqrt{\dot{\theta}(t)}\,X(\theta(t))$ where $X$ is a zero-mean stationary Gaussian process with covariance $R_X$ and $R_X''(0)<0$. Then the expected number of zeros of $Z(t)$ in $[a,b]$ is:
\[
\mathbb{E}[N_{[a,b]}]=\frac{\theta(b)-\theta(a)}{\pi}\sqrt{\frac{-R_X''(0)}{R_X(0)}}
\]
\end{corollary}

\begin{proof}
At any time $t$, the covariance of $Z$ is:
\[
K_Z(t,s)=\sqrt{\dot{\theta}(t)\dot{\theta}(s)}\,R_X(\theta(t)-\theta(s))
\]
At $s=t$, the variance is $K_Z(t,t)=\dot{\theta}(t)R_X(0)$.

Consider a zero $t_0$ where $Z(t_0)=0$. Since $\dot{\theta}(t_0)>0$, we must have $X(\theta(t_0))=0$. Differentiating $Z(t)$:
\[
\dot{Z}(t)=\frac{\ddot{\theta}(t)}{2\sqrt{\dot{\theta}(t)}}X(\theta(t))+\dot{\theta}(t)^{3/2}\dot{X}(\theta(t))
\]
At the zero $t_0$, the first term vanishes (because $X(\theta(t_0))=0$), leaving:
\[
\dot{Z}(t_0)=\dot{\theta}(t_0)^{3/2}\dot{X}(\theta(t_0))
\]

For the stationary process $X$, Bulinskaya's theorem (Theorem 2.4 below) shows that conditional on $X(u)=0$, the derivative $\dot{X}(u)$ is independent Gaussian with mean zero and variance $-R_X''(0)$. Therefore:
\[
\mathbb{E}[|\dot{Z}(t_0)|\mid Z(t_0)=0]=\dot{\theta}(t_0)^{3/2}\sqrt{\frac{2}{\pi}(-R_X''(0))}
\]

The marginal density of $Z(t)$ at zero is:
\[
p_{Z(t)}(0)=\frac{1}{\sqrt{2\pi K_Z(t,t)}}=\frac{1}{\sqrt{2\pi\dot{\theta}(t)R_X(0)}}
\]

By the Kac-Rice meta-theorem, the zero density is:
\[
\rho(t)=p_{Z(t)}(0)\,\mathbb{E}[|\dot{Z}(t)|\mid Z(t)=0]=\frac{\dot{\theta}(t)}{\pi}\sqrt{\frac{-R_X''(0)}{R_X(0)}}
\]

Integrating from $a$ to $b$ gives the result.
\end{proof}

\subsubsection{Bulinskaya's Theorem on Simplicity of Zero Crossings}

\begin{theorem}
[Bulinskaya's Theorem] Let $X(t)$ be a real-valued, zero-mean stationary Gaussian process with covariance function $R(h)=\mathbb{E}[X(t)X(t+h)]$. Suppose $R(h)$ is twice continuously differentiable in a neighborhood of $h=0$ with $R''(0)<0$. Then almost surely all zeros of $X(t)$ are simple:
\[
X(t_0)=0\quad\Rightarrow\quad\dot{X}(t_0)\neq0\quad\text{almost surely}
\]
\end{theorem}

\begin{proof}
The twice continuous differentiability of $R(h)$ ensures $X(t)$ has mean-square continuous first derivative $\dot{X}(t)$, and $(X(t),\dot{X}(t))$ is jointly Gaussian. For a stationary process, $R'(0)=0$, giving zero covariance:
\[
\mathbb{E}[X(t)\dot{X}(t)]=R'(0)=0
\]
Thus the covariance matrix is $\operatorname{diag}(R(0),-R''(0))$, so $X(t)$ and $\dot{X}(t)$ are independent at each fixed $t$.

For any fixed $t_0$, $P(X(t_0)=0)=0$ (continuous Gaussian). By independence:
\[
P(X(t_0)=0\text{ and }\dot{X}(t_0)=0)=P(X(t_0)=0)P(\dot{X}(t_0)=0)=0
\]

The Kac-Rice formula shows the expected number of zeros in any interval is finite, so zeros form a discrete (hence countable) set. Taking a countable union over all zeros gives probability zero of any double zero occurring.
\end{proof}

\begin{corollary}
Let $Z(t)=\sqrt{\dot{\theta}(t)}\,X(\theta(t))$ be the unitarily time-changed stationary Gaussian process where $X$ has twice continuously differentiable covariance with $R_X''(0)<0$. Then almost surely all zeros of $Z(t)$ are simple.
\end{corollary}

\begin{proof}
At a zero $t_0$ of $Z$, we have $X(\theta(t_0))=0$. As shown in the Kac-Rice proof above, $\dot{Z}(t_0)=\dot{\theta}(t_0)^{3/2}\dot{X}(\theta(t_0))$. Since $\dot{\theta}(t_0)>0$ and $\dot{X}(\theta(t_0))\neq0$ a.s. by Bulinskaya, the result follows.
\end{proof}

\subsection{Zero Localization Measure}

\begin{definition}
[Zero Localization Measure] Let $Z$ be a real-valued oscillatory process satisfying the hypotheses of the previous corollary. Define, for Borel $B\subset\mathbb{R}$:
\[
\mu(B)\assign\int_B\delta(Z(t))|\dot{Z}(t)|\,dt
\]
This places unit mass at each simple zero of $Z$. The multiplication operator $(Lf)(t)=tf(t)$ is essentially self-adjoint on $L^2(\mu)$ with pure point spectrum equal to the zero set.
\end{definition}

\section{Asymptotic Expansion of $\Theta'(t)$}

\subsection{Stirling's Formula and Application}

\begin{lemma}
[Stirling's Formula] For $z$ with $|\arg(z)|<\pi$:
\[
\log\Gamma(z)=\left(z-\frac12\right)\log z-z+\frac12\log(2\pi)+O(|z|^{-1})
\]
\end{lemma}

\begin{definition}
[Riemann-Siegel Theta Function] The theta function is defined as:
\[
\theta(t)=\operatorname{Im}\!\Bigl[\log\Gamma\Bigl(\frac14+\frac{it}{2}\Bigr)\Bigr]-\frac{t}{2}\log\pi
\]
\end{definition}

\begin{theorem}
[Asymptotic Expansion of $\theta'(t)$]
\[
\theta'(t)=\frac12\log\frac{t}{2\pi}+O(t^{-1})
\]
\end{theorem}

\begin{proof}
Let $z=\frac14+\frac{it}{2}$. For $t>0$:
\[
|z|=\sqrt{\frac{1}{16}+\frac{t^2}{4}}=\frac{t}{2}\sqrt{1+\frac{1}{4t^2}}=\frac{t}{2}(1+O(t^{-2}))
\]
\[
\arg(z)=\arctan(2t)=\frac{\pi}{2}-\frac{1}{2t}+O(t^{-3})
\]
Thus:
\[
\log z=\log\frac{t}{2}+O(t^{-2})+i\Bigl(\frac{\pi}{2}-\frac{1}{2t}+O(t^{-3})\Bigr)
\]
The imaginary part of $(z-\frac12)\log z$ is:
\[
\operatorname{Im}[(z-\tfrac12)\log z]=-\frac{\pi}{8}+\frac{1}{8t}+\frac{t}{2}\log\frac{t}{2}+O(t^{-2})
\]
Applying Stirling's formula and simplifying gives:
\[
\theta(t)=-\frac{\pi}{8}+\frac{t}{2}\log\frac{t}{2\pi e}+O(t^{-1})
\]
Differentiating yields the result.
\end{proof}

\section{Vanishing of the Logarithmic Ratio}

\begin{theorem}
[Vanishing of $\log n/\Theta'(t)$] For any fixed integer $n\ge1$:
\[
\lim_{t\to\infty}\frac{\log n}{\Theta'(t)}=0
\]
More precisely:
\[
\frac{\log n}{\Theta'(t)}=O\!\Bigl(\frac{\log n}{\log t}\Bigr)=o(1)\quad\text{as }t\to\infty
\]
\end{theorem}

\begin{proof}
Since $\Theta'(t)=\theta'(t)=\frac12\log\frac{t}{2\pi}+O(t^{-1})$, we have:
\[
\frac{\log n}{\Theta'(t)}=\frac{2\log n}{\log(t/(2\pi))}(1+o(1))\to0
\]
\end{proof}

\section{The Riemann-Siegel Representation}

\begin{definition}
[Hardy Z-Function] The Hardy Z-function is defined as:
\[
Z(t)=e^{i\theta(t)}\zeta(\tfrac12+it)
\]
where $\theta(t)$ is the Riemann-Siegel theta function and $\zeta(s)$ is the Riemann zeta function.
\end{definition}

\begin{theorem}
[Riemann-Siegel Representation] The Hardy Z-function admits the asymptotic expansion:
\[
Z(t)=2\sum_{n=1}^{N(t)}n^{-1/2}\cos(\theta(t)-t\log n)+R(t)
\]
where $N(t)=\bigl\lfloor\sqrt{t/(2\pi)}\bigr\rfloor$ and $R(t)=O(t^{-1/4})$.
\end{theorem}

\begin{proof}
This is the classical Riemann-Siegel formula (Siegel 1932). The proof uses the functional equation for $\zeta(s)$, Poisson summation, and stationary phase analysis. The key steps are:
\begin{enumerate}
\item Apply the functional equation $\zeta(s)=\chi(s)\zeta(1-s)$ at $s=\frac12+it$
\item Use Poisson summation to convert the series into an integral with controllable error
\item Apply stationary phase to identify the optimal truncation at $N(t)=\lfloor\sqrt{t/(2\pi)}\rfloor$
\item Use Van der Corput's lemma to bound the remainder by $O(t^{-1/4})$
\end{enumerate}
The detailed derivation is given in standard references (Edwards 1974, Titchmarsh 1986).
\end{proof}

\section{Transformation to $u$-Coordinates}

\begin{definition}
[Underlying Stationary Process] Define the process $X$ on $[\Theta(0),\infty)$ by:
\[
X(u)=(U_{\Theta}^{-1}Z)(u)=\frac{Z(\Theta^{-1}(u))}{\sqrt{\Theta'(\Theta^{-1}(u))}}
\]
\end{definition}

\begin{theorem}
[Exact Reconstruction] The original Hardy Z-function is exactly reconstructed by:
\[
Z(t)=(U_{\Theta}X)(t)=\sqrt{\Theta'(t)}\,X(\Theta(t))
\]
This is a unitarily time-changed stationary process.
\end{theorem}

\begin{proof}
Immediate from the inverse property $U_{\Theta}U_{\Theta}^{-1}=I$.
\end{proof}

\begin{theorem}
[Riemann-Siegel in $u$-Coordinates] In the transformed coordinates $u=\Theta(t)$, with $t=\Theta^{-1}(u)$, define the phase:
\[
\Phi_n(u)=\theta(\Theta^{-1}(u))-\Theta^{-1}(u)\log n
\]
Then:
\[
X(u)=\frac{1}{\sqrt{\Theta'(\Theta^{-1}(u))}}\Bigl[2\sum_{n=1}^{N(\Theta^{-1}(u))}n^{-1/2}\cos(\Phi_n(u))+R(\Theta^{-1}(u))\Bigr]
\]
\end{theorem}

\subsection{Analysis of Phase Differences}

\begin{lemma}
[Phase Difference Convergence] For fixed $h\in\mathbb{R}$ and fixed $n\ge1$:
\[
\lim_{u\to\infty}[\Phi_n(u)-\Phi_n(u+h)]=-h
\]
\end{lemma}

\begin{proof}
Expanding the difference:
\begin{align*}
\Phi_n(u)-\Phi_n(u+h)&=[\theta(\Theta^{-1}(u))-\theta(\Theta^{-1}(u+h))]\\
&\quad-[\Theta^{-1}(u)-\Theta^{-1}(u+h)]\log n
\end{align*}
By the mean value theorem, for some $\xi_u\in(u,u+h)$:
\[
\Theta^{-1}(u+h)-\Theta^{-1}(u)=\frac{h}{\Theta'(\Theta^{-1}(\xi_u))}
\]
The logarithmic term becomes:
\[
[\Theta^{-1}(u+h)-\Theta^{-1}(u)]\log n=\frac{h\log n}{\Theta'(\Theta^{-1}(\xi_u))}\to0
\]
by Theorem 4.1. For the $\theta$ difference, by the mean value theorem for integrals:
\[
\theta(\Theta^{-1}(u+h))-\theta(\Theta^{-1}(u))=\int_{\Theta^{-1}(u)}^{\Theta^{-1}(u+h)}\theta'(s)\,ds
\]
Since $\theta'(s)=\Theta'(s)$, we have:
\[
\int_{\Theta^{-1}(u)}^{\Theta^{-1}(u+h)}\theta'(s)\,ds\sim\Theta'(\Theta^{-1}(u))\cdot\frac{h}{\Theta'(\Theta^{-1}(u))}=h+o(1)
\]
Combining gives $\Phi_n(u)-\Phi_n(u+h)=-(h+o(1))+h\cdot o(1)=-h+o(1)$.
\end{proof}

\section{Ces{\`a}ro Averaging and Stationary Limit}

\subsection{Van der Corput Lemma}

\begin{lemma}
[Van der Corput] Let $\phi:[a,b]\to\mathbb{R}$ be continuously differentiable with $|\phi'(x)|\ge\lambda>0$ for all $x\in[a,b]$. Then:
\[
\Bigl|\int_a^b e^{i\phi(x)}\,dx\Bigr|\le\frac{4}{\lambda}
\]
In particular, $\bigl|\int_a^b\cos(\phi(x))\,dx\bigr|=O(1/\lambda)$.
\end{lemma}

\subsection{Phase Sum Derivative}

\begin{lemma}
[Phase Sum Derivative] For the phase sum $\Psi_n(u)\assign\Phi_n(u)+\Phi_n(u+h)$:
\[
\frac{d\Psi_n}{du}(u)=\frac{\theta'(\Theta^{-1}(u))}{\Theta'(\Theta^{-1}(u))}+\frac{\theta'(\Theta^{-1}(u+h))}{\Theta'(\Theta^{-1}(u+h))}
-\frac{\log n}{\Theta'(\Theta^{-1}(u))}-\frac{\log n}{\Theta'(\Theta^{-1}(u+h))}
\]
As $u\to\infty$, this tends to $2$.
\end{lemma}

\begin{proof}
Differentiating $\Phi_n(u)$ gives:
\[
\frac{d\Phi_n}{du}(u)=\frac{\theta'(\Theta^{-1}(u))-\log n}{\Theta'(\Theta^{-1}(u))}
\]
Summing two copies and using $\theta'/\Theta'\to1$ and $\log n/\Theta'\to0$ yields the limit.
\end{proof}

\subsection{Analysis of Diagonal and Off-Diagonal Terms}

\begin{proposition}
[Diagonal Oscillations Vanish] For each fixed $n$:
\[
\lim_{U\to\infty}\frac{1}{U}\int_{\Theta(0)}^U\cos(\Phi_n(u)+\Phi_n(u+h))\,du=0
\]
\end{proposition}

\begin{proof}
For sufficiently large $u$, Lemma 7.2 gives $|d[\Phi_n(u)+\Phi_n(u+h)]/du|\ge1$. By Van der Corput, the integral is $O(1)$. Division by $U\to\infty$ gives zero.
\end{proof}

\begin{proposition}
[Diagonal Difference Converges] For each fixed $n$ and $h$:
\[
\lim_{U\to\infty}\frac{1}{U}\int_{\Theta(0)}^U\cos(\Phi_n(u)-\Phi_n(u+h))\,du=\cos h
\]
\end{proposition}

\begin{proof}
By Lemma 6.1, $\Phi_n(u)-\Phi_n(u+h)=-h+o(1)$. Since $|\cos|\le1$, dominated convergence applies and the Cesàro average of $\cos(-h+o(1))$ converges to $\cos h$.
\end{proof}

\begin{proposition}
[Off-Diagonal Terms Vanish] For $n\neq m$, the cross terms contribute $o(1)$ to the Cesàro average.
\end{proposition}

\begin{proof}
The phase sum $\Phi_n(u)+\Phi_m(u+h)$ has derivative tending to $2$ by Lemma 7.2 (the different indices do not affect the asymptotic). Van der Corput gives $O(1)$ integral bound, and division by $U$ yields zero.
\end{proof}

\begin{proposition}
[Remainder Contribution Negligible] The remainder $R(t)=O(t^{-1/4})$ contributes $o(1)$ to the Cesàro average.
\end{proposition}

\begin{proof}
The weight factor is $W(u,h)=O((\log u)^{-1})$. The finite sum has $O(\sqrt{u})$ terms, each bounded by $O(1)$. The product $W\cdot O(\sqrt{u})\cdot O(u^{-1/4})=O(1/\log u)$. Integration over $[\Theta(0),U]$ gives $O(U/\log U)$, and division by $U$ yields $o(1)$.
\end{proof}

\begin{lemma}
[Cesàro Average Independence] For any bounded integrable $f$ and starting points $u_0,\tilde{u}_0\ge\Theta(0)$:
\[
\Bigl|\frac{1}{U}\int_{u_0}^{u_0+U}f\,du-\frac{1}{U}\int_{\tilde{u}_0}^{\tilde{u}_0+U}f\,du\Bigr|\le\frac{2|\tilde{u}_0-u_0|\sup|f|}{U}\to0
\]
\end{lemma}

\section{Main Theorem: Ces{\`a}ro Stationarity}

\begin{theorem}
[Cesàro Covariance Convergence] For the process $X(u)=(U_{\Theta}^{-1}Z)(u)$ defined via the inverse unitary transform of the Hardy Z-function, the Cesàro covariance:
\[
C(h)=\lim_{U\to\infty}\frac{1}{U-\Theta(0)}\int_{\Theta(0)}^U X(u)X(u+h)\,du
\]
exists for all $h\in\mathbb{R}$ and is independent of the starting point. This establishes $X$ as wide-sense stationary in the Cesàro sense, and consequently $Z(t)=\sqrt{\Theta'(t)}X(\Theta(t))$ is a unitarily time-changed oscillatory process.
\end{theorem}

\begin{proof}
Combining all previous results:
\begin{enumerate}
\item Theorem 3.1 gives the asymptotic behavior of $\Theta'(t)$.
\item Theorem 4.1 ensures harmonic decoherence via $\log n/\Theta'(t)\to0$.
\item Theorem 5.1 provides the Riemann-Siegel representation.
\item Lemma 6.1 shows phase differences converge to $-h$.
\item Lemma 7.2 enables Van der Corput estimates with derivative bound $\ge1$.
\item Propositions 7.1-7.4 show that only the diagonal difference term survives, yielding contribution proportional to $\cos h$.
\item Proposition 7.5 demonstrates off-diagonal terms vanish.
\item Proposition 7.6 proves remainder contributions are negligible.
\item Lemma 7.3 shows the limit is independent of starting point.
\end{enumerate}
Therefore $C(h)$ exists and characterizes $X$ as stationary in the Cesàro sense. By exact reconstruction, $Z$ is a unitarily time-changed oscillatory process.
\end{proof}

\begin{thebibliography}{99}
\bibitem{edwards1974} H.~M.~Edwards, \textit{Riemann's Zeta Function}, Academic Press, 1974.

\bibitem{titchmarsh1986} E.~C.~Titchmarsh, \textit{The Theory of the Riemann Zeta-Function}, 2nd ed., Oxford University Press, 1986.

\bibitem{siegel1932} C.~L.~Siegel, ``{\"U}ber Riemanns Nachlass zur analytischen Zahlentheorie,'' Quellen und Studien zur Geschichte der Mathematik, 1932.

\bibitem{van-der-corput} J.~G.~van der Corput, ``On trigonometric sums,'' Mathematische Annalen, vol.~120, pp.~369--382, 1948.

\bibitem{priestley1965} M.~B.~Priestley, ``Evolutionary spectra and non-stationary processes,'' Journal of the Royal Statistical Society, Series B, vol.~27, pp.~204--237, 1965.
\end{thebibliography}

\end{document}