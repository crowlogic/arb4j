\documentclass{article}
\usepackage[english]{babel}
\usepackage{geometry,amsmath,amssymb,latexsym,theorem}
\geometry{letterpaper}

%%%%%%%%%% Start TeXmacs macros
\newcommand{\assign}{:=}
\newcommand{\tmaffiliation}[1]{\\ #1}
\newenvironment{proof}{\noindent\textbf{Proof\ }}{\hspace*{\fill}$\Box$\medskip}
\newtheorem{definition}{Definition}
{\theorembodyfont{\rmfamily}\newtheorem{remark}{Remark}}
\newtheorem{theorem}{Theorem}
%%%%%%%%%% End TeXmacs macros

\begin{document}

\title{Unitary Time Changes of Stationary Processes Yield Oscillatory Processes}

\author{
  Stephen Crowley
  \tmaffiliation{August 12, 2025}
}

\date{}

\maketitle

\begin{theorem}
  [Unitary time change operator] Let $\theta : \mathbb{R} \to \mathbb{R}$ be a diffeomorphism (bijective and absolutely continuous) with $\theta'(t) \neq 0$ almost everywhere. The unitary time change operator $U_{\theta}$ on $L^2(\mathbb{R})$ is defined by
  \begin{equation}
    (U_{\theta} f)(t) \assign \sqrt{|\theta'(t)|} \hspace{0.17em} f(\theta(t)) \qquad \text{for } f \in L^2(\mathbb{R})
  \end{equation}
  This operator is unitary on $L^2(\mathbb{R})$.
\end{theorem}

\begin{proof}
  The change-of-variables formula gives
  \begin{equation}
    \int_{\mathbb{R}} |(U_{\theta} f)(t)|^2 \hspace{0.17em} dt = \int_{\mathbb{R}} |\theta'(t)| \hspace{0.17em} |f(\theta(t))|^2 \hspace{0.17em} dt = \int_{\mathbb{R}} |f(u)|^2 \hspace{0.17em} du
  \end{equation}
  so $U_{\theta}$ is isometric. Since $\theta$ is a diffeomorphism, $U_{\theta^{-1}}$ exists and provides the inverse, making $U_{\theta}$ unitary.
\end{proof}

\begin{definition}
  [Oscillatory processes in the sense of Priestley] An oscillatory process $Z$ is specified by a measurable gain function $A_t(\lambda)$ and has oscillatory function
  \begin{equation}
    \varphi_t(\lambda) \assign A_t(\lambda) \hspace{0.17em} e^{i \lambda t}
  \end{equation}
  The process $Z$ has spectral representation
  \begin{equation}
    Z(t) = \int_{\mathbb{R}} \varphi_t(\lambda) \hspace{0.17em} \Phi(d\lambda) = \int_{\mathbb{R}} A_t(\lambda) \hspace{0.17em} e^{i \lambda t} \hspace{0.17em} \Phi(d\lambda)
  \end{equation}
  where $\Phi$ is a complex orthogonal random measure on $\mathbb{R}$ with spectral measure $F$ satisfying
  \begin{equation}
    E\left[\Phi(d\lambda) \hspace{0.17em} \overline{\Phi(d\mu)}\right] = \mathbf{1}_{\{\lambda = \mu\}} \hspace{0.17em} dF(\lambda)
  \end{equation}
  The covariance kernel of $Z$ is
  \begin{equation}
    R_Z(t,s) \assign E[Z(t)\overline{Z(s)}] = \int_{\mathbb{R}} A_t(\lambda) \hspace{0.17em} \overline{A_s(\lambda)} \hspace{0.17em} e^{i\lambda(t-s)} \hspace{0.17em} dF(\lambda)
  \end{equation}
\end{definition}

\begin{remark}
  [Real-valuedness condition] The oscillatory process $Z$ is real-valued if and only if the gain satisfies conjugate symmetry:
  \begin{equation}
    A_t(-\lambda) = \overline{A_t(\lambda)} \quad \text{for $F$-almost every } \lambda, \text{for each fixed } t
  \end{equation}
\end{remark}

\begin{theorem}
  [Unitary time change of stationary process yields oscillatory process] Let $X$ be a zero-mean stationary Gaussian process with Cramér spectral representation
  \begin{equation}
    X(t) = \int_{\mathbb{R}} e^{i \lambda t} \hspace{0.17em} \Phi(d\lambda)
  \end{equation}
  where $\Phi$ is a complex orthogonal random measure with spectral measure $F$. Let $U_{\theta}$ be a unitary time change operator as defined above. Then the transformed process
  \begin{equation}
    Z(t) \assign (U_{\theta} X)(t) = \sqrt{|\theta'(t)|} \hspace{0.17em} X(\theta(t))
  \end{equation}
  is an oscillatory process in the sense of Priestley with oscillatory function
  \begin{equation}
    \varphi_t(\lambda) = \sqrt{|\theta'(t)|} \hspace{0.17em} e^{i \lambda \theta(t)}
  \end{equation}
\end{theorem}

\begin{proof}
  Starting from the stationary representation:
  \begin{align}
    Z(t) &= \sqrt{|\theta'(t)|} \hspace{0.17em} X(\theta(t)) \\
    &= \sqrt{|\theta'(t)|} \int_{\mathbb{R}} e^{i \lambda \theta(t)} \hspace{0.17em} \Phi(d\lambda) \\
    &= \int_{\mathbb{R}} \sqrt{|\theta'(t)|} \hspace{0.17em} e^{i \lambda \theta(t)} \hspace{0.17em} \Phi(d\lambda)
  \end{align}
  Defining $\varphi_t(\lambda) \assign \sqrt{|\theta'(t)|} \hspace{0.17em} e^{i \lambda \theta(t)}$, we have
  \begin{equation}
    Z(t) = \int_{\mathbb{R}} \varphi_t(\lambda) \hspace{0.17em} \Phi(d\lambda)
  \end{equation}
  which is the oscillatory form.
\end{proof}

\begin{theorem}
  [Explicit gain function for unitary time change] In the setting of the previous theorem, the gain function for the oscillatory process $Z(t) = (U_{\theta} X)(t)$ is given by
  \begin{equation}
    A_t(\lambda) = \sqrt{|\theta'(t)|} \hspace{0.17em} e^{i \lambda (\theta(t) - t)}
  \end{equation}
  The oscillatory function is
  \begin{equation}
    \varphi_t(\lambda) = A_t(\lambda) \hspace{0.17em} e^{i \lambda t} = \sqrt{|\theta'(t)|} \hspace{0.17em} e^{i \lambda \theta(t)}
  \end{equation}
  and the covariance kernel takes the form
  \begin{equation}
    R_Z(t,s) = \int_{\mathbb{R}} A_t(\lambda) \hspace{0.17em} \overline{A_s(\lambda)} \hspace{0.17em} e^{i \lambda (t-s)} \hspace{0.17em} dF(\lambda)
  \end{equation}
\end{theorem}

\begin{proof}
  Since $\varphi_t(\lambda) = A_t(\lambda) e^{i \lambda t}$, we solve for the gain:
  \begin{equation}
    A_t(\lambda) = \frac{\varphi_t(\lambda)}{e^{i \lambda t}} = \frac{\sqrt{|\theta'(t)|} \hspace{0.17em} e^{i \lambda \theta(t)}}{e^{i \lambda t}} = \sqrt{|\theta'(t)|} \hspace{0.17em} e^{i \lambda (\theta(t) - t)}
  \end{equation}
  The covariance formula follows by substitution:
  \begin{align}
    R_Z(t,s) &= \int_{\mathbb{R}} \varphi_t(\lambda) \hspace{0.17em} \overline{\varphi_s(\lambda)} \hspace{0.17em} dF(\lambda) \\
    &= \int_{\mathbb{R}} A_t(\lambda) e^{i \lambda t} \hspace{0.17em} \overline{A_s(\lambda) e^{i \lambda s}} \hspace{0.17em} dF(\lambda) \\
    &= \int_{\mathbb{R}} A_t(\lambda) \hspace{0.17em} \overline{A_s(\lambda)} \hspace{0.17em} e^{i \lambda (t-s)} \hspace{0.17em} dF(\lambda)
  \end{align}
\end{proof}

\end{document}
