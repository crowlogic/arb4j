\documentclass{article}
\usepackage[english]{babel}
\usepackage{geometry,amsmath,amssymb,latexsym,theorem}
\geometry{letterpaper}

%%%%%%%%%% Start TeXmacs macros
\newcommand{\assign}{:=}
\newcommand{\cdummy}{\cdot}
\newcommand{\mathd}{\mathrm{d}}
\newcommand{\tmmathbf}[1]{\ensuremath{\boldsymbol{#1}}}
\newcommand{\tmop}[1]{\ensuremath{\operatorname{#1}}}
\newcommand{\tmstrong}[1]{\textbf{#1}}
\newcommand{\tmtextbf}[1]{\text{{\bfseries{#1}}}}
\newcommand{\tmtextit}[1]{\text{{\itshape{#1}}}}
\newenvironment{proof}{\noindent\textbf{Proof\ }}{\hspace*{\fill}$\Box$\medskip}
\newtheorem{corollary}{Corollary}
\newtheorem{definition}{Definition}
\newtheorem{proposition}{Proposition}
{\theorembodyfont{\rmfamily}\newtheorem{remark}{Remark}}
\newtheorem{theorem}{Theorem}
%%%%%%%%%% End TeXmacs macros

\begin{document}

\title{
  Unitary Time Changes of Stationary Processes Yield Oscillatory Processes\\
  
}

\author{Stephen Crowley}

\date{September 16, 2025}

\maketitle

\begin{abstract}
  A unitary time-change operator $U_{\theta}$ is constructed for absolutely
  continuous, strictly increasing time reparametrizations $\theta$, acting on
  functions that are locally square-integrable (meaning over compact sets).
  Applying $U_{\theta}$ to the Cram{\'e}r spectral representation of a
  stationary process $X (t)$ produces the transformed process $Z (t) =
  U_{\theta} X (t) = \sqrt{\dot{\theta} (t)}  \hspace{0.17em} X (\theta (t)) =
  \sqrt{\dot{\theta} (t)}  \int_{\mathbb{R}} e^{i \lambda \theta (t)} 
  \hspace{0.17em} d \Phi (\lambda)$, which is an oscillatory process in the
  sense of Priestley with oscillatory function $\varphi_t (\lambda) =
  \sqrt{\dot{\theta} (t)}  \hspace{0.17em} e^{i \lambda \theta (t)}$,
  evolutionary spectrum $dF_t (\lambda) = \dot{\theta} (t)  \hspace{0.17em} dF
  (\lambda)$, and covariance kernel $K_Z (t, s) = \sqrt{\dot{\theta} (t) 
  \dot{\theta} (s)}  \hspace{0.17em} K_X (\theta (t), \theta (s))$ where $K_X$
  is the stationary covariance of $X (t) = \int_{\mathbb{R}} e^{i \lambda t} 
  \hspace{0.17em} d \Phi (\lambda)$, and the expected zero-counting function
  $\mathbb{E} [N_{[a, b]}]$ of the oscillatory process paths equals $\sqrt{-
  \ddot{K} (0)}  \hspace{0.17em} (\theta (a) - \theta (b))$. The sample paths
  of any non-degenerate second-order stationary process are locally square
  integrable, making the unitary time-change operator $U_{\theta}$ applicable
  to typical realizations. A zero-localization measure $d \mu (t) = \delta (Z
  (t)) \hspace{0.17em} | \dot{Z} (t) |  \hspace{0.17em} dt$ induces a Hilbert
  space $L^2 (\mu)$ on the zero set of each oscillatory process realization $Z
  (t)$, and the multiplication operator $(Lf) (t) = tf (t)$ has simple pure
  point spectrum equal to the zero crossing set of $Z$
\end{abstract}

{\tableofcontents}

\section{Gaussian Processes}

\subsection{Definition}

\begin{definition}
  \label{def:gaussian_process}\tmtextbf{(Gaussian process)} Let $(\Omega,
  \mathcal{F}, \mathbb{P})$ be a probability space and $T$ a nonempty index
  set. A family $\{X_t : t \in T\}$ of real-valued random variables on
  $(\Omega, \mathcal{F}, \mathbb{P})$ is called a Gaussian process if for
  every finite subset $\{t_1, \ldots, t_n \} \subset T$ the random vector
  $(X_{t_1}, \ldots, X_{t_n})$ is multivariate normal (possibly degenerate).
  Equivalently, every finite linear combination $\sum_{i = 1}^n a_i X_{t_i}$
  is either almost surely constant or Gaussian. The mean function is $m (t)
  \assign \mathbb{E} [X_t]$ and the covariance kernel is
  \begin{equation}
    \label{eq:covariance_kernel} K (s, t) = \mathrm{Cov} (X_s, X_t)
  \end{equation}
  For any finite $(t_i)_{i = 1}^n \subset T$, the matrix $K_{ij} = K (t_i,
  t_j)$ is symmetric positive semidefinite, and a Gaussian process is
  completely determined in law by $m$ and $K$
\end{definition}

\subsection{Stationary processes}

\begin{definition}
  \label{def:cramer}\tmtextbf{[Cram{\'e}r spectral
  representation]}{\cite{stationaryAndRelatedStochasticProcesses}} A zero-mean
  stationary process $X$ with spectral measure $F$ admits the sample path
  representation
  \begin{equation}
    \label{eq:cramer_representation} X (t) = \int_{\mathbb{R}} e^{i \lambda t}
    \hspace{0.17em} d \Phi (\lambda)
  \end{equation}
  which has covariance
  \begin{equation}
    \label{eq:stationary_covariance} R_X  (t - s) = \int_{\mathbb{R}} e^{i
    \lambda (t - s)}  \hspace{0.17em} dF (\lambda)
  \end{equation}
\end{definition}

\subsubsection{Sample path realizations}

\begin{definition}
  \label{def:L2loc}\tmtextbf{[Locally square-integrable functions]} Define
  \begin{equation}
    L^2_{\mathrm{loc}} (\mathbb{R}) \assign \left\{ f : \mathbb{R} \to
    \mathbb{C} \hspace{0.17em} : \hspace{0.17em} \int_K |f (t) |^2 
    \hspace{0.17em} dt < \infty \text{for every compact } K \subseteq
    \mathbb{R} \right\}
  \end{equation}
\end{definition}

\begin{remark}
  \label{rem:L2loc_properties}Every bounded measurable set in $\mathbb{R}$ is
  compact or contained in a compact set; hence $L^2_{\mathrm{loc}}
  (\mathbb{R})$ contains functions that are square-integrable on every bounded
  interval, including functions with polynomial growth at infinity
\end{remark}

\begin{theorem}
  \label{thm:paths_loc}\tmtextbf{[Sample paths in $L^2_{\mathrm{loc}}
  (\mathbb{R})$]} Let $\{X (t)\}_{t \in \mathbb{R}}$ be a second-order
  stationary process with
  \begin{equation}
    \label{eq:finite_variance} \sigma^2 \assign \mathbb{E} [X (t)^2] < \infty
  \end{equation}
  Then almost every sample path lies in $L^2_{\mathrm{loc}} (\mathbb{R})$
\end{theorem}

\begin{proof}
  Fix an arbitrary bounded interval $[a, b] \subset \mathbb{R}$ with $a < b$.
  Define
  \begin{equation}
    \label{eq:Yab_def} Y_{[a, b]} \assign \int_a^b X (t)^2  \hspace{0.17em} dt
  \end{equation}
  By Tonelli's theorem, since $X (t)^2 \ge 0$,
  \begin{equation}
    \label{eq:tonelli_application} \mathbb{E} [Y_{[a, b]}] =\mathbb{E}
    \hspace{-0.17em} \left[ \int_a^b X (t)^2  \hspace{0.17em} dt \right] =
    \int_a^b \mathbb{E} [X (t)^2]  \hspace{0.17em} dt
  \end{equation}
  By stationarity, $\mathbb{E} [X (t)^2] = \sigma^2$ for all $t$, hence
  \begin{equation}
    \label{eq:expectation_Yab} \mathbb{E} [Y_{[a, b]}] = \sigma^2  (b - a) <
    \infty
  \end{equation}
  Markov's inequality yields, for $M > 0$,
  \begin{equation}
    \label{eq:markov_inequality} \mathbb{P} (Y_{[a, b]} > M) \le
    \frac{\mathbb{E} [Y_{[a, b]}]}{M} = \frac{\sigma^2  (b - a)}{M}
  \end{equation}
  and letting $M \to \infty$ gives $\mathbb{P} (Y_{[a, b]} < \infty) = 1$. Now
  let $K \subset \mathbb{R}$ be compact, so $K \subseteq [- N, N]$ for some $N
  > 0$. Then
  \begin{equation}
    \int_K X (t)^2  \hspace{0.17em} dt \le \int_{- N}^N X (t)^2 
    \hspace{0.17em} dt < \infty \quad \text{a.s.}
  \end{equation}
  hence almost every path satisfies $\int_K |X (t, \omega) |^2 
  \hspace{0.17em} dt < \infty$ for every compact $K$, i.e. $X (\cdummy,
  \omega) \in L^2_{\mathrm{loc}} (\mathbb{R})$
\end{proof}

\subsection{(Non-Stationary) Oscillatory Processes}\label{sec:oscillatory}

\begin{definition}
  \label{def:osc_proc}\tmtextbf{[Oscillatory
  process]}{\cite{evolutionarySpectraAndNonStationaryProcesses}} Let $F$ be a
  finite nonnegative Borel measure on $\mathbb{R}$. Let
  \begin{equation}
    \label{eq:gain_L2} A_t \in L^2 (F) \quad \forall \hspace{0.17em} t \in
    \mathbb{R}
  \end{equation}
  be the gain function and
  \begin{equation}
    \label{eq:oscillatory_function} \varphi_t (\lambda) = A_t (\lambda) 
    \hspace{0.17em} e^{i \lambda t}
  \end{equation}
  the corresponding oscillatory function. An oscillatory process is a
  stochastic process represented as
  \begin{equation}
    \label{eq:oscillatory_process} Z (t) \hspace{0.27em} = \hspace{0.27em}
    \int_{\mathbb{R}} \varphi_t (\lambda)  \hspace{0.17em} d \Phi (\lambda)
    \hspace{0.27em} = \hspace{0.27em} \int_{\mathbb{R}} A_t (\lambda) 
    \hspace{0.17em} e^{i \lambda t}  \hspace{0.17em} d \Phi (\lambda)
  \end{equation}
  where $\Phi$ is a complex orthogonal random measure with spectral measure
  $F$ satisfying
  \begin{equation}
    \label{eq:orthogonality_phi} d\mathbb{E} \hspace{-0.17em} \left[ \Phi
    (\lambda) \hspace{0.17em} \overline{\Phi (\mu)} \right] \hspace{0.27em} =
    \hspace{0.27em} \delta (\lambda - \mu)  \hspace{0.17em} dF (\lambda)
  \end{equation}
  and covariance
  \begin{equation}
    \label{eq:oscillatory_covariance} \begin{array}{ll}
      R_Z (t, s) \hspace{0.27em} = \hspace{0.27em} \mathbb{E} \hspace{-0.17em}
      \left[ Z (t) \hspace{0.17em} \overline{Z (s)} \right] & =
      \int_{\mathbb{R}} A_t (\lambda) \hspace{0.17em} \overline{A_s (\lambda)}
      \hspace{0.17em} e^{i \lambda (t - s)}  \hspace{0.17em} dF (\lambda)\\
      & = \int_{\mathbb{R}} \varphi_t (\lambda) \hspace{0.17em}
      \overline{\varphi_s (\lambda)} \hspace{0.17em} dF (\lambda
    \end{array} \tmmathbf{}
  \end{equation}
  
\end{definition}

\begin{definition}
  \label{cor:evol_spec}\tmtextbf{[Evolutionary spectrum]} The evolutionary
  power spectral density of an oscillatory process is given by is
  \begin{equation}
    \begin{array}{ll}
      \label{eq:evolutionary_spectrum} dF_t (\lambda) & = S_t (\lambda) \mathd
      \lambda\\
      & = |A_t (\lambda) |^2  \hspace{0.17em} dF (\lambda) \tmmathbf{}\\
      & = |A_t (\lambda) |^2  \hspace{0.17em} S (\lambda) d \lambda
    \end{array}
  \end{equation}
\end{definition}

\begin{definition}
  {\tmstrong{[Variance of evolutionary process]}} The variance of an
  evolutionary process $Z (t)$ is given by integrating the evolutionary power
  spectral density $S_t (\lambda)$ over all frequencues
  \begin{equation}
    \tmop{var} (Z (t)) = \int_{- \infty}^{\infty} S_t (\lambda ) \mathd
    \lambda = \int_{- \infty}^{\infty} \mathd F_t (\lambda)
  \end{equation}
\end{definition}

\begin{theorem}
  \label{thm:realvaluedness}\tmtextbf{[Real-valuedness criterion for
  oscillatory processes]} Let $Z$ be an oscillatory process with oscillatory
  function $\varphi_t (\lambda) = A_t (\lambda)  \hspace{0.17em} e^{i \lambda
  t}$ and spectral measure $F$. Then $Z$ is real-valued if and only if
  \begin{equation}
    \label{eq:gain_symmetry} A_t  (- \lambda) = \overline{A_t (\lambda)} \quad
    \text{for } F \text{-a.e. } \lambda \in \mathbb{R}
  \end{equation}
  equivalently
  \begin{equation}
    \label{eq:osc_symmetry} \varphi_t  (- \lambda) = \overline{\varphi_t
    (\lambda)} \quad \text{for } F \text{-a.e. } \lambda \in \mathbb{R}
  \end{equation}
\end{theorem}

\begin{proof}
  If $Z$ is real-valued, then $Z (t) = \overline{Z (t)}$ for all $t$. Taking
  conjugates in the representation $Z (t) = \int \hspace{-0.17em} A_t
  (\lambda) e^{i \lambda t}  \hspace{0.17em} d \Phi (\lambda)$ and using the
  symmetry relation for the orthogonal random measure appropriate for
  real-valued processes, a change of variable $\mu = - \lambda$ shows that the
  $L^2 (F)$-integrands must agree $F$-a.e., i.e. $A_t (\lambda) =
  \overline{A_t  (- \lambda)}$, which is equivalent to
  \eqref{eq:gain_symmetry}. Using $\varphi_t (\lambda) = A_t (\lambda) e^{i
  \lambda t}$ then gives \eqref{eq:osc_symmetry}. The converse follows by
  reversing the steps
\end{proof}

\begin{theorem}
  \label{thm:existence_osc}\tmtextbf{[Existence of oscillatory processes with
  explicit $L^2$-limit construction]} Let $F$ be an absolutely continuous
  spectral measure and the gain function $A_t (\lambda) \in L^2 (F)$ for all
  $t \in \mathbb{R}$, measurable jointly in $(t, \lambda)$. Define the
  time-dependent spectrum
  \begin{equation}
    \label{eq:time_dependent_spectrum} S_t \assign \int_{\mathbb{R}} |A_t
    (\lambda) |^2  \hspace{0.17em} dF (\lambda) \hspace{0.27em} =
    \hspace{0.27em} \int_{\mathbb{R}} |A_t (\lambda) |^2  \hspace{0.17em} S
    (\lambda)  \hspace{0.17em} d \lambda \hspace{0.27em} < \hspace{0.27em}
    \infty
  \end{equation}
  Then there exists a complex orthogonal random measure $\Phi$ with spectral
  measure $F$ such that for each fixed $t$ the stochastic integral
  \begin{equation}
    \label{eq:oscillatory_well_defined} Z (t) \hspace{0.27em} =
    \hspace{0.27em} \int_{\mathbb{R}} A_t (\lambda)  \hspace{0.17em} e^{i
    \lambda t}  \hspace{0.17em} d \Phi (\lambda)
  \end{equation}
  is well-defined as an $L^2 (\Omega)$-limit and has covariance $R_Z$ as in
  \eqref{eq:oscillatory_covariance}
\end{theorem}

\begin{proof}
  Step 1 (simple functions and isometry). Let $\mathsf{S}$ denote the set of
  simple functions
  \begin{equation}
    \label{eq:simple_function} g (\lambda) \hspace{0.27em} = \hspace{0.27em}
    \sum_{j = 1}^n c_j  \hspace{0.17em} \textbf{1}_{E_j} (\lambda)
  \end{equation}
  with disjoint Borel $E_j$ and $F (E_j) < \infty$, $c_j \in \mathbb{C}$.
  Define the stochastic integral on $\mathsf{S}$ by
  \begin{equation}
    \label{eq:integral_simple} \int_{\mathbb{R}} g (\lambda)  \hspace{0.17em}
    d \Phi (\lambda) \hspace{0.27em} \assign \hspace{0.27em} \sum_{j = 1}^n
    c_j  \hspace{0.17em} \Phi (E_j)
  \end{equation}
  Using orthogonality of $\Phi$,
  \begin{equation}
    \label{eq:isometry_simple} \mathbb{E} \hspace{-0.17em} \left[ \left| \int
    g \hspace{0.17em} d \Phi \right|^2 \right] = \sum_{j = 1}^n |c_j |^2 
    \hspace{0.17em} F (E_j) = \int_{\mathbb{R}} |g (\lambda) |^2 
    \hspace{0.17em} dF (\lambda)
  \end{equation}
  Thus the map $I : \mathsf{S} \to L^2 (\Omega)$, $I (g) = \int g
  \hspace{0.17em} d \Phi$, is an isometry with respect to the $L^2 (F)$-norm.
  
  Step 2 (density and Cauchy property). Simple functions are dense in $L^2
  (F)$: for any $h \in L^2 (F)$ there exists $g_n \in \mathsf{S}$ with $\|h -
  g_n \|_{L^2 (F)} \to 0$. By \eqref{eq:isometry_simple},
  \begin{equation}
    \label{eq:cauchy_sequence} \mathbb{E} \hspace{-0.17em} \left[ \left| \int
    g_n  \hspace{0.17em} d \Phi - \int g_m  \hspace{0.17em} d \Phi \right|^2
    \right] = \|g_n - g_m \|_{L^2 (F)}^2 \hspace{0.27em} \xrightarrow[]{n, m
    \to \infty} 0
  \end{equation}
  so $\left\{ \int g_n  \hspace{0.17em} d \Phi \right\}$ is Cauchy in $L^2
  (\Omega)$.
  
  Step 3 (definition by $L^2$-limit and independence of approximating
  sequence). Since $L^2 (\Omega)$ is complete, the limit exists. Define, for
  $h \in L^2 (F)$,
  \begin{equation}
    \label{eq:L2_limit_def} \int_{\mathbb{R}} h (\lambda)  \hspace{0.17em} d
    \Phi (\lambda) \hspace{0.27em} \assign \hspace{0.27em} \lim_{n \to \infty}
    \int_{\mathbb{R}} g_n (\lambda)  \hspace{0.17em} d \Phi (\lambda)
  \end{equation}
  where $g_n \in \mathsf{S}$ and $\|h - g_n \|_{L^2 (F)} \to 0$. If $g_n$ and
  $\tilde{g}_n$ are two such approximating sequences, then $\|g_n -
  \tilde{g}_n \|_{L^2 (F)} \to 0$ and again by \eqref{eq:isometry_simple} the
  corresponding integrals differ by an $L^2 (\Omega)$-null sequence, so the
  limit is independent of the sequence.
  
  Step 4 (isometry and linearity extend). By continuity from
  \eqref{eq:isometry_simple} and \eqref{eq:L2_limit_def},
  \begin{equation}
    \label{eq:L2_isometry_extension} \mathbb{E} \hspace{-0.17em} \left[ \left|
    \int h \hspace{0.17em} d \Phi \right|^2 \right] = \int_{\mathbb{R}} |h
    (\lambda) |^2  \hspace{0.17em} dF (\lambda)
  \end{equation}
  for $h \in L^2 (F)$, and the map $h \mapsto \int h \hspace{0.17em} d \Phi$
  is linear and isometric.
  
  Step 5 (apply to $\varphi_t$). Since $|e^{i \lambda t} | = 1$, $\varphi_t
  (\lambda) = A_t (\lambda)  \hspace{0.17em} e^{i \lambda t} \in L^2 (F)$ and
  \begin{equation}
    \int_{\mathbb{R}} | \varphi_t (\lambda) |^2  \hspace{0.17em} dF (\lambda)
    = \int_{\mathbb{R}} |A_t (\lambda) |^2  \hspace{0.17em} dF (\lambda) = S_t
    < \infty
  \end{equation}
  Hence $Z (t)$ in \eqref{eq:oscillatory_well_defined} is well-defined as the
  $L^2 (\Omega)$-limit \eqref{eq:L2_limit_def} with $h = \varphi_t$. Computing
  covariance via sesquilinearity together with \eqref{eq:orthogonality_phi}
  yields \eqref{eq:oscillatory_covariance}
\end{proof}

\subsection{Operator Representations}

{\cite{characterizationOscillatoryProcesses}}

\section{Unitarily Time-Changed Stationary
Processes}\label{sec:stationary_timechange}

\subsection{Unitary time-change operator $U_{\theta} f$}

\begin{theorem}
  \label{thm:local_unitarity}\tmtextbf{[Unitary time-change and local
  isometry]} Let the time-scaling function $\theta : \mathbb{R} \to
  \mathbb{R}$ be absolutely continuous, strictly increasing, and bijective,
  with
  \begin{equation}
    \dot{\theta} (t) > 0
  \end{equation}
  almost everywhere and $\dot{\theta} (t) = 0$ only on sets of Lebesgue
  measure zero. For $f$ measurable, define
  \begin{equation}
    \label{eq:U_theta_def} (U_{\theta} f) (t) = \sqrt{\dot{\theta} (t)} 
    \hspace{0.17em} f (\theta (t))
  \end{equation}
  Its inverse is given by
  \begin{equation}
    \label{eq:U_theta_inverse} (U_{\theta}^{- 1} g) (s) = \frac{g (\theta^{-
    1} (s))}{\sqrt{\dot{\theta} (\theta^{- 1} (s))}}
  \end{equation}
  For every compact set $K \subseteq \mathbb{R}$ and $f \in L^2_{\mathrm{loc}}
  (\mathbb{R})$,
  \begin{equation}
    \label{eq:local_isometry} \int_K | (U_{\theta} f) (t) |^2 \hspace{0.17em}
    dt = \int_{\theta (K)} |f (s) |^2  \hspace{0.17em} ds
  \end{equation}
  Moreover, $U_{\theta}^{- 1}$ is the inverse of $U_{\theta}$ on
  $L^2_{\mathrm{loc}} (\mathbb{R})$
\end{theorem}

\begin{proof}
  By \eqref{eq:U_theta_def}, $\int_K | (U_{\theta} f) (t) |^2 dt = \int_K
  \dot{\theta} (t) \hspace{0.17em} |f (\theta (t)) |^2  \hspace{0.17em} dt$.
  With the change of variables $s = \theta (t)$ and $ds = \dot{\theta} (t)
  dt$, the domain maps to $\theta (K)$, giving \eqref{eq:local_isometry}. The
  two-sided inverse identities follow from direct substitution into
  \eqref{eq:U_theta_def} and \eqref{eq:U_theta_inverse}
\end{proof}

\subsection{Time-Varying (Convolution) Filter Representations}

\begin{theorem}
  TODO: insert time-varying filter representations (both forward and reverse)
\end{theorem}

\subsubsection{The Oscillatory Subclass $Z (t) = U_{\theta} X (t)$ }

\

\

\

\begin{theorem}
  \label{thm:inverse_filter}\tmtextbf{[Filter representations of unitarily
  time-changed stationary processes]} Let $\theta : \mathbb{R} \to \mathbb{R}$
  be absolutely continuous, strictly increasing, and bijective with $\theta'
  (t) > 0$ a.e. Let $X (u) = \int_{\mathbb{R}} e^{i \lambda t} \mathd \Phi
  (\lambda)$ be a realization of a stationary process , and set
  \begin{equation}
    \label{eq:Z_transformation} Z (t) = \sqrt{\dot{\theta} (t)} 
    \hspace{0.17em} Y (\theta (t))
  \end{equation}
  Then:
  \begin{enumerate}
    \item The forward filter kernel is
    \begin{equation}
      \label{eq:forward_kernel} h (t, u) = \sqrt{\dot{\theta} (t)} 
      \hspace{0.17em} \delta (u - \theta (t))
    \end{equation}
    \item The inverse filter kernel is
    \begin{equation}
      \label{eq:inverse_kernel} g (t, s) = \frac{\delta (s - \theta^{- 1}
      (t))}{\sqrt{\dot{\theta} (\theta^{- 1} (t))}}
    \end{equation}
    \item The composition $(g \circ h)$ recovers the identity:
    \begin{equation}
      \label{eq:filter_identity} Y (t) = \int_{\mathbb{R}} g (t, s) 
      \hspace{0.17em} Z (s)  \hspace{0.17em} ds = \frac{Z (\theta^{- 1}
      (t))}{\sqrt{\dot{\theta} (\theta^{- 1} (t))}}
    \end{equation}
  \end{enumerate}
\end{theorem}

\begin{proof}
  Using the sifting property of the Dirac delta in \eqref{eq:forward_kernel}
  gives \eqref{eq:Z_transformation}. Likewise, applying
  \eqref{eq:inverse_kernel}, then substituting \eqref{eq:Z_transformation} at
  $s = \theta^{- 1} (t)$ and $\theta \circ \theta^{- 1} = \mathrm{id}$ yields
  \eqref{eq:filter_identity}
\end{proof}

\subsection{Transformation of stationary to oscillatory processes via
$U_{\theta}$}

\begin{theorem}
  \label{thm:Utheta_to_osc}\tmtextbf{[Unitary time change produces oscillatory
  process]} Let $X$ be zero-mean stationary as in Definition \ref{def:cramer}.
  For a scaling function $\theta$ as in Theorem \ref{thm:local_unitarity},
  define
  \begin{equation}
    \label{eq:Z_def} Z (t) = (U_{\theta} X) (t) = \sqrt{\dot{\theta} (t)} 
    \hspace{0.17em} X (\theta (t))
  \end{equation}
  Then $Z$ is a realization of an oscillatory process with oscillatory
  function
  \begin{equation}
    \label{eq:oscillatory_function_Z} \varphi_t (\lambda) = \sqrt{\dot{\theta}
    (t)}  \hspace{0.17em} e^{i \lambda \theta (t)}
  \end{equation}
  gain function
  \begin{equation}
    \label{eq:gain_function_Z} A_t (\lambda) = \sqrt{\dot{\theta} (t)} 
    \hspace{0.17em} e^{i \lambda (\theta (t) - t)}
  \end{equation}
  and covariance
  \begin{equation}
    \begin{array}{ll}
      R_Z (t, s) & =\mathbb{E} \hspace{-0.17em} \left[ Z (t) \hspace{0.17em}
      \overline{Z (s)} \right]\\
      & = \sqrt{\dot{\theta} (t)  \dot{\theta} (s)}  \hspace{0.17em}
      \mathbb{E} \hspace{-0.17em} \left[ X (\theta (t)) \hspace{0.17em}
      \overline{X (\theta (s))} \right]\\
      & = \sqrt{\dot{\theta} (t)  \dot{\theta} (s)}  \hspace{0.17em} R_X 
      \hspace{-0.17em} (\theta (t) - \theta (s))\\
      & = \sqrt{\dot{\theta} (t)  \dot{\theta} (s)}  \int_{\mathbb{R}} e^{i
      \lambda (\theta (t) - \theta (s))}  \hspace{0.17em} dF (\lambda)
    \end{array} \label{UTCovar}
  \end{equation}
\end{theorem}

\begin{proof}
  From the Cram{\'e}r representation \eqref{eq:cramer_representation}, $X
  (\theta (t)) = \int e^{i \lambda \theta (t)}  \hspace{0.17em} d \Phi
  (\lambda)$. Therefore
  \[ Z (t) = \sqrt{\dot{\theta} (t)}  \int_{\mathbb{R}} e^{i \lambda \theta
     (t)}  \hspace{0.17em} d \Phi (\lambda) = \int_{\mathbb{R}} \left(
     \sqrt{\dot{\theta} (t)}  \hspace{0.17em} e^{i \lambda \theta (t)} \right)
     d \Phi (\lambda) = \int \varphi_t (\lambda)  \hspace{0.17em} d \Phi
     (\lambda) \]
  which is of the oscillatory form with $\varphi_t$ as in
  \eqref{eq:oscillatory_function_Z} and $A_t$ as in
  \eqref{eq:gain_function_Z}. The covariance follows from stationarity via
  \eqref{eq:stationary_covariance}
\end{proof}

\begin{corollary}
  \label{cor:evol_spec}\tmtextbf{[Evolutionary spectrum of unitarily
  time-changed stationary process]} The evolutionary spectrum is
  \begin{equation}
    \label{eq:evolutionary_spectrum} dF_t (\lambda) = |A_t (\lambda) |^2 
    \hspace{0.17em} dF (\lambda) = \dot{\theta} (t)  \hspace{0.17em} dF
    (\lambda)
  \end{equation}
\end{corollary}

\begin{proof}
  Since $|e^{i \alpha} | = 1$, $|A_t (\lambda) |^2 = \dot{\theta} (t)$, giving
  \eqref{eq:evolutionary_spectrum}
\end{proof}

\subsection{Covariance operator conjugation}

\begin{proposition}
  \label{prop:conjugation}\tmtextbf{[Operator conjugation]} Let
  \begin{equation}
    \label{eq:T_K_def} (T_K f) (t) \assign \int_{\mathbb{R}} K (|t - s|) 
    \hspace{0.17em} f (s)  \hspace{0.17em} ds
  \end{equation}
  with stationary kernel
  \begin{equation}
    \label{eq:K_def} K (h) = \int_{\mathbb{R}} e^{i \lambda h} 
    \hspace{0.17em} dF (\lambda)
  \end{equation}
  Define the transformed kernel
  \begin{equation}
    \label{eq:K_theta_def} K_{\theta} (s, t) \assign \sqrt{\dot{\theta} (t) 
    \dot{\theta} (s)}  \hspace{0.17em} K \hspace{-0.17em} (| \theta (t) -
    \theta (s) |)
  \end{equation}
  Then for all $f \in L^2_{\mathrm{loc}} (\mathbb{R})$,
  \begin{equation}
    \label{eq:conjugation} (T_{K_{\theta}} f) (t) = \left( U_{\theta} 
    \hspace{0.17em} T_K  \hspace{0.17em} U_{\theta}^{- 1} f \right) (t)
  \end{equation}
\end{proposition}

\begin{proof}
  Compute
  \[ (U_{\theta} T_K U_{\theta}^{- 1} f) (t) = \sqrt{\dot{\theta} (t)} 
     \hspace{0.17em} (T_K U_{\theta}^{- 1} f) (\theta (t)) =
     \sqrt{\dot{\theta} (t)}  \int_{\mathbb{R}} K (| \theta (t) - s|) \frac{f
     (\theta^{- 1} (s))}{\sqrt{\dot{\theta} (\theta^{- 1} (s))}} 
     \hspace{0.17em} ds \]
  With $s = \theta (u)$, $ds = \dot{\theta} (u)  \hspace{0.17em} du$, obtain
  \[ \sqrt{\dot{\theta} (t)}  \int_{\mathbb{R}} K (| \theta (t) - \theta (u)
     |) \hspace{0.17em} \sqrt{\dot{\theta} (u)}  \hspace{0.17em} f (u) 
     \hspace{0.17em} du = \int_{\mathbb{R}} K_{\theta} (u, t)  \hspace{0.17em}
     f (u)  \hspace{0.17em} du = (T_{K_{\theta}} f) (t) \]
\end{proof}

\section{Zero Localization}\label{sec:HP}

\begin{definition}
  \label{def:zeromeasure}\tmtextbf{[Zero localization measure]} Let $Z$ be
  real-valued with $Z \in C^1 (\mathbb{R})$ having only simple zeros
  \begin{equation}
    \label{eq:simple_zeros} Z (t_0) = 0 \hspace{0.27em} \Rightarrow
    \hspace{0.27em} \dot{Z} (t_0) \neq 0
  \end{equation}
  Define, for Borel $B \subset \mathbb{R}$,
  \begin{equation}
    \label{eq:mu_def} \mu (B) \assign \int_B \delta (Z (t)) \hspace{0.17em} |
    \dot{Z} (t) |  \hspace{0.17em} dt
  \end{equation}
  so that $\mu$ places unit mass at each simple zero of $Z$ counted by the
  co-area/change-of-variables identity for $C^1$ functions. The induced space
  $L^2 (\mu)$ consists of (equivalence classes of) functions supported on the
  zero set of $Z$, and the multiplication operator $(Lf) (t) = tf (t)$ is
  essentially self-adjoint on $C_c^{\infty}$ functions supported on the zero
  set with pure point spectrum equal to the zero-crossing set
\end{definition}

\

\begin{thebibliography}{1}
  \bibitem[1]{stationaryAndRelatedStochasticProcesses}Harald Cram{\'e}r  and 
  M.R.~Leadbetter. {\newblock}\tmtextit{Stationary and Related Processes:
  Sample Function Properties and Their Applications}. {\newblock}Wiley Series
  in Probability and Mathematical Statistics. 1967.{\newblock}
  
  \bibitem[2]{characterizationOscillatoryProcesses}V.~Mandrekar. {\newblock}A
  characterization of oscillatory processes and their prediction.
  {\newblock}\tmtextit{Proceedings of the American Mathematical Society},
  32(1):280--284, 1972.{\newblock}
  
  \bibitem[3]{evolutionarySpectraAndNonStationaryProcesses}Maurice~B
  Priestley. {\newblock}Evolutionary spectra and non-stationary processes.
  {\newblock}\tmtextit{Journal of the Royal Statistical Society: Series B
  (Methodological)}, 27(2):204--229, 1965.{\newblock}
\end{thebibliography}

\end{document}
