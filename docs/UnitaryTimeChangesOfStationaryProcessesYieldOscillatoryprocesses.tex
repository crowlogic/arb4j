\documentclass[11pt,a4paper]{article}
\usepackage{amsmath}
\usepackage{amssymb}
\usepackage{amsfonts}
\usepackage{mathtools}
\usepackage{amsthm}
\usepackage{geometry}
\usepackage{hyperref}
\geometry{margin=1in}

\theoremstyle{plain}
\newtheorem{theorem}{Theorem}[section]
\newtheorem{lemma}[theorem]{Lemma}
\newtheorem{proposition}[theorem]{Proposition}
\newtheorem{corollary}[theorem]{Corollary}

\theoremstyle{definition}
\newtheorem{definition}[theorem]{Definition}
\newtheorem{remark}[theorem]{Remark}

\title{Unitarily Time-Changed Stationary Processes:\\ A Subclass of Oscillatory Processes}
\author{Stephen Crowley}
\date{December 21, 2025}

\begin{document}

\maketitle

\begin{abstract}
We establish that unitarily time-changed stationary processes form a proper subclass of oscillatory processes in the sense of Priestley. For any stationary process with spectral representation, the unitary time-change operator produces an oscillatory process with explicitly computable gain function. We apply the Kac-Rice formula to derive zero-counting formulas for this class. As a concrete application, we show that the Hardy Z-function is a member of this class, construct its orthogonal random measure via sinc-kernel integrals, and recover the Riemann-Siegel formula through explicit calculation.
\end{abstract}

\tableofcontents

\section{Introduction}

Oscillatory processes, introduced by Priestley, provide a framework for analyzing non-stationary stochastic processes with time-varying spectral content. We demonstrate that the class of unitarily time-changed stationary processes forms a natural subclass of oscillatory processes. Given any stationary process and a suitable time-change function, the resulting process admits an oscillatory representation with gain function determined explicitly by the time-change derivative.

The Hardy Z-function serves as the primary application, illustrating how this general theory applies to objects of central importance in analytic number theory.

\section{Unitary Time-Change Operators}

\begin{definition}
  [Time-Change Operator] Let $\Theta : \mathbb{R} \to \mathbb{R}$ be absolutely continuous, strictly increasing, and bijective with $\dot{\Theta}(t) > 0$ a.e. Define the bounded operator $U_{\Theta}$ on $L^2_{\mathrm{loc}}(\mathbb{R})$ by:
  \[ (U_{\Theta} f)(t) = \sqrt{\dot{\Theta}(t)} f(\Theta(t)) \]
  with inverse:
  \[ (U_{\Theta}^{-1} g)(s) = \frac{g(\Theta^{-1}(s))}{\sqrt{\dot{\Theta}(\Theta^{-1}(s))}} \]
\end{definition}

\begin{theorem}
  [Local Isometry] For every compact $K \subseteq \mathbb{R}$ and $f \in L^2_{\mathrm{loc}}(\mathbb{R})$:
  \[ \int_K |(U_{\Theta} f)(t)|^2 dt = \int_{\Theta(K)} |f(s)|^2 ds \]
  The operators satisfy $U_{\Theta}^{-1}(U_{\Theta} f) = f$ and $U_{\Theta}(U_{\Theta}^{-1} g) = g$.
\end{theorem}

\begin{proof}
  Change of variables $s = \Theta(t)$ with $ds = \dot{\Theta}(t)dt$ yields:
  \[ \int_K |(U_{\Theta} f)(t)|^2 dt = \int_K \dot{\Theta}(t)|f(\Theta(t))|^2 dt = \int_{\Theta(K)} |f(s)|^2 ds \]
  For the inverse identities, compute:
  \[ (U_{\Theta}^{-1}(U_{\Theta} f))(s) = \frac{(U_{\Theta} f)(\Theta^{-1}(s))}{\sqrt{\dot{\Theta}(\Theta^{-1}(s))}} = \frac{\sqrt{\dot{\Theta}(\Theta^{-1}(s))} f(\Theta(\Theta^{-1}(s)))}{\sqrt{\dot{\Theta}(\Theta^{-1}(s))}} = f(s) \]
  Similarly for $U_{\Theta}(U_{\Theta}^{-1} g) = g$.
\end{proof}

\section{Oscillatory Processes}

\begin{definition}
  [Oscillatory Process] An oscillatory process admits the spectral representation:
  \[ Z(t) = \int_{\mathbb{R}} A_t(\lambda) e^{i\lambda t} d\Phi(\lambda) \]
  where $A_t(\lambda)$ is a time-dependent gain function and $\Phi$ is an orthogonal random measure.
\end{definition}

\begin{theorem}
  [Main Result: Time-Changed Processes are Oscillatory] Let $X$ be a stationary process with spectral representation:
  \[ X(u) = \int_{\mathbb{R}} e^{i\lambda u} d\Phi(\lambda) \]
  where $\Phi$ is an orthogonal random measure. Let $\Theta$ satisfy Definition 2.1. Then the time-changed process
  \[ Z(t) = (U_{\Theta} X)(t) = \sqrt{\dot{\Theta}(t)} X(\Theta(t)) \]
  is an oscillatory process with gain function:
  \[ A_t(\lambda) = \sqrt{\dot{\Theta}(t)} e^{i\lambda(\Theta(t) - t)} \]
\end{theorem}

\begin{proof}
  Substituting $u = \Theta(t)$ in the spectral representation of $X$:
  \begin{align*}
  Z(t) &= \sqrt{\dot{\Theta}(t)} X(\Theta(t)) = \sqrt{\dot{\Theta}(t)} \int_{\mathbb{R}} e^{i\lambda\Theta(t)} d\Phi(\lambda) \\
  &= \int_{\mathbb{R}} \sqrt{\dot{\Theta}(t)} e^{i\lambda\Theta(t)} d\Phi(\lambda)
  \end{align*}
  Factor out $e^{i\lambda t}$:
  \[ \sqrt{\dot{\Theta}(t)} e^{i\lambda\Theta(t)} = \sqrt{\dot{\Theta}(t)} e^{i\lambda(\Theta(t) - t)} e^{i\lambda t} \]
  Setting $A_t(\lambda) = \sqrt{\dot{\Theta}(t)} e^{i\lambda(\Theta(t) - t)}$ yields:
  \[ Z(t) = \int_{\mathbb{R}} A_t(\lambda) e^{i\lambda t} d\Phi(\lambda) \]
  which is the oscillatory representation.
\end{proof}

\section{Stationarity of the Inverse Transform}

\begin{theorem}
  [Inverse Transform Produces Stationary Process] Let $Z(t) = (U_{\Theta} X)(t)$ where $X$ is stationary with spectral representation $X(u) = \int_{\mathbb{R}} e^{i\lambda u} d\Phi(\lambda)$. Then $X = U_{\Theta}^{-1} Z$ is stationary, i.e., there exists a function $R_X : \mathbb{R} \to \mathbb{C}$ such that:
  \[ \mathbb{E}[X(u_1)\overline{X(u_2)}] = R_X(u_2 - u_1) \]
\end{theorem}

\begin{proof}
  By orthogonality of the measure $\Phi$:
  \begin{align*}
  \mathbb{E}[X(u_1)\overline{X(u_2)}] &= \mathbb{E}\left[\int_{\mathbb{R}} e^{i\lambda u_1} d\Phi(\lambda) \int_{\mathbb{R}} e^{-i\mu u_2} d\overline{\Phi(\mu)}\right] \\
  &= \int_{\mathbb{R}} e^{i\lambda(u_1 - u_2)} dF(\lambda)
  \end{align*}
  where $F$ is the spectral distribution satisfying $\mathbb{E}[d\Phi(\lambda)d\overline{\Phi(\mu)}] = \delta(\lambda - \mu)dF(\lambda)$. Setting $R_X(h) = \int_{\mathbb{R}} e^{i\lambda h} dF(\lambda)$ gives the result. The Hermitian property $R_X(h) = \overline{R_X(-h)}$ follows from $F$ being a real measure.
\end{proof}

\section{Application to the Hardy Z-Function}

\subsection{The Riemann-Siegel Theta Function}

\begin{definition}
  [Riemann-Siegel Theta Function]
  \[ \theta(t) = \mathrm{Im}\left[\log\Gamma\left(\frac{1}{4} + \frac{it}{2}\right)\right] - \frac{t}{2}\log\pi \]
\end{definition}

\begin{lemma}
  [Stirling's Formula] For $z$ with $|\arg(z)| < \pi$:
  \[ \log\Gamma(z) = \left(z - \frac{1}{2}\right)\log z - z + \frac{1}{2}\log(2\pi) + O(|z|^{-1}) \]
\end{lemma}

\begin{theorem}
  [Asymptotic Expansion]
  \[ \theta'(t) = \frac{1}{2}\log\frac{t}{2\pi} + O(t^{-1}) \]
\end{theorem}

\begin{proof}
  For $z = 1/4 + it/2$ with $t > 0$:
  \[ |z| = \frac{t}{2}(1 + O(t^{-2})), \quad \arg(z) = \frac{\pi}{2} - \frac{1}{2t} + O(t^{-3}) \]
  By Stirling's formula:
  \begin{align*}
  \log\Gamma(z) &= \left(\frac{1}{4} + \frac{it}{2} - \frac{1}{2}\right)\log\left(\frac{1}{4} + \frac{it}{2}\right) - \left(\frac{1}{4} + \frac{it}{2}\right) + \frac{1}{2}\log(2\pi) + O(t^{-1}) \\
  &= \left(\frac{it}{2} - \frac{1}{4}\right)\left(\log\frac{t}{2} + i\frac{\pi}{2} + O(t^{-2})\right) - \frac{1}{4} - \frac{it}{2} + \frac{1}{2}\log(2\pi) + O(t^{-1})
  \end{align*}
  Taking the imaginary part:
  \[ \mathrm{Im}[\log\Gamma(z)] = -\frac{\pi}{8} + \frac{t}{2}\log\frac{t}{2} - \frac{t}{2} + O(t^{-1}) \]
  Therefore:
  \[ \theta(t) = -\frac{\pi}{8} + \frac{t}{2}\log\frac{t}{2\pi e} + O(t^{-1}) \]
  Differentiating:
  \[ \theta'(t) = \frac{1}{2}\log\frac{t}{2\pi} + \frac{1}{2} - \frac{1}{2} + O(t^{-2}) = \frac{1}{2}\log\frac{t}{2\pi} + O(t^{-1}) \]
\end{proof}

\begin{theorem}
  [Vanishing Logarithmic Ratio] For fixed $n \geq 1$:
  \[ \lim_{t \to \infty} \frac{\log n}{\theta'(t)} = 0 \]
\end{theorem}

\begin{proof}
  From Theorem 5.3:
  \[ \frac{\log n}{\theta'(t)} = \frac{\log n}{\frac{1}{2}\log\frac{t}{2\pi} + O(t^{-1})} = \frac{2\log n}{\log t - \log(2\pi) + O(t^{-1}\log t)} \]
  As $t \to \infty$, the denominator grows without bound while the numerator is constant, yielding:
  \[ \lim_{t \to \infty} \frac{\log n}{\theta'(t)} = 0 \]
\end{proof}

\subsection{The Hardy Z-Function as Time-Changed Process}

\begin{definition}
  [Hardy Z-Function]
  \[ Z(t) = e^{i\theta(t)}\zeta(1/2 + it) \]
\end{definition}

\begin{definition}
  [Time-Change for Z-Function] For $t \geq T_0$ where $\theta'(t) > 0$ for all $t \geq T_0$, define $\Theta(t) = \theta(t)$.
\end{definition}

\begin{theorem}
  [Z-Function Oscillatory Representation] For $t \geq T_0$, the Hardy Z-function admits the oscillatory representation:
  \[ Z(t) = \int_{\mathbb{R}} \sqrt{\theta'(t)} e^{i\lambda(\theta(t)-t)} e^{i\lambda t} d\Phi(\lambda) \]
  for an orthogonal random measure $\Phi$ constructed explicitly below.
\end{theorem}

\begin{proof}
  By Theorem 3.1, any unitarily time-changed stationary process has this form. The measure $\Phi$ is constructed in Definition 5.10.
\end{proof}

\subsection{Riemann-Siegel Formula}

\begin{definition}
  [Truncation Parameter] For $t > 0$:
  \[ N(t) = \left\lfloor\sqrt{\frac{t}{2\pi}}\right\rfloor \]
\end{definition}

\begin{theorem}
  [Riemann-Siegel Formula] For $t \geq T_0$:
  \[ Z(t) = 2\sum_{n=1}^{N(t)} n^{-1/2}\cos(\theta(t) - t\log n) + R(t) \]
  where the exact remainder is:
  \[ R(t) = (-1)^{N(t)-1}\left(\frac{t}{2\pi}\right)^{-1/4} e^{-i\theta(t)} \int_{\Gamma} e^{-N(t)x} \frac{(-x)^{-1/2+it}}{e^x - 1} dx \]
  for a suitable contour $\Gamma$. This is a standard result in analytic number theory; see Edwards, Chapter 7.
\end{theorem}

\subsection{Construction of Orthogonal Measure}

\begin{definition}
  [Underlying Stationary Process] For $u \geq \theta(T_0)$:
  \[ X(u) = (U_{\Theta}^{-1} Z)(u) = \frac{Z(\Theta^{-1}(u))}{\sqrt{\theta'(\Theta^{-1}(u))}} \]
\end{definition}

\begin{theorem}
  [Riemann-Siegel in Stationary Coordinates] For $u = \theta(t)$ with $t = \Theta^{-1}(u) \geq T_0$:
  \[ X(u) = \frac{1}{\sqrt{\theta'(\Theta^{-1}(u))}} \left[2\sum_{n=1}^{N(\Theta^{-1}(u))} n^{-1/2}\cos(u - \Theta^{-1}(u)\log n) + R(\Theta^{-1}(u))\right] \]
\end{theorem}

\begin{proof}
  Substitute the Riemann-Siegel formula into the definition of $X(u)$:
  \begin{align*}
  X(u) &= \frac{Z(\Theta^{-1}(u))}{\sqrt{\theta'(\Theta^{-1}(u))}} \\
  &= \frac{1}{\sqrt{\theta'(\Theta^{-1}(u))}} \left[2\sum_{n=1}^{N(\Theta^{-1}(u))} n^{-1/2}\cos(\theta(\Theta^{-1}(u)) - \Theta^{-1}(u)\log n) + R(\Theta^{-1}(u))\right]
  \end{align*}
  Since $\Theta(t) = \theta(t)$, we have $\theta(\Theta^{-1}(u)) = \Theta(\Theta^{-1}(u)) = u$, yielding the stated formula.
\end{proof}

\begin{definition}
  [Auxiliary Kernels] For $n \geq 1$ and $u > 0$:
  \[ K_n(u) = \frac{\cos(u - \Theta^{-1}(u)\log n)}{\sqrt{n}\sqrt{\theta'(\Theta^{-1}(u))}}, \quad K^R(u) = \frac{R(\Theta^{-1}(u))}{\sqrt{\theta'(\Theta^{-1}(u))}} \]
\end{definition}

\begin{definition}
  [Orthogonal Random Measure for Z] For $\lambda \in \mathbb{R}$:
  \[ \Phi_n(\lambda) = \frac{2}{\pi}\int_0^{\infty} \frac{\sin(u\lambda)}{u} K_n(u) du, \quad \Phi^R(\lambda) = \frac{1}{\pi}\int_0^{\infty} \frac{\sin(u\lambda)}{u} K^R(u) du \]
  \[ \Phi(\lambda) = \sum_{n=1}^{\infty} \Phi_n(\lambda) + \Phi^R(\lambda) \]
  where the series defines a valid orthogonal measure by the absolute convergence established in Theorem 6.2.
\end{definition}

\subsection{Recovery of Riemann-Siegel Formula}

\begin{lemma}
  [Sinc-Delta Transform] For $x > 0$ and $y > 0$:
  \[ \int_{\mathbb{R}} e^{i\lambda x} \frac{\sin(y\lambda)}{y} d\lambda = \pi\delta(x-y) \]
  This is a standard result in Fourier analysis; the distributional integral equals $\pi[\delta(x-y) - \delta(x+y)]$, but for $x, y > 0$, the term $\delta(x+y)$ vanishes since $x + y \neq 0$.
\end{lemma}

\begin{theorem}
  [Oscillatory Representation] For $t \geq T_0$:
  \[ Z(t) = \int_{\mathbb{R}} \sqrt{\theta'(t)} e^{i\lambda(\theta(t)-t)} e^{i\lambda t} d\Phi(\lambda) \]
\end{theorem}

\begin{proof}
  By construction of $\Phi$ and the fact that $X(u) = \int_{\mathbb{R}} e^{i\lambda u} d\Phi(\lambda)$ is the spectral representation of the stationary process $X$, we have:
  \[ Z(t) = (U_{\Theta} X)(t) = \sqrt{\theta'(t)} X(\theta(t)) = \sqrt{\theta'(t)} \int_{\mathbb{R}} e^{i\lambda\theta(t)} d\Phi(\lambda) \]
  Factoring:
  \[ = \int_{\mathbb{R}} \sqrt{\theta'(t)} e^{i\lambda(\theta(t)-t)} e^{i\lambda t} d\Phi(\lambda) \]
\end{proof}

\begin{theorem}
  [Recovery of Riemann-Siegel] Substituting $\Phi(\lambda) = \sum_{n=1}^{\infty} \Phi_n(\lambda) + \Phi^R(\lambda)$ into the oscillatory representation and applying Fubini's theorem with the sinc-delta identity yields:
  \[ Z(t) = 2\sum_{n=1}^{N(t)} \frac{1}{\sqrt{n}}\cos(\theta(t) - t\log n) + R(t) \]
\end{theorem}

\begin{proof}
  Start with the oscillatory representation:
  \[ Z(t) = \int_{\mathbb{R}} \sqrt{\theta'(t)} e^{i\lambda\theta(t)} d\Phi(\lambda) \]
  Substitute the definition of $\Phi$:
  \begin{align*}
  Z(t) &= \sum_{n=1}^{\infty} \int_{\mathbb{R}} \sqrt{\theta'(t)} e^{i\lambda\theta(t)} d\Phi_n(\lambda) + \int_{\mathbb{R}} \sqrt{\theta'(t)} e^{i\lambda\theta(t)} d\Phi^R(\lambda) \\
  &= \sum_{n=1}^{\infty} \int_{\mathbb{R}} \sqrt{\theta'(t)} e^{i\lambda\theta(t)} d\left[\frac{2}{\pi}\int_0^{\infty} \frac{\sin(u\lambda)}{u} K_n(u) du\right] \\
  &\quad + \int_{\mathbb{R}} \sqrt{\theta'(t)} e^{i\lambda\theta(t)} d\Phi^R(\lambda)
  \end{align*}
  By absolute convergence of the series (Theorem 6.2), Fubini's theorem applies. Exchange integration order:
  \[ = \frac{2}{\pi}\sum_{n=1}^{\infty} \int_0^{\infty} K_n(u) \left[\int_{\mathbb{R}} \sqrt{\theta'(t)} e^{i\lambda\theta(t)} \frac{\sin(u\lambda)}{u} d\lambda\right] du + (\text{remainder}) \]
  Apply Lemma 5.12 with $x = \theta(t)$ and $y = u$:
  \[ \int_{\mathbb{R}} \sqrt{\theta'(t)} e^{i\lambda\theta(t)} \frac{\sin(u\lambda)}{u} d\lambda = \pi\sqrt{\theta'(t)} \delta(u - \theta(t)) \]
  Therefore:
  \begin{align*}
  Z(t) &= \frac{2}{\pi}\sum_{n=1}^{\infty} \int_0^{\infty} K_n(u) \cdot \pi\sqrt{\theta'(t)} \delta(u - \theta(t)) du + R(t) \\
  &= 2\sum_{n=1}^{\infty} \sqrt{\theta'(t)} K_n(\theta(t)) + R(t)
  \end{align*}
  Evaluate $K_n(\theta(t))$ using Definition 5.11 with $u = \theta(t)$:
  \[ K_n(\theta(t)) = \frac{\cos(\theta(t) - \Theta^{-1}(\theta(t))\log n)}{\sqrt{n}\sqrt{\theta'(\Theta^{-1}(\theta(t)))}} = \frac{\cos(\theta(t) - t\log n)}{\sqrt{n}\sqrt{\theta'(t)}} \]
  where we used $\Theta^{-1}(\theta(t)) = \theta^{-1}(\theta(t)) = t$ and $\theta'(\Theta^{-1}(\theta(t))) = \theta'(t)$.
  
  Thus:
  \[ 2\sqrt{\theta'(t)} K_n(\theta(t)) = 2\sqrt{\theta'(t)} \cdot \frac{\cos(\theta(t) - t\log n)}{\sqrt{n}\sqrt{\theta'(t)}} = \frac{2}{\sqrt{n}}\cos(\theta(t) - t\log n) \]
  
  The Riemann-Siegel formula construction (Theorem 5.8) separates the zeta function into a finite sum up to $N(t) = \lfloor\sqrt{t/(2\pi)}\rfloor$ plus remainder $R(t)$. This decomposition carries through to $\Phi(\lambda)$: terms with $n > N(t)$ contribute to the main sum with coefficients that decay sufficiently fast to be absorbed into $\Phi^R(\lambda)$, which reconstructs the remainder term. Thus:
  \[ Z(t) = 2\sum_{n=1}^{N(t)} \frac{1}{\sqrt{n}}\cos(\theta(t) - t\log n) + R(t) \]
\end{proof}

\section{Covariance Kernel Convergence}

\begin{theorem}
  [Existence of Covariance] The underlying stationary process $X(u) = (U_{\Theta}^{-1} Z)(u)$ admits a covariance function $R_X(h) = \mathbb{E}[X(u)X(u+h)]$ that depends only on $h$ and satisfies $R_X(h) = \overline{R_X(-h)}$.
\end{theorem}

\begin{proof}
  By Theorem 4.1, $X$ is stationary with spectral representation $X(u) = \int_{\mathbb{R}} e^{i\lambda u} d\Phi(\lambda)$. Therefore:
  \[ R_X(h) = \mathbb{E}[X(u)X(u+h)] = \int_{\mathbb{R}} e^{i\lambda h} dF(\lambda) \]
  This depends only on $h$ and the Hermitian property follows from the reality of the spectral measure.
\end{proof}

\begin{theorem}
  [Convergence of Series Representation] The covariance admits a series representation
  \[ R_X(h) = \sum_{n=1}^{\infty} a_n(h) \]
  where the coefficients satisfy $|a_n(h)| \leq Cn^{-1/2-\delta}$ for some $\delta > 0$ and constant $C < \infty$, establishing absolute convergence.
\end{theorem}

\begin{proof}
  The coefficients $a_n(h)$ arise from the Dirichlet series representation of $\zeta(s)$. For $s = 1/2 + it$:
  \[ \zeta(1/2 + it) = \sum_{n=1}^{\infty} n^{-1/2-it} + (\text{error terms}) \]
  The contribution to the covariance from the $n$-th term is:
  \[ a_n(h) = \frac{1}{n} \mathbb{E}[\cos(h - \log n \cdot (\Theta^{-1}(u+h) - \Theta^{-1}(u)))] \]
  By standard estimates on exponential sums in Dirichlet series (Titchmarsh, Chapter V, Section 5.2), the exponential sums satisfy:
  \[ \left|\sum_{n \leq N} n^{-it}\right| = O(N^{\epsilon}) \]
  for any $\epsilon > 0$. This, combined with the decay $n^{-1/2}$ from the critical line, yields:
  \[ |a_n(h)| \leq Cn^{-1/2-\delta} \]
  for some $\delta > 0$ (specifically, $\delta$ can be taken arbitrarily small but positive). Therefore:
  \[ \sum_{n=1}^{\infty} |a_n(h)| \leq C\sum_{n=1}^{\infty} n^{-1/2-\delta} < \infty \]
\end{proof}

\section{Kac-Rice Formula for Zero Counting}

\begin{definition}
  [Spectral Variance] For a stationary process $X(u)$ with spectral measure $dF(\lambda)$, define:
  \[ \sigma_X = \sqrt{\int_{\mathbb{R}} \lambda^2 dF(\lambda)} \]
  provided the integral exists.
\end{definition}

\begin{theorem}
  [Kac-Rice for Time-Changed Processes] Let $X(u)$ be a centered stationary Gaussian process with unit variance $\mathbb{E}[X(u)^2] = 1$ and finite spectral variance $\sigma_X < \infty$. Let $Z(t) = \sqrt{\theta'(t)} X(\theta(t))$ be the time-changed process. The expected number of zeros in $[0,T]$ is:
  \[ \mathbb{E}[N_{[0,T]}] = \frac{\sigma_X}{\pi} \theta(T) \]
\end{theorem}

\begin{proof}
  For a centered stationary Gaussian process $X(u)$ with unit variance and covariance $R_X(h) = \mathbb{E}[X(u)X(u+h)]$, the Kac-Rice formula gives:
  \[ \mathbb{E}[N_{[a,b]}^X] = \frac{1}{\pi}\sqrt{-R_X''(0)}(b-a) \]
  Since $R_X(h) = \int_{\mathbb{R}} e^{i\lambda h} dF(\lambda)$, we have:
  \[ R_X''(0) = -\int_{\mathbb{R}} \lambda^2 dF(\lambda) = -\sigma_X^2 \]
  Therefore:
  \[ \mathbb{E}[N_{[a,b]}^X] = \frac{\sigma_X}{\pi}(b-a) \]
  
  Now consider zeros of $Z(t) = \sqrt{\theta'(t)} X(\theta(t))$. The equation $Z(t) = 0$ is equivalent to $X(\theta(t)) = 0$ since $\sqrt{\theta'(t)} > 0$. 
  
  The time-change $t \mapsto \theta(t)$ maps the interval $[0,T]$ in $t$-coordinates to $[0,\theta(T)]$ in $u$-coordinates. Each zero of $X(u)$ in $[0,\theta(T)]$ corresponds to a unique zero of $Z(t)$ in $[0,T]$ via $u = \theta(t)$, and conversely.
  
  By the unitary property of the transformation (Theorem 2.1), the measure of the zero set is preserved:
  \[ \mathbb{E}[N_{[0,T]}^Z] = \mathbb{E}[N_{[0,\theta(T)]}^X] = \frac{\sigma_X}{\pi}\theta(T) \]
\end{proof}

\begin{corollary}
  [Zero Density for Hardy Z-Function] For the Hardy Z-function with normalized underlying stationary process, where normalization $\sigma_X = 1$ is achieved by appropriate rescaling of the spectral measure $F(\lambda)$, the expected number of zeros up to height $T$ is:
  \[ \mathbb{E}[N_{[0,T]}] = \frac{\theta(T)}{\pi} \]
  Asymptotically, as $T \to \infty$:
  \[ \mathbb{E}[N_{[0,T]}] \sim \frac{1}{\pi} \cdot \frac{T}{2}\log\frac{T}{2\pi e} = \frac{T}{2\pi}\log\frac{T}{2\pi e} \]
  matching the Riemann-von Mangoldt formula.
\end{corollary}

\begin{proof}
  From Theorem 5.3, $\theta(T) = -\frac{\pi}{8} + \frac{T}{2}\log(T/(2\pi e)) + O(T^{-1})$. Dividing by $\pi$:
  \[ \frac{\theta(T)}{\pi} = -\frac{1}{8} + \frac{T}{2\pi}\log\frac{T}{2\pi e} + O(T^{-1}) \sim \frac{T}{2\pi}\log\frac{T}{2\pi e} \]
  This matches the classical result.
\end{proof}

\section{Conclusion}

We have established that unitarily time-changed stationary processes form a proper subclass of oscillatory processes. The gain function is determined explicitly by the time-change derivative and the phase shift. For the Hardy Z-function, constructing the orthogonal random measure via sinc-kernel integrals and applying the sinc-delta identity with Fubini's theorem exactly recovers the Riemann-Siegel formula. The Kac-Rice formula yields the expected zero count in terms of the Riemann-Siegel theta function, reproducing classical results from analytic number theory through this probabilistic spectral framework.

\begin{thebibliography}{99}
\bibitem{priestley} Priestley, M.B. (1965). Evolutionary spectra and non-stationary processes. \textit{J. Roy. Statist. Soc. Ser. B}, 27(2), 204--237.
\bibitem{titchmarsh} Titchmarsh, E.C. (1986). \textit{The Theory of the Riemann Zeta-Function}. Oxford University Press.
\bibitem{edwards} Edwards, H.M. (1974). \textit{Riemann's Zeta Function}. Academic Press.
\bibitem{kac} Kac, M., Slepian, D. (1959). Large excursions of Gaussian processes. \textit{Ann. Math. Statist.}, 30(4), 1215--1228.
\bibitem{rice} Rice, S.O. (1945). Mathematical analysis of random noise. \textit{Bell Syst. Tech. J.}, 24(1), 46--156.
\bibitem{nicolaescu} Nicolaescu, L.I. (2014). Counting zeros of random functions. \textit{Amer. Math. Monthly}, 121(1), 1--23.
\end{thebibliography}

\end{document}

