\documentclass{article}
\usepackage[english]{babel}
\usepackage{geometry,amsmath,amssymb,latexsym,theorem}
\geometry{letterpaper}

%%%%%%%%%% Start TeXmacs macros
\newcommand{\assign}{:=}
\newcommand{\tmaffiliation}[1]{\\ #1}
\newenvironment{proof}{\noindent\textbf{Proof\ }}{\hspace*{\fill}$\Box$\medskip}
\newtheorem{definition}{Definition}
{\theorembodyfont{\rmfamily}\newtheorem{remark}{Remark}}
\newtheorem{theorem}{Theorem}
%%%%%%%%%% End TeXmacs macros

\begin{document}

\title{Unitary Time Changes of Stationary Processes Yield Oscillatory
Processes}

\author{
  Stephen Crowley
  \tmaffiliation{August 12, 2025}
}

\date{}

\maketitle

\begin{definition}
  [Unitary time change operator on $L^2 (\mathbb{R})$] Let $\theta :
  \mathbb{R} \to \mathbb{R}$ be absolutely continuous with $\theta' (t) \neq
  0$ almost everywhere. The unitary time change operator $U_{\theta}$ on $L^2
  (\mathbb{R})$ is defined by
  \begin{equation}
    (U_{\theta} f) (t) \assign \sqrt{| \theta' (t) |}  \hspace{0.17em} f
    (\theta (t))  \qquad \text{for } f \in L^2 (\mathbb{R})
  \end{equation}
\end{definition}

\begin{theorem}
  [Unitarity of $U_{\theta}$] The operator $U_{\theta}$ defined above is
  unitary on $L^2 (\mathbb{R})$.
\end{theorem}

\begin{proof}
  Absolute continuity with $\theta' (t) \neq 0$ a.e. implies the
  change-of-variables formula
  \begin{equation}
    \int_{\mathbb{R}} | (U_{\theta} f) (t) |^2 \hspace{0.17em} dt =
    \int_{\mathbb{R}} | \theta' (t) | \hspace{0.17em} |f (\theta (t)) |^2
    \hspace{0.17em} dt = \int_{\mathbb{R}} |f (u) |^2  \hspace{0.17em} du
  \end{equation}
  so $U_{\theta}$ is isometric. Surjectivity follows from the same
  change-of-variables applied to $U_{\theta^{- 1}}$, which exists almost
  everywhere under these hypotheses. Hence $U_{\theta}$ is unitary.
\end{proof}

\begin{definition}
  [Oscillatory processes in the sense of Priestley] An oscillatory process $Z$
  is specified by a measurable gain function $A_t (\lambda)$ and has
  oscillatory function
  \begin{equation}
    \varphi_t (\lambda) \assign A_t (\lambda)  \hspace{0.17em} e^{i \lambda t}
  \end{equation}
  The process $Z$ has spectral representation
  \begin{equation}
    Z (t) = \int_{\mathbb{R}} \varphi_t (\lambda)  \hspace{0.17em} \Phi (d
    \lambda) = \int_{\mathbb{R}} A_t (\lambda)  \hspace{0.17em} e^{i \lambda
    t}  \hspace{0.17em} \Phi (d \lambda)
  \end{equation}
  where $\Phi$ is a complex orthogonal random measure on $\mathbb{R}$ with
  spectral measure $F$ satisfying
  \begin{equation}
    E \left[ \Phi (d \lambda) \hspace{0.17em} \overline{\Phi (d \mu)} \right]
    = \textbf{1}_{\{\lambda = \mu\}}  \hspace{0.17em} dF (\lambda)
  \end{equation}
  The covariance kernel of $Z$ is
  \begin{equation}
    R_Z (t, s) \assign E [Z (t) \overline{Z (s)}] = \int_{\mathbb{R}} A_t
    (\lambda) \hspace{0.17em} \overline{A_s (\lambda)} \hspace{0.17em} e^{i
    \lambda (t - s)}  \hspace{0.17em} dF (\lambda)
  \end{equation}
\end{definition}

\begin{remark}
  [Real-valuedness condition] The oscillatory process $Z$ is real-valued if
  and only if the gain satisfies conjugate symmetry:
  \begin{equation}
    A_t  (- \lambda) = \overline{A_t (\lambda)} \quad \text{for $F$-almost
    every } \lambda, \text{for each fixed } t
  \end{equation}
\end{remark}

\begin{theorem}
  [Unitary time change of stationary process yields oscillatory process] Let
  $X$ be a zero-mean stationary Gaussian process with Cram{\'e}r spectral
  representation
  \begin{equation}
    X (t) = \int_{\mathbb{R}} e^{i \lambda t}  \hspace{0.17em} \Phi (d
    \lambda)
  \end{equation}
  where $\Phi$ is the same complex orthogonal random measure with spectral
  measure $F$ as in the oscillatory definition. Let $U_{\theta}$ be a unitary
  time change operator as defined above. Then the transformed process
  \begin{equation}
    Z (t) \assign (U_{\theta} X) (t) = \sqrt{| \theta' (t) |}  \hspace{0.17em}
    X (\theta (t))
  \end{equation}
  is an oscillatory process in the sense of Priestley with oscillatory
  function
  \begin{equation}
    \varphi_t (\lambda) = \sqrt{| \theta' (t) |}  \hspace{0.17em} e^{i \lambda
    \theta (t)}
  \end{equation}
\end{theorem}

\begin{proof}
  Starting from the stationary representation, we compute
  
  \begin{align}
    Z (t) & = \sqrt{| \theta' (t) |}  \hspace{0.17em} X (\theta (t)) \\
    & = \sqrt{| \theta' (t) |}  \int_{\mathbb{R}} e^{i \lambda \theta (t)} 
    \hspace{0.17em} \Phi (d \lambda) \\
    & = \int_{\mathbb{R}} \sqrt{| \theta' (t) |}  \hspace{0.17em} e^{i
    \lambda \theta (t)}  \hspace{0.17em} \Phi (d \lambda) 
  \end{align}
  
  Defining
  \begin{equation}
    \varphi_t (\lambda) \assign \sqrt{| \theta' (t) |}  \hspace{0.17em} e^{i
    \lambda \theta (t)}
  \end{equation}
  we have
  \[ Z (t) = \int_{\mathbb{R}} \varphi_t (\lambda)  \hspace{0.17em} \Phi (d
     \lambda) \]
  which is precisely the oscillatory form. The covariance kernel becomes
  \[ R_Z (t, s) = \int_{\mathbb{R}} \varphi_t (\lambda) \hspace{0.17em}
     \overline{\varphi_s (\lambda)} \hspace{0.17em} dF (\lambda) =
     \int_{\mathbb{R}} \sqrt{| \theta' (t) || \theta' (s) |}  \hspace{0.17em}
     e^{i \lambda (\theta (t) - \theta (s))}  \hspace{0.17em} dF (\lambda) \]
\end{proof}

\begin{theorem}
  [Explicit gain function for unitary time change] In the setting of the
  previous theorem, the gain function for the oscillatory process
  \begin{equation}
    Z (t) = (U_{\theta} X) (t)
  \end{equation}
  is given by
  \begin{equation}
    A_t (\lambda) = \sqrt{| \theta' (t) |}  \hspace{0.17em} e^{i \lambda
    (\theta (t) - t)}
  \end{equation}
  The oscillatory function is
  \begin{equation}
    \varphi_t (\lambda) = A_t (\lambda)  \hspace{0.17em} e^{i \lambda t} =
    \sqrt{| \theta' (t) |}  \hspace{0.17em} e^{i \lambda \theta (t)}
  \end{equation}
  and the covariance kernel takes the form
  \begin{equation}
    R_Z (t, s) = \int_{\mathbb{R}} A_t (\lambda) \hspace{0.17em} \overline{A_s
    (\lambda)} \hspace{0.17em} e^{i \lambda (t - s)}  \hspace{0.17em} dF
    (\lambda)
  \end{equation}
\end{theorem}

\begin{proof}
  From the previous theorem, we have
  \begin{equation}
    \varphi_t (\lambda) = \sqrt{| \theta' (t) |}  \hspace{0.17em} e^{i \lambda
    \theta (t)}
  \end{equation}
  Since the oscillatory function must satisfy
  \begin{equation}
    \varphi_t (\lambda) = A_t (\lambda) e^{i \lambda t}
  \end{equation}
  one solves for the gain:
  \[ A_t (\lambda) = \frac{\varphi_t (\lambda)}{e^{i \lambda t}} =
     \frac{\sqrt{| \theta' (t) |}  \hspace{0.17em} e^{i \lambda \theta
     (t)}}{e^{i \lambda t}} = \sqrt{| \theta' (t) |}  \hspace{0.17em} e^{i
     \lambda (\theta (t) - t)} \]
  and substitutes back into the covariance formula:
  
  \begin{align}
    R_Z (t, s) & = \int_{\mathbb{R}} \varphi_t (\lambda)  \hspace{0.17em}
    \bar{\varphi}_s (\lambda) \} \hspace{0.17em} dF (\lambda) \\
    & = \int_{\mathbb{R}} A_t (\lambda) e^{i \lambda t} \hspace{0.17em}
    \overline{A_s (\lambda) e^{i \lambda s}} \hspace{0.17em} dF (\lambda) \\
    & = \int_{\mathbb{R}} A_t (\lambda) \hspace{0.17em} \overline{A_s
    (\lambda)} \hspace{0.17em} e^{i \lambda (t - s)}  \hspace{0.17em} dF
    (\lambda) 
  \end{align}
\end{proof}

-----\hrulefill

\

\begin{theorem}
  [Unitary time change on $L^2 (\mathbb{R})$] Let $\theta : \mathbb{R} \to
  \mathbb{R}$ be absolutely continuous with $\theta' (t) \neq 0$ almost
  everywhere. Define the operator
  \[ (U_{\theta} f) (t) \assign \sqrt{| \theta' (t) |}  \hspace{0.17em} f
     (\theta (t))  \qquad \text{for } f \in L^2 (\mathbb{R}) . \]
  Then $U_{\theta}$ is unitary on $L^2 (\mathbb{R})$.
\end{theorem}

\begin{proof}
  By absolute continuity and $\theta' (t) \neq 0$ a.e., the
  change-of-variables formula gives
  \[ \int_{\mathbb{R}} | (U_{\theta} f) (t) |^2 \hspace{0.17em} dt =
     \int_{\mathbb{R}} | \theta' (t) | \hspace{0.17em} |f (\theta (t)) |^2
     \hspace{0.17em} dt = \int_{\mathbb{R}} |f (u) |^2  \hspace{0.17em} du, \]
  so $U_{\theta}$ is an isometry. The inverse time change $\theta^{- 1}$
  exists a.e. and is absolutely continuous, yielding an isometric inverse by
  the same computation; hence $U_{\theta}$ is unitary.
\end{proof}

\begin{theorem}
  [Oscillatory processes (Priestley framework)] Fix a finite nonnegative
  measure $F$ on $\mathbb{R}$. For each $t \in \mathbb{R}$, let $A_t :
  \mathbb{R} \to \mathbb{C}$ be measurable with
  \[ \int_{\mathbb{R}} |A_t (\lambda) |^2  \hspace{0.17em} dF (\lambda) <
     \infty . \]
  Define the oscillatory function by
  \[ \varphi_t (\lambda) \assign A_t (\lambda)  \hspace{0.17em} e^{i \lambda
     t} . \]
  There exists a complex orthogonal random measure $\Phi$ on $\mathbb{R}$ with
  spectral measure $F$ such that the stochastic integral
  \[ Z (t) \assign \int_{\mathbb{R}} \varphi_t (\lambda)  \hspace{0.17em} \Phi
     (d \lambda) \]
  is well-defined for each $t$, and the covariance kernel satisfies
  \[ R_Z (t, s) \assign \mathbb{E} [Z (t) \overline{Z (s)}] =
     \int_{\mathbb{R}} \varphi_t (\lambda) \hspace{0.17em} \overline{\varphi_s
     (\lambda)} \hspace{0.17em} dF (\lambda) = \int_{\mathbb{R}} A_t (\lambda)
     \hspace{0.17em} \overline{A_s (\lambda)} \hspace{0.17em} e^{i \lambda (t
     - s)}  \hspace{0.17em} dF (\lambda) . \]
  Moreover, if $X$ is a zero-mean stationary process with spectral
  representation $X (t) = \int_{\mathbb{R}} e^{i \lambda t}  \hspace{0.17em}
  \Phi (d \lambda)$ for the same $F$ and $\Phi$, then $Z$ reduces to $X$ when
  $A_t (\lambda) \equiv 1$.
\end{theorem}

\begin{proof}
  Given $F$, there exists a complex orthogonal random measure $\Phi$ with
  spectral measure $F$, i.e.,
  \[ \mathbb{E} \left[ \Phi (d \lambda) \hspace{0.17em} \overline{\Phi (d
     \mu)} \right] = \textbf{1}_{\{\lambda = \mu\}}  \hspace{0.17em} dF
     (\lambda) . \]
  Square-integrability of $\varphi_t$ with respect to $F$ ensures the
  stochastic integral isometric definition of $Z (t)$ and yields
  \[ \mathbb{E} [Z (t) \overline{Z (s)}] = \int \varphi_t (\lambda)
     \hspace{0.17em} \overline{\varphi_s (\lambda)} \hspace{0.17em} dF
     (\lambda) . \]
  Substituting $\varphi_t (\lambda) = A_t (\lambda) e^{i \lambda t}$ gives the
  stated kernel. If $A_t \equiv 1$, then $\varphi_t (\lambda) = e^{i \lambda
  t}$ and $Z$ coincides with the stationary Cram{\'e}r form $X$ built from the
  same $\Phi$.
\end{proof}

\begin{theorem}
  [Real-valuedness condition] Let $Z$ be as above with oscillatory function
  $\varphi_t (\lambda) = A_t (\lambda) e^{i \lambda t}$. The process $Z$ is
  real-valued if and only if, for each fixed $t$,
  \[ A_t  (- \lambda) = \overline{A_t (\lambda)} \quad \text{for } F
     \text{-almost every } \lambda, \]
  equivalently,
  \[ \varphi_t  (- \lambda) = \overline{\varphi_t (\lambda)} \quad \text{for }
     F \text{-almost every } \lambda . \]
\end{theorem}

\begin{proof}
  Write $Z (t) = \int \varphi_t (\lambda)  \hspace{0.17em} \Phi (d \lambda)$.
  Real-valuedness of $Z (t)$ is equivalent to $Z (t) = \overline{Z (t)}$ in
  $L^2 (\Omega)$, i.e.,
  \[ \int \varphi_t (\lambda)  \hspace{0.17em} \Phi (d \lambda) =
     \overline{\int \varphi_t (\lambda)  \hspace{0.17em} \Phi (d \lambda)} =
     \int \overline{\varphi_t (\lambda)} \hspace{0.17em} \overline{\Phi (d
     \lambda)} . \]
  Using the standard symmetry relation for complex orthogonal random measures
  associated with real processes (the negative-frequency part is the complex
  conjugate of the positive-frequency part in the $L^2$ sense), one arrives at
  the necessary and sufficient Hermitian symmetry of the integrand: $\varphi_t
  (- \lambda) = \overline{\varphi_t (\lambda)}$ $F$-a.e. As $e^{i (- \lambda)
  t} = \overline{e^{i \lambda t}}$, this is equivalent to $A_t  (- \lambda) =
  \overline{A_t (\lambda)}$ $F$-a.e.
\end{proof}

\begin{theorem}
  [Unitary time change of a stationary process is oscillatory; explicit gain]
  Let $X$ be a zero-mean stationary Gaussian process with spectral
  representation
  \[ X (t) = \int_{\mathbb{R}} e^{i \lambda t}  \hspace{0.17em} \Phi (d
     \lambda), \]
  for a complex orthogonal random measure $\Phi$ with spectral measure $F$.
  Let $\theta$ satisfy the hypotheses of the unitary theorem, and define
  \[ Z (t) \assign (U_{\theta} X) (t) = \sqrt{| \theta' (t) |} 
     \hspace{0.17em} X (\theta (t)) . \]
  Then $Z$ is an oscillatory process in the sense above with oscillatory
  function
  \[ \varphi_t (\lambda) = \sqrt{| \theta' (t) |}  \hspace{0.17em} e^{i
     \lambda \theta (t)}, \]
  and gain
  \[ A_t (\lambda) = \sqrt{| \theta' (t) |}  \hspace{0.17em} e^{i \lambda
     (\theta (t) - t)} . \]
  Its covariance kernel is
  \[ R_Z (t, s) = \int_{\mathbb{R}} \varphi_t (\lambda) \hspace{0.17em}
     \overline{\varphi_s (\lambda)} \hspace{0.17em} dF (\lambda) =
     \int_{\mathbb{R}} A_t (\lambda) \hspace{0.17em} \overline{A_s (\lambda)}
     \hspace{0.17em} e^{i \lambda (t - s)}  \hspace{0.17em} dF (\lambda) . \]
  Moreover, $Z$ is real-valued if and only if
  \[ A_t  (- \lambda) = \overline{A_t (\lambda)} \quad \text{for } F
     \text{-almost every } \lambda, \text{for each } t. \]
\end{theorem}

\begin{proof}
  From the previous theorem, we have $\varphi_t (\lambda) = \sqrt{| \theta'
  (t) |}  \hspace{0.17em} e^{i \lambda \theta (t)}$. Since the oscillatory
  function must satisfy $\varphi_t (\lambda) = A_t (\lambda) e^{i \lambda t}$,
  we solve for the gain:
  \[ A_t (\lambda) = \frac{\varphi_t (\lambda)}{e^{i \lambda t}} =
     \frac{\sqrt{| \theta' (t) |}  \hspace{0.17em} e^{i \lambda \theta
     (t)}}{e^{i \lambda t}} . \]
  Using the exponential division rule $\frac{e^a}{e^b} = e^{a - b}$, we get:
  \[ A_t (\lambda) = \sqrt{| \theta' (t) |}  \hspace{0.17em} \frac{e^{i
     \lambda \theta (t)}}{e^{i \lambda t}} = \sqrt{| \theta' (t) |} 
     \hspace{0.17em} e^{i \lambda \theta (t) - i \lambda t} = \sqrt{| \theta'
     (t) |}  \hspace{0.17em} e^{i \lambda (\theta (t) - t)} . \]
  Substituting back into the covariance formula:
  
  \begin{align*}
    R_Z (t, s) & = \int_{\mathbb{R}} \varphi_t (\lambda) \hspace{0.17em}
    \overline{\varphi_s (\lambda)} \hspace{0.17em} dF (\lambda)\\
    & = \int_{\mathbb{R}} A_t (\lambda) e^{i \lambda t} \hspace{0.17em}
    \overline{A_s (\lambda) e^{i \lambda s}} \hspace{0.17em} dF (\lambda)\\
    & = \int_{\mathbb{R}} A_t (\lambda) e^{i \lambda t} \hspace{0.17em}
    \overline{A_s (\lambda)} \hspace{0.17em} \overline{e^{i \lambda s}}
    \hspace{0.17em} dF (\lambda)\\
    & = \int_{\mathbb{R}} A_t (\lambda) e^{i \lambda t} \hspace{0.17em}
    \overline{A_s (\lambda)} \hspace{0.17em} e^{- i \lambda s} 
    \hspace{0.17em} dF (\lambda)\\
    & = \int_{\mathbb{R}} A_t (\lambda) \hspace{0.17em} \overline{A_s
    (\lambda)} \hspace{0.17em} e^{i \lambda t}  \hspace{0.17em} e^{- i \lambda
    s}  \hspace{0.17em} dF (\lambda)\\
    & = \int_{\mathbb{R}} A_t (\lambda) \hspace{0.17em} \overline{A_s
    (\lambda)} \hspace{0.17em} e^{i \lambda t - i \lambda s}  \hspace{0.17em}
    dF (\lambda)\\
    & = \int_{\mathbb{R}} A_t (\lambda) \hspace{0.17em} \overline{A_s
    (\lambda)} \hspace{0.17em} e^{i \lambda (t - s)}  \hspace{0.17em} dF
    (\lambda) .
  \end{align*}
\end{proof}

\end{document}
