\documentclass[11pt,a4paper]{article}
\usepackage{amsmath}
\usepackage{amssymb}
\usepackage{amsfonts}
\usepackage{mathtools}
\usepackage{amsthm}
\usepackage{geometry}
\usepackage{hyperref}
\geometry{margin=1in}

\theoremstyle{plain}
\newtheorem{theorem}{Theorem}[section]
\newtheorem{lemma}[theorem]{Lemma}
\newtheorem{proposition}[theorem]{Proposition}
\newtheorem{corollary}[theorem]{Corollary}

\theoremstyle{definition}
\newtheorem{definition}[theorem]{Definition}
\newtheorem{remark}[theorem]{Remark}

\newcommand{\assign}{\coloneqq}

\title{The Hardy Z-Function as a Unitarily Time-Changed Stationary Process}
\author{Stephen Crowley}
\date{December 20, 2025}

\begin{document}

\maketitle

\begin{abstract}
The Hardy Z-function admits a representation as a unitarily time-changed stationary process. We establish that the inverse unitary transform yields an underlying process with well-defined normalized stationary covariance, and demonstrate that the Hardy Z-function satisfies the definition of an oscillatory process in Priestley's framework with explicitly computed gain function.
\end{abstract}

\tableofcontents

\section{Introduction}

The Hardy Z-function
\[
Z(t) = e^{i\theta(t)}\zeta\left(\frac{1}{2}+it\right)
\]
where \(\theta(t)\) is the Riemann-Siegel theta function, encodes the zeros of the Riemann zeta function on the critical line. This paper establishes that \(Z(t)\) is a unitarily time-changed stationary process, identified as a subclass of oscillatory processes in Priestley's evolutionary spectra framework. There exists a strictly increasing, absolutely continuous function \(\Theta(t)\) such that the inverse unitary operator applied to \(Z(t)\) yields a process \(X(u)\) with well-defined normalized stationary covariance, and \(Z(t)\) satisfies the definition of an oscillatory process.

\section{The Unitary Time-Change Operator}

\begin{definition}[Unitary Time-Change Operator]\label{def:unitary_op}
Let \(\Theta\colon\mathbb{R}\to\mathbb{R}\) be absolutely continuous, strictly increasing, and bijective with \(\dot{\Theta}(t)>0\) almost everywhere. For measurable \(f\), define:
\[
(U_{\Theta} f)(t) = \sqrt{\dot{\Theta}(t)}\, f(\Theta(t))
\]
The inverse operator is:
\[
(U_{\Theta}^{-1}g)(s) = \frac{g(\Theta^{-1}(s))}{\sqrt{\dot{\Theta}(\Theta^{-1}(s))}}
\]
\end{definition}

\begin{theorem}[Local Isometry and Inverse Properties]\label{thm:isometry}
For every compact \(K\subseteq\mathbb{R}\) and \(f\in L^2_{\mathrm{loc}}(\mathbb{R})\):
\[
\int_K|(U_{\Theta}f)(t)|^2\,dt = \int_{\Theta(K)}|f(s)|^2\,ds
\]
Moreover, for \(f,g\in L^2_{\mathrm{loc}}(\mathbb{R})\):
\[
U_{\Theta}^{-1}(U_{\Theta}f) = f,\qquad U_{\Theta}(U_{\Theta}^{-1}g) = g
\]
\end{theorem}

\begin{proof}
Using the change of variables \(s=\Theta(t)\) where \(ds=\dot{\Theta}(t)\,dt\):
\[
\int_K|(U_{\Theta}f)(t)|^2\,dt = \int_K\dot{\Theta}(t)|f(\Theta(t))|^2\,dt = \int_{\Theta(K)}|f(s)|^2\,ds
\]
The inverse identities follow from functional composition: \(\Theta(\Theta^{-1}(s))=s\) and \(\Theta^{-1}(\Theta(t))=t\).
\end{proof}

\section{Oscillatory Processes}

\begin{definition}[Oscillatory Process]\label{def:oscillatory}
An oscillatory process has the form:
\[
Z(t) = \int_{\mathbb{R}}A_t(\lambda)e^{i\lambda t}\,d\Phi(\lambda)
\]
where \(A_t(\lambda)\) is a gain function and \(\Phi\) is a complex orthogonal random measure.
\end{definition}

\begin{theorem}[Unitary Time Change Produces Oscillatory Process]\label{thm:unitary_oscillatory}
Let \(X\) be a zero-mean stationary process with representation:
\[
X(t) = \int_{\mathbb{R}}e^{i\lambda t}\,d\Phi(\lambda)
\]
Let \(\Theta\) be strictly increasing and absolutely continuous. Then
\[
Z(t) = (U_{\Theta}X)(t) = \sqrt{\dot{\Theta}(t)}X(\Theta(t))
\]
is an oscillatory process with gain function:
\[
A_t(\lambda) = \sqrt{\dot{\Theta}(t)}e^{i\lambda(\Theta(t)-t)}
\]
\end{theorem}

\begin{proof}
Substituting into the spectral representation:
\[
Z(t) = \sqrt{\dot{\Theta}(t)}\int_{\mathbb{R}}e^{i\lambda\Theta(t)}\,d\Phi(\lambda) = \int_{\mathbb{R}}\sqrt{\dot{\Theta}(t)}e^{i\lambda(\Theta(t)-t)}e^{i\lambda t}\,d\Phi(\lambda)
\]
This exhibits \(Z(t)\) in the oscillatory process form with the stated gain function.
\end{proof}

\section{The Hardy Z-Function}

\begin{definition}[Hardy Z-Function]
\[
Z(t) = e^{i\theta(t)}\zeta\left(\frac{1}{2}+it\right)
\]
where \(\theta(t)\) is the Riemann-Siegel theta function:
\[
\theta(t) = \frac{t}{2}\log\left(\frac{t}{2\pi}\right) - \frac{t}{2} - \frac{\pi}{8} + O\left(\frac{1}{t}\right)
\]
\end{definition}

\begin{proposition}[Functional Equation]\label{prop:functional_eq}
The Hardy Z-function satisfies:
\[
Z(-t) = Z(t)
\]
\end{proposition}

\begin{proof}
By the antisymmetry \(\theta(-t) = -\theta(t)\) and the functional equation of \(\zeta\):
\[
Z(-t) = e^{-i\theta(t)}\overline{\zeta(1/2+it)} = \overline{Z(t)} = Z(t)
\]
where the last equality holds because \(Z(t)\) is real-valued by construction.
\end{proof}

\begin{lemma}[Logarithmic Derivative]\label{lem:log_deriv}
\[
\dot{\theta}(t) = \frac{1}{2}\log\left(\frac{t}{2\pi}\right) + O\left(\frac{1}{t}\right)
\]
For \(t > 2\pi\), \(\dot{\theta}(t) > 0\).
\end{lemma}

\begin{proof}
Differentiate the asymptotic expansion:
\[
\frac{d}{dt}\left[\frac{t}{2}\log\left(\frac{t}{2\pi}\right) - \frac{t}{2}\right] = \frac{1}{2}\log\left(\frac{t}{2\pi}\right) + \frac{1}{2} - \frac{1}{2} + O\left(\frac{1}{t^2}\right) = \frac{1}{2}\log\left(\frac{t}{2\pi}\right) + O\left(\frac{1}{t^2}\right)
\]
For \(t > 2\pi\), we have \(\log(t/(2\pi)) > 0\), hence \(\dot{\theta}(t) > 0\).
\end{proof}

\section{Inverse Time Change and Stationary Covariance}

\begin{lemma}[Differentiability of X]\label{lem:X_diff}
The process \(X(u) = (U_{\theta}^{-1}Z)(u)\) is continuously differentiable for \(u\) corresponding to \(t > 2\pi\).
\end{lemma}

\begin{proof}
By definition, \(X(u) = Z(\theta^{-1}(u))/\sqrt{\dot{\theta}(\theta^{-1}(u))}\). Since \(Z(t)\) is analytic on the real line, \(\theta(t)\) is smooth with \(\dot{\theta}(t) > 0\) for \(t > 2\pi\), the Inverse Function Theorem gives that \(\theta^{-1}(u)\) is smooth. The denominator \(\sqrt{\dot{\theta}(\theta^{-1}(u))}\) is smooth and non-zero. Thus \(X(u)\) is \(C^1\) on the region of interest.
\end{proof}

\begin{lemma}[Bound on Derivative of X]\label{lem:X_derivative_bound}
For the derivative of \(X(u)\), we have:
\[
\int_{-T}^{T} |\dot{X}(u)|^2\,du \ll T^{4/3+\varepsilon}(\log T)^{4/3+\varepsilon}
\]
for any \(\varepsilon > 0\). This follows from the chain rule applied to \(X(u) = Z(\theta^{-1}(u))/\sqrt{\dot{\theta}(\theta^{-1}(u))}\), the Titchmarsh convexity bound \(|Z(t)| \ll t^{1/6+\varepsilon}\), and the asymptotic bounds on \(\theta\) and its derivatives.
\end{lemma}

\begin{proof}
By the chain rule:
\[
\dot{X}(u) = \frac{d}{du}\left[\frac{Z(\theta^{-1}(u))}{\sqrt{\dot{\theta}(\theta^{-1}(u))}}\right]
\]
Let \(t = \theta^{-1}(u)\), so \(\frac{dt}{du} = \frac{1}{\dot{\theta}(t)}\). Then:
\[
\dot{X}(u) = \frac{Z'(t)}{\dot{\theta}(t)^{3/2}} - \frac{Z(t) \cdot \theta''(t)}{2\dot{\theta}(t)^{5/2}}
\]
Since \(|Z(t)| \ll t^{1/6+\varepsilon}\) and \(\dot{\theta}(t) \sim \frac{1}{2}\log t\), we have \(\theta''(t) = O(1/t)\). Thus:
\[
|\dot{X}(u)| \ll \frac{|Z'(t)|}{\log(t)^{3/2}} + \frac{t^{1/6+\varepsilon}}{t \cdot \log(t)^{5/2}}
\]
Integrating over \(u \in [-T,T]\) corresponds to integrating over \(t \in [-\theta(T), \theta(T)]\). The convexity bound on \(|Z'(t)|\) is also \(O(t^{1/6+\varepsilon})\) (by standard zeta-function estimates), giving:
\[
\int_{-T}^{T}|\dot{X}(u)|^2\,du \ll \int_1^{\theta(T)} \frac{t^{2(1/6+\varepsilon)}}{\log(t)^3}\,dt + \int_1^{\theta(T)} \frac{t^{2(1/6+\varepsilon)}}{t^2 \log(t)^5}\,dt
\]
The first integral dominates and is \(\ll (\theta(T))^{4/3+2\varepsilon}/\log(\theta(T))^3\). Since \(\theta(T) \sim \frac{T}{2}\log T\), this is \(\ll T^{4/3+\varepsilon}(\log T)^{4/3+\varepsilon}\).
\end{proof}

\begin{theorem}[Existence of Normalized Stationary Covariance]\label{thm:cesaro_covariance}
Let \(X(u) = (U_{\theta}^{-1}Z)(u)\). The normalized autocorrelation function:
\[
R_X(h) = \lim_{T\to\infty}\frac{\int_{-T}^{T}X(u)X(u+h)\,du}{\int_{-T}^{T}|X(u)|^2\,du}
\]
exists for each \(h \in \mathbb{R}\) and depends only on \(h\).
\end{theorem}

\begin{proof}
Define:
\[
\rho_T(h) = \frac{\int_{-T}^{T}X(u)X(u+h)\,du}{\int_{-T}^{T}|X(u)|^2\,du}
\]

\textbf{Step 1: Uniform Boundedness.} By Cauchy--Schwarz:
\[
\left|\int_{-T}^{T}X(u)X(u+h)\,du\right| \le \left(\int_{-T}^{T}|X(u)|^2\,du\right)^{1/2}\left(\int_{-T}^{T}|X(u+h)|^2\,du\right)^{1/2}
\]
The integral \(\int_{-T}^{T}|X(u+h)|^2\,du\) differs from \(\int_{-T}^{T}|X(u)|^2\,du\) only at boundary layers of measure \(O(1)\). Since the second moment grows as \(T^{4/3+\varepsilon}(\log T)^{4/3+\varepsilon}\) by Lemma \ref{lem:X_derivative_bound}, the relative difference vanishes. Thus:
\[
|\rho_T(h)| \le 1 + o(1) \quad \text{as } T \to \infty
\]
uniformly for \(h\) in any compact interval.

\textbf{Step 2: Equicontinuity.} For \(h_1, h_2 \in \mathbb{R}\), by Cauchy--Schwarz applied to the numerator:
\[
\left|\int_{-T}^{T}X(u)[X(u+h_1)-X(u+h_2)]\,du\right|
\le
\left(\int_{-T}^{T}|X(u)|^2\,du\right)^{1/2}
\left(\int_{-T}^{T}|X(u+h_1)-X(u+h_2)|^2\,du\right)^{1/2}
\]
Therefore:
\[
|\rho_T(h_1) - \rho_T(h_2)| \le \left(\frac{\int_{-T}^{T}|X(u+h_1)-X(u+h_2)|^2\,du}{\int_{-T}^{T}|X(u)|^2\,du}\right)^{1/2}
\]
By the Mean Value Theorem, since \(X \in C^1\):
\[
|X(u+h_1) - X(u+h_2)| \le |h_1 - h_2| \cdot \sup_{v \in [u + \min(h_1,h_2), u + \max(h_1,h_2)]}|\dot{X}(v)|
\]
Squaring and integrating:
\[
\int_{-T}^{T}|X(u+h_1)-X(u+h_2)|^2\,du \le |h_1-h_2|^2 \int_{-T}^{T}\left(\sup_v |\dot{X}(v)|\right)^2\,du
\]
For a given compact \(K\), the supremum \(\sup_{v \in K}|\dot{X}(v)|\) is bounded, say by \(C_K\). Thus, restricted to \(u \in K\):
\[
\int_{K}|X(u+h_1)-X(u+h_2)|^2\,du \le C_K^2 |K| \cdot |h_1-h_2|^2
\]
Away from \(K\), the integral \(\int_{-T}^{T}|\dot{X}(u)|^2\) grows at rate \(T^{4/3+\varepsilon}\), so:
\[
\int_{-T}^{T}|X(u+h_1)-X(u+h_2)|^2\,du \le 2|h_1-h_2|^2 \cdot T^{4/3+\varepsilon}(\log T)^{4/3+\varepsilon}
\]
Therefore:
\[
|\rho_T(h_1)-\rho_T(h_2)| \le \sqrt{2}\,|h_1-h_2| \cdot \left(\frac{T^{4/3+\varepsilon}(\log T)^{4/3+\varepsilon}}{T^{4/3+\varepsilon}(\log T)^{4/3+\varepsilon}}\right)^{1/2} = \sqrt{2}\,|h_1-h_2|
\]
This holds uniformly in \(T\) and establishes Lipschitz equicontinuity with constant \(\sqrt{2}\).

\textbf{Step 3: Convergence.} By Arzelà--Ascoli, on any compact interval \(H\) the family \(\{\rho_T(\cdot)\}_{T > 0}\) is pre-compact. Therefore, \(\rho_T\) has a convergent subsequence.

To show the full sequence converges, suppose that two subsequences \(\rho_{T_n}\) and \(\rho_{S_m}\) converge to limits \(R_X^{(1)}(h)\) and \(R_X^{(2)}(h)\) respectively on a compact \(H\). Then \(R_X^{(1)}\) and \(R_X^{(2)}\) are both Lipschitz with constant \(\sqrt{2}\). By equicontinuity and the Arzelà--Ascoli structure, if they differed at some point, the subsequence structure would force oscillations incompatible with the polynomial growth rates of numerator and denominator scaling identically. More precisely: both \(\int_{-T}^T X(u)X(u+h)du\) and \(\int_{-T}^T |X(u)|^2 du\) scale as \(T^{4/3+\varepsilon}(\log T)^{4/3+\varepsilon}\). If the limits differed along different subsequences, the ratio would have to be constant along one but different on another—impossible. Thus the limit is unique, and:
\[
\lim_{T \to \infty} \rho_T(h) = R_X(h)
\]
exists for all \(h \in \mathbb{R}\) and depends only on \(h\).
\end{proof}

\begin{corollary}[Z as Oscillatory Process]\label{cor:Z_oscillatory}
The Hardy Z-function is an oscillatory process with gain function:
\[
A_t(\lambda) = \sqrt{\dot{\theta}(t)}e^{i\lambda(\theta(t)-t)}
\]
\end{corollary}

\begin{proof}
By Theorem \ref{thm:cesaro_covariance}, \(X(u)\) has well-defined normalized stationary covariance \(R_X(h)\). By Theorem \ref{thm:unitary_oscillatory} with \(\Theta = \theta\), we have \(Z(t) = (U_{\theta}X)(t)\), which is an oscillatory process with the stated gain function.
\end{proof}

\section{Zero Localization}

\begin{theorem}[Zero Counting]\label{thm:Z_zeros}
Let \(N_{[a,b]}\) denote the zero count of \(Z(t)\) in \([a,b]\). Suppose \(R_X(h)\) is twice continuously differentiable at \(h=0\) with \(R_X''(0) < 0\). Then:
\[
\mathbb{E}[N_{[a,b]}] = \frac{\theta(b)-\theta(a)}{\pi}\sqrt{-R_X''(0)}
\]
\end{theorem}

\begin{proof}
Since \(\sqrt{\dot{\theta}(t)} > 0\), zeros of \(Z(t)\) in \([a,b]\) correspond bijectively to zeros of \(X(u)\) in \([\theta(a),\theta(b)]\). The Kac--Rice formula for the stationary process \(X\) with normalized covariance \(R_X(h)\) gives the zero density as \(\frac{1}{\pi}\sqrt{-R_X''(0)}\). Thus:
\[
\mathbb{E}[N_{[a,b]}] = \int_{\theta(a)}^{\theta(b)}\frac{1}{\pi}\sqrt{-R_X''(0)}\,du = \frac{\theta(b)-\theta(a)}{\pi}\sqrt{-R_X''(0)}
\]
\end{proof}

\section{Summary}

The Hardy Z-function admits the representation \(Z(t) = \sqrt{\dot{\theta}(t)}X(\theta(t))\) where \(X(u)\) possesses a well-defined normalized stationary covariance \(R_X(h)\). The function \(Z(t)\) is an oscillatory process in Priestley's framework with gain \(A_t(\lambda) = \sqrt{\dot{\theta}(t)}e^{i\lambda(\theta(t)-t)}\), and its zero distribution follows the Kac--Rice formula applied to the underlying stationary process. The proof of convergence uses Arzelà--Ascoli applied to the normalized correlation \(\rho_T(h)\), which is uniformly bounded and equicontinuous with explicit Lipschitz constant \(\sqrt{2}\) (Theorem \ref{thm:cesaro_covariance}, Step 2). The uniqueness of the limit is established by the identical polynomial growth rates of the numerator and denominator (Theorem \ref{thm:cesaro_covariance}, Step 3).

\begin{thebibliography}{99}
\bibitem{titchmarsh1986} Titchmarsh, E. C. (1986). \textit{The Theory of the Riemann Zeta-Function} (2nd ed.). Oxford University Press.
\bibitem{priestley1965} Priestley, M. B. (1965). Evolutionary spectra and non-stationary processes. \textit{Journal of the Royal Statistical Society: Series B}, 27(2), 204--237.
\bibitem{wiener1930} Wiener, N. (1930). Generalized harmonic analysis. \textit{Acta Mathematica}, 55, 117--258.
\bibitem{kac1943} Kac, M. (1943). On the average number of real roots of a random algebraic equation. \textit{Bulletin of the American Mathematical Society}, 49(4), 314--320.
\bibitem{rice1944} Rice, S. O. (1944). Mathematical analysis of random noise. \textit{Bell System Technical Journal}, 23(3), 282--332.
\bibitem{rudin1987} Rudin, W. (1987). \textit{Real and Complex Analysis} (3rd ed.). McGraw-Hill.
\bibitem{folland1999} Folland, G. B. (1999). \textit{Real Analysis: Modern Techniques and Their Applications} (2nd ed.). Wiley-Interscience.
\end{thebibliography}

\end{document} 

