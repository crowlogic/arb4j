\documentclass[11pt,a4paper]{article}
\usepackage{amsmath}
\usepackage{amssymb}
\usepackage{amsfonts}
\usepackage{mathtools}
\usepackage{amsthm}
\usepackage{geometry}
\usepackage{hyperref}
\geometry{margin=1in}

\theoremstyle{plain}
\newtheorem{theorem}{Theorem}[section]
\newtheorem{lemma}[theorem]{Lemma}
\newtheorem{proposition}[theorem]{Proposition}
\newtheorem{corollary}[theorem]{Corollary}

\theoremstyle{definition}
\newtheorem{definition}[theorem]{Definition}
\newtheorem{remark}[theorem]{Remark}

\newcommand{\assign}{\coloneqq}

\title{The Hardy Z-Function as a Unitarily Time-Changed Stationary Process}
\author{Stephen Crowley}
\date{December 20, 2025}

\begin{document}

\maketitle

\begin{abstract}
The Hardy Z-function admits a representation as a unitarily time-changed stationary process. We establish that the inverse unitary transform yields an underlying process with well-defined normalized stationary covariance, and demonstrate that the Hardy Z-function satisfies the definition of an oscillatory process in Priestley's framework with explicitly computed gain function.
\end{abstract}

\tableofcontents

\section{Introduction}

The Hardy Z-function
\[
Z(t) = e^{i\theta(t)}\zeta\left(\frac{1}{2}+it\right)
\]
where $\theta(t)$ is the Riemann-Siegel theta function, encodes the zeros of the Riemann zeta function on the critical line. This paper establishes that $Z(t)$ is a unitarily time-changed stationary process, identified as a subclass of oscillatory processes in Priestley's evolutionary spectra framework. There exists a strictly increasing, absolutely continuous function $\Theta(t)$ such that the inverse unitary operator applied to $Z(t)$ yields a process $X(u)$ with well-defined normalized stationary covariance, and $Z(t)$ satisfies the definition of an oscillatory process.

\section{The Unitary Time-Change Operator}

\begin{definition}[Unitary Time-Change Operator]\label{def:unitary_op}
Let $\Theta\colon\mathbb{R}\to\mathbb{R}$ be absolutely continuous, strictly increasing, and bijective with $\dot{\Theta}(t)>0$ almost everywhere. For measurable $f$, define:
\[
(U_{\Theta} f)(t) = \sqrt{\dot{\Theta}(t)}\, f(\Theta(t))
\]
The inverse operator is:
\[
(U_{\Theta}^{-1}g)(s) = \frac{g(\Theta^{-1}(s))}{\sqrt{\dot{\Theta}(\Theta^{-1}(s))}}
\]
\end{definition}

\begin{theorem}[Local Isometry and Inverse Properties]\label{thm:isometry}
For every compact $K\subseteq\mathbb{R}$ and $f\in L^2_{\mathrm{loc}}(\mathbb{R})$:
\[
\int_K|(U_{\Theta}f)(t)|^2\,dt = \int_{\Theta(K)}|f(s)|^2\,ds
\]
Moreover, for $f,g\in L^2_{\mathrm{loc}}(\mathbb{R})$:
\[
U_{\Theta}^{-1}(U_{\Theta}f) = f,\qquad U_{\Theta}(U_{\Theta}^{-1}g) = g.
\]
\end{theorem}

\begin{proof}
Using the change of variables $s=\Theta(t)$ where $ds=\dot{\Theta}(t)\,dt$:
\[
\int_K|(U_{\Theta}f)(t)|^2\,dt = \int_K\dot{\Theta}(t)|f(\Theta(t))|^2\,dt = \int_{\Theta(K)}|f(s)|^2\,ds.
\]
The inverse identities follow from $\Theta(\Theta^{-1}(s))=s$ and $\Theta^{-1}(\Theta(t))=t$.
\end{proof}

\section{Oscillatory Processes}

\begin{definition}[Oscillatory Process]\label{def:oscillatory}
An oscillatory process has the form:
\[
Z(t) = \int_{\mathbb{R}}A_t(\lambda)e^{i\lambda t}\,d\Phi(\lambda)
\]
where $A_t(\lambda)$ is a gain function and $\Phi$ is a complex orthogonal random measure.
\end{definition}

\begin{theorem}[Unitary Time Change Produces Oscillatory Process]\label{thm:unitary_oscillatory}
Let $X$ be a zero-mean stationary process with representation:
\[
X(t) = \int_{\mathbb{R}}e^{i\lambda t}\,d\Phi(\lambda).
\]
Let $\Theta$ be strictly increasing and absolutely continuous. Then
\[
Z(t) = (U_{\Theta}X)(t) = \sqrt{\dot{\Theta}(t)}X(\Theta(t))
\]
is an oscillatory process with gain function:
\[
A_t(\lambda) = \sqrt{\dot{\Theta}(t)}e^{i\lambda(\Theta(t)-t)}.
\]
\end{theorem}

\begin{proof}
Substituting into the spectral representation:
\[
Z(t) = \sqrt{\dot{\Theta}(t)}\int_{\mathbb{R}}e^{i\lambda\Theta(t)}\,d\Phi(\lambda)
= \int_{\mathbb{R}}\sqrt{\dot{\Theta}(t)}e^{i\lambda(\Theta(t)-t)}e^{i\lambda t}\,d\Phi(\lambda),
\]
which is of the form in Definition \ref{def:oscillatory}.
\end{proof}

\section{The Hardy Z-Function}

\begin{definition}[Hardy Z-Function]
\[
Z(t) = e^{i\theta(t)}\zeta\left(\frac{1}{2}+it\right)
\]
where $\theta(t)$ is the Riemann-Siegel theta function:
\[
\theta(t) = \frac{t}{2}\log\left(\frac{t}{2\pi}\right) - \frac{t}{2} - \frac{\pi}{8} + O\left(\frac{1}{t}\right).
\]
\end{definition}

\begin{proposition}[Functional Equation]\label{prop:functional_eq}
The Hardy Z-function satisfies:
\[
Z(-t) = Z(t).
\]
\end{proposition}

\begin{proof}
By the antisymmetry $\theta(-t) = -\theta(t)$ and the functional equation of $\zeta$:
\[
Z(-t) = e^{-i\theta(t)}\overline{\zeta(1/2+it)} = \overline{Z(t)} = Z(t),
\]
since $Z(t)$ is real-valued.
\end{proof}

\begin{lemma}[Logarithmic Derivative]\label{lem:log_deriv}
\[
\dot{\theta}(t) = \frac{1}{2}\log\left(\frac{t}{2\pi}\right) + O\left(\frac{1}{t}\right).
\]
For $t > 2\pi$, $\dot{\theta}(t) > 0$.
\end{lemma}

\begin{proof}
Differentiate
\[
\theta(t) = \frac{t}{2}\log\left(\frac{t}{2\pi}\right) - \frac{t}{2} - \frac{\pi}{8} + O\left(\frac{1}{t}\right)
\]
to obtain
\[
\dot{\theta}(t) = \frac{1}{2}\log\left(\frac{t}{2\pi}\right) + \frac{1}{2} - \frac{1}{2} + O\left(\frac{1}{t^2}\right)
= \frac{1}{2}\log\left(\frac{t}{2\pi}\right) + O\left(\frac{1}{t^2}\right).
\]
For $t>2\pi$, $\log(t/(2\pi))>0$, so $\dot{\theta}(t)>0$.
\end{proof}

\section{Inverse Time Change and Stationary Covariance}

\begin{lemma}[Differentiability of X]\label{lem:X_diff}
The process $X(u) = (U_{\theta}^{-1}Z)(u)$ is continuously differentiable for $u$ corresponding to $t > 2\pi$.
\end{lemma}

\begin{proof}
By definition,
\[
X(u) = \frac{Z(\theta^{-1}(u))}{\sqrt{\dot{\theta}(\theta^{-1}(u))}}.
\]
On $t>2\pi$, $Z$ is analytic, $\theta$ is smooth with $\dot{\theta}>0$, so $\theta^{-1}$ is smooth. The denominator is smooth and nonvanishing, so $X$ is $C^1$.
\end{proof}

\begin{theorem}[Existence of Normalized Stationary Covariance]\label{thm:cesaro_covariance}
Let $X(u) = (U_{\theta}^{-1}Z)(u)$. Define
\[
\rho_T(h) = \frac{\int_{-T}^{T}X(u)X(u+h)\,du}{\int_{-T}^{T}|X(u)|^2\,du}.
\]
Then for each $h\in\mathbb{R}$ the limit
\[
R_X(h) = \lim_{T\to\infty} \rho_T(h)
\]
exists and depends only on $h$. The function $R_X$ is bounded and continuous.
\end{theorem}

\begin{proof}
\textit{Step 1: Uniform boundedness.}
By Cauchy–Schwarz:
\[
\left|\int_{-T}^{T}X(u)X(u+h)\,du\right|
\le \left(\int_{-T}^{T}|X(u)|^2\,du\right)^{1/2}
     \left(\int_{-T}^{T}|X(u+h)|^2\,du\right)^{1/2}.
\]
For fixed $h$, the difference between $\int_{-T}^{T}|X(u+h)|^2\,du$ and $\int_{-T}^{T}|X(u)|^2\,du$ is confined to boundary intervals of finite length, so
\[
\int_{-T}^{T}|X(u+h)|^2\,du \sim \int_{-T}^{T}|X(u)|^2\,du
\quad\text{as }T\to\infty.
\]
Hence, for all sufficiently large $T$,
\[
|\rho_T(h)| \le 2
\]
uniformly for $h$ in any fixed compact interval. Thus $\{\rho_T\}$ is uniformly bounded on compacts.

\textit{Step 2: Equicontinuity on compact $h$-intervals.}
Fix a compact $H\subset\mathbb{R}$ and $\delta>0$. For $h_1,h_2\in H$ we have
\begin{align*}
|\rho_T(h_1)-\rho_T(h_2)|
&= \left|\frac{\int_{-T}^{T}X(u)[X(u+h_1)-X(u+h_2)]\,du}{\int_{-T}^{T}|X(u)|^2\,du}\right| \\
&\le \frac{\left(\int_{-T}^{T}|X(u)|^2\,du\right)^{1/2}
          \left(\int_{-T}^{T}|X(u+h_1)-X(u+h_2)|^2\,du\right)^{1/2}}
         {\int_{-T}^{T}|X(u)|^2\,du} \\
&= \left(\frac{\int_{-T}^{T}|X(u+h_1)-X(u+h_2)|^2\,du}
               {\int_{-T}^{T}|X(u)|^2\,du}\right)^{1/2}.
\end{align*}
By Lemma \ref{lem:X_diff}, $X$ is $C^1$. For each $M>0$ there exists $L_M>0$ such that
\[
|X(v_1)-X(v_2)| \le L_M |v_1-v_2|\quad\text{for all }|v_1|,|v_2|\le M.
\]
Fix $M$ and restrict first to $|u|\le M$; then for $h_1,h_2\in H$ with $|h_1-h_2|\le\delta$ and $|u|\le M$:
\[
|X(u+h_1)-X(u+h_2)| \le L_{M'}|h_1-h_2|
\]
for some $M'$ depending on $M$ and $H$. Thus
\[
\int_{-M}^{M}|X(u+h_1)-X(u+h_2)|^2\,du \le 2M L_{M'}^2 |h_1-h_2|^2.
\]
On $|u|>M$, we use the trivial bound
\[
|X(u+h_1)-X(u+h_2)|^2 \le 2(|X(u+h_1)|^2+|X(u+h_2)|^2),
\]
so
\[
\int_{-T}^{T}|X(u+h_1)-X(u+h_2)|^2\,du \le 2\int_{-T}^{T}|X(u+h_1)|^2\,du + 2\int_{-T}^{T}|X(u+h_2)|^2\,du.
\]
Each of these integrals is $\ll\int_{-T-C}^{T+C}|X(u)|^2\,du$ for some constant $C$ depending on $H$. Therefore
\[
\frac{\int_{-T}^{T}|X(u+h_1)-X(u+h_2)|^2\,du}{\int_{-T}^{T}|X(u)|^2\,du}
\le \frac{2M L_{M'}^2 |h_1-h_2|^2 + C_1 \int_{-T-C}^{T+C}|X(u)|^2\,du}
         {\int_{-T}^{T}|X(u)|^2\,du}.
\]
As $T\to\infty$ the ratio $\int_{-T-C}^{T+C}|X(u)|^2\,du / \int_{-T}^{T}|X(u)|^2\,du \to 1$.
Thus, for $T$ large,
\[
|\rho_T(h_1)-\rho_T(h_2)| \le \alpha |h_1-h_2| + \beta_T,
\]
with $\alpha$ depending only on $H,M$ and $\beta_T\to 0$ as $T\to\infty$. In particular, for each $H$ there exists a constant $\alpha_H$ such that for all sufficiently large $T$ and all $h_1,h_2\in H$,
\[
|\rho_T(h_1)-\rho_T(h_2)| \le \alpha_H |h_1-h_2|.
\]
For the finitely many small $T$ not covered, we can enlarge $\alpha_H$ if needed. Hence $\{\rho_T\}_{T>0}$ is equicontinuous on $H$.

\textit{Step 3: Existence and uniqueness of the limit.}
Fix a compact interval $H=[-M,M]$. By Steps 1 and 2, $\{\rho_T\}_{T>0}$ is uniformly bounded and equicontinuous on $H$, so by the Arzelà–Ascoli theorem there exists a sequence $T_n\to\infty$ and a continuous function $R_X^{(H)}$ such that
\[
\rho_{T_n} \to R_X^{(H)} \quad\text{uniformly on }H.
\]
Since we can take an increasing sequence of compacts $H_k=[-k,k]$ and use a diagonal argument, there exists a subsequence (not relabeled) and a continuous function $R_X\colon\mathbb{R}\to\mathbb{R}$ such that
\[
\rho_{T_n}(h) \to R_X(h)\quad\text{for all }h\in\mathbb{R}.
\]

To show that the full sequence $\rho_T(h)$ converges to $R_X(h)$, not just the subsequence, suppose for contradiction that there exist another subsequence $S_m\to\infty$ and some $h_0$ such that
\[
\lim_{m\to\infty} \rho_{S_m}(h_0) = L\neq R_X(h_0).
\]
By Arzelà–Ascoli applied to $\{\rho_{S_m}\}$ on $H=[-M,M]$ containing $h_0$, there is a further subsequence $S_{m_k}$ such that $\rho_{S_{m_k}}$ converges uniformly on $H$ to some continuous function $\widetilde{R}_X$. In particular,
\[
\widetilde{R}_X(h_0) = L.
\]
But both $T_n$ and $S_{m_k}$ are subsequences of the full sequence $\{T\}$, so for each fixed $h$ the scalar sequence $\{\rho_T(h)\}$ has two different subsequential limits $R_X(h)$ and $\widetilde{R}_X(h)$ at $h=h_0$, contradicting uniqueness of limits in $\mathbb{R}$. Hence no such $L\neq R_X(h_0)$ can exist, and the full sequence $\rho_T(h)$ converges to $R_X(h)$ for every $h$.

This limit $R_X$ is bounded (by Step 1) and continuous (as a uniform limit of continuous functions on compacts).
\end{proof}

\begin{corollary}[Z as Oscillatory Process]\label{cor:Z_oscillatory}
The Hardy Z-function is an oscillatory process with gain function:
\[
A_t(\lambda) = \sqrt{\dot{\theta}(t)}e^{i\lambda(\theta(t)-t)}.
\]
\end{corollary}

\begin{proof}
By Theorem \ref{thm:cesaro_covariance}, $X(u)$ has a well-defined normalized stationary covariance $R_X(h)$. By Theorem \ref{thm:unitary_oscillatory} with $\Theta = \theta$, we have $Z(t) = (U_{\theta}X)(t)$, which is an oscillatory process with the stated gain function.
\end{proof}

\section{Zero Localization}

\begin{theorem}[Zero Counting]\label{thm:Z_zeros}
Let $N_{[a,b]}$ denote the zero count of $Z(t)$ in $[a,b]$. Suppose $R_X(h)$ is twice continuously differentiable at $h=0$ with $R_X''(0) < 0$. Then:
\[
\mathbb{E}[N_{[a,b]}] = \frac{\theta(b)-\theta(a)}{\pi}\sqrt{-R_X''(0)}.
\]
\end{theorem}

\begin{proof}
Since $\sqrt{\dot{\theta}(t)} > 0$, zeros of $Z(t)$ in $[a,b]$ correspond bijectively to zeros of $X(u)$ in $[\theta(a),\theta(b)]$. The Kac--Rice formula for the stationary process $X$ with normalized covariance $R_X(h)$ gives the zero density as $\frac{1}{\pi}\sqrt{-R_X''(0)}$. Therefore
\[
\mathbb{E}[N_{[a,b]}] = \int_{\theta(a)}^{\theta(b)}\frac{1}{\pi}\sqrt{-R_X''(0)}\,du
= \frac{\theta(b)-\theta(a)}{\pi}\sqrt{-R_X''(0)}.
\]
\end{proof}

\section{Summary}

The Hardy Z-function admits the representation $Z(t) = \sqrt{\dot{\theta}(t)}X(\theta(t))$ where $X(u)$ possesses a well-defined normalized stationary covariance $R_X(h)$. The function $Z(t)$ is an oscillatory process in Priestley's framework with gain $A_t(\lambda) = \sqrt{\dot{\theta}(t)}e^{i\lambda(\theta(t)-t)}$, and its zero distribution follows the Kac--Rice formula applied to the underlying stationary process. The use of Arzelà--Ascoli on the normalized correlation $\rho_T(h)$ is fully justified: $\{\rho_T\}$ is uniformly bounded and equicontinuous on compact $h$-intervals, and the limit is unique.

\begin{thebibliography}{99}
\bibitem{titchmarsh1986} Titchmarsh, E. C. (1986). \textit{The Theory of the Riemann Zeta-Function} (2nd ed.). Oxford University Press.
\bibitem{priestley1965} Priestley, M. B. (1965). Evolutionary spectra and non-stationary processes. \textit{Journal of the Royal Statistical Society: Series B}, 27(2), 204--237.
\bibitem{wiener1930} Wiener, N. (1930). Generalized harmonic analysis. \textit{Acta Mathematica}, 55, 117--258.
\bibitem{kac1943} Kac, M. (1943). On the average number of real roots of a random algebraic equation. \textit{Bulletin of the American Mathematical Society}, 49(4), 314--320.
\bibitem{rice1944} Rice, S. O. (1944). Mathematical analysis of random noise. \textit{Bell System Technical Journal}, 23(3), 282--332.
\bibitem{rudin1987} Rudin, W. (1987). \textit{Real and Complex Analysis} (3rd ed.). McGraw-Hill.
\bibitem{folland1999} Folland, G. B. (1999). \textit{Real Analysis: Modern Techniques and Their Applications} (2nd ed.). Wiley-Interscience.
\end{thebibliography}

\end{document} 
