\documentclass{article}
\usepackage{amsmath, amsthm, amssymb, mathtools}
\usepackage{xcolor}
\title{Unitary Time Changes of Stationary Processes Yield Oscillatory Processes \\ and a Functional Framework Toward a Hilbert--P\'olya Construction}
\author{Stephen Crowley}
\date{September 16, 2025}
\newtheorem{theorem}{Theorem}[section]
\newtheorem{lemma}[theorem]{Lemma}
\newtheorem{corollary}[theorem]{Corollary}
\newtheorem{proposition}[theorem]{Proposition}
\newtheorem{definition}[theorem]{Definition}
\newtheorem{remark}[theorem]{Remark}

\begin{document}
\maketitle

\begin{abstract}
A unitary time-change operator $U_\theta$ is constructed for absolutely continuous, strictly increasing time reparametrizations $\theta$, acting on functions that are square-integrable over $\sigma$-compact sets. Applying $U_\theta$ to the Cram\'er spectral representation of a stationary process yields an oscillatory process in the sense of Priestley with oscillatory function $\varphi_t(\lambda)=\sqrt{\theta'(t)}\,e^{i\lambda\theta(t)}$ and evolutionary spectrum $dF_t(\lambda)=\theta'(t)\,dF(\lambda)$. It is proved that sample paths of any non-degenerate second-order stationary process almost surely lie in $L^2_{\sigma\text{-comp}}(\mathbb{R})$, making the operator applicable to typical realizations. A zero-localization measure $\mu(dt)=\delta(Z(t))\,|Z'(t)|\,dt$ induces a Hilbert space $L^2(\mu)$ on the zero set of an oscillatory process $Z$, and the multiplication operator $(Lf)(t)=t f(t)$ has pure point, simple spectrum equal to the zero set of $Z$. This produces a concrete operator scaffold consistent with a Hilbert--P\'olya-type viewpoint.
\end{abstract}

\tableofcontents

\section{Function Spaces and Unitary Time Change}\label{sec:functionspaces}

\subsection{$\sigma$-compact sets and local $L^2$}
\begin{definition}[$\sigma$-compact sets]\label{def:sigma_compact}
A subset $U\subseteq\mathbb{R}$ is $\sigma$-compact if $U=\bigcup_{n=1}^\infty K_n$ with each $K_n$ compact.
\end{definition}

\begin{definition}[Square-integrability on $\sigma$-compact sets]\label{def:L2sigma}
Define
\[
L^2_{\sigma\text{-comp}}(\mathbb{R})\coloneqq\left\{f:\mathbb{R}\to\mathbb{C}:\ \int_U |f(t)|^2\,dt<\infty\ \text{for every $\sigma$-compact }U\subseteq\mathbb{R}\right\}.
\]
\end{definition}

\begin{remark}
Every bounded measurable set in $\mathbb{R}$ is $\sigma$-compact; hence $L^2_{\sigma\text{-comp}}(\mathbb{R})$ contains functions that are square-integrable on every bounded interval, including functions with polynomial growth at infinity.
\end{remark}

\subsection{Unitary time-change operator}
\begin{definition}[Unitary time-change]\label{def:Utheta}
Let $\theta:\mathbb{R}\to\mathbb{R}$ be absolutely continuous, strictly increasing, and bijective, with $\theta'(t)>0$ almost everywhere and $\theta'(t)=0$ only on sets of Lebesgue measure zero. The function $\theta$ maps $\sigma$-compact sets to $\sigma$-compact sets. Define, for $f$ measurable,
\[
(U_\theta f)(t)\coloneqq \sqrt{\theta'(t)}\,f(\theta(t)).
\]
\end{definition}

\begin{proposition}[Inverse map]\label{prop:inverse}
The inverse map is given by
\[
(U_\theta^{-1}g)(s)=\frac{g(\theta^{-1}(s))}{\sqrt{\theta'(\theta^{-1}(s))}},
\]
which is well-defined almost everywhere on every $\sigma$-compact set.
\end{proposition}

\begin{proof}
Since $\theta'(t)=0$ only on sets of measure zero, and $\theta^{-1}$ maps sets of measure zero to sets of measure zero (as absolutely continuous bijective functions preserve measure-zero sets), the denominator $\sqrt{\theta'(\theta^{-1}(s))}$ is positive almost everywhere. The expression is therefore well-defined almost everywhere on every $\sigma$-compact set, which suffices for defining an element of $L^2_{\sigma\text{-comp}}(\mathbb{R})$.
\end{proof}

\begin{theorem}[Local unitarity on $\sigma$-compact sets]\label{thm:local_unitarity}
For every $\sigma$-compact set $U\subseteq\mathbb{R}$ and $f\in L^2_{\sigma\text{-comp}}(\mathbb{R})$,
\[
\int_U |(U_\theta f)(t)|^2\,dt=\int_{\theta(U)} |f(s)|^2\,ds.
\]
Moreover, $U_\theta^{-1}$ is the inverse of $U_\theta$ on $L^2_{\sigma\text{-comp}}(\mathbb{R})$.
\end{theorem}

\begin{proof}
Let $f\in L^2_{\sigma\text{-comp}}(\mathbb{R})$ and let $U$ be any $\sigma$-compact set. The local $L^2$-norm of $U_\theta f$ over $U$ is:
\begin{align}
\int_U |(U_\theta f)(t)|^2\,dt &= \int_U \left|\sqrt{\theta'(t)}f(\theta(t))\right|^2\,dt\\
&= \int_U \theta'(t)|f(\theta(t))|^2\,dt
\end{align}

Since $\theta$ is absolutely continuous and strictly increasing, applying the change of variables $s=\theta(t)$ gives $ds=\theta'(t)\,dt$ almost everywhere. Since $\theta$ maps $\sigma$-compact sets to $\sigma$-compact sets, as $t$ ranges over $U$, $s=\theta(t)$ ranges over $\theta(U)$, which is $\sigma$-compact. Therefore:
\[
\int_U \theta'(t)|f(\theta(t))|^2\,dt = \int_{\theta(U)} |f(s)|^2\,ds
\]

To verify that $U_\theta^{-1}$ is indeed the inverse, we compute explicitly. For any $f\in L^2_{\sigma\text{-comp}}(\mathbb{R})$:
\begin{align}
(U_\theta^{-1} U_\theta f)(s) &= (U_\theta^{-1})[\sqrt{\theta'(\cdot)}f(\theta(\cdot))](s)\\
&= \frac{[\sqrt{\theta'(\theta^{-1}(s))}f(\theta(\theta^{-1}(s)))])}{\sqrt{\theta'(\theta^{-1}(s))}}\\
&= \frac{\sqrt{\theta'(\theta^{-1}(s))}f(s)}{\sqrt{\theta'(\theta^{-1}(s))}}\\
&= f(s)
\end{align}
where we used $\theta(\theta^{-1}(s))=s$.

Similarly, for any $g\in L^2_{\sigma\text{-comp}}(\mathbb{R})$:
\begin{align}
(U_\theta U_\theta^{-1}g)(t) &= \sqrt{\theta'(t)}(U_\theta^{-1}g)(\theta(t))\\
&= \sqrt{\theta'(t)}\frac{g(\theta^{-1}(\theta(t)))}{\sqrt{\theta'(\theta^{-1}(\theta(t)))}}\\
&= \sqrt{\theta'(t)}\frac{g(t)}{\sqrt{\theta'(t)}}\\
&= g(t)
\end{align}
where we used $\theta^{-1}(\theta(t))=t$.

Therefore $U_\theta U_\theta^{-1} = U_\theta^{-1} U_\theta = I$ on $L^2_{\sigma\text{-comp}}(\mathbb{R})$.
\end{proof}

\begin{theorem}[Unitarity on $L^2(\mathbb{R})$]\label{thm:global_unitarity}
$U_\theta:L^2(\mathbb{R})\to L^2(\mathbb{R})$ is unitary: $\int_{\mathbb{R}} |(U_\theta f)(t)|^2\,dt=\int_{\mathbb{R}} |f(s)|^2\,ds$ and $U_\theta^{-1}$ is its inverse.
\end{theorem}

\begin{proof}
For $f\in L^2(\mathbb{R})$, we have:
\begin{align}
\int_{\mathbb{R}} |(U_\theta f)(t)|^2\,dt &= \int_{\mathbb{R}} \theta'(t)|f(\theta(t))|^2\,dt
\end{align}

By the change of variables $s=\theta(t)$ with $ds=\theta'(t)\,dt$, and since $\theta:\mathbb{R}\to\mathbb{R}$ is bijective:
\[
\int_{\mathbb{R}} \theta'(t)|f(\theta(t))|^2\,dt = \int_{\mathbb{R}} |f(s)|^2\,ds
\]

The inverse relationship follows from the same computation as in Theorem~\ref{thm:local_unitarity}, applied globally.
\end{proof}

\section{Oscillatory Processes (Priestley)}\label{sec:oscillatory}

\begin{definition}[Oscillatory process]\label{def:osc_proc}
Let $F$ be a finite nonnegative Borel measure on $\mathbb{R}$. For each $t\in\mathbb{R}$, let $A_t\in L^2(F)$ and set $\varphi_t(\lambda)\coloneqq A_t(\lambda)e^{i\lambda t}$. An \emph{oscillatory process} is a stochastic process
\[
Z(t)\coloneqq \int_{\mathbb{R}} \varphi_t(\lambda)\,\Phi(d\lambda)=\int_{\mathbb{R}} A_t(\lambda)e^{i\lambda t}\,\Phi(d\lambda),
\]
where $\Phi$ is a complex orthogonal random measure with spectral measure $F$, that is,
\[
\mathbb{E}\!\left[\Phi(d\lambda)\,\overline{\Phi(d\mu)}\right]=\delta(\lambda-\mu)\,dF(\lambda).
\]
Its covariance kernel is
\[
R_Z(t,s)=\mathbb{E}\!\left[Z(t)\overline{Z(s)}\right]=\int_{\mathbb{R}} A_t(\lambda)\,\overline{A_s(\lambda)}\,e^{i\lambda(t-s)}\,dF(\lambda).
\]
\end{definition}

\begin{remark}[Real-valuedness]
$Z$ is real-valued if and only if $A_t(-\lambda)=\overline{A_t(\lambda)}$ for $F$-a.e.\ $\lambda$, equivalently $\varphi_t(-\lambda)=\overline{\varphi_t(\lambda)}$ for $F$-a.e.\ $\lambda$.
\end{remark}

\begin{theorem}[Existence]\label{thm:existence_osc}
If $F$ is finite and $(A_t)_{t\in\mathbb{R}}$ is measurable in $t$ with $\int_{\mathbb{R}}|A_t(\lambda)|^2\,dF(\lambda)<\infty$ for each $t$, then there exists a complex orthogonal random measure $\Phi$ with spectral measure $F$ such that $Z(t)=\int_{\mathbb{R}} A_t(\lambda)e^{i\lambda t}\,\Phi(d\lambda)$ is well-defined in $L^2(\Omega)$ and has covariance $R_Z$ as above.
\end{theorem}

\begin{proof}
We construct the stochastic integral using the standard extension procedure. First, define the integral for simple functions of the form $g(\lambda)=\sum_{j=1}^n c_j \mathbf{1}_{E_j}(\lambda)$ where $\{E_j\}$ are disjoint Borel sets with $F(E_j)<\infty$ and $c_j\in\mathbb{C}$:
\[
\int_{\mathbb{R}} g(\lambda)\,\Phi(d\lambda) \coloneqq \sum_{j=1}^n c_j \Phi(E_j)
\]

For such simple functions, the isometry property holds:
\begin{align}
\mathbb{E}\left[\left|\int_{\mathbb{R}} g(\lambda)\,\Phi(d\lambda)\right|^2\right] &= \mathbb{E}\left[\left|\sum_{j=1}^n c_j \Phi(E_j)\right|^2\right]\\
&= \sum_{j=1}^n \sum_{k=1}^n c_j\overline{c_k}\mathbb{E}[\Phi(E_j)\overline{\Phi(E_k)}]\\
&= \sum_{j=1}^n |c_j|^2 F(E_j)\\
&= \int_{\mathbb{R}} |g(\lambda)|^2\,dF(\lambda)
\end{align}

Since simple functions are dense in $L^2(F)$, we extend by continuity to all $g\in L^2(F)$. For each $t$, since $\varphi_t(\lambda)=A_t(\lambda)e^{i\lambda t}$ and $A_t\in L^2(F)$, we have $\varphi_t\in L^2(F)$. Therefore $Z(t)=\int_{\mathbb{R}} \varphi_t(\lambda)\,\Phi(d\lambda)$ is well-defined in $L^2(\Omega)$.

The covariance is computed as:
\begin{align}
R_Z(t,s) &= \mathbb{E}[Z(t)\overline{Z(s)}]\\
&= \mathbb{E}\left[\int_{\mathbb{R}} \varphi_t(\lambda)\,\Phi(d\lambda) \int_{\mathbb{R}} \overline{\varphi_s(\mu)}\,\overline{\Phi(d\mu)}\right]\\
&= \int_{\mathbb{R}} \int_{\mathbb{R}} \varphi_t(\lambda)\overline{\varphi_s(\mu)}\mathbb{E}[\Phi(d\lambda)\overline{\Phi(d\mu)}]\\
&= \int_{\mathbb{R}} \varphi_t(\lambda)\overline{\varphi_s(\lambda)}\,dF(\lambda)\\
&= \int_{\mathbb{R}} A_t(\lambda)\overline{A_s(\lambda)}e^{i\lambda(t-s)}\,dF(\lambda)
\end{align}
\end{proof}

\section{Stationary Processes and Time Change}\label{sec:stationary_timechange}

\subsection{Stationary processes}
\begin{definition}[Cram\'er representation]\label{def:cramer}
A zero-mean stationary process $X$ with spectral measure $F$ admits
\[
X(t)=\int_{\mathbb{R}} e^{i\lambda t}\,\Phi(d\lambda),\qquad
R_X(t-s)=\int_{\mathbb{R}} e^{i\lambda(t-s)}\,dF(\lambda).
\]
\end{definition}

\subsection{Stationary $\to$ oscillatory via $U_\theta$}
\begin{theorem}[Time change yields oscillatory process]\label{thm:Utheta_to_osc}
Let $X$ be zero-mean stationary as in Definition~\ref{def:cramer}. For $\theta$ as in Definition~\ref{def:Utheta}, define
\[
Z(t)\coloneqq (U_\theta X)(t)=\sqrt{\theta'(t)}\,X(\theta(t)).
\]
Then $Z$ is oscillatory with oscillatory function
\[
\varphi_t(\lambda)=\sqrt{\theta'(t)}\,e^{i\lambda\theta(t)},
\qquad
A_t(\lambda)=\sqrt{\theta'(t)}\,e^{i\lambda(\theta(t)-t)},
\]
and covariance
\[
R_Z(t,s)=\int_{\mathbb{R}} \sqrt{\theta'(t)\theta'(s)}\,e^{i\lambda(\theta(t)-\theta(s))}\,dF(\lambda).
\]
\end{theorem}

\begin{proof}
Applying the unitary time change operator to the spectral representation of $X(t)$:
\begin{align}
Z(t) &= (U_\theta X)(t)\\
&= \sqrt{\theta'(t)}\,X(\theta(t))\\
&= \sqrt{\theta'(t)} \int_{\mathbb{R}} e^{i\lambda\theta(t)}\,\Phi(d\lambda)\\
&= \int_{\mathbb{R}} \sqrt{\theta'(t)}e^{i\lambda\theta(t)}\,\Phi(d\lambda)\\
&= \int_{\mathbb{R}} \varphi_t(\lambda)\,\Phi(d\lambda)
\end{align}
where $\varphi_t(\lambda)=\sqrt{\theta'(t)}e^{i\lambda\theta(t)}$.

To verify this constitutes an oscillatory representation according to Definition~\ref{def:osc_proc}, we must write $\varphi_t(\lambda)$ in the form $A_t(\lambda)e^{i\lambda t}$:
\begin{align}
\varphi_t(\lambda) &= \sqrt{\theta'(t)}e^{i\lambda\theta(t)}\\
&= \sqrt{\theta'(t)}e^{i\lambda(\theta(t)-t)}e^{i\lambda t}\\
&= A_t(\lambda)e^{i\lambda t}
\end{align}
where $A_t(\lambda)=\sqrt{\theta'(t)}e^{i\lambda(\theta(t)-t)}$.

Since $\theta'(t)\geq 0$ almost everywhere and $\theta'(t)=0$ only on sets of measure zero, $A_t(\lambda)$ is well-defined almost everywhere. Moreover, $A_t\in L^2(F)$ for each $t$ since:
\begin{align}
\int_{\mathbb{R}} |A_t(\lambda)|^2\,dF(\lambda) &= \int_{\mathbb{R}} \left|\sqrt{\theta'(t)}e^{i\lambda(\theta(t)-t)}\right|^2\,dF(\lambda)\\
&= \int_{\mathbb{R}} \theta'(t)|e^{i\lambda(\theta(t)-t)}|^2\,dF(\lambda)\\
&= \theta'(t)\int_{\mathbb{R}} dF(\lambda)\\
&= \theta'(t)F(\mathbb{R}) < \infty
\end{align}
where we used $|e^{i\alpha}|=1$ for all real $\alpha$.

The covariance is computed as:
\begin{align}
R_Z(t,s) &= \mathbb{E}[Z(t)\overline{Z(s)}]\\
&= \mathbb{E}\left[\sqrt{\theta'(t)}X(\theta(t))\sqrt{\theta'(s)}\overline{X(\theta(s))}\right]\\
&= \sqrt{\theta'(t)\theta'(s)}\mathbb{E}[X(\theta(t))\overline{X(\theta(s))}]\\
&= \sqrt{\theta'(t)\theta'(s)}R_X(\theta(t)-\theta(s))\\
&= \sqrt{\theta'(t)\theta'(s)}\int_{\mathbb{R}} e^{i\lambda(\theta(t)-\theta(s))}\,dF(\lambda)
\end{align}
\end{proof}

\begin{corollary}[Evolutionary spectrum]\label{cor:evol_spec}
The evolutionary spectrum is $dF_t(\lambda)=|A_t(\lambda)|^2\,dF(\lambda)=\theta'(t)\,dF(\lambda)$.
\end{corollary}

\begin{proof}
By definition of the evolutionary spectrum and using the gain function from Theorem~\ref{thm:Utheta_to_osc}:
\begin{align}
dF_t(\lambda) &= |A_t(\lambda)|^2\,dF(\lambda)\\
&= \left|\sqrt{\theta'(t)}e^{i\lambda(\theta(t)-t)}\right|^2\,dF(\lambda)\\
&= \theta'(t)|e^{i\lambda(\theta(t)-t)}|^2\,dF(\lambda)\\
&= \theta'(t)\,dF(\lambda)
\end{align}
since $|e^{i\alpha}|=1$ for all real $\alpha$.
\end{proof}

\subsection{Covariance operator conjugation}
\begin{proposition}[Operator conjugation]\label{prop:conjugation}
Let $(T_K f)(t)\coloneqq \int_{\mathbb{R}} K(|t-s|)\,f(s)\,ds$ with stationary kernel $K(h)=\int_{\mathbb{R}} e^{i\lambda h}\,dF(\lambda)$. Define the transformed kernel
\[
K_\theta(s,t)\coloneqq \sqrt{\theta'(t)\theta'(s)}\,K\!\left(|\theta(t)-\theta(s)|\right)
\]
and operator $(T_{K_\theta}f)(t)\coloneqq \int_{\mathbb{R}} K_\theta(s,t)\,f(s)\,ds$.
Then $T_{K_\theta}=U_\theta\,T_K\,U_\theta^{-1}$ on $L^2_{\sigma\text{-comp}}(\mathbb{R})$.
\end{proposition}

\begin{proof}
For any $g\in L^2_{\sigma\text{-comp}}(\mathbb{R})$, we compute $(U_\theta T_K U_\theta^{-1}g)(t)$ step by step.

First, $(U_\theta^{-1}g)(s) = \frac{g(\theta^{-1}(s))}{\sqrt{\theta'(\theta^{-1}(s))}}$.

Second, $(T_K U_\theta^{-1}g)(t) = \int_{\mathbb{R}} K(|t-s|)\frac{g(\theta^{-1}(s))}{\sqrt{\theta'(\theta^{-1}(s))}}\,ds$.

Apply change of variables $u=\theta^{-1}(s)$, so $s=\theta(u)$ and $ds=\theta'(u)\,du$:
\begin{align}
(T_K U_\theta^{-1}g)(t) &= \int_{\mathbb{R}} K(|t-\theta(u)|)\frac{g(u)}{\sqrt{\theta'(u)}}\theta'(u)\,du\\
&= \int_{\mathbb{R}} K(|t-\theta(u)|)g(u)\sqrt{\theta'(u)}\,du
\end{align}

Third, $(U_\theta T_K U_\theta^{-1}g)(t) = \sqrt{\theta'(t)}(T_K U_\theta^{-1}g)(\theta(t))$:
\begin{align}
&= \sqrt{\theta'(t)}\int_{\mathbb{R}} K(|\theta(t)-\theta(u)|)g(u)\sqrt{\theta'(u)}\,du\\
&= \int_{\mathbb{R}} \sqrt{\theta'(t)\theta'(u)}K(|\theta(t)-\theta(u)|)g(u)\,du
\end{align}

Finally, changing variables back with $s=\theta(u)$:
\begin{align}
&= \int_{\mathbb{R}} \sqrt{\theta'(t)\theta'(s)}K(|\theta(t)-\theta(s)|)g(s)\,ds\\
&= \int_{\mathbb{R}} K_\theta(s,t)g(s)\,ds\\
&= (T_{K_\theta}g)(t)
\end{align}

This establishes the conjugation relation $T_{K_\theta}=U_\theta T_K U_\theta^{-1}$.
\end{proof}

\section{Sample Paths Live in $L^2_{\sigma\text{-comp}}$}\label{sec:samplepaths}

\begin{theorem}[Sample paths in $L^2_{\sigma\text{-comp}}(\mathbb{R})$]\label{thm:paths_sigma_comp}
Let $\{X(t)\}_{t\in\mathbb{R}}$ be a second-order stationary process with $\sigma^2\coloneqq \mathbb{E}[X(t)^2]<\infty$. Then, almost surely, every sample path $t\mapsto X(\omega,t)$ belongs to $L^2_{\sigma\text{-comp}}(\mathbb{R})$.
\end{theorem}

\begin{proof}
Fix any bounded interval $[a,b]$ and consider the random variable $Y_{[a,b]}\coloneqq \int_a^b X(t)^2\,dt$. By stationarity and Fubini's theorem:
\[
\mathbb{E}[Y_{[a,b]}] = \mathbb{E}\left[\int_a^b X(t)^2\,dt\right] = \int_a^b \mathbb{E}[X(t)^2]\,dt = \int_a^b \sigma^2\,dt = \sigma^2(b-a)<\infty.
\]

By Markov's inequality, for any $M>0$:
\[
P(Y_{[a,b]}>M) \leq \frac{\mathbb{E}[Y_{[a,b]}]}{M} = \frac{\sigma^2(b-a)}{M}.
\]

Taking $M\to\infty$, we conclude $P(Y_{[a,b]}<\infty)=1$, i.e., almost surely the sample path is square-integrable on $[a,b]$.

Since $\mathbb{R}$ is the countable union of bounded intervals:
\[
\mathbb{R} = \bigcup_{n=1}^\infty [-n,n],
\]
by countable subadditivity of probability:
\[
P\left(\bigcap_{n=1}^\infty \left\{\int_{-n}^n X(t)^2\,dt<\infty\right\}\right) = 1.
\]

Now let $U$ be any $\sigma$-compact set. Then $U=\bigcup_{m=1}^\infty K_m$ where each $K_m$ is compact. Each compact set $K_m$ is bounded, so $K_m\subseteq [-N_m,N_m]$ for some $N_m$. Therefore:
\[
\int_U X(t)^2\,dt = \int_{\bigcup_{m=1}^\infty K_m} X(t)^2\,dt \leq \sum_{m=1}^\infty \int_{K_m} X(t)^2\,dt \leq \sum_{m=1}^\infty \int_{-N_m}^{N_m} X(t)^2\,dt.
\]

Since each integral $\int_{-N_m}^{N_m} X(t)^2\,dt<\infty$ almost surely, and the sum of countably many finite terms is finite, we have $\int_U X(t)^2\,dt<\infty$ almost surely.

This holds for every $\sigma$-compact set $U$, so almost surely every sample path lies in $L^2_{\sigma\text{-comp}}(\mathbb{R})$.
\end{proof}

\section{Zero Localization and Hilbert--P\'olya Scaffold}\label{sec:HP}

\subsection{Zero localization measure}
\begin{definition}[Zero localization measure]\label{def:zeromeasure}
Let $Z$ be real-valued with $Z\in C^1(\mathbb{R})$ and only simple zeros $Z(t_0)=0\Rightarrow Z'(t_0)\neq 0$. Define, for Borel $B\subset\mathbb{R}$,
\[
\mu(B)\coloneqq \int_{\mathbb{R}} \mathbf{1}_B(t)\,\delta(Z(t))\,|Z'(t)|\,dt.
\]
\end{definition}

\begin{theorem}[Atomicity on the zero set]\label{thm:atomic}
For every $\phi\in C_c^\infty(\mathbb{R})$,
\[
\int_{\mathbb{R}} \phi(t)\,\delta(Z(t))\,|Z'(t)|\,dt=\sum_{t_0:Z(t_0)=0}\phi(t_0),
\]
hence $\mu=\sum_{t_0:Z(t_0)=0}\delta_{t_0}$.
\end{theorem}

\begin{proof}
Since all zeros of $Z$ are simple and $Z\in C^1(\mathbb{R})$, by the inverse function theorem each zero $t_0$ is isolated. Near each zero $t_0$, $Z$ is locally monotonic, so we can apply the one-dimensional change of variables formula for the Dirac delta.

Specifically, near $t_0$ where $Z(t_0)=0$ and $Z'(t_0)\neq 0$, we have locally $Z(t)=(t-t_0)Z'(t_0)+O((t-t_0)^2)$. The distributional identity for the Dirac delta under smooth changes of variables gives:
\[
\delta(Z(t)) = \sum_{t_0:Z(t_0)=0} \frac{\delta(t-t_0)}{|Z'(t_0)|}.
\]

Therefore:
\begin{align}
\int_{\mathbb{R}} \phi(t)\,\delta(Z(t))\,|Z'(t)|\,dt &= \int_{\mathbb{R}} \phi(t)\,|Z'(t)|\sum_{t_0:Z(t_0)=0} \frac{\delta(t-t_0)}{|Z'(t_0)|}\,dt\\
&= \sum_{t_0:Z(t_0)=0} \int_{\mathbb{R}} \phi(t)\frac{|Z'(t)|\delta(t-t_0)}{|Z'(t_0)|}\,dt\\
&= \sum_{t_0:Z(t_0)=0} \frac{|Z'(t_0)|\phi(t_0)}{|Z'(t_0)|}\\
&= \sum_{t_0:Z(t_0)=0} \phi(t_0)
\end{align}

This shows that $\mu$ is the discrete measure $\mu=\sum_{t_0:Z(t_0)=0}\delta_{t_0}$ assigning unit mass to each zero.
\end{proof}

\subsection{Hilbert space on zeros and multiplication operator}
\begin{definition}[Hilbert space on the zero set]\label{def:Hmu}
Let $\mathcal{H}\coloneqq L^2(\mu)$ with inner product $\langle f,g\rangle=\int f(t)\overline{g(t)}\,\mu(dt)$.
\end{definition}

\begin{proposition}[Atomic structure]\label{prop:atomic}
With $\mu=\sum_{t_0:Z(t_0)=0}\delta_{t_0}$,
\[
\mathcal{H}\cong \left\{f:\{t_0:Z(t_0)=0\}\to\mathbb{C}:\ \sum_{t_0:Z(t_0)=0}|f(t_0)|^2<\infty\right\}\cong \ell^2,
\]
with orthonormal basis $\{e_{t_0}\}_{t_0:Z(t_0)=0}$, where $e_{t_0}(t_1)=\delta_{t_0 t_1}$.
\end{proposition}

\begin{proof}
By the atomic form of $\mu$, for any $f\in L^2(\mu)$:
\begin{align}
\|f\|_{\mathcal{H}}^2 &= \int |f(t)|^2\,\mu(dt)\\
&= \int |f(t)|^2\sum_{t_0:Z(t_0)=0}\delta_{t_0}(dt)\\
&= \sum_{t_0:Z(t_0)=0} |f(t_0)|^2
\end{align}

This shows the isomorphism with $\ell^2$. The functions $e_{t_0}$ defined by $e_{t_0}(t_1)=\delta_{t_0 t_1}$ satisfy:
\[
\langle e_{t_0}, e_{t_1}\rangle = \int e_{t_0}(t)\overline{e_{t_1}(t)}\,\mu(dt) = \sum_{t:Z(t)=0} \delta_{t_0 t}\delta_{t_1 t} = \delta_{t_0 t_1}
\]
so they form an orthonormal set. Any $f\in\mathcal{H}$ can be written as $f=\sum_{t_0:Z(t_0)=0} f(t_0)e_{t_0}$, proving they form a basis.
\end{proof}

\begin{definition}[Multiplication operator]\label{def:L}
Define $L:\mathcal{D}(L)\subset\mathcal{H}\to\mathcal{H}$ by $(Lf)(t)=t\,f(t)$ on $\operatorname{supp}(\mu)$ with domain
\[
\mathcal{D}(L)\coloneqq \left\{f\in\mathcal{H}:\ \int |t\,f(t)|^2\,\mu(dt)<\infty\right\}.
\]
\end{definition}

\begin{theorem}[Self-adjointness and spectrum]\label{thm:spectrum}
$L$ is self-adjoint on $\mathcal{H}$ and has pure point, simple spectrum
\[
\sigma(L)=\{t\in\mathbb{R}:\ Z(t)=0\},
\]
with eigenvalues $\lambda=t_0$ and eigenvectors $e_{t_0}$.
\end{theorem}

\begin{proof}
First, we verify self-adjointness. For $f,g\in\mathcal{D}(L)$:
\begin{align}
\langle Lf,g\rangle &= \int (Lf)(t)\overline{g(t)}\,\mu(dt)\\
&= \int t f(t)\overline{g(t)}\,\mu(dt)\\
&= \int f(t)\overline{t g(t)}\,\mu(dt)\\
&= \int f(t)\overline{(Lg)(t)}\,\mu(dt)\\
&= \langle f,Lg\rangle
\end{align}
Thus $L$ is symmetric. 

In the atomic representation, $L$ acts as $(Lf)(t_0)=t_0 f(t_0)$ for each $t_0$ where $Z(t_0)=0$. This is unitarily equivalent to the diagonal operator on $\ell^2$ with diagonal entries $\{t_0:Z(t_0)=0\}$. Such diagonal operators are self-adjoint.

For the spectrum calculation: $Le_{t_0}=t_0 e_{t_0}$, so each $t_0$ where $Z(t_0)=0$ is an eigenvalue with eigenvector $e_{t_0}$. Since $\{e_{t_0}\}$ forms an orthonormal basis, $L$ has pure point spectrum.

To show there are no other spectral points, suppose $\lambda\notin\{t_0:Z(t_0)=0\}$. Then for any $f\in\mathcal{D}(L)$, $(L-\lambda I)f$ has components $((L-\lambda I)f)(t_0)=(t_0-\lambda)f(t_0)$. Since $t_0-\lambda\neq 0$ for all zeros $t_0$, we can solve $(L-\lambda I)f=g$ uniquely for any $g\in\mathcal{H}$ by setting $f(t_0)=\frac{g(t_0)}{t_0-\lambda}$. This shows $L-\lambda I$ is invertible, so $\lambda\notin\sigma(L)$.

Therefore $\sigma(L)=\{t_0:Z(t_0)=0\}$ exactly, with simple eigenvalues.
\end{proof}

\begin{remark}[Operator scaffold]\label{rem:scaffold}
The construction
\[
\text{stationary }X \xrightarrow{\ U_\theta\ }\ \text{oscillatory }Z \xrightarrow{\ \mu=\delta(Z)|Z'|\,dt\ }\ L^2(\mu)\xrightarrow{\ L:t\cdot\ }\ (L,\sigma(L))
\]
produces a concrete self-adjoint operator whose spectrum equals the zero set of $Z$, determined by the choice of time-change $\theta$ and spectral measure $F$. This provides an explicit realization consistent with Hilbert--P\'olya approaches to encoding arithmetic information in operator spectra.
\end{remark}

\section{Appendix: Regularity and Simple Zeros}\label{sec:appendix}

\begin{definition}[Regularity and simplicity]\label{def:regularity}
Assume $Z\in C^1(\mathbb{R})$ and every zero is simple: $Z(t_0)=0\Rightarrow Z'(t_0)\neq 0$.
\end{definition}

\begin{lemma}[Local finiteness and delta decomposition]\label{lem:delta}
Under Definition~\ref{def:regularity}, zeros are locally finite and
\[
\delta(Z(t))=\sum_{t_0:Z(t_0)=0}\frac{\delta(t-t_0)}{|Z'(t_0)|},
\]
whence $\mu=\sum_{t_0:Z(t_0)=0}\delta_{t_0}$.
\end{lemma}

\begin{proof}
Since $Z\in C^1(\mathbb{R})$ and $Z'(t_0)\neq 0$ at each zero $t_0$, the inverse function theorem implies that $Z$ is locally invertible near each zero. Specifically, there exists a neighborhood $U_{t_0}$ of $t_0$ such that $Z|_{U_{t_0}}$ is strictly monotonic and invertible.

This implies zeros are isolated: if $Z(t_0)=0$ and $Z'(t_0)\neq 0$, then there exists $\epsilon>0$ such that $Z(t)\neq 0$ for $0<|t-t_0|<\epsilon$. Therefore zeros are locally finite (finitely many in any bounded interval).

For the distributional identity, consider the one-dimensional change of variables formula for the Dirac delta. If $g:I\to\mathbb{R}$ is $C^1$ on interval $I$ with $g'(x)\neq 0$ for all $x\in I$, then
\[
\delta(g(x))=\sum_{x_0:g(x_0)=0}\frac{\delta(x-x_0)}{|g'(x_0)|}.
\]

Applying this locally around each zero $t_0$ of $Z$, and since zeros are isolated, we can patch together the local results to obtain the global identity:
\[
\delta(Z(t))=\sum_{t_0:Z(t_0)=0}\frac{\delta(t-t_0)}{|Z'(t_0)|}.
\]

Consequently:
\[
\mu(dt) = \delta(Z(t))|Z'(t)|\,dt = \sum_{t_0:Z(t_0)=0}\frac{|Z'(t)|}{|Z'(t_0)|}\delta(t-t_0)\,dt = \sum_{t_0:Z(t_0)=0}\delta_{t_0}(dt),
\]
where the last equality uses the fact that $|Z'(t_0)|/|Z'(t_0)|=1$ when evaluating at $t=t_0$.
\end{proof}

\end{document}
