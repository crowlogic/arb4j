\documentclass[11pt,a4paper]{article}
\usepackage{amsmath}
\usepackage{amssymb}
\usepackage{amsfonts}
\usepackage{mathtools}
\usepackage{amsthm}
\usepackage{geometry}
\usepackage{hyperref}
\geometry{margin=1in}

\theoremstyle{plain}
\newtheorem{theorem}{Theorem}[section]
\newtheorem{lemma}[theorem]{Lemma}
\newtheorem{proposition}[theorem]{Proposition}
\newtheorem{corollary}[theorem]{Corollary}

\theoremstyle{definition}
\newtheorem{definition}[theorem]{Definition}
\newtheorem{remark}[theorem]{Remark}

\title{Unitarily Time-Changed Stationary Processes:\\A Subclass of Oscillatory Processes}
\author{Stephen Crowley}
\date{December 21, 2025}

\begin{document}

\maketitle

\begin{abstract}
Unitarily time-changed stationary processes form a proper subclass of oscillatory processes in the sense of Priestley. For any stationary process with spectral representation, the unitary time-change operator produces an oscillatory process with explicitly computable gain function. The Hardy Z-function is shown to be a member of this class through construction of its orthogonal random measure via spectral inversion. The underlying stationary process possesses well-defined Cesàro covariance structure, and the Kac-Rice formula yields zero-counting results that correspond to the smooth part of the Backlund counting function.
\end{abstract}

\tableofcontents

\section{Introduction}

The framework of oscillatory processes provides tools for studying stochastic processes where spectral characteristics vary with time. This work demonstrates that unitarily time-changed stationary processes form a natural subclass of oscillatory processes. Given any stationary process and a suitable time-change function satisfying required monotonicity properties, the resulting process admits an oscillatory representation with gain function determined explicitly by the time-change derivative.

The Hardy Z-function provides a concrete instantiation of this theory, illustrating connections between analytic number theory and stochastic process theory.

\section{Unitary Time-Change Operators}

\begin{definition}
[Time-Change Operator] Let \(\Theta : \mathbb{R} \to \mathbb{R}\) be absolutely continuous, strictly increasing, and bijective with \(\dot{\Theta}(t) > 0\) almost everywhere. The bounded operator \(U_{\Theta}\) on \(L^2_{\mathrm{loc}}(\mathbb{R})\) is defined by:
\[ (U_{\Theta} f)(t) = \sqrt{\dot{\Theta}(t)} f(\Theta(t)) \]
with inverse:
\[ (U_{\Theta}^{-1} g)(s) = \frac{g(\Theta^{-1}(s))}{\sqrt{\dot{\Theta}(\Theta^{-1}(s))}} \]
\end{definition}

\begin{theorem}
[Local Isometry] For every compact \(K \subseteq \mathbb{R}\) and \(f \in L^2_{\mathrm{loc}}(\mathbb{R})\):
\[ \int_K |(U_{\Theta} f)(t)|^2 dt = \int_{\Theta(K)} |f(s)|^2 ds \]
The operators satisfy \((U_{\Theta}^{-1} \circ U_{\Theta})f = f\) and \((U_{\Theta} \circ U_{\Theta}^{-1})g = g\).
\end{theorem}

\begin{proof}
The change of variables \(s = \Theta(t)\) with \(ds = \dot{\Theta}(t)dt\) yields:
\[ \int_K |(U_{\Theta} f)(t)|^2 dt = \int_K \dot{\Theta}(t)|f(\Theta(t))|^2 dt = \int_{\Theta(K)} |f(s)|^2 ds \]
For the inverse identities:
\[ (U_{\Theta}^{-1}(U_{\Theta} f))(s) = \frac{(U_{\Theta} f)(\Theta^{-1}(s))}{\sqrt{\dot{\Theta}(\Theta^{-1}(s))}} = \frac{\sqrt{\dot{\Theta}(\Theta^{-1}(s))} f(\Theta(\Theta^{-1}(s)))}{\sqrt{\dot{\Theta}(\Theta^{-1}(s))}} = f(s) \]
Similarly, \((U_{\Theta}(U_{\Theta}^{-1} g))(t) = g(t)\).
\end{proof}

\section{Oscillatory Processes}

\begin{definition}
[Oscillatory Process] An oscillatory process possesses a spectral representation:
\[ Z(t) = \int_{\mathbb{R}} A_t(\lambda) e^{i\lambda t} d\Phi(\lambda) \]
where \(A_t(\lambda)\) is a time-dependent gain function and \(\Phi\) is an orthogonal random measure.
\end{definition}

\begin{theorem}
[Time-Changed Processes are Oscillatory] Let \(X\) be a stationary process with spectral representation:
\[ X(u) = \int_{\mathbb{R}} e^{i\lambda u} d\Phi(\lambda) \]
where \(\Phi\) is an orthogonal random measure. Let \(\Theta\) satisfy Definition 2.1. Then the time-changed process
\[ Z(t) = (U_{\Theta} X)(t) = \sqrt{\dot{\Theta}(t)} X(\Theta(t)) \]
is an oscillatory process with gain function:
\[ A_t(\lambda) = \sqrt{\dot{\Theta}(t)} e^{i\lambda(\Theta(t) - t)} \]
\end{theorem}

\begin{proof}
Substituting \(u = \Theta(t)\) in the spectral representation of \(X\):
\begin{align*}
Z(t) &= \sqrt{\dot{\Theta}(t)} X(\Theta(t)) = \sqrt{\dot{\Theta}(t)} \int_{\mathbb{R}} e^{i\lambda\Theta(t)} d\Phi(\lambda) \\
&= \int_{\mathbb{R}} \sqrt{\dot{\Theta}(t)} e^{i\lambda\Theta(t)} d\Phi(\lambda)
\end{align*}
Factoring \(e^{i\lambda\Theta(t)} = e^{i\lambda(\Theta(t) - t)} e^{i\lambda t}\) and setting \(A_t(\lambda) = \sqrt{\dot{\Theta}(t)} e^{i\lambda(\Theta(t) - t)}\) yields the oscillatory representation.
\end{proof}

\section{Application to the Hardy Z-Function}

\subsection{The Riemann-Siegel Theta Function}

\begin{definition}
[Riemann-Siegel Theta Function]
\[ \theta(t) = \mathrm{Im}\left[\log\Gamma\left(\frac{1}{4} + \frac{it}{2}\right)\right] - \frac{t}{2}\log\pi \]
\end{definition}

\begin{lemma}
[Stirling's Formula] For \(z\) with \(|\arg(z)| < \pi\):
\[ \log\Gamma(z) = \left(z - \frac{1}{2}\right)\log z - z + \frac{1}{2}\log(2\pi) + O(|z|^{-1}) \]
\end{lemma}

\begin{theorem}
[Asymptotic Expansion]
\[ \theta'(t) = \frac{1}{2}\log\frac{t}{2\pi} + O(t^{-1}) \]
\end{theorem}

\begin{proof}
For \(z = 1/4 + it/2\) with \(t > 0\), the modulus and argument are:
\[ |z| = \frac{t}{2}(1 + O(t^{-2})), \quad \arg(z) = \frac{\pi}{2} - \frac{1}{2t} + O(t^{-3}) \]
Applying Stirling's formula and extracting the imaginary part yields:
\[ \theta(t) = \frac{t}{2}\log\frac{t}{2\pi e} - \frac{\pi}{8} + O(t^{-1}) \]
Differentiation gives the result.
\end{proof}

\begin{theorem}
[Vanishing Logarithmic Ratio] For fixed \(n \geq 1\):
\[ \lim_{t \to \infty} \frac{\log n}{\theta'(t)} = 0 \]
\end{theorem}

\begin{proof}
From the previous theorem, \(\theta'(t) = \frac{1}{2}\log\frac{t}{2\pi} + O(t^{-1})\). As \(t \to \infty\), the denominator grows unboundedly while the numerator remains constant.
\end{proof}

\subsection{The Hardy Z-Function as Time-Changed Process}

\begin{definition}
[Hardy Z-Function]
\[ Z(t) = e^{i\theta(t)}\zeta(1/2 + it) \]
\end{definition}

\begin{definition}
[Monotonized Time-Change] Define \(\tau(t)\) by:
\[ \tau(t) = \begin{cases} 2\theta(a) - \theta(t) & t < a \\ \theta(t) & t \geq a \end{cases} \]
where \(a\) is the critical point satisfying \(\theta'(a) = 0\).
\end{definition}

\subsection{Spectral Inversion Formula}

\begin{theorem}
[Spectral Representation of Stationary Process] For the stationary process \(X(u) = (U_{\tau}^{-1} Z)(u)\), there exists an orthogonal random measure \(\Phi\) such that:
\[ X(u) = \int_{\mathbb{R}} e^{i\lambda u} d\Phi(\lambda) \]
and the cumulative orthogonal random measure is:
\[ \Phi(\lambda) = \frac{1}{\pi} \int_0^\infty \frac{\sin(u\lambda)}{u} \cdot \frac{Z(\tau^{-1}(u))}{\sqrt{\tau'(\tau^{-1}(u))}} du \]
\end{theorem}

\subsection{Reconstruction of Z from Spectral Representation}

The spectral representation with the orthogonal random measure substituted inline is:

\[ Z(t) = \sqrt{\tau'(t)} \int_{\mathbb{R}} e^{i\lambda \tau(t)} d\left[\frac{1}{\pi} \int_0^\infty \frac{\sin(v\lambda)}{v} \cdot \frac{Z(\tau^{-1}(v))}{\sqrt{\tau'(\tau^{-1}(v))}} dv\right] \]

Applying Fubini to interchange the order of integration:

\[ Z(t) = \frac{\sqrt{\tau'(t)}}{\pi} \int_0^\infty \frac{Z(\tau^{-1}(v))}{\sqrt{\tau'(\tau^{-1}(v))}} \frac{1}{v} \left[\int_{\mathbb{R}} e^{i\lambda \tau(t)} \sin(v\lambda) d\lambda\right] dv \]

The inner integral over \(\lambda\) is:

\[ \int_{\mathbb{R}} e^{i\lambda \tau(t)} \sin(v\lambda) d\lambda = \pi[\delta(v - \tau(t)) - \delta(v + \tau(t))] \]

For \(v > 0\) and \(\tau(t) > 0\), only the first delta function contributes:

\[ Z(t) = \sqrt{\tau'(t)} \int_0^\infty \frac{Z(\tau^{-1}(v))}{\sqrt{\tau'(\tau^{-1}(v))}} \delta(v - \tau(t)) dv \]

Evaluating the delta function at \(v = \tau(t)\):

\[ Z(t) = \sqrt{\tau'(t)} \cdot \frac{Z(t)}{\sqrt{\tau'(t)}} = Z(t) \]

This recovers the Hardy Z-function.

\section{Cesàro Stationarity}

\subsection{Phase Analysis}

\begin{definition}
[Underlying Stationary Process] For \(u \geq \tau(T_0)\):
\[ X(u) = (U_{\tau}^{-1} Z)(u) = \frac{Z(\tau^{-1}(u))}{\sqrt{\tau'(\tau^{-1}(u))}} \]
\end{definition}

\begin{theorem}
[Riemann-Siegel in Stationary Coordinates] For \(u = \tau(t)\) with \(t = \tau^{-1}(u) \geq T_0\):
\[ X(u) = \frac{1}{\sqrt{\tau'(\tau^{-1}(u))}} \left[2\sum_{n=1}^{N(\tau^{-1}(u))} n^{-1/2}\cos(u - \tau^{-1}(u)\log n) + R(\tau^{-1}(u))\right] \]
where \(N(t) = \left\lfloor\sqrt{\frac{t}{2\pi}}\right\rfloor\).
\end{theorem}

\subsection{Cesàro Convergence}

Define the phase:
\[ \Phi_n(u) = \tau(\tau^{-1}(u)) - \tau^{-1}(u)\log n = u - \tau^{-1}(u)\log n \]

\begin{lemma}
[Phase Difference] For fixed \(h \in \mathbb{R}\) and fixed \(n \geq 1\):
\[ \lim_{u \to \infty} [\Phi_n(u) - \Phi_n(u+h)] = -h \]
\end{lemma}

\begin{proof}
\[ \Phi_n(u) - \Phi_n(u+h) = [u - (u+h)] - [\tau^{-1}(u) - \tau^{-1}(u+h)]\log n = -h - [\tau^{-1}(u) - \tau^{-1}(u+h)]\log n \]
By mean-value theorem, for some \(\xi \in (u, u+h)\):
\[ \tau^{-1}(u+h) - \tau^{-1}(u) = \frac{h}{\tau'(\tau^{-1}(\xi))} \]
Therefore:
\[ [\tau^{-1}(u) - \tau^{-1}(u+h)]\log n = -\frac{h\log n}{\tau'(\tau^{-1}(\xi))} \to 0 \]
by Theorem 4.3.
\end{proof}

\begin{theorem}
[Cesàro Covariance] The Cesàro limit
\[ R_X(h) = \lim_{U \to \infty} \frac{1}{U} \int_0^U X(u)X(u+h) du \]
exists and equals:
\[ R_X(h) = 2\sum_{n=1}^\infty n^{-1} \cos(h) \]
where convergence is in the Cesàro sense.
\end{theorem}

\section{Kac-Rice Formula and Zero Counting}

\begin{theorem}
[Kac-Rice for Time-Changed Processes] Let \(X(u)\) be a centered stationary Gaussian process with unit variance and finite spectral variance \(\sigma_X < \infty\). Let \(Z(t) = \sqrt{\tau'(t)} X(\tau(t))\). The expected number of zeros in \([0,T]\) is:
\[ \mathbb{E}[N_{[0,T]}] = \frac{\sigma_X}{\pi} \tau(T) \]
\end{theorem}

\begin{definition}
[Backlund Counting Function] The exact number of zeros of \(\zeta(1/2 + it)\) in \(0 < t \leq T\) is:
\[ N(T) = \frac{\theta(T)}{\pi} + 1 + S(T) \]
where \(S(T) = \frac{1}{\pi}\arg\zeta(1/2 + iT)\).
\end{definition}

\begin{corollary}
[Zero Density] With normalization \(\sigma_X = 1\), the expected zero count is:
\[ \mathbb{E}[N_{[0,T]}] = \frac{\theta(T)}{\pi} \]
The Backlund function factorizes as expected count plus fluctuation \(S(T)\).
\end{corollary}

\section{Conclusion}

Unitarily time-changed stationary processes form a proper subclass of oscillatory processes. The gain function is determined explicitly by the time-change derivative and phase shift. For the Hardy Z-function, the spectral inversion formula provides the orthogonal random measure, and the underlying process possesses Cesàro covariance structure. The Kac-Rice formula yields an expected zero count \(\frac{\theta(T)}{\pi}\) that corresponds to the smooth part of the Backlund counting function.

\begin{thebibliography}{99}
\bibitem{priestley} Priestley, M.B. (1965). Evolutionary spectra and non-stationary processes. J. Roy. Statist. Soc. Ser. B, 27(2), 204--237.
\bibitem{titchmarsh} Titchmarsh, E.C. (1986). The Theory of the Riemann Zeta-Function. Oxford University Press.
\bibitem{edwards} Edwards, H.M. (1974). Riemann's Zeta Function. Academic Press.
\bibitem{kac} Kac, M., Slepian, D. (1959). Large excursions of Gaussian processes. Ann. Math. Statist., 30(4), 1215--1228.
\bibitem{rice} Rice, S.O. (1945). Mathematical analysis of random noise. Bell Syst. Tech. J., 24(1), 46--156.
\end{thebibliography}

\end{document}
