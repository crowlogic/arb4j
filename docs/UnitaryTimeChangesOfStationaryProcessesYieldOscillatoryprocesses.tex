\documentclass{article}
\usepackage[english]{babel}
\usepackage{geometry,amsmath,amssymb,latexsym,theorem}
\geometry{letterpaper}

%%%%%%%%%% Start TeXmacs macros
\newcommand{\assign}{:=}
\newcommand{\tmem}[1]{{\em #1\/}}
\newenvironment{proof}{\noindent\textbf{Proof\ }}{\hspace*{\fill}$\Box$\medskip}
\newtheorem{definition}{Definition}
\newcounter{nndefinition}
\def\thenndefinition{\unskip}
\newtheorem{definition*}[nndefinition]{Definition}
\newtheorem{lemma}{Lemma}
\newtheorem{proposition}{Proposition}
{\theorembodyfont{\rmfamily}\newtheorem{remark}{Remark}}
\newtheorem{theorem}{Theorem}
%%%%%%%%%% End TeXmacs macros

\begin{document}

\title{
  Unitary Time Changes of Stationary Processes Yield Oscillatory Processes\\
  and a Functional Framework Toward a Hilbert--P{\'o}lya Construction
}

\author{Stephen Crowley}

\date{September 17, 2025}

\maketitle

{\tableofcontents}

\section{Unitary Time Change on $L^2 (\mathbb{R})$}

\begin{definition}
  [Unitary time change operator on $L^2 (\mathbb{R})$] Let $\theta :
  \mathbb{R} \to \mathbb{R}$ be absolutely continuous with $\theta' (t) \neq
  0$ almost everywhere. Define $U_{\theta} : L^2 (\mathbb{R}) \to L^2
  (\mathbb{R})$ by
  \begin{equation}
    (U_{\theta} f) (t) \assign \sqrt{| \theta' (t) |}  \hspace{0.17em} f
    (\theta (t))  \qquad (f \in L^2 (\mathbb{R})) .
  \end{equation}
\end{definition}

\begin{theorem}
  [Unitarity of $U_{\theta}$] $U_{\theta}$ is unitary on $L^2 (\mathbb{R})$.
\end{theorem}

\begin{proof}
  By absolute continuity and $\theta' (t) \neq 0$ a.e., the
  change-of-variables formula gives
  \[ \int_{\mathbb{R}} | (U_{\theta} f) (t) |^2 \hspace{0.17em} dt =
     \int_{\mathbb{R}} | \theta' (t) | \hspace{0.17em} |f (\theta (t)) |^2
     \hspace{0.17em} dt = \int_{\mathbb{R}} |f (u) |^2  \hspace{0.17em} du, \]
  so $U_{\theta}$ is an isometry. Since $\theta$ admits an a.e. inverse
  $\theta^{- 1}$ with the same regularity and nonvanishing derivative a.e.,
  one has $U_{\theta^{- 1}} U_{\theta} = \mathrm{Id}$ and $U_{\theta}
  U_{\theta^{- 1}} = \mathrm{Id}$ a.e., hence $U_{\theta}$ is unitary.
\end{proof}

\section{Oscillatory Processes in the Sense of Priestley}

\begin{definition}
  [Oscillatory process, gain and oscillatory function] Let $F$ be a finite
  nonnegative Borel measure on $\mathbb{R}$. For each $t \in \mathbb{R}$ let
  $A_t : \mathbb{R} \to \mathbb{C}$ be measurable and square-integrable with
  respect to $F$. Define
  \begin{equation}
    \varphi_t (\lambda) \assign A_t (\lambda)  \hspace{0.17em} e^{i \lambda t}
  \end{equation}
  An {\tmem{oscillatory process}} $Z$ is a stochastic process with spectral
  representation
  \begin{equation}
    Z (t) \assign \int_{\mathbb{R}} \varphi_t (\lambda)  \hspace{0.17em} \Phi
    (d \lambda) = \int_{\mathbb{R}} A_t (\lambda)  \hspace{0.17em} e^{i
    \lambda t}  \hspace{0.17em} \Phi (d \lambda)
  \end{equation}
  where $\Phi$ is a complex orthogonal random measure with spectral measure
  $F$ satisfying the orthogonality of infinitesimal increments
  \begin{equation}
    \mathbb{E} \hspace{-0.17em} \left[ \Phi (d \lambda) \hspace{0.17em}
    \overline{\Phi (d \mu)} \right] = \delta (\lambda - \mu)  \hspace{0.17em}
    dF (\lambda)
  \end{equation}
  The covariance kernel is
  \begin{equation}
    R_Z (t, s) \assign \mathbb{E} \left[ Z (t) \hspace{0.17em} \overline{Z
    (s)} \right] = \int_{\mathbb{R}} A_t (\lambda) \hspace{0.17em}
    \overline{A_s (\lambda)} \hspace{0.17em} e^{i \lambda (t - s)} 
    \hspace{0.17em} dF (\lambda)
  \end{equation}
\end{definition}

\begin{remark}
  [Real-valuedness] $Z$ is real-valued if and only if, for each fixed $t$,
  $A_t  (- \lambda) = \overline{A_t (\lambda)}$ for $F$-a.e. $\lambda$,
  equivalently $\varphi_t  (- \lambda) = \overline{\varphi_t (\lambda)}$ for
  $F$-a.e. $\lambda$.
\end{remark}

\begin{theorem}
  [Existence of oscillatory processes with prescribed $(A_t)_t$] Let $F$ be
  finite and $(A_t)_t$ measurable with $\int |A_t (\lambda) |^2 
  \hspace{0.17em} dF (\lambda) < \infty$ for each $t$. There exists a complex
  orthogonal random measure $\Phi$ on $\mathbb{R}$ with spectral measure $F$
  such that $Z (t) = \int \varphi_t (\lambda)  \hspace{0.17em} \Phi (d
  \lambda)$ is well-defined in $L^2 (\Omega)$ and has covariance
  \begin{equation}
    R_Z (t, s) = \int_{\mathbb{R}} \varphi_t (\lambda) \hspace{0.17em}
    \overline{\varphi_s (\lambda)} \hspace{0.17em} dF (\lambda) =
    \int_{\mathbb{R}} A_t (\lambda) \hspace{0.17em} \overline{A_s (\lambda)}
    \hspace{0.17em} e^{i \lambda (t - s)}  \hspace{0.17em} dF (\lambda)
  \end{equation}
\end{theorem}

\begin{proof}
  Construct the stochastic integral first for simple functions in $L^2 
  (\mathbb{R}, F)$ and extend by isometry using
  \begin{equation}
    \mathbb{E} \left[ | \int g (\lambda) \hspace{0.17em} \Phi (d \lambda) |^2
    \right] = \int |g (\lambda) |^2  \hspace{0.17em} dF (\lambda)
  \end{equation}
  Apply with $g = \varphi_t$ to obtain $Z (t)$ and the stated covariance.
\end{proof}

\section{Unitary Time Changes Map Stationary to Oscillatory}

\begin{definition}
  [Stationary process via Cram{\'e}r representation] A zero-mean stationary
  process $X$ with spectral measure $F$ admits
  \begin{equation}
    X (t) = \int_{\mathbb{R}} e^{i \lambda t}  \hspace{0.17em} \Phi (d
    \lambda)
  \end{equation}
  with covariance
  \begin{equation}
    R_X  (t - s) = \int_{\mathbb{R}} e^{i \lambda (t - s)}  \hspace{0.17em} dF
    (\lambda)
  \end{equation}
\end{definition}

\begin{theorem}
  [Unitary time change yields an oscillatory process] Let $X$ be zero-mean
  stationary with
  \begin{equation}
    X (t) = \int_{\mathbb{R}} e^{i \lambda t}  \hspace{0.17em} \Phi (d
    \lambda)
  \end{equation}
  Let $\theta$ satisfy the hypotheses of the unitary time change and set
  \begin{equation}
    Z (t) \assign (U_{\theta} X) (t) = \sqrt{| \theta' (t) |}  \hspace{0.17em}
    X (\theta (t))
  \end{equation}
  Then $Z$ is an oscillatory process with oscillatory function
  \begin{equation}
    \varphi_t (\lambda) = \sqrt{| \theta' (t) |}  \hspace{0.17em} e^{i \lambda
    \theta (t)}
  \end{equation}
  and gain
  \begin{equation}
    A_t (\lambda) = \sqrt{| \theta' (t) |}  \hspace{0.17em} e^{i \lambda
    (\theta (t) - t)}
  \end{equation}
  The covariance is
  \begin{equation}
    \begin{array}{ll}
      R_Z (t, s) & = \int_{\mathbb{R}} A_t (\lambda) \hspace{0.17em}
      \overline{A_s (\lambda)} \hspace{0.17em} e^{i \lambda (t - s)} 
      \hspace{0.17em} dF (\lambda)\\
      & = \int_{\mathbb{R}} \sqrt{| \theta' (t) \theta' (s) |} 
      \hspace{0.17em} e^{i \lambda (\theta (t) - \theta (s))}  \hspace{0.17em}
      dF (\lambda)
    \end{array}
  \end{equation}
\end{theorem}

\begin{proof}
  Compute
  \begin{equation}
    \begin{array}{ll}
      Z (t) & = \sqrt{| \theta' (t) |}  \hspace{0.17em} X (\theta (t))\\
      & = \sqrt{| \theta' (t) |}  \int_{\mathbb{R}} e^{i \lambda \theta (t)} 
      \hspace{0.17em} \Phi (d \lambda)\\
      & = \int_{\mathbb{R}} \sqrt{| \theta' (t) |}  \hspace{0.17em} e^{i
      \lambda \theta (t)}  \hspace{0.17em} \Phi (d \lambda)
    \end{array}
  \end{equation}
  Thus
  \begin{equation}
    \varphi_t (\lambda) = \sqrt{| \theta' (t) |}  \hspace{0.17em} e^{i \lambda
    \theta (t)}
  \end{equation}
  and
  \begin{equation}
    A_t (\lambda) = \varphi_t (\lambda) e^{- i \lambda t}
  \end{equation}
  The covariance follows from orthogonality of $\Phi$.
\end{proof}

\begin{remark}
  [Real-valuedness under time change] If $X$ is real-valued and $\theta$ is
  real with $\theta' (t) > 0$ a.e., then $Z$ is real-valued by the Hermitian
  symmetry of $A_t$.
\end{remark}

\section{Zero Localization by a Functional Measure}

\begin{definition}
  [Zero localization measure] Let $Z$ be real-valued, with sample paths in
  $C^1 (\mathbb{R})$, and such that every zero of $Z$ is simple (i.e. $Z (t_0)
  = 0 \Longrightarrow Z' (t_0) \neq 0$). Define the measure on Borel $B
  \subset \mathbb{R}$ by
  \begin{equation}
    \mu (B) \assign \int_{\mathbb{R}} \textbf{1}_B (t)  \hspace{0.17em} \delta
    (Z (t)) \hspace{0.17em} |Z' (t) |  \hspace{0.17em} dt
  \end{equation}
\end{definition}

\begin{theorem}
  [Support and mass on the zero set] For any test function $\phi \in
  C_c^{\infty} (\mathbb{R})$,
  \begin{equation}
    \int_{\mathbb{R}} \phi (t)  \hspace{0.17em} \delta (Z (t)) \hspace{0.17em}
    |Z' (t) |  \hspace{0.17em} dt = \sum_{t_0 : Z (t_0) = 0} \phi (t_0)
  \end{equation}
  and hence $\mu = \sum_{t_0 : Z (t_0) = 0} \delta_{t_0}$ is a discrete
  measure assigning unit mass to each simple zero of $Z$.
\end{theorem}

\begin{proof}
  At a simple zero $t_0$, the distributional identity holds:
  \begin{equation}
    \delta (Z (t)) = \frac{\delta (t - t_0)}{|Z' (t_0) |} + \sum_{t_1 \neq t_0
    : Z (t_1) = 0} \frac{\delta (t - t_1)}{|Z' (t_1) |}
  \end{equation}
  Multiplying by $|Z' (t) |$ and integrating against $\phi$ yields the stated
  identity and the atomic form of $\mu$.
\end{proof}

\section{Hilbert Space on the Zero Set and Multiplication Operator}

\begin{definition*}
  [Hilbert space on the zero set via $\mu$] Define
  \begin{equation}
    \mathcal{H} \assign L^2 (\mu) = \left\{ f : \mathbb{R} \to \mathbb{C}
    \hspace{0.17em} : \hspace{0.17em} \|f\|_{\mathcal{H}}^2 = \int |f (t) |^2 
    \hspace{0.17em} \delta (Z (t)) \hspace{0.17em} |Z' (t) | \hspace{0.17em}
    dt < \infty \right\}
  \end{equation}
  The inner product is
  \begin{equation}
    \langle f, g \rangle = \int f (t) \overline{g (t)} \hspace{0.17em} \delta
    (Z (t)) \hspace{0.17em} |Z' (t) |  \hspace{0.17em} dt
  \end{equation}
\end{definition*}

\begin{proposition}
  [Atomic structure] With $\mu = \sum_{t_0 : Z (t_0) = 0} \delta_{t_0}$, one
  has
  \begin{equation}
    \mathcal{H}= \left\{ f : \{t_0 : Z (t_0) = 0\} \to \mathbb{C}
    \hspace{0.17em} : \hspace{0.17em} \sum_{Z (t_0) = 0} |f (t_0) |^2 < \infty
    \right\} \cong \ell^2
  \end{equation}
  and the functions $e_{t_0}$ defined by $e_{t_0} (t_1) = \delta_{t_0 t_1}$
  form an orthonormal basis.
\end{proposition}

\begin{proof}
  Substitute the atomic form of $\mu$ into the $L^2$-definition to obtain the
  $\ell^2$-structure; the canonical coordinate functions form an ONB.
\end{proof}

\begin{definition}
  [Multiplication operator] Define $L : \mathcal{D} (L) \subset \mathcal{H}
  \to \mathcal{H}$ by $(Lf) (t) = t \hspace{0.17em} f (t)$ on $\sup (\mu)$,
  with
  \begin{equation}
    \mathcal{D} (L) = \left\{ f \in \mathcal{H} \hspace{0.17em} :
    \hspace{0.17em} \int |t \hspace{0.17em} f (t) |^2  \hspace{0.17em} \delta
    (Z (t)) \hspace{0.17em} |Z' (t) | \hspace{0.17em} dt < \infty \right\}
  \end{equation}
\end{definition}

\begin{theorem}
  [Self-adjointness and spectrum] $L$ is self-adjoint on $\mathcal{H}$, and
  its spectrum is exactly
  \begin{equation}
    \sigma (L) = \{ \hspace{0.17em} t \in \mathbb{R} \hspace{0.17em} :
    \hspace{0.17em} Z (t) = 0 \hspace{0.17em} \}
  \end{equation}
  with pure point spectrum consisting of simple eigenvalues $\lambda = t_0$
  (for each zero $t_0$) and eigenvectors $e_{t_0}$.
\end{theorem}

\begin{proof}
  For $f, g \in \mathcal{D} (L)$,
  \begin{equation}
    \langle Lf, g \rangle = \int t \hspace{0.17em} f (t) \hspace{0.17em}
    \overline{g (t)} \hspace{0.17em} \delta (Z (t)) \hspace{0.17em} |Z' (t) | 
    \hspace{0.17em} dt = \int f (t) \hspace{0.17em} \overline{t
    \hspace{0.17em} g (t)} \hspace{0.17em} \delta (Z (t)) \hspace{0.17em} |Z'
    (t) |  \hspace{0.17em} dt = \langle f, Lg \rangle
  \end{equation}
  so $L$ is symmetric. On the atomic space, $L$ is unitarily equivalent to the
  diagonal operator $(c_{t_0}) \mapsto (t_0 c_{t_0})$ on $\ell^2$, which is
  self-adjoint with spectrum equal to the set of diagonal entries $\{t_0 : Z
  (t_0) = 0\}$, each simple, with eigenvectors the coordinate basis identified
  with $e_{t_0}$.
\end{proof}

\section{Time-Changed Stationary Processes and a Hilbert--P{\'o}lya Scaffold}

\begin{definition}
  [Arithmetic phase time change] Let $\theta : \mathbb{R} \to \mathbb{R}$ be
  an absolutely continuous phase with $\theta' (t) > 0$ a.e. encoding the
  target arithmetic structure (e.g. a Riemann--Siegel-type phase). Let $X$ be
  zero-mean stationary with spectral measure $F$ and orthogonal random measure
  $\Phi$. Define the time-changed oscillatory process
  \begin{equation}
    Z (t) = \int_{\mathbb{R}} \sqrt{| \theta' (t) |}  \hspace{0.17em} e^{i
    \lambda \theta (t)}  \hspace{0.17em} \Phi (d \lambda)
  \end{equation}
\end{definition}

\begin{proposition}
  [Covariance under time change]
  \begin{equation}
    R_Z (t, s) = \int_{\mathbb{R}} \sqrt{| \theta' (t) \theta' (s) |} 
    \hspace{0.17em} e^{i \lambda (\theta (t) - \theta (s))}  \hspace{0.17em}
    dF (\lambda)
  \end{equation}
  In particular, if $F$ is chosen so that $R_Z$ concentrates along $\theta (t)
  = \theta (s)$, then the correlation structure of $Z$ is phase-aligned with
  $\theta$.
\end{proposition}

\begin{proof}
  Insert the oscillatory function into the covariance integral and use the
  orthogonality relation for $\Phi$.
\end{proof}

\begin{definition}
  [Zero-localized Hilbert space and operator] With the zero localization
  measure $\mu (dt) = \delta (Z (t)) \hspace{0.17em} |Z' (t) | 
  \hspace{0.17em} dt$, define $\mathcal{H}= L^2 (\mu)$ and $L$ as
  multiplication by $t$ on $\mathcal{H}$.
\end{definition}

\begin{theorem}
  [Spectral encoding of zero set] The spectrum of $L$ is the zero set of $Z$:
  \begin{equation}
    \sigma (L) = \{t : Z (t) = 0\}
  \end{equation}
  and $L$ has simple pure point spectrum with eigenvectors supported at
  individual zeros.
\end{theorem}

\begin{proof}
  Follows from the established atomic structure of $\mu$ and the diagonal form
  of $L$ on $L^2 (\mu)$.
\end{proof}

\begin{remark}
  [Operator scaffold] The sequence
  \begin{equation}
    \text{stationary } X \xrightarrow{\hspace{0.17em} U_{\theta}
    \hspace{0.17em}} \text{oscillatory } Z \xrightarrow{\hspace{0.17em} \delta
    (Z) \hspace{0.17em} |Z' |  \hspace{0.17em} dt \hspace{0.17em}} \mu
    \xrightarrow{\hspace{0.17em} L^2 (\mu) \hspace{0.17em}} \mathcal{H}
    \xrightarrow{\hspace{0.17em} t \cdot \hspace{0.17em}} L
  \end{equation}
  produces a concrete self-adjoint operator whose spectrum equals the
  (constructed) zero set governed by the choice of $\theta$ and $F$. Aligning
  $\theta$ and $F$ to a prescribed arithmetic target sets the stage for a
  Hilbert--P{\'o}lya-type identification.
\end{remark}

\section{Appendix: Regularity and Simple Zeros}

\begin{definition}
  [Regularity and simplicity] Assume $Z \in C^1 (\mathbb{R})$ and every zero
  of $Z$ is simple: $Z (t_0) = 0 \Longrightarrow Z' (t_0) \neq 0$.
\end{definition}

\begin{lemma}
  [Local finiteness and decomposition] Under the above condition, zeros are
  locally finite and the distributional identity
  \begin{equation}
    \delta (Z (t)) = \sum_{t_0 : Z (t_0) = 0} \frac{\delta (t - t_0)}{|Z'
    (t_0) |}
  \end{equation}
  holds, yielding
  \begin{equation}
    \mu = \sum_{t_0} \delta_{t_0}
  \end{equation}
\end{lemma}

\begin{proof}
  Continuity and $Z' (t_0) \neq 0$ imply isolated zeros by the inverse
  function theorem; the distributional identity is standard from the
  one-dimensional change-of-variables formula for the Dirac delta under
  monotone $C^1$ maps near each zero.
\end{proof}

\end{document}
