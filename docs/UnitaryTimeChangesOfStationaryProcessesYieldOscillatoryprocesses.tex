\documentclass{article}
\usepackage[english]{babel}
\usepackage{geometry,amsmath,amssymb,latexsym,theorem}
\geometry{letterpaper}

%%%%%%%%%% Start TeXmacs macros
\newcommand{\assign}{:=}
\newcommand{\tmaffiliation}[1]{\\ #1}
\newenvironment{proof}{\noindent\textbf{Proof\ }}{\hspace*{\fill}$\Box$\medskip}
\newtheorem{definition}{Definition}
{\theorembodyfont{\rmfamily}\newtheorem{remark}{Remark}}
\newtheorem{theorem}{Theorem}
%%%%%%%%%% End TeXmacs macros

\begin{document}

\title{Unitary Time Changes of Stationary Processes Yield Oscillatory
Processes}

\author{
  Stephen Crowley
  \tmaffiliation{August 12, 2025}
}

\date{}

\maketitle

\begin{definition}
  [Unitary time change operator on $L^2 (\mathbb{R})$] Let $\theta :
  \mathbb{R} \to \mathbb{R}$ be absolutely continuous with $\theta' (t) \neq
  0$ almost everywhere. The unitary time change operator $U_{\theta}$ on $L^2
  (\mathbb{R})$ is defined by
  \begin{equation}
    (U_{\theta} f) (t) \assign \sqrt{| \theta' (t) |}  \hspace{0.17em} f
    (\theta (t))  \qquad \text{for } f \in L^2 (\mathbb{R})
  \end{equation}
\end{definition}

\begin{theorem}
  [Unitarity of $U_{\theta}$] The operator $U_{\theta}$ defined above is
  unitary on $L^2 (\mathbb{R})$.
\end{theorem}

\begin{proof}
  Absolute continuity with $\theta' (t) \neq 0$ a.e. implies the
  change-of-variables formula
  \begin{equation}
    \int_{\mathbb{R}} | (U_{\theta} f) (t) |^2 \hspace{0.17em} dt =
    \int_{\mathbb{R}} | \theta' (t) | \hspace{0.17em} |f (\theta (t)) |^2
    \hspace{0.17em} dt = \int_{\mathbb{R}} |f (u) |^2  \hspace{0.17em} du
  \end{equation}
  so $U_{\theta}$ is isometric. Surjectivity follows from the same
  change-of-variables applied to $U_{\theta^{- 1}}$, which exists almost
  everywhere under these hypotheses. Hence $U_{\theta}$ is unitary.
\end{proof}

\begin{definition}
  [Oscillatory processes in the sense of Priestley] An oscillatory process $Z$
  is specified by a measurable gain function $A_t (\lambda)$ and has
  oscillatory function
  \begin{equation}
    \varphi_t (\lambda) \assign A_t (\lambda)  \hspace{0.17em} e^{i \lambda t}
  \end{equation}
  The process $Z$ has spectral representation
  \begin{equation}
    Z (t) = \int_{\mathbb{R}} \varphi_t (\lambda)  \hspace{0.17em} \Phi (d
    \lambda) = \int_{\mathbb{R}} A_t (\lambda)  \hspace{0.17em} e^{i \lambda
    t}  \hspace{0.17em} \Phi (d \lambda)
  \end{equation}
  where $\Phi$ is a complex orthogonal random measure on $\mathbb{R}$ with
  spectral measure $F$ satisfying
  \begin{equation}
    E \left[ \Phi (d \lambda) \hspace{0.17em} \overline{\Phi (d \mu)} \right]
    = \textbf{1}_{\{\lambda = \mu\}}  \hspace{0.17em} dF (\lambda)
  \end{equation}
  The covariance kernel of $Z$ is
  \begin{equation}
    R_Z (t, s) \assign E [Z (t) \overline{Z (s)}] = \int_{\mathbb{R}} A_t
    (\lambda) \hspace{0.17em} \overline{A_s (\lambda)} \hspace{0.17em} e^{i
    \lambda (t - s)}  \hspace{0.17em} dF (\lambda)
  \end{equation}
\end{definition}

\begin{remark}
  [Real-valuedness condition] The oscillatory process $Z$ is real-valued if
  and only if the gain satisfies conjugate symmetry:
  \begin{equation}
    A_t  (- \lambda) = \overline{A_t (\lambda)} \quad \text{for $F$-almost
    every } \lambda, \text{for each fixed } t
  \end{equation}
\end{remark}

\begin{theorem}
  [Unitary time change of stationary process yields oscillatory process] Let
  $X$ be a zero-mean stationary Gaussian process with Cram{\'e}r spectral
  representation
  \begin{equation}
    X (t) = \int_{\mathbb{R}} e^{i \lambda t}  \hspace{0.17em} \Phi (d
    \lambda)
  \end{equation}
  where $\Phi$ is the same complex orthogonal random measure with spectral
  measure $F$ as in the oscillatory definition. Let $U_{\theta}$ be a unitary
  time change operator as defined above. Then the transformed process
  \begin{equation}
    Z (t) \assign (U_{\theta} X) (t) = \sqrt{| \theta' (t) |}  \hspace{0.17em}
    X (\theta (t))
  \end{equation}
  is an oscillatory process in the sense of Priestley with oscillatory
  function
  \begin{equation}
    \varphi_t (\lambda) = \sqrt{| \theta' (t) |}  \hspace{0.17em} e^{i \lambda
    \theta (t)}
  \end{equation}
\end{theorem}

\begin{proof}
  Starting from the stationary representation, we compute
  
  \begin{align}
    Z (t) & = \sqrt{| \theta' (t) |}  \hspace{0.17em} X (\theta (t)) \\
    & = \sqrt{| \theta' (t) |}  \int_{\mathbb{R}} e^{i \lambda \theta (t)} 
    \hspace{0.17em} \Phi (d \lambda) \\
    & = \int_{\mathbb{R}} \sqrt{| \theta' (t) |}  \hspace{0.17em} e^{i
    \lambda \theta (t)}  \hspace{0.17em} \Phi (d \lambda) 
  \end{align}
  
  Defining
  \begin{equation}
    \varphi_t (\lambda) \assign \sqrt{| \theta' (t) |}  \hspace{0.17em} e^{i
    \lambda \theta (t)}
  \end{equation}
  we have
  \[ Z (t) = \int_{\mathbb{R}} \varphi_t (\lambda)  \hspace{0.17em} \Phi (d
     \lambda) \]
  which is precisely the oscillatory form. The covariance kernel becomes
  \[ R_Z (t, s) = \int_{\mathbb{R}} \varphi_t (\lambda) \hspace{0.17em}
     \overline{\varphi_s (\lambda)} \hspace{0.17em} dF (\lambda) =
     \int_{\mathbb{R}} \sqrt{| \theta' (t) || \theta' (s) |}  \hspace{0.17em}
     e^{i \lambda (\theta (t) - \theta (s))}  \hspace{0.17em} dF (\lambda) \]
\end{proof}

\begin{theorem}
  [Explicit gain function for unitary time change] In the setting of the
  previous theorem, the gain function for the oscillatory process
  \begin{equation}
    Z (t) = (U_{\theta} X) (t)
  \end{equation}
  is given by
  \begin{equation}
    A_t (\lambda) = \sqrt{| \theta' (t) |}  \hspace{0.17em} e^{i \lambda
    (\theta (t) - t)}
  \end{equation}
  The oscillatory function is
  \begin{equation}
    \varphi_t (\lambda) = A_t (\lambda)  \hspace{0.17em} e^{i \lambda t} =
    \sqrt{| \theta' (t) |}  \hspace{0.17em} e^{i \lambda \theta (t)}
  \end{equation}
  and the covariance kernel takes the form
  \begin{equation}
    R_Z (t, s) = \int_{\mathbb{R}} A_t (\lambda) \hspace{0.17em} \overline{A_s
    (\lambda)} \hspace{0.17em} e^{i \lambda (t - s)}  \hspace{0.17em} dF
    (\lambda)
  \end{equation}
\end{theorem}

\begin{proof}
  From the previous theorem, we have
  \begin{equation}
    \varphi_t (\lambda) = \sqrt{| \theta' (t) |}  \hspace{0.17em} e^{i \lambda
    \theta (t)}
  \end{equation}
  Since the oscillatory function must satisfy
  \begin{equation}
    \varphi_t (\lambda) = A_t (\lambda) e^{i \lambda t}
  \end{equation}
  one solves for the gain:
  \[ A_t (\lambda) = \frac{\varphi_t (\lambda)}{e^{i \lambda t}} =
     \frac{\sqrt{| \theta' (t) |}  \hspace{0.17em} e^{i \lambda \theta
     (t)}}{e^{i \lambda t}} = \sqrt{| \theta' (t) |}  \hspace{0.17em} e^{i
     \lambda (\theta (t) - t)} \]
  and substitutes back into the covariance formula:
  
  \begin{align}
    R_Z (t, s) & = \int_{\mathbb{R}} \varphi_t (\lambda)  \hspace{0.17em}
    \bar{\varphi}_s (\lambda) \} \hspace{0.17em} dF (\lambda) \\
    & = \int_{\mathbb{R}} A_t (\lambda) e^{i \lambda t} \hspace{0.17em}
    \overline{A_s (\lambda) e^{i \lambda s}} \hspace{0.17em} dF (\lambda) \\
    & = \int_{\mathbb{R}} A_t (\lambda) \hspace{0.17em} \overline{A_s
    (\lambda)} \hspace{0.17em} e^{i \lambda (t - s)}  \hspace{0.17em} dF
    (\lambda) 
  \end{align}
\end{proof}

\end{document}
