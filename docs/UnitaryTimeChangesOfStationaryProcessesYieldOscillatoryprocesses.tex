\documentclass{article}
\usepackage[english]{babel}
\usepackage{geometry,amsmath,amssymb,latexsym,theorem}
\geometry{letterpaper}

%%%%%%%%%% Start TeXmacs macros
\newcommand{\assign}{:=}
\newcommand{\cdummy}{\cdot}
\newcommand{\tmtextbf}[1]{\text{{\bfseries{#1}}}}
\newenvironment{proof}{\noindent\textbf{Proof\ }}{\hspace*{\fill}$\Box$\medskip}
\newtheorem{corollary}{Corollary}
\newtheorem{definition}{Definition}
\newtheorem{lemma}{Lemma}
{\theorembodyfont{\rmfamily}\newtheorem{remark}{Remark}}
\newtheorem{theorem}{Theorem}
%%%%%%%%%% End TeXmacs macros

\begin{document}

\title{Unitarily Time-Changed Stationary Processes: A Subclass of Oscillatory
Processes}

\author{Stephen Crowley}

\date{December 13, 2025}

\maketitle

\begin{abstract}
  A unitary time-change operator $U_{\theta}$ is constructed for absolutely
  continuous, strictly increasing time reparametrizations $\theta$, acting on
  functions that are locally square-integrable. Applying $U_{\theta}$ to the
  Cram{\'e}r spectral representation of a stationary process $X (t)$ produces
  the transformed process
  \[ Z (t) = (U_{\theta} X) (t) = \sqrt{\dot{\theta} (t)} X (\theta (t)) =
     \sqrt{\dot{\theta} (t)}  \int_{\mathbb{R}} e^{i \lambda \theta (t)} d
     \Phi (\lambda) \]
  which is an oscillatory process with oscillatory function $\phi_t (\lambda)
  = \sqrt{\dot{\theta} (t)} e^{i \lambda \theta (t)}$, evolutionary power
  spectral density $S_t (\lambda) = \dot{\theta} (t) S (\lambda)$, and
  covariance kernel
  \[ K_Z (t, s) = \sqrt{\dot{\theta} (t)  \dot{\theta} (s)} K_X (\theta (t),
     \theta (s)) \]
  where $K_X$ is the stationary covariance of $X (t) = \int_{\mathbb{R}} e^{i
  \lambda t} d \Phi (\lambda)$. Following Mandrekar's characterization theorem
  {\cite{mandrekar1972}}, every oscillatory process admits a stationary
  representation via shift-commuting operators. The generalized Kac-Rice
  formula for non-stationary processes gives the expected zero-counting
  function. By Bulinskaya's theorem, when the covariance is twice continuously
  differentiable with $R'' (0) < 0$, almost all zeros are simple.
\end{abstract}

{\tableofcontents}

\section{Gaussian Processes}

\subsection{Definition}

\begin{definition}
  \label{def:gaussian_process}(Gaussian process) Let $(\Omega, \mathcal{F},
  \mathbb{P})$ be a probability space and $T$ a nonempty index set. A family
  $\{X_t : t \in T\}$ of real-valued random variables on $(\Omega,
  \mathcal{F}, \mathbb{P})$ is called a Gaussian process if for every finite
  subset $\{t_1, \ldots, t_n \} \subset T$ the random vector $(X_{t_1},
  \ldots, X_{t_n})$ is multivariate normal (possibly degenerate).
  Equivalently, every finite linear combination $\sum_{i = 1}^n a_i X_{t_i}$
  is either almost surely constant or Gaussian. The mean function is $m (t)
  \assign \mathbb{E} [X_t]$ and the covariance kernel is
  \begin{equation}
    \label{eq:covariance_kernel} K (s, t) = \mathrm{Cov} (X_s, X_t)
  \end{equation}
  For any finite $(t_i)_{i = 1}^n \subset T$, the matrix $K_{ij} = K (t_i,
  t_j)$ is symmetric positive semidefinite, and a Gaussian process is
  completely determined in law by $m$ and $K$.
\end{definition}

\subsection{Stationary Processes}

\begin{definition}
  \label{def:cramer_representation}(Cram{\'e}r spectral representation) A
  zero-mean stationary process $X$ with spectral measure $F$ admits the sample
  path representation
  \begin{equation}
    \label{eq:cramer_spectral} X (t) = \int_{\mathbb{R}} e^{i \lambda t} d
    \Phi (\lambda)
  \end{equation}
  which has covariance
  \begin{equation}
    \label{eq:stationary_covariance} R_X  (t - s) = \int_{\mathbb{R}} e^{i
    \lambda (t - s)} dF (\lambda)
  \end{equation}
\end{definition}

\subsection{Sample Path Realizations}

\begin{definition}
  \label{def:L2loc}(Locally square-integrable functions) Define
  \begin{equation}
    \label{eq:L2loc_def} L^2_{\mathrm{loc}} (\mathbb{R}) \assign \left\{ f :
    \mathbb{R} \to \mathbb{C}: \int_K |f (t) |^2 dt < \infty \text{for every
    compact } K \subseteq \mathbb{R} \right\}
  \end{equation}
\end{definition}

\begin{remark}
  \label{rem:bounded_compact}Every bounded measurable set in $\mathbb{R}$ is
  contained in a compact set; hence $L^2_{\mathrm{loc}} (\mathbb{R})$ contains
  functions that are square-integrable on every bounded interval, including
  functions with polynomial growth at infinity.
\end{remark}

\begin{theorem}
  \label{thm:sample_paths_L2loc}(Sample paths in $L^2_{\mathrm{loc}}
  (\mathbb{R})$) Let $\{X (t)\}_{t \in \mathbb{R}}$ be a second-order
  stationary process with
  \begin{equation}
    \label{eq:finite_variance} \sigma^2 \assign \mathbb{E} [X (t)^2] < \infty
  \end{equation}
  Then almost every sample path lies in $L^2_{\mathrm{loc}} (\mathbb{R})$.
\end{theorem}

\begin{proof}
  Fix a bounded interval $[a, b] \subset \mathbb{R}$ with $a < b$ and define
  \begin{equation}
    \label{eq:Y_ab} Y_{[a, b]} \assign \int_a^b X (t)^2 dt
  \end{equation}
  By Tonelli's theorem,
  \begin{equation}
    \label{eq:tonelli_expectation} \mathbb{E} [Y_{[a, b]}] = \int_a^b
    \mathbb{E} [X (t)^2] dt
  \end{equation}
  By stationarity, $\mathbb{E} [X (t)^2] = \sigma^2$, hence
  \begin{equation}
    \label{eq:expected_Y} \mathbb{E} [Y_{[a, b]}] = \sigma^2  (b - a) < \infty
  \end{equation}
  Markov's inequality yields
  \begin{equation}
    \label{eq:markov_bound} \mathbb{P} (Y_{[a, b]} > M) \leq \frac{\sigma^2 
    (b - a)}{M}
  \end{equation}
  so $\mathbb{P} (Y_{[a, b]} < \infty) = 1$. If $K \subset \mathbb{R}$ is
  compact then $K \subseteq [- N, N]$ for some $N > 0$, so
  \begin{equation}
    \label{eq:compact_bound} \int_K X (t)^2 dt \leq \int_{- N}^N X (t)^2 dt <
    \infty \text{a.s.}
  \end{equation}
  Thus $X (\cdummy, \omega) \in L^2_{\mathrm{loc}} (\mathbb{R})$ for almost
  every $\omega$.
\end{proof}

\section{Oscillatory Processes}

\subsection{Definition}

\begin{definition}
  \label{def:oscillatory_process}(Oscillatory process) Let $F$ be a finite
  nonnegative Borel measure on $\mathbb{R}$. Let
  \begin{equation}
    \label{eq:gain_condition} A_t \in L^2 (F) \quad \forall t \in \mathbb{R}
  \end{equation}
  be the gain function and
  \begin{equation}
    \label{eq:oscillatory_function} \phi_t (\lambda) = A_t (\lambda) e^{i
    \lambda t}
  \end{equation}
  the corresponding oscillatory function. An oscillatory process is a
  stochastic process represented as
  \begin{equation}
    \label{eq:oscillatory_representation} \begin{array}{ll}
      Z (t) & = \int_{\mathbb{R}} \phi_t (\lambda) d \Phi (\lambda)\\
      & = \int_{\mathbb{R}} A_t (\lambda) e^{i \lambda t} d \Phi (\lambda)
    \end{array}
  \end{equation}
  where $\Phi$ is a complex orthogonal random measure with spectral measure
  $F$ satisfying
  \begin{equation}
    \label{eq:orthogonal_measure} \mathbb{E} [\Phi (\lambda) \overline{\Phi
    (\mu)}] = \delta (\lambda - \mu) dF (\lambda)
  \end{equation}
  and covariance
  \begin{equation}
    \label{eq:oscillatory_covariance} \begin{array}{ll}
      R_Z (t, s) & =\mathbb{E} [Z (t) \overline{Z (s)}]\\
      & = \int_{\mathbb{R}} A_t (\lambda) \overline{A_s (\lambda)} e^{i
      \lambda (t - s)} dF (\lambda)\\
      & = \int_{\mathbb{R}} \phi_t (\lambda) \overline{\phi_s (\lambda)} dF
      (\lambda)
    \end{array}
  \end{equation}
\end{definition}

\begin{definition}
  \label{def:EPSD}(Evolutionary power spectral density) If $dF (\lambda) = S
  (\lambda) d \lambda$, define
  \begin{equation}
    \label{eq:EPSD} S_t (\lambda) \assign |A_t (\lambda) |^2 S (\lambda)
  \end{equation}
  so that
  \begin{equation}
    \label{eq:EPSD_measure} \begin{array}{ll}
      dF_t (\lambda) & = S_t (\lambda) d \lambda\\
      & = |A_t (\lambda) |^2 dF (\lambda)\\
      & = |A_t (\lambda) |^2 S (\lambda) d \lambda
    \end{array}
  \end{equation}
\end{definition}

\begin{theorem}
  \label{thm:real_valued_criterion}(Real-valuedness criterion for oscillatory
  processes) Let $Z$ be an oscillatory process with $\phi_t (\lambda) = A_t
  (\lambda) e^{i \lambda t}$ and spectral measure $F$. Then $Z$ is real-valued
  if and only if
  \begin{equation}
    \label{eq:real_gain_condition} A_t  (- \lambda) = \overline{A_t (\lambda)}
    \text{for } F \text{-a.e. } \lambda \in \mathbb{R}
  \end{equation}
  equivalently
  \begin{equation}
    \label{eq:real_oscillatory_condition} \phi_t  (- \lambda) =
    \overline{\phi_t (\lambda)} \text{for } F \text{-a.e. } \lambda \in
    \mathbb{R}
  \end{equation}
\end{theorem}

\begin{proof}
  Taking complex conjugates of (\ref{eq:oscillatory_representation}) and
  applying the symmetry $d \overline{\Phi (\lambda)} = d \Phi (- \lambda)$ for
  real processes, with change of variables $\mu = - \lambda$, yields $A_t
  (\lambda) = \overline{A_t  (- \lambda)}$ $F$-a.e. Reversing the steps gives
  the converse.
\end{proof}

\begin{theorem}
  \label{thm:oscillatory_existence}(Existence of oscillatory processes with
  explicit $L^2$-limit construction) Let $F$ be absolutely continuous with
  density $S (\lambda)$ and let $A_t (\lambda) \in L^2 (F)$ for all $t \in
  \mathbb{R}$, measurable jointly in $(t, \lambda)$. Define
  \begin{equation}
    \label{eq:variance_t} \sigma_t^2 \assign \int_{\mathbb{R}} |A_t (\lambda)
    |^2 dF (\lambda) < \infty
  \end{equation}
  Then there exists a complex orthogonal random measure $\Phi$ with spectral
  measure $F$ such that for each fixed $t$ the stochastic integral
  \begin{equation}
    \label{eq:stochastic_integral} Z (t) = \int_{\mathbb{R}} A_t (\lambda)
    e^{i \lambda t} d \Phi (\lambda)
  \end{equation}
  is well-defined as an $L^2 (\Omega)$-limit and has covariance
  (\ref{eq:oscillatory_covariance}).
\end{theorem}

\begin{proof}
  Let $S$ be the set of simple functions $g (\lambda) = \sum_{j = 1}^n c_j 
  \textbf{1}_{E_j} (\lambda)$ with disjoint Borel $E_j$ and $F (E_j) <
  \infty$. Define $\int g \hspace{0.17em} d \Phi \assign \sum_{j = 1}^n c_j
  \Phi (E_j)$. Orthogonality gives the isometry:
  \begin{equation}
    \label{eq:isometry} \mathbb{E} \left| \int g \hspace{0.17em} d \Phi
    \right|^2 = \int_{\mathbb{R}} |g (\lambda) |^2 dF (\lambda)
  \end{equation}
  For $h \in L^2 (F)$, choose $g_n \in S$ with $\|h - g_n \|_{L^2 (F)} \to 0$.
  Then:
  \begin{equation}
    \label{eq:cauchy_sequence} \mathbb{E} \left| \int g_n d \Phi - \int g_m d
    \Phi \right|^2 = \|g_n - g_m \|_{L^2 (F)}^2
  \end{equation}
  and $\lim_{n, m \to \infty} \|g_n - g_m \|_{L^2 (F)}^2 = 0$. Completeness of
  $L^2 (\Omega)$ yields the limit, and the isometry shows independence of the
  approximating sequence.
\end{proof}

\section{Unitarily Time-Changed Stationary Processes}

\subsection{Unitary Time-Change Operator}

\begin{theorem}
  \label{thm:unitary_time_change}(Unitary time-change and local isometry) Let
  $\theta : \mathbb{R} \to \mathbb{R}$ be absolutely continuous, strictly
  increasing, and bijective with $\dot{\theta} (t) > 0$ a.e. For measurable
  $f$, define:
  \begin{equation}
    \label{eq:U_theta} (U_{\theta} f) (t) = \sqrt{\dot{\theta} (t)} f (\theta
    (t))
  \end{equation}
  Define the inverse map:
  \begin{equation}
    \label{eq:U_theta_inverse} (U_{\theta}^{- 1} g) (s) = \frac{g (\theta^{-
    1} (s))}{\sqrt{\dot{\theta} (\theta^{- 1} (s))}}
  \end{equation}
  For every compact $K \subseteq \mathbb{R}$ and $f \in L^2_{\mathrm{loc}}
  (\mathbb{R})$:
  \begin{equation}
    \label{eq:local_isometry} \int_K | (U_{\theta} f) (t) |^2 dt =
    \int_{\theta (K)} |f (s) |^2 ds
  \end{equation}
  Moreover, for $f, g \in L^2_{\mathrm{loc}} (\mathbb{R})$:
  \begin{equation}
    \label{eq:inverse_identities} U_{\theta}^{- 1}  (U_{\theta} f) = f, \quad
    U_{\theta}  (U_{\theta}^{- 1} g) = g
  \end{equation}
\end{theorem}

\begin{proof}
  Using change of variables $s = \theta (t)$, $ds = \dot{\theta} (t) dt$:
  \begin{equation}
    \label{eq:change_of_variables} \int_K \dot{\theta} (t)  |f (\theta (t))
    |^2 dt = \int_{\theta (K)} |f (s) |^2 ds
  \end{equation}
  Direct substitution verifies the inverse identities
  (\ref{eq:inverse_identities}).
\end{proof}

\begin{theorem}
  \label{thm:mandrekar_characterization}(Fundamental inversion via stationary
  representation {\cite{mandrekar1972}}) Let $Z (t)$ be an oscillatory process
  with spectral representation
  \begin{equation}
    \label{eq:Z_oscillatory} Z (t) = \int_{\mathbb{R}} A_t (\lambda) e^{i
    \lambda t} d \Phi (\lambda)
  \end{equation}
  where $A_t \in L^2 (F)$ for each $t$ and $\Phi$ is an orthogonal random
  measure with spectral measure $F$. Then there exists a stationary process
  \begin{equation}
    \label{eq:X_stationary} X (t) = \int_{\mathbb{R}} e^{i \lambda t} d \Phi
    (\lambda)
  \end{equation}
  and for each $t \in \mathbb{R}$ a closed, densely-defined operator $L_t$
  acting on the Hilbert space $H_X (\infty) = \overline{\mathrm{span}} \{X (s)
  : s \in \mathbb{R}\}$ such that
  \begin{equation}
    \label{eq:Z_as_LtX} Z (t) = L_t X (0)
  \end{equation}
  where each operator $L_t$ is defined by the spectral integral
  \begin{equation}
    \label{eq:Lt_operator} L_t = \int_{\mathbb{R}} A_t (\lambda) e^{i \lambda
    t} E (d \lambda)
  \end{equation}
  with domain $D (L_t) \supseteq \{X (s) : s \in \mathbb{R}\}$, where $E$ is
  the spectral measure of the shift group $\{U_s \}_{s \in \mathbb{R}}$
  defined by $U_s X (r) = X (r + s)$. The family of operators $\{L_t \}_{t \in
  \mathbb{R}}$ commutes with the shift group:
  \begin{equation}
    \label{eq:shift_commutation} L_t U_s = U_s L_t  \quad \text{for all } s, t
    \in \mathbb{R}
  \end{equation}
  The random spectral measure $\Phi$ is uniquely determined by $X$ via $\Phi
  (B) = (E (B) X) (0)$ for all Borel $B$.
\end{theorem}

\begin{proof}
  This is Mandrekar's characterization theorem {\cite{mandrekar1972}}. We
  outline the key steps:
  
  Forward direction: Given oscillatory $Z (t)$ as in (\ref{eq:Z_oscillatory}),
  define the stationary curve
  \begin{equation}
    \label{eq:X_from_Z} X (t) = \int_{\mathbb{R}} e^{i \lambda t} d \Phi
    (\lambda)
  \end{equation}
  By Stone's theorem, there exists a unitary shift group $\{U_s \}$ and
  spectral measure $E$ such that $X (t) = U_t X (0)$ and
  \begin{equation}
    \label{eq:X_spectral} X (t) = \int_{\mathbb{R}} e^{i \lambda t} E (d
    \lambda) X (0)
  \end{equation}
  with $\Phi (B) = E (B) X (0)$. Define the operator as in
  (\ref{eq:Lt_operator}). By Dunford-Schwartz spectral theory, each $L_t$ is a
  closed operator with domain containing $\{X (s) : s \in \mathbb{R}\}$. The
  commutation relation (\ref{eq:shift_commutation}) follows from $U_s E (B) =
  E (B) U_s$ for all Borel $B$. Computing:
  \begin{equation}
    \label{eq:LtX_computation} \begin{array}{ll}
      L_t X (0) & = \int A_t (\lambda) e^{i \lambda t} E (d \lambda) X (0)\\
      & = \int A_t (\lambda) e^{i \lambda t} d \Phi (\lambda) = Z (t)
    \end{array}
  \end{equation}
  Reverse direction: If $Z (t) = L_t X (0)$ where $X$ is stationary and $L_t
  U_s = U_s L_t$, then by the Stone-von Neumann theorem on commutants of
  unitary groups, there exists a Borel measurable function $A_t (\cdummy)$
  such that (\ref{eq:Lt_operator}) holds. The domain condition $\{X (s) : s
  \in \mathbb{R}\} \subseteq D (L_t)$ implies
  \begin{equation}
    \label{eq:domain_condition} \int_{\mathbb{R}} |A_t (\lambda) |^2  \|E (d
    \lambda) X (0)\|^2 < \infty
  \end{equation}
  for each $t$, giving $A_t \in L^2 (F)$ where $dF (\lambda) = \|E (d \lambda)
  X (0)\|^2$. This yields the oscillatory representation.
\end{proof}

\begin{remark}
  \label{rem:generality}(Generality of the stationary representation) Theorem
  \ref{thm:mandrekar_characterization} establishes that every oscillatory
  process is a deformed stationary curve in the sense of Mandrekar
  {\cite{mandrekar1972}}. The key requirement is shift-commutation
  (\ref{eq:shift_commutation}). Unitarily time-changed processes arise as a
  particular explicit subclass where $A_t (\lambda) = \sqrt{\dot{\theta} (t)}
  e^{i \lambda (\theta (t) - t)}$. The theorem guarantees that for any choice
  of gain function $A_t (\lambda) \in L^2 (F)$, there exists an underlying
  stationary process and family of operators recovering the oscillatory
  process.
\end{remark}

\begin{definition}
  \label{def:unitarily_time_changed}(Unitarily time-changed stationary
  process) Let $X = \{X (t)\}_{t \in \mathbb{R}}$ be a second-order stationary
  process with sample paths in $L^2_{\mathrm{loc}} (\mathbb{R})$. Let $\theta$
  satisfy Theorem \ref{thm:unitary_time_change}. Define:
  \begin{equation}
    \label{eq:Z_time_changed} Z (t) \assign (U_{\theta} X) (t) =
    \sqrt{\dot{\theta} (t)} X (\theta (t))
  \end{equation}
  Then $Z$ is called a unitarily time-changed stationary process.
\end{definition}

\begin{lemma}
  \label{lem:exact_recovery}(Exact recovery of $X$) If $Z$ is defined as in
  (\ref{eq:Z_time_changed}), then:
  \begin{equation}
    \label{eq:X_recovery} X = U_{\theta}^{- 1} Z
  \end{equation}
\end{lemma}

\begin{proof}
  This is precisely (\ref{eq:inverse_identities}) from Theorem
  \ref{thm:unitary_time_change}.
\end{proof}

\subsection{Stationary to Oscillatory}

\begin{theorem}
  \label{thm:time_change_oscillatory}(Unitary time change produces oscillatory
  process) Let $X$ be zero-mean stationary with spectral representation
  (\ref{eq:cramer_spectral}). Let $\theta$ satisfy Theorem
  \ref{thm:unitary_time_change}. Define $Z (t)$ as in
  (\ref{eq:Z_time_changed}). Then $Z$ is an oscillatory process with
  oscillatory function:
  \begin{equation}
    \label{eq:phi_t_time_change} \begin{array}{ll}
      \phi_t (\lambda) & = A_t (\lambda) e^{i \lambda t}\\
      & = \sqrt{\dot{\theta} (t)} e^{i \lambda (\theta (t) - t)} e^{i \lambda
      t}\\
      & = \sqrt{\dot{\theta} (t)} e^{i \lambda \theta (t)}
    \end{array}
  \end{equation}
  where the gain function is:
  \begin{equation}
    \label{eq:gain_time_change} A_t (\lambda) = \sqrt{\dot{\theta} (t)} e^{i
    \lambda (\theta (t) - t)}
  \end{equation}
\end{theorem}

\begin{proof}
  Substituting $t \mapsto \theta (t)$ in (\ref{eq:cramer_spectral}):
  \begin{equation}
    \label{eq:Z_substitution} \begin{array}{ll}
      Z (t) & = \sqrt{\dot{\theta} (t)}  \int_{\mathbb{R}} e^{i \lambda \theta
      (t)} d \Phi (\lambda)\\
      & = \int_{\mathbb{R}} \left( \sqrt{\dot{\theta} (t)} e^{i \lambda
      \theta (t)} \right) d \Phi (\lambda)
    \end{array}
  \end{equation}
  Thus $\phi_t (\lambda) = \sqrt{\dot{\theta} (t)} e^{i \lambda \theta (t)}$
  and $A_t (\lambda) = \sqrt{\dot{\theta} (t)} e^{i \lambda (\theta (t) - t)}$
  since $\phi_t (\lambda) = A_t (\lambda) e^{i \lambda t}$ by
  (\ref{eq:oscillatory_function}).
\end{proof}

\begin{corollary}
  \label{cor:EPSD_time_change}(EPSD for the unitary time change) If $dF
  (\lambda) = S (\lambda) d \lambda$, then:
  \begin{equation}
    \label{eq:EPSD_time_change} S_t (\lambda) = |A_t (\lambda) |^2 S (\lambda)
    = \dot{\theta} (t) S (\lambda)
  \end{equation}
\end{corollary}

\begin{proof}
  From (\ref{eq:gain_time_change}):
  \begin{equation}
    \label{eq:gain_squared} |A_t (\lambda) |^2 = \dot{\theta} (t) |e^{i
    \lambda (\theta (t) - t)} |^2 = \dot{\theta} (t)
  \end{equation}
\end{proof}

\section{Zero Localization}

\subsection{Kac-Rice Formula}

\begin{theorem}
  \label{thm:kac_rice}(Generalized Kac-Rice formula) Let $Z (t)$ be a
  real-valued, zero-mean Gaussian process with covariance $K (t, s)
  =\mathbb{E} [Z (t) Z (s)]$. Assume $K (t, t) > 0$ and that $K (t, s)$ is
  twice continuously differentiable in a neighborhood of $(t, t)$. Define:
  \begin{equation}
    \label{eq:K_derivatives} K (t) \assign K (t, t), \quad K_s (t) \assign
    \left. \frac{\partial K (t, s)}{\partial s} \right|_{s = t}, \quad K_{ss}
    (t) \assign \left. \frac{\partial^2 K (t, s)}{\partial s^2} \right|_{s =
    t}
  \end{equation}
  Assume
  \begin{equation}
    \label{eq:V_condition} V (t) \assign K (t) K_{ss} (t) - [K_s (t)]^2 > 0
  \end{equation}
  for $t \in [a, b]$. Then:
  \begin{equation}
    \label{eq:kac_rice_formula} \mathbb{E} [N_{[a, b]}] = \int_a^b
    \frac{1}{\pi}  \sqrt{\frac{V (t)}{K (t)^2}}  \hspace{0.17em} dt
  \end{equation}
\end{theorem}

\begin{proof}
  The joint density of $(Z (t), \dot{Z} (t))$ is Gaussian with covariance
  matrix $\Sigma (t) = \left(\begin{array}{cc}
    K (t) & K_s (t)\\
    K_s (t) & K_{ss} (t)
  \end{array}\right)$. The Kac-Rice formula gives:
  \begin{equation}
    \label{eq:kac_rice_derivation} \begin{array}{ll}
      \mathbb{E} [N_{[a, b]}] & = \int_a^b \mathbb{E} [| \dot{Z} (t) | \mid Z
      (t) = 0] p_{Z (t)} (0)  \hspace{0.17em} dt\\
      & = \int_a^b \frac{1}{\sqrt{2 \pi K (t)}}  \sqrt{\frac{2}{\pi}  \frac{K
      (t) K_{ss} (t) - K_s (t)^2}{K (t)^2}}  \hspace{0.17em} dt
    \end{array}
  \end{equation}
  Simplifying yields (\ref{eq:kac_rice_formula}).
\end{proof}

\subsubsection{Kac-Rice Formula for Unitarily Time-Changed Processes}

\begin{theorem}
  \label{thm:kac_rice_time_change}(Kac-Rice formula for unitary time change)
  Let $X (t)$ be a zero-mean, stationary Gaussian process with covariance $R
  (h) =\mathbb{E} [X (t) X (t + h)]$ satisfying $R (0) > 0$ and $R'' (0) < 0$.
  
  Let $\theta : \mathbb{R} \to \mathbb{R}$ be absolutely continuous, strictly
  increasing, and bijective with $\dot{\theta} (t) > 0$ almost everywhere.
  Define the unitarily time-changed process:
  \begin{equation}
    Z (t) = \sqrt{\dot{\theta} (t)} X (\theta (t))
  \end{equation}
  Then the expected zero-crossing density of $Z$ at time $t$ is:
  \begin{equation}
    \frac{d\mathbb{E} [N (t)]}{dt} = \frac{\dot{\theta} (t)}{\pi} 
    \sqrt{\frac{- R'' (0)}{R (0)}}
  \end{equation}
  Equivalently, the expected number of zeros of $Z$ in the interval $[a, b]$
  is:
  \begin{equation}
    \mathbb{E} [N ([a, b])] = \frac{\theta (b) - \theta (a)}{\pi} 
    \sqrt{\frac{- R'' (0)}{R (0)}}
  \end{equation}
\end{theorem}

\begin{proof}
  \ 
\end{proof}

\subsubsection{Evolutionary Power Spectral Density as Factorized Spectrum}

\begin{corollary}
  \label{cor:EPSD_factorization}(Factorization of evolutionary spectrum) For
  the unitarily time-changed process $Z (t) = \sqrt{\dot{\theta} (t)} X
  (\theta (t))$ with stationary spectral density $S (\lambda)$, the
  evolutionary power spectral density factorizes as:
  \[ S_t (\lambda) = \dot{\theta} (t) \cdot S (\lambda) \]
  Time-dependence and frequency-dependence separate completely: the spectral
  energy density at time $t$ and frequency $\lambda$ is the product of the
  instantaneous time-dilation $\dot{\theta} (t)$ and the base spectral density
  $S (\lambda)$.
\end{corollary}

\begin{proof}
  The gain function is $A_t (\lambda) = \sqrt{\dot{\theta} (t)} e^{i \lambda
  (\theta (t) - t)}$, so:
  \[ |A_t (\lambda) |^2 = \dot{\theta} (t) \]
  By definition:
  \[ S_t (\lambda) = |A_t (\lambda) |^2 S (\lambda) = \dot{\theta} (t) \cdot S
     (\lambda) \]
\end{proof}

\subsection{Bulinskaya's Theorem}

\begin{theorem}
  \label{thm:bulinskaya}(Bulinskaya) Let $X (t)$ be a real-valued, zero-mean
  stationary Gaussian process with covariance $R (h) =\mathbb{E} [X (t) X (t +
  h)]$. If $R$ is twice continuously differentiable in a neighborhood of 0 and
  $R'' (0) < 0$, then with probability 1 all zeros of $X$ are simple.
\end{theorem}

\begin{proof}
  For fixed $t_0$, $(X (t_0), \dot{X} (t_0))$ is jointly Gaussian.
  Stationarity gives $\mathbb{E} [X (t_0) \dot{X} (t_0)] = R' (0) = 0$, so
  they are independent. Since $R'' (0) < 0$, $\dot{X} (t_0)$ is non-degenerate
  and $\mathbb{P} (\dot{X} (t_0) = 0) = 0$. Thus $\mathbb{P} (X (t_0) = 0
  \text{and } \dot{X} (t_0) = 0) = 0$. By continuity and countable union over
  rationals, all zeros are simple almost surely.
\end{proof}

\section{Example: The Hardy Z-Function}

This section demonstrates that the Hardy Z-function is a concrete instance of
a unitarily time-changed stationary process. We prove that the transformed
process, when expressed via the inverse unitary operator, possesses a
well-defined stationary covariance structure in the Ces{\`a}ro sense.

\subsection{Definitions}

\begin{definition}
  \label{def:hardy_Z}(Hardy Z-function) Let $\zeta (s)$ be the Riemann zeta
  function and let $\theta (t)$ denote the Riemann-Siegel theta function:
  \begin{equation}
    \label{eq:theta_def} \theta (t) = \Im \log \Gamma \left( \frac{1}{4} +
    \frac{it}{2} \right) - \frac{t}{2} \log \pi
  \end{equation}
  Define:
  \begin{equation}
    \label{eq:hardy_Z} Z (t) = e^{i \theta (t)} \zeta (1 / 2 + it)
  \end{equation}
\end{definition}

\begin{definition}
  \label{def:monotonized_theta}(Monotonized theta time change) Let $a > 0$ be
  the unique critical point of $\theta$ in $(0, \infty)$ where $\dot{\theta}
  (a) = 0$. Define $\Theta : [0, \infty) \to [\Theta (0), \infty)$ by:
  \begin{equation}
    \label{eq:Theta_def} \Theta (t) = \left\{ \begin{array}{ll}
      2 \theta (a) - \theta (t) & 0 \leq t \leq a\\
      \theta (t) & t \geq a
    \end{array} \right.
  \end{equation}
\end{definition}

\subsection{Unitary Time Change Representation}

We apply the unitary time-change operator $U_{\Theta}$ from Theorem
\ref{thm:unitary_time_change} to reveal the underlying stationary structure.

\begin{definition}
  \label{def:X_hardy}(Underlying stationary process) Define the process $X$
  via the inverse unitary transform $X = U_{\Theta}^{- 1} Z$:
  \begin{equation}
    \label{eq:X_hardy} X (u) = (U_{\Theta}^{- 1} Z) (u) = \frac{Z (\Theta^{-
    1} (u))}{\sqrt{\Theta' (\Theta^{- 1} (u))}}
  \end{equation}
  for $u \in [\Theta (0), \infty)$.
\end{definition}

By Lemma \ref{lem:exact_recovery}, we have the exact reconstruction:
\begin{equation}
  \label{eq:Z_reconstruction} Z (t) = (U_{\Theta} X) (t) = \sqrt{\Theta' (t)}
  X (\Theta (t))
\end{equation}
which is precisely the form of a unitarily time-changed process from
Definition \ref{def:unitarily_time_changed}.

\subsubsection{Stationarity}

\begin{lemma}
  \label{lem:van_der_corput}(van der Corput lemma) Let $\phi : [a, b] \to
  \mathbb{R}$ be continuously differentiable. If $| \phi' (x) | \geq \lambda >
  0$ for all $x \in [a, b]$, then:
  \begin{equation}
    \left| \int_a^b e^{i \phi (x)} dx \right| \leq \frac{4}{\lambda}
  \end{equation}
  In particular, $\left| \int_a^b \cos (\phi (x)) dx \right| = O (1 /
  \lambda)$ when $| \phi' (x) | \geq \lambda$.
\end{lemma}

\begin{theorem}[Ces{\`a}ro covariance convergence]
  \label{thm:cesaro_hardy}For the process $X (u)$ defined in
  (\ref{eq:X_hardy}), the Ces{\`a}ro covariance
  \begin{equation}
    \label{eq:cesaro_cov} C (h) = \lim_{U \to \infty}  \frac{1}{U - \Theta
    (0)}  \int_{\Theta (0)}^U X (u) X (u + h)  \hspace{0.17em} du
  \end{equation}
  exists for all $h \in \mathbb{R}$ and is independent of the starting point.
  This establishes that $X$ is a wide-sense stationary process in the
  Ces{\`a}ro sense, and consequently $Z$ is a unitarily time-changed
  oscillatory process.
\end{theorem}

\begin{proof}
  The proof relies on explicit asymptotic analysis of the Riemann-Siegel
  representation of $Z (t)$.
  
  \subsubsection*{Step 1: Asymptotic expansion of $\Theta' (t)$}
  
  Starting from the definition (\ref{eq:theta_def}), apply Stirling's formula
  for $\log \Gamma (z)$:
  \begin{equation}
    \log \Gamma (z) = \left( z - \frac{1}{2} \right) \log z - z + \frac{1}{2}
    \log (2 \pi) + O (|z|^{- 1})
  \end{equation}
  For $z = \frac{1}{4} + \frac{it}{2}$ with $t \to \infty$, compute:
  \begin{equation}
    |z| = \sqrt{\frac{1}{16} + \frac{t^2}{4}} = \frac{t}{2}  (1 + O (t^{- 2}))
  \end{equation}
  \begin{equation}
    \arg z = \arctan (2 t) = \frac{\pi}{2} - \frac{1}{2 t} + O (t^{- 3})
  \end{equation}
  Therefore:
  \begin{equation}
    \log z = \log \frac{t}{2} + i \left( \frac{\pi}{2} - \frac{1}{2 t} + O
    (t^{- 3}) \right)
  \end{equation}
  Computing $(z - 1 / 2) \log z$ and taking the imaginary part yields:
  \begin{equation}
    \Im [(z - 1 / 2) \log z] = \frac{t}{2} \log \frac{t}{2} - \frac{\pi}{8} -
    \frac{t \pi}{4} + O (t^{- 1})
  \end{equation}
  Combining with the $- \frac{t}{2} \log \pi$ term:
  \begin{equation}
    \label{eq:theta_asymptotic} \theta (t) = \frac{t}{2} \log \frac{t}{2 \pi}
    - \frac{t}{2} - \frac{\pi}{8} + O (t^{- 1})
  \end{equation}
  Differentiating term by term:
  \begin{equation}
    \theta' (t) = \frac{1}{2} \log \frac{t}{2 \pi} + \frac{1}{2} \cdot
    \frac{t}{t} - \frac{1}{2} + O (t^{- 2})
  \end{equation}
  Simplifying:
  \begin{equation}
    \label{eq:theta_prime} \theta' (t) = \frac{1}{2} \log \frac{t}{2 \pi} + O
    (t^{- 1})
  \end{equation}
  For $t \geq a$, $\Theta (t) = \theta (t)$, so $\Theta' (t)$ has the same
  asymptotic.
  
  \tmtextbf{Key consequence:} For any fixed $n$,
  \begin{equation}
    \label{eq:log_ratio_vanishes} \frac{\log n}{\Theta' (t)} = \frac{2 \log
    n}{\log (t / (2 \pi))} \to 0 \quad \text{as } t \to \infty
  \end{equation}
  
  \subsubsection*{Step 2: Riemann-Siegel representation}
  
  The Hardy Z-function admits the Riemann-Siegel expansion:
  \begin{equation}
    \label{eq:RS_expansion} Z (t) = 2 \sum_{n = 1}^{N (t)} n^{- 1 / 2} \cos
    (\theta (t) - t \log n) + R (t)
  \end{equation}
  where $N (t) = \lfloor \sqrt{t / (2 \pi)} \rfloor$ and the remainder
  satisfies $R (t) = O (t^{- 1 / 4})$.
  
  Transforming to $u$-coordinates with $t = \Theta^{- 1} (u)$, define:
  \begin{equation}
    \Phi_n (u) = u - \Theta^{- 1} (u) \log n
  \end{equation}
  Then:
  \begin{equation}
    X (u) = \frac{2}{\sqrt{\Theta' (\Theta^{- 1} (u))}}  \sum_{n = 1}^{N
    (\Theta^{- 1} (u))} n^{- 1 / 2} \cos (\Phi_n (u)) + \frac{R (\Theta^{- 1}
    (u))}{\sqrt{\Theta' (\Theta^{- 1} (u))}}
  \end{equation}
  
  \subsubsection*{Step 3: Diagonal terms remain bounded}
  
  Consider the product $X (u) X (u + h)$. For diagonal terms ($n = m$):
  \begin{equation}
    \cos (\Phi_n (u)) \cos (\Phi_n (u + h)) = \frac{1}{2}  [\cos (\Phi_n (u) -
    \Phi_n (u + h)) + \cos (\Phi_n (u) + \Phi_n (u + h))]
  \end{equation}
  \tmtextbf{Phase difference:}
  \begin{equation}
    \Phi_n (u) - \Phi_n  (u + h) = - h + [\Theta^{- 1} (u + h) - \Theta^{- 1}
    (u)] \log n
  \end{equation}
  By the Mean Value Theorem, for some $\xi_u \in (u, u + h)$:
  \begin{equation}
    \Theta^{- 1}  (u + h) - \Theta^{- 1} (u) = \frac{h}{\Theta' (\Theta^{- 1}
    (\xi_u))}
  \end{equation}
  Therefore:
  \begin{equation}
    [\Theta^{- 1} (u + h) - \Theta^{- 1} (u)] \log n = \frac{h \log n}{\Theta'
    (\Theta^{- 1} (\xi_u))}
  \end{equation}
  By (\ref{eq:log_ratio_vanishes}), as $u \to \infty$ (so $\Theta^{- 1}
  (\xi_u) \to \infty$):
  \begin{equation}
    \frac{h \log n}{\Theta' (\Theta^{- 1} (\xi_u))} \to 0
  \end{equation}
  Hence:
  \begin{equation}
    \label{eq:phase_diff_limit} \Phi_n (u) - \Phi_n  (u + h) \to - h
  \end{equation}
  The diagonal oscillatory term $\cos (\Phi_n (u) - \Phi_n (u + h))$ remains
  bounded by 1.
  
  \tmtextbf{Phase sum:} The sum $\Phi_n (u) + \Phi_n  (u + h) = 2 u + h -
  \Theta^{- 1} (u) \log n - \Theta^{- 1}  (u + h) \log n$ has derivative:
  \begin{equation}
    \frac{d}{du}  [\Phi_n (u) + \Phi_n (u + h)] = 2 - \frac{\log n}{\Theta'
    (\Theta^{- 1} (u))} - \frac{\log n}{\Theta'  (\Theta^{- 1} (u + h))}
  \end{equation}
  By (\ref{eq:log_ratio_vanishes}), both reciprocal terms vanish as $u \to
  \infty$, so:
  \begin{equation}
    \frac{d}{du}  [\Phi_n (u) + \Phi_n (u + h)] \to 2
  \end{equation}
  For sufficiently large $u > U_0$, we have $\left| \frac{d}{du} [\Phi_n (u) +
  \Phi_n (u + h)] \right| \geq 1$.
  
  By van der Corput's lemma (Lemma \ref{lem:van_der_corput}):
  \begin{equation}
    \left| \int_{U_0}^U \cos (\Phi_n (u) + \Phi_n (u + h)) \hspace{0.17em} du
    \right| = O (1)
  \end{equation}
  Therefore, the Ces{\`a}ro contribution from the phase sum:
  \begin{equation}
    \frac{1}{U}  \int_{\Theta (0)}^U \cos (\Phi_n (u) + \Phi_n (u + h)) 
    \hspace{0.17em} du = O (U^{- 1}) \to 0
  \end{equation}
  
  \subsubsection*{Step 4: Off-diagonal terms vanish}
  
  For $n \neq m$, the cross term has phase:
  \begin{equation}
    \Phi_n (u) + \Phi_m  (u + h) = 2 u + h - \Theta^{- 1} (u) \log n -
    \Theta^{- 1}  (u + h) \log m
  \end{equation}
  The derivative is:
  \begin{equation}
    \frac{d}{du}  [\Phi_n (u) + \Phi_m (u + h)] = 2 - \frac{\log n}{\Theta'
    (\Theta^{- 1} (u))} - \frac{\log m}{\Theta'  (\Theta^{- 1} (u + h))} \to 2
  \end{equation}
  Identically to Step 3, van der Corput's lemma applies and:
  \begin{equation}
    \frac{1}{U}  \int_{\Theta (0)}^U \cos (\Phi_n (u) + \Phi_m (u + h)) 
    \hspace{0.17em} du = O (U^{- 1}) \to 0
  \end{equation}
  
  \subsubsection*{Step 5: Remainder terms vanish}
  
  The weight factor in the transformation is:
  \begin{equation}
    W (u, h) = \frac{1}{\sqrt{\Theta' (\Theta^{- 1} (u)) \Theta'  (\Theta^{-
    1} (u + h))}} \sim \frac{1}{\log (\Theta^{- 1} (u))}
  \end{equation}
  The sum $\sum_{n = 1}^{N (t)} n^{- 1 / 2} \cos (\Phi_n (u))$ is bounded by
  $O (\sqrt{N (t)}) = O (t^{1 / 4})$ where $t = \Theta^{- 1} (u)$.
  
  Cross terms involving the remainder $R (\Theta^{- 1} (u + h)) = O (t^{- 1 /
  4})$ give:
  \begin{equation}
    W (u, h) \cdot O (t^{1 / 4}) \cdot O (t^{- 1 / 4}) = O ((\log t)^{- 1})
  \end{equation}
  Changing to $t$-coordinates with $u = \Theta (t)$ and $du = \Theta' (t) dt$:
  \begin{equation}
    \frac{1}{U}  \int_{t_0}^{t_1} \frac{\Theta' (t)}{\log t}  \hspace{0.17em}
    dt \sim \frac{1}{U}  \int_{t_0}^{t_1} \frac{\frac{1}{2} \log t}{\log t} 
    \hspace{0.17em} dt = \frac{t_1 - t_0}{2 U}
  \end{equation}
  Since $U = \Theta (t_1) - \Theta (t_0) \sim \frac{t_1}{2} \log t_1$ for
  large $t_1$:
  \begin{equation}
    \frac{t_1 - t_0}{2 U} \sim \frac{t_1}{t_1 \log t_1} = (\log t_1)^{- 1} \to
    0
  \end{equation}
  Similarly, $R (\Theta^{- 1} (u)) R (\Theta^{- 1} (u + h)) = O (t^{- 1 / 2})$
  gives Ces{\`a}ro average $O (t^{- 1 / 2}) \to 0$.
  
  \subsubsection*{Step 6: Independence of starting point}
  
  For any bounded integrable function $f$ and starting points $u_0,
  \tilde{u}_0 \geq \Theta (0)$:
  \begin{equation}
    \left| \frac{1}{U}  \int_{u_0}^{u_0 + U} f \hspace{0.17em} du -
    \frac{1}{U}  \int_{\tilde{u}_0}^{\tilde{u}_0 + U} f \hspace{0.17em} du
    \right| \leq \frac{2 | \tilde{u}_0 - u_0 | \sup |f|}{U} \to 0
  \end{equation}
  
  \subsubsection*{Conclusion}
  
  Combining Steps 3--6, the Ces{\`a}ro covariance limit
  \begin{equation}
    C (h) = \lim_{U \to \infty}  \frac{1}{U - \Theta (0)}  \int_{\Theta (0)}^U
    X (u) X (u + h)  \hspace{0.17em} du
  \end{equation}
  exists and is independent of the starting point $\Theta (0)$. This
  establishes that $X (u)$ is wide-sense stationary in the Ces{\`a}ro sense.
\end{proof}

\begin{corollary}
  \label{cor:hardy_oscillatory_structure}The Hardy Z-function is a unitarily
  time-changed stationary process $Z = U_{\Theta} X$, where $X$ is the
  Ces{\`a}ro-stationary process characterized by Theorem
  \ref{thm:cesaro_hardy}. Therefore, $Z$ is an oscillatory process with
  evolutionary power spectral density
  \begin{equation}
    \label{eq:hardy_epsd} S_t (\lambda) = \Theta' (t) S_X (\lambda)
  \end{equation}
  where $S_X (\lambda)$ is the spectral density of $X$.
\end{corollary}

\begin{remark}
  The convergence of the Ces{\`a}ro covariance rigorously establishes that the
  Hardy Z-function, when viewed through theta-time coordinates, admits a
  well-defined stationary structure. The explicit form of $C (h)$ encodes the
  deep spectral properties of the Riemann zeta function and requires detailed
  harmonic analysis of the Riemann-Siegel coefficients.
\end{remark}

\begin{thebibliography}{99}
  {\bibitem{cramerLeadbetter1967}}Harald Cram{\'e}r and M. R. Leadbetter.
  Stationary and Related Processes: Sample Function Properties and Their
  Applications. Wiley, 1967.
  
  {\bibitem{mandrekar1972}}V. Mandrekar. A characterization of oscillatory
  processes and their prediction. Proc. Amer. Math. Soc., 32(1):280--284,
  1972.
  
  {\bibitem{priestley1965}}Maurice B. Priestley. Evolutionary spectra and
  non-stationary processes. J. R. Stat. Soc. B, 27(2):204--229, 1965.
  
  {\bibitem{priestley1981}}Maurice B. Priestley. Spectral Analysis and Time
  Series. Academic Press, 1981.
  
  {\bibitem{bulinskaya1961}}E. V. Bulinskaya. On the mean number of crossings
  of a level by a stationary Gaussian process. Theory Probab. Appl.,
  6(4):435--438, 1961.
\end{thebibliography}

\end{document}
