\documentclass{article}
\usepackage[english]{babel}
\usepackage{geometry,amsmath}
\geometry{letterpaper}

%%%%%%%%%% Start TeXmacs macros
\newcommand{\tmtextbf}[1]{\text{{\bfseries{#1}}}}
%%%%%%%%%% End TeXmacs macros

\begin{document}

\section*{Definition}

A \tmtextbf{current} $T$ of degree $k$ on a manifold $M$ is a continuous
linear functional on the space of compactly supported smooth differential
forms of complementary degree $n - k$ (where $n$ is the dimension of the
manifold). This is expressed as:
\begin{equation}
  T (\omega)
\end{equation}
for a compactly supported smooth $(n - k)$-form $\omega$ on $M$.

\section*{Properties and Operations}

\begin{enumerate}
  \item \tmtextbf{Boundary of a Current:} If $T$ is a $k$-current, its
  boundary $\partial T$ is a $(k - 1)$-current defined by:
  \begin{equation}
    \partial T (\omega) = T (d \omega)
  \end{equation}
  where $d \omega$ is the exterior derivative of $\omega$.
  
  \item \tmtextbf{Pushforward and Pullback:} Given a smooth map $f : M \to N$,
  the pushforward of a current can be defined from $M$ to $N$. The pullback,
  however, is generally not well-defined for currents.
  
  \item \tmtextbf{Integration of Currents:} A $n$-dimensional current $T$ on
  $M$ can be integrated over $M$, represented by:
  \begin{equation}
    \int_M T
  \end{equation}
\end{enumerate}

\section*{Examples of Currents}

\begin{enumerate}
  \item \tmtextbf{Dirac Delta Current:} For a point $p$ in $M$, the Dirac
  delta current $\delta_p$ of degree $n$ acts on a $n$-form $\omega$ by:
  \begin{equation}
    \delta_p (\omega) = \omega (p)
  \end{equation}
  \item \tmtextbf{Integration Along a Submanifold:} Let $S$ be a
  $k$-dimensional oriented smooth submanifold of $M$. The current $[S]$
  associated with $S$ acts on a $(n - k)$-form $\omega$ as:
  \begin{equation}
    [S] (\omega) = \int_S \omega
  \end{equation}
\end{enumerate}

\section*{Application in Equidistribution}

Currents can be used to study the asymptotic distribution of zeros of
holomorphic sections in Hermitian vector bundles, particularly how these zeros
distribute themselves across the manifold, often converging to a certain
limiting current.

\end{document}
