\documentclass{article}

\usepackage{amsmath, amssymb, amsthm}
\newtheorem{theorem}{Theorem}
\newtheorem{corollary}{Corollary}
\newtheorem{definition}{Definition}
\newtheorem{example}{Example}

\title{Sample Path Relationships in Monotonically Modulated Gaussian Processes: \\ Structural and Analytical Transformations}
\author{}
\date{}

\begin{document}
\maketitle

\begin{abstract}
This work establishes explicit relationships between sample paths of original and modulated Gaussian processes through time-warping transformations. Key contributions include: pathwise differentiability transfer under modulation, Hölder continuity propagation laws, extreme value correlations, and concrete quantitative comparisons of covariance gradients. Methodologies combine operator conjugation techniques with stochastic Taylor expansions.
\end{abstract}

\section{Temporal Warping Mechanics} 

\begin{definition}[Modulated Process]
Let $X(t)$ be a Gaussian process with stationary covariance $K(|t-s|)$. Under $\theta \in C^1(\mathbb{R})$ strictly increasing, define the modulated process:
\begin{equation*}
Y(t) = X(\theta(t))\sqrt{\dot{\theta}(t)}
\end{equation*}
with transformed covariance:
\begin{equation*}
\mathbb{E}[Y(t)Y(s)] = K(|\theta(t)-\theta(s)|)\sqrt{\dot{\theta}(t)\dot{\theta}(s)}
\end{equation*}
\end{definition}

\section{Pathwise Differentiability Transfer}

\begin{theorem}[Differentiated Sample Path Relationship]
If $X(t)$ has $C^k$-sample paths and $\theta \in C^{k+1}(\mathbb{R})$, then $Y(t)$ is $C^k$ with derivatives:
\begin{equation*}
Y^{(m)}(t) = \sum_{k=0}^m \binom{m}{k} X^{(k)}(\theta(t)) \cdot \frac{d^k}{dt^k}\sqrt{\dot{\theta}(t)}
\end{equation*}
\end{theorem}

\begin{proof}
By recursive application of Faà di Bruno's formula for higher-order chain rules. The base case ($m=1$) gives:
\begin{align*}
Y'(t) &= \frac{d}{dt}\left[X(\theta(t))\sqrt{\dot{\theta}(t)}\right] \\
&= X'(\theta(t))\dot{\theta}(t)\sqrt{\dot{\theta}(t)} + X(\theta(t))\frac{\ddot{\theta}(t)}{2\sqrt{\dot{\theta}(t)}} \\
&= X'(\theta(t))[\dot{\theta}(t)]^{3/2} + X(\theta(t))\frac{\ddot{\theta}(t)}{2\sqrt{\dot{\theta}(t)}}
\end{align*}
The general form follows by induction on $m$ through Leibniz differentiation.
\end{proof}

\section{Oscillatory Behavior Analysis}

\subsection{Instantaneous Frequency Modulation}

\begin{theorem}[Quadratic Covariance Gradient]
The modulated process satisfies:
\begin{equation*}
\partial^2_{ts} \mathbb{E}[Y(t)Y(s)]|_{t=s} = -\ddot{K}(0)[\dot{\theta}(t)]^2 + \dot{K}(0)\ddot{\theta}(t)
\end{equation*}
\end{theorem}

\begin{proof}
Direct computation using covariance expression:
\begin{align*}
\partial_t \mathbb{E}[Y(t)Y(s)] &= \dot{K}(|\theta(t)-\theta(s)|)\dot{\theta}(t)\text{sgn}(t-s)\sqrt{\dot{\theta}(t)\dot{\theta}(s)} \\
&\quad + \frac{1}{2}K(|\theta(t)-\theta(s)|)\frac{\ddot{\theta}(t)}{\sqrt{\dot{\theta}(t)\dot{\theta}(s)}}
\end{align*}
Evaluating the limit as $s \to t$ yields second-order behavior dominated by $\ddot{K}(0)$-term.
\end{proof}

\section{Extreme Value Correlations}

\begin{theorem}[Maxima Coupling]
Let $M_X(I) = \sup_{t \in I}X(t)$ for interval $I$. Then:
\begin{equation*}
\mathbb{P}(M_Y([a,b]) > u) = \mathbb{P}\left(M_X([\theta(a),\theta(b)]) > u/\sqrt{\dot{\theta}\circ\theta^{-1}(s)}\right)
\end{equation*}
\end{theorem}

\begin{proof}
Through time-change $s = \theta(t)$, the supremum transforms as:
\begin{equation*}
\sup_{t \in [a,b]} Y(t) = \sup_{s \in [\theta(a),\theta(b)]} X(s)\sqrt{\dot{\theta}(\theta^{-1}(s))}
\end{equation*}
Result follows from measure preservation under diffeomorphism.
\end{proof}

\section{Practical Modulation Examples}

\subsection{Power Law Modulation}

\begin{example}[Polynomial Time Warp]
For $\theta(t) = t^\alpha$ with $\alpha > 1$:
\begin{equation*}
\mathbb{E}[N_Y([0,T])] = \sqrt{-\ddot{K}(0)}(T^\alpha - 0) = \mathcal{O}(T^\alpha)
\end{equation*}
Zero crossings grow polynomially rather than linearly.
\end{example}

\subsection{Discontinuous Modulation}

\begin{theorem}[Piecewise Constant Derivative]
Let $\dot{\theta}(t) = \begin{cases} v_1 & t < t_0 \\ v_2 & t \geq t_0 \end{cases}$. Then:
\begin{equation*}
\mathbb{E}[Y(t)Y(s)] = \begin{cases} 
K(|\theta(t)-\theta(s)|)v_1 & t,s < t_0 \\
K(|\theta(t)-\theta(s)|)\sqrt{v_1v_2} & t < t_0 \leq s \\
K(|\theta(t)-\theta(s)|)v_2 & t,s \geq t_0
\end{cases}
\end{equation*}
\end{theorem}

\end{document}
