\documentclass{article}
\usepackage[english]{babel}
\usepackage{amssymb}

%


\begin{document}

\section*{Proposition 8}

Let $K : T \times T \to \mathbb{C}$ be a covariance function such that the
associated RKHS $\mathcal{H}_K$ is separable where $T \subset \mathbb{R}$.
Then there exists a family of vector functions $\Psi (t, x) = (\psi_n (t, x),
n \geq 1), t \in T$ and a Borel measure $\mu$ on $T$ such that $\psi_n (t, x)
\in L^2 (T, \mu)$, in terms of which $K$ is representable as:
\[ K (s, t) = \int_T \sum_{n = 1}^{\infty} \psi_n (s, x) \overline{\psi_n (t,
   x)} d \mu (x) \]
The vector functions $\Psi (s, .), s \in T$ and the measure $\mu$ may not be
unique, but all such $(\Psi, .), .)$ determine $K$ and $H_K$ uniquely and the
cardinality of the components determining $K$ remains the same.

\subsection*{Important Remarks:}

1. If $\Psi (t, .)$ is a scalar, then we have $K (s, t) = \int_T \Psi (s, x)
\overline{\Psi (t, x)} d \mu (x)$, which includes the tri-diagonal triangular
covariance with $\mu$ absolutely continuous relative to the Lebesgue measure.

2. The following notational simplification of (25) can be made. Let $n = R
\times Z_+ = S \otimes P$, where $P$ is the power set of integers Z, and let P
= u @ o where o is the counting measure. Then
\[ \Psi (t, n) = (\psi_n (t, x), n \in Z) \]
Hence
\[ | \Psi^{\ast} (t) |^2_{L^2} = \int_T | \psi_n (t, x) |^2 d \mu (x) \]
This content is adapted from MM Rao's book, *Stochastic Processes: Inference
Theory*.

\end{document}
