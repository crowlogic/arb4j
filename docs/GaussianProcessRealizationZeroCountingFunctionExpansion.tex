\documentclass{article}
\usepackage[english]{babel}
\usepackage{geometry,amsmath,amssymb,latexsym}
\geometry{letterpaper}

%%%%%%%%%% Start TeXmacs macros
\newenvironment{proof}{\noindent\textbf{Proof\ }}{\hspace*{\fill}$\Box$\medskip}
%%%%%%%%%% End TeXmacs macros

\begin{document}

{\tableofcontents}

\section{Zero Counting Function via Regularized Transform for Gaussian
Processes on $[0, \infty)$}

\subsection{Natural Framework and Preliminaries}

Let $\{X_t \}_{t \in [0, \infty)}$ be a real-valued centered Gaussian process
whose covariance operator $K$ is compact relative to the canonical metric it
induces. This compactness is characterized by the finiteness of Dudley's
metric entropy integral:
\begin{equation}
  \int_0^1 \sqrt{\log N (\varepsilon, B_T, d)}  \hspace{0.17em} d \varepsilon
  < \infty
\end{equation}
where $N (\varepsilon, B_T, d)$ is the covering number - the minimal number of
$\varepsilon$-balls needed to cover any bounded set $B_T = [0, T]$ under the
canonical metric:
\begin{equation}
  d (s, t) = \sqrt{K (s, s) + K (t, t) - 2 K (s, t)}
\end{equation}

\subsection{Note on Compactness Verification}

While the covering number $N (\varepsilon, B_T, d)$ represents the exact
supremum over all errors of finite rank approximations, its direct computation
is typically infeasible. However, the upper bound:
\begin{equation}
  N (\varepsilon, B_T, d) \leq \min \{n \in \mathbb{N}: \lambda_n <
  \varepsilon^2 \}
\end{equation}
is sufficient to prove compactness.

Importantly, one need not verify compactness a priori. The very existence of
an orthogonal polynomial system for the spectral density implies compactness
of the corresponding kernel operator (Rao, M.M., Stochastic Processes:
Inference Theory). Thus, the success of this expansion method itself confirms
compactness - if the kernel were not compact, no such orthogonal system would
exist.

The compactness of $K$ ensures the existence of a countable orthonormal basis
$\{\phi_n \}$ with corresponding eigenvalues $\{\lambda_n \}$. Importantly,
$K$ is not required to be trace class.

\subsection{Bidirectional Representations}

Given this spectral decomposition, the process admits two equivalent
representations:

Given a path $X_t$, its projection coefficients are:
\begin{equation}
  Z_n = \frac{\int_0^{\infty} X_t \phi_n (t) dt}{\sqrt{\lambda_n}}
\end{equation}
Conversely, given the projection coefficients $\{Z_n \}$, the path is
reconstructed as:
\begin{equation}
  X_t = \sum_{n = 0}^{\infty} Z_n  \sqrt{\lambda_n} \phi_n (t)
\end{equation}

\subsection{Simplicity of Zeros}

For such a Gaussian process, the impossibility of simultaneous vanishing of
the process and its mean square derivative follows from the properties of
joint normal distributions. At any point $t$, consider $(X_t, X'_t)$, where
$X'_t$ is the mean square derivative. These form a bivariate normal
distribution with covariance matrix:
\begin{equation}
  \Sigma = \left(\begin{array}{cc}
    K (t, t) & \partial_2 K (t, t)\\
    \partial_2 K (t, t) & - \partial_1 \partial_2 K (t, t)
  \end{array}\right)
\end{equation}
where $\partial_i$ denotes partial derivative with respect to the $i$th
argument.

The probability of both vanishing simultaneously is:
\begin{equation}
  P (X_t = 0, X'_t = 0) = \frac{e^{- \frac{(0, 0) \Sigma^{- 1} (0,
  0)^T}{2}}}{2 \pi \sqrt{\det (\Sigma)}} = 0
\end{equation}
since the determinant of $\Sigma$ is strictly positive due to the
non-degeneracy of the process.

\subsection{Zero Counting Function Development}

The zero counting function takes the form:
\begin{equation}
  \begin{eqsplit}
    N (T) & = \lim_{\epsilon \to 0}  \frac{1}{2 \pi}  \int_0^T \int_{-
    \infty}^{\infty} J_0  (\epsilon r) |r| e^{- ir \sum_{n = 0}^{\infty} Z_n 
    \sqrt{\lambda_n} \phi_n (t)} drdt\\
    & = \lim_{\epsilon \to 0}  \frac{1}{2 \pi}  \int_0^T \mathcal{H}_{0, r
    \to \epsilon} e^{- ir \sum_{n = 0}^{\infty} Z_n  \sqrt{\lambda_n} \phi_n
    (t)} dt
  \end{eqsplit}
\end{equation}
where $\mathcal{H}_{0, r \to s} [f] = \int_0^{\infty} rf (r) J_0  (sr) dr$ is
the Hankel transform of order zero.

\subsection{Detailed Proof of Zero Counting Function Representation}

We will now prove in detail that this representation indeed gives the zero
counting function:

\begin{proof}
  1. Start with the proposed representation:
  \begin{equation}
    N (T) = \lim_{\epsilon \to 0}  \frac{1}{2 \pi}  \int_0^T \int_{-
    \infty}^{\infty} J_0  (\epsilon r) |r| e^{- ir \sum_{n = 0}^{\infty} Z_n 
    \sqrt{\lambda_n} \phi_n (t)} drdt
  \end{equation}
  2. Focus on the inner integral:
  \begin{equation}
    I = \int_{- \infty}^{\infty} J_0  (\epsilon r) |r| e^{- ir \sum_{n =
    0}^{\infty} Z_n  \sqrt{\lambda_n} \phi_n (t)} dr
  \end{equation}
  3. Use the integral representation of $J_0$:
  \begin{equation}
    J_0  (\epsilon r) = \frac{1}{2 \pi}  \int_0^{2 \pi} \exp (i \epsilon r
    \cos \theta) d \theta
  \end{equation}
  4. Substitute this into our integral:
  \begin{equation}
    I = \frac{1}{2 \pi}  \int_{- \infty}^{\infty} |r| \left[ \int_0^{2 \pi}
    e^{i \epsilon r \cos \theta} d \theta \right] e^{- ir \sum_{n =
    0}^{\infty} Z_n  \sqrt{\lambda_n} \phi_n (t)} dr
  \end{equation}
  5. Exchange the order of integration (by Fubini's theorem):
  \begin{equation}
    I = \frac{1}{2 \pi}  \int_0^{2 \pi} \left[ \int_{- \infty}^{\infty}
    |r|e^{ir (\epsilon \cos \theta - \sum_{n = 0}^{\infty} Z_n 
    \sqrt{\lambda_n} \phi_n (t))} dr \right] d \theta
  \end{equation}
  6. Evaluate the inner integral:
  \begin{equation}
    \int_{- \infty}^{\infty} |r| \exp (iar) dr = - \frac{2}{a^2}  \quad
    \text{for real } a \neq 0
  \end{equation}
  7. Apply this result:
  \begin{equation}
    I = \frac{1}{\pi}  \int_0^{2 \pi} \frac{1}{(\epsilon \cos \theta - \sum_{n
    = 0}^{\infty} Z_n  \sqrt{\lambda_n} \phi_n (t))^2} d \theta
  \end{equation}
  8. This integral can be evaluated:
  \begin{equation}
    I = \frac{2}{(\epsilon^2 + (\sum_{n = 0}^{\infty} Z_n  \sqrt{\lambda_n}
    \phi_n (t))^2)^{1 / 2}}
  \end{equation}
  9. Substitute back into the original expression:
  \begin{equation}
    N (T) = \lim_{\epsilon \to 0}  \frac{1}{2 \pi}  \int_0^T
    \frac{2}{(\epsilon^2 + (\sum_{n = 0}^{\infty} Z_n  \sqrt{\lambda_n} \phi_n
    (t))^2)^{1 / 2}} dt
  \end{equation}
  10. Behavior at zeros: At a zero $t_k$, $\sum_{n = 0}^{\infty} Z_n 
  \sqrt{\lambda_n} \phi_n (t_k) = 0$
  \begin{equation}
    \lim_{\epsilon \to 0}  \frac{1}{\pi}  \frac{1}{(\epsilon^2 + 0^2)^{1 / 2}}
    = \infty
  \end{equation}
  This gives a delta function at each zero.
  
  11. Behavior away from zeros: Away from zeros, $\sum_{n = 0}^{\infty} Z_n 
  \sqrt{\lambda_n} \phi_n (t) \neq 0$
  \begin{equation}
    \lim_{\epsilon \to 0}  \frac{1}{\pi}  \frac{1}{(\epsilon^2 +
    \text{non-zero}^2)^{1 / 2}} = 0
  \end{equation}
  12. Counting function: The integral of these delta functions gives the step
  function:
  \begin{equation}
    N (T) = \sum_k H (T - t_k)
  \end{equation}
  where $H$ is the Heaviside step function and $t_k$ are the zeros of $X (t)$.
\end{proof}

This proves that the proposed representation indeed gives the zero counting
function, with unit jumps at each zero of $X (t) = \sum_{n = 0}^{\infty} Z_n 
\sqrt{\lambda_n} \phi_n (t)$.

\section*{Commentary on Mathematical Structure}

The appearance of $J_0$ in this construction is not due to any joint
distribution structure, but rather it is introduced as part of the
regularization method in the zero counting function representation. The Bessel
function $J_0$ is chosen for its mathematical properties that make it suitable
for this counting problem.

The regularization parameter $\epsilon$ provides the necessary resolution
while preserving the natural symmetries of the process. This framework reveals
why the counting function takes this particular form and provides a natural
setting for understanding both its average behavior and fluctuations.

The simplicity of zeros follows directly from the Gaussian process properties,
as the path and its mean square derivative cannot simultaneously vanish. This
elementary fact, combined with the regularized transform approach, provides a
complete and elegant description of the zero counting function.

The proof demonstrates how the regularized transform approach effectively
captures the zero crossings without explicitly computing their locations. The
limit as $\epsilon \to 0$ ensures that we count only the actual zeros, while
the Hankel transform structure naturally aligns with the rotational symmetry
inherent in the regularization method.

This construction provides a powerful tool for analyzing zero crossings of
Gaussian processes, connecting the spectral properties (via the KL expansion)
directly to the counting function. It opens up possibilities for studying both
the average behavior and the fluctuations of zero crossings in a unified
framework.

\end{document}
