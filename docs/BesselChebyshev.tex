
\documentclass{article}
\usepackage{amsmath}
\usepackage{amssymb}

\begin{document}

\title{Relationship Between Bessel Function Integrals and Chebyshev Polynomials}
\date{}
\maketitle

\section{Introduction}
We begin with the following integral formula:
\begin{equation}
  \int_0^{\infty} J_n (x) e^{- ixy} dx = \frac{(\cos (n \arcsin (y)) - i \sin (n \arcsin (y)))}{\sqrt{1 - y^2}}
\end{equation}
where $J_n (x)$ is the Bessel function of the first kind of order $n$. This formula has a direct connection to Chebyshev polynomials, which we will rigorously derive.

\section{Chebyshev Polynomials}
Recall the definitions of Chebyshev polynomials of the first and second kind:
\begin{align}
  T_n (x) & = \cos (n \arccos (x)) \\
  U_n (x) & = \frac{\sin ((n + 1) \arccos (x))}{\sin (\arccos (x))} 
\end{align}

\section{Key Trigonometric Identity}
We will use the following fundamental trigonometric identity:
\begin{equation}
  \arcsin (y) + \arccos (\sqrt{1 - y^2}) = \frac{\pi}{2}
\end{equation}
This allows us to express $\arcsin (y)$ in terms of $\arccos (\sqrt{1 - y^2})$:
\begin{equation}
  \arcsin (y) = \frac{\pi}{2} - \arccos (\sqrt{1 - y^2})
\end{equation}

\section{Detailed Derivation}
\subsection{Substitution}
Let's substitute this expression into the numerator of the integral result:
\begin{multline}
  \cos (n \arcsin (y)) - i \sin (n \arcsin (y)) \\
  = \cos (n (\frac{\pi}{2} - \arccos (\sqrt{1 - y^2}))) - i \sin (n (\frac{\pi}{2} - \arccos (\sqrt{1 - y^2})))
\end{multline}

Expanding using the angle subtraction formulas:
\begin{multline}
  = \cos (n\frac{\pi}{2}) \cos (n \arccos (\sqrt{1 - y^2})) + \sin (n\frac{\pi}{2}) \sin (n \arccos (\sqrt{1 - y^2})) \\
  - i [\sin (n\frac{\pi}{2}) \cos (n \arccos (\sqrt{1 - y^2})) - \cos (n\frac{\pi}{2}) \sin (n \arccos (\sqrt{1 - y^2}))]
\end{multline}

\subsection{Connecting to Chebyshev Polynomials}
We can express $\sin (n \arccos (x))$ in terms of $U_{n-1}(x)$:
\begin{equation}
  \sin (n \arccos (x)) = \sqrt{1 - x^2} [x U_{n-1}(x) + T_{n-1}(x)]
\end{equation}

Substituting the Chebyshev polynomial definitions and the above expression:
\begin{multline}
  = \cos (n\frac{\pi}{2}) T_n(\sqrt{1 - y^2}) + \sin (n\frac{\pi}{2}) y [\sqrt{1 - y^2} U_{n-1}(\sqrt{1 - y^2}) + T_{n-1}(\sqrt{1 - y^2})] \\
  - i [\sin (n\frac{\pi}{2}) T_n(\sqrt{1 - y^2}) - \cos (n\frac{\pi}{2}) y [\sqrt{1 - y^2} U_{n-1}(\sqrt{1 - y^2}) + T_{n-1}(\sqrt{1 - y^2})]]
\end{multline}

\subsection{Simplification}
Using Euler's formula, we have:
\begin{equation}
  = e^{-i n\frac{\pi}{2}} [T_n(\sqrt{1 - y^2}) + i y [\sqrt{1 - y^2} U_{n-1}(\sqrt{1 - y^2}) + T_{n-1}(\sqrt{1 - y^2})]]
\end{equation}

Note that $e^{-i n\frac{\pi}{2}}$ is a complex number of unit magnitude. Specifically:

\begin{equation}
  e^{-i n\frac{\pi}{2}} = 
  \begin{cases} 
    1 & \text{if } n \equiv 0 \pmod{4} \\
    -i & \text{if } n \equiv 1 \pmod{4} \\
    -1 & \text{if } n \equiv 2 \pmod{4} \\
    i & \text{if } n \equiv 3 \pmod{4}
  \end{cases}
\end{equation}

This factor contributes only to the phase of the result and does not affect the relationship between the Bessel function and the Chebyshev polynomials. For the purpose of establishing this relationship, we can factor it out:

\begin{equation}
  = e^{-i n\frac{\pi}{2}} \cdot [T_n(\sqrt{1 - y^2}) + i y [\sqrt{1 - y^2} U_{n-1}(\sqrt{1 - y^2}) + T_{n-1}(\sqrt{1 - y^2})]]
\end{equation}

We will focus on the term in square brackets to continue our derivation:

\begin{equation}
  [T_n(\sqrt{1 - y^2}) + i y [\sqrt{1 - y^2} U_{n-1}(\sqrt{1 - y^2}) + T_{n-1}(\sqrt{1 - y^2})]]
\end{equation}

Further simplifying:
\begin{equation}
  = [T_n(\sqrt{1 - y^2}) + i y T_{n-1}(\sqrt{1 - y^2}) + i y \sqrt{1 - y^2} U_{n-1}(\sqrt{1 - y^2})]
\end{equation}

\subsection{Using Chebyshev Polynomial Identities}
Using the recurrence relation $T_{n-1}(x) + x U_{n-1}(x) = U_n(x)$:
\begin{equation}
  = [T_n(\sqrt{1 - y^2}) + i y U_n(\sqrt{1 - y^2})]
\end{equation}

Using $U_n(x) = 2T_n(x) + U_{n-2}(x)$:
\begin{equation}
  = T_n(\sqrt{1 - y^2}) [1 + 2iy] + i y U_{n-2}(\sqrt{1 - y^2})
\end{equation}

Using $U_{n-2}(x) = x U_{n-1}(x) - T_{n-1}(x)$:
\begin{equation}
  = T_n(\sqrt{1 - y^2}) [1 + 2iy] + i y [\sqrt{1 - y^2} U_{n-1}(\sqrt{1 - y^2}) - T_{n-1}(\sqrt{1 - y^2})]
\end{equation}

\subsection{Final Steps}
Rearranging and using $2T_n(x) = U_n(x) - U_{n-2}(x)$:
\begin{equation}
  = T_n(\sqrt{1 - y^2}) + i y [T_{n-1}(\sqrt{1 - y^2}) + \sqrt{1 - y^2} U_{n-1}(\sqrt{1 - y^2})]
\end{equation}

Finally, using $T_{n-1}(x) + x U_{n-1}(x) = U_n(x)$:
\begin{equation}
  = T_n(\sqrt{1 - y^2}) + i y U_n(\sqrt{1 - y^2})
\end{equation}

\section{Conclusion}
Therefore, our integral formula can be rewritten as:
\begin{equation}
  \int_0^{\infty} J_n (x) e^{- ixy} dx = e^{-i n\frac{\pi}{2}} \cdot \frac{T_n (\sqrt{1 - y^2}) + i y U_n (\sqrt{1 - y^2})}{\sqrt{1 - y^2}}
\end{equation}
This directly relates the original integral to Chebyshev polynomials of the first and second kind, with an additional phase factor $e^{-i n\frac{\pi}{2}}$ that depends on the order $n$ of the Bessel function.

\end{document}
