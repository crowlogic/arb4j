\documentclass{article}
\usepackage[english]{babel}
\usepackage{latexsym}

%%%%%%%%%% Start TeXmacs macros
\newenvironment{proof}{\noindent\textbf{Proof\ }}{\hspace*{\fill}$\Box$\medskip}
\newtheorem{lemma}{Lemma}
\newtheorem{theorem}{Theorem}
%%%%%%%%%% End TeXmacs macros

%


\begin{document}

\title{Sketch Finite and Full Rank Kernel Analysis for Translation-Invariant
Kernels with Proofs}

\date{November 22, 2024}

\maketitle

\section{Finite Rank Kernels and Their Eigenfunctions}

Given:
\begin{itemize}
  \item The kernel $K (s, t)$ is translation-invariant, i.e., $K (s, t) = K (t
  - s)$
  
  \item We have a basis $\{\psi_n (t - s)\}_{n = 1}^{\infty}$ that uniformly
  converges to the kernel:
  \[ K (t - s) = \lim_{N \to \infty}  \sum_{n = 1}^N \psi_n  (t - s) \]
\end{itemize}

\section{Finite Rank Kernels}

For each finite $N$, we have a finite-rank kernel:
\[ K_N (s, t) = \sum_{n = 1}^N \psi_n  (t - s) \]
Key properties:
\begin{enumerate}
  \item The rank of $K_N$ is exactly $N$.
  
  \item $K_N$ has exactly $N$ non-zero eigenvalues and corresponding
  eigenfunctions.
  
  \item These eigenfunctions are exact for $K_N$, not approximations within
  $K_N$.
\end{enumerate}

\section{Eigenvalue Problem for Finite Rank Kernels}

For each $K_N$, we solve:
\[ A_N \mathbf{C}_N = \lambda_N \mathbf{C}_N \]
where:
\begin{itemize}
  \item $A_N = [A_{ij}]_{i, j = 1}^N$ with $A_{ij} = \int \psi_i  (t - s)
  \psi_j (t) dt$
  
  \item $\mathbf{C}_N = [c_{1 k}, c_{2 k}, ..., c_{Nk}]^T$ is the $k$-th
  eigenvector
  
  \item $\lambda_N$ is the corresponding eigenvalue
\end{itemize}
This yields exactly $N$ eigenvalue-eigenfunction pairs $(\lambda_k^{(N)},
\phi_k^{(N)})$, where:
\[ \phi_k^{(N)} (s) = \sum_{n = 1}^N c_{nk}^{(N)} \psi_n (s) \]
These are exact eigenfunctions for $K_N$, not approximations within $K_N$.

\section{Relationship to Full Rank Kernel}

As $N \to \infty$, we approach the full-rank kernel:
\[ \lim_{N \to \infty} K_N (s, t) = K (s, t) \]
Important observations:
\begin{enumerate}
  \item The eigenfunctions $\phi_k^{(N)}$ of $K_N$ are only approximate
  eigenfunctions of the full-rank kernel $K$.
  
  \item As $N$ increases, these approximations improve.
  
  \item In the limit $N \to \infty$, we obtain the true eigenfunctions of the
  full-rank kernel $K$.
\end{enumerate}

\section{Convergence Process}

1. For each finite $N$, we have $N$ exact eigenfunctions for $K_N$. 2. As $N$
increases, we get more eigenfunctions, and existing ones are refined. 3. In
the limit, we obtain infinitely many eigenfunctions of the full-rank kernel
$K$.

\section{Computational Implications}

When implementing this in a program:

1. Choose a finite $N$ based on computational resources and desired accuracy.
2. Solve the eigenvalue problem for $K_N$, obtaining $N$
eigenvalue-eigenfunction pairs. 3. These pairs are exact for $K_N$ but
approximate for the full-rank $K$. 4. To improve accuracy, increase $N$ and
repeat, knowing that: a) You'll get more eigenvalue-eigenfunction pairs. b)
Existing pairs will be refined, becoming better approximations of $K$'s
eigenfunctions.

\section{Proofs}

\subsection{Rank of Finite Kernels}

\begin{theorem}
  The rank of the finite kernel $K_N (s, t) = \sum_{n = 1}^N \psi_n  (t - s)$
  is exactly $N$.
\end{theorem}

\begin{proof}
  Let $\mathcal{H}_N$ be the reproducing kernel Hilbert space (RKHS)
  associated with $K_N$.
  \begin{enumerate}
    \item By construction, $K_N (s, t) = \sum_{n = 1}^N \psi_n  (t - s)$.
    
    \item The set $\{\psi_n \}_{n = 1}^N$ spans $\mathcal{H}_N$.
    
    \item These functions are linearly independent (as they form a basis).
    
    \item Therefore, $\dim (\mathcal{H}_N) = N$.
    
    \item For a positive definite kernel, the rank is equal to the dimension
    of its RKHS.
    
    \item Thus, rank($K_N$) = $\dim (\mathcal{H}_N) = N$.
  \end{enumerate}
\end{proof}

\subsection{Number of Non-zero Eigenvalues}

\begin{theorem}
  The finite-rank kernel $K_N$ has exactly $N$ non-zero eigenvalues.
\end{theorem}

\begin{proof}
  \begin{enumerate}
    \item The eigenvalue equation for $K_N$ is:
    \[ \int K_N (s, t) \phi (t) dt = \lambda \phi (s) \]
    \item Substituting the expression for $K_N$:
    \[ \int \sum_{n = 1}^N \psi_n  (t - s) \phi (t) dt = \lambda \phi (s) \]
    \item This can be written as a matrix equation: $A_N \mathbf{c} = \lambda
    \mathbf{c}$ where $A_N = [A_{ij}]_{i, j = 1}^N$ with $A_{ij} = \int \psi_i
    (t - s) \psi_j (t) dt$
    
    \item $A_N$ is an $N \times N$ matrix.
    
    \item By the fundamental theorem of linear algebra, the number of non-zero
    eigenvalues of $A_N$ is equal to its rank.
    
    \item We proved earlier that rank($K_N$) = $N$.
    
    \item Therefore, $K_N$ has exactly $N$ non-zero eigenvalues.
  \end{enumerate}
\end{proof}

\subsection{Exactness of Eigenfunctions for $K_N$}

\begin{lemma}
  The eigenfunctions $\phi_k^{(N)}$ obtained from solving $A_N \mathbf{C}_N =
  \lambda_N \mathbf{C}_N$ are exact eigenfunctions of $K_N$.
\end{lemma}

\begin{proof}
  \begin{enumerate}
    \item Let $\phi_k^{(N)} (s) = \sum_{n = 1}^N c_{nk}^{(N)} \psi_n (s)$ be
    an eigenfunction obtained from $A_N \mathbf{C}_N = \lambda_N
    \mathbf{C}_N$.
    
    \item Substitute this into the eigenvalue equation for $K_N$:
    \[ \int K_N (s, t) \phi_k^{(N)} (t) dt = \lambda_k^{(N)} \phi_k^{(N)} (s)
    \]
    \[ \int \sum_{i = 1}^N \psi_i  (t - s)  \sum_{j = 1}^N c_{jk}^{(N)} \psi_j
       (t) dt = \lambda_k^{(N)}  \sum_{n = 1}^N c_{nk}^{(N)} \psi_n (s) \]
    \[ \sum_{i = 1}^N \sum_{j = 1}^N c_{jk}^{(N)}  \int \psi_i  (t - s) \psi_j
       (t) dt = \lambda_k^{(N)}  \sum_{n = 1}^N c_{nk}^{(N)} \psi_n (s) \]
    \[ \sum_{i = 1}^N \sum_{j = 1}^N A_{ij} c_{jk}^{(N)} \psi_i (s) =
       \lambda_k^{(N)}  \sum_{n = 1}^N c_{nk}^{(N)} \psi_n (s) \]
    \item This is exactly satisfied by the solution to $A_N \mathbf{C}_N =
    \lambda_N \mathbf{C}_N$.
    
    \item Therefore, $\phi_k^{(N)}$ is an exact eigenfunction of $K_N$ with
    eigenvalue $\lambda_k^{(N)}$.
  \end{enumerate}
\end{proof}

\subsection{Convergence to Full-Rank Kernel}

\begin{theorem}
  As $N \to \infty$, $K_N$ converges uniformly to $K$.
\end{theorem}

\begin{proof}
  \begin{enumerate}
    \item By definition, $K (t - s) = \lim_{N \to \infty}  \sum_{n = 1}^N
    \psi_n  (t - s)$ uniformly.
    
    \item This means that for any $\epsilon > 0$, there exists an $N_0$ such
    that for all $N > N_0$:
    \[ \sup_{s, t} |K (t - s) - \sum_{n = 1}^N \psi_n (t - s) | < \epsilon \]
    \item But $\sum_{n = 1}^N \psi_n  (t - s)$ is exactly $K_N (s, t)$.
    
    \item Therefore, $\sup_{s, t} |K (s, t) - K_N (s, t) | < \epsilon$ for all
    $N > N_0$.
    
    \item This is the definition of uniform convergence of $K_N$ to $K$.
  \end{enumerate}
\end{proof}

\subsection{Approximation of Full-Rank Eigenfunctions}

\begin{theorem}
  As $N \to \infty$, the eigenfunctions of $K_N$ converge to the
  eigenfunctions of $K$.
\end{theorem}

\begin{proof}
  (Sketch, as a full proof requires more advanced functional analysis)
  \begin{enumerate}
    \item Let $(\lambda, \phi)$ be an eigenvalue-eigenfunction pair of $K$.
    
    \item Define the projection operator $P_N$ onto the span of $\{\psi_n
    \}_{n = 1}^N$.
    
    \item Consider the sequence $\{(\lambda_N, P_N \phi)\}_{N = 1}^{\infty}$.
    
    \item As $N \to \infty$, $P_N \phi \to \phi$ in the $L^2$ norm.
    
    \item The uniform convergence of $K_N$ to $K$ implies that $\lambda_N \to
    \lambda$.
    
    \item Therefore, the eigenfunctions of $K_N$ converge to those of $K$.
  \end{enumerate}
\end{proof}

This completes the proofs of the main statements in our analysis, providing a
rigorous foundation for understanding the relationship between finite-rank
kernels and the full-rank kernel, as well as the behavior of their respective
eigenfunctions.

\end{document}
