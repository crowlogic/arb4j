\documentclass{article}
\usepackage[english]{babel}
\usepackage{geometry}
\usepackage{amsmath}
\usepackage{amsthm}
\geometry{letterpaper}

\newtheorem{theorem}{Theorem}

\begin{document}

\title{Uniformly Convergent Expansions of Bounded Positive Definite Functions}

\author{Stephen Crowley \\ \texttt{stephencrowley214@gmail.com}}
\date{October 23, 2024}

\maketitle

\begin{theorem}
  Any bounded positive definite function $K(t)$ has a uniformly convergent expansion 
  in terms of functions from the orthogonal complement of the null space of the 
  inner product defined by $K$. This uniform convergence holds initially on the 
  real line and extends to the entire complex plane. The boundedness condition 
  ensures the existence of the associated Reproducing Kernel Hilbert Space (RKHS).
\end{theorem}

\begin{proof}
  Let $K(t)$ be a bounded positive definite function. By Bochner's theorem, 
  there exists a finite positive measure $\mu$ (the spectral measure) such that:
  \begin{equation}
    K(t) = \int e^{i\omega t} d\mu(\omega)
  \end{equation}
  
  Let $\{P_n(\omega)\}_{n=0}^{\infty}$ be the orthogonal polynomials with respect 
  to this spectral measure $\mu$, and $\{f_n(t)\}_{n=0}^{\infty}$ their Fourier 
  transforms defined as:
  \begin{equation}
    f_n(t) = \int P_n(\omega) e^{i\omega t} d\mu(\omega)
  \end{equation}

  1) The orthogonality of the polynomials $P_n(\omega)$ is established:
  
  For $m \neq n$:
  \begin{equation}
    \int P_m(\omega) P_n(\omega) d\mu(\omega) = 0
  \end{equation}

  2) The null space property of $\{f_n(t)\}_{n=1}^{\infty}$ follows:

  Consider the inner product $\langle f_n, K \rangle$ for $n \geq 1$:
  \begin{equation}
    \langle f_n, K \rangle = \int f_n(t) K(t) dt = \int f_n(t) \left(\int 
    e^{i\omega t} d\mu(\omega)\right) dt
  \end{equation}

  By Fubini's theorem:
  \begin{equation}
    \langle f_n, K \rangle = \int \left(\int f_n(t) e^{i\omega t} dt\right) 
    d\mu(\omega) = \int P_n(\omega) d\mu(\omega) = 0
  \end{equation}

  3) Apply the Gram-Schmidt process to $\{f_n(t)\}_{n=0}^{\infty}$ to obtain 
  an orthonormal basis $\{g_n(t)\}_{n=0}^{\infty}$ for the orthogonal complement 
  of the null space:
  \begin{equation}
    \tilde{g}_0(t) = f_0(t)
  \end{equation}
  \begin{equation}
    g_0(t) = \frac{\tilde{g}_0(t)}{\|\tilde{g}_0(t)\|}
  \end{equation}

  For $n \geq 1$:
  \begin{equation}
    \tilde{g}_n(t) = f_n(t) - \sum_{k=0}^{n-1} \langle f_n, g_k \rangle g_k(t)
  \end{equation}
  \begin{equation}
    g_n(t) = \frac{\tilde{g}_n(t)}{\|\tilde{g}_n(t)\|}
  \end{equation}

  4) Express $K(t)$ in terms of this basis:
  \begin{equation}
    K(t) = \sum_{n=0}^{\infty} \alpha_n g_n(t)
  \end{equation}
  where $\alpha_n = \langle K, g_n \rangle$

  5) Define partial sums:
  \begin{equation}
    S_N(t) = \sum_{n=0}^N \alpha_n g_n(t)
  \end{equation}

  6) The sequence $\{S_N(t)\}$ converges uniformly to $K(t)$ in the canonical 
  metric:
  \begin{equation}
    d(f,g) = \sqrt{\int\int (f(t)-g(t))(f(s)-g(s))K(t-s)dtds}
  \end{equation}

  7) The boundedness of $K(t)$ ensures that this metric is well-defined and 
  the associated RKHS exists. The error in this metric satisfies:
  \begin{equation}
    d(K,S_N)^2 = \int\int (K(t)-S_N(t))(K(s)-S_N(s))K(t-s)dtds \to 0
  \end{equation}

  8) Extension to the Complex Plane:
  
  The boundedness of $K(t)$ implies it is analytic in the complex plane. The 
  partial sums $S_N(t)$ are finite sums of analytic functions. By the Identity 
  Theorem for analytic functions, the uniform convergence on the real line extends 
  to the entire complex plane.

  Thus, we have shown that any bounded positive definite function has a uniformly 
  convergent expansion in terms of functions from the orthogonal complement of 
  the null space of its induced inner product, with convergence extending to 
  the entire complex plane.
\end{proof}

\end{document}