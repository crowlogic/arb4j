\documentclass{article}
\usepackage[english]{babel}
\usepackage{geometry,latexsym}
\geometry{letterpaper}

%%%%%%%%%% Start TeXmacs macros
\catcode`\<=\active \def<{
\fontencoding{T1}\selectfont\symbol{60}\fontencoding{\encodingdefault}}
\catcode`\>=\active \def>{
\fontencoding{T1}\selectfont\symbol{62}\fontencoding{\encodingdefault}}
\newcommand{\cdummy}{\cdot}
\newcommand{\tmaffiliation}[1]{\\ #1}
\newcommand{\tmrsub}[1]{\ensuremath{_{\textrm{#1}}}}
\newenvironment{proof}{\noindent\textbf{Proof\ }}{\hspace*{\fill}$\Box$\medskip}
\newtheorem{theorem}{Theorem}
%%%%%%%%%% End TeXmacs macros

\begin{document}

\title{Uniformly Convergence Expansions of Positive Definite Functions}

\author{
  Stephen Crowley <stephencrowley214@gmail.com>
  \tmaffiliation{October 22, 2024}
}

\maketitle

\begin{theorem}
  The covariance function $K (t)$ of a stationary Gaussian process has a
  uniformly convergent expansion in terms of functions from the orthogonal
  complement of the null space of the inner product defined by $K$. This
  uniform convergence holds initially on the real line and extends to the
  entire complex plane.
\end{theorem}

\begin{proof}
  Let $\{P_n (\omega)\}_{n = 0}^{\infty}$ be the orthogonal polynomials with
  respect to the spectral density $S (\omega)$ of a stationary Gaussian
  process, and $\{f_n (t)\}_{n = 0}^{\infty}$ their Fourier transforms defined
  as:
  \begin{equation}
    f_n (t) = \int P_n (\omega) e^{i \omega t} d \omega
  \end{equation}
  Let $K (t)$ be the covariance function of the Gaussian process.
  
  1) First, the orthogonality of the polynomials $P_n (\omega)$ is
  established:
  
  a) By definition of orthogonal polynomials, for $m \neq n$:
  \begin{equation}
    \int P_m (\omega) P_n (\omega) S (\omega) d \omega = 0
  \end{equation}
  b) The spectral density and covariance function form a Fourier transform
  pair:
  \begin{equation}
    K (t) = \int S (\omega) e^{i \omega t} d \omega
  \end{equation}
  2) The Gram-Schmidt process is applied to the Fourier transforms $\{f_n
  (t)\}_{n = 0}^{\infty}$ to obtain an orthonormal basis $\{g_n (t)\}_{n =
  0}^{\infty}$:
  \begin{equation}
    \tilde{g}_0 (t) = f_0 (t)
  \end{equation}
  \begin{equation}
    g_0 (t) = \frac{\tilde{g}_0 (t)}{\| \tilde{g}_0 (t)\|}
  \end{equation}
  For $n \geq 1$:
  \begin{equation}
    \tilde{g}_n (t) = f_n (t) - \sum_{k = 0}^{n - 1} \langle f_n, g_k \rangle
    g_k (t)
  \end{equation}
  \begin{equation}
    g_n (t) = \frac{\tilde{g}_n (t)}{\| \tilde{g}_n (t)\|}
  \end{equation}
  where $\| \cdummy \|$ and $\langle \cdummy, \cdummy \rangle$ denote the norm
  and inner product induced by $K$, respectively.
  
  3) $K (t)$ can be expressed in terms of this basis:
  \begin{equation}
    K (t) = \sum_{n = 0}^{\infty} \alpha_n g_n (t)
  \end{equation}
  where $\alpha_n = \langle K, g_n \rangle$ are the projections of $K$ onto
  $g_n (t)$.
  
  4) The partial sum is defined as:
  \begin{equation}
    S_N (t) = \sum_{n = 0}^N \alpha_n g_n (t)
  \end{equation}
  5) The sequence of partial sums $S_N (t)$ converges uniformly to $K (t)$ in
  the canonical metric induced by the kernel as $N \to \infty$.
  
  6) To realize this, recall that the canonical metric is defined as:
  \begin{equation}
    d (f, g) = \sqrt{\int \int (f (t) - g (t))  (f (s) - g (s)) K (t - s)
    dtds}
  \end{equation}
  7) The error in this metric is considered:
  \begin{equation}
    d (K, S_N)^2 = \int \int (K (t) - S_N (t))  (K (s) - S_N (s)) K (t - s)
    dtds
  \end{equation}
  8) As the kernel operator is compact in this metric:
  
  For every positive epsilon, there exists an N (which depends on epsilon)
  less than n, such that the distance between K and S\tmrsub{n} is less than
  epsilon.
  \begin{equation}
    \exists N (\epsilon) < n : d (K, S_n) < \epsilon \quad \forall \epsilon >
    0
  \end{equation}
  9) Extension to the Complex Plane:
  
  a) The covariance function $K (t)$ of a stationary Gaussian process is
  positive definite and therefore analytic in the complex plane.
  
  b) The partial sum $S_N (t)$ is a finite sum of analytic functions (as $g_n
  (t)$ are analytic), and is thus analytic in the complex plane.
  
  c) The convergence of $S_N (t)$ to $K (t)$ on the real line is uniform, as
  shown in steps 1-8.
  
  d) Consider any open disk D in the complex plane that intersects the real
  line. The intersection of D with the real line contains an accumulation
  point.
  
  e) By the Identity Theorem for analytic functions, since $K (t)$ and $S_N
  (t)$ agree on a set with an accumulation point within D (namely, the
  intersection of D with the real line), they must agree on the entire disk D.
  
  f) As this holds for any disk intersecting the real line, and such disks
  cover the entire complex plane, the uniform convergence of $S_N (t)$ to $K
  (t)$ extends to the entire complex plane.
  
  Thus, it has been shown that the covariance function $K (t)$ has a uniformly
  convergent expansion in terms of functions from the orthogonal complement of
  the null space of the inner product defined by $K$. This uniform convergence
  holds initially on the real line and extends to the entire complex plane.
\end{proof}

\end{document}
