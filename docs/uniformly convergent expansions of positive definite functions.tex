\documentclass{article}
\usepackage{amsmath}
\usepackage{amsfonts}
\usepackage{amssymb}
\usepackage{amsthm}

\title{Uniform Convergence of Positive Definite Functions}
\author{Your Name}
\date{\today}

\newtheorem{theorem}{Theorem}

\begin{document}

\maketitle

\begin{theorem}
The covariance function $K(t)$ of a stationary Gaussian process can be uniformly approximated by partial sums of functions from the orthogonal complement of the null space of the inner product defined by $K$. This uniform convergence holds not only on the real line but extends to the entire complex plane.
\end{theorem}

\begin{proof}
Let $\{P_n(\omega)\}_{n=0}^{\infty}$ be the orthogonal polynomials with respect to the spectral density $S(\omega)$ of a stationary Gaussian process, and $\{f_n(t)\}_{n=0}^{\infty}$ their Fourier transforms. Let $K(t)$ be the covariance function of the Gaussian process.

1) First, we prove that $f_n(t)$ for $n \geq 1$ are orthogonal to $K(t)$:

   a) By definition of orthogonal polynomials, for $n \geq 1$:
      $$\int P_n(\omega)S(\omega)d\omega = 0$$

   b) The spectral density and covariance function form a Fourier transform pair:
      $$K(t) = \int S(\omega)e^{i\omega t}d\omega$$

   c) Consider the inner product $\langle f_n, K \rangle$ for $n \geq 1$:
      $$\langle f_n, K \rangle = \int f_n(t)K(t)dt = \int f_n(t) \left(\int S(\omega)e^{i\omega t}d\omega\right) dt$$

   d) Applying Fubini's theorem:
      $$\langle f_n, K \rangle = \int S(\omega) \left(\int f_n(t)e^{i\omega t}dt\right) d\omega = \int S(\omega)P_n(\omega)d\omega = 0$$

   Thus, $\{f_n(t)\}_{n=1}^{\infty}$ are orthogonal to $K(t)$.

2) The orthogonal complement of the space spanned by $\{f_n(t)\}_{n=1}^{\infty}$ is uniquely determined. Let $\{g_n(t)\}_{n=0}^{\infty}$ be the orthonormal basis for this complement, obtained through the Gram-Schmidt process:

   $$\tilde{g}_0(t) = K(t)$$
   $$g_0(t) = \frac{\tilde{g}_0(t)}{\|\tilde{g}_0(t)\|}$$

   For $n \geq 1$:
   $$\tilde{g}_n(t) = K(t) - \sum_{k=0}^{n-1} \langle K, g_k \rangle g_k(t)$$
   $$g_n(t) = \frac{\tilde{g}_n(t)}{\|\tilde{g}_n(t)\|}$$

   where $\|\cdot\|$ denotes the norm induced by the inner product $\langle f, g \rangle = \int f(t)g(t)dt$.

3) We can express $K(t)$ in terms of this basis:
   $$K(t) = \sum_{n=0}^{\infty} \alpha_n g_n(t)$$
   where $\alpha_n = \langle K, g_n \rangle$ are the projections of $K$ onto $g_n(t)$.

4) Define the partial sum:
   $$S_N(t) = \sum_{n=0}^N \alpha_n g_n(t)$$

5) We claim that $S_N(t)$ converges to $K(t)$ in the canonical metric induced by the kernel as $N \to \infty$.

6) The canonical metric is defined as:
   $$d(f,g) = \sqrt{\int\int (f(t) - g(t))(f(s) - g(s))K(t-s)dtds}$$

7) Consider the error in this metric:
   $$d(K, S_N)^2 = \int\int (K(t) - S_N(t))(K(s) - S_N(s))K(t-s)dtds$$

8) As the kernel operator is compact in this metric, for any $\epsilon > 0$, there exists an $N$ such that for all $n > N$:
   $$d(K, S_n) < \epsilon$$

9) Extension to the Complex Plane:
   
   a) The covariance function $K(t)$ of a stationary Gaussian process is positive definite and therefore analytic.
   
   b) The partial sum $S_N(t)$ is a finite sum of analytic functions (as $g_n(t)$ are analytic), and is thus analytic.
   
   c) The convergence of $S_N(t)$ to $K(t)$ on the real line is uniform, as shown in steps 1-8.
   
   d) By the identity theorem for analytic functions, if two analytic functions agree on a set with an accumulation point (in this case, the real line), they must agree everywhere in their domain of analyticity.
   
   e) Therefore, the uniform convergence of $S_N(t)$ to $K(t)$ extends to the entire complex plane.

Thus, we have shown that the covariance function $K(t)$ can be uniformly approximated by partial sums of functions from the orthogonal complement of the null space of the inner product defined by $K$. This uniform convergence holds  initially on the real line but is easily shown to extend  to the entire complex plane.
\end{proof}

\end{document}