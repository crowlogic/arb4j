\documentclass{article}
\usepackage{amsmath, amssymb, amsthm}
\usepackage{natbib}
\usepackage{hyperref}
\usepackage{cleveref}

\newtheorem{theorem}{Theorem}
\newtheorem{lemma}[theorem]{Lemma}
\newtheorem{corollary}[theorem]{Corollary}

\theoremstyle{definition}
\newtheorem{definition}[theorem]{Definition}
\newtheorem{example}[theorem]{Example}
\newtheorem{remark}[theorem]{Remark}

\bibliographystyle{plainnat}

\begin{document}

\title{Uniform Convergence of Orthonormal Basis Projections in RKHS}
\author{}
\date{\today}
\maketitle

\begin{definition}[Reproducing Kernel Hilbert Space]
A Hilbert space $H$ of functions on a set $D$ is called a reproducing kernel Hilbert space (RKHS) if there exists a function $k: D \times D \to \mathbb{R}$ such that:
\begin{enumerate}
    \item For every $x \in D$, the function $k_x(\cdot) = k(\cdot, x)$ belongs to $H$.
    \item For every $x \in D$ and every $f \in H$, the reproducing property holds: $f(x) = \langle f, k_x \rangle_H$.
\end{enumerate}
The function $k$ is called the reproducing kernel of $H$.
\end{definition}

\begin{definition}[Orthonormal Basis in RKHS]
\label{def:orthonormal_basis}
A sequence of functions $\{e_n\}_{n=1}^{\infty} \subset H$ is an orthonormal basis of the RKHS $H$ if:
\begin{enumerate}
    \item Orthonormality: For all indices $n, m$, $\langle e_n, e_m \rangle_H = \delta_{nm}$, where $\delta_{nm}$ is the Kronecker delta.
    \item Completeness: The span of $\{e_n\}_{n=1}^{\infty}$ is dense in $H$, which means:
    \begin{enumerate}
        \item For any $f \in H$, if $\langle f, e_n \rangle_H = 0$ for all $n$, then $f = 0$.
        \item Equivalently, every function $f \in H$ can be represented as
        \[
        f = \sum_{n=1}^{\infty} \langle f, e_n \rangle_H e_n
        \]
        with convergence in the $H$-norm:
        \[
        \lim_{N\to\infty} \left\|f - \sum_{n=1}^{N} \langle f, e_n \rangle_H e_n\right\|_H = 0.
        \]
    \end{enumerate}
    \item Parseval's Identity: For any $f \in H$,
    \[
    \|f\|_H^2 = \sum_{n=1}^{\infty} |\langle f, e_n \rangle_H|^2.
    \]
\end{enumerate}
In an RKHS, each basis function satisfies the reproducing property: $e_n(x) = \langle e_n, k(\cdot,x) \rangle_H$ for all $x \in D$.
\end{definition}

\begin{theorem}
\label{thm:uniform_convergence}
Let $H$ be a reproducing kernel Hilbert space (RKHS) on a set $D$ with reproducing kernel $k$. Suppose that:
\begin{enumerate}
    \item $\{e_n\}_{n=1}^{\infty}$ is an orthonormal basis of $H$ as defined in Definition \ref{def:orthonormal_basis}.
    \item The kernel is uniformly bounded on $D$; that is, there exists a constant $M > 0$ such that
    \[
    \sup_{x \in D} \|k(\cdot, x)\|_H \le M.
    \]
\end{enumerate}
Then for any function $f \in H$ with orthonormal expansion
\[
f = \sum_{n=1}^{\infty} c_n e_n,
\]
where $c_n = \langle f, e_n \rangle_H$, the partial sums
\[
S_N f = \sum_{n=1}^N c_n e_n
\]
converge uniformly to $f$ on $D$; in other words,
\[
\lim_{N\to\infty} \sup_{x\in D} \bigl| S_N f(x) - f(x) \bigr| = 0.
\]
\end{theorem}

\begin{proof}
By the completeness property of the orthonormal basis (Definition \ref{def:orthonormal_basis}), every function $f \in H$ can be represented by its orthonormal expansion that converges in the $H$-norm. Since $H$ is an RKHS, the evaluation functional at any $x \in D$ satisfies:
\[
\bigl| f(x) - S_N f(x) \bigr| = \Bigl| \langle f - S_N f, k(\cdot, x) \rangle_H \Bigr|.
\]

Using the Cauchy-Schwarz inequality:
\[
\bigl| f(x) - S_N f(x) \bigr| \le \|f - S_N f\|_H \, \|k(\cdot, x)\|_H.
\]

Taking the supremum over $x \in D$ yields:
\[
\sup_{x \in D} \bigl| f(x) - S_N f(x) \bigr| \le \|f - S_N f\|_H \, \sup_{x \in D} \|k(\cdot, x)\|_H.
\]

By the uniform boundedness assumption:
\[
\sup_{x \in D} \bigl| f(x) - S_N f(x) \bigr| \le M \, \|f - S_N f\|_H.
\]

From the convergence property of orthonormal bases:
\[
\lim_{N\to\infty} \|f - S_N f\|_H = 0.
\]

For any $\varepsilon > 0$, choose $N$ such that for all $n \ge N$:
\[
\|f - S_n f\|_H < \frac{\varepsilon}{M}.
\]

Then for all $n \ge N$:
\[
\sup_{x \in D} \bigl| f(x) - S_n f(x) \bigr| < \varepsilon.
\]

Thus:
\[
\lim_{N\to\infty} \sup_{x\in D} |S_N f(x) - f(x)| = 0.
\]
\end{proof}

\begin{remark}
The uniform boundedness condition on $\|k(\cdot, x)\|_H$ is essential. Without it, norm convergence in the RKHS does not guarantee uniform convergence of evaluations on $D$. This condition ensures the kernel's feature maps are uniformly bounded across the entire domain.
\end{remark}

\begin{remark}
The domain $D$ need not be compact. The result holds for arbitrary domains (e.g., $D = \mathbb{R}^n$) as long as $\sup_{x \in D} \|k(\cdot, x)\|_H \le M$.
\end{remark}

\begin{remark} \label{rem:mercer_uniqueness}
The uniform convergence applies to expansions of functions $f \in H$ in any orthonormal basis. For expansions of the kernel $k(x, y)$ itself, uniform convergence holds only for the Mercer eigenbasis $\{e_n^*\}$ satisfying:
\[
\int_D k(x, y)e_n^*(y)\,dy = \lambda_n e_n^*(x).
\]
Non-Mercer bases yield pointwise convergence.
\end{remark}

\bibliographystyle{plainnat}
\begin{thebibliography}{9}

\bibitem[Riesz(1907)]{riesz1907}
Riesz, F. (1907). Sur les syst{\`e}mes orthogonaux de fonctions. \emph{Comptes rendus de l'Acad{\'e}mie des sciences}, 144:615--619.

\bibitem[Fischer(1907)]{fischer1907}
Fischer, E. (1907). Sur la convergence en moyenne. \emph{Comptes rendus de l'Acad{\'e}mie des sciences}, 144:1022--1024.

\bibitem[Aronszajn(1950)]{aronszajn1950}
Aronszajn, N. (1950). Theory of reproducing kernels. \emph{Transactions of the American Mathematical Society}, 68(3):337--404.

\bibitem[Berlinet and Thomas-Agnan(2004)]{berlinet2004}
Berlinet, A. and Thomas-Agnan, C. (2004). \emph{Reproducing Kernel Hilbert Spaces in Probability and Statistics}. Springer, Boston, MA.

\end{thebibliography}

\end{document}
