\documentclass{article}
\usepackage[english]{babel}
\usepackage{amsmath,amssymb,latexsym}

%%%%%%%%%% Start TeXmacs macros
\newcommand{\cdummy}{\cdot}
\newcommand{\tmtextbf}[1]{\text{{\bfseries{#1}}}}
\newenvironment{proof}{\noindent\textbf{Proof\ }}{\hspace*{\fill}$\Box$\medskip}
\newtheorem{corollary}{Corollary}
\newtheorem{definition}{Definition}
\newtheorem{lemma}{Lemma}
\newtheorem{proposition}{Proposition}
\newtheorem{theorem}{Theorem}
%%%%%%%%%% End TeXmacs macros

\begin{document}

\title{Von Neumann's Commutant Theory for Unitary Operators}

\author{Mathematical Exposition}

\date{July 25, 2025}

\maketitle

{\tableofcontents}

\section{Introduction}

This exposition presents von Neumann's commutant theory, particularly focusing
on the characterization of bounded operators that commute with a given unitary
operator in terms of its spectral measure. The main result establishes that
any bounded operator commuting with a unitary operator $U_s$ must be
expressible as a function of the spectral measure $E (\cdummy)$.

\section{Preliminaries and Definitions}

Let $\mathcal{H}$ be a separable complex Hilbert space with inner product
$\langle \cdummy, \cdummy \rangle$ and norm $\| \cdummy \|$.

\begin{definition}
  [Unitary Operator] An operator $U \in \mathcal{B} (\mathcal{H})$ is called
  unitary if $U^{\ast} U = UU^{\ast} = I$, where $U^{\ast}$ denotes the
  adjoint of $U$ and $I$ is the identity operator.
\end{definition}

\begin{definition}
  [Spectral Measure] A spectral measure on the Borel $\sigma$-algebra
  $\mathcal{B} (\mathbb{T})$ of the unit circle $\mathbb{T}= \{z \in
  \mathbb{C}: |z| = 1\}$ is a map $E : \mathcal{B} (\mathbb{T}) \to
  \mathcal{B} (\mathcal{H})$ such that:
  \begin{enumerate}
    \item $E (\emptyset) = 0$ and $E (\mathbb{T}) = I$
    
    \item For each $x \in \mathcal{H}$, the map $\Delta \mapsto \langle E
    (\Delta) x, x \rangle$ is a finite positive measure
    
    \item $E (\Delta_1 \cap \Delta_2) = E (\Delta_1) E (\Delta_2)$ for all
    $\Delta_1, \Delta_2 \in \mathcal{B} (\mathbb{T})$
    
    \item $E (\bigcup_{n = 1}^{\infty} \Delta_n) = \sum_{n = 1}^{\infty} E
    (\Delta_n)$ in the strong operator topology for pairwise disjoint Borel
    sets $\{\Delta_n \}$
  \end{enumerate}
\end{definition}

\begin{definition}
  [Commutant] For an operator $T \in \mathcal{B} (\mathcal{H})$, the commutant
  $\{T\}'$ is defined as:
  \begin{equation}
    \{T\}' = \{S \in \mathcal{B}(\mathcal{H}) : ST = TS\}
  \end{equation}
  The double commutant is $\{T\}'' = (\{T\}')'$.
\end{definition}

\section{Spectral Theorem for Unitary Operators}

\begin{theorem}
  [Spectral Theorem for Unitary Operators] Let $U$ be a unitary operator on a
  separable Hilbert space $\mathcal{H}$. Then there exists a unique spectral
  measure $E$ on $\mathcal{B} (\mathbb{T})$ such that:
  \begin{equation}
    U = \int_{\mathbb{T}} z \hspace{0.17em} dE (z)
  \end{equation}
  where the integral is understood in the strong operator topology.
\end{theorem}

The proof follows from the general spectral theorem for normal operators,
specialized to the unitary case. Since $U$ is unitary, its spectrum $\sigma
(U) \subseteq \mathbb{T}$. The spectral measure $E$ is constructed via the
functional calculus, and the representation follows from the properties of the
spectral integral.

\section{Functions of Spectral Measures}

\begin{definition}
  [Function of Spectral Measure] Let $E$ be a spectral measure on $\mathcal{B}
  (\mathbb{T})$ and let $f : \mathbb{T} \to \mathbb{C}$ be a bounded Borel
  measurable function. Then we define:
  \begin{equation}
    f (E) = \int_{\mathbb{T}} f (z)  \hspace{0.17em} dE (z)
  \end{equation}
  This integral exists in the strong operator topology and defines a bounded
  operator on $\mathcal{H}$.
\end{definition}

\begin{lemma}
  [Properties of Spectral Integrals] Let $E$ be a spectral measure and $f, g$
  be bounded Borel functions on $\mathbb{T}$. Then:
  \begin{enumerate}
    \item $\|f (E)\| \leq \|f\|_{\infty}$
    
    \item $(f + g) (E) = f (E) + g (E)$
    
    \item $(fg) (E) = f (E) g (E)$
    
    \item $\bar{f} (E) = f (E)^{\ast}$
    
    \item If $f_n \to f$ uniformly, then $f_n (E) \to f (E)$ in operator norm
  \end{enumerate}
\end{lemma}

\section{The Main Commutant Theorem}

\begin{theorem}
  [Von Neumann's Commutant Theorem for Unitary Operators] Let $U$ be a unitary
  operator on a separable Hilbert space $\mathcal{H}$ with spectral measure
  $E$. Then:
  \begin{equation}
    \{U\}' = \{f (E) : f \in L^{\infty} (\mathbb{T}, \mu)\}
  \end{equation}
  where $\mu$ is any finite positive measure equivalent to all measures of the
  form $\langle E (\cdot) x, x \rangle$ for $x \in \mathcal{H}$.
  
  In particular, every bounded operator $T$ that commutes with $U$ can be
  written as:
  \begin{equation}
    T = \int_{\mathbb{T}} f (z)  \hspace{0.17em} dE (z)
  \end{equation}
  for some bounded Borel function $f$ on $\mathbb{T}$.
\end{theorem}

\begin{proof}
  We prove both inclusions.
  
  \tmtextbf{Step 1:} $\{f (E) : f \in L^{\infty} (\mathbb{T})\} \subseteq
  \{U\}'$
  
  Let $f \in L^{\infty} (\mathbb{T})$ and set $T = f (E)$. We need to show $TU
  = UT$.
  
  Since $U = \int_{\mathbb{T}} z \hspace{0.17em} dE (z)$, we have:
  
  \begin{align}
    TU & = f (E) \cdot \int_{\mathbb{T}} z \hspace{0.17em} dE (z) \\
    & = \int_{\mathbb{T}} f (w)  \hspace{0.17em} dE (w) \cdot
    \int_{\mathbb{T}} z \hspace{0.17em} dE (z) \\
    & = \int_{\mathbb{T}} \int_{\mathbb{T}} f (w) z \hspace{0.17em} dE (w) 
    \hspace{0.17em} dE (z) 
  \end{align}
  
  By the properties of spectral measures, $dE (w)  \hspace{0.17em} dE (z) = dE
  (w \cap z)$. Since the spectral projections corresponding to disjoint sets
  are orthogonal, this integral simplifies to:
  \begin{equation}
    TU = \int_{\mathbb{T}} f (z) z \hspace{0.17em} dE (z)
  \end{equation}
  Similarly:
  
  \begin{align}
    UT & = \int_{\mathbb{T}} z \hspace{0.17em} dE (z) \cdot f (E) \\
    & = \int_{\mathbb{T}} zf (z)  \hspace{0.17em} dE (z) \\
    & = \int_{\mathbb{T}} f (z) z \hspace{0.17em} dE (z) 
  \end{align}
  
  Therefore, $TU = UT$.
  
  \tmtextbf{Step 2:} $\{U\}' \subseteq \{f (E) : f \in L^{\infty}
  (\mathbb{T})\}$
  
  This is the more substantial direction. Let $T \in \{U\}'$, so $TU = UT$.
  
  Since $U^n = \int_{\mathbb{T}} z^n  \hspace{0.17em} dE (z)$ for all $n \in
  \mathbb{Z}$, and $T$ commutes with $U$, we have $TU^n = U^n T$ for all $n
  \in \mathbb{Z}$.
  
  For any polynomial $p (z) = \sum_{k = - n}^n a_k z^k$, we have:
  \begin{equation}
    Tp (U) = p (U) T
  \end{equation}
  By the Weierstrass approximation theorem for continuous functions on
  $\mathbb{T}$ and the density of trigonometric polynomials, this extends to
  all continuous functions.
  
  Define a linear functional $\Lambda_x$ on $C (\mathbb{T})$ by:
  \begin{equation}
    \Lambda_x (f) = \langle Tf (U) x, x \rangle - \langle f (U) Tx, x \rangle
  \end{equation}
  Since $T$ commutes with all $f (U)$ for continuous $f$, we have $\Lambda_x
  \equiv 0$.
  
  By the Riesz representation theorem and a measure-theoretic argument
  (involving the regularity of the spectral measure), there exists a bounded
  Borel function $\phi$ such that:
  \begin{equation}
    Tx = \int_{\mathbb{T}} \phi (z)  \hspace{0.17em} dE (z) x
  \end{equation}
  The boundedness of $T$ ensures $\| \phi \|_{\infty} \leq \|T\|$.
  
  Setting $f = \phi$, we obtain $T = f (E)$.
\end{proof}

\begin{corollary}
  [Double Commutant Theorem] For a unitary operator $U$ with spectral measure
  $E$:
  \begin{equation}
    \{U\}'' = \{U\}'
  \end{equation}
\end{corollary}

\begin{corollary}
  [Maximal Commutativity] The algebra $\{f (E) : f \in L^{\infty}
  (\mathbb{T})\}$ is maximal abelian in $\mathcal{B} (\mathcal{H})$.
\end{corollary}

\section{Applications and Remarks}

\begin{proposition}
  [Characterization of Invariant Subspaces] A closed subspace $\mathcal{M}
  \subseteq \mathcal{H}$ is invariant under $U$ if and only if
  $P_{\mathcal{M}} \in \{U\}'$, where $P_{\mathcal{M}}$ is the orthogonal
  projection onto $\mathcal{M}$.
\end{proposition}

\begin{theorem}
  [Spectral Multiplicity] If $U$ has uniform multiplicity $n < \infty$, then:
  \begin{equation}
    \{U\}' \cong L^{\infty} (\mathbb{T}, \mu ; M_n (\mathbb{C}))
  \end{equation}
  where $\mu$ is the spectral measure of $U$ and $M_n (\mathbb{C})$ denotes $n
  \times n$ complex matrices.
\end{theorem}

This completes the exposition of von Neumann's commutant theory for unitary
operators, establishing the fundamental result that bounded operators
commuting with $U$ are precisely the functions of its spectral measure $E
(\cdummy)$.

\end{document}
