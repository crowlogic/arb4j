\documentclass{article}
\usepackage[english]{babel}
\usepackage{amsmath,amssymb,latexsym,theorem}

%%%%%%%%%% Start TeXmacs macros
\newcommand{\assign}{:=}
\newcommand{\tmaffiliation}[1]{\\ #1}
\newcommand{\tmem}[1]{{\em #1\/}}
\newcommand{\tmtextit}[1]{\text{{\itshape{#1}}}}
\newenvironment{proof}{\noindent\textbf{Proof\ }}{\hspace*{\fill}$\Box$\medskip}
\newtheorem{definition}{Definition}
\newtheorem{proposition}{Proposition}
{\theorembodyfont{\rmfamily}\newtheorem{remark}{Remark}}
\newtheorem{theorem}{Theorem}
%%%%%%%%%% End TeXmacs macros

\begin{document}

\title{The Covariance of Ergodic Stationary Processes}

\author{
  Stephen Crowley
  \tmaffiliation{July 27, 2025}
}

\maketitle

\begin{abstract}
  This note establishes in detail the equality, for strictly stationary and
  ergodic real-valued stochastic processes with finite second moment, between
  the covariance function as defined by the expectation of the product of
  observations at a given lag and the almost sure limit of temporal
  expectation of products along a single path. The argument is developed from
  first principles, specifying measure-theoretic structure, shift invariance,
  ergodicity, and relevant properties of function spaces, and invoking the
  continuous-time ergodic theorem in full generality for integrable functions.
\end{abstract}

\section*{1. Preliminaries}

\begin{definition}
  [Probability Space and Process] Let $(\Omega, \mathcal{F}, P)$ be a
  probability space. A {\tmem{stochastic process}} $\xi : \mathbb{R} \times
  \Omega \to \mathbb{R}$ is a collection of real random variables $\xi (t) :
  \Omega \to \mathbb{R}$, indexed by $t \in \mathbb{R}$. The process is
  {\tmem{jointly measurable}} if the mapping $(t, \omega) \mapsto \xi (t,
  \omega)$ is measurable with respect to the product $\sigma$-algebra
  $\mathcal{B} (\mathbb{R}) \otimes \mathcal{F}$.
\end{definition}

\begin{definition}
  [Strict Stationarity] A stochastic process $\xi = (\xi (t))_{t \in
  \mathbb{R}}$ is {\tmem{strictly stationary}} if, for every $n \in
  \mathbb{N}$, every choice $t_1, \ldots, t_n \in \mathbb{R}$, and every $h
  \in \mathbb{R}$,
  \[ (\xi (t_1 + h), \ldots, \xi (t_n + h)) \overset{d}{=} (\xi (t_1), \ldots,
     \xi (t_n)), \]
  that is, the finite-dimensional distributions are invariant under time
  shifts.
\end{definition}

\begin{definition}
  [Covariance Function] For a stochastic process $\xi$ with $\mathbb{E} [\xi
  (0)^2] < \infty$, define the {\tmem{covariance function}} $r$ by
  \[ r (\tau) = \text{Cov} (\xi (0), \xi (\tau)) =\mathbb{E} [(\xi (0) - m)
     (\xi (\tau) - m)], \quad m =\mathbb{E} [\xi (0)] . \]
  If $\mathbb{E} [\xi (0)] = 0$, then $r (\tau) =\mathbb{E} [\xi (0) \xi
  (\tau)]$.
\end{definition}

\begin{definition}
  [Shift Operator (Path Space Version)] Let $E =\mathbb{R}$ and consider the
  canonical space $\Omega =\mathbb{R}^{\mathbb{R}}$ consisting of all
  functions $x : \mathbb{R} \to \mathbb{R}$. For each $h \in \mathbb{R}$
  define the shift operator $T_h : \Omega \to \Omega$ by
  \[ (T_h x) (t) = x (t + h) \]
  for all $t \in \mathbb{R}$, $x \in \Omega$. If $\xi$ is a process on an
  abstract probability space, interpret $x$ as a sample path $x_{\omega} (t) =
  \xi (t, \omega)$.
\end{definition}

\begin{proposition}
  [Shift Invariance] Let $\mu$ be a probability measure on $(\Omega,
  \mathcal{F})$ such that the coordinate process $x (t)$ under $\mu$ has the
  same law as $\xi (t)$ for all $t$. If $\xi$ is strictly stationary, then for
  every $h \in \mathbb{R}$ and every $A \in \mathcal{F}$, $\mu (T_h^{- 1} A) =
  \mu (A)$.
\end{proposition}

\begin{proof}
  Let $A \in \mathcal{F}$ be a cylinder set of the form
  \[ A = \{ x \in \Omega : (x (t_1), \ldots, x (t_n)) \in B \} \]
  where $t_1, \ldots, t_n \in \mathbb{R}$ and $B$ is a Borel set in
  $\mathbb{R}^n$. Then
  \[ T_h^{- 1} A = \{ x \in \Omega : (x (t_1 + h), \ldots, x (t_n + h)) \in B
     \} . \]
  Since under $\mu$ the law of $(x (t_1 + h), \ldots, x (t_n + h))$ coincides
  with that of $(x (t_1), \ldots, x (t_n))$ by stationarity,
  \[ \mu ((x (t_1 + h), \ldots, x (t_n + h)) \in B) = \mu ((x (t_1), \ldots, x
     (t_n)) \in B) = \mu (A) . \]
  Extension from cylinder sets to $\mathcal{F}$ proceeds by the monotone class
  theorem or standard arguments.
\end{proof}

\begin{definition}
  [Ergodicity] The measure-preserving flow $(T_h)_{h \in \mathbb{R}}$ on
  $(\Omega, \mathcal{F}, \mu)$ is called {\tmem{ergodic}} if, for every $A \in
  \mathcal{F}$ satisfying $T_h^{- 1} A = A$ for all $h \in \mathbb{R}$, either
  $\mu (A) = 0$ or $\mu (A) = 1$.
\end{definition}

\begin{remark}
  Ergodicity is equivalent to the triviality of the shift-invariant
  $\sigma$-algebra:
  \[ \mathcal{I}= \{A \in \mathcal{F}: T_h^{- 1} A = A \text{for all } h \in
     \mathbb{R}\} . \]
\end{remark}

\section*{2. The Covariance Function and Pathwise Limit}

\begin{theorem}
  [Pathwise Determination of Covariance Function]\label{mainthm}Let $(\Omega,
  \mathcal{F}, \mu)$ and the canonical process $x (t)$ be as above. Suppose
  that under $\mu$, $x (t)$ is strictly stationary, ergodic with respect to
  $(T_h)_{h \in \mathbb{R}}$, and $\mathbb{E}_{\mu} [x (0)^2] < \infty$. Fix
  $\tau \in \mathbb{R}$. Then for $\mu$-almost every $x \in \Omega$,
  \[ \lim_{T \to \infty}  \frac{1}{2 T}  \int_{- T}^T x (t)  \hspace{0.17em} x
     (t + \tau)  \hspace{0.17em} dt = r (\tau) \]
  where $r (\tau) =\mathbb{E}_{\mu}  [x (0) x (\tau)]$.
\end{theorem}

\begin{proof}
  The steps follow as below.
  \begin{enumerate}
    \item The map $x \mapsto x (0) x (\tau)$ is measurable as a product of
    coordinate projections, hence Borel measurable on $\Omega$.
    
    \item Since $\mathbb{E}_{\mu} [x (0)^2] < \infty$, by the Cauchy-Schwarz
    inequality,
    \[ \mathbb{E}_{\mu} [\hspace{0.17em} |x (0) x (\tau) | \hspace{0.17em}]
       \leq \sqrt{\mathbb{E}_{\mu} [x (0)^2]  \hspace{0.27em} \mathbb{E}_{\mu}
       [x (\tau)^2]} =\mathbb{E}_{\mu} [x (0)^2] < \infty \]
    by stationarity, so $x \mapsto x (0) x (\tau)$ is integrable.
    
    \item Consider the function $F : \Omega \to \mathbb{R}$ given by $F (x) =
    x (0) x (\tau)$. For each $t \in \mathbb{R}$, define $F \circ T_t (x) = x
    (t) x (t + \tau)$. As above, this is measurable and integrable for each
    $t$.
    
    \item For each $x \in \Omega$ and $T > 0$, set
    \[ A_T (x) \assign \frac{1}{2 T}  \int_{- T}^T F (T_t x)  \hspace{0.17em}
       dt = \frac{1}{2 T}  \int_{- T}^T x (t)  \hspace{0.17em} x (t + \tau) 
       \hspace{0.17em} dt. \]
    \item The Birkhoff (Khintchine) ergodic theorem in continuous time for
    flows of measure-preserving transformations applies under the above
    conditions. Thus, for $\mu$-almost every $x \in \Omega$,
    \[ \lim_{T \to \infty} A_T (x) =\mathbb{E}_{\mu} [F] = r (\tau) . \]
  \end{enumerate}
  This matches the claimed formula.
\end{proof}

\begin{remark}
  The set of $x \in \Omega$ for which the limit in Theorem \ref{mainthm} fails
  has measure zero under $\mu$. The limit is a measurable function of $x$. The
  limit equals the covariance function for each fixed $\tau \in \mathbb{R}$.
  Almost sure convergence for all $\tau$ simultaneously generally holds only
  for countable subsets of $\mathbb{R}$.
\end{remark}

\section*{3. Measure-Theoretic and Technical Details}

\begin{enumerate}
  \item The canonical path space $\Omega =\mathbb{R}^{\mathbb{R}}$ with the
  product $\sigma$-algebra supports all coordinate projections and the shift
  operator. The measure $\mu$ is defined such that the law of $(x (t_1),
  \ldots, x (t_n))$ under $\mu$ is that of $(\xi (t_1), \ldots, \xi (t_n))$.
  
  \item The measurability and integrability of $x \mapsto x (0) x (\tau)$
  follow from the structure of $\Omega$ and the moment assumption.
  
  \item The shift flow $(T_h)_{h \in \mathbb{R}}$ acts measurably and
  preserves $\mu$.
  
  \item The ergodic theorem for flows applies, as all conditions
  (integrability, invariance, ergodicity) are satisfied.
\end{enumerate}

\section*{4. Almost Sure Convergence}

\begin{definition}
  [Almost Surely] Let $(\Omega, \mathcal{F}, \mu)$ be a probability space. A
  property holds {\tmem{almost surely}} if the set of $x \in \Omega$ for which
  it fails has $\mu$-measure zero.
\end{definition}

\begin{remark}
  If a sequence of measurable maps $f_n : \Omega \to \mathbb{R}$ converges
  almost surely to $f$, this means $\mu (\{x : \lim_{n \to \infty} f_n (x) = f
  (x)\}) = 1$. Sets of measure zero do not affect expectation or
  measure-theoretic statements.
\end{remark}

\begin{thebibliography}{10}
  {\bibitem{doob}}Joseph L. Doob, \tmtextit{Stochastic Processes}, Wiley,
  1953.
  
  {\bibitem{krengel}}Ulrich Krengel, \tmtextit{Ergodic Theorems}, de Gruyter,
  1985.
  
  {\bibitem{cohn}}Donald L. Cohn, \tmtextit{Measure Theory}, Birkh{\"a}user,
  2013.
  
  {\bibitem{cornfeld}}Ilya P. Cornfeld, Sergei V. Fomin, Yakov G. Sinai,
  \tmtextit{Ergodic Theory}, Springer, 1982.
  
  {\bibitem{yosida}}K{\^o}saku Yosida, \tmtextit{Ergodic Theorems}, Springer,
  1965.
\end{thebibliography}

\

\end{document}
