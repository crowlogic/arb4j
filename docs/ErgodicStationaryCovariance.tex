\documentclass{article}
\usepackage[english]{babel}
\usepackage{amsmath,amssymb,latexsym,theorem}

%%%%%%%%%% Start TeXmacs macros
\newcommand{\tmaffiliation}[1]{\\ #1}
\newcommand{\tmem}[1]{{\em #1\/}}
\newcommand{\tmtextbf}[1]{\text{{\bfseries{#1}}}}
\newcommand{\tmtextit}[1]{\text{{\itshape{#1}}}}
\newenvironment{proof}{\noindent\textbf{Proof\ }}{\hspace*{\fill}$\Box$\medskip}
\newtheorem{definition}{Definition}
{\theorembodyfont{\rmfamily}\newtheorem{remark}{Remark}}
\newtheorem{theorem}{Theorem}
%%%%%%%%%% End TeXmacs macros

\begin{document}

\title{The Covariance of Ergodic Stationary Processes}

\author{
  Stephen Crowley
  \tmaffiliation{July 27, 2025}
}

\maketitle

\begin{abstract}
  This short note presents a fundamental result concerning the covariance of
  real-valued, zero-mean, strictly stationary, and ergodic stochastic
  processes with finite second moments. It is shown that, for such processes,
  the ensemble covariance function can be consistently recovered from a
  single, sufficiently long sample path. Specifically, the temporal average of
  products of observations at time-lagged points converges almost surely to
  the ensemble covariance as the observation window extends to infinity. This
  result relies upon the application of the Birkhoff--Khinchin ergodic theorem
  to sample paths of the process, ensuring that, for almost every realization,
  the empirical and ensemble covariances coincide in the limit. 
\end{abstract}

\begin{definition}
  A stochastic process $\xi (t)$, $t \in \mathbb{R}$, is called
  {\tmem{strictly stationary}} if for all $t_1, t_2, \ldots, t_n$ and all
  $\tau \in \mathbb{R}$,
  \begin{equation}
    (\xi (t_1 + \tau), \ldots, \xi (t_n + \tau)) \overset{d}{=} (\xi (t_1),
    \ldots, \xi (t_n))
  \end{equation}
  A strictly stationary process is called {\tmem{ergodic}} if every invariant
  event under the temporal shift transformation has probability zero or one.
\end{definition}

\begin{theorem}[Exact Covariance Function from a Single Sample Path]
  \label{thm:covariance}Let $\xi (t)$ be a real-valued, zero-mean, strictly
  stationary, and ergodic process with $\mathbb{E} [\xi^2 (0)] < \infty$. Let
  $x (t)$ be a realization of $\xi (t)$. Then for every fixed $\tau \in
  \mathbb{R}$,
  \begin{equation}
    r (\tau) = \lim_{T \to \infty}  \frac{1}{2 T}  \int_{- T}^T x (t) x (t +
    \tau)  \hspace{0.17em} dt
  \end{equation}
  almost surely, where $r (\tau) =\mathbb{E} [\xi (0) \xi (\tau)]$ is the
  covariance function.
\end{theorem}

\begin{proof}
  \tmtextbf{Step 1: Establish integrability conditions.}\\
  Since $\xi (t)$ is strictly stationary, $\mathbb{E} [\xi^2 (t)] =\mathbb{E}
  [\xi^2 (0)] < \infty$ for all $t \in \mathbb{R}$. For any fixed $\tau \in
  \mathbb{R}$, the Cauchy-Schwarz inequality yields
  \begin{equation}
    \begin{array}{ll}
      \mathbb{E} [| \xi (0) \xi (\tau) |] & \leq \sqrt{\mathbb{E} [\xi^2 (0)]
      \cdot \mathbb{E} [\xi^2 (\tau)]}\\
      & = \sqrt{\mathbb{E} [\xi^2 (0)] \cdot \mathbb{E} [\xi^2 (0)]}\\
      & =\mathbb{E} [\xi^2 (0)] < \infty
    \end{array}
  \end{equation}
  Therefore, the random variable $\xi (0) \xi (\tau)$ is integrable.
  
  \tmtextbf{Step 2: Define the measurable function and shift operator.}\\
  Consider the function $f : \mathbb{R} \to \mathbb{R}$ defined by
  \begin{equation}
    f (s) = \xi (s) \xi (s + \tau)
  \end{equation}
  for fixed $\tau$. Let $T_h$ denote the shift operator defined by
  \begin{equation}
    (T_h \xi) (t) = \xi (t + h)
  \end{equation}
  for $h \in \mathbb{R}$. The strict stationarity condition implies that the
  measure induced by $\xi$ is invariant under $T_h$ for all $h$.
  
  \tmtextbf{Step 3: Verify ergodicity conditions.}\\
  Since $\xi (t)$ is ergodic, the shift-invariant $\sigma$-algebra has trivial
  tail structure: every shift-invariant event has probability 0 or 1. This
  ensures that the conditions of the Birkhoff-Khinchin ergodic theorem are
  satisfied for the dynamical system $(\Omega, \mathcal{F}, P, T_h)$ where
  $\Omega$ is the sample space of the process.
  
  \tmtextbf{Step 4: Apply the Birkhoff-Khinchin ergodic theorem.}\\
  For the integrable function
  \begin{equation}
    f (s) = \xi (s) \xi (s + \tau)
  \end{equation}
  the ergodic theorem states that
  \begin{equation}
    \lim_{T \to \infty}  \frac{1}{2 T}  \int_{- T}^T f (s)  \hspace{0.17em} ds
    =\mathbb{E} [f (0)]
  \end{equation}
  almost surely with respect to the probability measure of the process.
  Substituting our function:
  \begin{equation}
    \lim_{T \to \infty}  \frac{\int_{- T}^T \xi (s) \xi (s + \tau) 
    \hspace{0.17em} ds}{2 T} =\mathbb{E} [\xi (0) \xi (\tau)]
  \end{equation}
  almost surely.
  
  \tmtextbf{Step 5: Connect to sample path realization.}\\
  For any particular realization $x (t) = \xi (t, \omega)$ where $\omega$
  belongs to the set of full measure on which the ergodic theorem holds, we
  have
  \begin{equation}
    \lim_{T \to \infty}  \frac{\int_{- T}^T x (s) x (s + \tau) 
    \hspace{0.17em} ds}{2 T} =\mathbb{E} [\xi (0) \xi (\tau)]
  \end{equation}
  \tmtextbf{Step 6: Establish covariance function equality.}\\
  By definition of the covariance function for a zero-mean process:
  \begin{equation}
    \begin{array}{ll}
      r (\tau) & = \text{Cov} (\xi (0), \xi (\tau))\\
      & =\mathbb{E} [\xi (0) \xi (\tau)] -\mathbb{E} [\xi (0)] \mathbb{E}
      [\xi (\tau)]\\
      & =\mathbb{E} [\xi (0) \xi (\tau)] - 0 \cdot 0\\
      & =\mathbb{E} [\xi (0) \xi (\tau)]
    \end{array}
  \end{equation}
  \tmtextbf{Step 7: Conclude the main result.}\\
  Combining Steps 5 and 6:
  \begin{equation}
    r (\tau) =\mathbb{E} [\xi (0) \xi (\tau)] = \lim_{T \to \infty} 
    \frac{\int_{- T}^T x (t) x (t + \tau)  \hspace{0.17em} dt}{2 T} 
  \end{equation}
  almost surely. The exceptional set (where this equality fails) has
  probability zero by the ergodic theorem.
\end{proof}

\begin{remark}
  The almost sure convergence implies that for any specific realization drawn
  from the process, the temporal average will equal the ensemble covariance
  function.
\end{remark}

\begin{thebibliography}{9}
  {\bibitem{doob}}Joseph L. Doob, \tmtextit{Stochastic Processes}, Wiley,
  1953.
  
  {\bibitem{cornfeld}}Ilya P. Cornfeld, Sergei V. Fomin, Yakov G. Sinai,
  \tmtextit{Ergodic Theory}, Springer, 1982.
  
  {\bibitem{krengel}}Ulrich Krengel, \tmtextit{Ergodic Theorems}, de Gruyter,
  1985.
  
  {\bibitem{yosida}}K{\^o}saku Yosida, \tmtextit{Ergodic Theorems}, Springer,
  1965.
  
  {\bibitem{cohn}}Donald L. Cohn, \tmtextit{Measure Theory}, Birkh{\"a}user,
  2013. [See Chapter 9 for ergodic theorems in continuous time]
\end{thebibliography}

\end{document}
