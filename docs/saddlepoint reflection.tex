\section{Basis for the Modified Riemann-Siegel Theta Function}

The focus is on establishing a basis for the modified Riemann-Siegel theta function $\theta^*(t)$, ensuring it retains the desired properties of positive definiteness and monotonicity.

\subsection{Key Points}
1. **Basis for the Theta Function**: We can construct an orthonormal basis for the modified theta function $\theta^*(t)$, ensuring that it maintains the desired properties.

2. **Positive Definiteness**: The modification of $\theta(t)$ to $\theta^*(t)$ through reflection will help maintain positive definiteness, enabling us to derive a suitable kernel.

3. **Kernel Derivation**: Once we have the appropriate orthonormal basis, we can express the modified theta function in terms of this basis, allowing for the study of its properties and behavior.

\subsection{Next Steps}
- Establish the orthonormal basis for $\theta^*(t)$.
- Ensure that the constructed kernel based on $\theta^*(t)$ is positive definite and monotonic.

This approach will facilitate the exploration of the properties of the modified Riemann-Siegel theta function and its applications.