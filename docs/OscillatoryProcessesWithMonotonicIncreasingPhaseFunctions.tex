\documentclass{article}
\usepackage[english]{babel}
\usepackage{amssymb,latexsym}

%%%%%%%%%% Start TeXmacs macros
\newcommand{\tmaffiliation}[1]{\\ #1}
\newenvironment{proof}{\noindent\textbf{Proof\ }}{\hspace*{\fill}$\Box$\medskip}
\newtheorem{corollary}{Corollary}
\newtheorem{definition}{Definition}
\newtheorem{theorem}{Theorem}
%%%%%%%%%% End TeXmacs macros

\begin{document}

\title{Oscillatory Processes with Monotonic Phase Functions}

\author{
  Stephen Crowley
  \tmaffiliation{July 24, 2025}
}

\maketitle

\begin{definition}
  [Oscillatory Process] Let $\{X_t \}_{t \in \mathbb{R}}$ be a complex
  second-order process. $\{X_t \}$ is called oscillatory if
  \begin{equation}
    X_t = \int_{- \infty}^{\infty} \phi_t (\omega)  \hspace{0.17em} dZ
    (\omega)
  \end{equation}
  where $Z (\omega)$ has orthogonal increments where
  \begin{equation}
    E |dZ (\omega) |^2 = d \mu (\omega)
  \end{equation}
  and
  \begin{equation}
    \phi_t (\omega) = A_t (\omega) e^{i \omega t}
  \end{equation}
\end{definition}

\begin{theorem}
  Let $\theta : \mathbb{R} \to \mathbb{R}$ be smooth and strictly
  monotonically increasing. Define
  \begin{equation}
    \phi_t (\omega) = e^{i \omega \theta (t)}
  \end{equation}
  Set
  \begin{equation}
    A_t (\omega) = e^{i \omega (\theta (t) - t)}
  \end{equation}
  Then the time-varying impulse response $h_t (u)$ defined by
  \begin{equation}
    A_t (\omega) = \int_{- \infty}^{\infty} e^{i \omega u} h_t (u) 
    \hspace{0.17em} du
  \end{equation}
  is
  \begin{equation}
    h_t (u) = \delta (u - [\theta (t) - t])
  \end{equation}
\end{theorem}

\begin{proof}
  The inverse Fourier transform yields
  \begin{equation}
    \begin{array}{ll}
      h_t (u) & = \frac{1}{2 \pi}  \int_{- \infty}^{\infty} e^{- i \omega u}
      A_t (\omega)  \hspace{0.17em} d \omega\\
      & = \frac{1}{2 \pi}  \int_{- \infty}^{\infty} e^{- i \omega u} e^{i
      \omega (\theta (t) - t)} d \omega\\
      & = \frac{1}{2 \pi}  \int_{- \infty}^{\infty} e^{i \omega (\theta (t) -
      t - u)} d \omega\\
      & = \delta (\theta (t) - t - u)\\
      & = \delta (u - [\theta (t) - t])
    \end{array}
  \end{equation}
  
\end{proof}

\begin{corollary}
  Let $S_t = \int_{- \infty}^{\infty} e^{i \omega t} dZ (\omega)$. Then
  \begin{equation}
    X_t = \int_{- \infty}^{\infty} S_{t - u}  \hspace{0.17em} h_t (u) 
    \hspace{0.17em} du = S_{2 t - \theta (t)}
  \end{equation}
  and
  \begin{equation}
    X_t = \int_{- \infty}^{\infty} e^{i \omega \theta (t)}  \hspace{0.17em} dZ
    (\omega)
  \end{equation}
\end{corollary}

\begin{proof}
  By the sifting property of the Dirac delta,
  \begin{equation}
    \begin{array}{ll}
      X_t & = \int_{- \infty}^{\infty} S_{t - u} \delta (u - [\theta (t) - t])
      \hspace{0.17em} du\\
      & = S_{t - [\theta (t) - t]}\\
      & = S_{2 t - \theta (t)}
    \end{array}
  \end{equation}
  By direct substitution, the oscillatory representation holds by definition.
\end{proof}

\end{document}
