\documentclass{article}
\usepackage{amsmath, amssymb, amsthm}
\usepackage{hyperref}
\usepackage{enumitem}

\title{A UNIQUENESS PROBLEM FOR THE ENVELOPE OF AN OSCILLATORY PROCESS}
\author{A. M. HASOFER\thanks{Postal address: School of Mathematics, Department of Statistics, The University of New South Wales, P.O.Box 1, Kensington, N.S.W. 2033, Australia.}}
\date{}

\begin{document}
\maketitle

\begin{abstract}
In a previous paper, the author has described a method for obtaining envelope processes for oscillatory stochastic processes. These are processes which can be represented as the output of a time-varying linear filter whose input is a stationary process.

It is shown in this paper that the proposed definition of the envelope process may not be unique, but may depend on the particular representation of the oscillatory process chosen.

It is then shown that for a class of oscillatory processes which is of particular interest, the class of transient processes, there is a class of natural representations which all lead to a unique envelope process.

NON-STATIONARY STOCHASTIC PROCESSES; TIME-VARYING FILTERS; ENVELOPE PROCESSES
\end{abstract}

\section*{Introduction}
In a previous paper~\cite{hasoferpetocz}, the author described a method for obtaining an envelope process for a class of non-stationary stochastic processes. This is the class of oscillatory processes introduced by Priestley~\cite{priestley}. For an oscillatory process $Y(t)$, one defines a `quadrature process' $\hat{Y}(t)$. The envelope process, $R(t)$, is then defined by
\begin{equation}
R(t) = \sqrt{Y^2(t) + \hat{Y}^2(t)}.
\label{eq:envelope}
\end{equation}

It has come to the author's notice that similar definitions for the envelope were given by Arens in 1957~\cite{arens}, and Yang in 1972~\cite{yang}.

Further analysis has, however, uncovered the fact that the proposed definition of the quadrature process may not be unique, but may depend on the particular representation chosen for $Y(t)$ in terms of an underlying stationary process. This lack of uniqueness is exhibited by a counterexample.

\footnotetext{Received 20 September 1978; revision received 21 November 1978.}

It is then shown that for a class of oscillatory processes which is of particular interest, the class of transient processes, there is a class of natural representations which all lead to a unique quadrature process.

\section*{Background and notation}

Let $X(t)$ be a real-valued stationary stochastic process with zero mean and finite variance. Let its spectral representation be
\begin{equation}
X(t) = \int_0^{\infty} \cos \lambda t\, dU(\lambda) + \sin \lambda t\, dV(\lambda).
\label{eq:spectral}
\end{equation}
(See Hasofer and Petocz~\cite{hasoferpetocz}.)

It is well known that $X(t)$ uniquely determines the two uncorrelated processes with orthogonal increments $U(\lambda)$ and $V(\lambda)$, e.g. through the formulae~\cite[ p. 136]{cramerleadbetter}
\begin{align}
U(\lambda) &= \lim_{T \to \infty} \frac{1}{\pi} \int_{-T}^{T} \frac{\sin \lambda t}{t} X(t)\, dt, \label{eq:Ulambda} \\
V(\lambda) &= \lim_{T \to \infty} \frac{1}{\pi} \int_{-T}^{T} \frac{1 - \cos \lambda t}{t} X(t)\, dt. \label{eq:Vlambda}
\end{align}

The Hilbert transform of $X(t)$, $\hat{X}(t)$, is defined by the formula
\begin{equation}
\hat{X}(t) = \int_0^{\infty} \sin \lambda t\, dU(\lambda) - \cos \lambda t\, dV(\lambda).
\label{eq:hilbert}
\end{equation}

It is to be noted that $\hat{X}(t)$ defines $X(t)$ uniquely. This is easily seen from the fact that the Hilbert transform of $\hat{X}(t)$ is $-X(t)$.

Next we define the oscillatory process $Y(t)$ by the formula
\begin{equation}
Y(t) = \int_{-\infty}^{+\infty} h(t, u) X(u)\, du
\label{eq:Ydef}
\end{equation}
where $h(t, u)$ is the impulse response function of a linear, non-time-invariant filter. This may be written in the form
\begin{equation}
Y(t) = \int_0^{\infty} \cos \lambda t\, dU^*(t, \lambda) + \sin \lambda t\, dV^*(t, \lambda)
\label{eq:Yspectral}
\end{equation}
(see~\cite{hasoferpetocz}).

We then define the quadrature process $\hat{Y}(t)$ by the equation
\begin{equation}
\hat{Y}(t) = \int_0^{\infty} \sin \lambda t\, dU^*(t, \lambda) - \cos \lambda t\, dV^*(t, \lambda).
\label{eq:Yquad}
\end{equation}
It is easy to see that this definition is equivalent to
\begin{equation}
\hat{Y}(t) = \int_{-\infty}^{\infty} h(t, u) \hat{X}(u)\, du.
\label{eq:Yquadequiv}
\end{equation}

Finally, it is useful to have a definition of $\hat{X}(t), \hat{Y}(t)$ in terms of the complex-valued representation of $X(t)$.

Let
\begin{equation}
X(t) = \int_{-\infty}^{+\infty} e^{i t \lambda} d\zeta(\lambda).
\label{eq:Xcomplex}
\end{equation}

Assuming that $\zeta(\lambda)$ has a.s. no jump at the origin (corresponding to no jump for the spectrum $F(\lambda)$ of $X(t)$), we have
\begin{equation}
\hat{X}(t) = \operatorname{Im} 2 \int_0^{+\infty} e^{i t \lambda} d\zeta(\lambda).
\label{eq:Xhatcomplex}
\end{equation}

Further we can write
\begin{equation}
Y(t) = \int_{-\infty}^{+\infty} A(t, \lambda) e^{i t \lambda} d\zeta(\lambda)
\label{eq:Ycomplex}
\end{equation}
where
\begin{equation}
A(t, \lambda) e^{i t \lambda} = \int_{-\infty}^{+\infty} h(t, u) e^{i u \lambda} du.
\label{eq:Adef}
\end{equation}

We then easily see that
\begin{equation}
\hat{Y}(t) = \operatorname{Im} 2 \int_0^{\infty} A(t, \lambda) e^{i t \lambda} d\zeta(\lambda).
\label{eq:Yhatcomplex}
\end{equation}

There is a difficulty about the definition~\eqref{eq:Yquadequiv} in the following sense. Suppose that there exists another representation of $Y(t)$ of the form
\begin{equation}
Y(t) = \int_{-\infty}^{+\infty} h'(t, u) X'(u)\, du
\label{eq:Yalt}
\end{equation}
where $X'(t)$ is a stationary stochastic process with zero mean and finite variance. The existence of an infinity of representations of $Y(t)$ of the form~\eqref{eq:Yalt} has been pointed out by Priestley~\cite[ p. 205]{priestley}. The quadrature process of $Y(t)$ derived from the representation~\eqref{eq:Yalt} is
\begin{equation}
\hat{Y}'(t) = \int_{-\infty}^{+\infty} h'(t, u) \hat{X}'(u)\, du.
\label{eq:Yquadprime}
\end{equation}

The question then arises: is $\hat{Y}'(t)$ equal to $\hat{Y}(t)$, or in other words, is the definition of the quadrature process given in~\eqref{eq:Yquadequiv} independent of the particular representation of $Y(t)$ as the output of a linear, non-time-invariant filter with a stationary input process?

In this note, we shall answer this question in the negative. In fact we shall exhibit a counterexample, which yields two completely different quadrature processes $\hat{Y}(t)$ and $\hat{Y}'(t)$ for two different representations of $Y(t)$.

However, the lack of invariance of $\hat{Y}(t)$ does not necessarily invalidate its use. In fact we shall show that for a particularly interesting class of oscillatory processes, namely the class of 'transient' processes, there is a class of natural representations, all of which lead to the same quadrature process, which can then be taken as the natural one.

\section*{The counterexample}

Let $(\xi_1, \xi_2, \xi_3, \xi_4)$ be four independent standard normal random variables. We consider the stochastic process $X(t)$ defined by
\begin{equation}
X(t) = \xi_1 \cos t + \xi_2 \cos 2t + \xi_3 \sin t + \xi_4 \sin 2t.
\label{eq:Xcounter}
\end{equation}

This process can be easily checked to have zero mean and finite variance and to be stationary.

The Hilbert space spanned by $X(t)$ (see Cramér and Leadbetter~\cite[ p. 105]{cramerleadbetter}) is in this case just an ordinary four-dimensional Euclidian space, and all linear operators in this space are four-by-four matrices. In the sequel of this section we shall use the Hilbert space operator notation interchangeably with matrix notation.

Let $Y(t)$ be defined by
\begin{equation}
Y(t) = \int_{-\infty}^{+\infty} h(t, u) X(u)\, du
\label{eq:Ycounter}
\end{equation}
where $h(t, u)$ is the impulse response function of a linear, non-time-invariant filter.

To simplify the exposition, we introduce the following notation. Let $\boldsymbol{\xi} = (\xi_1, \xi_2, \xi_3, \xi_4)'$ be the column vector of the four random variables $\xi_1, \xi_2, \xi_3, \xi_4$.

Let $F$ denote the mapping $\boldsymbol{\xi} \rightarrow X(t)$. As pointed out above, this mapping is invertible. In the present case, this is almost obvious. For instance
\begin{equation}
\xi_2 = \frac{1}{\pi} \int_{-\pi}^{+\pi} X(t) \cos 2t\, dt.
\label{eq:xi2}
\end{equation}

Let $H$ denote the mapping $X(t) \rightarrow \hat{X}(t)$. In our case, we have
\begin{equation}
\hat{X}(t) = \xi_1 \sin t + \xi_2 \sin 2t - \xi_3 \cos t - \xi_4 \cos 2t.
\label{eq:Xhatcounter}
\end{equation}

Finally let $K$ denote the operator representing the non-time-invariant filter. Thus we can write
\begin{align}
Y &= K F \xi \label{eq:YKFxi} \\
\hat{Y} &= K H F \xi \label{eq:YKHFXI}
\end{align}

Now let $T$ denote a four-by-four orthogonal matrix. Then the vector $\xi' = T \xi$ will still consist of four independent standard normal variables. We now note that we can write $Y$ in the form
\begin{equation}
Y = K F T^{-1} F^{-1} F T \xi
\label{eq:Yaltform1}
\end{equation}
or
\begin{equation}
Y = K' F \xi'
\label{eq:Yaltform2}
\end{equation}
where
\begin{equation}
K' = K F T^{-1} F^{-1}
\label{eq:Kprime}
\end{equation}
and
\begin{equation}
\xi' = T \xi.
\label{eq:xiprime}
\end{equation}

This constitutes an alternative representation to~\eqref{eq:YKFxi}. Corresponding to it, we have
\begin{equation}
\hat{Y}' = K' H F \xi'.
\label{eq:Yhatprime}
\end{equation}

We shall take as $T$ the well-known orthogonal matrix
\begin{equation}
T = \frac{1}{2}
\begin{bmatrix}
1 & 1 & 1 & 1 \\
-1 & -1 & 1 & 1 \\
1 & -1 & -1 & 1 \\
-1 & 1 & -1 & 1
\end{bmatrix}
\label{eq:Tmatrix}
\end{equation}
and carry out the operations described above. We obtain
\begin{equation}
\hat{Y}(t) = \int_{-\infty}^{+\infty} h(t, u) \left[-\xi_3 \cos u - \xi_4 \cos 2u + \xi_1 \sin u + \xi_2 \sin 2u\right] du
\label{eq:Yhatcounter}
\end{equation}
while
\begin{equation}
\hat{Y}'(t) = \int_{-\infty}^{+\infty} h(t, u) \left[\xi_2 \cos u - \xi_1 \cos 2u - \xi_4 \sin u + \xi_3 \sin 2u\right] du
\label{eq:Yhatprimecounter}
\end{equation}

From these representations it is easy to calculate the cross-covariances, which turn out to be
\begin{align}
E[Y(u)\hat{Y}(v)] &= \int_{-\infty}^{+\infty} \int_{-\infty}^{+\infty} h(u, t_1) h(v, t_2) \left[\sin(t_2 - t_1) + \sin 2(t_2 - t_1)\right] dt_1 dt_2 \label{eq:cov1} \\
E[Y(u)\hat{Y}'(v)] &= \int_{-\infty}^{+\infty} \int_{-\infty}^{+\infty} h(u, t_1) h(v, t_2) \left[\cos(2 t_1 + t_2) - \cos(t_1 + 2 t_2)\right] dt_1 dt_2 \label{eq:cov2}
\end{align}
two utterly dissimilar covariances.

\section*{The class of transient processes}

We shall now consider a special class of oscillatory processes, which is of great practical interest, and for which there exists a class of 'natural' representations. We refer to the case of 'transient' processes. By this we mean processes which are asymptotically stationary. For example, if the process $Y(t)$ is the solution of a linear differential equation with given initial conditions, and a forcing function which is a stationary process $X(t)$, $Y(t)$ would often be asymptotically stationary, and we would call it a 'transient' process.

We shall make our definition precise in the following way. We assume that we have a family of oscillatory processes
\begin{equation}
Y(t_0; t) = \int_{-\infty}^{+\infty} h(t_0; t, u) X(u)\, du
\label{eq:Yt0t}
\end{equation}
(We think of $t_0$ as being the point at which the process $Y$ is initiated.)

We further assume that as $t_0$ tend to $-\infty$, $h(t_0; t, u)$ tends to a limit $h(t-u)$, and $Y(t_0; t)$ tends to a stationary process
\begin{equation}
Y(t) = \int_{-\infty}^{+\infty} h(t-u) X(u)\, du
\label{eq:Ystationary}
\end{equation}

We shall further assume that the kernel $h(t-u)$ is invertible. Writing for short
\begin{align}
Y_{t_0} &= K_{t_0} X \label{eq:Yt0K} \\
Y &= K X \label{eq:YK}
\end{align}
we have
\begin{equation}
X = K^{-1} Y
\label{eq:XfromY}
\end{equation}

Suppose now that there exists a second representation of $Y(t_0; t)$ of the form
\begin{equation}
Y(t_0; t) = \int_{-\infty}^{+\infty} h'(t_0; t, u) X'(u)\, du
\label{eq:Yt0tprime}
\end{equation}
which, as $t_0 \rightarrow -\infty$, tends to
\begin{equation}
Y(t) = \int_{-\infty}^{+\infty} h'(t-u) X'(u)\, du
\label{eq:Ytprime}
\end{equation}

We write these relations as
\begin{align}
Y_{t_0} &= K_{t_0}' X' \label{eq:Yt0Kprime} \\
Y &= K' X' \label{eq:YKprime}
\end{align}

We have, on account of~\eqref{eq:XfromY},
\begin{equation}
X = K^{-1} K' X'
\label{eq:XKprime}
\end{equation}

We now note that $K^{-1} K'$ is a time-invariant linear filter. It is easily verified that on account of this fact we have
\begin{equation}
\hat{X} = K^{-1} K' \hat{X}'
\label{eq:XhatKprime}
\end{equation}
for the Hilbert transformation commutes with time-invariant filters.

It follows from~\eqref{eq:XKprime} and~\eqref{eq:XhatKprime} that
\begin{equation}
Y_{t_0} = K_{t_0} K^{-1} K' X' = K_{t_0}' X'
\label{eq:Yt0KprimeX}
\end{equation}
and that
\begin{align}
\hat{Y}_{t_0} &= K_{t_0} \hat{X} \notag \\
&= K_{t_0} K^{-1} K' \hat{X}' \label{eq:YhatKprime}
\end{align}
while
\begin{equation}
\hat{Y}_{t_0}' = K_{t_0}' \hat{X}'
\label{eq:YhatprimeK}
\end{equation}

We use now the spectral representation of $X'$:
\begin{equation}
X'(t) = \int_{-\infty}^{+\infty} e^{i \lambda} d\zeta(\lambda)
\label{eq:XprimeSpectral}
\end{equation}
and obtain
\begin{align}
Y(t_0; t) &= \int_{-\infty}^{+\infty} A'(t_0; t, \lambda) e^{i t \lambda} d\zeta(\lambda) \label{eq:Yt0A} \\
&= \int_{-\infty}^{+\infty} A''(t_0; t, \lambda) e^{i t \lambda} d\zeta(\lambda) \label{eq:Yt0Aprime}
\end{align}
where $A'$ corresponds to the operator $K_{t_0} K^{-1} K'$ and $A''$ corresponds to $K_{t_0}'$.

It easily follows that
\begin{equation}
\int_{-\infty}^{+\infty} |A' - A''|^2 dF'(\lambda) = 0 \quad \text{for each } t
\label{eq:AAdiff}
\end{equation}
where $F'(\lambda)$ is the spectral distribution of $X'$.

Finally we have
\begin{align}
\hat{Y}(t_0; t) &= \operatorname{Im} 2 \int_0^{\infty} A'(t_0; t, \lambda) e^{i t \lambda} d\zeta(\lambda) \label{eq:YhatA} \\
\hat{Y}'(t_0; t) &= \operatorname{Im} 2 \int_0^{\infty} A''(t_0; t, \lambda) e^{i t \lambda} d\zeta(\lambda) \label{eq:YhatAprime}
\end{align}
from which it follows that $\hat{Y}(t_0; t) = \hat{Y}'(t_0, t)$. Thus uniqueness of the quadrature process is established for all representations of $Y(t_0, t)$ satisfying the stated assumptions, and it can be taken as the natural quadrature process.

As an example of a transient process, consider the process $Y(t)$ which is the solution of the second-order linear differential equation with constant coefficients
\begin{equation}
\frac{d^2 Y}{dt^2} + 2 \zeta \omega_0 \frac{dY}{dt} + \omega_0^2 Y = X(t)
\label{eq:ODE}
\end{equation}
where $X(t)$ is a stationary process with zero mean and finite variance. Initial conditions are
\begin{equation}
Y(t_0) = 0, \quad \frac{dY(t_0)}{dt} = 0
\label{eq:ODEinit}
\end{equation}

The solution can be written as
\begin{equation}
Y(t_0, t) = \int_{t_0}^t X(\tau) h(t - \tau) d\tau, \quad (t \geq t_0)
\label{eq:ODEsol}
\end{equation}
where
\begin{equation}
h(t) = \frac{1}{\omega_1} e^{-\zeta \omega_0 t} \sin \omega_1 t
\label{eq:ODEh}
\end{equation}
and $\omega_1 = \omega_0 \sqrt{1 - \zeta^2}$.

It is easy to verify that all the conditions specified above are fulfilled and thus in this case, a `natural' representation to use for obtaining the quadrature process is~\eqref{eq:Yt0t}.

\section*{References}
\begin{enumerate}[label={[\arabic*]}]
\item\label{arens} Arens, R. (1957) Complex processes for envelopes of normal noise. IRE Trans. Inf. Theory 3, 204-207.
\item\label{cramerleadbetter} Cramér, H. and Leadbetter, M. R. (1967) Stationary and Related Stochastic Processes. Wiley, New York.
\item\label{hasoferpetocz} Hasofer, A. M. and Petocz, P. (1978) The envelope of an oscillatory process and its upcrossings (abstract). Adv. Appl. Prob. 10, 711-716.
\item\label{priestley} Priestley, M. B. (1965) Evolutionary spectra and non-stationary processes. J. R. Statist. Soc. B 27, 204-229.
\item\label{yang} Yang, J. N. (1972) Non-stationary envelope process and first-excursion probability. J. Structural Mech. 1, 231-248.
\end{enumerate}

\end{document}
