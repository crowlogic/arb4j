\documentclass{article}
\usepackage{amsmath}
\usepackage{amssymb}
\usepackage{amsthm}

% Define theorem-like environments
\newtheorem{theorem}{Theorem}
\newtheorem{lemma}[theorem]{Lemma}
\newtheorem{definition}{Definition}
\newtheorem{example}{Example}

\title{A UNIQUENESS PROBLEM FOR THE ENVELOPE\\
OF AN OSCILLATORY PROCESS}
\author{A. M. HASOFER\thanks{School of Mathematics, Department of Statistics, The University of New South Wales, P.O.Box 1, Kensington, N.S.W. 2033, Australia.}}
\date{\textit{J. Appl. Prob.} 16, 822-829 (1979)\\Received 20 September 1978; revision received 21 November 1978.}

\begin{document}

\maketitle

\begin{abstract}
In a previous paper, the author has described a method for obtaining
envelope processes for oscillatory stochastic processes. These are processes
which can be represented as the output of a time-varying linear filter whose
input is a stationary process.

It is shown in this paper that the proposed definition of the envelope process
may not be unique, but may depend on the particular representation of the
oscillatory process chosen.

It is then shown that for a class of oscillatory processes which is of particular
interest, the class of transient processes, there is a class of natural representa-
tions which all lead to a unique envelope process.
\end{abstract}

\noindent\textbf{NON-STATIONARY STOCHASTIC PROCESSES; TIME-VARYING FILTERS; ENVELOPE PROCESSES}

\section{Introduction}
In a previous paper \cite{hasofer1978}, the author described a method for obtaining an
envelope process for a class of non-stationary stochastic processes. This is the
class of oscillatory processes introduced by Priestley \cite{priestley1965}. For an oscillatory
process $Y(t)$, one defines a 'quadrature process' $\hat{Y}(t)$. The envelope process,
$R(t)$, is then defined by
\begin{equation}
\label{eq:envelope}
R(t) = \sqrt{Y^2(t) + \hat{Y}^2(t)}.
\end{equation}

It has come to the author's notice that similar definitions for the envelope
were given by Arens in 1957 \cite{arens1957}, and Yang in 1972 \cite{yang1972}.

Further analysis has, however, uncovered the fact that the proposed definition
of the quadrature process may not be unique, but may depend on the particular
representation chosen for $Y(t)$ in terms of an underlying stationary process. This
lack of uniqueness is exhibited by a counterexample.

It is then shown that for a class of oscillatory processes which is of particular
interest, the class of transient processes, there is a class of natural representa-
tions which all lead to a unique quadrature process.

\section{Background and notation}

Let $X(t)$ be a real-valued stationary stochastic process with zero mean and
finite variance. Let its spectral representation be
\begin{equation}
\label{eq:spectral}
X(t) = \int \cos \lambda t dU(\lambda) + \sin \lambda t dV(\lambda).
\end{equation}

(See Hasofer and Petocz \cite{hasofer1978}.)

It is well known that $X(t)$ uniquely determines the two uncorrelated processes
with orthogonal increments $U(\lambda)$ and $V(\lambda)$, e.g. through the formulae (\cite{cramer1967}, p.
136)
\begin{equation}
\label{eq:u_lambda}
U(\lambda) = \text{l.i.m.}_{T \to \infty} \frac{1}{\pi} \int_{-T}^{T} \frac{\sin \lambda t}{t} X(t)dt,
\end{equation}
\begin{equation}
\label{eq:v_lambda}
V(\lambda) = \text{l.i.m.}_{T \to \infty} \frac{1}{\pi} \int_{-T}^{T} \frac{1 - \cos \lambda t}{t} X(t)dt.
\end{equation}

The Hilbert transform of $X(t)$, $\hat{X}(t)$, is defined by the formula
\begin{equation}
\label{eq:hilbert}
\hat{X}(t) = \int \sin \lambda t dU(\lambda) - \cos \lambda t dV(\lambda).
\end{equation}

It is to be noted that $X(t)$ defines $\hat{X}(t)$ uniquely.
This is easily seen from the fact that the Hilbert transform of $\hat{X}(t)$ is $-X(t)$.

Next we define the oscillatory process $Y(t)$ by the formula
\begin{equation}
\label{eq:oscillatory}
Y(t) = \int h(t, u)X(u)du,
\end{equation}
where $h(t, u)$ is the impulse response function of a linear, non-time-invariant
filter. This may be written in the form
\begin{equation}
\label{eq:y_spectral}
Y(t) = \int \cos \lambda t dU^*(t, \lambda) + \sin \lambda t dV^*(t, \lambda)
\end{equation}
(see \cite{hasofer1978}).

We then define the quadrature process $\hat{Y}(t)$ by the equation
\begin{equation}
\label{eq:quadrature}
\hat{Y}(t) = \int \sin \lambda t dU^*(t, \lambda) - \cos \lambda t dV^*(t, \lambda).
\end{equation}

It is easy to see that this definition is equivalent to
\begin{equation}
\label{eq:quadrature_alt}
\hat{Y}(t) = \int h(t, u)\hat{X}(u)du.
\end{equation}

Finally, it is useful to have a definition of $\hat{X}(t)$, $\hat{Y}(t)$ in terms of the
complex-valued representation of $X(t)$.

Let
\begin{equation}
\label{eq:complex_rep}
X(t) = \int e^{it\lambda}d\zeta(\lambda).
\end{equation}

Assuming that $\zeta(\lambda)$ has a.s. no jump at the origin (corresponding to no jump
for the spectrum $F(\lambda)$ of $X(t)$), we have
\begin{equation}
\label{eq:x_hat_complex}
\hat{X}(t) = \text{Im } 2 \int e^{it\lambda}d\zeta(\lambda).
\end{equation}

Further we can write
\begin{equation}
\label{eq:y_complex}
Y(t) = \int A(t, \lambda)e^{it\lambda}d\zeta(\lambda)
\end{equation}
where
\begin{equation}
\label{eq:a_t_lambda}
A(t, \lambda)e^{it\lambda} = \int h(t, u)e^{i\lambda u}du.
\end{equation}

We then easily see that
\begin{equation}
\label{eq:y_hat_complex}
\hat{Y}(t) = \text{Im } 2 \int A(t, \lambda)e^{it\lambda}d\zeta(\lambda).
\end{equation}

There is a difficulty about the definition \eqref{eq:quadrature_alt} in the following sense. Suppose
that there exists another representation of $Y(t)$ of the form
\begin{equation}
\label{eq:alt_rep}
Y(t) = \int h'(t, u)X'(u)du,
\end{equation}
where $X'(t)$ is a stationary stochastic process with zero mean and finite variance.
The existence of an infinity of representations of $Y(t)$ of the form \eqref{eq:alt_rep} has been
pointed out by Priestley (\cite{priestley1965}, p. 205). The quadrature process of $Y(t)$ derived
from the representation \eqref{eq:alt_rep} is
\begin{equation}
\label{eq:alt_quadrature}
\hat{Y}'(t) = \int h'(t, u)\hat{X}'(u)du.
\end{equation}

The question then arises: is $\hat{Y}'(t)$ equal to $\hat{Y}(t)$, or in other words, is the
definition of the quadrature process given in \eqref{eq:quadrature_alt} independent of the particular
representation of $Y(t)$ as the output of a linear, non-time-invariant filter with a
stationary input process?

In this note, we shall answer this question in the negative. In fact we shall exhibit a counterexample, which yields two completely different quadrature
processes $\hat{Y}(t)$ and $\hat{Y}'(t)$ for two different representations of $Y(t)$.

However, the lack of invariance of $\hat{Y}(t)$ does not necessarily invalidate its use.
In fact we shall show that for a particularly interesting class of oscillatory
processes, namely the class of 'transient' processes, there is a class of natural
representations, all of which lead to the same quadrature process, which can
then be taken as the natural one.

\section{The counterexample}

Let $(\xi_1, \xi_2, \xi_3, \xi_4)$ be four independent standard normal random variables. We
consider the stochastic process $X(t)$ defined by
\begin{equation}
\label{eq:x_counter}
X(t) = \xi_1 \cos t + \xi_2 \cos 2t + \xi_3 \sin t + \xi_4 \sin 2t.
\end{equation}

This process can be easily checked to have zero mean and finite variance and to
be stationary.

The Hilbert space spanned by $X(t)$ (see Cramér and Leadbetter \cite{cramer1967}, p. 105) is
in this case just an ordinary four-dimensional Euclidian space, and all linear operators in this space are four-by-four matrices. In the sequel of this section we
shall use the Hilbert space operator notation interchangeably with matrix
notation.

Let $Y(t)$ be defined by
\begin{equation}
\label{eq:y_counter}
Y(t) = \int h(t, u)X(u)du,
\end{equation}
where $h(t, u)$ is the impulse response function of a linear, non-time-invariant
filter.

To simplify the exposition, we introduce the following notation. Let $\xi =
(\xi_1, \xi_2, \xi_3, \xi_4)'$ be the column vector of the four random variables $\xi_1, \xi_2, \xi_3, \xi_4$.
Let $F$ denote the mapping $\xi \to X(t)$. As pointed out above, this mapping is
invertible. In the present case, this is almost obvious. For instance
\begin{equation}
\label{eq:xi_example}
\xi_2 = \frac{1}{\pi} \int X(t) \cos 2t dt.
\end{equation}

Let $H$ denote the mapping $X(t) \to \hat{X}(t)$. In our case, we have
\begin{equation}
\label{eq:x_hat_counter}
\hat{X}(t) = \xi_1 \sin t + \xi_2 \sin 2t - \xi_3 \cos t - \xi_4 \cos 2t.
\end{equation}

Finally let $K$ denote the operator representing the non-time-invariant filter.
Thus we can write
\begin{align}
Y &= KF\xi, \label{eq:y_operator} \\
\hat{Y} &= KHF\xi. \label{eq:y_hat_operator}
\end{align}

Now let $T$ denote a four-by-four orthogonal matrix. Then the vector $\xi' = T\xi$
will still consist of four independent standard normal variables. We now note
that we can write $Y$ in the form
\begin{equation}
\label{eq:y_alt_form}
Y = KFT^{-1}T\xi,
\end{equation}
or
\begin{equation}
\label{eq:y_alt_form2}
Y = K'F\xi',
\end{equation}
where
\begin{equation}
\label{eq:k_prime}
K' = KFT^{-1}F^{-1}
\end{equation}
and
\begin{equation}
\label{eq:xi_prime}
\xi' = T\xi.
\end{equation}

This constitutes an alternative representation to \eqref{eq:y_operator}. Corresponding to it, we
have
\begin{equation}
\label{eq:y_hat_prime}
\hat{Y}' = K'HF\xi'.
\end{equation}

We shall take as $T$ the well-known orthogonal matrix
\begin{equation}
\label{eq:t_matrix}
T = \begin{pmatrix} 
1 & 1 & 1 & 1 \\
1 & -1 & 1 & -1 \\
1 & 1 & -1 & -1 \\
1 & -1 & -1 & 1 
\end{pmatrix}
\end{equation}
and carry out the operations described above. We obtain
\begin{equation}
\label{eq:y_hat_result}
\hat{Y}(t) = \int h(t, u) [-\xi_3 \cos u - \xi_4 \cos 2u + \xi_1 \sin u + \xi_2 \sin 2u]du
\end{equation}
while
\begin{equation}
\label{eq:y_hat_prime_result}
\hat{Y}'(t) = \int h(t, u) [\xi_2 \cos u - \xi_4 \cos 2u - \xi_4 \sin u + \xi_3 \sin 2u ]du.
\end{equation}

From these representations it is easy to calculate the cross-covariances, which
turn out to be
\begin{equation}
\label{eq:cov1}
E[Y(u)\hat{Y}(v)] = \int \int h(u, t_1) h(v, t_2) [\sin(t_2-t_1) + \sin 2(t_2-t_1)]dt_1dt_2
\end{equation}
\begin{equation}
\label{eq:cov2}
E[Y(u)\hat{Y}'(v)] = \int \int h(u, t_1) h(v, t_2) [\cos(2t_1 + t_2) - \cos (t_1 + 2t_2)]dt_1dt_2;
\end{equation}
two utterly dissimilar covariances.

\section{The class of transient processes}

We shall now consider a special class of oscillatory processes, which is of great
practical interest, and for which there exists a class of 'natural' representations.

We refer to the case of 'transient' processes. By this we mean processes which
are asymptotically stationary. For example, if the process $Y(t)$ is the solution of
a linear differential equation with given initial conditions, and a forcing function
which is a stationary process $X(t)$, $Y(t)$ would often be asymptotically stationary, and we would call it a 'transient' process.

We shall make our definition precise in the following way. We assume that we
have a family of oscillatory processes
\begin{equation}
\label{eq:y_t0}
Y(t_0; t) = \int h(t_0; t, u)X(u)du.
\end{equation}

(We think of $t_0$ as being the point at which the process $Y$ is initiated.)

We further assume that as $t_0$ tend to $-\infty$, $h(t_0; t, u)$ tends to a limit $h(t - u)$,
and $Y(t_0; t)$ tends to a stationary process
\begin{equation}
\label{eq:y_stationary}
Y(t) = \int h(t-u)X(u)du.
\end{equation}

We shall further assume that the kernel $h(t-u)$ is invertible. Writing for
short
\begin{align}
Y &= KX, \label{eq:y_short} \\
\hat{Y} &= K\hat{X}, \label{eq:y_hat_short}
\end{align}
we have
\begin{equation}
\label{eq:x_from_y}
X = K^{-1}Y.
\end{equation}

Suppose now that there exists a second representation of $Y(t_0; t)$ of the form
\begin{equation}
\label{eq:y_t0_alt}
Y(t_0; t) = \int h'(t_0; t, u)X'(u)du,
\end{equation}
which, as $t_0 \to -\infty$, tends to
\begin{equation}
\label{eq:y_stationary_alt}
Y(t) = \int h'(t-u)X'(u)du.
\end{equation}

We write these relations as
\begin{align}
Y &= K'X', \label{eq:y_short_alt} \\
\hat{Y} &= K'\hat{X}', \label{eq:y_hat_short_alt}
\end{align}

We have, on account of \eqref{eq:x_from_y},
\begin{equation}
\label{eq:x_rel}
X = K^{-1}K'X'.
\end{equation}

We now note that $K^{-1}K'$ is a time-invariant linear filter. It is easily verified
that on account of this fact we have
\begin{equation}
\label{eq:x_hat_rel}
\hat{X} = K^{-1}K'\hat{X}',
\end{equation}
for the Hilbert transformation commutes with time-invariant filters.

It follows from \eqref{eq:x_rel} and \eqref{eq:x_hat_rel} that
\begin{equation}
\label{eq:y_t0_rel}
Y(t_0) = KK^{-1}K'X' = K'X',
\end{equation}
and that
\begin{equation}
\label{eq:y_hat_t0_rel}
\hat{Y}(t_0) = K\hat{X} = KK^{-1}K'\hat{X}',
\end{equation}
while
\begin{equation}
\label{eq:y_hat_prime_t0}
\hat{Y}'(t_0) = K'\hat{X}'.
\end{equation}

We use now the spectral representation of $X'$:
\begin{equation}
\label{eq:x_prime_spectral}
X'(t) = \int e^{it\lambda}d\zeta(\lambda),
\end{equation}
and obtain
\begin{align}
Y(t_0; t) &= \int A'(t_0; t, \lambda)e^{it\lambda}d\zeta(\lambda), \label{eq:y_t0_spectral1} \\
&= \int A''(t_0; t, \lambda)e^{it\lambda}d\zeta(\lambda), \label{eq:y_t0_spectral2}
\end{align}
where $A'$ corresponds to the operator $KK^{-1}K'$ and $A''$ corresponds to $K'$.

It easily follows that
\begin{equation}
\label{eq:a_diff}
\int |A' - A''|^2 dF'(\lambda) = 0 \text{ for each } t,
\end{equation}
where $F'(\lambda)$ is the spectral distribution of $X'$. This means that $A'(t_0; t, \lambda) = A''(t_0; t, \lambda)$ for $F'$-almost every $\lambda$, and thus the two representations are essentially the same in the limit as $t_0 \to -\infty$.

Therefore, the quadrature process $\hat{Y}(t)$ is uniquely determined in the class of representations corresponding to asymptotically stationary (transient) processes with invertible, time-invariant limiting kernels.

\begin{theorem}\label{thm:uniqueness}
Let $Y(t)$ be a transient process, i.e., $Y(t) = \int h(t-u) X(u) du$ with $h$ invertible and $X$ stationary. Then, for any alternative representation $Y(t) = \int h'(t-u) X'(u) du$ with $X'$ stationary and $h'$ invertible, the quadrature process $\hat{Y}(t)$ defined via
\[
\hat{Y}(t) = \int h(t-u) \hat{X}(u) du
\]
is unique (up to $F'$-null sets), i.e., does not depend on the choice of representation.
\end{theorem}

\begin{proof}
Given two such representations, we have $X = K^{-1} K' X'$, where $K$ and $K'$ are the convolution operators associated with $h$ and $h'$, respectively. Since $K^{-1} K'$ is time-invariant, the Hilbert transform commutes with it:
\[
\hat{X} = K^{-1} K' \hat{X}'.
\]
Therefore,
\[
\hat{Y}(t) = K \hat{X} = K K^{-1} K' \hat{X}' = K' \hat{X}' = \hat{Y}'(t).
\]
Thus, the quadrature process is independent of the representation.
\end{proof}

\section{Conclusion}

We have shown that the definition of the quadrature process, and hence the envelope process, for a general oscillatory process may depend on the particular representation chosen for the process as the output of a linear, non-time-invariant filter with stationary input. However, for the important subclass of transient processes, there exists a class of natural representations (those with invertible, time-invariant limiting kernels) which all lead to the same quadrature process. Thus, in this case, the envelope process is uniquely defined.

\section*{Acknowledgements}

The author thanks Professor M. B. Priestley for helpful discussions.

\begin{thebibliography}{9}
\bibitem{hasofer1978}
A. M. Hasofer and A. Petocz (1978) ``Envelope processes for oscillatory stochastic processes,'' \textit{J. Appl. Prob.} \textbf{15}, 1--15.

\bibitem{priestley1965}
M. B. Priestley (1965) ``Evolutionary spectra and non-stationary processes,'' \textit{J. Roy. Statist. Soc. Ser. B} \textbf{27}, 204--237.

\bibitem{arens1957}
T. Arens (1957) ``Über die Hüllkurve zufälliger Schwingungen,'' \textit{Z. Angew. Math. Mech.} \textbf{37}, 385--398.

\bibitem{yang1972}
C. T. Yang (1972) ``On the envelope of a nonstationary random process,'' \textit{IEEE Trans. Inform. Theory} \textbf{18}, 603--606.

\bibitem{cramer1967}
H. Cramér and M. R. Leadbetter (1967) \textit{Stationary and Related Stochastic Processes}, Wiley, New York.
\end{thebibliography}

\end{document}
