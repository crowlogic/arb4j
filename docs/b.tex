\documentclass{article}
\usepackage[english]{babel}
\usepackage{amsmath,amssymb,latexsym,theorem}

%%%%%%%%%% Start TeXmacs macros
\newcommand{\cdummy}{\cdot}
\newcommand{\tmem}[1]{{\em #1\/}}
\newenvironment{proof}{\noindent\textbf{Proof\ }}{\hspace*{\fill}$\Box$\medskip}
\newtheorem{corollary}{Corollary}
\newtheorem{definition}{Definition}
\newtheorem{lemma}{Lemma}
\newtheorem{proposition}{Proposition}
{\theorembodyfont{\rmfamily}\newtheorem{remark}{Remark}}
\newtheorem{theorem}{Theorem}
%%%%%%%%%% End TeXmacs macros

\newcommand{\R}{\mathbb{R}}
\newcommand{\C}{\mathbb{C}}
\newcommand{\E}{\mathbb{E}}
\providecommand{\tmem}[1]{{\tmem{#1}}}

\begin{document}

\title{The Spectral Representation of Oscillatory Processes}

\author{Stephen Crowley}

\date{November 25, 2025}

\maketitle

{\tableofcontents}

\section{Definitions and Assumptions}

\begin{remark}
  Unless otherwise stated:
  \begin{enumerate}
    \item All parametric families $\{f_t (\omega)\}$ are jointly measurable
    with respect to $\mathcal{B} (\R) \otimes \mathcal{B} (\R)$.
    
    \item Dirac delta identities such as $\int e^{i (\mu - \lambda) u} 
    \hspace{0.17em} du = 2 \pi \delta (\mu - \lambda)$ are distributional.
    
    \item Integrals of the form $\int f (\omega)  \hspace{0.17em} d \nu
    (\omega)$ denote Lebesgue-Stieltjes integration with respect to measure
    $\nu$, while $\int g (u)  \hspace{0.17em} du$ denotes Lebesgue
    integration.
    
    \item Integrals with respect to orthogonal random measures $\Phi$, written
    $\int h (\omega)  \hspace{0.17em} d \Phi (\omega)$, are Lebesgue-Stieltjes
    integrals in $L^2 (\Omega)$ with variance $\E [| \int h \hspace{0.17em} d
    \Phi |^2] = \int |h|^2  \hspace{0.17em} d \mu$.
    
    \item Limit interchange is justified by dominated convergence under
    uniform $L^2$ bounds.
  \end{enumerate}
\end{remark}

Throughout, $(\Omega, \mathcal{F}, \mathbb{P})$ is a fixed probability space
and $\mu$ is a finite Borel measure on $\R$.

\begin{definition}
  [Oscillatory Function Family]\label{def:oscillatory}A family $\{\phi_t
  (\omega)\}_{t \in \R} \subset L^2 (\R, \mu)$ is called {\tmem{oscillatory}}
  if there exists a family $\{A_t (\omega)\}_{t \in \R, \omega \in \R}$
  satisfying:
  \begin{enumerate}
    \item {\tmem{Factorization:}} For all $t, \omega \in \R$,
    \[ \phi_t (\omega) = A_t (\omega) e^{i \omega t} . \]
    \item {\tmem{Amplitude Representation:}} For $\mu$-almost every $\omega$,
    there exists a probability measure $H_{\omega}$ on $\R$ with $\int |
    \lambda |  \hspace{0.17em} dH_{\omega} (\lambda) < \infty$ such that
    \[ A_t (\omega) = \int_{- \infty}^{\infty} e^{i \lambda t} 
       \hspace{0.17em} dH_{\omega} (\lambda)  \quad \text{for all } t \in \R .
    \]
    \item {\tmem{Concentration:}} For $\mu$-almost every $\omega$, the measure
    $H_{\omega}$ concentrates at the origin in the sense that
    \[ H_{\omega} (\{0\}) > H_{\omega}  (\R \setminus \{0\}), \]
    equivalently, the point mass at zero exceeds the total remaining mass.
    
    \item {\tmem{Normalization:}} For $\mu$-almost every $\omega$, $A_0
    (\omega) = 1$.
  \end{enumerate}
  The family $\{A_t (\omega)\}$ is called the {\tmem{amplitude family}}
  associated to $\{\phi_t \}$.
\end{definition}

\begin{proposition}
  [Non-vanishing from Concentration]\label{prop:nonvanishing}Under the
  concentration condition, the amplitude satisfies $A_t (\omega) \neq 0$ for
  all $t \in \R$ and $\mu$-almost every $\omega$.
\end{proposition}

\begin{proof}
  Fix $\omega$ in the full-measure set where the representation and
  concentration properties hold. Write
  \[ H_{\omega} = h_0  \hspace{0.17em} \delta_0 + \tilde{H}_{\omega}, \]
  where $h_0 = H_{\omega} (\{0\})$ and $\tilde{H}_{\omega}$ is supported on
  $\R \setminus \{0\}$ with total mass
  \[ m = \tilde{H}_{\omega}  (\R \setminus \{0\}) = H_{\omega}  (\R \setminus
     \{0\}) . \]
  The concentration condition gives $h_0 > m$. The Fourier--Stieltjes
  representation of $A_t (\omega)$ becomes
  \[ A_t (\omega) = \int_{\R} e^{i \lambda t}  \hspace{0.17em} dH_{\omega}
     (\lambda) = h_0 + \int_{\R \setminus \{0\}} e^{i \lambda t} 
     \hspace{0.17em} d \tilde{H}_{\omega} (\lambda) . \]
  Since $|e^{i \lambda t} | = 1$ for all $\lambda, t \in \R$, the triangle
  inequality for complex integrals yields
  \[ \left| \int_{\R \setminus \{0\}} e^{i \lambda t}  \hspace{0.17em} d
     \tilde{H}_{\omega} (\lambda) \right| \leq \int_{\R \setminus \{0\}} |e^{i
     \lambda t} |  \hspace{0.17em} d \tilde{H}_{\omega} (\lambda) = \int_{\R
     \setminus \{0\}} d \tilde{H}_{\omega} (\lambda) = m. \]
  The real part of $A_t (\omega)$ can be bounded from below:
  \[ \Re A_t (\omega) = \Re \left( h_0 + \int_{\R \setminus \{0\}} e^{i
     \lambda t}  \hspace{0.17em} d \tilde{H}_{\omega} (\lambda) \right) \geq
     h_0 - \left| \int_{\R \setminus \{0\}} e^{i \lambda t}  \hspace{0.17em} d
     \tilde{H}_{\omega} (\lambda) \right| \geq h_0 - m > 0 \]
  for all $t \in \R$. A complex number with strictly positive real part cannot
  be zero, so $A_t (\omega) \neq 0$ for all $t \in \R$ and $\mu$-a.e.
  $\omega$.
\end{proof}

\begin{definition}
  [Oscillatory Stochastic Process]\label{def:oscillatory-process}A centered
  stochastic process $\{X_t \}_{t \in \R}$ is {\tmem{oscillatory}} if there
  exists:
  \begin{enumerate}
    \item a finite Borel measure $\mu$ on $\R$,
    
    \item a complex orthogonal random measure $\Phi$ on $(\R,
    \mathcal{B}(\R))$ with $\E [| \Phi (B) |^2] = \mu (B)$,
    
    \item an amplitude family $\{A_t (\omega)\}$ satisfying
    Definition~\ref{def:oscillatory},
  \end{enumerate}
  such that for each $t \in \R$,
  \[ X_t = \int_{\R} A_t (\omega) e^{i \omega t}  \hspace{0.17em} d \Phi
     (\omega)  \quad \text{in } L^2 (\Omega) . \]
\end{definition}

\section{Existence and Regularity}

\begin{lemma}
  [Existence and Regularity of Oscillatory
  Processes]\label{lem:existence-regularity}Let $\{X_t \}_{t \in \R}$ be a
  centered stochastic process with covariance function $R_{s, t} = \E [X_s
  \overline{X_t}]$. Suppose $R_{s, t}$ admits a representation
  \[ R_{s, t} = \int_{\R} \phi_s (\omega) \overline{\phi_t (\omega)}
     \hspace{0.17em} d \mu (\omega) \]
  for some finite Borel measure $\mu$ and an oscillatory function family
  $\{\phi_t \}$ in the sense of Definition~\ref{def:oscillatory} with
  associated amplitude family $\{A_t \}$. Then the process admits the spectral
  representation
  \[ X_t = \int_{\R} A_t (\omega) e^{i \omega t}  \hspace{0.17em} d \Phi
     (\omega) \]
  where $\Phi$ is a complex orthogonal random measure with $\E [| \Phi (B)
  |^2] = \mu (B)$, and the amplitude family satisfies the following regularity
  conditions:
  \begin{enumerate}
    \item \label{cond:nonvanishing} {\tmem{Non-vanishing:}} For $\mu$-almost
    every $\omega \in \R$,
    \[ A_t (\omega) \neq 0 \quad \text{for all } t \in \R . \]
    \item \label{cond:fourier} {\tmem{Fourier--Stieltjes structure:}} For
    $\mu$-almost every $\omega \in \R$, the representation
    \[ A_t (\omega) = \int_{- \infty}^{\infty} e^{i \lambda t} 
       \hspace{0.17em} dH_{\omega} (\lambda) \]
    holds with $H_{\omega} (\R) = 1$, $H_{\omega} (\{0\}) > H_{\omega}  (\R
    \setminus \{0\})$, and $\int | \lambda |  \hspace{0.17em} dH_{\omega}
    (\lambda) < \infty$.
    
    \item \label{cond:l2integrability} {\tmem{Quadratic integrability:}} For
    each fixed $t \in \R$,
    \[ \int_{\R} |A_t (\omega) |^2  \hspace{0.17em} d \mu (\omega) < \infty .
    \]
    \item \label{cond:normalization} {\tmem{Normalization:}} For $\mu$-almost
    every $\omega \in \R$,
    \[ A_0 (\omega) = 1. \]
  \end{enumerate}
\end{lemma}

\begin{proof}
  Factorization and existence of $\Phi$ follow from
  Definition~\ref{def:oscillatory} and the spectral theorem for covariance
  kernels. The non-vanishing property is given by
  Proposition~\ref{prop:nonvanishing}.
  
  For quadratic integrability, for $\mu$-a.e. $\omega$,
  \[ |A_t (\omega) | \leq \int |e^{i \lambda t} |  \hspace{0.17em} dH_{\omega}
     (\lambda) = H_{\omega} (\R) = 1. \]
  Therefore,
  \[ \int_{\R} |A_t (\omega) |^2  \hspace{0.17em} d \mu (\omega) \leq
     \int_{\R} 1 \hspace{0.17em} d \mu (\omega) = \mu (\R) < \infty, \]
  which proves condition~\ref{cond:l2integrability} and ensures the stochastic
  integral defining $X_t$ is well-defined in $L^2 (\Omega)$.
  
  For covariance verification, use orthogonality of $\Phi$ and the isometry
  property of stochastic integrals:
  
  \begin{align*}
    \E [X_s \overline{X_t}] & = \E \left[ \int_{\R} A_s (\omega) e^{i \omega
    s}  \hspace{0.17em} d \Phi (\omega) \cdot \overline{\int_{\R} A_t (\nu)
    e^{i \nu t}  \hspace{0.17em} d \Phi (\nu)} \right]\\
    & = \int_{\R} A_s (\omega) \overline{A_t (\omega)} e^{i \omega (s - t)} 
    \hspace{0.17em} d \E [| \Phi (\omega) |^2]\\
    & = \int_{\R} A_s (\omega) \overline{A_t (\omega)} e^{i \omega (s - t)} 
    \hspace{0.17em} d \mu (\omega)\\
    & = R_{s, t} .
  \end{align*}
\end{proof}

\begin{corollary}
  [Derived Regularity Properties]\label{cor:regularity}Under the conditions of
  Lemma~\ref{lem:existence-regularity}, the following hold:
  \begin{enumerate}
    \item \label{prop:local-bound} For any compact interval $[a, b] \subset
    \R$ and Borel set $E$ with $\mu (E) < \infty$,
    \[ \sup_{t \in [a, b]}  \int_E |A_t (\omega) |^2  \hspace{0.17em} d \mu
       (\omega) \leq \mu (E) . \]
    \item \label{prop:differentiable} For $\mu$-almost every $\omega \in \R$,
    the temporal derivative
    \[ \frac{\partial A_t (\omega)}{\partial t} = \int_{- \infty}^{\infty} i
       \lambda e^{i \lambda t}  \hspace{0.17em} dH_{\omega} (\lambda) \]
    exists for all $t \in \R$ and is continuous in $t$.
  \end{enumerate}
\end{corollary}

\begin{proof}
  For Item~\ref{prop:local-bound}, Lemma~\ref{lem:existence-regularity}
  implies $|A_t (\omega) | \leq 1$ for $\mu$-a.e. $\omega$ and all $t \in \R$.
  Therefore,
  \[ \sup_{t \in [a, b]}  \int_E |A_t (\omega) |^2  \hspace{0.17em} d \mu
     (\omega) \leq \int_E 1 \hspace{0.17em} d \mu (\omega) = \mu (E) < \infty
     . \]
  For Item~\ref{prop:differentiable}, the Fourier--Stieltjes representation
  gives
  \[ A_t (\omega) = \int_{- \infty}^{\infty} e^{i \lambda t}  \hspace{0.17em}
     dH_{\omega} (\lambda) . \]
  Differentiation under the integral sign is justified by the finite first
  moment condition $\int | \lambda |  \hspace{0.17em} dH_{\omega} (\lambda) <
  \infty$ and dominated convergence. For any $t \in \R$ and $h \neq 0$,
  \[ \frac{A_{t + h} (\omega) - A_t (\omega)}{h} = \int_{- \infty}^{\infty}
     e^{i \lambda t}  \frac{e^{i \lambda h} - 1}{h}  \hspace{0.17em}
     dH_{\omega} (\lambda) . \]
  Since $| \frac{e^{i \lambda h} - 1}{h} | \leq | \lambda |$ for all $h \neq
  0$ and $\int | \lambda |  \hspace{0.17em} dH_{\omega} (\lambda) < \infty$,
  dominated convergence gives
  \[ \lim_{h \to 0}  \frac{A_{t + h} (\omega) - A_t (\omega)}{h} = \int_{-
     \infty}^{\infty} i \lambda e^{i \lambda t}  \hspace{0.17em} dH_{\omega}
     (\lambda) . \]
  Continuity of the derivative follows from the same argument applied to the
  derivative itself.
\end{proof}

\section{Spectral Representation Theorems}

\begin{theorem}
  [Existence of Spectral Representation]\label{thm:existence}Let $\{X_t \}$ be
  a centered oscillatory process with covariance function $R_{s, t} = \E [X_s
  \overline{X_t}]$. Then there exist a finite Borel measure $\mu$ on $\R$, a
  family of oscillatory functions $\{A_t (\omega)\}$ satisfying
  Lemma~\ref{lem:existence-regularity}, and a complex orthogonal random
  measure $\Phi$ with $\E [| \Phi (B) |^2] = \mu (B)$, such that
  \[ X_t = \int_{\R} A_t (\omega) e^{i \omega t}  \hspace{0.17em} d \Phi
     (\omega) \]
  and
  \[ R_{s, t} = \int_{\R} A_s (\omega) \overline{A_t (\omega)} e^{i \omega (s
     - t)}  \hspace{0.17em} d \mu (\omega) . \]
\end{theorem}

\begin{proof}
  By Definition~\ref{def:oscillatory-process}, an oscillatory process has a
  representation $X_t = \int_{\R} A_t (\omega) e^{i \omega t}  \hspace{0.17em}
  d \Phi (\omega)$. The covariance formula follows from the covariance
  verification in Lemma~\ref{lem:existence-regularity}. The existence of such
  a representation is guaranteed by the construction in that lemma.
\end{proof}

\begin{theorem}
  [Uniqueness of the Spectral Triple]\label{thm:uniqueness}The triple $(\mu,
  \Phi, A_t)$ is unique pathwise up to a scalar multiple: if two triples
  $(\mu_1, \Phi_1, A_t^{(1)})$ and $(\mu_2, \Phi_2, A_t^{(2)})$ generate the
  same process $\{X_t \}$, then $\mu_1 = \mu_2$ (modulo null sets), $A_t^{(1)}
  (\omega) = A_t^{(2)} (\omega)$ for $\mu$-almost every $\omega$, and there
  exists $c \in \C$ with $|c| = 1$ such that $\Phi_2 (B) = c \Phi_1 (B)$ for
  all Borel sets $B$.
\end{theorem}

\begin{proof}
  Suppose
  \[ X_t = \int_{\R} A_t^{(1)} (\omega) e^{i \omega t}  \hspace{0.17em} d
     \Phi_1 (\omega) = \int_{\R} A_t^{(2)} (\omega) e^{i \omega t} 
     \hspace{0.17em} d \Phi_2 (\omega) . \]
  Setting $s = t$ gives
  \[ \int_{\R} |A_t^{(1)} (\omega) |^2  \hspace{0.17em} d \mu_1 (\omega) = \E
     [|X_t |^2] = \int_{\R} |A_t^{(2)} (\omega) |^2  \hspace{0.17em} d \mu_2
     (\omega) \]
  for all $t$. Taking $t = 0$ and using $A_0^{(1)} = A_0^{(2)} = 1$, one
  obtains $\mu_1 (\R) = \mu_2 (\R)$.
  
  For any Borel set $E$, define $Y_t^{(E)} = \int_E A_t^{(1)} (\omega) e^{i
  \omega t}  \hspace{0.17em} d \Phi_1 (\omega)$. By orthogonality,
  \[ \E [|Y_t^{(E)} |^2] = \int_E |A_t^{(1)} (\omega) |^2  \hspace{0.17em} d
     \mu_1 (\omega) . \]
  The same quantity computed using $(\mu_2, \Phi_2, A_t^{(2)})$ must be equal,
  so by polarization and the non-vanishing property, $\mu_1 (E) = \mu_2 (E)$
  is obtained. Hence $\mu_1 = \mu_2$ as measures.
  
  With $\mu_1 = \mu_2$, the equality of spectral representations implies that
  for each $t$,
  \[ \int_{\R} (A_t^{(1)} (\omega) - A_t^{(2)} (\omega)) e^{i \omega t} 
     \hspace{0.17em} d \Phi_1 (\omega) = 0 \quad \text{a.s.} \]
  By the non-vanishing property and the isometry, $A_t^{(1)} (\omega) =
  A_t^{(2)} (\omega)$ for $\mu$-a.e. $\omega$.
  
  Finally, since the amplitude family is unique, the random measures must
  satisfy $\Phi_2 (B) = c \Phi_1 (B)$ for some $c$ with $|c| = 1$ by the
  uniqueness of orthogonal random measures with prescribed variance.
\end{proof}

\section{The Shift Operator}

\begin{theorem}
  [Action of the Shift Operator]\label{thm:shift}Let $U_{\tau} : \mathcal{H}_T
  \to \mathcal{H}_T$ be the time-shift operator defined by $U_{\tau} X_t =
  X_{t + \tau}$. Under the spectral representation in
  Definition~\ref{def:oscillatory-process},
  \[ U_{\tau} X_t = X_{t + \tau} = \int_{\R} A_{t + \tau} (\omega) e^{i \omega
     (t + \tau)}  \hspace{0.17em} d \Phi (\omega) = \int_{\R} A_{t + \tau}
     (\omega) e^{i \omega t} e^{i \omega \tau}  \hspace{0.17em} d \Phi
     (\omega) . \]
\end{theorem}

\begin{proof}
  By Definition~\ref{def:oscillatory-process},
  \[ X_{t + \tau} = \int_{\R} A_{t + \tau} (\omega) e^{i \omega (t + \tau)} 
     \hspace{0.17em} d \Phi (\omega) . \]
  Factoring the exponential $e^{i \omega (t + \tau)} = e^{i \omega t} e^{i
  \omega \tau}$ and applying the isometry property of stochastic integrals:
  \[ \E \left| \int_{\R} A_{t + \tau} (\omega) e^{i \omega t} e^{i \omega
     \tau}  \hspace{0.17em} d \Phi (\omega) \right|^2 = \int_{\R} |A_{t +
     \tau} (\omega) |^2 |e^{i \omega \tau} |^2  \hspace{0.17em} d \mu (\omega)
     = \int_{\R} |A_{t + \tau} (\omega) |^2  \hspace{0.17em} d \mu (\omega) =
     \E [|X_{t + \tau} |^2] < \infty . \]
  Thus the representation holds in $L^2 (\Omega)$.
\end{proof}

\section{Time-Dependent Convolution Representation}

\begin{theorem}
  [Filter Representation]\label{thm:filter}Let $Y$ be a zero-mean stationary
  process with spectral representation
  \[ Y (u) = \int_{\R} e^{i \lambda u}  \hspace{0.17em} d \Psi (\lambda) \]
  and spectral measure with associated orthogonal random measure $\Psi$. Let
  $X$ be an oscillatory process with oscillatory function $\varphi_t (\lambda)
  = A_t (\lambda) e^{i \lambda t}$ and the same orthogonal random measure
  $\Psi$. Then
  \[ X (t) = \int_{- \infty}^{\infty} h (t, u) Y (t - u)  \hspace{0.17em} du
  \]
  where
  \[ h (t, u) = \frac{1}{2 \pi}  \int_{- \infty}^{\infty} A_t (\lambda) e^{- i
     \lambda u}  \hspace{0.17em} d \lambda \]
  is the time-dependent impulse response function.
\end{theorem}

\begin{proof}
  Substitute the definitions of $h (t, u)$ and $Y (u)$:
  
  \begin{align*}
    \int_{- \infty}^{\infty} h (t, u) Y (t - u)  \hspace{0.17em} du & =
    \int_{- \infty}^{\infty} \frac{1}{2 \pi}  \int_{- \infty}^{\infty} A_t
    (\lambda) e^{- i \lambda (t - u)}  \hspace{0.17em} d \lambda \int_{-
    \infty}^{\infty} e^{i \nu u}  \hspace{0.17em} d \Psi (\nu) 
    \hspace{0.17em} du
  \end{align*}
  
  By Fubini's theorem (justified by absolute convergence), the order of
  integration may be exchanged:
  
  \begin{align*}
    & = \frac{1}{2 \pi}  \int_{- \infty}^{\infty} \int_{- \infty}^{\infty}
    A_t (\lambda) \left[ \int_{- \infty}^{\infty} e^{i (\nu - \lambda) (t -
    u)}  \hspace{0.17em} du \right] d \lambda \hspace{0.17em} d \Psi (\nu)
  \end{align*}
  
  The inner integral over $u$ is the distributional identity $\int e^{i (\nu -
  \lambda)  (t - u)}  \hspace{0.17em} du = 2 \pi \delta (t (\nu - \lambda))$.
  Applying the sifting property:
  
  \begin{align*}
    & = \frac{1}{2 \pi}  \int_{- \infty}^{\infty} \int_{- \infty}^{\infty}
    A_t (\lambda) \cdot 2 \pi \delta (t (\nu - \lambda))  \hspace{0.17em} d
    \lambda \hspace{0.17em} d \Psi (\nu)\\
    & = \int_{- \infty}^{\infty} \int_{- \infty}^{\infty} A_t (\lambda)
    \delta (t (\nu - \lambda))  \hspace{0.17em} d \lambda \hspace{0.17em} d
    \Psi (\nu)\\
    & = \int_{- \infty}^{\infty} A_t (\nu) e^{i \nu t}  \hspace{0.17em} d
    \Psi (\nu)\\
    & = X (t) .
  \end{align*}
\end{proof}

\section{Isomorphism and Bidirectional Determination}

\begin{theorem}
  [Unitary Isomorphism]\label{thm:unitary}Define the map $U : \mathcal{H}_T
  \to L^2 (\R, \mu)$ by $U (X_t) (\omega) = A_t (\omega) e^{i \omega t}$. The
  map extends to a unitary isomorphism with
  \[ \langle X_s, X_t \rangle_{L^2 (\Omega)} = \langle U (X_s), U (X_t)
     \rangle_{L^2 (\R, \mu)} . \]
\end{theorem}

\begin{proof}
  The map $U$ is well-defined on the linear span $\mathcal{H}_0 =
  \mathrm{span} \{X_t : t \in \R \}$. For $Y = \sum_{j = 1}^n c_j X_{t_j}$,
  define
  \[ U (Y) (\omega) = \sum_{j = 1}^n c_j A_{t_j} (\omega) e^{i \omega t_j} .
  \]
  Linearity follows from the definition. The isometry property follows from
  the covariance calculation:
  \[ \langle X_s, X_t \rangle_{L^2 (\Omega)} = \E [X_s \overline{X_t}] =
     \int_{\R} A_s (\omega) \overline{A_t (\omega)} e^{i \omega (s - t)} 
     \hspace{0.17em} d \mu (\omega) = \langle U (X_s), U (X_t) \rangle_{L^2
     (\R, \mu)} . \]
  Thus $\|U (Y)\|_{L^2 (\mu)}^2 = \|Y\|_{L^2 (\Omega)}^2$ for all $Y \in
  \mathcal{H}_0$, so $U$ is an isometry. Since $\mathcal{H}_0$ is dense in
  $\mathcal{H}_T$ and the image $U (\mathcal{H}_0)$ contains all functions of
  the form $\sum c_j A_{t_j} (\omega) e^{i \omega t_j}$, which are dense in
  $L^2 (\R, \mu)$ by non-vanishing, $U$ extends uniquely to a unitary
  isomorphism.
\end{proof}

\begin{theorem}
  [Bidirectional Determination]\label{thm:bidirectional}The oscillatory
  process $\{X_t \}$ and the triple $(\mu, \Phi, \{A_t (\omega)\})$ are in
  one-to-one correspondence. Given $\{X_t \}$, the triple is uniquely
  determined (up to measure-theoretic equivalence) by
  Theorem~\ref{thm:uniqueness}. Conversely, given the triple satisfying
  Lemma~\ref{lem:existence-regularity}, the process is uniquely reconstructed
  via Definition~\ref{def:oscillatory-process}.
\end{theorem}

\begin{proof}
  The forward direction is covered by Theorem~\ref{thm:uniqueness}. For the
  reverse direction, given $(\mu, \Phi, \{A_t \})$ satisfying
  Lemma~\ref{lem:existence-regularity}, the stochastic integral
  \[ X_t = \int_{\R} A_t (\omega) e^{i \omega t}  \hspace{0.17em} d \Phi
     (\omega) \]
  is well-defined in $L^2 (\Omega)$ by condition~\ref{cond:l2integrability}.
  The covariance is given by
  \[ R_{s, t} = \E [X_s \overline{X_t}] = \int_{\R} A_s (\omega) \overline{A_t
     (\omega)} e^{i \omega (s - t)}  \hspace{0.17em} d \mu (\omega) . \]
  For a Gaussian process, the pair $(X_t, R_{s, t})$ uniquely determines the
  finite-dimensional distributions and hence the law of the process,
  establishing the bijective correspondence.
\end{proof}

\section{Gaussian Oscillatory Processes}

\begin{theorem}
  [Gaussian Structure]\label{thm:gaussian}If $\{X_t \}$ is a Gaussian
  oscillatory process satisfying Definition~\ref{def:oscillatory-process}, the
  orthogonal random measure $\Phi$ is Gaussian: for each Borel set $B$, $\Phi
  (B)$ is a complex Gaussian random variable with
  \[ \Phi (B) = \Phi_R (B) + i \Phi_I (B) \]
  where $\Phi_R, \Phi_I$ are independent real Gaussian orthogonal measures
  with $\E [\Phi_R (B)^2] = \E [\Phi_I (B)^2] = \mu (B) / 2$.
\end{theorem}

\begin{proof}
  Since $X_t$ is Gaussian and any finite linear combination $\sum_{j = 1}^n
  c_j X_{t_j}$ is Gaussian, the spectral representation gives
  \[ \sum_{j = 1}^n c_j X_{t_j} = \int_{\R} \left[ \sum_{j = 1}^n c_j A_{t_j}
     (\omega) e^{i \omega t_j} \right] d \Phi (\omega) \]
  as Gaussian for all $c_j, t_j$. By the Cram{\'e}r--Wold theorem, $\Phi (B)$
  is Gaussian for all Borel $B$. The complex orthogonality condition $\E [\Phi
  (B_1) \overline{\Phi (B_2)}] = 0$ for disjoint $B_1, B_2$ implies $\E
  [\Phi_R (B_1) \Phi_R (B_2)] = \E [\Phi_I (B_1) \Phi_I (B_2)] = 0$ and $\E
  [\Phi_R (B_1) \Phi_I (B_2)] = 0$. The variance is
  \[ \mu (B) = \E [| \Phi (B) |^2] = \E [\Phi_R (B)^2] + \E [\Phi_I (B)^2], \]
  and by symmetry $\E [\Phi_R (B)^2] = \E [\Phi_I (B)^2] = \mu (B) / 2$.
\end{proof}

\subsection{The Envelope Spectrum}

\begin{definition}
  [Envelope Spectrum]\label{def:envelope}The temporal Fourier transform of the
  amplitude,
  \[ \hat{A}_{\omega} (\lambda) = \int_{- \infty}^{\infty} A_t (\omega) e^{i
     \lambda t}  \hspace{0.17em} dt, \]
  is called the {\tmem{envelope spectrum}} at carrier frequency $\omega$. It
  describes how the modulation $A_t (\omega)$ distributes across temporal
  frequencies $\lambda$.
\end{definition}

\section{Inverse Spectral Theorem}

\begin{theorem}
  [Inverse Spectral Theorem for Oscillatory Processes]\label{thm:inverse}Let
  $\{X_t \}_{t \in \R}$ be a Gaussian oscillatory process satisfying
  Definition~\ref{def:oscillatory-process} with amplitude family satisfying
  Lemma~\ref{lem:existence-regularity}. Given a sample path $\{X_t \}_{t \in
  \R}$, the underlying stationary process $\{Y_u \}_{u \in \R}$ is first
  recovered via the inverse time-dependent filter by removing the amplitude
  modulation. Specifically, for each $t \in \R$, form
  \[ \tilde{Y}_t = \int_{- \infty}^{\infty} \frac{e^{- i \lambda t}}{A_t
     (\lambda)} X_t  \hspace{0.17em} d \lambda . \]
  The resulting process is stationary in its spectral decomposition. The
  orthogonal random measure $\Phi$ is then recovered pathwise via the
  Wiener--Khinchin--type inversion from the stationary component:
  \[ \Phi (B) = \lim_{T \to \infty}  \frac{1}{2 \pi T}  \int_{- T}^T
     \tilde{Y}_u \left[ \int_B e^{- i \omega u}  \hspace{0.17em} d \omega
     \right] du. \]
  Equivalently, the measure $\Phi$ may be recovered directly via
  \[ \Phi (B) = \lim_{T \to \infty}  \frac{1}{2 \pi T}  \int_{- T}^T X_t
     \left[ \int_B \frac{e^{- i \omega t}}{A_t (\omega)}  \hspace{0.17em} d
     \omega \right] dt, \]
  where the division by $A_t (\omega)$ is well-defined by
  condition~\ref{cond:nonvanishing} of Lemma~\ref{lem:existence-regularity}.
\end{theorem}

\begin{proof}
  The inversion proceeds in two conceptual steps, which can be combined.
  
  \paragraph{Step 1: De-modulation to Recover Stationary Component.} Write the
  oscillatory process as
  \[ X_t = \int_{\R} A_t (\omega) e^{i \omega t}  \hspace{0.17em} d \Phi
     (\omega) . \]
  Multiply both sides formally by the inverse of the amplitude via
  time-dependent inverse filtering:
  \[ \frac{X_t}{A_t (\cdummy)} = \int_{\R} e^{i \omega t}  \hspace{0.17em} d
     \Phi (\omega) = : Y_t . \]
  This formal product represents the de-modulated process, which is now purely
  stationary in spectral form, with no time-dependent amplitude $A_t
  (\omega)$.
  
  \paragraph{Step 2: Recover the Measure from the Stationary Component.} For
  the stationary process $\{Y_t \}_{t \in \R}$ with spectral representation
  \[ Y_t = \int_{\R} e^{i \omega t}  \hspace{0.17em} d \Phi (\omega), \]
  the standard Wiener--Khinchin inverse theorem applies. Define the
  approximate identity kernel
  \[ K_T (\lambda, \omega) = \frac{1}{2 \pi T}  \int_{- T}^T e^{i (\lambda -
     \omega) t}  \hspace{0.17em} dt = \frac{\sin ((\lambda - \omega) T)}{\pi
     (\lambda - \omega) T} . \]
  For each fixed $\lambda$, as $T \to \infty$, $K_T (\lambda, \cdot)$
  approximates the Dirac delta: $K_T (\lambda, \omega) \to \delta_{\lambda}
  (\omega)$.
  
  Compute
  \[ \frac{1}{2 \pi T}  \int_{- T}^T Y_t \left[ \int_B e^{- i \omega t} 
     \hspace{0.17em} d \omega \right] dt = \frac{1}{2 \pi T}  \int_{- T}^T
     \left[ \int_{\R} e^{i \lambda t}  \hspace{0.17em} d \Phi (\lambda)
     \right] \left[ \int_B e^{- i \omega t}  \hspace{0.17em} d \omega \right]
     dt. \]
  By Fubini's theorem,
  \[ = \int_{\R} \left[ \int_B \frac{1}{2 \pi T}  \int_{- T}^T e^{i (\lambda -
     \omega) t}  \hspace{0.17em} dt \hspace{0.17em} d \omega \right] d \Phi
     (\lambda) = \int_{\R} \left[ \int_B K_T (\lambda, \omega) \hspace{0.17em}
     d \omega \right] d \Phi (\lambda) . \]
  As $T \to \infty$, $\int_B K_T (\lambda, \omega)  \hspace{0.17em} d \omega
  \to \textbf{1}_B (\lambda)$ for $\mu$-a.e. $\lambda$. By dominated
  convergence for orthogonal random measures,
  \[ \lim_{T \to \infty}  \int_{\R} \left[ \int_B K_T (\lambda, \omega)
     \hspace{0.17em} d \omega \right] d \Phi (\lambda) = \Phi (B) . \]
  \paragraph{Step 3: Direct Combined Inversion Formula.} To recover $\Phi$
  directly from $X_t$ without explicitly constructing the intermediate
  de-modulated process, substitute $Y_t = \int_{\R} e^{i \omega t} 
  \hspace{0.17em} d \Phi (\omega)$ and $X_t = \int_{\R} A_t (\omega) e^{i
  \omega t}  \hspace{0.17em} d \Phi (\omega)$ into the stationary inversion
  formula. The ratio $X_t / A_t (\cdummy)$ is understood as a generalized form
  acting on the spectral measure. This gives
  \[ \Phi (B) = \lim_{T \to \infty}  \frac{1}{2 \pi T}  \int_{- T}^T X_t
     \left[ \int_B \frac{e^{- i \omega t}}{A_t (\omega)}  \hspace{0.17em} d
     \omega \right] dt. \]
  The well-posedness of the division by $A_t (\omega)$ is guaranteed by
  condition~\ref{cond:nonvanishing}, which ensures $A_t (\omega) \neq 0$ for
  all $t$ and $\mu$-a.e. $\omega$. The dominated convergence for orthogonal
  random measures and standard approximate identity arguments complete the
  proof.
\end{proof}

\end{document}
