\documentclass{article}
\usepackage[english]{babel}
\usepackage{geometry,amsmath,amssymb,latexsym}
\usepackage{mathtools} % for \coloneqq and other alignment symbols

\geometry{letterpaper}

%%%%%%%%%% Start TeXmacs macros
\newcommand{\tmaffiliation}[1]{\\ #1}
\newcommand{\tmem}[1]{{\em #1\/}}
\newenvironment{proof}{\noindent\textbf{Proof\ }}{\hspace*{\fill}$\Box$\medskip}
\newtheorem{definition}{Definition}
\newtheorem{lemma}{Lemma}
\newtheorem{theorem}{Theorem}
%%%%%%%%%% End TeXmacs macros

\begin{document}

\title{Eigenfunction Properties of Stationary and Oscillatory Stochastic
Processes}

\author{
  Stephen Crowley
  \tmaffiliation{August 7, 2025}
}

\date{}

\maketitle

\section*{Introduction}

Oscillatory processes generalize stationary stochastic processes by allowing
their spectral properties to evolve over time. Central to this representation
is the {\tmem{gain function}} $A (t, \omega)$, a complex-valued function that
works in conjunction with an underlying spectral density $f (\omega)$ to
produce time-varying spectral characteristics. The magnitude $|A (t, \omega)
|$ scales the spectral power at each frequency and time, while the argument
$\arg (A (t, \omega))$ introduces frequency-dependent phase shifts. The
effective spectral density at time $t$ becomes $|A (t, \omega) |^2 f
(\omega)$, showing how the gain function and underlying spectral density work
together multiplicatively.

\begin{definition}
  [Stationary Process] A stochastic process $\{X (t), t \in \mathbb{R}\}$ is
  called stationary if its covariance function satisfies $R (s, t) = R (t -
  s)$ for all $s, t \in \mathbb{R}$.
\end{definition}

\begin{definition}
  [Oscillatory Process (Priestley)] A stochastic process $\{X (t), t \in
  \mathbb{R}\}$ is called oscillatory if it possesses an evolutionary spectral
  representation
  \begin{equation}
    X (t) = \int_{- \infty}^{\infty} A (t, \omega) e^{i \omega t} \, dZ (\omega)
  \end{equation}
  where:
  \begin{itemize}
  \item $A (t, \omega)$ is the gain function (complex-valued, deterministic),
  \item $Z (\omega)$ is an orthogonal increment process,
  \item $dF (\omega) = f (\omega) \, d \omega$ is its spectral measure,
  \item $f (\omega)$ is the underlying spectral density.
  \end{itemize}
\end{definition}

\begin{theorem}
  [Covariance Structure of Oscillatory Processes]
  For an oscillatory process with gain function $A (t, \omega)$ and underlying spectral density $f
  (\omega)$, the covariance function is given by
  \begin{equation}
    C (s, t) = \int_{- \infty}^{\infty} A (s, \omega) A^{\ast} (t, \omega) \, e^{i\omega(s-t)} \, f
    (\omega) \, d \omega.
  \end{equation}
\end{theorem}

\begin{proof}
We start from the evolutionary spectral representation
\[
X(s) = \int_{-\infty}^\infty A(s,\omega_1)\, e^{i\omega_1 s} \, dZ(\omega_1),
\]
\[
X^*(t) = \int_{-\infty}^\infty A^*(t,\omega_2)\, e^{-i\omega_2 t} \, dZ^*(\omega_2).
\]
The covariance is
\begin{align*}
C(s,t) &= \mathbb{E}\big[ X(s) X^*(t) \big] \\
&= \mathbb{E}\left[ \int_{\mathbb{R}} A(s,\omega_1)e^{i\omega_1 s}dZ(\omega_1)
        \int_{\mathbb{R}} A^*(t,\omega_2) e^{-i\omega_2 t}dZ^*(\omega_2) \right].
\end{align*}
By Fubini's theorem (justified by square integrability),
\begin{align*}
C(s,t) &= \int_{\mathbb{R}} \int_{\mathbb{R}} 
     A(s,\omega_1) e^{i\omega_1 s} \, A^*(t,\omega_2) e^{-i\omega_2 t} \,
     \mathbb{E}[\,dZ(\omega_1)\,dZ^*(\omega_2)\,] .
\end{align*}
Orthogonality of increments gives
\[
\mathbb{E}[\,dZ(\omega_1)\, dZ^*(\omega_2)\,] =
       \delta(\omega_1 - \omega_2) \, f(\omega_1)\, d\omega_1 .
\]
Integrating out $\omega_2$ using $\delta$:
\begin{align*}
C(s,t) &= \int_{\mathbb{R}} A(s,\omega) e^{i\omega s} \, A^*(t,\omega) e^{-i\omega t} \,
    f(\omega) \, d\omega \\
&= \int_{\mathbb{R}} A(s,\omega) A^*(t,\omega) e^{i\omega(s-t)} f(\omega)\, d\omega.
\end{align*}
This is the claimed form.
\end{proof}

\begin{theorem}
  [Eigenfunction Property for Stationary Processes]
  Let $\{X (t)\}$ be stationary with covariance function $R (\tau)$
  and covariance operator
  \[
  (Kf) (t) = \int_{\mathbb{R}} R (t - s) f(s)\, ds.
  \]
  Then $e^{i \omega t}$ are eigenfunctions of $K$
  with eigenvalues $S (\omega)$, the power spectral density.
\end{theorem}

\begin{proof}
Take $f(s) = e^{i\omega s}$. Then
\[
(K e^{i\omega \cdot})(t) = \int_{\mathbb{R}} R(t-s)\, e^{i\omega s} \, ds.
\]
Let $\tau = t - s$ so $s = t - \tau$, $ds = -d\tau$. Bounds $\pm\infty$ swap but because we integrate over $\mathbb{R}$ the order is irrelevant:
\begin{align*}
(K e^{i\omega \cdot})(t) &= \int_{\mathbb{R}} R(\tau)\, e^{i\omega (t-\tau)} \, d\tau \\
&= e^{i\omega t} \int_{\mathbb{R}} R(\tau) e^{-i\omega \tau} \, d\tau.
\end{align*}
The term
\[
S(\omega) \coloneqq \int_{\mathbb{R}} R(\tau) e^{-i\omega \tau} \, d\tau
\]
is the Wiener--Khintchine representation of the power spectrum. Thus
\[
(K e^{i\omega\cdot})(t) = S(\omega) \, e^{i\omega t}.
\]
\end{proof}

\begin{theorem}
  [Eigenfunction Property for Oscillatory Processes]
  Let $X(t)$ be oscillatory with representation
  \[
  X (t) = \int_{\mathbb{R}} A (t, \omega) e^{i \omega t} \, dZ (\omega)
  \]
  and covariance function
  \[
  C(s, t) = \int_{\mathbb{R}} A(s, \omega) A^{\ast} (t, \omega) e^{i\omega(s-t)} f(\omega) \, d \omega.
  \]
  Then $\phi(t,\omega) = A(t,\omega) e^{i\omega t}$ are eigenfunctions of
  \[
  (K g)(t) = \int_{\mathbb{R}} C(t,s) g(s) \, ds
  \]
  with eigenvalues $f(\omega)$.
\end{theorem}

\begin{proof}
Let $g(s) = \phi(s,\omega) = A(s,\omega) e^{i\omega s}$. Then
\begin{align*}
(K\phi(\cdot,\omega))(t) &= \int_{\mathbb{R}} C(t,s) \, A(s,\omega) e^{i\omega s}\, ds \\
&= \int_{\mathbb{R}} \left[ \int_{\mathbb{R}} A(t,\lambda) A^*(s,\lambda) e^{i\lambda(t-s)} f(\lambda)\, d\lambda \right]
   A(s,\omega) e^{i\omega s} ds.
\end{align*}
Swap integration order by Fubini:
\begin{align*}
&= \int_{\mathbb{R}} A(t,\lambda) f(\lambda) e^{i\lambda t}
    \left[ \int_{\mathbb{R}} A^*(s,\lambda) A(s,\omega) e^{-i\lambda s} e^{i\omega s} ds \right] d\lambda.
\end{align*}
The exponent factor $e^{-i\lambda s} e^{i\omega s} = e^{i(\omega-\lambda)s}$.  
By the fundamental orthogonality condition of evolutionary spectral representations,
\[
\int_{\mathbb{R}} A^*(s,\lambda) A(s,\omega) e^{i(\omega-\lambda)s} ds = \delta(\lambda-\omega).
\]
Substitute into the above:
\begin{align*}
(K\phi(\cdot,\omega))(t) &= \int_{\mathbb{R}} A(t,\lambda) f(\lambda) e^{i\lambda t} \, \delta(\lambda-\omega)\, d\lambda \\
&= A(t,\omega) e^{i\omega t} f(\omega) \\
&= \phi(t,\omega) \, f(\omega).
\end{align*}
\end{proof}

\begin{theorem}
  [Reality Conditions for Oscillatory Processes]
  Let
  \[
  X (t) = \int_{\mathbb{R}} A (t, \omega) e^{i \omega t} \, dZ (\omega)
  \]
  be oscillatory. Then $X(t)$ is real-valued for all $t$
  if and only if:
  \begin{enumerate}
    \item $A (t, \omega) = A^{\ast} (t, - \omega)$ (gain conjugate symmetry),
    \item $dZ (- \omega) = dZ^{\ast} (\omega)$ (increment conjugate symmetry).
  \end{enumerate}
\end{theorem}

\begin{proof}
Necessary:  
If $X(t)$ is real-valued then $X^*(t) = X(t)$ for each $t$. Compute $X^*(t)$:
\begin{align*}
X^*(t) &= \left[ \int_{\mathbb{R}} A(t,\omega) e^{i\omega t} \, dZ(\omega) \right]^* \\
&= \int_{\mathbb{R}} A^*(t,\omega) e^{-i\omega t} \, dZ^*(\omega).
\end{align*}
Make substitution $\omega' = -\omega$:
\begin{align*}
X^*(t) &= \int_{\mathbb{R}} A^*(t,-\omega) e^{i\omega t} \, dZ^*(-\omega).
\end{align*}
Equality $X^*(t) = X(t)$ for all $t$ means
\[
A(t,\omega) \, dZ(\omega) = A^*(t,-\omega) \, dZ^*(-\omega)
\]
for all $\omega$. Splitting the statement into deterministic and stochastic factors:
\[
A(t,\omega) = A^*(t,-\omega), \quad dZ(\omega) = dZ^*(-\omega),
\]
which rearranges to the stated conditions.

Sufficiency:  
If these two symmetry properties hold, then the expression for $X^*(t)$ above becomes identical to $X(t)$. Therefore $X(t)$ is real-valued.
\end{proof}

\begin{theorem}
  [Equivalence of Evolutionary Spectral and Filter Representations]
  Given the evolutionary spectral representation
  \[
  X(t) = \int_{\mathbb{R}} A(t,\omega) e^{i\omega t} \, dZ(\omega),
  \]
  with $dZ(\omega) = \int_{\mathbb{R}} e^{-i\omega s} dW(s)$ for orthogonal increments $dW(s)$,
  one can write
  \[
  X(t) = \int_{\mathbb{R}} h_t(t-s) \, dW(s),
  \]
  where $h_t(u)$ is the time-dependent filter kernel.
\end{theorem}

\begin{proof}
Let $\phi(t,\omega) = A(t,\omega) e^{i\omega t}$. Define the kernel by
\[
h_t(u) = \int_{\mathbb{R}} \phi(t,\omega) e^{-i\omega u} \, d\omega
       = \int_{\mathbb{R}} A(t,\omega) e^{i\omega t} e^{-i\omega u} d\omega.
\]
Note $u = t-s$ will appear naturally. Now substitute the assumed $dZ$ expression into the spectral form of $X(t)$:
\begin{align*}
X(t) &= \int_{\mathbb{R}} A(t,\omega) e^{i\omega t} \left[ \int_{\mathbb{R}} e^{-i\omega s} dW(s) \right] d\omega.
\end{align*}
By Fubini,
\begin{align*}
X(t) &= \int_{\mathbb{R}} \left[ \int_{\mathbb{R}} A(t,\omega) e^{i\omega t} e^{-i\omega s} d\omega \right] dW(s) \\
&= \int_{\mathbb{R}} h_t(t-s) \, dW(s),
\end{align*}
with $h_t(t-s)$ exactly as above. This establishes the equivalence.
\end{proof}

\begin{theorem}
  [Fourier Transform Relationships]
  The quantities $A(t,\omega)$, $\phi(t,\omega)$ and $h_t(u)$ obey:
  \begin{align}
    A(t, \omega) &= \int_{\mathbb{R}} h_t(t - s) e^{- i \omega s} ds, \\
    \phi(t, \omega) &= A(t, \omega) e^{i \omega t} = \int_{\mathbb{R}} h_t(u) e^{- i \omega (t - u)} du, \\
    h_t(t - s) &= \int_{\mathbb{R}} A(t, \omega) e^{i \omega s} \, d\omega 
              = \int_{\mathbb{R}} \phi(t, \omega) e^{- i \omega (t - s)} d \omega.
  \end{align}
\end{theorem}

\begin{proof}
1. By the inverse Fourier transform in $\omega$:
\[
A(t,\omega) = \int_{\mathbb{R}} h_t(t-s) e^{-i\omega s} \, ds.
\]

2. Multiplying $A(t,\omega)$ by $e^{i\omega t}$:
\[
\phi(t,\omega) = e^{i\omega t} \int_{\mathbb{R}} h_t(t-s) e^{-i\omega s} ds.
\]
Let $u = t-s$, $ds = -du$, limits still $\mathbb{R}$:
\[
\phi(t,\omega) = \int_{\mathbb{R}} h_t(u) e^{-i\omega (t-u)} du.
\]

3. Inverse-transform $A(t,\omega)$ back:
\[
h_t(t-s) = \int_{\mathbb{R}} A(t,\omega) e^{i\omega s} \, d\omega.
\]
Similarly, replacing $A(t,\omega)$ by $\phi(t,\omega)e^{-i\omega t}$ yields the other form.
\end{proof}

\begin{lemma}
  [Orthogonality Property]
  In the evolutionary spectral representation,
  \[
  \int_{\mathbb{R}} A^{\ast} (s, \lambda) A (s, \omega) e^{i (\omega-\lambda) s} ds = \delta (\lambda - \omega).
  \]
\end{lemma}

\begin{proof}
The key assumption for $dZ(\omega)$ is
\[
\mathbb{E}[\, dZ(\lambda) dZ^*(\omega) \,] = \delta(\lambda-\omega) f(\lambda) \, d\lambda.
\]
Writing
\[
dZ(\lambda) = \int_{\mathbb{R}} X(s) \, \psi^*_\lambda(s) \, ds
\]
for some dual family $\psi_\lambda$ given by the oscillatory functions $\phi(s,\lambda)$, the uncorrelatedness in $\lambda\neq\omega$ forces
\[
\int_{\mathbb{R}} \phi^*(s,\lambda) \phi(s,\omega) ds = \delta(\lambda-\omega).
\]
Substituting $\phi(s,\omega) = A(s,\omega)e^{i\omega s}$ yields the stated form.
\end{proof}

\begin{theorem}
  [Correspondence Principle]
  If $A(t,\omega) = A(\omega)$ is independent of $t$, then $\phi(t,\omega) = A(\omega) e^{i\omega t}$ reduces to stationary eigenfunctions and the covariance is that of a stationary process.
\end{theorem}

\begin{proof}
If $A$ is independent of $t$ then
\[
\phi(t,\omega) = A(\omega) e^{i \omega t}.
\]
The covariance from Theorem~\ref{thm:cov} becomes:
\[
C(s,t) = \int_{\mathbb{R}} A(\omega) A^*(\omega) e^{i\omega(s-t)} f(\omega)\, d\omega.
\]
This depends only on $\tau = s-t$ and is thus stationary, with power spectral density $|A(\omega)|^2 f(\omega)$.
\end{proof}

\end{document}
