\documentclass{article}
\usepackage[english]{babel}
\usepackage{amsmath,amssymb,latexsym}

%%%%%%%%%% Start TeXmacs macros
\newcommand{\tmaffiliation}[1]{\\ #1}
\newcommand{\tmem}[1]{{\em #1\/}}
\newenvironment{proof}{\noindent\textbf{Proof\ }}{\hspace*{\fill}$\Box$\medskip}
\newtheorem{definition}{Definition}
\newtheorem{lemma}{Lemma}
\newtheorem{theorem}{Theorem}
%%%%%%%%%% End TeXmacs macros

\begin{document}

\title{Eigenfunction Properties of Stationary and Oscillatory Stochastic
Processes}

\author{
  Stephen Crowley
  \tmaffiliation{August 7, 2025}
}

\date{}

\maketitle

\section*{Introduction}

Oscillatory processes generalize stationary stochastic processes by allowing
their spectral properties to evolve over time. Central to this representation
is the {\tmem{gain function}} $A (t, \omega)$, a complex-valued function that
works in conjunction with an underlying spectral density $f (\omega)$ to
produce time-varying spectral characteristics. The magnitude $|A (t, \omega)
|$ scales the spectral power at each frequency and time, while the argument
$\arg (A (t, \omega))$ introduces frequency-dependent phase shifts. The
effective spectral density at time $t$ becomes $|A (t, \omega) |^2 f
(\omega)$, showing how the gain function and underlying spectral density work
together multiplicatively.

\begin{definition}
  [Stationary Process] A stochastic process $\{X (t), t \in \mathbb{R}\}$ is
  called stationary if its covariance function satisfies $R (s, t) = R (t -
  s)$ for all $s, t \in \mathbb{R}$.
\end{definition}

\begin{definition}
  [Oscillatory Process (Priestley)] A stochastic process $\{X (t), t \in
  \mathbb{R}\}$ is called oscillatory if it possesses an evolutionary spectral
  representation
  \begin{equation}
    X (t) = \int_{- \infty}^{\infty} A (t, \omega) e^{i \omega t} dZ (\omega)
  \end{equation}
  where $A (t, \omega)$ is the gain function and $Z (\omega)$ is an orthogonal
  increment process with spectral measure $dF (\omega) = f (\omega) d \omega$,
  where $f (\omega)$ is the underlying spectral density.
\end{definition}

\begin{theorem}
  [Covariance Structure of Oscillatory Processes] For an oscillatory process
  with gain function $A (t, \omega)$ and underlying spectral density $f
  (\omega)$, the covariance function is given by
  \begin{equation}
    C (s, t) = \int_{- \infty}^{\infty} A (s, \omega) A^{\ast} (t, \omega) f
    (\omega) d \omega
  \end{equation}
  This shows that the gain function works in conjunction with the underlying
  spectral density, with the effective spectral density at times $s$ and $t$
  being the product $A (s, \omega) A^{\ast} (t, \omega) f (\omega)$.
\end{theorem}

\begin{proof}
  From the evolutionary spectral representation and the orthogonality property
  $\mathbb{E} [dZ (\omega_1) dZ^{\ast} (\omega_2)] = \delta (\omega_1 -
  \omega_2) f (\omega_1) d \omega_1$:
  
  \begin{align}
    C (s, t) & =\mathbb{E} [X (s) X^{\ast} (t)] \\
    & =\mathbb{E} \left[ \int_{- \infty}^{\infty} A (s, \omega_1) e^{i
    \omega_1 s} dZ (\omega_1) \int_{- \infty}^{\infty} A^{\ast} (t, \omega_2)
    e^{- i \omega_2 t} dZ^{\ast} (\omega_2) \right] \\
    & = \int_{- \infty}^{\infty} \int_{- \infty}^{\infty} A (s, \omega_1)
    A^{\ast} (t, \omega_2) e^{i \omega_1 s} e^{- i \omega_2 t} \mathbb{E} [dZ
    (\omega_1) dZ^{\ast} (\omega_2)] \\
    & = \int_{- \infty}^{\infty} A (s, \omega) A^{\ast} (t, \omega) e^{i
    \omega (s - t)} f (\omega) d \omega 
  \end{align}
\end{proof}

\begin{theorem}
  [Eigenfunction Property for Stationary Processes] Let $\{X (t), t \in
  \mathbb{R}\}$ be a stationary process with covariance function $R (\tau)$
  and covariance operator
  \begin{equation}
    (Kf) (t) = \int_{- \infty}^{\infty} R (t - s) f (s) ds
  \end{equation}
  Then the complex exponentials $e^{i \omega t}$ are eigenfunctions of $K$
  with eigenvalues equal to the power spectral density $S (\omega)$.
\end{theorem}

\begin{proof}
  Consider the action of $K$ on $e^{i \omega t}$:
  
  \begin{align}
    (Ke^{i \omega t}) (t) & = \int_{- \infty}^{\infty} R (t - s) e^{i \omega
    s} ds 
  \end{align}
  
  Substituting $\tau = t - s$:
  
  \begin{align}
    & = e^{i \omega t}  \int_{- \infty}^{\infty} R (\tau) e^{- i \omega \tau}
    d \tau \\
    & = e^{i \omega t} \cdot S (\omega) 
  \end{align}
  
  where $S (\omega) = \int_{- \infty}^{\infty} R (\tau) e^{- i \omega \tau} d
  \tau$ is the power spectral density by the Wiener-Khintchine theorem.
\end{proof}

\begin{theorem}
  [Eigenfunction Property for Oscillatory Processes] Let $\{X (t), t \in
  \mathbb{R}\}$ be an oscillatory process with evolutionary spectral
  representation
  \begin{equation}
    X (t) = \int_{- \infty}^{\infty} A (t, \omega) e^{i \omega t} dZ (\omega)
  \end{equation}
  and covariance function
  \begin{equation}
    C (s, t) = \int_{- \infty}^{\infty} A (s, \omega) A^{\ast} (t, \omega) f
    (\omega) d \omega
  \end{equation}
  where $f (\omega)$ is the underlying spectral density. Then the oscillatory
  functions $\phi (t, \omega) = A (t, \omega) e^{i \omega t}$ are
  eigenfunctions of the covariance operator
  \begin{equation}
    (Kf) (t) = \int_{- \infty}^{\infty} C (t, s) f (s) ds
  \end{equation}
  with eigenvalues $f (\omega)$.
\end{theorem}

\begin{proof}
  Consider the action of $K$ on the oscillatory function $\phi (s, \omega) = A
  (s, \omega) e^{i \omega s}$:
  
  \begin{align}
    (K \phi) (t) & = \int_{- \infty}^{\infty} C (t, s) A (s, \omega) e^{i
    \omega s} ds \\
    & = \int_{- \infty}^{\infty} \left[ \int_{- \infty}^{\infty} A (t,
    \lambda) A^{\ast} (s, \lambda) f (\lambda) d \lambda \right] A (s, \omega)
    e^{i \omega s} ds 
  \end{align}
  
  By Fubini's theorem, the order of integration may be exchanged:
  
  \begin{align}
    & = \int_{- \infty}^{\infty} A (t, \lambda) f (\lambda) \left[ \int_{-
    \infty}^{\infty} A^{\ast} (s, \lambda) A (s, \omega) e^{i \omega s} ds
    \right] d \lambda 
  \end{align}
  
  The inner integral represents the orthogonality condition in the
  evolutionary spectral representation. By the fundamental property of
  evolutionary spectral representations:
  \begin{equation}
    \int_{- \infty}^{\infty} A^{\ast} (s, \lambda) A (s, \omega) e^{i \omega
    s} ds = \delta (\lambda - \omega)
  \end{equation}
  where $\delta (\lambda - \omega)$ is the Dirac delta function.
  
  Therefore:
  
  \begin{align}
    (K \phi) (t) & = \int_{- \infty}^{\infty} A (t, \lambda) f (\lambda)
    \delta (\lambda - \omega) d \lambda \\
    & = A (t, \omega) f (\omega) \\
    & = \phi (t, \omega) \cdot f (\omega) 
  \end{align}
  
  This establishes that $\phi (t, \omega) = A (t, \omega) e^{i \omega t}$ are
  eigenfunctions with eigenvalues $f (\omega)$.
\end{proof}

\begin{theorem}
  [Reality Conditions for Oscillatory Processes] Let
  \[ X (t) = \int_{- \infty}^{\infty} A (t, \omega) e^{i \omega t} dZ (\omega)
  \]
  with gain function $A (t, \omega)$ and orthogonal increment process $dZ
  (\omega)$. The process $X (t)$ is real-valued for all $t$ if and only if the
  following conditions hold for all $t$ and almost all $\omega$:
  \begin{enumerate}
    \item $A (t, \omega) = A^{\ast} (t, - \omega)$ {\hspace*{\fill}}(conjugate
    symmetry of the gain function),
    
    \item $dZ (- \omega) = dZ^{\ast} (\omega)$ {\hspace*{\fill}}(conjugate
    symmetry of the increments).
  \end{enumerate}
\end{theorem}

\begin{proof}
  The process $X (t)$ is real-valued if and only if $X^{\ast} (t) = X (t)$ for
  each $t$.
  
  Compute the complex conjugate:
  \[ X^{\ast} (t) = \left( \int_{- \infty}^{\infty} A (t, \omega) e^{i \omega
     t} dZ (\omega) \right)^{\ast} = \int_{- \infty}^{\infty} A^{\ast} (t,
     \omega) e^{- i \omega t} dZ^{\ast} (\omega) \]
  Make the substitution $\omega' = - \omega$ (so $d \omega' = - d \omega$),
  and note that as the limits are infinite, the domain is unchanged under sign
  reversal:
  \[ X^{\ast} (t) = \int_{- \infty}^{\infty} A^{\ast} (t, - \omega) e^{i
     \omega t} dZ^{\ast}  (- \omega) \]
  For the process to be real-valued for all $t$, it must hold that $X^{\ast}
  (t) = X (t)$ for all $t$, i.e.,
  \[ \int_{- \infty}^{\infty} A (t, \omega) e^{i \omega t} dZ (\omega) =
     \int_{- \infty}^{\infty} A^{\ast} (t, - \omega) e^{i \omega t} dZ^{\ast} 
     (- \omega) \]
  This equality holds if and only if the integrands are equal for all
  $\omega$, up to a set of measure zero. Thus, the following must hold:
  \[ A (t, \omega) dZ (\omega) = A^{\ast} (t, - \omega) dZ^{\ast}  (- \omega)
  \]
  for all $t$ and $\omega$. This is equivalent to demanding:
  
  \begin{align*}
    A^{\ast} (t, - \omega) & = A (t, \omega)\\
    dZ^{\ast}  (- \omega) & = dZ (\omega)
  \end{align*}
  
  Taking complex conjugates of both sides in the second line:
  \[ dZ (- \omega) = dZ^{\ast} (\omega) \]
  So, the process is real-valued if and only if the gain function and the
  increment process each have conjugate symmetry:
  \[ A (t, \omega) = A^{\ast} (t, - \omega), \quad dZ (- \omega) = dZ^{\ast}
     (\omega) \]
\end{proof}

\begin{theorem}
  [Equivalence of Evolutionary Spectral and Filter Representations] Let $X
  (t)$ be a stochastic process. The evolutionary spectral representation
  \begin{equation}
    X (t) = \int_{- \infty}^{\infty} A (t, \omega) e^{i \omega t} dZ (\omega)
  \end{equation}
  where $A (t, \omega)$ is the gain function and $dZ (\omega)$ is an
  orthogonal increment process, is equivalent to the time-domain filter
  representation
  \begin{equation}
    X (t) = \int_{- \infty}^{\infty} h_t  (t - s) dW (s)
  \end{equation}
  where $h_t  (t - s)$ is a time-dependent filter kernel and $dW (s)$ is an
  orthogonal increment process.
\end{theorem}

\begin{proof}
  The filter kernel $h_t  (t - s)$ relates to the gain function and the
  oscillatory function via Fourier transform relationships:
  
  \begin{align}
    h_t  (t - s) & = \int_{- \infty}^{\infty} \phi (t, \omega) e^{- i \omega
    (t - s)} d \omega \\
    & = \int_{- \infty}^{\infty} A (t, \omega) e^{i \omega t} e^{- i \omega
    (t - s)} d \omega \\
    & = \int_{- \infty}^{\infty} A (t, \omega) e^{i \omega s} d \omega 
  \end{align}
  
  where $\phi (t, \omega) = A (t, \omega) e^{i \omega t}$ is the oscillatory
  function.
  
  To establish equivalence, substitute the orthogonal increment relationship
  $dZ (\omega) = \int_{- \infty}^{\infty} e^{- i \omega s} dW (s)$ into the
  evolutionary spectral representation:
  
  \begin{align}
    X (t) & = \int_{- \infty}^{\infty} A (t, \omega) e^{i \omega t} dZ
    (\omega) \\
    & = \int_{- \infty}^{\infty} A (t, \omega) e^{i \omega t} \left[ \int_{-
    \infty}^{\infty} e^{- i \omega s} dW (s) \right] d \omega \\
    & = \int_{- \infty}^{\infty} \left[ \int_{- \infty}^{\infty} A (t,
    \omega) e^{i \omega t} e^{- i \omega s} d \omega \right] dW (s) \\
    & = \int_{- \infty}^{\infty} \left[ \int_{- \infty}^{\infty} A (t,
    \omega) e^{i \omega (t - s)} d \omega \right] dW (s) \\
    & = \int_{- \infty}^{\infty} h_t  (t - s) dW (s) 
  \end{align}
  
  where the last equality follows from the definition of $h_t  (t - s)$ with
  $u = t - s$.
\end{proof}

\begin{theorem}
  [Fourier Transform Relationships] The gain function $A (t, \omega)$,
  oscillatory function $\phi (t, \omega)$, and filter kernel $h_t (u)$ satisfy
  the following Fourier transform relationships:
  
  \begin{align}
    A (t, \omega) & = \int_{- \infty}^{\infty} h_t  (t - s) e^{- i \omega s}
    ds \\
    \phi (t, \omega) & = A (t, \omega) e^{i \omega t} = \int_{-
    \infty}^{\infty} h_t (u) e^{- i \omega (t - u)} du \\
    h_t  (t - s) & = \int_{- \infty}^{\infty} A (t, \omega) e^{i \omega s} d
    \omega = \int_{- \infty}^{\infty} \phi (t, \omega) e^{- i \omega (t - s)}
    d \omega 
  \end{align}
\end{theorem}

\begin{proof}
  For the first relationship, apply the inverse Fourier transform to $h_t  (t
  - s)$:
  
  \begin{align}
    A (t, \omega) & =\mathcal{F}_s^{- 1}  [h_t (t - s)] \\
    & = \int_{- \infty}^{\infty} h_t  (t - s) e^{- i \omega s} ds 
  \end{align}
  
  For the oscillatory function relationship, substitute the definition $\phi
  (t, \omega) = A (t, \omega) e^{i \omega t}$:
  
  \begin{align}
    \phi (t, \omega) & = A (t, \omega) e^{i \omega t} \\
    & = \left[ \int_{- \infty}^{\infty} h_t (t - s) e^{- i \omega s} ds
    \right] e^{i \omega t} \\
    & = \int_{- \infty}^{\infty} h_t  (t - s) e^{- i \omega s} e^{i \omega t}
    ds \\
    & = \int_{- \infty}^{\infty} h_t  (t - s) e^{- i \omega (s - t)} ds \\
    & = \int_{- \infty}^{\infty} h_t (u) e^{- i \omega (t - u)} du 
  \end{align}
  
  where $u = t - s$ in the last step.
  
  For the inverse relationships, apply the Fourier transform to recover $h_t 
  (t - s)$:
  
  \begin{align}
    h_t  (t - s) & =\mathcal{F}_{\omega}^{- 1}  [A (t, \omega) e^{i \omega s}]
    \\
    & = \int_{- \infty}^{\infty} A (t, \omega) e^{i \omega s} d \omega 
  \end{align}
  
  Similarly:
  
  \begin{align}
    h_t  (t - s) & =\mathcal{F}_{\omega}^{- 1}  [\phi (t, \omega) e^{- i
    \omega t}] \\
    & = \int_{- \infty}^{\infty} \phi (t, \omega) e^{- i \omega t} e^{i
    \omega (t - s)} d \omega \\
    & = \int_{- \infty}^{\infty} \phi (t, \omega) e^{- i \omega (t - s)} d
    \omega 
  \end{align}
\end{proof}

\begin{lemma}
  [Orthogonality Property] For the evolutionary spectral representation, the
  orthogonality condition
  \begin{equation}
    \int_{- \infty}^{\infty} A^{\ast} (s, \lambda) A (s, \omega) e^{i \omega
    s} ds = \delta (\lambda - \omega)
  \end{equation}
  follows from the requirement that $dZ (\omega)$ be an orthogonal increment
  process.
\end{lemma}

\begin{proof}
  The orthogonality of $dZ (\omega)$ requires $\mathbb{E} [dZ (\lambda)
  dZ^{\ast} (\omega)] = \delta (\lambda - \omega) f (\lambda) d \lambda$. This
  condition, combined with the evolutionary spectral representation, directly
  implies the stated orthogonality property for the gain functions.
\end{proof}

\begin{theorem}
  [Correspondence Principle] The eigenfunction properties of oscillatory
  processes reduce to those of stationary processes when the gain function
  becomes constant: $A (t, \omega) = A (\omega)$.
\end{theorem}

\begin{proof}
  When $A (t, \omega) = A (\omega)$ is independent of time, the oscillatory
  functions become $\phi (t, \omega) = A (\omega) e^{i \omega t}$, which are
  scalar multiples of the complex exponentials $e^{i \omega t}$. The
  covariance function reduces to
  \begin{equation}
    C (s, t) = \int_{- \infty}^{\infty} |A (\omega) |^2 f (\omega) e^{i \omega
    (s - t)} d \omega
  \end{equation}
  which depends only on $s - t$, recovering the stationary case with effective
  spectral density $|A (\omega) |^2 f (\omega)$.
\end{proof}

\end{document}
