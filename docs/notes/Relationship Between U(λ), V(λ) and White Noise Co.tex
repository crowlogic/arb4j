% Options for packages loaded elsewhere
\PassOptionsToPackage{unicode}{hyperref}
\PassOptionsToPackage{hyphens}{url}
%
\documentclass[
]{article}
\usepackage{amsmath,amssymb}
\usepackage{iftex}
\ifPDFTeX
  \usepackage[T1]{fontenc}
  \usepackage[utf8]{inputenc}
  \usepackage{textcomp} % provide euro and other symbols
\else % if luatex or xetex
  \usepackage{unicode-math} % this also loads fontspec
  \defaultfontfeatures{Scale=MatchLowercase}
  \defaultfontfeatures[\rmfamily]{Ligatures=TeX,Scale=1}
\fi
\usepackage{lmodern}
\ifPDFTeX\else
  % xetex/luatex font selection
\fi
% Use upquote if available, for straight quotes in verbatim environments
\IfFileExists{upquote.sty}{\usepackage{upquote}}{}
\IfFileExists{microtype.sty}{% use microtype if available
  \usepackage[]{microtype}
  \UseMicrotypeSet[protrusion]{basicmath} % disable protrusion for tt fonts
}{}
\makeatletter
\@ifundefined{KOMAClassName}{% if non-KOMA class
  \IfFileExists{parskip.sty}{%
    \usepackage{parskip}
  }{% else
    \setlength{\parindent}{0pt}
    \setlength{\parskip}{6pt plus 2pt minus 1pt}}
}{% if KOMA class
  \KOMAoptions{parskip=half}}
\makeatother
\usepackage{xcolor}
\usepackage{color}
\usepackage{fancyvrb}
\newcommand{\VerbBar}{|}
\newcommand{\VERB}{\Verb[commandchars=\\\{\}]}
\DefineVerbatimEnvironment{Highlighting}{Verbatim}{commandchars=\\\{\}}
% Add ',fontsize=\small' for more characters per line
\newenvironment{Shaded}{}{}
\newcommand{\AlertTok}[1]{\textcolor[rgb]{1.00,0.00,0.00}{\textbf{#1}}}
\newcommand{\AnnotationTok}[1]{\textcolor[rgb]{0.38,0.63,0.69}{\textbf{\textit{#1}}}}
\newcommand{\AttributeTok}[1]{\textcolor[rgb]{0.49,0.56,0.16}{#1}}
\newcommand{\BaseNTok}[1]{\textcolor[rgb]{0.25,0.63,0.44}{#1}}
\newcommand{\BuiltInTok}[1]{\textcolor[rgb]{0.00,0.50,0.00}{#1}}
\newcommand{\CharTok}[1]{\textcolor[rgb]{0.25,0.44,0.63}{#1}}
\newcommand{\CommentTok}[1]{\textcolor[rgb]{0.38,0.63,0.69}{\textit{#1}}}
\newcommand{\CommentVarTok}[1]{\textcolor[rgb]{0.38,0.63,0.69}{\textbf{\textit{#1}}}}
\newcommand{\ConstantTok}[1]{\textcolor[rgb]{0.53,0.00,0.00}{#1}}
\newcommand{\ControlFlowTok}[1]{\textcolor[rgb]{0.00,0.44,0.13}{\textbf{#1}}}
\newcommand{\DataTypeTok}[1]{\textcolor[rgb]{0.56,0.13,0.00}{#1}}
\newcommand{\DecValTok}[1]{\textcolor[rgb]{0.25,0.63,0.44}{#1}}
\newcommand{\DocumentationTok}[1]{\textcolor[rgb]{0.73,0.13,0.13}{\textit{#1}}}
\newcommand{\ErrorTok}[1]{\textcolor[rgb]{1.00,0.00,0.00}{\textbf{#1}}}
\newcommand{\ExtensionTok}[1]{#1}
\newcommand{\FloatTok}[1]{\textcolor[rgb]{0.25,0.63,0.44}{#1}}
\newcommand{\FunctionTok}[1]{\textcolor[rgb]{0.02,0.16,0.49}{#1}}
\newcommand{\ImportTok}[1]{\textcolor[rgb]{0.00,0.50,0.00}{\textbf{#1}}}
\newcommand{\InformationTok}[1]{\textcolor[rgb]{0.38,0.63,0.69}{\textbf{\textit{#1}}}}
\newcommand{\KeywordTok}[1]{\textcolor[rgb]{0.00,0.44,0.13}{\textbf{#1}}}
\newcommand{\NormalTok}[1]{#1}
\newcommand{\OperatorTok}[1]{\textcolor[rgb]{0.40,0.40,0.40}{#1}}
\newcommand{\OtherTok}[1]{\textcolor[rgb]{0.00,0.44,0.13}{#1}}
\newcommand{\PreprocessorTok}[1]{\textcolor[rgb]{0.74,0.48,0.00}{#1}}
\newcommand{\RegionMarkerTok}[1]{#1}
\newcommand{\SpecialCharTok}[1]{\textcolor[rgb]{0.25,0.44,0.63}{#1}}
\newcommand{\SpecialStringTok}[1]{\textcolor[rgb]{0.73,0.40,0.53}{#1}}
\newcommand{\StringTok}[1]{\textcolor[rgb]{0.25,0.44,0.63}{#1}}
\newcommand{\VariableTok}[1]{\textcolor[rgb]{0.10,0.09,0.49}{#1}}
\newcommand{\VerbatimStringTok}[1]{\textcolor[rgb]{0.25,0.44,0.63}{#1}}
\newcommand{\WarningTok}[1]{\textcolor[rgb]{0.38,0.63,0.69}{\textbf{\textit{#1}}}}
\setlength{\emergencystretch}{3em} % prevent overfull lines
\providecommand{\tightlist}{%
  \setlength{\itemsep}{0pt}\setlength{\parskip}{0pt}}
\setcounter{secnumdepth}{-\maxdimen} % remove section numbering
\ifLuaTeX
  \usepackage{selnolig}  % disable illegal ligatures
\fi
\usepackage{bookmark}
\IfFileExists{xurl.sty}{\usepackage{xurl}}{} % add URL line breaks if available
\urlstyle{same}
\hypersetup{
  hidelinks,
  pdfcreator={LaTeX via pandoc}}

\author{}
\date{}

\begin{document}

\section{Relationship Between U(λ), V(λ) and White Noise
Components}\label{relationship-between-uux3bb-vux3bb-and-white-noise-components}

The orthogonal processes U(λ) and V(λ) are \textbf{direct linear
transformations} of the underlying white noise components, scaled by the
square root of the power spectral density. This relationship embodies
the fundamental connection between time-domain randomness and
frequency-domain spectral structure.

\subsection{Mathematical Foundation}\label{mathematical-foundation}

In the spectral representation theorem, a real-valued stationary
Gaussian process has the form:

\(X(t) = \int_0^{\infty} [\cos(\lambda t) \, dU(\lambda) + \sin(\lambda t) \, dV(\lambda)]\)

The orthogonal increment processes U(λ) and V(λ) are constructed from
\textbf{complex-valued white noise measures} as described in the
literature\footnote{https://arxiv.org/pdf/2111.01084.pdf}.

\subsection{Direct Construction
Relationship}\label{direct-construction-relationship}

In the discrete implementation, the relationship is explicit:

\textbf{White Noise Components:}

\begin{itemize}
\tightlist
\item
  \(W_k^{re} \sim \mathcal{N}(0,1)\) (real part)
\item
  \(W_k^{im} \sim \mathcal{N}(0,1)\) (imaginary part)
\item
  Independent across frequencies and between real/imaginary parts
\end{itemize}

\textbf{Spectral Scaling:}
\(Z_k = \sqrt{S(\lambda_k) \Delta\lambda} \cdot (W_k^{re} + i W_k^{im})\)

\textbf{Orthogonal Process Increments:}

\begin{itemize}
\tightlist
\item
  \(dU(\lambda_k) = \text{Re}(Z_k) = \sqrt{S(\lambda_k) \Delta\lambda} \cdot W_k^{re}\)
\item
  \(dV(\lambda_k) = \text{Im}(Z_k) = \sqrt{S(\lambda_k) \Delta\lambda} \cdot W_k^{im}\)
\end{itemize}

\subsection{Key Properties}\label{key-properties}

\textbf{Isometry Preservation:} The white noise isometry
property\footnote{https://www.math.utah.edu/\textasciitilde davar/math7880/S15/Chapter6.pdf}
ensures that orthogonal white noise components map to orthogonal
increments in U(λ) and V(λ). This means:

\(E[dU(\lambda_i) dU(\lambda_j)] = E[dV(\lambda_i) dV(\lambda_j)] = 0 \text{ for } i \neq j\)
\(E[dU(\lambda_i) dV(\lambda_j)] = 0 \text{ for all } i,j\)

\textbf{Independence Structure:} Since the underlying white noise
components are independent Gaussians, and linear transformations
preserve Gaussian distributions, the orthogonal increments maintain
independence across frequencies\footnote{https://dsp.stackexchange.com/questions/35802/gaussian-white-noise-relation-between-distribution-and-correlation}\footnote{https://www.math.utah.edu/\textasciitilde davar/math7880/S15/Chapter6.pdf}.

\textbf{Spectral Coloring:} The power spectral density S(λ) acts as a
\textbf{frequency-dependent amplification factor} that transforms white
(flat spectrum) noise into colored noise with the desired spectral
characteristics.

\subsection{Physical Interpretation}\label{physical-interpretation}

\textbf{U(λ) Process:} Captures the \textbf{cosine components} of the
spectral decomposition. Each increment \(dU(\lambda_k)\) represents the
contribution of frequency \(\lambda_k\) to the ``even'' or ``symmetric''
part of the process.

\textbf{V(λ) Process:} Captures the \textbf{sine components} of the
spectral decomposition. Each increment \(dV(\lambda_k)\) represents the
contribution of frequency \(\lambda_k\) to the ``odd'' or
``antisymmetric'' part of the process.

\textbf{Randomness Inheritance:} The statistical properties
(Gaussianity, independence, zero mean) are \textbf{inherited directly}
from the white noise, while the frequency-dependent variance structure
comes from the spectral density.

\subsection{Computational
Implementation}\label{computational-implementation}

In the code implementation:

\begin{Shaded}
\begin{Highlighting}[]
\CommentTok{// White noise generation (innovation)}
\NormalTok{element}\OperatorTok{.}\FunctionTok{re}\OperatorTok{().}\FunctionTok{set}\OperatorTok{(}\NormalTok{random}\OperatorTok{.}\FunctionTok{nextGaussian}\OperatorTok{());}  \CommentTok{// W\_k\^{}re}
\NormalTok{element}\OperatorTok{.}\FunctionTok{im}\OperatorTok{().}\FunctionTok{set}\OperatorTok{(}\NormalTok{random}\OperatorTok{.}\FunctionTok{nextGaussian}\OperatorTok{());}  \CommentTok{// W\_k\^{}im}

\CommentTok{// Spectral scaling (coloring)}
\NormalTok{complexSignal}\OperatorTok{.}\FunctionTok{get}\OperatorTok{(}\NormalTok{k}\OperatorTok{).}\FunctionTok{set}\OperatorTok{(}\NormalTok{element}\OperatorTok{).}\FunctionTok{mul}\OperatorTok{(}\NormalTok{mag}\OperatorTok{,}\NormalTok{ bits}\OperatorTok{);}  \CommentTok{// Z\_k = sqrt(S(λ\_k)) * W\_k}

\CommentTok{// Orthogonal process extraction}
\NormalTok{uProcess}\OperatorTok{[}\NormalTok{k}\OperatorTok{]} \OperatorTok{=}\NormalTok{ complexSignal}\OperatorTok{.}\FunctionTok{get}\OperatorTok{(}\NormalTok{k}\OperatorTok{).}\FunctionTok{re}\OperatorTok{().}\FunctionTok{doubleValue}\OperatorTok{();}  \CommentTok{// dU(λ\_k)}
\NormalTok{vProcess}\OperatorTok{[}\NormalTok{k}\OperatorTok{]} \OperatorTok{=}\NormalTok{ complexSignal}\OperatorTok{.}\FunctionTok{get}\OperatorTok{(}\NormalTok{k}\OperatorTok{).}\FunctionTok{im}\OperatorTok{().}\FunctionTok{doubleValue}\OperatorTok{();}  \CommentTok{// dV(λ\_k)}
\end{Highlighting}
\end{Shaded}

This reveals that U(λ) and V(λ) are \textbf{not independent random
processes}, but rather \textbf{deterministic linear functionals} of the
same underlying white noise field, differentiated only by their real
versus imaginary parts and their trigonometric roles in the spectral
representation.

The white noise provides the fundamental \textbf{innovation} or
\textbf{unpredictability}, while the spectral density determines how
this innovation is \textbf{distributed across frequencies} to create the
desired correlation structure in the time domain.

⁂

\end{document}
