\documentclass{article}
\usepackage{amsmath}
\usepackage{amssymb}
\usepackage{amsthm}
\usepackage{amsfonts}
\usepackage{mathtools}

\newtheorem{theorem}{Theorem}
\newtheorem{proposition}{Proposition}
\newtheorem{corollary}{Corollary}

% Ensure all equations are numbered
\mathtoolsset{showonlyrefs=false}

\title{A NOTE ON HARMONIZABLE AND STATIONARY SEQUENCES}
\author{JOSE L. ABREU}
\date{}

\begin{document}
\maketitle

\begin{abstract}
In this paper we prove that harmonizable Hilbert sequences are projections of
stationary ones, and we use this result to give a sufficient condition for a harmonizable sequence to be deterministic.
\end{abstract}

Z will denote the set of integers; $T^n$ the n-dimensional torus for $n = 1, 2, \ldots$
and $C(T^n)$ the space of continuous complex valued functions on $T^n$. For every
$f, g\in C(T)$, $f \otimes g$ will denote the tensor product of $f$ and $g$, i.e. $(f \otimes g)(s, t) =
f(s)g(t)$ for $s, t \in T$. It is clear that $f \otimes g\in C(T^2)$ whenever $f, g\in C(T)$. c.l.s
will stand for closed linear span and l.s. for linear span.

Let $H$ be a Hilbert space with inner product $(,)$ and let $x_n \in H$ for $n \in Z$.
Define $H_{\infty}(x) = \text{c.l.s.}\{x_n:n \in Z\}$; $H_n(x) = \text{c.l.s.}\{x_k:k \leq n\}$ for each $n \in Z$,
and $H_{-\infty}(X) = \cap_{n\in Z}H_n(x)$. The Hilbert sequence $\{x_n\}$ is deterministic if
$H_{-\infty}(x) = H_{\infty}(x)$, and it is linearly free if $H_{-\infty}(x) = \{0\}$.

The covariance of a Hilbert sequence $\{x_n\}$ is defined by $B(m, n) = (x_m, x_n)$ for $m, n \in Z$. The Hilbert sequence $\{x_n\}$ is stationary if its covariance depends
only on the difference $m - n$.

A Hilbert sequence $\{x_n\}$ is called harmonizable \cite{cramer1, cramer2} if for some complex
valued Borel measure $\mu$ on $T^2$
\begin{equation}
(x_m, x_n) = \int \int_{T^2} e^{2\pi i(mx-ny)} d\mu(x, y)
\end{equation}
for every $m, n \in Z$. $\mu$ is called the covariance measure of $\{x_n\}$.

A Hilbert sequence $\{x_n\}$ is stationary if and only if it is harmonizable and its
covariance measure is concentrated on the diagonal of $T^2$.

\begin{theorem}
If $\{x_n\}$ is a harmonizable Hilbert sequence then there exists a
Hilbert space $H'$ containing $H_{\infty}(x)$ as a subspace, and a stationary Hilbert sequence $\{z_n\} \subset H'$ such that if $P:H' \to H_{\infty}(x)$ is the orthogonal projection, then
$x_n = Pz_n$ for every $n \in Z$.
\end{theorem}

\begin{proof}
Let $\mu$ be the covariance measure of $\{x_n\}$, and let $|\mu|$ denote the total
variation of $\mu$. Then $|\mu|$ is a positive, finite and symmetric Borel measure on $T^2$.
Let $\mu_0$ be the finite positive Borel measure on $T$ defined by
\begin{equation}
\mu_0(\Delta) = |\mu|(\Delta \times T) = \frac{1}{2}(|\mu|(\Delta \times T) + |\mu|(T \times \Delta))
\end{equation}
for each Borel subset $\Delta$ of $T$. Then for every continuous complex valued function $f$ on $T$, we have
\begin{equation}
\int_T f d\mu_0 = \frac{1}{2} \int \int_{T^2} f \otimes 1 d|\mu| + \frac{1}{2} \int \int_{T^2} 1 \otimes f d|\mu|.
\end{equation}

Let $\phi \in C(T)$. Then
\begin{equation}
0 \leq \int \int_{T^2} \phi \otimes \phi d\mu \leq \int \int_{T^2} |\phi \otimes \phi| d|\mu|
\end{equation}

But $|\phi(t)\phi(s)| \leq \frac{1}{2}(|\phi(t)|^2 + |\phi(s)|^2)$ for $t, s \in T$. Therefore
\begin{equation}
\int \int_{T^2} |\phi \otimes \phi| d|\mu| \leq \frac{1}{2} \int \int_{T^2} |\phi|^2 \otimes 1 d|\mu| + \frac{1}{2} \int \int_{T^2} 1 \otimes |\phi|^2 d|\mu|,
\end{equation}
and in using (3) we obtain
\begin{equation}
\int \int_{T^2} \phi \otimes \phi d\mu \leq \int_T |\phi|^2 d\mu_0.
\end{equation}

Let $\mu_1$ be the Borel measure on $T^2$ which is concentrated in the diagonal and
satisfies
\begin{equation}
\int \int_{T^2} f(x, y) d\mu_1(x, y) = \int_T f(x, x) d\mu_0(x)
\end{equation}
for every $f \in C(T^2)$. Now define $\mu_2 = \mu_1 - \mu$.
From (6) it follows that
\begin{equation}
0 \leq \int \int_{T^2} \phi \otimes \phi d\mu_2
\end{equation}
for every $\phi \in C(T)$. Therefore $\mu_2$ is the covariance measure of some harmonizable
Hilbert sequence $\{y_n\}$. (For example let $(f, g)_2 = \int \int_{T^2} f \otimes \bar{g} d\mu_2$ for $f, g \in C(T)$.
Let $\text{ker}\ \mu_2 = \{f \in C(T): (f, f)_2 = 0\}$ and define $H''$ to be the Hilbert space obtained by completing $C(T)/\text{ker}\ \mu_2$ with respect to the norm $\|f\|_2 = (f, f)_2^{1/2}$. Then let $y_n$ be the image of the function $x \mapsto e^{2\pi inx}$ under the canonical map
$C(T) \to H''$ for each $n \in Z$.) Now put $H_{\infty}(y) = \text{c.l.s.}\{y_n:n \in Z\}$ and let
$H' = H_{\infty}(x) \oplus H_{\infty}(y)$. Identify $H_{\infty}(x)$ with the subspace $H_{\infty}(x) \oplus \{0\}$ of $H'$.
$(,)_1$ will denote the innerproduct in $H'$ and $(,)_2$ will denote the innerproduct
in $H_{\infty}(y)$.

Let $z_n = x_n + y_n$ for each $n \in Z$. Then $(z_m, z_n)_1 = (x_m, x_n) + (y_m, y_n)_2$ for
$m, n \in Z$. Hence the covariance measure of $\{z_n\}$ is $\mu + \mu_2 = \mu_1$ which shows that
$\{z_n\}$ is stationary.

Finally, if $P:H' \to H_{\infty}(x)$ denotes the orthogonal projection onto $H_{\infty}(x)$,
it is obvious by construction that $x_n = Pz_n$ for each $n \in Z$. 
\end{proof}

We have shown that harmonizable Hilbert sequences are projections of stationary Hilbert sequences. Furthermore, the measure $\mu_0$ determining the covariance of the stationary sequence obtained is given by $\mu_0(\Delta) = |\mu|(\Delta \times T)$
for each Borel set $\Delta \subset T$, where $\mu$ is the covariance measure of the harmonizable
Hilbert sequence.

\begin{proposition}
Let $H'$ and $H''$ be two Hilbert spaces and $A:H' \to H''$ a bounded
linear transformation. Let $\{x_n\} \subset H'$ be a deterministic Hilbert sequence. Then
$\{Ax_n\} \subset H''$ is also a deterministic Hilbert sequence.
\end{proposition}

\begin{proof}
$\text{l.s.}\{Ax_n; n \leq k\} = A(\text{l.s.}\{x_n:n \leq k\})$ for every $k$. Since $A$ is continuous
it follows that
\begin{equation}
\text{c.l.s.}\{Ax_n:n \leq k\} \supset A(\text{c.l.s.}\{x_n:n \leq k\}) \quad k \in Z.
\end{equation}
But since $\{x_n\}$ is deterministic, $H_k(x) = H_{-\infty}(x)$ and therefore
\begin{equation}
\text{c.l.s.}\{Ax_n:n \leq k\} \supset A(H_{-\infty}(x)),
\end{equation}
and since $A(H_{-\infty}(x))$ is dense in $\text{c.l.s.}\{Ax_n:n \in Z\}$ it follows that for every $k \in Z$
\begin{equation}
\text{c.l.s.}\{Ax_n:n \leq k\} = \text{c.l.s.}\{Ax_n:n \in Z\},
\end{equation}
therefore $\{Ax_n\}$ is deterministic.
\end{proof}

Combining this proposition with the previous theorem we obtain that if the
stationary sequence $\{z_n\}$, whose projection onto $H_{\infty}(x)$ is $\{x_n\}$, is deterministic,
then $\{x_n\}$ is deterministic too.

A well-known result for stationary sequences \cite{doob, rozanov} says that a necessary
and sufficient condition for $\{z_n\}$ to be deterministic is that
\begin{equation}
\int_0^1 \log G'(t) dt = -\infty
\end{equation}
where $G(t) = \mu_0([0, t))$ for every $t \in (0, 1]$. We just proved:

\begin{corollary}
If $\{x_n\}$ is a harmonizable Hilbert sequence whose covariance
measure is $\mu$, then a sufficient condition for $\{x_n\}$ to be deterministic is that
\begin{equation}
\int_0^1 \log G'(t) dt = -\infty
\end{equation}
where $G(t) = |\mu|([0, t) \times T)$ for every $t \in (0, 1]$.
\end{corollary}

This result was proved by Cramér \cite{cramer1, cramer2} and later by Dudley \cite{dudley} using a
different approach, this new proof shows clearly that it is essentially an application of the results known for stationary sequences.

The characterization of linearly free and deterministic Hilbert sequences in
terms of their covariance distributions has been studied by Dudley \cite{dudley}. Whether
deterministic harmonizable Hilbert sequences can be characterized analytically
or not seems to be an open problem.

To conclude this paper we prove that it is not necessary for a Hilbert sequence
to be harmonizable in order to be the projection of a stationary Hilbert sequence.

Let $\{e_n\}$, $n \in Z$ be an orthonormal sequence in a Hilbert space $H$. Define the
Hilbert sequence $\{x_n\}$ as follows: $x_n = 0$ if $n \leq 0$; $x_n = e_n$ if $n > 0$.

Let $P:H \to H_{\infty}(x)$ be the projection operator onto $H_{\infty}(x) = \text{c.l.s.}\{e_n:n > 0\}$.
Then $x_n = Pe_n$ for every $n \in Z$. $\{e_n\}$ is obviously a stationary Hilbert sequence.
However, $\{x_n\}$ is not harmonizable. Indeed, suppose $\mu$ is the covariance measure
of $\{x_n\}$. The Fourier coefficients of $\mu$ are
\begin{equation}
\hat{\mu}(m, n) = \int \int_{T^2} e^{2\pi i(mx+ny)} d\mu(x, y) = (x_m, x_{-n}) =
\begin{cases}
1 & \text{if } m=-n > 0 \\
0 & \text{otherwise}.
\end{cases}
\end{equation}

Thus $\hat{\mu}(m, n)$ vanishes outside the cone $\{(m, n) \in Z^2:m = -n > 0\}$. Therefore
by a theorem of Bochner (see \cite{helson} or \cite{bochner}), it follows that $\mu$ is absolutely continuous
with respect to Lebesgue measure and, by the Riemann-Lebesgue lemma,
$\hat{\mu}(m, n) \to 0$ as $m^2 + n^2 \to \infty$. This is false according to (11).
Hence $\{x_n\}$ is not harmonizable.

This paper is part of the Ph.D. Thesis which was presented by the author to
the department of Mathematics of the Massachusetts Institute of Technology
on June, 1970. The author wishes to express his gratitude to Dr. Richard M.
Dudley for his guidance in the preparation of the original manuscript.

\begin{center}
FACULTAD DE CIENCIAS E\\
INSTITUTO DE GEOFÍSICA DE LA\\
UNIVERSIDAD NACIONAL AUTÓNOMA DE MÉXICO.
\end{center}

\begin{thebibliography}{99}
\bibitem{cramer1} H. Cramér, On some classes of non-stationary stochastic processes. Proc. IV Berkeley Symp. on Math. Stat. Prob. 2, 57-78. Univ. of California Press.
\bibitem{cramer2} H. Cramér, On the structure of purely non-deterministic stochastic processes, Ark. Mat. 4 (1961), 249-66.
\bibitem{doob} J. L. Doob, Stochastic Processes, Wiley, New York, 1953.
\bibitem{rozanov} Y. A. Rozanov. Stationary Random Processes, Holden-Day, San Francisco, Cal., 1967.
\bibitem{dudley} R. M. Dudley, Prediction theory for non-stationary sequences. Proc. V. Berkeley Symp. on Math. Stat. Prob. 2, 223-34. Univ. of California Press.
\bibitem{helson} H. Helson and D. Lowdenslager, Prediction theory and Fourier Series in several variables, Acta Math. 99 (1968), 165-202.
\bibitem{bochner} S. Bochner, Boundary values of analytic functions in several variables and almost periodic functions. Ann. Math. 45 (1944), 708-22.
\end{thebibliography}

\end{document}
