\documentclass{article}
\usepackage{amsmath}
\usepackage{amssymb}
\usepackage{amsthm}
\usepackage{hyperref}

\newtheorem{theorem}{Theorem}

\title{Tight Bounds on Covering Numbers for Integral Operators}
\author{Claude Assistant}
\date{}

\begin{document}

\maketitle

\section{Introduction}

In the study of statistical learning theory and empirical processes, covering numbers play a crucial role in quantifying the complexity of function spaces. This article focuses on the covering numbers of reproducing kernel Hilbert spaces (RKHS) associated with integral operators, presenting two bounds and discussing their relative tightness.

\section{Main Results}

Let $K$ be a symmetric positive definite kernel on a probability space $(X, \mu)$, and let $(\lambda_j)_{j\geq1}$ be the eigenvalues of the associated integral operator $T_K$, arranged in non-increasing order. The following theorem provides an upper bound on the covering number of the unit ball in the RKHS $\mathcal{H}_K$.

\begin{theorem}[Carl and Stephani, 1990]
For any $\epsilon > 0$, we have:
\begin{equation}
\log N(\epsilon, \mathcal{H}_K, L_2(\mu)) \leq \sum_{j=1}^{\infty} \log\left(1 + \frac{\lambda_j}{\epsilon^2}\right)
\end{equation}
where $N(\epsilon, \mathcal{H}_K, L_2(\mu))$ denotes the covering number of the unit ball in $\mathcal{H}_K$ with respect to the $L_2(\mu)$ norm.
\end{theorem}

This bound, originally derived by Carl and Stephani \cite{carl1990}, provides a tight estimate of the covering number in terms of the eigenvalues of the integral operator.

\section{Comparison with a Looser Bound}

For comparison, we present a looser bound that is sometimes encountered in the literature:

\begin{equation}
\log N(\epsilon, \mathcal{H}_K, L_2(\mu)) \leq 4 \sum_{j=1}^{\infty} \log\left(1 + \frac{2\sqrt{\lambda_j}}{\epsilon}\right)
\end{equation}

This bound is demonstrably looser than the one provided in Theorem 1. To see this, note that for any $\lambda_j > 0$ and $\epsilon > 0$:

\begin{equation}
\frac{\lambda_j}{\epsilon^2} \leq \left(\frac{2\sqrt{\lambda_j}}{\epsilon}\right)^2 = \frac{4\lambda_j}{\epsilon^2}
\end{equation}

Given the monotonicity of the logarithm function and the additional factor of 4 outside the sum in the looser bound, we can conclude that the bound in Theorem 1 is always tighter.

\section{Conclusion}

The tighter bound provided by Carl and Stephani offers a more precise characterization of the covering numbers for integral operators. This result has significant implications for the analysis of learning algorithms and the derivation of generalization bounds in statistical learning theory.

\bibliographystyle{plain}
\begin{thebibliography}{9}

\bibitem{carl1990}
Carl, B., \& Stephani, I. (1990). 
\textit{Entropy, compactness and the approximation of operators}. 
Cambridge University Press.

\end{thebibliography}

\end{document}
