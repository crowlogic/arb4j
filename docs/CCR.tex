\documentclass{article}
\usepackage{amsmath, amssymb}

\title{Canonical Commutation Relations: Heisenberg, Weyl, and Schrödinger Forms}
\author{Author's Name}
\date{\today}

\begin{document}
\maketitle

\section{Introduction}
This document aims to explore the canonical commutation relations in quantum mechanics, focusing on the Heisenberg, Weyl, and Schrödinger forms, and their mathematical structures and implications, including a detailed examination of the Baker-Campbell-Hausdorff (BCH) formula.

\section{Heisenberg Commutation Relation}
The Heisenberg commutation relation is defined by:
\begin{equation}
    [\hat{x}, \hat{p}] = i\hbar
\end{equation}
where \(\hat{x}\) and \(\hat{p}\) represent the position and momentum operators, respectively. This relationship underpins the Heisenberg Uncertainty Principle:
\begin{equation}
    \Delta x \Delta p \geq \frac{\hbar}{2}
\end{equation}

\section{Weyl Commutation Relation}
The Weyl form uses unitary operators constructed from exponentiating the position and momentum operators:
\begin{equation}
    e^{i\alpha \hat{x}} e^{i\beta \hat{p}} = e^{-i\alpha \beta \hbar} e^{i\beta \hat{p}} e^{i\alpha \hat{x}}
\end{equation}
where the exponentials are defined by the series expansions:
\begin{align}
    e^{i\alpha \hat{x}} &= \sum_{n=0}^\infty \frac{(i\alpha \hat{x})^n}{n!}, \\
    e^{i\beta \hat{p}} &= \sum_{n=0}^\infty \frac{(i\beta \hat{p})^n}{n!}.
\end{align}

\section{Schrödinger Commutation Relation}
The Schrödinger form of the canonical commutation relations often refers to the differential operator representation in the position basis:
\begin{equation}
    \hat{x} = x, \quad \hat{p} = -i\hbar \frac{\partial}{\partial x}
\end{equation}
In this representation, the commutation relation verifies:
\begin{equation}
    [\hat{x}, \hat{p}] = \hat{x}\hat{p} - \hat{p}\hat{x} = x(-i\hbar \frac{\partial}{\partial x}) - (-i\hbar \frac{\partial}{\partial x})x = i\hbar
\end{equation}
This illustrates the foundational postulate of quantum mechanics where measurements of position and momentum are inherently linked to the wave nature of particles.

\section{Baker-Campbell-Hausdorff Formula}
The BCH formula, fundamental in quantum mechanics and Lie algebra, is given by an infinite series:
\begin{equation}
    e^X e^Y = e^{Z(X, Y)}
\end{equation}
where \(Z(X, Y)\) is expanded as:
\begin{equation}
    Z = X + Y + \sum_{n=1}^\infty \frac{(-1)^{n-1}}{n} \sum_{\substack{r+s=n \\ r,s>0}} \frac{(ad\ X)^r (ad\ Y)^s}{r! s!} Y + \frac{(ad\ Y)^s (ad\ X)^r}{s! r!} X
\end{equation}

\section{Definition and Properties of the Lie Bracket}
The Lie bracket, denoted by \([A, B]\), is a binary operation defined in the context of Lie algebras:
\begin{equation}
    [A, B] = AB - BA
\end{equation}
Properties of the Lie bracket include:
\begin{itemize}
    \item Antisymmetry: \([A, B] = -[B, A]\)
    \item Bilinearity: \([aA + bB, C] = a[A, C] + b[B, C]\)
    \item Jacobi Identity: \([A, [B, C]] + [B, [C, A]] + [C, [A, B]] = 0\)
\end{itemize}

\end{document}
