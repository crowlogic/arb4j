\documentclass{article}
\usepackage[english]{babel}
\usepackage{geometry,amsmath,amssymb}
\geometry{letterpaper}

%%%%%%%%%% Start TeXmacs macros
\newcommand{\tmaffiliation}[1]{\\ #1}
\newtheorem{conjecture}{Conjecture}
\newtheorem{definition}{Definition}
\newtheorem{theorem}{Theorem}
%%%%%%%%%% End TeXmacs macros

\begin{document}

\title{The Birch and Swinnerton-Dyer Conjecture On The Rank Of Elliptic Curves
Over Rational Numbers}

\author{
  Stephen Crowley
  \tmaffiliation{August 28, 2025}
}

\maketitle

{\tableofcontents}

\section{The Birch and Swinnerton-Dyer Conjecture}

The Birch and Swinnerton-Dyer conjecture is fundamentally about elliptic
curves over the rational numbers and specifically about understanding when
these curves have infinitely many rational solutions versus only finitely
many.

\subsection{Foundational Definitions}

\begin{definition}
  The integers $\mathbb{Z}$ are the set $\{\ldots, - 2, - 1, 0, 1, 2,
  \ldots\}$.
\end{definition}

\begin{definition}
  The rational numbers $\mathbb{Q}$ are the set $\{p / q : p, q \in
  \mathbb{Z}, q \neq 0\}$.
\end{definition}

\begin{definition}
  A monomial in variables $x_1, \ldots, x_n$ is an expression of the form
  $x_1^{a_1} x_2^{a_2} \cdots x_n^{a_n}$ where each $a_i \geq 0$ is a
  nonnegative integer.
\end{definition}

\begin{definition}
  The degree of a monomial $x_1^{a_1} x_2^{a_2} \cdots x_n^{a_n}$ is the sum
  $a_1 + a_2 + \cdots + a_n$.
\end{definition}

\begin{definition}
  A polynomial in variables $x_1, \ldots, x_n$ with coefficients in
  $\mathbb{Q}$ is a finite linear combination of monomials: $f (x_1, \ldots,
  x_n) = \sum c_{\mathbf{a}} x_1^{a_1} \cdots x_n^{a_n}$ where $c_{\mathbf{a}}
  \in \mathbb{Q}$ and only finitely many $c_{\mathbf{a}}$ are nonzero.
\end{definition}

\begin{definition}
  A homogeneous polynomial of degree $d$ in variables $x_1, \ldots, x_n$ is a
  polynomial $f$ such that every monomial term in $f$ has total degree $d$.
  That is, if $f = \sum c_{\mathbf{a}} x_1^{a_1} \cdots x_n^{a_n}$ where
  $c_{\mathbf{a}} \neq 0$, then $a_1 + \cdots + a_n = d$ for all such terms.
\end{definition}

\begin{definition}
  The projective plane $\mathbb{P}^2 (\mathbb{Q})$ over $\mathbb{Q}$ consists
  of equivalence classes $[x : y : z]$ where $(x, y, z) \in \mathbb{Q}^3
  \setminus \{(0, 0, 0)\}$ and $(x, y, z) \sim (\lambda x, \lambda y, \lambda
  z)$ for any nonzero $\lambda \in \mathbb{Q}$.
\end{definition}

\begin{definition}
  A projective curve $C$ in $\mathbb{P}^2 (\mathbb{Q})$ is the set $C = \{[x :
  y : z] \in \mathbb{P}^2 (\mathbb{Q}) : F (x, y, z) = 0\}$ where $F (x, y,
  z)$ is a homogeneous polynomial with coefficients in $\mathbb{Q}$.
\end{definition}

\begin{definition}
  The partial derivative of a polynomial $F (x, y, z)$ with respect to $x$ is
  the polynomial $\frac{\partial F}{\partial x}$ obtained by differentiating
  each term: if $F = \sum c_{ijk} x^i y^j z^k$, then $\frac{\partial
  F}{\partial x} = \sum i \cdot c_{ijk} x^{i - 1} y^j z^k$.
\end{definition}

\begin{definition}
  A point $P = [a : b : c]$ on a projective curve $C$ defined by $F (x, y, z)
  = 0$ is singular if all three partial derivatives vanish at $P$:
  \[ \frac{\partial F}{\partial x}  (a, b, c) = \frac{\partial F}{\partial y} 
     (a, b, c) = \frac{\partial F}{\partial z}  (a, b, c) = 0 \]
\end{definition}

\begin{definition}
  A projective curve is non-singular (or smooth) if it contains no singular
  points.
\end{definition}

\begin{definition}
  The genus of a non-singular projective curve defined by a homogeneous
  polynomial of degree $d$ is $g = \frac{(d - 1)  (d - 2)}{2}$.
\end{definition}

\begin{definition}
  An elliptic curve over $\mathbb{Q}$ is a non-singular projective curve of
  genus 1 equipped with a specified rational point. It can be written in
  Weierstrass form as:
  \[ E : y^2 z = x^3 + axz^2 + bz^3 \]
  where $a, b \in \mathbb{Q}$ and the discriminant $\Delta = - 16 (4 a^3 + 27
  b^2) \neq 0$.
\end{definition}

\begin{definition}
  The point at infinity on an elliptic curve in Weierstrass form is $O = [0 :
  1 : 0]$.
\end{definition}

\begin{definition}
  An abelian group is a set $G$ with an operation $+ : G \times G \to G$ such
  that:
  \begin{enumerate}
    \item (Associativity) $(a + b) + c = a + (b + c)$ for all $a, b, c \in G$
    
    \item (Identity) There exists $0 \in G$ such that $a + 0 = 0 + a = a$ for
    all $a \in G$
    
    \item (Inverse) For each $a \in G$, there exists $- a \in G$ such that $a
    + (- a) = 0$
    
    \item (Commutativity) $a + b = b + a$ for all $a, b \in G$
  \end{enumerate}
\end{definition}

\begin{definition}
  The set $E (\mathbb{Q})$ of rational points on an elliptic curve $E$ forms
  an abelian group under the chord-and-tangent law with identity element $O$
  and group operation defined as follows: For distinct points $P = [x_1 : y_1
  : 1], Q = [x_2 : y_2 : 1] \in E (\mathbb{Q})$ with $P, Q \neq O$:
  \begin{enumerate}
    \item If $x_1 \neq x_2$, let $\ell$ be the line through $P$ and $Q$. This
    line intersects $E$ at exactly three points: $P$, $Q$, and a third point
    $R$. Define $P + Q$ to be the point such that $P + Q + R = O$ under the
    group law.
    
    \item If $x_1 = x_2$ and $y_1 = - y_2$, then $P + Q = O$.
    
    \item If $P = Q$ and $y_1 \neq 0$, let $\ell$ be the tangent line to $E$
    at $P$. This intersects $E$ at $P$ (with multiplicity 2) and one other
    point $R$. Define $2 P$ such that $2 P + R = O$.
    
    \item For any $P \in E (\mathbb{Q})$: $P + O = O + P = P$.
  \end{enumerate}
\end{definition}

\begin{definition}
  The rank of an abelian group $G$ is the dimension of $G \otimes \mathbb{Q}$
  as a $\mathbb{Q}$-vector space.
\end{definition}

\begin{definition}
  A square-free integer is an integer $n$ such that no perfect square other
  than 1 divides $n$.
\end{definition}

\subsection{L-Functions}

\begin{definition}
  Let $\mathbb{F}_p$ denote the field with $p$ elements, where $p$ is prime.
\end{definition}

\begin{definition}
  An elliptic curve $E$ over $\mathbb{Q}$ has good reduction at a prime $p$ if
  the curve obtained by reducing the coefficients of its Weierstrass equation
  modulo $p$ is non-singular over $\mathbb{F}_p$.
\end{definition}

\begin{definition}
  An elliptic curve $E$ over $\mathbb{Q}$ has multiplicative reduction at a
  prime $p$ if the reduced curve modulo $p$ has exactly one singular point,
  which is a node (intersection of two distinct lines).
\end{definition}

\begin{definition}
  An elliptic curve $E$ over $\mathbb{Q}$ has additive reduction at a prime
  $p$ if the reduced curve modulo $p$ has a cusp or worse singularity.
\end{definition}

\begin{definition}
  The Hasse-Weil L-function $L (E, s)$ of an elliptic curve $E$ over
  $\mathbb{Q}$ is defined as the Euler product:
  \[ L (E, s) = \prod_{p \text{prime}} L_p (E, s)^{- 1} \]
  which converges absolutely for $\mathrm{Re} (s) > \frac{3}{2}$, where each
  local L-factor $L_p (E, s)$ is defined as:
  \begin{enumerate}
    \item If $E$ has good reduction at $p$: $L_p (E, s) = 1 - a_p p^{- s} +
    p^{1 - 2 s}$ where $a_p = p + 1 - |E (\mathbb{F}_p) |$
    
    \item If $E$ has multiplicative reduction at $p$: $L_p (E, s) = 1 - a_p
    p^{- s}$ where $a_p = \pm 1$
    
    \item If $E$ has additive reduction at $p$: $L_p (E, s) = 1$
  \end{enumerate}
\end{definition}

\begin{definition}
  The order of vanishing of a function $f (s)$ at $s = s_0$ is the largest
  integer $k$ such that $(s - s_0)^k$ divides $f (s)$ in a neighborhood of
  $s_0$.
\end{definition}

\begin{definition}
  The Tamagawa number $c_p (E)$ of an elliptic curve $E$ at a prime $p$ is the
  index $[E (\mathbb{Q}_p) : E^0 (\mathbb{Q}_p)]$, where $E^0 (\mathbb{Q}_p)$
  is the subgroup of points with good reduction.
\end{definition}

\begin{definition}
  The real period $\Omega_E$ of an elliptic curve $E$ is $\int_{E
  (\mathbb{R})} | \omega |$ where $\omega$ is the invariant differential on
  $E$.
\end{definition}

\begin{definition}
  The Shafarevich-Tate group $\mathrm{X} (E /\mathbb{Q})$ is the kernel of the
  map $H^1 (\mathbb{Q}, E) \to \prod_v H^1 (\mathbb{Q}_v, E)$ where the
  product runs over all places $v$ of $\mathbb{Q}$.
\end{definition}

\begin{definition}
  The regulator $\mathrm{Reg} (E /\mathbb{Q})$ is the determinant of the Gram
  matrix of the canonical height pairing on the free part of $E (\mathbb{Q})$.
\end{definition}

\subsection{The Conjecture}

\begin{conjecture}
  [Birch and Swinnerton-Dyer] Let $E$ be an elliptic curve over $\mathbb{Q}$.
  Then:
  \begin{enumerate}
    \item The Shafarevich-Tate group $\mathrm{X} (E /\mathbb{Q})$ is finite.
    
    \item $\mathrm{ord}_{s = 1} L (E, s) = \mathrm{rank}_{\mathbb{Z}} E
    (\mathbb{Q})$
    
    \item $\lim_{s \to 1}  \frac{L (E, s)}{(s - 1)^r} = \frac{\Omega_E \cdot
    \mathrm{Reg} (E /\mathbb{Q}) \cdot | \mathrm{X} (E /\mathbb{Q}) |  \prod_p
    c_p (E)}{|E (\mathbb{Q})_{\mathrm{tors}} |^2}$ where $r =
    \mathrm{rank}_{\mathbb{Z}} E (\mathbb{Q})$.
  \end{enumerate}
\end{conjecture}

\subsection{Connection to Square-Free Numbers}

\begin{definition}
  The quadratic twist of an elliptic curve $E : y^2 = x^3 + ax + b$ by a
  square-free integer $n$ is the curve $E_n : ny^2 = x^3 + ax + b$.
\end{definition}

\begin{definition}
  A congruent number is a square-free positive integer $n$ that is the area of
  a right triangle with rational side lengths.
\end{definition}

\begin{theorem}
  Let $n$ be a square-free positive integer. Then $n$ is a congruent number if
  and only if the elliptic curve $E_n : y^2 = x^3 - n^2 x$ has positive rank.
  By the Birch and Swinnerton-Dyer conjecture, this is equivalent to $L (E_n,
  1) = 0$.
\end{theorem}

The conjecture involves square-free numbers because the behavior of
L-functions $L (E_n, s)$ at $s = 1$ for quadratic twists by square-free
integers $n$ determines the solvability of fundamental Diophantine equations.

\end{document}
