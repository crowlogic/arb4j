\documentclass{article}
\usepackage[english]{babel}
\usepackage{geometry,amssymb,latexsym,theorem}
\geometry{letterpaper}

%%%%%%%%%% Start TeXmacs macros
\newcommand{\cdummy}{\cdot}
\newcommand{\nobracket}{}
\newcommand{\tmem}[1]{{\em #1\/}}
\newenvironment{proof}{\noindent\textbf{Proof\ }}{\hspace*{\fill}$\Box$\medskip}
\newtheorem{definition}{Definition}
{\theorembodyfont{\rmfamily}\newtheorem{remark}{Remark}}
\newtheorem{theorem}{Theorem}
%%%%%%%%%% End TeXmacs macros

\begin{document}

\title{Uniform Convergence of Orthonormal Basis Projections in RKHS}

\author{Stephen Crowley}

\date{March 28, 2025}

\maketitle

\begin{definition}
  [Reproducing Kernel Hilbert Space] A Hilbert space $H$ of functions on a set
  $D$ is called a reproducing kernel Hilbert space (RKHS) if there exists a
  function $k : D \times D \to \mathbb{R}$ such that:
  \begin{enumerate}
    \item For every $x \in D$, the function $k_x (\cdummy) = k (\cdummy, x)$
    belongs to $H$.
    
    \item For every $x \in D$ and every $f \in H$, the reproducing property
    holds: $f (x) = \langle f, k_x \rangle_H$.
  \end{enumerate}
  The function $k$ is called the reproducing kernel of $H$.
\end{definition}

\begin{definition}
  [Orthonormal Basis in RKHS]\label{def:orthonormal_basis}A sequence of
  functions $\{e_n \}_{n = 1}^{\infty} \subset H$ is an orthonormal basis of
  the RKHS $H$ if:
  \begin{enumerate}
    \item Orthonormality: For all indices $n, m$, $\langle e_n, e_m \rangle_H
    = \delta_{nm}$, where $\delta_{nm}$ is the Kronecker delta.
    
    \item Completeness: The span of $\{e_n \}_{n = 1}^{\infty}$ is dense in
    $H$, which means:
    \begin{enumerate}
      \item For any $f \in H$, if $\langle f, e_n \rangle_H = 0$ for all $n$,
      then $f = 0$.
      
      \item Equivalently, every function $f \in H$ can be represented as
      \[ f = \sum_{n = 1}^{\infty} \langle f, e_n \rangle_H e_n \]
      with convergence in the $H$-norm:
      \[ \lim_{N \to \infty} \left| f - \sum_{n = 1}^N \langle f, e_n
         \rangle_H e_n \right|_H = 0. \]
    \end{enumerate}
    \item Parseval's Identity: For any $f \in H$,
    \[ \|f\|_H^2 = \sum_{n = 1}^{\infty} | \langle f, e_n \rangle_H |^2 . \]
  \end{enumerate}
  In an RKHS, each basis function satisfies the reproducing property: $e_n (x)
  = \langle e_n, k (\cdot, x) \rangle_H$ for all $x \in D$.
\end{definition}

\begin{theorem}
  \label{thm:uniform_convergence}Let $H$ be a reproducing kernel Hilbert space
  (RKHS) on a set $D$ with reproducing kernel $k$. Suppose that:
  \begin{enumerate}
    \item $\{e_n \}_{n = 1}^{\infty}$ is an orthonormal basis of $H$ as
    defined in Definition \ref{def:orthonormal_basis}.
    
    \item The kernel is uniformly bounded on $D$; that is, there exists a
    constant $M > 0$ such that
    \[ \sup_{x \in D}  \sqrt{k (x, x)} \le M. \]
  \end{enumerate}
  Then for any function $f \in H$ with orthonormal expansion
  \[ f = \sum_{n = 1}^{\infty} c_n e_n, \]
  where $c_n = \langle f, e_n \rangle_H$, the partial sums
  \[ S_N f = \sum_{n = 1}^N c_n e_n \]
  converge uniformly to $f$ on $D$; in other words,
  \[ \lim_{N \to \infty} \sup_{x \in D} | S_N f (x) - f (x) | = 0. \]
\end{theorem}

\begin{proof}
  By the completeness property of the orthonormal basis (Definition
  \ref{def:orthonormal_basis}), every function $f \in H$ can be represented by
  its orthonormal expansion that converges in the $H$-norm. Since $H$ is an
  RKHS, the evaluation functional at any $x \in D$ is bounded by $\sqrt{k (x,
  x)}$. Specifically, for each fixed $x \in D$:
  \[ \nobracket | f (x) - S_N f (x) | = | \langle f - S_N f, k (\cdot, x)
     \rangle_H | \le \|f - S_N f\|_H  \sqrt{k (x, x)} . \]
  Taking the supremum over $x \in D$ yields
  \[ \sup_{x \in D} | f (x) - S_N f (x) | \le \|f - S_N f\|_H  \hspace{0.17em}
     \sup_{x \in D}  \sqrt{k (x, x)} \le M \|f - S_N f\|_H . \]
  From the convergence property of orthonormal bases, we have:
  \[ \lim_{N \to \infty} \|f - S_N f\|_H = 0. \]
  For any $\varepsilon > 0$, choose $N$ such that for all $n \ge N$:
  \[ \|f - S_n f\|_H < \frac{\varepsilon}{M} . \]
  Then for all $n \ge N$:
  \[ \sup_{x \in D} | f (x) - S_n f (x) | < \varepsilon . \]
  Thus:
  \[ \lim_{N \to \infty} \sup_{x \in D} |S_N f (x) - f (x) | = 0. \]
\end{proof}

\begin{remark}
  The uniform boundedness condition on the kernel is essential. Without it,
  norm convergence in the RKHS would not necessarily imply uniform convergence
  of the function evaluations on the domain.
\end{remark}

\begin{remark}
  It is important to emphasize that the domain $D$ in Theorem
  \ref{thm:uniform_convergence} is not required to be compact. The result
  holds for any domain, including unbounded domains such as $D =\mathbb{R}^n$
  or $D = [0, \infty)$, provided that the kernel is uniformly bounded on that
  domain.
\end{remark}

\begin{remark}
  \label{rem:mercer_uniqueness}The uniform convergence described in Theorem
  \ref{thm:uniform_convergence} applies to any orthonormal basis when
  expanding functions in the RKHS $H$, whereas when expanding the reproducing
  kernel $k (x, y)$ itself, only the Mercer eigenbasis $\{e_n^{\ast} \}$,
  defined by the equation
  \[ \int_D k (x, y) e_n^{\ast} (y)  \hspace{0.17em} dy = \lambda_n e_n^{\ast}
     (x), \]
  converges uniformly, whereas non-Mercer orthonormal bases converge
  pointwise.
\end{remark}

\begin{thebibliography}{9}
  {\bibitemwithkey{Riesz(1907)}{riesz1907}} Riesz, F. (1907). Sur les
  syst{\`e}mes orthogonaux de fonctions. {\tmem{Comptes rendus de
  l'Acad{\'e}mie des sciences}}, 144:615--619.
  
  {\bibitemwithkey{Fischer(1907)}{fischer1907}} Fischer, E. (1907). Sur la
  convergence en moyenne. {\tmem{Comptes rendus de l'Acad{\'e}mie des
  sciences}}, 144:1022--1024.
  
  {\bibitemwithkey{Aronszajn(1950)}{aronszajn1950}} Aronszajn, N. (1950).
  Theory of reproducing kernels. {\tmem{Transactions of the American
  Mathematical Society}}, 68(3):337--404.
  
  {\bibitemwithkey{Berlinet and Thomas-Agnan(2004)}{berlinet2004}} Berlinet,
  A. and Thomas-Agnan, C. (2004). {\tmem{Reproducing Kernel Hilbert Spaces in
  Probability and Statistics}}. Springer, Boston, MA.
\end{thebibliography}

q

\end{document}
