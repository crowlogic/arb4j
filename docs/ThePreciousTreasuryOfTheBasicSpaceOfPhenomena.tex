\documentclass{article}
\usepackage[english]{babel}
\usepackage{geometry,amsmath,latexsym,theorem}
\geometry{letterpaper}

%%%%%%%%%% Start TeXmacs macros
\newcommand{\tmtextbf}[1]{\text{{\bfseries{#1}}}}
\newenvironment{proof}{\noindent\textbf{Proof\ }}{\hspace*{\fill}$\Box$\medskip}
\newtheorem{corollary}{Corollary}
\newtheorem{definition}{Definition}
\newtheorem{lemma}{Lemma}
\newtheorem{proposition}{Proposition}
{\theorembodyfont{\rmfamily}\newtheorem{remark}{Remark}}
\newtheorem{theorem}{Theorem}
%%%%%%%%%% End TeXmacs macros

\begin{document}

\title{
  The Precious Treasury of the Basic Space of Phenomena:\\
  A Mathematical Formulation of Enlightened Intent
}

\author{Dzogchen Mathematical Framework}

\date{}

\maketitle

\begin{definition}
  [Awakened Mind] Awakened mind is by nature primordially pure.
\end{definition}

\begin{definition}
  [True Nature of Phenomena] The true nature of phenomena is such that there
  is nothing to discard or adopt, nothing that comes or goes, nothing to
  achieve by trying.
\end{definition}

\begin{theorem}
  [Utter Lucidity] The sun and moon of utter lucidity arise when one rests
  naturally in the spacious expanse that is the true nature of phenomena.
\end{theorem}

\begin{proof}
  Without sense objects being blocked or mind being reified, if there is no
  straying from the natural state of spontaneous equalness, you arrive at the
  enlightened intent of supreme spaciousness---Samantabhadra.
\end{proof}

\begin{definition}
  [Naturally Limpid State] Without the arising and subsiding of thoughts,
  there is a naturally limpid, pristine state, like the unwavering evenness of
  a limpid ocean.
\end{definition}

\begin{theorem}
  [Naturally Occurring Timeless Awareness] Free of the occurrence of or
  involvement in thoughts, free of hope or fear, you abide within the state of
  naturally occurring timeless awareness, the true nature of which is
  profoundly lucid.
\end{theorem}

\begin{proof}
  Without the compulsions of ordinary mind, there is an unfeigned state, a
  natural settling, uncontrived and unadulterated, though it cannot be
  characterized with words. This absorption in the expanse of being, the true
  nature of which cannot be characterized, involves neither meditation nor
  something to meditate on, and so laxity and agitation dissipate naturally,
  and enlightened intent occurs naturally.
\end{proof}

\begin{lemma}
  [Dynamic Energy of Awareness] All consuming thought patterns cannot be
  abandoned by being renounced, for they are the dynamic energy of awareness.
\end{lemma}

\begin{proof}
  Their true nature is such that there are no distinctions, nothing to
  differentiate or exclude. So that nature is not ensured by achievement, but
  arises as basic space.
\end{proof}

\begin{corollary}
  [Perception of Samsara] Without rejecting samsara, you perceive it to be
  naturally occurring timeless awareness, through the pure yoga of the dynamic
  energy of the vast expanse of being.
\end{corollary}

\begin{theorem}
  [Meditative Absorption] In the timeless unity of sensory appearances and
  mind, the naturally settled state that is the true nature of phenomena,
  meditative absorption is experienced as an unwavering ongoing flow.
\end{theorem}

\begin{proof}
  Thus the Vajrapani, the most excellent enlightened mind of Samantabhadra, is
  the most sublime, spacious state, equal to space.
\end{proof}

\begin{proposition}
  [Supreme Meditation] The most sublime meditation of all involves no
  differentiation or exclusion. It is spontaneously present as a superb,
  timelessly infinite monarch.
\end{proposition}

\begin{theorem}
  [Utter Lucidity Flow] The ongoing flow of utter lucidity, timeless and
  omnipresent, is spontaneously present within this context in which nothing
  is discarded or adopted.
\end{theorem}

\begin{proof}
  And so it is the most sublime enlightened intent, the basic space of
  phenomena, the nature of samsara and nirvana.
\end{proof}

\begin{definition}
  [Vast Expanse] This vast expanse, unwavering, indescribable and equal to
  space, is timelessly and innately present in all beings.
\end{definition}

\begin{lemma}
  [Ordinary Confused Mind] It is the ordinary confused mind that perceives
  sensory appearances to be something other than oneself. It is the ordinary
  confused mind that believes in meditation and making an effort.
\end{lemma}

\begin{theorem}
  [True Nature of Confusion] The true nature of confusion is the realm of
  equalness, the natural state of rest, the natural expanse that is unwavering
  and primordially pure.
\end{theorem}

\begin{proof}
  There is nothing to do and no effort to make. Whether or not you are resting
  is irrelevant.
\end{proof}

\begin{theorem}
  [Four Aspects of Omnipresent Awareness] There exist four fundamental aspects
  of omnipresent awareness: view, meditation, conduct, and fruition.
\end{theorem}

\begin{proof}
  \tmtextbf{View:} Given the unchanging, spontaneously present nature of
  phenomena, if you look again and again with self-knowing awareness, free of
  any complicating conceptual framework, you will see that there is nothing to
  look at, nothing to look at. This is the view of omnipresent awareness.
  
  \tmtextbf{Meditation:} Given awareness which is not cultivated in meditation
  and which nothing is discarded or adopted, if you meditate again and again,
  you will see that there is nothing to cultivate in meditation, nothing to
  cultivate in meditation. This is the meditation of omnipresent awareness.
  
  \tmtextbf{Conduct:} Given the way of abiding, non-dual and free of
  acceptance and rejection, if you engage in conduct again and again, you will
  see that there is no conduct to enact, no conduct to enact. This is the
  conduct of omnipresent awareness.
  
  \tmtextbf{Fruition:} Given spontaneous presence, timelessly ensured and free
  of hope and fear, if you strive to achieve again and again, you will see
  that there is nothing to achieve, nothing to achieve. This is the fruition
  of omnipresent awareness.
\end{proof}

\begin{definition}
  [State of Equalness] Within the state of equalness, there are no thoughts
  about sense objects and no reification of ordinary mind.
\end{definition}

\begin{lemma}
  [Pacification of Hope and Fear] So the occurrence of an involvement in hope
  and fear are pacified.
\end{lemma}

\begin{theorem}
  [Omnipresent State] Abiding in the equalness of sense objects and mind means
  that as a matter of course, there is no straying from the expanse that is
  the true nature of phenomena. One abides in an omnipresent state in which
  what are characterized as sense objects do not exist as objects.
\end{theorem}

\begin{corollary}
  [Great Perfection] Since there is omnipresent awareness, timeless and
  non-dual within the state of great perfection, the indivisibility of samsara
  and nirvana, everything is in a state of infinite evenness without
  acceptance or rejection.
\end{corollary}

\begin{theorem}
  [Universal Equality in Basic Space] All phenomena are equal in basic space.
\end{theorem}

\begin{proof}
  What is tangible and what is intangible are equal in basic space. Buddhas
  and ordinary beings are equal in basic space. Relative reality and ultimate
  reality are equal in basic space. Flaws and positive qualities are equal in
  basic space. And all directions above, below and in between are equal in
  basic space.
\end{proof}

\begin{corollary}
  [Equal Arising] Therefore, whatever display arises from that naturally
  occurring state, even as it arises, things arise equally, none being better
  or worse. What need is there to accept or reject them by applying antidotes?
\end{corollary}

\begin{proposition}
  [Equal Abiding and Freedom] When things abide, they abide equally, none
  being better or worse. Whatever is now taking place in your mind, rest in
  natural peace. When things are free, they are equally free, none being
  better or worse. In the wake of being conscious of them, do not continue to
  suppress or indulge in them.
\end{proposition}

\begin{theorem}
  [Dynamic Energy and Display] Within awakened mind itself, the expanse of the
  ground of being, the way in which everything arises as its dynamic energy
  and display is unpredictable.
\end{theorem}

\begin{proof}
  Even as things arise equally, they arise within that primordial expanse.
  Even as they arise unequally, they arise within the basic space of their
  equalness. Even as they abide equally, their true nature is a natural state
  of rest. Even as they abide unequally, they abide within the basic space of
  their equalness. Even as they are freed equally, this constitutes the
  expanse of naturally occurring timeless awareness. Even as they are freed
  unequally, they are freed within the basic space of their equalness.
\end{proof}

\begin{lemma}
  [Timeless Non-Existence] Given naturally occurring awareness, the timeless
  equalness of everything:
  \begin{itemize}
    \item Arising and non-arising are timelessly non-existent in basic space
    
    \item Abiding and non-abiding are timelessly non-existent in basic space
    
    \item Freedom and the absence of freedom are timelessly non-existent in
    basic space
  \end{itemize}
\end{lemma}

\begin{theorem}
  [Natural Holding to Place] Within awareness, a supreme state of unwavering
  equalness, even as things arise, they arise naturally, holding to their own
  place. Even as they abide, they abide naturally, holding to their own place.
  Even as they are freed, they are freed naturally, holding to their own
  place.
\end{theorem}

\begin{theorem}
  [Nature of Space and Timelessness] Given that awareness is unchanging and
  free of elaboration, everything is of the nature of space.
\end{theorem}

\begin{proof}
  What arises, arises timelessly. What abides, abides timelessly. And what is
  free, is free, timelessly.
\end{proof}

\begin{lemma}
  [Simultaneous Arising and Freedom] Thoughts arise, abide and are freed.
  Their simultaneous arising and being freed is uninterrupted.
\end{lemma}

\begin{theorem}
  [Transcendence of Causality] Since it is uninterrupted, there is no
  separation into cause and effect. Since there is no cause and effect, the
  abyss of samsara has been crossed. Since there is no longer an abyss, where
  could one go astray?
\end{theorem}

\begin{corollary}
  [Unchanging Expanse] The expanse of Samantabhadra is timelessly unchanging.
  The expanse of Vajrasattva is without transition or change. The term
  Buddhahood is nothing more than a label, for what is simply recognition of
  the very essence of being, the way of abiding.
\end{corollary}

\begin{theorem}
  [Infinite Evenness] With the realization of this, there are no phenomena to
  accept or reject. So all things are in a state of infinite evenness, that is
  their sole true nature.
\end{theorem}

\begin{proof}
  As on the isle of gold, there is no division or exclusion. This nature is
  not subject to limitation, for error and obscuration have been seen through.
\end{proof}

\begin{proposition}
  [Three Kayas Spontaneous Perfection] Within awakened mind itself, in which
  there are no pitfalls, the three kayas involving no effort are spontaneously
  perfect. So the phrase "beyond imagination or expression" is a mere figure
  of speech.
\end{proposition}

\begin{theorem}
  [Unrestricted Appearances] Sensory appearances are unrestricted, awareness
  is evident and naturally occurring.
\end{theorem}

\begin{proof}
  Since the genuine state of uncontrived rest is unobscured and unobstructed,
  with no division into outer and inner, it is evident as the supreme nature
  of phenomena.
\end{proof}

\begin{remark}
  [Relaxation Instructions] Let your mind and body relax deeply in a carefree
  state, with an easy going attitude like a person who has nothing more to do.
  Let your mind and body rest in whatever way is comfortable, neither tense
  nor loose.
\end{remark}

\begin{theorem}
  [Movement Within Fundamental Nature] However things stay, they stay within
  their fundamental nature. However they dwell, they dwell within their
  fundamental nature. However they move, they move within their fundamental
  nature.
\end{theorem}

\begin{corollary}
  [No Coming or Going] Fundamentally there is no coming or going within the
  basic space of enlightenment. The enlightened forms of victorious ones do
  not come or go.
\end{corollary}

\begin{lemma}
  [Description and Expression] However description occurs, it occurs within
  its fundamental nature. However expression occurs, it occurs within its
  fundamental nature. Fundamentally there is no description or expression
  within awakened mind. The enlightened speech of the victorious ones of the
  three times is indescribable and inexpressible.
\end{lemma}

\begin{lemma}
  [Thinking and Conceptualization] However thinking occurs, it occurs within
  its fundamental nature. However conceptualization occurs, it occurs within
  its fundamental nature. There is never any thinking or conceptualizing
  within awakened mind. The enlightened mind of the victorious ones of the
  three times is free of thinking and conceptualizing.
\end{lemma}

\begin{definition}
  [Three Kayas Structure] Since what is non-existent can occur in any way at
  all, there is nirmanakaya. Since the richness of being enjoys itself, there
  is sambhogakaya. Since no substantial basis for these two exist, there is
  dharmakaya. Through fruition is the expanse within which the three kayas are
  spontaneously present.
\end{definition}

\begin{theorem}
  [Enlightened Intent Recognition] Within the very state that is the vast
  expanse of awakened mind, the concepts of ordinary thinking do not occur. If
  the characteristics of ordinary consciousness do not stir in the mind, that
  itself is enlightened intent, the unique state of Buddhahood.
\end{theorem}

\begin{proposition}
  [Nature of Enlightenment] The nature of enlightenment is similar to the
  spacious vault of the sky. The most sublime form of meditation involves no
  recollection or thinking.
\end{proposition}

\begin{theorem}
  [Unchanging Nature] One's nature is unwavering and uncontrived. Unplanned
  and completely free of the formation of ideas, the true nature of phenomena,
  the naturally settled state is without transition or change throughout the
  three times.
\end{theorem}

\begin{corollary}
  [Supreme Meditation Form] The most sublime form of meditation involves no
  stirring or proliferation of all consuming thoughts.
\end{corollary}

\begin{theorem}
  [Sacred State of Mind] Any abiding in suchness is the sacred state of mind,
  the unique state of Buddhahood, free of all characterization.
\end{theorem}

\begin{proof}
  It is the unwavering basic space of phenomena, a state of evenness that
  transcends reifying concepts. This is the expanse of the enlightened intent
  of the victorious ones, the sublime, spacious nature of being.
\end{proof}

\begin{lemma}
  [Abandonment of Contrivance] When the bonds of physical and mental
  contrivance are abandoned, there is an unfeigned relaxation.
\end{lemma}

\begin{theorem}
  [Non-Wavering from True Nature] No matter what recollection stirs in the
  mind, if you do not waver from the context of the true nature of phenomena,
  that of resting in the ground of being, everything is a spacious expanse of
  the enlightened intent of Samantabhadra.
\end{theorem}

\begin{corollary}
  [Natural Settling] Since nothing is reified or discarded, there is none of
  the tension or laxity of the compulsive mind. The unrestricted state of
  natural settling, just as it is, is ensured as a matter of course.
  Unwavering, infinite evenness is an expanse with no fixed dimension.
\end{corollary}

\begin{proposition}
  [Sky-like Enlightened Intent] If all ordinary thinking occurs naturally and
  is pacified naturally, that is the sky-like enlightened intent of
  Vajrasattva.
\end{proposition}

\begin{theorem}
  [Uncontrived Expanse Warning] If you maintain an undistracted state within
  the uncontrived expanse of being, even engaging in thoughts concerning sense
  objects is within the scope of the true nature of phenomena. As for that
  true nature, it is non-conceptual and as spacious as the sky.
\end{theorem}

\begin{lemma}
  [Danger of Contrivance] But if you try to contrive it deliberately and
  compulsively, it becomes a cage of characteristics. Though you may spend day
  and night in such meditation, that is the bondage of fixation, pure and
  simple. As a victorious one stated that it resembles the meditative
  stability of the gods.
\end{lemma}

\begin{corollary}
  [Crucial Settling] Therefore, it is extremely crucial that your mind, which
  is undistracted and which effort and striving have been eradicated, settle
  naturally beyond reifying effort.
\end{corollary}

\begin{theorem}
  [Limitation-Free Awareness] Since naturally occurring, timeless awareness is
  without limitation or bias, it cannot be characterized as some thing, for
  all elaboration naturally subsides.
\end{theorem}

\begin{proof}
  Therefore, give up creating more concepts. Train in the ultimate meaning of
  supreme spaciousness, free of any foundation.
\end{proof}

\begin{definition}
  [The Unique Nature Quintuple] The unique nature of phenomena is naturally
  occurring timeless awareness. The unique view is freedom from the
  limitations of elaboration. In the unique meditation, nothing is discarded
  or adopted. Nothing comes or goes. In the unique conduct, there is no
  duality of acceptance and rejection. The unique fruition is free of the
  duality of renunciation and attainment. This is the enlightened intent of
  naturally occurring spontaneous presence.
\end{definition}

\begin{theorem}
  [Universal Primordial State] The true nature of all phenomena, in their
  entirety, the universe of appearances and possibilities, whether of samsara
  or nirvana, is the primordial state.
\end{theorem}

\begin{proof}
  Since it does not stray from naturally occurring timeless awareness itself,
  understand it to be enlightened intent with everything at rest in the ground
  of being.
\end{proof}

\begin{remark}
  [How to Rest] Concerning phenomena that manifest as myriad sense objects,
  without thinking in any way, this is how to rest. Rest spontaneously in the
  naturally settled state, free of the proliferation and resolution of
  thoughts. Abide as a matter of course within the expanse of equalness, the
  true nature of phenomena.
  
  Neither focusing your senses on, nor letting your gaze wander to, the
  manifestations of sensory appearances in all their variety. Neither thinking
  of self, nor conceiving of other, rest naturally lucid in the supremely
  spacious state of complete openness.
\end{remark}

\begin{theorem}
  [Meditative Absorption Experience] Given the enlightened intent of naturally
  occurring timeless awareness, in which everything is equal, expansive and
  elevated mind, free of the proliferation and resolution of thoughts, the
  experience of blending with space, without any division into outer and
  inner, or in between, arises as meditative absorption that is blissful,
  clear and free of elaboration.
\end{theorem}

\begin{lemma}
  [No Division into Outer and Inner] Given the enlightened intent of the true
  nature of phenomena, which never strays from a state of rest, the ground of
  being, there is no division into outer and inner. For that nature is free of
  the elaborations of dualistic perception.
\end{lemma}

\begin{proposition}
  [Freedom from Fixation] There is no ordinary mind fixating on something
  other, a sense object. So nothing is reified as an object, and your
  perceptions of the universe are free of fixation. No context exists for
  taking rebirth in samsara. This is similar to space.
\end{proposition}

\begin{theorem}
  [Dharmakaya Realization] There is no inner concept of mind as self. So
  nothing is reified as a subject. And the all-consuming thought patterns of
  conditioned existence are stilled. The potential for rebirth in samsara is
  cut through at the root.
\end{theorem}

\begin{proof}
  At that point, you have arrived at the enlightened intent of dharmakaya,
  like space, in which there is no division into outer and inner, and no frame
  of reference for phenomena based on confusion. You have touched on the point
  of resolution, and since there is no coming or going, everything is an
  infinite expanse, the pure realm of Samantabhadra. You have reached the
  sublime palace of dharmakaya.
\end{proof}

\begin{theorem}
  [Freedom from Conditioned Existence] If awareness in the moment does not
  stray from the ground of being, familiarization with that experience negates
  any furthering of conditioned existence.
\end{theorem}

\begin{proof}
  You are free of the karma and habitual patterns that perpetuate rebirth. You
  have come to the decisive experience of causality, described as the
  equalness of samsara and nirvana. You have arrived at the heart essence of
  enlightenment, which does not abide in conditioned existence, nor the state
  of peace. It is crucial that you distinguish between this and a one-pointed
  state of calm abiding. This is the enlightened intent of natural great
  perfection.
\end{proof}

\begin{lemma}
  [Straying from Fundamental Nature] If you stray from your fundamental
  nature, the functioning of conceptual mind is samsara, pure and simple. It
  involves cause and effect. You have not come to the decisive experience. The
  person who makes this mistake falls lower and lower.
\end{lemma}

\begin{theorem}
  [Secret Great Perfection] Therefore, the sublime secret great perfection
  does not stray from basic space and the expressions of dynamic energy
  resolve within the ground of being. Enlightened intent abides as an
  unwavering state of equalness.
\end{theorem}

\begin{corollary}
  [No Cause and Effect in Context] Within this context there is no cause and
  effect, no concerted effort. View, for example, cannot be cultivated in
  meditation. Although the mode of cessation is described as having neither
  center nor limit, when dynamic energy itself deviates from this natural
  state, the myriad display arises as the multiplicity of the universe of
  appearances and possibilities. So never say categorically there is no cause
  and effect.
\end{corollary}

\begin{theorem}
  [Interdependence] Interdependence ensures that conditioned composite
  phenomena are beyond enumeration and imagination. Confused perception in
  samsara and even states of peace and bliss are beyond enumeration and
  imagination.
\end{theorem}

\begin{proof}
  All of this constitutes the very process of interdependence, which is the
  coming together of causes and conditions.
\end{proof}

\begin{theorem}
  [Fundamentally Unconditioned Nature] If you evaluate your fundamentally
  unconditioned nature, you find it has never existed as anything whatsoever.
\end{theorem}

\begin{proof}
  So too, in taking this as your path, you have no frame of reference
  whatsoever for straying from that fundamentally unconditioned nature in all
  its immediacy. Rather, you appreciate it within the context of enlightened
  intent. Having reached the ultimate state in the immediacy of your
  fundamentally unconditioned nature, you are not sullied by anything at all.
  
  Afflictive emotions, karma and habitual patterns have no support within this
  vast expanse, but are the playing out of magical games of illusion. You must
  be liberated from this, so please come to a decisive experience of
  causality. As a means of doing so, there is nothing superior to this
  approach.
\end{proof}

\begin{theorem}
  [Final Advice] Therefore, it is crucial not to stray from the enlightened
  intent of the true nature of phenomena. This is the expanse of my profound
  and heartfelt advice. It is crucial to go beyond what everything is or is
  not. Transcending is and is not.
\end{theorem}

\begin{remark}
  [Section Identification] This is the tenth section of the precious treasury
  of the basic space of phenomena, demonstrating that enlightened intent does
  not deviate from the true nature of phenomena.
\end{remark}

\end{document}
