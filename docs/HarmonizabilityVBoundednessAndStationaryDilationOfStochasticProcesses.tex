\documentclass{article}
\usepackage{amsmath,amssymb,amsthm}
\usepackage{mathrsfs}
\usepackage{lineno}

% Set up theorem environments
\newtheorem{theorem}{Theorem}
\newtheorem{lemma}[theorem]{Lemma}
\newtheorem{corollary}[theorem]{Corollary}
\newtheorem{definition}{Definition}
\newtheorem{remark}{Remark}

% For equation numbering
\linenumbers

\title{Harmonizability, V-Boundedness and Stationary Dilation of Stochastic Processes}
\author{A. G. MIAMEE \& H. SALEHI}
\date{January-February, 1978}

\begin{document}

\maketitle

The main purpose of this paper is to show that a stochastic process $X_{t}, t$ in the real line $R$, with values in a given complex Hilbert space $\mathscr{H}$ is harmonizable if and only if it is the projection of some stationary process taking values in a larger Hilbert space $\mathscr{K}$. Harmonizable processes are natural generalizations of stationary processes where the inner product $(X_{t}, X_{s})$ is continuous and depends only on $t-s$. It is well known that any stationary process $X_{t}, t \in R$, can be represented in the form
\begin{equation}
X_{t}=\int_{R} e^{-i t u} d \Phi(u)
\end{equation}
where $\Phi$ is a countably additive orthogonally scattered $\mathscr{H}$-valued measure on $R$ such that the spectral measure $F$ defined by
\begin{equation}
F(A \cap B)=(\Phi(A), \Phi(B)), A \quad \text{ and } \quad B \quad \text{ Borel sets}
\end{equation}
is a bounded countably additive nonnegative measure (for the definition of orthogonally scattered measures, see \cite{masani}). In this case the correlation function $R(t, s)=(X_{t}, X_{s})$ has the representation
\begin{equation}
R(t, s)=\int_{R} e^{-i(t-s) u} d F(u)
\end{equation}

The notion of harmonizable processes was introduced by M. Loève \cite{loeve} as those processes $X_{t}$ for which
\begin{equation}
X_{t}=\int_{R} e^{-i t u} d \Phi(u)
\end{equation}
and
\begin{equation}
R(t, s)=\iint_{R^{2}} e^{-i(t v-s u)} d F(u, v)
\end{equation}
with $\Phi$ being a countably additive $\mathscr{H}$-valued measure (not necessarily orthogonally scattered) and the spectral measure $F$ being a complex-valued measure of
bounded variation on the plane. We note that for a stationary process the spectral measure $F$ can be realized as a measure on the plane which is concentrated on its diagonal. But for a harmonizable process this is not necessarily the case.

Subsequently H. Cramér \cite{cramer} in his study of non-stationary processes obtained several important results concerning harmonizable sequences. Later J. L. Abreu \cite{abreu} considered these harmonizable processes and showed that they are projections of stationary processes taking values in some larger Hilbert space. With this setting the class of harmonizable processes fails to include projection of all stationary processes, a fact which was observed by Abreu in his work. A weaker definition of harmonizability was introduced by Yu. A. Rozanov in \cite{rozanov}. More precisely, he defined an $\mathscr{H}$-valued process $X_{t}$ to be harmonizable if it can be represented as
\begin{equation}
X_{t}=\int_{R} e^{-i t u} d \Phi(u)
\end{equation}
where $\Phi$ is a countably additive measure with values in $\mathscr{H}$, which is "strongly continuous" and of bounded semi-variation. The term "strong continuity" used by Rozanov means that the semi-variation $\|\Phi\|(\cdot)$ is continuous from above at zero (for definition of the semi-variation $\|\Phi\|$ see \cite{dunford} and \cite{rozanov}). In view of IV. 10.5 \cite{dunford} the "strong continuity" of $\Phi$ assumed by Rozanov is automatically true. Hence we adopt the following definition for harmonizability:

\begin{definition}
An $\mathscr{H}$-valued process $X_{t}, t \in R$, is said to be harmonizable if $X_{t}$ can be represented as
\begin{equation}
X_{t}=\int_{R} e^{-i t u} d \Phi(u)
\end{equation}
with $\Phi$ being a countably additive $\mathscr{H}$-valued measure on $R$ which is of bounded semi-variation.
\end{definition}

It is clear, as Rozanov pointed out in \cite{rozanov}, that the class of harmonizable processes contains projections of stationary processes. The main purpose of this article is to show that in fact these two class of processes are precisely the same (Theorem 5). In other words our aim is to prove that an $\mathscr{H}$-valued process is harmonizable if and only if it has a stationary dilation, i.e., it is the projection of some stationary process taking values in some larger Hilbert space $\mathscr{K}$.

Recently H. Niemi in a sequence of articles \cite{niemi1}, \cite{niemi2}, \cite{niemi3} studied bounded vector measures and their dilation as projection of orthogonally scattered measures. Using these results he characterized \cite{niemi4} continuous $V$-bounded processes, and showed that they are the same as projections of stationary processes. Combining this and our main result mentioned above, one concludes that the following are equivalent:
(i) $X_{t}, t \in R$, is a harmonizable process,
(ii) $X_{t}, t \in R$, is projection of some stationary process,
(iii) $X_{t}, t \in R$, is a continuous $V$-bounded process.

However, we will obtain a direct proof for the equivalence of (i) and (iii), (see also \cite{kluvanek}), and hence providing a considerably shorter proof for Niemi's result. We should also point out that in Niemi's work a bounded vector measure means a bounded linear operator on $C_{0}(R)$ into a Hilbert space, whereas in our study we work with the usual vector valued measures, which naturally appear in the study of spectral theory of stochastic processes.

In the course of our proof we will make extensive use of the ideas and the techniques occurring in the works of R. Rogge \cite{rogge} and A. Pietsch \cite{pietsch} concerning 2 majorizable operators on a Hilbert space. In this connection see also \cite{dubinsky} and \cite{grothendieck}.

In this paper we use the concept of integration with respect to a finitely additive measure $F$ as introduced by Rozanov in \cite{rozanov}. The reader may find it convenient to be familiar with this paper. We will use the following characterization of harmonizable processes which is a slight modification of Theorem 1.2, \cite{rozanov}.

\begin{theorem}
In order that an $\mathscr{H}$-valued process $X_{t}, t \in R$, be harmonizable, it is necessary and sufficient that the correlation function $R(t, s)$ can be represented in the form
\begin{equation}
R(t, s)=\iint_{R^{2}} e^{-i(t v-s u)} d F(u, v)
\end{equation}
where $F(A, B)=(\Phi(A), \Phi(B)), A$ and B Borel sets; $\Phi$ as in Definition 1, and the spectral measure $F$ has the following properties:
\begin{enumerate}
\item[(a)] $F$ is finitely additive on the algebra of rectangles,
\item[(b)] $F$ is continuous from above at zero, in the sense that $F(A_{i}, B_{j})$ tends to zero if $A_{i}$ and $B_{j}$ converge monotonically to the empty set.
\item[(c)] $F$ is positive definite, i.e., for any set of complex numbers $\alpha_{1}, \alpha_{2}, \cdots \alpha_{n}$,
\begin{equation}
\sum_{i, j=1}^{n} a_{i} \bar{\alpha}_{j} F(A_{i}, A_{j}) \geq 0
\end{equation}
\item[(d)] $F$ is of bounded semi-variation.
\end{enumerate}
\end{theorem}

In his paper \cite{rozanov} Rozanov, in addition, assumed that $F$ is "strongly continuous'', i.e., the semi-variation of $F$ is continuous from above at zero. It is not hard to show that properties $(a)-(d)$ imply this "strong continuity."

In order to establish our main theorem we first prove two lemmas. In the proof of these lemmas we have borrowed the techniques contained in \cite{pietsch} and \cite{rogge}, and we are happy to acknowledge it here. For the sake of completeness and the benefit of readers we summarize some preliminary results contained in \cite{rogge}. Let $R^{n}(n \geq 2)$ be the $n$-dimensional real Euclidean space with the usual norm and inner product $\|\cdot\|,(\cdot, \cdot)$; $A_{n}$ the unit sphere of $R^{n}$, and $m_{n}$ the normalized Haar measure on $A_{n}$. By elementary integration one can obtain:
\begin{equation}
\left\{\begin{array}{l}
\int_{A_{n}}(r, t)(s, t) d m_{n}(t)=\frac{1}{n}(r, s) \\
\|r\| \int_{A_{n}} \operatorname{sign}[(r, t)](s, t) d m_{n}(t)=\rho_{n}(r, s) \\
\int_{A_{n}}|(r, t)| d m_{n}(t)=\rho_{n}\|r\|
\end{array}\right.
\end{equation}
where $\rho_{n}=\frac{\Gamma\left(\frac{n}{2}\right)}{\sqrt{\pi} \Gamma\left(\frac{n+1}{2}\right)}$. Moreover, one can see that $n \rho_{n}^{2}>\frac{2}{\pi}$.

Let $L_{2}(A_{n}, m_{n})$ be the usual Hilbert space of real-valued square summable functions on $A_{n}$. We define two operators $P_{n}$ and $Q_{n}$ on $L_{2}(A_{n}, m_{n})$ as follows:
\begin{equation}
\begin{gathered}
(P_{n} f)(s)=n \int_{A_{n}}(s, t) f(t) d m_{n}(t) \\
(Q_{n} f)(s)=\int_{A_{n}} \operatorname{sign}[(s, t)] f(t) d m_{n}(t)
\end{gathered}
\end{equation}
$P_{n}$ and $Q_{n}$ are linear operators. As shown in \cite{rogge}, using relations (3), one can verify that $\frac{\pi}{2} Q_{n}^{2}-\frac{1}{n} P_{n}$ is a non-negative Hermitian operator in $L_{2}(A_{n}, m_{n})$, i.e.,
\begin{equation}
\frac{\pi}{2}(Q_{n}^{2} f, f) \geq \frac{1}{n}(P_{n} f, f), \quad \text{ for all } f \in L_{2}(A_{n}, m_{n})
\end{equation}

Define the kernel $L(\cdot, \cdot)$ by
\begin{equation}
L(r, s)=\int_{A_{n}} \operatorname{sign}[(r, t)] \quad \operatorname{sign}[(s, t)] d m_{n}(t)
\end{equation}

Simple computation on (4) shows that for all $f \in L_{2}(A_{n}, m_{n})$,
\begin{equation}
\begin{aligned}
& \int_{A_{n}} \int_{A_{n}}(r, s) f(r) f(s) d m_{n}(r) d m_{n}(s) \\
\leq & \frac{\pi}{2} \int_{A_{n}} \int_{A_{n}} L(r, s) f(r) f(s) d m_{n}(r) d m_{n}(s)
\end{aligned}
\end{equation}

Let $\lambda_{1}, \lambda_{2} \cdots, \lambda_{n}$ be any set of real numbers and $t_{1}, t_{2}, \cdots, t_{n}$ be any set of vectors in $R^{n}$. Apply (5) to the function $f(r)=\sum_{i=1}^{n} \lambda_{i}(r, t_{i})$. Using relations (3) and the definition of the kernel $L(\cdot, \cdot)$, some routine calculations show that
\begin{equation}
\sum_{i=1}^{m} \sum_{j=1}^{m}(t_{i}, t_{j}) \lambda_{i} \lambda_{j} \leq \frac{\pi}{2} \sum_{i=1}^{m} \sum_{j=1}^{m} L(t_{i}, t_{j})\|t_{i}\|\|t_{j}\| \lambda_{i} \lambda_{j}.
\end{equation}

Now since the matrix $[\frac{\pi}{2} L(t_{i}, t_{j})\|t_{i}\|\|t_{i}\|-(t_{i}, t_{j})]_{i, j=1}^{n}$ is symmetric with real entries, (6) implies the corresponding relation for the complex numbers $\lambda_{1} \lambda_{2}, \cdots, \lambda_{n}$, i.e., we have
\begin{equation}
\sum_{i=1}^{m} \sum_{j=1}^{m}(t_{i}, t_{j}) \lambda_{i} \bar{\lambda}_{j} \leq \frac{\pi}{2} \sum_{i=1}^{m} \sum_{j=1}^{m} L(t_{i}, t_{j})\|t_{i}\|\|t_{j}\| \lambda_{i} \bar{\lambda}_{j},
\end{equation}
for any set $\lambda_{1}, \lambda_{2}, \cdots, \lambda_{n}$ of complex numbers and any set $t_{1}, t_{2}, \cdots, t_{n}$ of vectors in $R^{n}$.

\begin{lemma}
Let $X_{t}, t \in R$, be an $\mathscr{H}$-valued harmonizable process with the spectral measure F as in Theorem 2, then there exists a positive number $c$ such that
\begin{enumerate}
\item[(a)]
\begin{equation}
\left|\iint_{R^{2}} \varphi(s) \psi(t) d F(s, t)\right| \leq c\|\varphi\|_{\infty}\|\psi\|_{\infty},
\end{equation}
for any pair $\varphi$ and $\psi$ of bounded measurable complex-valued functions on $R$, and
\item[(b)]
\begin{equation}
\sum_{k=1}^{n} \iint_{R^{2}} \varphi_{k}(s) \bar{\varphi}_{k}(t) d F(s, t) \leq \pi c \sup_{t \in R} \sum_{k=1}^{n} |\varphi_{k}(t)|^2,
\end{equation}
for any set $\{\varphi_1, \varphi_2, \cdots, \varphi_n\}$ of bounded measurable complex-valued functions on $R$.

\begin{proof}
(a) This follows from \cite{rozanov} with $c = \|F\|(R)$. (b) First we assume that the function $\varphi_1, \varphi_2, \cdots, \varphi_n$ are real-valued. Without loss of generality we may assume that $c$ in part (a) is one, and hence assuming $\|\varphi_k\|_{\infty}^2 = 1$, we show that
\begin{equation}
\sum_{k=1}^{n} \iint_{R^{2}} \varphi_{k}(s) \varphi_{k}(t) d F(s, t) \leq \pi.
\end{equation}

It suffices to assume that $\varphi_1, \varphi_2, \cdots, \varphi_n$ are simple functions (the general case then follows from approximating the $\varphi$'s by simple functions). We may write $\varphi_k = \sum_{i=1}^{m} \varphi_k(t_i) \chi_{U_i}$, $k = 1, 2, \ldots, n$; where $U_1, U_2, \cdots, U_m$ is a disjoint covering of $R$ and $t_i \in U_i$ for each $i$. We have
\begin{equation}
\sum_{k=1}^{n} \iint_{R^{2}} \varphi_{k}(s) \varphi_{k}(t) d F(s, t) = \sum_{k=1}^{n} \sum_{i=1}^{m} \sum_{j=1}^{m} \varphi_k(t_i) \varphi_k(t_j) F(U_i, U_j).
\end{equation}

Now for any set of complex numbers $\{\gamma_i\}_{i=1}^{m}$ with $\sum_{i=1}^{m} |\gamma_i|^2 = 1$, we have
\begin{equation}
\left| \iint_{R^{2}} \left(\sum_{i=1}^{m} \gamma_i \chi_{U_i}(s)\right) \left(\sum_{j=1}^{m} \gamma_j \chi_{U_j}(t)\right) d F(s, t) \right| \leq 1,
\end{equation}
(because $\|\sum_{i=1}^{m} \gamma_i \chi_{U_i}\|_{\infty} \leq 1$, and (a) applies with $\varphi = \psi = \sum_{i=1}^{m} \gamma_i \chi_{U_i}$). Define the operator $S : R \to R^n$ by
\begin{equation}
S(t) = (\varphi_1(t), \varphi_2(t), \cdots, \varphi_n(t)).
\end{equation}

For any $s, t \in R$ we have
\begin{equation}
(S(s), S(t)) = \sum_{k=1}^{n} \varphi_k(s) \varphi_k(t) \quad \text{and} \quad \|S(t)\| \leq 1.
\end{equation}

Now from (9) we can write
\begin{align}
\sum_{k=1}^{n} \iint_{R^{2}} \varphi_{k}(s) \varphi_{k}(t) d F(s, t) &= \sum_{i=1}^{m} \sum_{j=1}^{m} (S(t_i), S(t_j)) F(U_i, U_j) \\
&= \sum_{i=1}^{m} \sum_{j=1}^{m} \sum_{\beta} (\Phi(U_i), e_\beta) (\Phi(U_j), e_\beta) (S(t_i), S(t_j)),
\end{align}
where $\{e_\beta\}$ is an orthonormal system for the Hilbert space $\mathscr{H}$, and $\Phi$ is as in Theorem 2. Now bringing the sum over $\beta$ out and using (7) we get
\begin{align}
\sum_{k=1}^{n} \iint_{R^{2}} \varphi_{k}(s) \varphi_{k}(t) d F(s, t) &\leq \frac{\pi}{2} \sum_{i=1}^{m} \sum_{j=1}^{m} \sum_{\beta} L(S(t_i), S(t_j)) \|S(t_i)\| \|S(t_j)\| (\Phi(U_i), e_\beta) (\Phi(U_j), e_\beta) \\
&= \frac{\pi}{2} \sum_{i=1}^{m} \sum_{j=1}^{m} L(S(t_i), S(t_j)) \|S(t_i)\| \|S(t_j)\| (\Phi(U_i), \Phi(U_j)) \\
&= \frac{\pi}{2} \sum_{i=1}^{m} \sum_{j=1}^{m} L(S(t_i), S(t_j)) \|S(t_i)\| \|S(t_j)\| F(U_i, U_j) \\
&= \frac{\pi}{2} \int_{A_n} \sum_{i=1}^{m} \sum_{j=1}^{m} F(U_i, U_j) \text{sign}[S(t_i), p] \text{sign}[S(t_j), p] \|S(t_i)\| \|S(t_j)\| dm_n(p).
\end{align}

Using (10) with $\gamma_i = \text{sign}[S(t_i), p] \|S(t_i)\|$ we get
\begin{align}
\sum_{k=1}^{n} \iint_{R^{2}} \varphi_{k}(s) \varphi_{k}(t) d F(s, t) &\leq \frac{\pi}{2} \int_{A_n} \left| \iint_{R^{2}} \left(\sum_{i=1}^{m} \text{sign}[S(t_i), p] \|S(t_i)\| \chi_{U_i}(s)\right) \right. \\
&\quad \left. \left(\sum_{j=1}^{m} \text{sign}[S(t_j), p] \|S(t_j)\| \chi_{U_j}(t)\right) dF(s, t) \right| dm_n(p) \\
&\leq \frac{\pi}{2} \int_{A_n} dm_n(p) = \frac{\pi}{2} \cdot 2 = \pi.
\end{align}

Thus for any arbitrary real-valued bounded measurable functions $\varphi_1, \varphi_2, \cdots, \varphi_n$ we have
\begin{equation}
\sum_{k=1}^{n} \iint_{R^{2}} \varphi_{k}(s) \varphi_{k}(t) d F(s, t) \leq \pi c \sup_{t \in R} \sum_{k=1}^{n} \varphi_k(t)^2.
\end{equation}

This completes the proof of (b) for the real-valued $\varphi$'s. Now let $\varphi_1, \varphi_2, \cdots, \varphi_n$ be complex-valued bounded measurable functions. Writing $\varphi_k(t) = \alpha_k(t) + i\beta_k(t)$, $k = 1, 2, \ldots, n$; we have
\begin{align}
0 \leq \sum_{k=1}^{n} \iint_{R^{2}} \varphi_{k}(s) \bar{\varphi}_{k}(t) d F(s, t) &= \sum_{k=1}^{n} \iint_{R^{2}} (\alpha_k(s) + i\beta_k(s)) (\alpha_k(t) - i\beta_k(t)) d F(s, t) \\
&= \sum_{k=1}^{n} \iint_{R^{2}} \alpha_k(s) \alpha_k(t) d F(s, t) + \sum_{k=1}^{n} \iint_{R^{2}} \beta_k(s) \beta_k(t) d F(s, t) \\
&\quad + i \sum_{k=1}^{n} \iint_{R^{2}} (\beta_k(s) \alpha_k(t) - \alpha_k(s) \beta_k(t)) d F(s, t).
\end{align}

Clearly the third term is zero. By (11) each of the first two terms are less than or equal to $\frac{\pi c}{2} \sup_{t \in R} \sum_{k=1}^{n} (|\alpha_k(t)|^2 + |\beta_k(t)|^2) = \frac{\pi c}{2} \sup_{t \in R} \sum_{k=1}^{n} |\varphi_k(t)|^2$. These complete the proof of part (b).
\end{proof}

\begin{lemma}
Let $X_t$ be a harmonizable process with the spectral measure $F$, then there exists a bounded regular countably additive nonnegative measure $\mu$ on $R$ such that
\begin{equation}
\left| \iint_{R^{2}} \varphi(s) \psi(t) d F(s, t) \right| \leq \left( \int_{R} |\varphi(t)|^2 d\mu(t) \right)^{1/2} \left( \int_{R} |\psi(t)|^2 d\mu(t) \right)^{1/2},
\end{equation}
for any bounded measurable complex-valued function $\varphi$ on $R$.
\end{lemma}

\begin{proof}
By Lemma 3 there exists a positive number $\rho$ such that
\begin{equation}
\sum_{k=1}^{n} \iint_{R^{2}} \varphi_{k}(s) \bar{\varphi}_{k}(t) d F(s, t) \leq \rho \sup_{t \in R} \sum_{k=1}^{n} |\varphi_k(t)|^2,
\end{equation}
for any set $\{\varphi_1, \varphi_2, \cdots, \varphi_n\}$ of bounded measurable complex-valued functions. Define a functional $S$ on the class $C_R(R)$ of all bounded continuous real-valued functions by
\begin{equation}
S(\psi) = \inf \left\{ \sup_{t \in R} \sum_{k=1}^{n} |\varphi_k(t)|^2 + \rho \sum_{k=1}^{n} |\psi(t) - \varphi_k(t)|^2 - \sum_{k=1}^{n} \iint_{R^{2}} \varphi_{k}(s) \bar{\varphi}_{k}(t) d F(s, t) \right\},
\end{equation}
where the infinum is taken over all finite sets $\{\varphi_1, \varphi_2, \cdots, \varphi_n\}$ of bounded measurable complex-valued functions. It is easy to see that $S$ is a homogeneous subadditive real-valued functional for which
\begin{equation}
\inf_{t \in R} \psi(t) \leq S(\psi) \leq \sup_{t \in R} \psi(t), \text{ for all } \psi \in C_R(R)
\end{equation}
and hence
\begin{align}
S(\psi) \geq 0 \text{ if } \psi \geq 0 \\
S(\psi) \leq 0 \text{ if } \psi \leq 0
\end{align}

Thus, by the Hahn-Banach theorem there exists a linear functional $T$ on $C_R(R)$ such that (see for example \cite{dunford} pp. 62-63),
\begin{equation}
S(-\psi) \leq T(\psi) \leq S(\psi), \psi \in C_R(R).
\end{equation}

Obviously, $T$ is a bounded nonnegative linear functional on the normed vector space $C_R(R)$. Let $C_C(R)$ be the class of all bounded continuous complex-valued functions on $R$. We extend $T$ to $C_C(R)$ by
\begin{equation}
\hat{T}(\varphi + i\psi) = T(\varphi) + iT(\psi); \varphi, \psi \in C_R(R).
\end{equation}

Clearly $\hat{T}$ is a bounded nonnegative linear functional on $C_C(R)$. Since $C_C(R)$ contains $C_0(R)$, the class of all continuous complex-valued functions vanishing at $\infty$, according to the Riesz representation theorem (\cite{rudin}, p. 131, Theorem 6.19) there exists a complex regular measure $\mu_0$ such that
\begin{equation}
T(\psi) = \int_{R} \psi(t) d\mu_0(t), \psi \in C_C(R).
\end{equation}

Let $E$ be any measurable set. Given any $\epsilon > 0$, one can find a nonnegative function $\psi$ in $C_C(R)$ such that $\int_{E} (1 - \psi(t)) d\mu_0(t) \leq \epsilon$. This implies that $\mu_0(E) = \int_{E} 1 d\mu_0(t)$ is nonnegative. Hence $\mu_0$ is a bounded regular countably additive nonnegative measure such that
\begin{equation}
T(\psi) = \int_{R} \psi(t) d\mu_0(t), \psi \in C_R(R).
\end{equation}

For any bounded continuous complex-valued function $\psi$ we have
\begin{equation}
S(-\rho|\psi|^2) \leq T(-\rho|\psi|^2),
\end{equation}
or
\begin{equation}
S(-\rho|\psi|^2) \leq \sup_{t \in R} \{|\psi(t)|^2 + \rho|\varphi(t) - \psi(t)|^2 - \iint_{R^{2}} \varphi(s) \bar{\varphi}(t) d F(s, t)\},
\end{equation}
where the supremum is taken over all bounded continuous complex-valued functions $\varphi$.

Thus for the specific choice $\varphi = \psi$ we get
\begin{equation}
S(-\rho|\psi|^2) \leq \sup_{t \in R} |\psi(t)|^2 - \iint_{R^{2}} \psi(s) \bar{\psi}(t) d F(s, t),
\end{equation}
or
\begin{equation}
-\rho T(|\psi|^2) = T(-\rho|\psi|^2) \leq S(-\rho|\psi|^2) \leq \iint_{R^{2}} \psi(s) \bar{\psi}(t) d F(s, t).
\end{equation}

Thus for any function $\psi$ in $C_C(R)$ we have
\begin{equation}
\iint_{R^{2}} \psi(s) \bar{\psi}(t) d F(s, t) \leq \rho T(|\psi|^2) = \rho \int_{R} |\psi(t)|^2 d\mu_0(t),
\end{equation}
or equivalently
\begin{equation}
\left( \int_{R} \psi(s) d\Phi(s), \int_{R} \psi(s) d\Phi(s) \right) \leq \rho \int_{R} |\psi(t)|^2 d\mu_0(t).
\end{equation}

Let $\psi$ be in $C_C(R)$. Take a sequence $\psi_n$ in $C_C(R)$ such that $\psi_n \to \psi$ pointwise and $|\psi_n(s)| \leq \sup_{t \in R} |\psi(t)|$. Since $\mu_0$ is bounded by the usual Lebesgue dominated convergence theorem $\int_{R} |\psi_n(t)|^2 d\mu_0(t)$ converges to $\int_{R} |\psi(t)|^2 d\mu_0(t)$. Also by (\cite{dunford}, p. 328, Theorem 10) $\int_{R} \psi_n(t) d\Phi(t)$ converges to $\int_{R} \psi(t) d\Phi(t)$. Therefore applying (12) to the sequence $\psi_n$ and then taking limits one obtains
\begin{equation}
\iint_{R^{2}} \psi(s) \bar{\psi}(t) d F(s, t) = \left( \int_{R} \psi(s) d\Phi(s), \int_{R} \psi(s) d\Phi(s) \right) \leq \rho \int_{R} |\psi(t)|^2 d\mu_0(t), \psi \in C_C(R).
\end{equation}

Hence $\mu = \rho \mu_0$ is the desired measure.
\end{proof}

\begin{theorem}[Main theorem]
An $\mathscr{H}$-valued process $X_t, t \in R$, is harmonizable if and only if there exists a Hilbert space $\mathscr{K}$ containing $\mathscr{H}$ and a $\mathscr{K}$-valued stationary process $Y_t$ such that
\begin{equation}
X_t = PY_t,
\end{equation}
where $P$ is the projection from $\mathscr{K}$ onto $\mathscr{H}$.
\end{theorem}

\begin{proof}
If $X$ is the projection of some stationary process $Y$, it is easy, as was pointed out in \cite{rozanov}, to see that $X_t$ is harmonizable. Now let $X_t, t \in R$, be a harmonizable process with the spectral measure $F$. Let $\mu$ be the dominating measure of Lemma 4. There exists a countably additive Borel measure $\mu_1$ on $R \times R$ which is concentrated on its diagonal and satisfies
\begin{equation}
\int_{R^2} f(s, t) d\mu_1(s, t) = \int_{R} f(t, t) d\mu(t),
\end{equation}
for every bounded continuous complex-valued function $f$ on $R \times R$. Now since by Lemma 4
\begin{equation}
\int_{R^2} \varphi(s) \bar{\psi}(t) d\mu_1(s, t) - \int_{R^2} \varphi(s) \bar{\psi}(t) dF(s, t) \geq 0,
\end{equation}
we define an inner product $(\cdot, \cdot)'$ on the class $C_C(R)$ by
\begin{equation}
(\varphi, \psi)' = \int_{R^2} \varphi(s) \bar{\psi}(t) d\mu_1(s, t) - \int_{R^2} \varphi(s) \bar{\psi}(t) dF(s, t).
\end{equation}
Let $\mathscr{H}'$ be the Hilbert space obtained from completing $C_C(R)/\{f \in C_C(R)|(f, f)' = 0\}$ with respect to the norm $\|f\|' = \sqrt{(f, f)'}$, and $Z_t$ be the image of the functional $u \to e^{-iut}$ under the canonical mapping $C_C(R) \to \mathscr{H}'$. Now let $\mathscr{K} = \mathscr{H} \oplus \mathscr{H}'$, identifying $\mathscr{H}$ with the subspace $\mathscr{H}\oplus \{0\}$ of $\mathscr{K}$. Let $Y_t = X_t + Z_t$, $t\in R$. Obviously, $X_t = PY_t$ and
\begin{equation}
(Y_s, Y_t)_{\mathscr{K}} = (X_s, X_t)_{\mathscr{H}} + (Z_s, Z_t)', \text{ for all } s, t \in R.
\end{equation}

Hence
\begin{align}
(Y_t, Y_s)_{\mathscr{K}} &= \iint_{R^2} e^{-i(tv-su)} dF(u, v) + \iint_{R^2} e^{-i(tv-su)} d\mu_1(u, v) - \iint_{R^2} e^{-i(tv-su)} dF(u, v)\\
\end{align}
which reduces to
\begin{align}
(Y_t, Y_s)_{\mathscr{K}} &= \iint_{R^2} e^{-i(tv-su)} d\mu_1(u, v)\\
&= \int_R e^{-iu(t-s)} d\mu(u)
\end{align}
which in turn means that $Y_t$ is stationary and completes the proof.
\end{proof}

The idea of constructing the Hilbert space $\mathscr{K}$ and the stationary process $Y$ was originated by Abreu in \cite{abreu}. In his case the measure $F$ was of bounded variation and this fact played an important part in establishing the dominating measure. However, in our case $F$ is a finitely additive measure which is only of bounded semi-variation and his line of proof will not induce the desired dominating measure. Nevertheless, we did achieve our goal in establishing the existence of this crucial measure via our lemmas.

As a consequence of our main theorem we obtain the following result which relates countably additive vector valued measures to projections of orthogonally scattered measures (see also \cite{niemi2}).

\begin{corollary}
A countably additive $\mathscr{H}$-valued measure $\Phi$ is a projection of an orthogonally scattered $\mathscr{K}$-valued measure $E$ if and only if $\Phi$ is of bounded semi-variation.
\end{corollary}

\begin{proof}
It is clear that if $E$ is an orthogonally scattered $\mathscr{K}$-valued measure and $P$ is the projection of the Hilbert space $\mathscr{K}$ onto $\mathscr{H} \subset \mathscr{K}$, then $\Phi = PE$ is a countably additive $\mathscr{H}$-valued measure which is of bounded semi-variation. Now suppose that $\Phi$ is a countably additive $\mathscr{H}$-valued measure which is of bounded semi-variation. Then the process $X_t$ defined by
\begin{equation}
X_t = \int_R e^{-itu} d\Phi(u),
\end{equation}
is harmonizable. Now by Theorem 5 there exists a Hilbert space $\mathscr{K}$ containing $\mathscr{H}$ and a stationary process $Y_t, t \in R$, such that $X_t = PY_t, t \in R$. Let $E$ be the well-known corresponding orthogonally scattered $\mathscr{K}$-valued measure. We have
\begin{equation}
X_t = PY_t = P\left(\int_R e^{-itu} dE(u)\right) = \int_R e^{-itu} PE(u).
\end{equation}

Thus
\begin{equation}
\int_R e^{-itu} d\Phi(u) = \int_R e^{-itu} PE(u), \text{ for all } t \in R.
\end{equation}
Hence for any $h \in \mathscr{H}$ and any real number $t$, we have
\begin{equation}
\int_R e^{-itu} d(\Phi(u), h)_{\mathscr{H}} = \left(\int_R e^{-itu} d\Phi(u), h\right)_{\mathscr{H}} = \left(\int_R e^{-itu} PE(u), h\right)_{\mathscr{H}} = \int_R e^{-itu} d(PE(u), h)_{\mathscr{H}}
\end{equation}
which implies
\begin{equation}
\Phi = PE.
\end{equation}
\end{proof}

Our next theorem relates harmonizable processes with the continuous $V$-bounded processes which were originally studied by S. Bochner in \cite{bochner}. We first state the definition of continuous $V$-bounded processes as given in \cite{niemi4}.

\begin{definition}
A continuous $\mathscr{H}$-valued process $X_t, t \in R$, is $V$-bounded if it is bounded and if there exists a constant $C$ such that
\begin{equation}
\iint_{R^2} R(t,s)\hat{\varphi}(s)\hat{\psi}(t)ds dt \leq C \|\varphi\|_1\|\psi\|_1,
\end{equation}
where $\varphi$ and $\psi$ are functions in $L_1(R)$ with the Fourier transforms $\hat{\varphi}$ and $\hat{\psi}$ respectively, and $R(\cdot, \cdot)$ is the correlation of $X_t, t \in R$.
\end{definition}

The proof of the following lemma is clear.

\begin{lemma}
Let $X_t$ be a harmonizable process, i.e., let
\begin{equation}
X_t = \int_R e^{-itu} d\Phi(u),
\end{equation}
where $\Phi$ as in (1). Then we have
\begin{equation}
\int_R \hat{f}(t)X_t dt = \int_R f(t)d\Phi(t), \forall f \in L_1(R).
\end{equation}
\end{lemma}

\begin{theorem}
An $\mathscr{H}$-valued process $X_t, t \in R$, is harmonizable if and only if it is a continuous $V$-bounded process.
\end{theorem}

\begin{proof}
Suppose that $X_t$ is harmonizable, i.e., suppose that
\begin{equation}
X_t = \int_R e^{-itu} d\Phi(u)
\end{equation}
and
\begin{equation}
R(t, s) = (X_t, X_s)_{\mathscr{H}} = \iint_{R^2} e^{-i(tv-su)} dF(u, v),
\end{equation}
where $\Phi$ and $F$ are as in Theorem 2. It is clear by Theorem 5 that $X_t$ is bounded and continuous. Now using Lemma 8 we note that for any two functions $\varphi$ and $\psi$ in $L_1(R)$ we have
\begin{align}
\iint_{R^2} \hat{\varphi}(u)\hat{\psi}(v)R(u, v)du dv &= \left(\int_R \hat{\varphi}(u)X_u du, \int_R \hat{\psi}(v)X_v dv\right)_{\mathscr{H}}\\
&= \left(\int_R \varphi(u)d\Phi(u), \int_R \psi(v)d\Phi(v)\right)_{\mathscr{H}}\\
&\leq C\|\varphi\|_1\|\psi\|_1, \quad C = \|F\|(R).
\end{align}

Conversely, suppose that $X_t$ is a continuous $V$-bounded process with the correlation function $R(s, t)$. Then besides being bounded and continuous, we have
\begin{equation}
\left|\iint_{R^2} \hat{\varphi}(u)\hat{\psi}(v)R(u, v)du dv\right| \leq C \|\varphi\|_1\|\psi\|_1, \quad \forall \varphi, \psi \in L_1(R).
\end{equation}

Define the bilinear form $K$ on $L_1(R)$ by
\begin{equation}
K(\varphi, \psi) = \iint_{R^2} \hat{\varphi}(u)\hat{\psi}(v)R(u, v)du dv.
\end{equation}

By (13) we have
\begin{equation}
|K(\varphi, \psi)| \leq C \|\varphi\|_1\|\psi\|_1
\end{equation}
$K(\cdot,\cdot)$ is a positive definite kernel. Consider the corresponding reproducing kernel Hilbert space $\mathscr{F}$. We can define the operator $T : L_1(R) \to \mathscr{F}$ by
\begin{equation}
T(\varphi) = K(\varphi, \cdot).
\end{equation}
From the properties of reproducing kernel Hilbert spaces we have
\begin{equation}
K(\varphi, \psi) = (K(\varphi, \cdot), K(\psi, \cdot))_{\mathscr{F}} = (T(\varphi), T(\psi))_{\mathscr{F}}
\end{equation}
which along with (14) implies that $T$ is a bounded operator on $L_1(R)$. Since $L_1(R)$ is dense in the class $C_0(R)$ (see \cite{rudin}, p. 195) one can extend $T$ to a bounded operator on $C_0(R)$ into $\mathscr{F}$. But by Lemma 2 of \cite{kluvanek} and its corollaries, there exists an $\mathscr{F}$-valued regular measure $M$ on $R$ which is of bounded semi-variation and satisfies
\begin{equation}
T(\varphi) = \int_R \varphi(t)dM(t), \text{ for all } \varphi \in C_0(R).
\end{equation}

Define the set function $F$ for each pair $A$ and $B$ of Borel sets by
\begin{equation}
F(A, B) = (M(A), M(B))_{\mathscr{F}}
\end{equation}
and then define $R_1$ by
\begin{equation}
R_1(t, s) = \iint_{R^2} e^{-i(tv-su)} dF(u, v).
\end{equation}

Using Lemma 8 and Theorem 2 that for any $\varphi$ and $\psi$ in $L_1(R)$ we have
\begin{equation}
\iint_{R^2} \hat{\varphi}(u)\hat{\psi}(v)R_1(u, v)du dv = \iint_{R^2} \varphi(u)\psi(v)dF(u, v).
\end{equation}

On the other hand, we have
\begin{equation}
\iint_{R^2} \hat{\varphi}(u)\hat{\psi}(v)R(u, v)du dv = \iint_{R^2} \varphi(u)\psi(v)dF(u, v),
\end{equation}
for all pairs $\varphi$ and $\psi$ of functions in $L_1(R)$. Combining (15) and (16) we get
\begin{equation}
\iint_{R^2} \hat{\varphi}(u)\hat{\psi}(v)[R(u, v) - R_1(u, v)]du dv = 0,
\end{equation}
for all $\varphi$ and $\psi$ in $L_1(R)$. Thus, we have
\begin{equation}
R(t, s) = R_1(t, s) = \iint_{R^2} e^{-i(tv-su)} dF(u, v),
\end{equation}
and Theorem 2, now implies that $X_t$ is harmonizable.
\end{proof}

As a corollary to our work we obtain the following ergodic theorem:

\begin{corollary}
If $X$ is an $\mathscr{H}$-valued harmonizable or equivalently continuous $V$-bounded process, then
\begin{equation}
\lim_{T\to\infty} \frac{1}{T} \int_0^T X_t e^{-iu_0 t} dt = \Phi(\{u_0\}),
\end{equation}
where $X_t = \int_R e^{-itu} d\Phi(u)$.
\end{corollary}

\begin{proof}
By our earlier results there exists a Hilbert space $\mathscr{K}$ containing $\mathscr{H}$, a $\mathscr{K}$-valued stationary process $Y_t$ and an orthogonally scattered $\mathscr{K}$-valued measure $E$ such that $X_t = PY_t$, $Y_t = \int_R e^{-iut} dE(u)$, and $\Phi = PE$; where $P$ is the projection of $\mathscr{K}$ onto $\mathscr{H}$. Now using the corresponding well known ergodic theorem for the stationary process $Y_t$ we get
\begin{align}
\lim_{T\to\infty} \frac{1}{T} \int_0^T X_t e^{-iu_0 t} dt &= \lim_{T\to\infty} \frac{1}{T} \int_0^T PY_t e^{-iu_0 t} dt\\
&= P \lim_{T\to\infty} \frac{1}{T} \int_0^T Y_t e^{-iu_0 t} dt\\
&= P E(\{u_0\}) = \Phi(\{u_0\}).
\end{align}
\end{proof}

Such an ergodic theorem for harmonizable processes was obtained in \cite{rozanov}.

\begin{remark}
In this article we only considered the continuous parameter processes, however our main theorem can be established in an analogous way for the discrete parameter case as well.
\end{remark}

\begin{thebibliography}{99}
\bibitem{abreu} J. L. Abreu, A note on harmonizable and stationary sequences, Bol. Soc. Mat. Mexicana 15 (1970), 58-41.
\bibitem{bochner} S. Bochner, Stationarity, boundedness, almost periodicity of random valued functions, Proc. III Berkeley Sym. Math. Stat. Prob., 2, 7-27, Univ. Calif. Press, 1956.
\bibitem{cramer} H. Cramér, On some class of non-stationary stochastic processes, Proc. IV Berkeley Sym. Math. Stat. prob., 2, 57-78, Univ. Calif. Press, 1962.
\bibitem{dubinsky} E. Dubinsky, A. Pełczyński \& H. P. Rosenthal, On Banach spaces X for which $\Pi_2(L_\infty, X) = B(L_\infty, X)$, Studia Math. 44 (1972), 617-648.
\bibitem{dunford} N. Dunford \& J. T. Schwartz, Linear Operators, I, Wiley, New York, 1953.
\bibitem{grothendieck} A. Grothendieck, Résumé de la théorie metrique des produits tensoriels topologiques, Bol. Soc. Matem. Sao Paulo 8 (1956), 1-79.
\bibitem{kluvanek} I. Kluvánek, Characterization of Fourier-Stieltjes transform of vector and operator valued measures, Czechoslovak Math. J. 17 (1967), 261-277.
\bibitem{loeve} M. Loève, Fonctions aléatoires du second ordre, Revue Sci. 84 (1946), 195-206.
\bibitem{masani} P. Masani, Quasi-isometric measures and their applications, Bull. Amer. Math. Soc. 76 (1970), 427-528.
\bibitem{niemi1} H. Niemi, On stationary dilations and the linear prediction of certain stochastic processes, Soc. Sci. Fennica Comment Phys.-Math. 38 (1972), 1-30.
\bibitem{niemi2} H. Niemi, Stochastic operators on a Hilbert space and stochastically continuous semigroups, J. Functional Analysis 13 (1973), 63-76.
\bibitem{niemi3} H. Niemi, On orthogonally scattered dilations of bounded vector measures, Ann. Acad. Sci. Fennicae, Ser. A, I Math. 2 (1976), 97-105.
\bibitem{niemi4} H. Niemi, On the support of a $V$-bounded stochastic measure and the dilation of its covariance measure, Thesis, Turku, 1975.
\bibitem{pietsch} A. Pietsch, Absolut $p$-summierende Abbildungen in normierten Räumen, Studia Math. 28 (1967), 333-353.
\bibitem{rogge} R. Rogge, Über $2$-majorisierende Operatoren, Studia Math. 29 (1967), 41-52.
\bibitem{rozanov} Yu. A. Rozanov, Spectral analysis of abstract functions, Theory Prob. Appl. 4 (1959), 271-287.
\bibitem{rudin} W. Rudin, Real and Complex Analysis, McGraw-Hill, New York, 1966.
\end{thebibliography}
