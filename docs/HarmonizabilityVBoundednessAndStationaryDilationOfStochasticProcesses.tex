\documentclass{article}
\usepackage[english]{babel}
\usepackage{amsmath,amssymb,mathrsfs,latexsym}

%%%%%%%%%% Start TeXmacs macros
\newcommand{\cdummy}{\cdot}
\newenvironment{proof}{\noindent\textbf{Proof\ }}{\hspace*{\fill}$\Box$\medskip}
\newtheorem{lemma}{Lemma}
\newtheorem{theorem}{Theorem}
%%%%%%%%%% End TeXmacs macros

\begin{document}

\title{Harmonizability, V-Boundedness and Stationary Dilation of Stochastic
Processes}

\author{A. G. MIAMEE, H. SALEHI}

\date{January-February, 1978}

\maketitle

\begin{abstract}
  This paper presents a new orthogonal series representation and a new
  orthogonal integral representation for harmonizable stochastic processes. It
  also demonstrates the importance of harmonizable stochastic processes in
  systems analysis by showing that the output of a wide class of systems is a
  harmonizable process.
\end{abstract}

\section{Introduction}\label{sec:intro}

Harmonizable stochastic processes, a generalization of wide sense stationary
processes, have been investigated in connection with a variety of subjects.
Properties related to their integral representation have been studied by
Lo{\`e}ve~{\cite{loeve1963}} and Rosanov~{\cite{rosanov1959}}, their special
role in the multiplicity theory of purely nondeterministic stochastic
processes has been demonstrated by Cramer~{\cite{cramer1964}}, and sampling
theorems have been derived by Piranashvili~{\cite{piranashvili1967}} and
Rao~{\cite{rao1967}}.

This paper makes two contributions related to the class of harmonizable
stochastic processes. First, it is proved under some general conditions that
the output of a linear system is a harmonizable stochastic process; the system
may be randomly time-varying and the input process need not be stationary nor
even harmonizable. Thus, harmonizable processes are the most general processes
that need be considered in the analysis of a wide class of linear systems. In
such analyses as well as in many other problems in communications and
controls, series representation of stochastic processes has been used as a
powerful tool. The second contribution of this paper is a series
representation for harmonizable stochastic processes. A constructive procedure
for obtaining the series representation is given. This representation is valid
over the entire real line, while the well-known Karhunen-Lo{\`e}ve series
representation is valid only on compact intervals. An orthogonal integral
representation for harmonizable stochastic processes is also derived in this
paper.

\section{Notation}\label{sec:notation}

Some definitions and basic properties are briefly mentioned here. For a more
detailed discussion the reader is referred to Karhunen~{\cite{karhunen1947}},
Cramer~{\cite{cramer1951}}, Rosanov~{\cite{rosanov1959}},
Parzen~{\cite{parzen1967}}, and Masani~{\cite{masani1968}}.

Consider a probability space $(\Omega, \mathscr{M}, P)$ and the Hilbert space
$L_2 (\Omega, \mathscr{M}, P)$ of all complex valued random variables with
finite mean square value. A random measure $X$ on a class of subsets
$\mathscr{N}$ of a set $E$ is a countably additive function on $\mathscr{N}$
to $L_2 (\Omega, \mathscr{M}, P)$, i.e., $X (S, \omega) = \sum_{n =
1}^{\infty} X (S_n, \omega)$ in the mean square sense whenever the disjoint
sets $S_n, n = 1, 2, \ldots$, are in $\mathscr{N}$ and $S = \bigcup_{n =
1}^{\infty} S_n \in \mathscr{N}$. To each $X$ on $\mathscr{N}$ there
corresponds a complex measure $r_X$ defined on $\mathscr{N} \times
\mathscr{N}$ by
\begin{equation}
  r_X  (S_1 \times S_2) = \mathscr{E} [X (S_1, \omega) X^{\ast} (S_2,
  \omega)], \label{eq:rX}
\end{equation}
where $\mathscr{E}$ is the expectation operator and the star denotes complex
conjugate. $r_X$ is nonnegative definite on $\mathscr{N} \times \mathscr{N}$.
$X$ is orthogonal if and only if $r_X  (S_1 \times S_2) = 0$, whenever $S_1$
and $S_2$ are disjoint. To each orthogonal $X$ on $\mathscr{N}$ there
corresponds a nonnegative measure $Q_X$ defined on $\mathscr{N}$ by
\begin{equation}
  Q_X (S) = \mathscr{E} |X (S, \omega) |^2 . \label{eq:QX}
\end{equation}
Usually $X$ is initially defined on a semiring of subsets $\mathscr{N}$. If
$r_X$ is of bounded variation over $\mathscr{N} \times \mathscr{N}$ then $X$
can be extended to the restricted $\sigma$-ring $\mathscr{B}_0 = \{S \in
\sigma (\mathscr{N}) = \mathscr{B} ; |r_X | (S \times S) < \infty\}$. We then
say that $X$ is a random measure on $(E, \mathscr{B})$.

Let $L_2 (X) = \sigma \{X (S, \omega) ; S \in \mathscr{B}_0 \}$ denote the
span of $X$ in $L_2 (\Omega, \mathscr{M}, P)$ and let $\Lambda_2 (r_X)$ be the
set of all complex valued, $\mathscr{B}$-measurable functions $f$ on $E$ such
that
\begin{equation}
  \int_{E \times E} f (t) f^{\ast} (s) r_X (dt, ds) < \infty .
  \label{eq:lambda2}
\end{equation}
Then, upon considering two functions $f$ and $g$ in $\Lambda_2 (r_x)$ as
identical if and only if
\begin{equation}
  \int_{E \times E} [f (t) - g (t)]  [f^{\ast} (s) - g^{\ast} (s)] r_X (dt,
  ds) = 0, \label{eq:lambda2-ident}
\end{equation}
$\Lambda_2 (r_X)$ becomes a Hilbert space with inner product
\begin{equation}
  (f, g)_{\Lambda_2 (r_X)} = \int_{E \times E} f (t) g^{\ast} (s) r_X (dt, ds)
  \label{eq:lambda2-inner}
\end{equation}
and $\Lambda_2 (r_X) = \sigma \{I_S (t) ; S \in \mathscr{B}_0 \}$, where $I$
is the indicator function. $L_2 (X)$ and $\Lambda_2 (r_X)$ are isomorphic with
corresponding elements $X (S, \omega)$ and $I_S (t)$, $S \in \mathscr{B}_0$,
and integration of functions in $A_2 (r_X)$ with respect to $X$ is defined by
\begin{equation}
  \xi (\omega) = \int_E f (t) X (dt, \omega), \label{eq:xi}
\end{equation}
where $\xi$ and $f$ are corresponding elements under the isomorphism.

\section{Representation of Harmonizable Stochastic
Processes}\label{sec:representation}

Harmonizable stochastic processes have been introduced by
Lo{\`e}ve~{\cite{loeve1963}} as a first step generalization of wide sense
stationary mean square continuous stochastic processes. A second-order
stochastic process $\{x (t, \omega), t \in \mathbb{R}, \omega \in \Omega\}$ is
harmonizable if and only if it has the integral representation
\begin{equation}
  x (t, \omega) = \int_{- \infty}^{\infty} e^{itu} X (du, \omega)  \quad
  \text{for all } t \in \mathbb{R} \label{eq:proc-integral}
\end{equation}
where $X$ is a random measure defined for all Borel sets $\mathscr{B}^1$ of
the real line $\mathbb{R}$.

It is shown by Lo{\`e}ve~{\cite{loeve1963}} and Cram{\`e}r~{\cite{cramer1951}}
that a second-order stochastic process $x (t, \omega)$ is harmonizable if and
only if its autocorrelation function $R_{xx} (t, s)$ has the integral
representation
\begin{equation}
  R_{xx} (t, s) = \int_{- \infty}^{\infty} \int_{- \infty}^{\infty} e^{i (tu -
  sv)} r_X (du, dv)  \quad \text{for all } t, s \in \mathbb{R}
  \label{eq:autocorr}
\end{equation}
where $r_X$ is a measure of bounded variation on the whole plane
$\mathbb{R}^2$ and nonnegative definite on $\mathscr{B}^1 \times
\mathscr{B}^1$. $r_X$ can be regarded as the two-dimensional spectral measure
of the harmonizable process $x (t, \omega)$, with two-dimensional spectral
distribution $\rho_{xx} (u, v) = r_X  ((- \infty, u] \times (- \infty, v])$.

It is clear that a mean square continuous wide sense stationary stochastic
process is harmonizable, with $X$ an orthogonal random measure and $r_X$ a
nonnegative measure concentrated on the diagonal of the plane.

The functions $\{e^{itu}, t \in \mathbb{R}\}$ span the whole $\Lambda_2 (r_X)$
space and this implies that $L_2 (x) = L_2 (X)$, where $L_2 (x) = \sigma \{x
(t, \omega), t \in \mathbb{R}\}$ denotes the span of $x$ in $L_2 (\Omega,
\mathscr{M}, P)$.

\section{Orthogonal Series Representation of a Harmonizable Stochastic
Process}\label{sec:orthogonal-series}

An orthogonal series representation for harmonizable stochastic processes is
provided by the following:

\begin{theorem}
  \label{thm:series-rep}A harmonizable stochastic process $\{x (t, \omega), t
  \in \mathbb{R}, \omega \in \Omega\}$ is uniformly mean square continuous and
  has an orthogonal series expansion
  \begin{equation}
    x (t, \omega) = \sum_n a_n (t) \xi_n (\omega) \label{eq:series}
  \end{equation}
  in the mean square sense for all $t \in \mathbb{R}$, where
  \begin{equation}
    \xi_n (\omega) = \int_{- \infty}^{\infty} f_n (u) X (du, \omega), \quad
    \mathscr{E} [\xi_n \xi_m^{\ast}] = \delta_{nm} \label{eq:xin}
  \end{equation}
  and
  \begin{equation}
    a_n (t) = \iint_{- \infty}^{\infty} e^{itu} f_n^{\ast} (v) r_X (du, dv)
    \label{eq:an}
  \end{equation}
  and $\{f_n (\cdot)\}$ is an orthonormal and complete set of functions in
  $\Lambda_2 (r_X)$.
\end{theorem}

\begin{proof}
  The autocorrelation function $R_{xx} (t, s)$ of $x (t, \omega)$ is given by
  \eqref{eq:autocorr}. Since $e^{i (tu - sv)}$ is continuous in $t, s$
  uniformly in $u, v$ and is bounded by $1$, which is integrable with respect
  to $r_X$, it follows by the bounded convergence theorem that $R_{xx} (t, s)$
  is uniformly continuous in $t, s$. Hence $x (t, \omega)$ is uniformly mean
  square continuous.
  
  Since $x (t, \omega)$ is mean square continuous, $L_2 (x)$ is separable, as
  shown by Parzen~{\cite{parzen1967}} in Theorem 2C. Hence $L_2 (X) = L_2 (x)$
  is separable and so is its isomorphic space $\Lambda_2 (r_X)$. Let $\{f_n
  (\cdot)\}$ be an orthonormal basis in $\Lambda_2 (r_X)$. If for each $n,
  \xi_n (\omega)$ is the element of $L_2 (x)$ corresponding to $f_n (\cdummy)
  \in \Lambda_2 (r_X)$ under the isomorphism, then $\{\xi_n (\omega)\}$ is an
  orthonormal basis in $L_2 (X) = L_2 (x)$, i.e., $\mathscr{E} [\xi_n
  \xi_m^{\ast}] = \delta_{nm}$, and
  \begin{equation}
    \xi_n (\omega) = \int_{- \infty}^{\infty} f_n (u) X (du, \omega)
    \label{eq:xin-again}
  \end{equation}
  Hence, for all $t \in \mathbb{R}$, we have
  \begin{equation}
    x (t, \omega) = \sum_n a_n (t) \xi_n (\omega) \label{eq:series-again}
  \end{equation}
  in the mean square sense and
  \begin{equation}
    a_n (t) = \mathscr{E} [x (t) \xi_n^{\ast}] \label{eq:an-expect}
  \end{equation}
  Since $x (t, \omega)$ and $e^{itu}$, as well as $\xi_n (\omega)$ and $f_n
  (u)$, are corresponding elements of $L_2 (X)$ and $\Lambda_2 (r_X)$ under
  the isomorphism, we obtain
  \begin{equation}
    a_n (t) = (e^{itu}, f_n (u))_{\Lambda_2 (r_X)} = \int_{- \infty}^{\infty}
    \int_{- \infty}^{\infty} e^{itu} f_n^{\ast} (v) r_X (du, dv)
    \label{eq:an-inner}
  \end{equation}
\end{proof}

It follows from \eqref{eq:series} and \eqref{eq:an} that the autocorrelation
function of a harmonizable stochastic process $x (t, \omega)$ has a series
expansion
\begin{equation}
  R_{xx} (t, s) = \sum_n a_n (t) a_n^{\ast} (s)  \quad \text{for all } t, s
  \in \mathbb{R} \label{eq:autocorr-series}
\end{equation}
where the $a_n$'s are given by \eqref{eq:an}.

For all $f \in \Lambda_2 (r_x)$ we have $e^{itu} f (u) \in \Lambda_2 (r_X)$
for all $t \in \mathbb{R}$, since
\begin{equation}
  |e^{itu} f (u) | \leqslant |f (u) | \in \Lambda_2 (r_X) \label{eq:expf}
\end{equation}
Hence
\begin{equation}
  y (t, \omega) = \int_{- \infty}^{\infty} e^{itu} f (u) X (du, \omega)
  \label{eq:y}
\end{equation}
is well-defined in $L_2 (X) = L_2 (x)$ and is thus a linear operation on $x
(t, \omega)$. $y (t, \omega)$ is harmonizable itself, since if a random
measure $Y$ is defined by
\begin{equation}
  Y (S, \omega) = \int_S f (u) X (du, \omega), \quad S \in \mathscr{B}^1
  \label{eq:Y}
\end{equation}
then $Y$ is finite on $\mathbb{R}$ and
\begin{equation}
  y (t, \omega) = \int_{- \infty}^{\infty} e^{itu} Y (du, \omega)
  \label{eq:yY}
\end{equation}
Let $y_n  (t, \omega)$ be the linear operation on $x (t, \omega)$ defined by
\eqref{eq:y} when $f (u) = f_n (u)$. Then $\xi_n (\omega) = y_n  (0, \omega)$.
If $f_n$ has Fourier transform $h_n$, i.e., if
\begin{equation}
  f_n (u) = \int_{- \infty}^{\infty} h_n (v) e^{- iuv} m (dv), \quad h_n \in
  L_1 (\mathbb{R}, \mathscr{B}^1, m) \label{eq:fnFourier}
\end{equation}
then, as it is shown by Rosanov~{\cite{rosanov1959}}, p. 278, the order of
integration in \eqref{eq:y} can be interchanged to give
\begin{equation}
  y_n  (t, \omega) = \int_{- \infty}^{\infty} h_n (v) x (t - v, \omega) m (dv)
  \label{eq:yn}
\end{equation}
Hence $\xi_n (\omega)$ may be regarded as the output at time $t = 0$ of a
linear time invariant system with impulse response $h_n$ and input $x (t,
\omega)$. We also obtain from \eqref{eq:an}
\begin{equation}
  a_n (t) = \int_{- \infty}^{\infty} h_n^{\ast} (\tau) R_{xx} (t, - \tau) m (d
  \tau) \label{eq:an-hn}
\end{equation}
It should be noted that the series representation of Theorem
\ref{thm:series-rep} is by no means unique, since for each orthonormal and
complete set of functions in $\Lambda_2 (r_X)$ a distinct representation
\eqref{eq:series} is obtained by \eqref{eq:xin} and \eqref{eq:an}. However, in
the context of a particular problem, one may be able to determine those
representations, if any, which have some optimal properties. The significance
of the representation is primarily in the fact that it exists and it is
orthogonal, which enables one to use it as a model for the harmonizable
process in problems involving mean square error criteria.

Theorem \ref{thm:series-rep} has been proven for mean square continuous, wide
sense stationary stochastic processes by Masry et al.~{\cite{masry1968}} and
also by Campbell~{\cite{campbell1969}}.

\section{Orthonormal and Complete Sets in $\Lambda_2
(r_X)$}\label{sec:orthonormal-sets}

It is clear from Theorem \ref{thm:series-rep} that an explicit series
expansion of a harmonizable process can be obtained by using \eqref{eq:xin}
and \eqref{eq:an} provided an orthonormal and complete set of functions in
$\Lambda_2 (r_X)$ can be constructed.

If the harmonizable process $x$ is stationary then $\Lambda_2 (r_X)$ is
isomorphic to $L_2 (\mathbb{R}, \mathscr{B}^1, Q_X)$, where $Q_X$ is a finite
nonnegative measure, and a general procedure to construct an orthonormal basis
in the latter space is presented by Masry et al.~{\cite{masry1968}}.

Since
\begin{equation}
  \Lambda_2 (r_X) = \sigma \{e^{itu}, t \text{real } \} = \sigma \{e^{itu}, t
  \text{rational } \}, \label{eq:span}
\end{equation}
an orthonormal and complete set of functions in $\Lambda_2 (r_X)$ can be
obtained by orthonormalizing the countable set of functions $\{e^{itu}, t$
rational $\}$ using the Gram-Schmidt procedure. However, this procedure solves
the problem of finding an orthonormal basis in $\Lambda_2 (r_X)$ only in
principle.

The following theorem gives a complete set of functions $\{F_n (t)\}$ in
$\Lambda_2 (r_X)$. By orthonormalizing the set $\{F_n (t)\}$ using the
Gram-Schmidt procedure, an orthonormal and complete set $\{f_n (t)\}$ is
obtained.

\begin{theorem}
  \label{thm:complete-set}Let $\mu$ be any finite, nonnegative measure on
  $(\mathbb{R}, \mathscr{B}^1)$, absolutely continuous with respect to the
  Lebesgue measure $m$ with Radon-Nikodym derivative $[d \mu / dm] (t) = h (t)
  \neq 0$ a.e. $[m]$, and let $\{\phi_n (t)\}$ be any complete set of
  functions in $L_2 (\mathbb{R}, \mathscr{B}^1, \mu) = L_2 (\mu)$. Then the
  set $\{F_n (t)\}$ given by
  \begin{equation}
    F_n (t) = \int_{- \infty}^{\infty} \phi_n^{\ast} (u) e^{itu} \mu (du)
    \label{eq:Fn}
  \end{equation}
  is complete in $\Lambda_2 (r_X)$.
\end{theorem}

\begin{proof}
  Since $\mu$ is finite, $\phi_n \in L_2 (\mu)$ implies $\phi_n \in L_1 (\mu)$
  and hence the functions $F_n (t)$ are well-defined by \eqref{eq:Fn}
  everywhere and
  \begin{equation}
    |F_n (t) | \leqslant \int_{- \infty}^{\infty} | \phi_n^{\ast} (u) | \mu
    (du) = \| \phi_n \|_{L_1 (\mu)} \label{eq:Fn-bound}
  \end{equation}
  This implies that $F_n \in \Lambda_2 (r_X)$ and
  \begin{equation}
    \|F_n \|_{\Lambda_2 (r_X)}^2 \leqslant \| \phi_n \|_{L_1 (\mu)}^2 |r_X |
    (R^2), \label{eq:Fn-norm}
  \end{equation}
  where $|r_X |$ denotes the total variation of $r_X$.
  
  The completeness of the set $\{F_n (t)\}$ in $\Lambda_2 (r_X)$ is shown as
  follows. Let $f \in \Lambda_2 (r_X)$ and $(F_n, f)_{\Lambda_2 (r_X)} = 0$
  for all $n$. Then
  
  \begin{align}
    0 & = \int_{- \infty}^{\infty} \int_{- \infty}^{\infty} F_n (u) f^{\ast}
    (v) r_X (du, dv) \\
    & = \int_{- \infty}^{\infty} \int_{- \infty}^{\infty} \int_{-
    \infty}^{\infty} \phi_n^{\ast} (t) e^{itu} f^{\ast} (v) \mu (dt) r_X (du,
    dv) \\
    & = \int_{- \infty}^{\infty} F (t) \phi_n^{\ast} (t) \mu (dt) 
    \label{eq:Fn-complete}
  \end{align}
  
  where
  \begin{equation}
    F (t) = \int_{- \infty}^{\infty} \int_{- \infty}^{\infty} e^{itu} f^{\ast}
    (v) r_X (du, dv) = (e^{itu}, f (u))_{\Lambda_2 (r_X)} \label{eq:F}
  \end{equation}
  It follows by \eqref{eq:Fn-complete} that
  
  \begin{align}
    |F (t) |^2 & \leqslant \|e^{itu} \|_{\Lambda_2 (r_X)}^2 \|f\|_{\Lambda_2
    (r_X)}^2 = R_{xx} (t, t) \|f\|_{\Lambda_2 (r_X)}^2 \\
    & \leqslant |r_X | (R^2) \|f\|_{\Lambda_2 (r_X)}^2  \label{eq:F-bound}
  \end{align}
  
  and
  \begin{equation}
    \int_{- \infty}^{\infty} |F (t) |^2 \mu (dt) \leqslant |r_X | (R^2)
    \|f\|_{\Lambda_2 (r_X)}^2 \mu (\mathbb{R}) < \infty \label{eq:F-L2}
  \end{equation}
  i.e., $F \in L_2 (\mu)$. The completeness of the set $\{\phi_n (t)\}$ in
  $L_2 (\mu)$ and \eqref{eq:Fn-complete} imply that $F = 0$ in $L_2 (\mu)$,
  i.e., $|F (t) |^2 h (t) = 0$ a.e. $[m]$. Since $h (t) \neq 0$ a.e. [m], it
  follows that $F (t) = 0$ a.e. [m]. But the continuity of $e^{itu}$ in $t$
  and the bounded convergence theorem imply by \eqref{eq:F} that $F (t)$ is
  continuous in $t$. Hence $F (t) = 0$ for all $t \in \mathbb{R}$ and by
  \eqref{eq:F} $f$ is orthogonal to the set $\{e^{itu}, t \in \mathbb{R}\}$,
  which is dense in $\Lambda_2 (r_X)$. It follows that $f = 0$ in $\Lambda_2
  (r_X)$ and hence the set $\{F_n (t)\}$ is complete in $\Lambda_2 (r_X)$.
\end{proof}

It should be pointed out that the set of functions $\{F_n (t)\}$ given by
\eqref{eq:Fn}, which is shown in Theorem \ref{thm:complete-set} to be complete
in any $\Lambda_2 (r_X)$ space with $r_X$ of bounded variation on the entire
plane, is independent of the measure $r_X$ and is completely determined by
$\mu$ and $\{\phi_n (t)\}$. It is clear, however, that the orthonormal and
complete set of functions $\{f_n (t)\}$ in $\Lambda_2 (r_X)$, obtained by
orthonormalizing the complete set $\{F_n (t)\}$, depends on the measure $r_X$.

Theorem \ref{thm:complete-set} allows considerable freedom in the choice of
the measure $\mu$ and complete freedom in the choice of the complete set of
functions $\{\phi_n (t)\}$ in $L_2 (\mu)$. As a complete set of functions in
$L_2 (\mu)$ one can choose the orthonormal and complete set $\{\phi_n (t)\}$
given by Masry et al.~{\cite{masry1968}}:
\begin{equation}
  \phi_n (t) = \frac{1}{\sqrt{\mu (\mathbb{R})}} \exp \left( \frac{in 2
  \pi}{\mu (\mathbb{R})}  \int_{- \infty}^t h (u) m (du) \right), \quad n = 0,
  \pm 1, \pm 2, \ldots \label{eq:phin}
\end{equation}
Upon normalizing $\mu, \mu (\mathbb{R}) = 1$, and using the complete set
\eqref{eq:phin} in \eqref{eq:Fn} we obtain the complete set of functions

\begin{align}
  F_n (t) & = \int_{- \infty}^{\infty} \exp [i\{tu - n 2 \pi H (u)\}] h (u) m
  (du)  \quad n = 0, \pm 1, \pm 2, \ldots \\
  & = \int_0^1 \exp [i\{tH^{- 1} (v) - n 2 \pi v\}] m (dv)  \label{eq:Fn-exp}
\end{align}

where $h$ is any probability density with $h (u) \neq 0$ a.e. [$m$], and
\begin{equation}
  H (u) = \int_{- \infty}^u h (v) m (dv) \label{eq:H}
\end{equation}
It is clear that the functions given by \eqref{eq:Fn} are uniformly continuous
and uniformly bounded for fixed $n$, and that the family of functions given by
\eqref{eq:Fn-exp} is equicontinuous and uniformly bounded.

\subsection*{Examples}

By choosing probability densities $h$ of particular form, we obtain by
\eqref{eq:Fn-exp} various sets of complete functions in $\Lambda_2 (r_X)$.
However, as it is illustrated by the following examples, the integral in
\eqref{eq:Fn-exp} is not easily expressed in terms of the elementary and the
special functions.
\begin{enumerate}
  \item The density of the normal distribution, $h (u) = (1 / \sqrt{2 \pi})
  e^{- \frac{1}{2} u^2}$, gives
  \begin{equation}
    F_n (t) = (- 1)^n \sqrt{\frac{2}{\pi}}  \int_0^{\infty} \cos \left[ tu - n
    \pi \Phi \left( \frac{u}{\sqrt{2}} \right) \right] e^{- \frac{1}{2} u^2} m
    (du) \label{eq:Fn-normal}
  \end{equation}
  where $\Phi (u) = (2 / \sqrt{\pi})  \int_0^u e^{- v^2} m (dv)$.
  
  \item The density of the double exponential distribution, $h (u) =
  \frac{1}{2} e^{- |u|}$, gives
  
  \begin{align}
    F_n (t) & = \int_0^{\infty} e^{- u} \cos [tu + n \pi e^{- u}] m (du) \\
    & = \int_0^1 \cos [t \ln v - n \pi v] m (dv)  \label{eq:Fn-double-exp}
  \end{align}
  
  \item The density of the Cauchy distribution, $h (u) = (1 / \pi (1 + u^2))$,
  gives
  
  \begin{align}
    F_n (t) & = (- 1)^n \frac{2}{\pi}  \int_0^{\infty} \cos [tu - 2 n \tan^{-
    1} u]  \frac{m (du)}{1 + u^2} \\
    & = (- 1)^n \frac{2}{\pi}  \int_0^{\pi / 2} \cos [t \tan v - 2 nv] m (dv)
    \label{eq:Fn-cauchy}
  \end{align}
  
  \item The probability density
  \begin{equation}
    h (u) = \left\{ \begin{array}{ll}
      \alpha^{- k} & \text{on } (k - 1, k), \quad k \leqslant - 1\\
      \frac{1 - 3 \alpha}{2 (1 - \alpha)} & \text{on } (- 1, + 1),\\
      \alpha^k & \text{on } (k, k + 1), \quad k \geqslant 1
    \end{array} \right. \label{eq:h-custom}
  \end{equation}
  where $0 < \alpha < \frac{1}{3}$, gives
  \begin{equation}
    F_n (t) = 2 (- 1)^n  \sum_{k = 0}^{\infty} \alpha^k \cos \left[ \left( k +
    \frac{1}{2} \right) t - n \pi c_k \right]  \frac{\sin \left( \frac{1}{2} t
    - n \pi \alpha^k \right)}{\frac{1}{2} t - n \pi \alpha^k}
    \label{eq:Fn-custom}
  \end{equation}
  where $c_k = 1 - (1 + \alpha / 1 - \alpha) \alpha^k$.
  
  \item If $r_X$ is supported by $(a, b) \times (a, b)$, then by using the
  density of the uniform distribution on $(a, b), h (u) = (1 / b - a)$ on $(a,
  b)$ and zero elsewhere, we obtain the complete set of functions
  \begin{equation}
    F_n (t) = \frac{e^{ibt} - e^{iat}}{i [(b - a) t - n 2 \pi]}
    \label{eq:Fn-uniform}
  \end{equation}
  which in the particular case where $a = - T, b = T$ gives
  \begin{equation}
    F_n (t) = (- 1)^n \frac{\sin \left[ \pi \left( \frac{T}{\pi} t - n \right)
    \right]}{\pi \left( \frac{T}{\pi} t - n \right)}
    \label{eq:Fn-uniform-symm}
  \end{equation}
\end{enumerate}
\section{Orthogonal Integral Representation of a Harmonizable Stochastic
Process}\label{sec:orthogonal-integral}

Clearly any second order stochastic process $x (t, \omega)$ having an
orthogonal series expansion of the form \eqref{eq:series} admits a trivial
orthogonal integral representation of the form
\begin{equation}
  x (t, \omega) = \int_{- \infty}^{\infty} f (t, u) Y (du, \omega),
  \label{eq:integral-rep}
\end{equation}
where the orthogonal random measure $Y$ is concentrated on the set of integers
with $Y (\{n\}, \omega) = \xi_n (\omega)$ and $f (t, n) = a_n (t)$.

The following theorem shows that an explicit (nontrivial) orthogonal integral
representation of a harmonizable stochastic process can always be obtained and
that the nonnegative measure associated with the orthogonal random measure can
be chosen arbitrarily from a wide class of measures.

\begin{theorem}
  \label{thm:integral-rep}Let $\mu$ be any nonnegative measure on
  $(\mathbb{R}, \mathscr{B}^1)$, finite on the bounded Borel sets
  $\mathscr{B}$ and such that $L_2 (\mathbb{R}, \mathscr{B}^1, \mu) = L_2
  (\mu)$ is infinite dimensional. Let $\{\varphi_n (\cdot)\}$ be an
  orthonormal and complete set of functions in $L_2 (\mu)$. Then every
  harmonizable stochastic process $x (t, \omega)$ admits an orthogonal
  integral representation
  \begin{equation}
    x (t, \omega) = \int_{- \infty}^{\infty} f (t, u) Y (du, \omega)
    \label{eq:integral-rep-thm}
  \end{equation}
  The function $f (t, u)$ is given by
  \begin{equation}
    f (t, u) = \sum_n a_n (t) \varphi_n (u) \label{eq:f-integral}
  \end{equation}
  in $L_2 (\mu)$ for all $t \in \mathbb{R}$. The orthogonal random measure $Y$
  is defined on $\mathscr{B}$ by
  \begin{equation}
    Y (S, \omega) = \sum_n \xi_n (\omega)  \int_S \varphi_n^{\ast} (u) \mu
    (du) \label{eq:Y-thm}
  \end{equation}
  in the mean square sense for all $S \in \mathscr{B}$ and has $Q_Y = \mu$.
  The $a_n$'s and the $\xi_n$'s are given in Theorem \ref{thm:series-rep}.
\end{theorem}

\begin{proof}
  We first show that $Y$ is well-defined on $\mathscr{B}$ by \eqref{eq:Y-thm}
  and that it is orthogonal. Since $\mu$ is finite on $\mathscr{B}, I_S \in
  L_2 (\mu)$ for all $S \in \mathscr{B}$. Hence
  \begin{equation}
    I_S (u) = \sum_n b_n (S) \varphi_n (u) \label{eq:IS}
  \end{equation}
  where
  \begin{equation}
    b_n (S) = \int_S \varphi_n^{\ast} (u) \mu (du) \label{eq:bn}
  \end{equation}
  It follows that
  
  \begin{align}
    & \sum_n \int_{S_1} \varphi_n^{\ast} (u) \mu (du)  \int_{S_2} \varphi_n
    (u) \mu (du) \\
    & \quad = (I_{S_1}, I_{S_2})_{L_2 (\mu)} = \mu (S_1 \cap S_2) < \infty
    \quad \text{for all } \quad S_1, S_2 \in \mathscr{B} 
    \label{eq:sum-finite}
  \end{align}
  
  and in particular that
  \begin{equation}
    \sum_n \left| \int_S \varphi_n^{\ast} (u) \mu (du) \right|^2 = \|I_S
    \|_{L_2 (\mu)}^2 = \mu (S) < \infty \label{eq:mu-finite}
  \end{equation}
  Hence $Y$ is well-defined by \eqref{eq:Y-thm} in $L_2 (\Omega, \mathscr{M},
  P)$ for all $S \in \mathscr{B}$. It follows by \eqref{eq:Y-thm} and
  \eqref{eq:bn} that
  \begin{equation}
    r_Y  (S_1 \times S_2) = \mu (S_1 \cap S_2) \label{eq:rY}
  \end{equation}
  and thus $Y$ is orthogonal with $Q_Y = \mu$.
  
  We next show that $f (t, u)$ is well-defined in $L_2 (\mu)$ by
  \eqref{eq:f-integral} for all $t \in \mathbb{R}$. This is clear since from
  \eqref{eq:autocorr-series}
  \begin{equation}
    \sum_n |a_n (t) |^2 = R_{xx} (t, t) < \infty \quad \text{for all } \quad t
    \in \mathbb{R} \label{eq:an-finite}
  \end{equation}
  Hence the integral in \eqref{eq:integral-rep-thm} is well-defined and from
  \eqref{eq:f-integral}, \eqref{eq:Y-thm}, and \eqref{eq:series} we have
  
  \begin{align}
    \int_{- \infty}^{\infty} f (t, u) Y (du, \omega) & = \sum_n a_n (t) 
    \int_{- \infty}^{\infty} \varphi_n (u) Y (du, \omega) \\
    & = \sum_n a_n (t)  \sum_m \int_{- \infty}^{\infty} \varphi_n (u)
    \varphi_m^{\ast} (u) \mu (du) \xi_m (\omega) \\
    & = \sum_n a_n (t) \xi_n (\omega) = x (t, \omega) 
    \label{eq:integral-proof}
  \end{align}
  
  which proves \eqref{eq:integral-rep-thm}.
\end{proof}

The freedom in choosing the measure $\mu$ enables us to obtain various
orthogonal integral representations \eqref{eq:integral-rep-thm} of particular
form. If $\mu$ is chosen to be a finite nonnegative measure on $\mathbb{R}$
then $Y$ will be finite on the whole real line and the $\varphi_n$'s can be
chosen as in \eqref{eq:phin} given by Masry et al.~{\cite{masry1968}}. If
$\mu$ is chosen to be the Lebesgue measure or the Lebesgue measure restricted
to the half line or to an interval, then the $\varphi_n$'s may be chosen to be
well-known orthonormal and complete sets of functions such as the
Tchebysheff-Hermite functions, the Tchebysheff-Laguerre functions, the
Legendre polynomials or the trigonometric system. In this latter case it is
clear from \eqref{eq:rY} that the orthogonal random measure $Y$ has stationary
values.

A harmonizable stochastic process is shown to have the nonorthogonal integral
representation \eqref{eq:proc-integral}, the orthogonal series expansion
\eqref{eq:series} and the orthogonal integral representation
\eqref{eq:integral-rep-thm}. The relationship between the orthogonal random
measure $Y$ and the random measure $X$ is
\begin{equation}
  Y (S, \omega) = \int_{- \infty}^{\infty} \left[ \sum_n f_n (v) \int_S
  \varphi_n^{\ast} (u) \mu (du) \right] X (dv, \omega) \label{eq:Y-X}
\end{equation}
for all $S \in \mathscr{B}$, which can be obtained by \eqref{eq:Y-thm} and
\eqref{eq:xin}.

\section{Moving Average Representations and Harmonizable Stochastic
Processes}\label{sec:moving-avg}

A second-order stochastic process $\{x (t, \omega), t \in \mathbb{R}, \omega
\in \Omega\}$ is said to have a moving average representation if and only if
\begin{equation}
  x (t, \omega) = \int_{- \infty}^{\infty} f (t - u) X (du, \omega)  \quad
  \text{for all } \quad t \in \mathbb{R} \label{eq:moving-avg}
\end{equation}
where $X$ is a random measure defined on the bounded Borel sets of
$\mathbb{R}$ and $f (t - \cdot) \in \Lambda_2 (r_X)$ for all $t \in
\mathbb{R}$. This is a generalization of the usual definition which assumes
$X$ to be orthogonal and $Q_X$ to be the Lebesgue measure. A moving average
representation is orthogonal if and only if $X$ is orthogonal and in this case
$f (t - \cdot) \in L_2 (Q_X)$ for all $t \in \mathbb{R}$.

It is shown by Karhunen~{\cite{karhunen1947}} and Doob~{\cite{doob1953}} that
(i) a second-order stochastic process which has an orthogonal moving average
representation with $Q_X = m$, the Lebesgue measure, and $f \in L_2
(\mathbb{R}, \mathscr{B}^1, m) = L_2 (m)$ is mean square continuous wide sense
stationary and has absolutely continuous spectrum; and conversely that (ii) a
mean square continuous wide sense stationary process with absolutely
continuous spectrum has a moving average representation with $Q_X = m$ and $f
\in L_2 (m)$ is the Fourier transform of the square root of its spectral
density.

Sufficient conditions for the harmonizability of a stochastic process which
has a moving average representation are given in the following

\begin{theorem}
  \label{thm:moving-avg-harmonizable}If a second-order stochastic process $x
  (t, \omega)$ has a moving average representation with $r_X$ a measure of
  bounded variation on the entire plane $\mathbb{R}^2, f \in L_1 (m)$ and its
  Fourier transform $F \in L_1 (m)$, then $x (t, \omega)$ is harmonizable.
\end{theorem}

\begin{proof}
  We have
  \begin{equation}
    f (\tau) = \frac{1}{\sqrt{2 \pi}}  \int_{- \infty}^{\infty} F (\rho) e^{i
    \tau \rho} m (d \rho) \label{eq:f-fourier}
  \end{equation}
  Since $F \in L_1 (m)$ and $r_X$ is finite, by interchanging the order of
  integration (see Rosanov~{\cite{rosanov1959}}, p. 287), we obtain from
  \eqref{eq:moving-avg}
  \begin{equation}
    x (t, \omega) = \frac{1}{\sqrt{2 \pi}}  \int_{- \infty}^{\infty} \int_{-
    \infty}^{\infty} F (\rho) e^{- iu \rho} e^{it \rho} X (du, \omega) m (d
    \rho) \label{eq:x-interchange}
  \end{equation}
  Also, $F \in L_1 (m)$ and $r_X$ finite imply that for all $S \in
  \mathscr{B}^1$
  \begin{equation}
    \int_S F (\rho) e^{- iu \rho} m (d \rho) \in \Lambda_2 (r_X)
    \label{eq:F-Lambda}
  \end{equation}
  Hence the random measure $Y$ on $(\mathbb{R}, \mathscr{B}^1)$ is
  well-defined by
  \begin{equation}
    Y (S, \omega) = \frac{1}{\sqrt{2 \pi}}  \int_{- \infty}^{\infty} \left[
    \int_S F (\rho) e^{- iu \rho} m (d \rho) \right] X (du, \omega)
    \label{eq:Y-define}
  \end{equation}
  for all $S \in \mathscr{B}^1$ and by interchanging the order of integration
  we obtain
  \begin{equation}
    Y (S, \omega) = \frac{1}{\sqrt{2 \pi}}  \int_S F (\rho) \left[ \int_{-
    \infty}^{\infty} e^{- iu \rho} X (du, \omega) \right] m (d \rho)
    \label{eq:Y-interchange}
  \end{equation}
  i.e.,
  \begin{equation}
    \left[ \frac{dY}{dm} \right] (\rho, \omega) = \frac{F (\rho)}{\sqrt{2
    \pi}}  \int_{- \infty}^{\infty} e^{- iu \rho} X (du, \omega)
    \label{eq:dY-dm}
  \end{equation}
  Since $F \in L_1 (m)$ and $r_X$ is finite, the lemma which follows this
  proof applies, and \eqref{eq:x-interchange} and \eqref{eq:dY-dm} imply
  \begin{equation}
    x (t, \omega) = \int_{- \infty}^{\infty} e^{it \rho} Y (d \rho, \omega)
    \label{eq:x-harmonize}
  \end{equation}
  Therefore, $x (t, \omega)$ is harmonizable.
\end{proof}

The property used in the last step of the proof of Theorem
\ref{thm:moving-avg-harmonizable} will be proven now. It corresponds to the
familiar property of Radon-Nikodym derivative in the scalar case and is used
in later sections of this paper.

\begin{lemma}
  \label{lem:radon-nikodym}If the second-order stochastic process $\{y (t,
  \omega), t \in \mathbb{R}, \omega \in \Omega\}$ is such that
  \begin{equation}
    \int_{- \infty}^{\infty} \int_{- \infty}^{\infty} |R_{yy} (t, s) | m (dt)
    m (ds) < \infty \label{eq:Ryy-finite}
  \end{equation}
  then
  \begin{equation}
    Y (S, \omega) = \int_S y (t, \omega) m (dt), \quad S \in \mathscr{B}^1
    \label{eq:Y-lemma}
  \end{equation}
  defines a random measure $Y$ on $(\mathbb{R}, \mathscr{B}^1)$ with $r_Y$ of
  bounded variation on $\mathbb{R}^2$, and for all $g \in \Lambda_2 (r_Y)$
  \begin{equation}
    \int_{- \infty}^{\infty} g (t) Y (dt, \omega) = \int_{- \infty}^{\infty} g
    (t) y (t, \omega) m (dt) \label{eq:g-interchange}
  \end{equation}
  all equalities being in the mean square sense.
\end{lemma}

\begin{proof}
  It is clear from \eqref{eq:Ryy-finite} that \eqref{eq:Y-lemma} defines a
  random measure $Y$ on $(\mathbb{R}, \mathscr{B}^1)$ with $r_Y$ of bounded
  variation on $\mathbb{R}^2$.
  
  If we put
  \begin{equation}
    \xi (\omega) = \int_{- \infty}^{\infty} g (t) Y (dt, \omega)  \quad
    \text{and } \quad \eta (\omega) = \int_{- \infty}^{\infty} g (t) y (t,
    \omega) m (dt) \label{eq:xi-eta}
  \end{equation}
  it suffices to show that
  \begin{equation}
    \mathscr{E} [| \xi - \eta |^2] = \mathscr{E} [| \xi |^2] + \mathscr{E} [|
    \eta |^2] - \mathscr{E} [\xi \eta^{\ast}] - \mathscr{E} [\eta \xi^{\ast}]
    = 0 \label{eq:xi-eta-equal}
  \end{equation}
  We have
  
  \begin{align}
    \mathscr{E} [| \xi |^2] & = \int_{- \infty}^{\infty} \int_{-
    \infty}^{\infty} g (t) g^{\ast} (s) r_Y (dt, ds) \\
    & = \int_{- \infty}^{\infty} \int_{- \infty}^{\infty} g (t) g^{\ast} (s)
    R_{yy} (t, s) m (dt) m (ds) \\
    \mathscr{E} [| \eta |^2] & = \int_{- \infty}^{\infty} \int_{-
    \infty}^{\infty} g (t) g^{\ast} (s) R_{yy} (t, s) m (dt) m (ds), \\
    \mathscr{E} [\eta \xi^{\ast}] & = \int_{- \infty}^{\infty} g (t)
    \mathscr{E} [y (t) \xi^{\ast}] m (dt) \\
    \mathscr{E} [y (t) \xi^{\ast}] & = \int_{- \infty}^{\infty} g^{\ast} (s)
    \lambda_t  (ds)  \label{eq:E-terms}
  \end{align}
  
  where the measure $\lambda_t$ on $(\mathbb{R}, \mathscr{B}^1)$ is defined by
  \begin{equation}
    \lambda_t (S) = \mathscr{E} [y (t) Y^{\ast} (S)] = \int_S R_{yy} (t, s) m
    (ds) \label{eq:lambda}
  \end{equation}
  for all $S \in \mathscr{B}^1$. It follows by \eqref{eq:E-terms} and
  \eqref{eq:lambda} that
  \begin{equation}
    \mathscr{E} [y (t) \xi^{\ast}] = \int_{- \infty}^{\infty} g^{\ast} (s)
    R_{yy} (t, s) m (ds) \label{eq:E-y-xi}
  \end{equation}
  and by \eqref{eq:E-terms}
  \begin{equation}
    \mathscr{E} [\eta \xi^{\ast}] = \int_{- \infty}^{\infty} \int_{-
    \infty}^{\infty} g (t) g^{\ast} (s) R_{yy} (t, s) m (dt) m (ds) =
    \mathscr{E} [\xi \eta^{\ast}] \label{eq:E-eta-xi}
  \end{equation}
  The validity of \eqref{eq:xi-eta-equal} follows from \eqref{eq:E-terms},
  \eqref{eq:E-eta-xi} and the proof is complete.
\end{proof}

If the moving average representation in Theorem
\ref{thm:moving-avg-harmonizable} is orthogonal, then the condition of bounded
variation of $r_X$ is equivalent to the finiteness of $Q_X$.

If the second-order stochastic process $x (t, \omega)$ has a moving average
representation and $X$ has Radon-Nikodym derivative with respect to the
Lebesgue measure the second-order stochastic process $y (t, \omega), [dY / dm]
= y$, then
\begin{equation}
  x (t, \omega) = \int_{- \infty}^{\infty} f (t - u) y (u, \omega) m (du)
  \label{eq:x-conv}
\end{equation}
In this case $x (t, \omega)$ can be regarded as the output of a linear time
invariant system with impulse response $f$ and input the stochastic process $y
(t, \omega)$. Theorem \ref{thm:moving-avg-harmonizable} then implies that if
$y (t, \omega)$ is integrable over $\mathbb{R}$ in the mean square sense,
i.e., if $R_{yy} (t, s)$ is integrable over $\mathbb{R}^2$, and $f, F \in L_1
(m)$ then the output $x (t, \omega)$ is a harmonizable stochastic process. In
the following section a more general result is proven which includes time
varying linear systems.

\section{Linear Time Varying Systems and Harmonizable Stochastic
Processes}\label{sec:linear-systems}

Two kinds of linear time varying systems characterized by their impulse
response $h (t, \tau)$, i.e., the response at time $t$ to a unit impulse input
at time $\tau$, are considered in this section. Systems with $h$ a
deterministic function and systems with $h$ a sample function of a stochastic
process.

\subsection{Deterministic Linear Time Varying
Systems}\label{subsec:deterministic}

Consider a linear time varying system with impulse response $h (t, \tau)$. If
the input process $x$ is such that
\begin{equation}
  \int_{- \infty}^{\infty} \int_{- \infty}^{\infty} h (t, u) h^{\ast}  (t, v)
  R_{xx} (u, v) m (du) m (dv) < \infty \quad \text{for all } \quad t \in
  \mathbb{R} \label{eq:impulse-finite}
\end{equation}
then the output of the system is the second-order stochastic process $y$
defined by
\begin{equation}
  y (t, \omega) = \int_{- \infty}^{\infty} h (t, u) x (u, \omega) m (du) 
  \quad \text{for all } \quad t \in \mathbb{R} \label{eq:y-output}
\end{equation}
It is apparent that a sufficient condition for the output of the system to be
a stochastic process of second-order for all input processes $x$ which have
uniformly bounded autocorrelation functions
\begin{equation}
  |R_{xx} (u, v) | \leqslant M < \infty \quad \text{for all } u, v \in
  \mathbb{R} \label{eq:Rxx-bounded}
\end{equation}
is
\begin{equation}
  h (t, \cdot) \in L_1 (m)  \quad \text{for all } t \in \mathbb{R}
  \label{eq:h-L1}
\end{equation}
The wide sense stationary processes $x$ belong to this class since
\begin{equation}
  |R_{xx} (u, v) | = |R_{xx} (u - v) | \leqslant R_{xx} (0) < \infty \quad
  \text{for all } u, v \in \mathbb{R} \label{eq:Rxx-stationary}
\end{equation}
and so do the harmonizable stochastic processes $x$, since
\begin{equation}
  |R_{xx} (u, v) | \leqslant |r_X | (\mathbb{R}^2) < \infty \quad \text{for
  all } u, v \in \mathbb{R} \label{eq:Rxx-harmonizable}
\end{equation}
The following theorem provides a set of sufficient conditions which imply the
harmonizability of the output of a linear time varying system.

\begin{theorem}
  \label{thm:linear-system-harmonizable}Let $h (t, \tau)$ be the impulse
  response of a linear time varying system and $x (t, \omega)$ be the input
  stochastic process. If $h (\cdummy, \tau)$ is the Fourier transform of a
  function $g (\cdummy, \tau) \in L_1 (m)$ for all $\tau \in \mathbb{R}$ and
  if $g$ satisfies
  \begin{equation}
    \int_{- \infty}^{\infty} \int_{- \infty}^{\infty} \int_{- \infty}^{\infty}
    \int_{- \infty}^{\infty} |g (\tau, u) | |g^{\ast} (\sigma, v) | |R_{xx}
    (u, v) | m (du) m (dv) m (d \tau) m (d \sigma) < \infty
    \label{eq:g-condition}
  \end{equation}
  then the output stochastic process is harmonizable.
\end{theorem}

\begin{proof}
  For all $\tau \in \mathbb{R}$ we have
  \begin{equation}
    h (t, \tau) = \frac{1}{\sqrt{2 \pi}}  \int_{- \infty}^{\infty} g (s, \tau)
    e^{its} m (ds) \label{eq:h-g-transform}
  \end{equation}
  Hence \eqref{eq:impulse-finite} implies \eqref{eq:impulse-finite} and $y (t,
  \omega)$ is well-defined by \eqref{eq:y-output} in the stochastic mean. It
  follows by \eqref{eq:g-condition} that the random measure $Y$ defined on
  $(\mathbb{R}, \mathscr{B}^1)$ by
  \begin{equation}
    \left[ \frac{dY}{dm} \right] (\tau, \omega) = \frac{1}{\sqrt{2 \pi}} 
    \int_{- \infty}^{\infty} g (\tau, u) x (u, \omega) m (du)
    \label{eq:dY-dm-g}
  \end{equation}
  has $r_Y$ of finite variation on the entire plane $\mathbb{R}^2$. For all
  $S_1, S_2 \in \mathscr{B}^1$
  \begin{equation}
    r_Y  (S_1 \times S_2) = \frac{1}{2 \pi}  \int_{- \infty}^{\infty} \int_{-
    \infty}^{\infty} \int_{S_1} \int_{S_2} g (\tau, u) g^{\ast} (\sigma, v)
    R_{xx} (u, v) m (d \tau) m (d \sigma) m (du) m (dv) \label{eq:rY-g}
  \end{equation}
  It is clear from \eqref{eq:y-output}, \eqref{eq:h-g-transform},
  \eqref{eq:dY-dm-g} and Lemma \ref{lem:radon-nikodym} that
  \begin{equation}
    y (t, \omega) = \int_{- \infty}^{\infty} e^{it \tau} \left[ \frac{dY}{dm}
    \right] (\tau, \omega) m (d \tau) = \int_{- \infty}^{\infty} e^{it \tau} Y
    (d \tau, \omega) \label{eq:y-harmonize}
  \end{equation}
  in the stochastic mean sense and hence $y$ is harmonizable.
\end{proof}

A sufficient condition for \eqref{eq:g-condition} to be satisfied for the
class of input processes with uniformly bounded autocorrelation functions is
clearly
\begin{equation}
  g (s, \tau) \in L_1  (m \times m) \label{eq:g-L1}
\end{equation}
As an example, let
\begin{equation}
  h (t, \tau) = \frac{\alpha + \gamma | \tau |}{t^2 + (\alpha + \gamma | \tau
  |)^2} e^{- \beta | \tau |}  \quad \alpha, \beta > 0, \quad \gamma \geqslant
  0 \label{eq:h-example}
\end{equation}
Then $h (t, \tau)$ is the Fourier transform of $g (s, \tau)$ :
\begin{equation}
  g (s, \tau) = \sqrt{\frac{\pi}{2}} e^{- \alpha |s | - \beta | \tau | -
  \gamma |s \tau |} \label{eq:g-example}
\end{equation}
and $g$ satisfies \eqref{eq:g-L1}. The condition \eqref{eq:g-condition} can be
written in the form
\begin{equation}
  \int_{- \infty}^{\infty} \int_{- \infty}^{\infty} \frac{e^{- \beta
  |u|}}{\alpha + \gamma |u|} \cdot \frac{e^{- \beta |v|}}{\alpha + \gamma |v|}
  |R_{xx} (u, v) | m (du) m (dv) < \infty \label{eq:g-condition-example}
\end{equation}
and the output of the system with impulse response $h (t, \tau)$ to all input
processes which satisfy this condition is a harmonizable stochastic process.
In particular, if $x$ is harmonizable with
\begin{equation}
  x (t, \omega) = \int_{- \infty}^{\infty} e^{it \lambda} X (d \lambda,
  \omega) \label{eq:x-harmonize-example}
\end{equation}
then it follows by \eqref{eq:dY-dm-g} that
\begin{equation}
  \left[ \frac{dY}{dm} \right] (\tau, \omega) = (\beta + \gamma | \tau |) e^{-
  \alpha | \tau |}  \int_{- \infty}^{\infty} \frac{X (d \lambda,
  \omega)}{\lambda^2 + (\beta + \gamma | \tau |)^2} \label{eq:dY-dm-example}
\end{equation}
and the output $y$ has the harmonizable representation \eqref{eq:y-harmonize}
which can also be written in the form
\begin{equation}
  y (t, \omega) = \int_{- \infty}^{\infty} \left[ \int_{- \infty}^{\infty}
  \frac{\beta + \gamma | \tau |}{\lambda^2 + (\beta + \gamma | \tau |)^2} e^{-
  \alpha | \tau | + it \tau} m (d \tau) \right] X (d \lambda, \omega)
  \label{eq:y-full-example}
\end{equation}
This representation takes the following simple form in the particular case
where $\gamma = 0$, i.e.,
\begin{equation}
  h (t, \tau) = \frac{\alpha e^{- \beta | \tau |}}{\alpha^2 + t^2}
\end{equation}
\begin{equation}
  y (t, \omega) = \frac{2 \beta}{\alpha^2 + t^2}  \int_{- \infty}^{\infty}
  \frac{X (d \lambda, \omega)}{\beta^2 + \lambda^2}
  \label{eq:y-simple-example}
\end{equation}
\subsection{Linear Randomly Time Varying Systems}\label{subsec:random-systems}

Let the impulse response of a linear time varying system be a sample function
of a stochastic process of second order $h (t, \tau, \omega)$ with
autocorrelation function $R_{hh} (t, s ; \tau, \sigma)$. For all second-order
input processes $x (t, \omega)$ independent of $h$ and such that
\begin{equation}
  \int_{- \infty}^{\infty} \int_{- \infty}^{\infty} R_{hh} (t, t ; u, v)
  R_{xx} (u, v) m (du) m (dv) < \infty \quad \text{for all } \quad t \in
  \mathbb{R} \label{eq:Rhh-condition}
\end{equation}
the output of the system is the second-order stochastic process $y$ defined by
\begin{equation}
  y (t, \omega) = \int_{- \infty}^{\infty} h (t, u, \omega) x (u, \omega) m
  (du) \label{eq:y-random}
\end{equation}
A sufficient condition for \eqref{eq:Rhh-condition} to hold for all input
processes $x$ with uniformly bounded autocorrelation functions is clearly
$R_{hh} (t, t ; \cdot, \cdot) \in L_1  (m \times m)$ for all $t \in
\mathbb{R}$.

A set of sufficient conditions for the harmonizability of the output of a
linear randomly time varying system is given in the following theorem. Its
proof is similar to the proof of Theorem \ref{thm:linear-system-harmonizable}
and as such it is omitted.

\begin{theorem}
  \label{thm:random-system-harmonizable}If the impulse response $h (t, u,
  \omega)$ of a linear randomly time varying system is the Fourier transform
  in the stochastic mean sense of a second-order stochastic process $H (\rho,
  u, \omega)$, i.e.,
  \begin{equation}
    h (t, u, \omega) = \frac{1}{\sqrt{2 \pi}}  \int_{- \infty}^{\infty} H
    (\rho, u, \omega) e^{it \rho} m (d \rho) \label{eq:h-H-transform}
  \end{equation}
  which is such that $R_{HH} (\cdummy, \cdummy ; u, u) \in L_1  (m \times m)$
  for all $u \in \mathbb{R}$ and
  \begin{equation}
    \int_{- \infty}^{\infty} \int_{- \infty}^{\infty} \int_{- \infty}^{\infty}
    \int_{- \infty}^{\infty} |R_{HH} (\tau, \sigma ; u, v) ||R_{xx} (u, v) | m
    (d \tau) m (d \sigma) m (du) m (dv) < \infty \label{eq:RHH-condition}
  \end{equation}
  and if the input $x$ is independent of $h$, then the output stochastic
  process $y (t, \omega)$ is harmonizable.
\end{theorem}

We have
\begin{equation}
  y (t, \omega) = \int_{- \infty}^{\infty} e^{it \rho} Y (d \rho, \omega) =
  \int_{- \infty}^{\infty} e^{it \rho} \left[ \frac{dY}{dm} \right] (\rho,
  \omega) m (d \rho), \label{eq:y-harmonize-random}
\end{equation}
where
\begin{equation}
  \left[ \frac{dY}{dm} \right] (\rho, \omega) = \frac{1}{\sqrt{2 \pi}} 
  \int_{- \infty}^{\infty} H (\rho, u, \omega) x (u, \omega) m (du)
  \label{eq:dY-dm-H}
\end{equation}
A sufficient condition for \eqref{eq:RHH-condition} to hold for all input
processes $x$ with uniformly bounded autocorrelation functions is clearly
$R_{HH} \in L_1 (m^4)$.

\begin{thebibliography}{99}
  {\bibitem{campbell1969}}Campbell, L. L. (1969), Series expansions for random
  processes, in ``Proceedings of the International Symposium on Probability
  and Information Theory,'' Lecture Notes in Mathematics, No. 89, pp. 77-95,
  Springer, New York.
  
  {\bibitem{cramer1951}}Cramer, H. (1951), A contribution to the theory of
  stochastic processes, 2nd Berkeley Symp. Math. Stat. Probability, 329-339.
  
  {\bibitem{cramer1964}}Cramer, H. (1964), Stochastic processes as curves in
  Hilbert space, Theor. Probability Appl. 9, 169-179.
  
  {\bibitem{doob1953}}Doob, J. L. (1953), ``Stochastic Processes,'' Wiley, New
  York.
  
  {\bibitem{karhunen1947}}Karhunen, K. (1947), Uber lineare Methoden in der
  Wahrscheinlichkeitsrechnung, Ann. Acad. Sci. Fenn. Ser. Al No. 37, 1-79.
  
  {\bibitem{loeve1963}}Lo{\`e}ve, M. (1963), ``Probability Theory,'' Van
  Nostrand, Princeton, N. J.
  
  {\bibitem{masani1968}}Masani, P. (1968), Orthogonally scattered measures,
  Advan. Math. 2, 61-117.
  
  {\bibitem{masry1968}}Masry, E., Liu, B., And Steiglitz, K. (1968), Series
  expansion of wide-sense stationary random processes, IEEE Trans. IT-14,
  792-796.
  
  {\bibitem{parzen1967}}Parzen, E. (1967), Statistical inference on time
  series by Hilbert space methods, in ``Time series Analysis Papers,''
  Holden-Day, San Francisco, Calif.
  
  {\bibitem{piranashvili1967}}Piranashyili, Z. A. (1967), On the problem of
  interpolation of random processes, Theor. Probability Appl. 12, 647-657.
  
  {\bibitem{rao1967}}Rao, M. M. (1967), Inference in stochastic processes-III,
  Z. Wahrscheinlichkeitstheorie und verw. Gebiete 8, 49-72.
  
  {\bibitem{rosanov1959}}Rosanov, Y. A. (1959), Spectral analysis of abstract
  functions, Theor. Probability Appl. 4, 271-287.
\end{thebibliography}

\end{document}
