\documentclass{article}
\usepackage[english]{babel}
\usepackage{geometry,amsmath,amssymb,latexsym}
\geometry{letterpaper}

%%%%%%%%%% Start TeXmacs macros
\newcommand{\assign}{:=}
\newenvironment{proof}{\noindent\textbf{Proof\ }}{\hspace*{\fill}$\Box$\medskip}
\newtheorem{theorem}{Theorem}
%%%%%%%%%% End TeXmacs macros

\begin{document}

\

\

\begin{theorem}
  [Integral correlation length for the Bessel kernel] Let $\rho (t) = J_0  (2
  \pi t)$ for $t \in \mathbb{R}$, where $J_0$ denotes the Bessel function of
  the first kind of order $0$. Define the integral correlation length
  \begin{equation}
    \ell \hspace{0.27em} = \hspace{0.27em} \int_{- \infty}^{\infty} \rho (t) 
    \hspace{0.17em} dt \hspace{0.27em} = \hspace{0.27em} 2 \int_0^{\infty}
    \rho (t)  \hspace{0.17em} dt
  \end{equation}
  Then $\ell = 1$. Consequently, the corresponding characteristic frequency
  scale $f_c \assign 1 / \ell$ equals $1$ (in cycles per unit of the time
  variable used in $\rho$).
\end{theorem}

\begin{proof}
  By evenness of $J_0$, one has
  \begin{equation}
    \ell \hspace{0.27em} = \hspace{0.27em} 2 \int_0^{\infty} J_0  (2 \pi t) 
    \hspace{0.17em} dt
  \end{equation}
  Apply the change of variables $u = 2 \pi t$, so that $dt = \frac{du}{2 \pi}$
  and $t \in [0, \infty)$ corresponds to $u \in [0, \infty)$. This gives
  \begin{equation}
    \int_0^{\infty} J_0  (2 \pi t)  \hspace{0.17em} dt \hspace{0.27em} =
    \hspace{0.27em} \frac{1}{2 \pi}  \int_0^{\infty} J_0 (u)  \hspace{0.17em}
    du
  \end{equation}
  It is a classical integral identity that
  \begin{equation}
    \int_0^{\infty} J_0 (u)  \hspace{0.17em} du \hspace{0.27em} =
    \hspace{0.27em} 1
  \end{equation}
  Therefore,
  \begin{equation}
    \int_0^{\infty} J_0  (2 \pi t)  \hspace{0.17em} dt \hspace{0.27em} =
    \hspace{0.27em} \frac{1}{2 \pi}, \quad \text{and hence} \quad \ell
    \hspace{0.27em} = \hspace{0.27em} 2 \cdot \frac{1}{2 \pi} \hspace{0.27em}
    = \hspace{0.27em} 1
  \end{equation}
  Defining the characteristic frequency by $f_c \assign 1 / \ell$ (cycles per
  unit time) yields $f_c = 1$.
\end{proof}

\end{document}
