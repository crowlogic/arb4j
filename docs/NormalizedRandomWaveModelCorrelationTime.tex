\documentclass{article}
\usepackage{amsmath, amssymb, amsthm}

\begin{document}

\theoremstyle{plain}
\newtheorem{theorem}{Theorem}

\begin{theorem}
Let \( R(\tau) = J_0(|\tau|) \) for \( -\infty < \tau < \infty \), where \( J_0 \) is the Bessel function of the first kind of order zero. Then:
\begin{enumerate}
    \item \(
        \displaystyle
        \int_{0}^{\infty}J_{0}(\tau)\,d\tau = 1
        \).
    \item The correlation time is
        \[
            \tau_{c}
            =\frac{\displaystyle\int_{-\infty}^{\infty}R(\tau)\,d\tau}{R(0)}
            =2\!\int_{0}^{\infty}J_{0}(\tau)\,d\tau
            =2.
        \]
    \item The Fourier transform satisfies
        \[
            S(\omega)=\int_{-\infty}^{\infty}\!e^{-i\omega\tau}J_{0}(|\tau|)\,d\tau
            =\frac{2}{\sqrt{1-\omega^{2}}}\;\mathbf 1_{|\omega|<1},
        \]
        so the spectrum is strictly band-limited to \( |\omega| < 1 \). The support width is \(2\), matching the correlation time.
    \item Time–bandwidth identity:
        \[
            \tau_{c}\,\Delta\omega = 2\times1=2,
        \]
        so the kernel saturates the exact time–bandwidth product on \(\mathbb{R}\).
\end{enumerate}
\end{theorem}

\begin{proof}
Item (1) follows from the integral property of the Bessel function of the first kind: \( \int_{0}^{\infty}J_{0}(\tau)\,d\tau = 1 \).

Item (2) uses the fact that \( R(0)=J_{0}(0)=1 \) and the evenness of the kernel:
\[
\tau_{c}
   =\frac{\displaystyle\int_{-\infty}^{\infty}R(\tau)\,d\tau}{R(0)}
   =2\!\int_{0}^{\infty}J_{0}(\tau)\,d\tau
   =2.
\]

Item (3) is the classical cosine–Fourier transform result:
\[
S(\omega)=\int_{-\infty}^{\infty}\!e^{-i\omega\tau}J_{0}(|\tau|)\,d\tau
         =\frac{2}{\sqrt{1-\omega^{2}}}\;\mathbf 1_{|\omega|<1},
\]
which vanishes identically for \( |\omega|\ge 1 \), so the spectrum is strictly band-limited to \( [-1,1] \); the support width is \(2\).

Item (4) computes the time–bandwidth product of the kernel,
\[
\tau_{c}\,\Delta\omega = 2\times1=2,
\]
with \( \tau_c=2 \) and half-width \( \Delta\omega=1 \).
\end{proof}

\end{document}
