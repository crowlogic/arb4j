\title{Eigenfunction Expansion of Integral Operators}

\author{Claude}

\date{}

\maketitle

\begin{theorem}
Let $K(\alpha, \gamma)$ be a symmetric kernel defined on $[0, \infty) \times [0, \infty)$, and let $\{\phi_n(\alpha)\}_{n=0}^\infty$ be the set of orthonormal eigenfunctions of the integral equation
\begin{equation}
  \phi(\alpha) = \lambda \int_0^\infty K(\alpha, \xi) \phi(\xi) \hspace{0.17em}
  \mathrm{d} \xi
\end{equation}
with corresponding eigenvalues $\{\lambda_n\}_{n=0}^\infty$. The eigenfunctions satisfy the orthonormality condition:
\begin{equation}
  \int_0^\infty \phi_n(\alpha) \phi_m(\alpha) \hspace{0.17em} \mathrm{d}\alpha = \delta_{nm}
\end{equation}
where $\delta_{nm}$ is the Kronecker delta. Then, if the series
\begin{equation}
  \sum_{n=0}^{\infty} \frac{\phi_n(\alpha) \phi_n(\gamma)}{\lambda_n}
\end{equation}
is uniformly convergent for $0 \leq \alpha, \gamma < \infty$, we have
\begin{equation}
  K(\alpha, \gamma) = \sum_{n=0}^{\infty} \frac{\phi_n(\alpha) \phi_n(\gamma)}{\lambda_n}
\end{equation}
\end{theorem}

\begin{proof}
Let
\begin{equation}
  S(\alpha, \gamma) = \sum_{n=0}^{\infty} \frac{\phi_n(\alpha) \phi_n(\gamma)}{\lambda_n}
\end{equation}

Consider the action of $S(\alpha, \gamma)$ on an eigenfunction $\phi_m(\gamma)$:

\begin{align}
  \int_0^\infty S(\alpha, \gamma) \phi_m(\gamma) \hspace{0.17em} \mathrm{d}\gamma 
  &= \int_0^\infty \sum_{n=0}^{\infty} \frac{\phi_n(\alpha) \phi_n(\gamma)}{\lambda_n} \phi_m(\gamma) \hspace{0.17em} \mathrm{d}\gamma \\
  &= \sum_{n=0}^{\infty} \frac{\phi_n(\alpha)}{\lambda_n} \int_0^\infty \phi_n(\gamma) \phi_m(\gamma) \hspace{0.17em} \mathrm{d}\gamma \\
  &= \sum_{n=0}^{\infty} \frac{\phi_n(\alpha)}{\lambda_n} \delta_{nm} \\
  &= \frac{\phi_m(\alpha)}{\lambda_m}
\end{align}

The interchange of summation and integration is justified by the uniform convergence of the series. For the eigenfunction $\phi_m(\alpha)$:

\begin{equation}
  \phi_m(\alpha) = \lambda_m \int_0^\infty K(\alpha, \gamma) \phi_m(\gamma) \hspace{0.17em} \mathrm{d}\gamma
\end{equation}

Comparing this with our result for $S(\alpha, \gamma)$, we see that

\begin{equation}
  \int_0^\infty S(\alpha, \gamma) \phi_m(\gamma) \hspace{0.17em} \mathrm{d}\gamma = \int_0^\infty K(\alpha, \gamma) \phi_m(\gamma) \hspace{0.17em} \mathrm{d}\gamma
\end{equation}

for all eigenfunctions $\phi_m(\alpha)$. For any square-integrable function $f(\alpha)$:

\begin{equation}
  f(\alpha) = \sum_{m=0}^{\infty} c_m \phi_m(\alpha)
\end{equation}

where $c_m = \int_0^\infty f(\gamma) \phi_m(\gamma) \hspace{0.17em} \mathrm{d}\gamma$. Then:

\begin{align}
  \int_0^\infty S(\alpha, \gamma) f(\gamma) \hspace{0.17em} \mathrm{d}\gamma 
  &= \int_0^\infty S(\alpha, \gamma) \sum_{m=0}^{\infty} c_m \phi_m(\gamma) \hspace{0.17em} \mathrm{d}\gamma \\
  &= \sum_{m=0}^{\infty} c_m \int_0^\infty S(\alpha, \gamma) \phi_m(\gamma) \hspace{0.17em} \mathrm{d}\gamma \\
  &= \sum_{m=0}^{\infty} c_m \int_0^\infty K(\alpha, \gamma) \phi_m(\gamma) \hspace{0.17em} \mathrm{d}\gamma \\
  &= \int_0^\infty K(\alpha, \gamma) \sum_{m=0}^{\infty} c_m \phi_m(\gamma) \hspace{0.17em} \mathrm{d}\gamma \\
  &= \int_0^\infty K(\alpha, \gamma) f(\gamma) \hspace{0.17em} \mathrm{d}\gamma
\end{align}

Since this equality holds for all square-integrable functions $f(\alpha)$, we conclude that

\begin{equation}
  S(\alpha, \gamma) = K(\alpha, \gamma)
\end{equation}

To prove uniqueness, suppose there exists another expansion

\begin{equation}
  K(\alpha, \gamma) = \sum_{n=0}^{\infty} \frac{\phi_n(\alpha) \phi_n(\gamma)}{\lambda_n} + H(\alpha, \gamma)
\end{equation}

where $H(\alpha, \gamma)$ is a non-zero symmetric function. Then for any eigenfunction $\phi_m(\alpha)$:

\begin{equation}
  \int_0^\infty H(\alpha, \gamma) \phi_m(\gamma) \hspace{0.17em} \mathrm{d}\gamma = 0
\end{equation}

This implies $H(\alpha, \gamma)$ must be identically zero, contradicting our assumption. Therefore, the expansion is unique.

For a specific eigenfunction $\phi_k(\alpha)$:

\begin{equation}
  \phi_k(\alpha) = \lambda_k \frac{\phi_k(\alpha)}{\lambda_k} \int_0^\infty [\phi_k(\xi)]^2 \hspace{0.17em} \mathrm{d}\xi
\end{equation}

This leads to the normalization condition: $\int_0^\infty [\phi_k(\xi)]^2 \hspace{0.17em} \mathrm{d}\xi = 1$.
\end{proof}

This theorem provides a foundation for analyzing symmetric kernels and their associated integral equations, with applications in spectral theory and functional analysis.
