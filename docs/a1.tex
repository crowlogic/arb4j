\documentclass[12pt]{article}

\usepackage{amsmath,amsthm,amssymb,amsfonts}
\usepackage{mathtools}
\usepackage{mathrsfs}
\usepackage[T1]{fontenc}
\usepackage[utf8]{inputenc}
\usepackage{microtype}
\usepackage{geometry}
\geometry{margin=1in}

% Environments
\newtheorem{theorem}{Theorem}
\newtheorem{lemma}{Lemma}
\newtheorem{claim}{Claim}
\newtheorem{proposition}{Proposition}
\newtheorem{corollary}{Corollary}
\theoremstyle{definition}
\newtheorem{definition}{Definition}
\theoremstyle{remark}
\newtheorem{remark}{Remark}

% Notation helpers
\DeclareMathOperator{\supp}{supp}
\DeclareMathOperator{\Var}{Var}
\DeclareMathOperator{\Cov}{Cov}
\DeclareMathOperator{\sgn}{sgn}

\title{Unitary Time Changes of Stationary Processes Yield Oscillatory Processes\\
and a Functional Framework Toward a Hilbert--P\'olya Construction}
\author{Stephen Crowley}
\date{\today}

\begin{document}
\maketitle

\section{Unitary Time Change on $L^2(\mathbb{R})$}

\begin{definition}[Unitary time change operator on $L^2(\mathbb{R})$]
Let $\theta:\mathbb{R}\to\mathbb{R}$ be absolutely continuous with $\theta'(t)\neq 0$ almost everywhere.
Define $U_\theta:L^2(\mathbb{R})\to L^2(\mathbb{R})$ by
\begin{equation}
  (U_\theta f)(t)\coloneqq \sqrt{|\theta'(t)|}\, f(\theta(t))\qquad (f\in L^2(\mathbb{R})).
\end{equation}
\end{definition}

\begin{theorem}[Unitarity of $U_\theta$]
$U_\theta$ is unitary on $L^2(\mathbb{R})$.
\end{theorem}

\begin{proof}
By absolute continuity and $\theta'(t)\neq 0$ a.e., the change-of-variables formula gives
\[
\int_{\mathbb{R}} |(U_\theta f)(t)|^2\,dt
= \int_{\mathbb{R}} |\theta'(t)|\,|f(\theta(t))|^2\,dt
= \int_{\mathbb{R}} |f(u)|^2\,du,
\]
so $U_\theta$ is an isometry. Since $\theta$ is a.e. strictly monotone on measurable pieces, it admits an a.e.\ inverse $\theta^{-1}$ that is absolutely continuous and satisfies the same hypotheses, hence $U_{\theta^{-1}}$ is well-defined and isometric. One checks directly $U_{\theta^{-1}}U_\theta=\mathrm{Id}$ and $U_\theta U_{\theta^{-1}}=\mathrm{Id}$ a.e., therefore $U_\theta$ is onto and hence unitary.
\end{proof}

\section{Oscillatory Processes in the Sense of Priestley}

\begin{definition}[Oscillatory process, gain and oscillatory function]
Let $F$ be a finite nonnegative Borel measure on $\mathbb{R}$. For each $t\in\mathbb{R}$ let $A_t:\mathbb{R}\to\mathbb{C}$ be measurable and square-integrable with respect to $F$.
Define
\[
\varphi_t(\lambda)\coloneqq A_t(\lambda)\,e^{i\lambda t}.
\]
An \emph{oscillatory process} $Z$ is a stochastic process with spectral representation
\begin{equation}
  Z(t) \coloneqq \int_{\mathbb{R}} \varphi_t(\lambda)\,\Phi(d\lambda)
  = \int_{\mathbb{R}} A_t(\lambda)\,e^{i\lambda t}\,\Phi(d\lambda),
\end{equation}
where $\Phi$ is a complex orthogonal random measure with spectral measure $F$ satisfying, in the sense of orthogonality of infinitesimal increments,
\begin{equation}
  \mathbb{E}\!\left[\Phi(d\lambda)\,\overline{\Phi(d\mu)}\right] = \delta(\lambda-\mu)\,dF(\lambda).
\end{equation}
The covariance kernel is then
\begin{equation}
  R_Z(t,s)\coloneqq \mathbb{E}\big[Z(t)\,\overline{Z(s)}\big]
  = \int_{\mathbb{R}} A_t(\lambda)\,\overline{A_s(\lambda)}\,e^{i\lambda(t-s)}\,dF(\lambda).
\end{equation}
\end{definition}

\begin{remark}[Real-valuedness]
$Z$ is real-valued if and only if, for each fixed $t$, $A_t(-\lambda)=\overline{A_t(\lambda)}$ for $F$-a.e.\ $\lambda$, equivalently $\varphi_t(-\lambda)=\overline{\varphi_t(\lambda)}$ for $F$-a.e.\ $\lambda$.
\end{remark}

\begin{theorem}[Existence of oscillatory processes with prescribed $(A_t)_t$]
Let $F$ be finite and $(A_t)_t$ measurable with $\int |A_t(\lambda)|^2\,dF(\lambda)<\infty$ for each $t$. There exists a complex orthogonal random measure $\Phi$ on $\mathbb{R}$ with spectral measure $F$ such that $Z(t)=\int \varphi_t(\lambda)\,\Phi(d\lambda)$ is well-defined in $L^2(\Omega)$ and has covariance
\[
R_Z(t,s)=\int_{\mathbb{R}} \varphi_t(\lambda)\,\overline{\varphi_s(\lambda)}\,dF(\lambda)
=\int_{\mathbb{R}} A_t(\lambda)\,\overline{A_s(\lambda)}\,e^{i\lambda(t-s)}\,dF(\lambda).
\]
\end{theorem}

\begin{proof}
Standard isonormal construction: define the stochastic integral first for simple functions in $L^2(\mathbb{R},F)$ and extend by isometry, using $\mathbb{E}\big[\big|\int g(\lambda)\,\Phi(d\lambda)\big|^2\big]=\int |g(\lambda)|^2\,dF(\lambda)$. Apply with $g=\varphi_t$.
\end{proof}

\section{Unitary Time Changes Map Stationary to Oscillatory}

\begin{definition}[Stationary process via Cram\'er representation]
A zero-mean stationary process $X$ with spectral measure $F$ admits
\begin{equation}
  X(t) = \int_{\mathbb{R}} e^{i\lambda t}\,\Phi(d\lambda),
\end{equation}
with $\Phi, F$ as above, and covariance
\begin{equation}
  R_X(t-s)=\int_{\mathbb{R}} e^{i\lambda(t-s)}\,dF(\lambda).
\end{equation}
\end{definition}

\begin{theorem}[Unitary time change yields an oscillatory process]
Let $X$ be zero-mean stationary with
\[
X(t)=\int_{\mathbb{R}} e^{i\lambda t}\,\Phi(d\lambda).
\]
Let $\theta$ satisfy the hypotheses of the unitary time change and set
\begin{equation}
  Z(t)\coloneqq (U_\theta X)(t)=\sqrt{|\theta'(t)|}\,X(\theta(t)).
\end{equation}
Then $Z$ is an oscillatory process with oscillatory function
\begin{equation}
  \varphi_t(\lambda)=\sqrt{|\theta'(t)|}\,e^{i\lambda \theta(t)},
\end{equation}
and gain
\begin{equation}
  A_t(\lambda)=\sqrt{|\theta'(t)|}\,e^{i\lambda(\theta(t)-t)}.
\end{equation}
The covariance is
\begin{equation}
  R_Z(t,s)=\int_{\mathbb{R}} A_t(\lambda)\,\overline{A_s(\lambda)}\,e^{i\lambda(t-s)}\,dF(\lambda)
  = \int_{\mathbb{R}} \sqrt{|\theta'(t)\theta'(s)|}\,e^{i\lambda(\theta(t)-\theta(s))}\,dF(\lambda).
\end{equation}
\end{theorem}

\begin{proof}
Compute
\[
Z(t)=\sqrt{|\theta'(t)|}\,X(\theta(t))
=\sqrt{|\theta'(t)|}\int_{\mathbb{R}} e^{i\lambda \theta(t)}\,\Phi(d\lambda)
=\int_{\mathbb{R}} \sqrt{|\theta'(t)|}\,e^{i\lambda \theta(t)}\,\Phi(d\lambda).
\]
Thus $\varphi_t(\lambda)=\sqrt{|\theta'(t)|}\,e^{i\lambda \theta(t)}$. Since $\varphi_t=A_t e^{i\lambda t}$, solve
\[
A_t(\lambda)=\frac{\varphi_t(\lambda)}{e^{i\lambda t}}
=\sqrt{|\theta'(t)|}\,e^{i\lambda(\theta(t)-t)}.
\]
The covariance is obtained by orthogonality of $\Phi$:
\[
R_Z(t,s)=\int \varphi_t(\lambda)\,\overline{\varphi_s(\lambda)}\,dF(\lambda)
=\int \sqrt{|\theta'(t)\theta'(s)|}\,e^{i\lambda(\theta(t)-\theta(s))}\,dF(\lambda).
\]
Equivalently, substituting $\varphi_t=A_t e^{i\lambda t}$ yields the alternative form.
\end{proof}

\begin{remark}[Real-valuedness under time change]
If $X$ is real-valued, then $\Phi$ satisfies the usual Hermitian symmetry. The condition $A_t(-\lambda)=\overline{A_t(\lambda)}$ holds provided $\theta$ is real-valued and $\theta'(t)>0$ a.e., ensuring $Z$ is real-valued.
\end{remark}

\section{Zero Localization by a Functional Measure}

\begin{definition}[Zero localization measure]
Let $Z$ be a real-valued process with $C^1$ paths and such that $\mathbb{P}(Z(t)=0\Rightarrow Z'(t)=0)=0$ (by Bulinskaya-type non-tangency). Define the measure on Borel $B\subset\mathbb{R}$ by
\begin{equation}
  \mu(B)\coloneqq \int_{\mathbb{R}} \mathbf{1}_B(t)\,\delta(Z(t))\,|Z'(t)|\,dt.
\end{equation}
\end{definition}

\begin{theorem}[Support and mass on the zero set]
Under the stated regularity, for any test function $\phi\in C_c^\infty(\mathbb{R})$,
\begin{equation}
  \int_{\mathbb{R}} \phi(t)\,\delta(Z(t))\,|Z'(t)|\,dt
  = \sum_{t: Z(t)=0} \phi(t),
\end{equation}
and hence $\mu$ is a discrete measure with unit atoms at the simple zeros of $Z$.
\end{theorem}

\begin{proof}
Apply the change-of-variables formula for distributions to $Z$ at each simple zero $t_0$:
\[
\delta(Z(t))=\sum_{t_0: Z(t_0)=0}\frac{\delta(t-t_0)}{|Z'(t_0)|}.
\]
Multiplying by $|Z'(t)|$ and integrating against $\phi$ yields the identity and the discrete form of $\mu$.
\end{proof}

\section{Hilbert Space on the Zero Set and Multiplication Operator}

\begin{definition}[Hilbert space on the zero set via $\mu$]
Define
\[
\mathcal{H}\coloneqq L^2(\mu)
=\Big\{ f:\mathbb{R}\to\mathbb{C}\,:\, \|f\|_{\mathcal{H}}^2=\int |f(t)|^2\,\delta(Z(t))\,|Z'(t)|\,dt<\infty\Big\}.
\]
The inner product is $\langle f,g\rangle=\int f(t)\overline{g(t)}\,\delta(Z(t))\,|Z'(t)|\,dt$.
\end{definition}

\begin{proposition}[Atomic structure]
$\mu=\sum_{n} \delta_{t_n}$ where $\{t_n\}$ are the simple zeros of $Z$. Consequently,
\[
\mathcal{H}=\Big\{ f:\{t_n\}\to\mathbb{C}\,:\, \sum_n |f(t_n)|^2<\infty\Big\}\cong \ell^2,
\]
and the functions $e_n$ defined by $e_n(t_m)=\delta_{nm}$ form an orthonormal basis.
\end{proposition}

\begin{proof}
Immediate from the previous theorem and the definition of $L^2(\mu)$.
\end{proof}

\begin{definition}[Multiplication operator]
Define $L:\mathcal{D}(L)\subset\mathcal{H}\to\mathcal{H}$ by $(Lf)(t)=t\,f(t)$ on the support of $\mu$, with
\[
\mathcal{D}(L)=\Big\{ f\in\mathcal{H}\,:\, \int |t\,f(t)|^2\,\delta(Z(t))\,|Z'(t)|\,dt<\infty\Big\}.
\]
\end{definition}

\begin{theorem}[Self-adjointness and spectrum]
$L$ is self-adjoint on $\mathcal{H}$, and its spectrum is
\[
\sigma(L)=\{\, t\in\mathbb{R}\,:\, Z(t)=0 \,\},
\]
with pure point spectrum consisting of simple eigenvalues $\lambda_n=t_n$ and eigenvectors $e_n$.
\end{theorem}

\begin{proof}
For $f,g\in\mathcal{D}(L)$,
\[
\langle Lf,g\rangle = \int t\,f(t)\,\overline{g(t)}\,\delta(Z(t))\,|Z'(t)|\,dt
= \int f(t)\,\overline{t\,g(t)}\,\delta(Z(t))\,|Z'(t)|\,dt = \langle f,Lg\rangle,
\]
so $L$ is symmetric. On the atomic space, $L$ is unitarily equivalent to the diagonal operator $(c_n)\mapsto (t_n c_n)$ on $\ell^2$, which is self-adjoint with spectrum equal to $\{t_n\}$, each of multiplicity one, and eigenvectors the canonical basis, corresponding here to $e_n$.
\end{proof}

\section{Hilbert--P\'olya Framework via Time-Changed Stationary Processes}

\begin{definition}[Time-change aligned to an arithmetic phase]
Let $\theta:\mathbb{R}\to\mathbb{R}$ be an absolutely continuous phase function (e.g.\ a Riemann--Siegel-type phase) with $\theta'(t)>0$ a.e. Let $X$ be zero-mean stationary with spectral measure $F$. Define the time-changed oscillatory process
\[
Z(t)=\int_{\mathbb{R}} \sqrt{|\theta'(t)|}\,e^{i\lambda \theta(t)}\,\Phi(d\lambda).
\]
\end{definition}

\begin{proposition}[Covariance under time-change]
$R_Z(t,s)=\int \sqrt{|\theta'(t)\theta'(s)|}\,e^{i\lambda(\theta(t)-\theta(s))}\,dF(\lambda)$. In particular, if $F$ is chosen so that $R_Z$ has oscillatory correlation concentrated near the level sets $\theta(t)=\theta(s)$, then $Z$ exhibits zero-crossings modulated by the phase $\theta$.
\end{proposition}

\begin{proof}
Substitute the oscillatory function and integrate using the orthogonality of $\Phi$.
\end{proof}

\begin{definition}[Zero-based functional localization and operator]
With the zero localization measure $\mu(dt)=\delta(Z(t))\,|Z'(t)|\,dt$, define $\mathcal{H}=L^2(\mu)$ and $L$ as the multiplication operator $(Lf)(t)=t\,f(t)$ on $\mathcal{H}$.
\end{definition}

\begin{theorem}[Spectral encoding of zero set]
The pure point spectrum of $L$ equals the zero set of $Z$:
\[
\sigma(L)=\{t: Z(t)=0\},
\]
and the eigenbasis is indexed by simple zeros. If $Z(t)=e^{i\theta(t)}\Xi(t)$ for a target complex function $\Xi$ whose real/imaginary parts the construction aims to annul at a prescribed set, then the operator $L$ diagonalizes on those locations.
\end{theorem}

\begin{proof}
Follows from the previous self-adjointness and the atomic form of $\mu$.
\end{proof}

\begin{remark}[Programmatic path to Hilbert--P\'olya]
The pipeline is:
\begin{itemize}
\item Start from a stationary Gaussian $X$ with chosen spectral measure $F$.
\item Apply the unitary time change $U_\theta$ to obtain an oscillatory process $Z$ with oscillatory function $\sqrt{|\theta'(t)|}\,e^{i\lambda \theta(t)}$.
\item Localize the zero set of $Z$ via $\mu(dt)=\delta(Z(t))\,|Z'(t)|\,dt$ to build $\mathcal{H}=L^2(\mu)$.
\item Act by the multiplication operator $L$; its spectrum is the zero set.
\end{itemize}
By choosing $\theta$ and $F$ to reflect arithmetic symmetries, the spectral set can be molded to align with a prescribed collection of ordinates, providing a concrete operator-theoretic scaffold toward a Hilbert--P\'olya-type correspondence.
\end{remark}

\section{Technical Regularity and Non-Tangency}

\begin{definition}[Regularity hypothesis]
Assume $Z$ has sample paths in $C^1(\mathbb{R})$ and satisfies the non-tangency condition: $\mathbb{P}(Z(t)=0 \Rightarrow Z'(t)=0)=0$ for each fixed $t$.
\end{definition}

\begin{lemma}[Non-tangency implies simple zeros]
Under the regularity hypothesis, all zeros of $Z$ are simple, and the zero set is locally finite.
\end{lemma}

\begin{proof}
By continuity and non-tangency, zeros cannot accumulate in compact sets and cannot be of higher order. Standard arguments (e.g.\ implicit function theorem in random setting) yield local finiteness and simplicity.
\end{proof>

\begin{remark}[Induced real-valuedness]
If $X$ is real-valued stationary and $\theta$ is real with $\theta'(t)>0$ a.e., then $Z$ is real-valued by the symmetry $A_t(-\lambda)=\overline{A_t(\lambda)}$, ensuring the zero localization measure is well-defined as above.
\end{remark}

\end{document}

