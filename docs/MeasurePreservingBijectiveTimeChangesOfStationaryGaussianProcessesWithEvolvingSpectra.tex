\documentclass{article}
\usepackage[english]{babel}
\usepackage{geometry,amsmath,amssymb,latexsym,theorem,hyperref}
\geometry{letterpaper}

%%%%%%%%%% Start TeXmacs macros
\newcommand{\cdummy}{\cdot}
\newcommand{\mathd}{\mathrm{d}}
\newcommand{\tmem}[1]{{\em #1\/}}
\newcommand{\tmemail}[1]{\\ \textit{Email:} \texttt{#1}}
\newcommand{\tmop}[1]{\ensuremath{\operatorname{#1}}}
\newenvironment{proof}{\noindent\textbf{Proof\ }}{\hspace*{\fill}$\Box$\medskip}
\newtheorem{corollary}{Corollary}
\newtheorem{definition}{Definition}
\newtheorem{lemma}{Lemma}
{\theorembodyfont{\rmfamily}\newtheorem{remark}{Remark}}
\newtheorem{theorem}{Theorem}
%%%%%%%%%% End TeXmacs macros

\begin{document}

\title{Measure-Preserving Bijective Time Changes of Stationary Gaussian Processes Generates A Subclass of Oscillatory Processes}

\author{
  Stephen Crowley
  \tmemail{stephencrowley214@gmail.com}
}

\date{August 1, 2025}

\maketitle

\begin{abstract}
  This article establishes that Gaussian processes obtained through
  measure-preserving bijective unitary time transformations of stationary
  processes constitute a subclass of oscillatory processes in the sense of
  Priestley{\cite{priestley1965}}. The transformation $Z (t) = \sqrt{\dot{\theta} (t)} X (\theta
  (t))$, where $X (t)$ is a realization of stationary Gaussian process and
  $\theta$ is a strictly increasing $C^1$ differentiable monotonic function,
  yields an oscillatory process with evolutionary power spectrum $dF_t
  (\omega) = \dot{\theta} (t) d \mu (\omega)$. An explicit unitary
  transformation between the input stationary process and the transformed
  oscillatory process is established, preserving the $L^2$-norm and providing
  a complete spectral characterization.
\end{abstract}

{\tableofcontents}

\section{Scaling Functions}\label{sec:scaling}

\begin{definition}
  [Scaling Functions]\label{def:scaling} Let $\mathcal{F}$ denote the set of
  functions $\theta : \mathbb{R} \to \mathbb{R}$ satisfying
  \begin{enumerate}
    \item $\theta$ is absolutely continuous with
    \begin{equation}
      \dot{\theta} (t) = \frac{\mathd}{\mathd t} \theta (t) \geq 0
    \end{equation}
    almost everywhere and $\dot{\theta} (t) = 0$ only on sets of Lebesgue
    measure zero
    
    \item $\theta$ is strictly increasing and bijective.
  \end{enumerate}
\end{definition}

\begin{remark}
  \label{rem:inverse_properties}The conditions in Definition~\ref{def:scaling}
  ensure that $\theta^{- 1}$ exists and is absolutely continuous. By the
  inverse function theorem for absolutely continuous functions,
  \begin{equation}
    \frac{\mathd}{\mathd s} (\theta^{- 1}) (s) = \frac{1}{\dot{\theta}
    (\theta^{- 1} (s))}
  \end{equation}
  for almost all $s$ in the range of $\theta$. The condition that
  $\dot{\theta} (t) = 0$ only on sets of measure zero ensures that
  $\frac{1}{\dot{\theta} (\theta^{- 1} (s))}$ is well-defined almost
  everywhere.
\end{remark}

\section{Oscillatory Processes}\label{sec:oscillatory}

\begin{definition}
  [Oscillatory Process]\label{def:oscillatory} A complex-valued, second-order
  process $\{X (t) \}_{t \in \mathbb{R}}$ is called {\tmem{oscillatory}} if
  there exist
  \begin{enumerate}
    \item a family of oscillatory basis functions $\{\phi_t (\omega)\}_{t \in
    \mathbb{R}}$ with
    \begin{equation}
      \phi_t (\omega) = A_t (\omega) e^{i \omega t}
    \end{equation}
    and a given gain function
    \begin{equation}
      A_t (\cdummy) \in L^2 (\mu) \label{envelope}
    \end{equation}
    \item and a complex orthogonal random measure $\Phi (\omega)$ with
    \begin{equation}
      E \lvert d \Phi (\omega) \rvert^2 = d \mu (\omega) = S (\omega)
    \end{equation}
  \end{enumerate}
  such that
  \begin{equation}
    \label{eq:oscillatory_rep} \begin{array}{ll}
      Z (t) & = \int_{- \infty}^{\infty} \phi_t (\omega)  \hspace{0.17em} d
      \Phi (\omega)\\
      & = \int_{- \infty}^{\infty} A_t (\omega) e^{i \omega t} d \Phi
      (\omega)
    \end{array}
  \end{equation}
  All stationary processes are oscillatory with $A_t (\omega) = 1$
\end{definition}

\section{Stationary Reference Process}\label{sec:stationary}

Let $\{X (t) \}_{t \in \mathbb{R}}$ be a stationary Gaussian process with
continuous spectral representation
\begin{equation}
  \label{eq:stationary_rep} X (t) = \int_{- \infty}^{\infty} e^{i \omega t} 
  \hspace{0.17em} d \Phi (\omega)
\end{equation}
where $\Phi (\omega)$ is an orthogonal-increment process with spectral density
\begin{equation}
  E \lvert d \Phi (\omega) \rvert^2 = d \mu (\omega) = S (\omega) = <
  \tmop{fourier} \tmop{transform} \tmop{of} K_X >
\end{equation}
and $\mu$ is a finite measure on $\mathbb{R}$.

\section{Time-Changed Process}\label{sec:time_change}

\subsection{Definition and Unitary Operator}

\begin{definition}
  [Unitary Time-Change Operator]\label{def:unitary_op}For $\theta \in
  \mathcal{F}$, define the operator $M_{\theta} : L^2 (\mathbb{R}) \to L^2
  (\mathbb{R})$ by
  \begin{equation}
    \label{eq:unitary_op} (M_{\theta} f) (t) = \sqrt{\dot{\theta} (t)} 
    \hspace{0.17em} f (\theta (t))
  \end{equation}
\end{definition}

\begin{definition}
  [Unitarily Time-Changed Stationary Process]\label{def:time_changed_proc} For
  $\theta \in \mathcal{F}$, apply the unitary time change operator
  $M_{\theta}$ from Definition-\ref{def:unitary_op} \ to a realization of a
  stationary process $X (t)$ from the ensemble $\{ X (t) \}$ to define a
  realization of the unitarily time-changed process
  \begin{equation}
    \label{eq:time_change} Z (t) = \sqrt{\dot{\theta} (t)}  \hspace{0.17em} X
    (\theta (t))  \forall t \in \mathbb{R}
  \end{equation}
\end{definition}

\

\begin{definition}
  [Inverse Unitary Time-Change Operator]\label{def:inverse_unitary_op} The
  inverse operator $M_{\theta}^{- 1} : L^2 (\mathbb{R}) \to L^2 (\mathbb{R})$
  corresponding to the unitary time-change operator $(M_{\theta} f) (t)$
  defined in Equation-\ref{eq:unitary_op} is given by
  \begin{equation}
    \label{eq:unitary_inverse} (M_{\theta}^{- 1} g) (s) = \frac{g (\theta^{-
    1} (s))}{\sqrt{\dot{\theta} (\theta^{- 1} (s))}}
  \end{equation}
\end{definition}

\begin{lemma}
  [Well-Definedness of Inverse Operator]\label{lem:inverse_well_defined} The
  operator $M_{\theta}^{- 1}$ in Definition~\ref{def:inverse_unitary_op} is
  well-defined $\forall \theta \in \mathcal{F}$.
\end{lemma}

\begin{proof}
  Since $\dot{\theta} (t) = 0$ only on sets of measure zero by
  Definition~\ref{def:scaling}, and $\theta^{- 1}$ maps sets of measure zero
  to sets of measure zero (as it preserves absolute continuity), the
  denominator $\sqrt{\dot{\theta} (\theta^{- 1} (s))}$ is positive almost
  everywhere. The expression in equation~\eqref{eq:unitary_inverse} is
  therefore well-defined almost everywhere, which is sufficient for defining
  an element of $L^2 (\mathbb{R})$.
\end{proof}

\begin{theorem}
  [Unitarity of Transformation Operator]\label{thm:unitary} The operator
  $M_{\theta}$ defined in equation~\eqref{eq:unitary_op} is unitary, i.e.,
  \begin{equation}
    \label{eq:L2_preserve} \int_{\mathbb{R}} \lvert (M_{\theta} f) (t)
    \rvert^2  \hspace{0.17em} dt = \int_{\mathbb{R}} \lvert f (s) \rvert^2 
    \hspace{0.17em} ds \forall f \in L^2 (\mathbb{R})
  \end{equation}
\end{theorem}

\begin{proof}
  Let $f \in L^2 (\mathbb{R})$. The $L^2$-norm of $M_{\theta} f$ is computed
  as follows:
  
  \begin{align}
    \int_{\mathbb{R}} \lvert (M_{\theta} f) (t) \rvert^2  \hspace{0.17em} dt &
    = \int_{\mathbb{R}} \left| \sqrt{\dot{\theta} (t)}  \hspace{0.17em} f
    (\theta (t)) \right|^2  \hspace{0.17em} dt \\
    & = \int_{\mathbb{R}} \dot{\theta} (t) \hspace{0.17em} \lvert f (\theta
    (t)) \rvert^2  \hspace{0.17em} dt 
  \end{align}
  
  Apply the change of variables $s = \theta (t)$. Since $\theta$ is absolutely
  continuous and strictly increasing, its Jacobian is given by
  \begin{equation}
    ds = \dot{\theta} (t)  \hspace{0.17em} dt
  \end{equation}
  almost everywhere. As $t$ ranges over $\mathbb{R}$, $s = \theta (t)$ ranges
  over $\mathbb{R}$ due to the bijectivity of $\theta$. Therefore:
  
  \begin{align}
    \int_{\mathbb{R}} \dot{\theta} (t) \hspace{0.17em} \lvert f (\theta (t))
    \rvert^2  \hspace{0.17em} dt & = \int_{\mathbb{R}} \lvert f (s) \rvert^2 
    \hspace{0.17em} ds 
  \end{align}
  
  This establishes equation~\eqref{eq:L2_preserve}. To complete the proof of
  unitarity, it remains to show that $M_{\theta}^{- 1}$ is indeed the inverse
  of $M_{\theta}$. For any $f \in L^2 (\mathbb{R})$:
  
  \begin{align}
    (M_{\theta}^{- 1} M_{\theta} f) (s) & = (M_{\theta}^{- 1})  \left[
    \sqrt{\dot{\theta} (\cdummy)}  \hspace{0.17em} f (\theta (\cdot)) \right]
    (s) \\
    & = \frac{\sqrt{\dot{\theta} (\theta^{- 1} (s))}  \hspace{0.17em} f
    (\theta (\theta^{- 1} (s)))}{\sqrt{\dot{\theta} (\theta^{- 1} (s))}} \\
    & = f (s) 
  \end{align}
  
  where the last equality uses $\theta (\theta^{- 1} (s)) = s$. Similarly, for
  any $g \in L^2 (\mathbb{R})$:
  
  \begin{align}
    (M_{\theta} M_{\theta}^{- 1} g) (t) & = \sqrt{\dot{\theta} (t)} 
    \hspace{0.17em} (M_{\theta}^{- 1} g) (\theta (t)) \\
    & = \sqrt{\dot{\theta} (t)}  \hspace{0.17em} \frac{g (\theta^{- 1}
    (\theta (t)))}{\sqrt{\dot{\theta} (\theta^{- 1} (\theta (t)))}} \\
    & = \sqrt{\dot{\theta} (t)}  \hspace{0.17em} \frac{g
    (t)}{\sqrt{\dot{\theta} (t)}} \\
    & = g (t) 
  \end{align}
  
  Therefore
  \begin{equation}
    M_{\theta} M_{\theta}^{- 1} = M_{\theta}^{- 1} M_{\theta} = I
  \end{equation}
  proving that $M_{\theta}$ is unitary.
\end{proof}

\begin{corollary}
  [Measure Preservation]\label{cor:measure_preserve} The transformation
  $M_{\theta}$ preserves the $L^2$-measure in the sense that for any
  measurable set $A \subseteq \mathbb{R}$
  \begin{equation}
    \label{eq:measure_preserve_sets} \int_A \lvert (M_{\theta} f) (t) \rvert^2
    \hspace{0.17em} dt = \int_{\theta (A)} \lvert f (s) \rvert^2 
    \hspace{0.17em} ds
  \end{equation}
\end{corollary}

\begin{proof}
  The proof follows the same change of variables argument as in
  Theorem~\ref{thm:unitary}, applied to the characteristic function of the set
  $A$.
\end{proof}

\subsection{$L^2$-Norm Preservation}\label{sec:norm_preservation}

\begin{theorem}
  [Measure Preservation]\label{thm:measure_preserve} The transformation
  defined in equation~\eqref{eq:time_change} preserves the $L^2$-norm in the
  sense that
  \begin{equation}
    \label{eq:measure_preserve} \int_I \mathrm{var} (Z (t))  \hspace{0.17em}
    dt = \int_{\theta (I)} \mathrm{var} (X (s))  \hspace{0.17em} ds
  \end{equation}
  for any measurable set $I \subseteq \mathbb{R}$.
\end{theorem}

\begin{proof}
  Using the change of variables $s = \theta (t)$ with $ds = \dot{\theta} (t) 
  \hspace{0.17em} dt$:
  
  \begin{align}
    \int_I \mathrm{var} (X (t))  \hspace{0.17em} dt & = \int_I \mathrm{var}
    \left( \sqrt{\dot{\theta} (t)}  \hspace{0.17em} X (\theta (t)) \right) 
    \hspace{0.17em} dt \\
    & = \int_I \dot{\theta} (t) \hspace{0.17em} \mathrm{var} (X (\theta (t)))
    \hspace{0.17em} dt \\
    & = \int_{\theta (I)} \mathrm{var} (X (s))  \hspace{0.17em} ds 
  \end{align}
\end{proof}

\subsection{Oscillatory Representation}

\begin{theorem}
  [Oscillatory Form]\label{thm:osc_rep} The process $\{ Z (t) \}$ defined in
  equation~\eqref{eq:time_change} is oscillatory with oscillatory functions
  \begin{equation}
    \label{eq:phi_def} \phi_t (\omega) = A_t (\omega) e^{i \omega t} =
    \sqrt{\dot{\theta} (t)}  \hspace{0.17em} e^{i \omega \theta (t)}
  \end{equation}
  and gain functions \
  \begin{equation}
    A_t (\omega) = \sqrt{\dot{\theta} (t)}  \hspace{0.17em} e^{i \omega
    (\theta (t) - t)}
  \end{equation}
\end{theorem}

\begin{proof}
  From the spectral representation~\eqref{eq:stationary_rep} of the stationary
  process $X (t)$:
  
  \begin{align}
    X (t) & = \sqrt{\dot{\theta} (t)} X (\theta (t)) \\
    & = \sqrt{\dot{\theta} (t)}  \int_{- \infty}^{\infty} e^{i \omega \theta
    (t)}  \hspace{0.17em} d \Phi (\omega) \\
    & = \int_{- \infty}^{\infty} \sqrt{\dot{\theta} (t)}  \hspace{0.17em}
    e^{i \omega \theta (t)}  \hspace{0.17em} d \phi (\omega) \\
    & = \int_{- \infty}^{\infty} \phi_t (\omega)  \hspace{0.17em} d \Phi
    (\omega) 
  \end{align}
  
  where
  \begin{equation}
    \phi_t (\omega) = \sqrt{\dot{\theta} (t)}  \hspace{0.17em} e^{i \omega
    \theta (t)}
  \end{equation}
  To verify this is an oscillatory representation according to
  Definition~\ref{def:oscillatory}, express $\phi_t (\omega)$ in the form of a
  function of the time-dependent gain $A_t (\lambda)$ as required
  \begin{equation}
    \begin{array}{ll}
      \phi_t (\omega) & = A_t (\omega) e^{i \omega t}\\
      & = \sqrt{\dot{\theta} (t)}  \hspace{0.17em} e^{i \omega (\theta (t) -
      t)}  \hspace{0.17em} e^{i \omega t}\\
      & = \sqrt{\dot{\theta} (t)} e^{i \omega (\theta (t) - t + t)}\\
      & = \sqrt{\dot{\theta} (t)}  \hspace{0.17em} e^{i \omega \theta (t)}
    \end{array}
  \end{equation}
  where
  \begin{equation}
    A_t (\omega) = \sqrt{\dot{\theta} (t)}  \hspace{0.17em} e^{i \omega
    (\theta (t) - t)}
  \end{equation}
  .
  
  Since $\dot{\theta} (t) \geq 0$ almost everywhere and $\dot{\theta} (t) = 0$
  only on sets of measure zero, the function $A_t (\omega)$ is well-defined
  almost everywhere. Moreover, $A_t (\cdummy) \in L^2 (\mu)$ for each $t$
  since:
  
  \begin{align}
    \int_{- \infty}^{\infty} \lvert A_t (\omega) \rvert^2  \hspace{0.17em} d
    \mu (\omega) & = \int_{- \infty}^{\infty} \dot{\theta} (t) 
    \hspace{0.17em} d \mu (\omega) \\
    & = \dot{\theta} (t) \int_{- \infty}^{\infty}  \hspace{0.17em} d \mu
    (\omega) \nonumber\\
    & = \dot{\theta} (t) \mu (\mathbb{R}) < \infty 
  \end{align}
  
  where the finiteness follows from $\mu$ being a finite measure and
  $\dot{\theta} (t)$ being finite almost everywhere.
\end{proof}

\subsection{Envelope and Evolutionary Spectrum}

\

\begin{corollary}
  [Evolutionary Spectrum]\label{cor:evolving_spec} The evolutionary power
  spectrum is
  \begin{equation}
    \label{eq:evolutionary_spec} \begin{array}{ll}
      dF_t (\omega) & = \lvert A_t (\omega) \rvert^2  \hspace{0.17em} d \mu
      (\omega)\\
      & = \dot{\theta} (t)  \hspace{0.17em} d \mu (\omega)
    \end{array}
  \end{equation}
\end{corollary}

\begin{proof}
  By Definition~\ref{def:oscillatory} and the envelope from
  Equation~\ref{envelope}, the evolutionary power spectrum is:
  
  \begin{align}
    dF_t (\omega) & = \lvert A_t (\omega) \rvert^2  \hspace{0.17em} d \mu
    (\omega) \\
    & = \left| \sqrt{\dot{\theta} (t)}  \hspace{0.17em} e^{i \omega (\theta
    (t) - t)} \right|^2  \hspace{0.17em} d \mu (\omega) \\
    & = \dot{\theta} (t) \hspace{0.17em} | e^{i \omega (\theta (t) - t)} |^2 
    \hspace{0.17em} d \mu (\omega) \\
    & = \dot{\theta} (t)  \hspace{0.17em} d \mu (\omega) 
  \end{align}
  
  since
  \begin{equation}
    |e^{i \alpha} | = 1 \forall \alpha \in \mathbb{R}
  \end{equation}
\end{proof}

\section{Operator Conjugation}\label{sec:conjugation}

\begin{theorem}
  [Operator Conjugation]\label{thm:operator_conjugation} Let $T_K$ be the \
  integral covariance operator defined by
  \begin{equation}
    \label{eq:integral_op_original} \begin{array}{ll}
      (T_K f) (t) & = \int_{- \infty}^{\infty} K (|t - s|) f (s) 
      \hspace{0.17em} ds
    \end{array}
  \end{equation}
  where $K (h)$ is the stationary kernel
  \begin{equation}
    K (h) = \int_{- \infty}^{\infty} S (\lambda) e^{i \lambda h} \mathd
    \lambda
  \end{equation}
  , and let $T_{K_{\theta}}$ be the integral covariance operator defined by
  \begin{equation}
    \label{eq:integral_op_transformed} \begin{array}{ll}
      (T_{K_{\theta}} f) (t) & = \int_{- \infty}^{\infty} K_{\theta} (s, t) f
      (s)  \hspace{0.17em} ds\\
      & = \int_{- \infty}^{\infty} K (| \theta (t) - \theta (s) |) 
      \sqrt{\dot{\theta} (t) \dot{\theta} (s)} f (s)  \hspace{0.17em} ds
    \end{array}
  \end{equation}
  for the unitarily time-changed kernel
  \begin{equation}
    K_{\theta} (s, t) = K (| \theta (t) - \theta (s) |) \sqrt{\dot{\theta} (t)
    \dot{\theta} (s)}
  \end{equation}
  . Then
  \begin{equation}
    \label{eq:conjugation} T_{K_{\theta}} = M_{\theta} T_K M_{\theta}^{- 1}
  \end{equation}
\end{theorem}

\begin{proof}
  For any $g \in L^2 (\mathbb{R})$, compute $(M_{\theta} T_K M_{\theta}^{- 1}
  g) (t)$:
  
  \begin{align}
    (M_{\theta}^{- 1} g) (s) & = \frac{g (\theta^{- 1} (s))}{\sqrt{\theta'
    (\theta^{- 1} (s))}}, \\
    (T_K M_{\theta}^{- 1} g) (t) & = \int_{- \infty}^{\infty} K (|t - s|)
    \frac{g (\theta^{- 1} (s))}{\sqrt{\dot{\theta} (\theta^{- 1} (s))}} ds. 
  \end{align}
  
  Apply the change of variables $u = \theta^{- 1} (s)$, so $s = \theta (u)$
  and $ds = \dot{\theta} (u) du$:
  
  \begin{align}
    (T_K M_{\theta}^{- 1} g) (t) & = \int_{- \infty}^{\infty} K (|t - \theta
    (u) |) \frac{g (u)}{\sqrt{\dot{\theta} (u)}}  \dot{\theta} (u) du \\
    & = \int_{- \infty}^{\infty} K (|t - \theta (u) |) g (u)
    \sqrt{\dot{\theta} (u)} du. 
  \end{align}
  
  Now apply $M_{\theta}$:
  
  \begin{align}
    (M_{\theta} T_K M_{\theta}^{- 1} g) (t) & = \sqrt{\dot{\theta} (t)}  (T_K
    M_{\theta}^{- 1} g) (\theta (t)) \\
    & = \sqrt{\dot{\theta} (t)}  \int_{- \infty}^{\infty} K (| \theta (t) -
    \theta (u) |) g (u) \sqrt{\dot{\theta} (u)} du. 
  \end{align}
  
  Apply the change of variables $s = \theta (u)$ in the reverse direction:
  
  \begin{align}
    (M_{\theta} T_K M_{\theta}^{- 1} g) (t) & = \int_{- \infty}^{\infty} K (|
    \theta (t) - \theta (s) |) g (s) ds \\
    & = (T_{K_{\theta}} g) (t) 
  \end{align}
  
  This establishes the conjugation relation~\eqref{eq:conjugation}.
\end{proof}

\section{Expected Zero Count}\label{sec:zero_count}

\begin{theorem}
  [Expected Zero-Counting Function]\label{thm:zero_count} Let $\theta \in
  \mathcal{F}$ and let
  \begin{equation}
    K (\tau) = \mathrm{cov} (X (t), X (\tau))
  \end{equation}
  be twice differentiable at $\tau = 0$. The expected number of zeros of the
  process $X_t$ in $[a, b]$ is
  \begin{equation}
    \label{eq:zero_count} \mathbb{E} [N_{[a, b]}] = \sqrt{- \ddot{K} (0)} 
    \hspace{0.17em} (\theta (b) - \theta (a))
  \end{equation}
\end{theorem}

\begin{proof}
  The covariance function of the time-changed process is
  \begin{equation}
    \label{eq:time_changed_cov} K_{\theta} (s, t) = \mathrm{cov} (X_s, X_t) =
    \sqrt{\dot{\theta} (s)  \dot{\theta} (t)}  \hspace{0.17em} K (| \theta (t)
    - \theta (s) |)
  \end{equation}
  For the zero-crossing analysis, consider the normalized process. By the
  Kac-Rice formula:
  \begin{equation}
    \label{eq:kac_rice} \mathbb{E} [N_{[a, b]}] = \int_a^b \sqrt{- \lim_{s \to
    t}  \frac{\partial^2}{\partial s \partial t} K_{\theta} (s, t)} 
    \hspace{0.17em} dt
  \end{equation}
  Computing the mixed partial derivative:
  
  \begin{align}
    \frac{\partial}{\partial t} K_{\theta} (s, t) & = \frac{1}{2} 
    \frac{\ddot{\theta} (t)}{\sqrt{\dot{\theta} (t)}}  \sqrt{\theta' (s)} K (|
    \theta (t) - \theta (s) |) \\
    & \quad + \sqrt{\dot{\theta} (s)  \dot{\theta} (t)}  \dot{K} (| \theta
    (t) - \theta (s) |) \mathrm{sgn} (\theta (t) - \theta (s))  \dot{\theta}
    (t) . 
  \end{align}
  
  Taking the limit as $s \to t$ and using the fact that $\dot{K} (0) = 0$ for
  stationary processes:
  
  \begin{align}
    \lim_{s \to t}  \frac{\partial^2}{\partial s \partial t} K_{\theta} (s, t)
    & = \dot{\theta} (s)  \dot{\theta} (t)  \ddot{K} (0) \\
    & = \dot{\theta} (t)^2  \ddot{K} (0) 
  \end{align}
  
  Substituting into the Kac-Rice formula:
  
  \begin{align}
    \mathbb{E} [N_{[a, b]}] & = \int_a^b \sqrt{- \dot{\theta} (t)^2  \ddot{K}
    (0)}  \hspace{0.17em} dt \\
    & = \sqrt{- \ddot{K} (0)}  \int_a^b \dot{\theta} (t)  \hspace{0.17em} dt
    \\
    & = \sqrt{- \ddot{K} (0)}  \hspace{0.17em} (\theta (b) - \theta (a)) 
  \end{align}
  
  Here the second equality uses $\dot{\theta} (t) \geq 0$ almost everywhere.
\end{proof}

\section{Conclusion}\label{sec:conclusion}

This analysis establishes that Gaussian processes generated by
measure-preserving bijective time changes of stationary processes form a
well-defined subclass of oscillatory processes. The key contributions include:
\begin{enumerate}
  \item The rigorous construction of the unitary operator $M_{\theta}$ and its
  inverse, with proper treatment of the case where $\dot{\theta} (t) = 0$ on
  sets of measure zero.
  
  \item The explicit oscillatory representation with envelope function $A_t
  (\omega) = \sqrt{\dot{\theta} (t)} e^{i \omega (\theta (t) - t)}$.
  
  \item The evolutionary power spectrum formula $dF_t (\omega) = \dot{\theta}
  (t) d \mu (\omega)$.
  
  \item The operator conjugation relationship $T_{K_{\theta}} = M_{\theta} T_K
  M_{\theta}^{- 1}$.
  
  \item A closed-form expression for the expected zero count in terms of the
  range of the time transformation.
\end{enumerate}
\begin{thebibliography}{99}
\bibitem{priestley1965} M.B. Priestley. Evolutionary spectra and non-stationary processes. \textit{Journal of the Royal Statistical Society, Series B}, 27(2):204--237, 1965.

\bibitem{cramer1967} H. Cramer and M.R. Leadbetter. \textit{Stationary and Related Stochastic Processes}. Wiley, 1967.

\bibitem{kac1943} M. Kac. On the average number of real roots of a random algebraic equation. \textit{Bulletin of the American Mathematical Society}, 49(4):314--320, 1943.

\bibitem{rice1945} S.O. Rice. Mathematical analysis of random noise. \textit{Bell System Technical Journal}, 24(1):46--156, 1945.
\end{thebibliography}


\end{document}
