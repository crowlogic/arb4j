\documentclass[11pt]{article}
\usepackage{amsmath,amssymb,amsthm,mathtools}
\usepackage{enumitem}
\usepackage{hyperref}

\newtheorem{definition}{Definition}
\newtheorem{theorem}{Theorem}
\newtheorem{lemma}{Lemma}
\newtheorem{corollary}{Corollary}
\newtheorem{remark}{Remark}

\title{Measure-Preserving Bijective Time Changes of Stationary Gaussian Processes Generate Oscillatory Processes With Evolving Spectra}

\author{Stephen Crowley\thanks{Email: \texttt{stephencrowley214@gmail.com}}}

\date{August 1, 2025}

\begin{document}

\maketitle

\begin{abstract}
This article establishes that Gaussian processes obtained through measure-preserving bijective unitary time transformations of stationary processes constitute a subclass of oscillatory processes in the sense of Priestley. The transformation $X_t = \sqrt{\theta'(t)} S_{\theta(t)}$, where $S_t$ is a stationary Gaussian process and $\theta$ is a strictly monotonic function, yields an oscillatory process with evolutionary power spectrum $dF_t(\omega) = \theta'(t) d\mu(\omega)$. An explicit unitary transformation between the original stationary process and the transformed oscillatory process is established, preserving the $L^2$-norm and providing a complete spectral characterization.
\end{abstract}

\section{Scaling Functions}\label{sec:scaling}

\begin{definition}[Scaling Functions]\label{def:scaling}
Let $\mathcal{F}$ denote the set of functions $\theta\colon\mathbb{R}\to\mathbb{R}$ satisfying
\begin{enumerate}[label=(\alph*)]
    \item $\theta$ is absolutely continuous with $\theta'(t) \geq 0$ almost everywhere and $\theta'(t) = 0$ only on sets of Lebesgue measure zero,
    \item $\theta$ is strictly increasing and bijective.
\end{enumerate}
\end{definition}

\begin{remark}\label{rem:inverse_properties}
The conditions in Definition~\ref{def:scaling} ensure that $\theta^{-1}$ exists and is absolutely continuous. By the inverse function theorem for absolutely continuous functions, $(\theta^{-1})'(s) = \frac{1}{\theta'(\theta^{-1}(s))}$ for almost all $s$ in the range of $\theta$. The condition that $\theta'(t) = 0$ only on sets of measure zero ensures that $\frac{1}{\theta'(\theta^{-1}(s))}$ is well-defined almost everywhere.
\end{remark}

\section{Oscillatory Processes}\label{sec:oscillatory}

\begin{definition}[Oscillatory Process]\label{def:oscillatory}
A complex-valued, second-order process $\{X_t\}_{t\in\mathbb{R}}$ is called \emph{oscillatory} if there exist
\begin{enumerate}[label=(\roman*)]
    \item a family of functions $\{\phi_t(\omega)\}_{t\in\mathbb{R}}$ with $\phi_t(\omega)=A_t(\omega)e^{i\omega t}$ and $A_t(\cdot)\in L^2(\mu)$,
    \item a complex orthogonal-increment process $Z(\omega)$ with $E\lvert dZ(\omega)\rvert^2=d\mu(\omega)$,
\end{enumerate}
such that
\begin{equation}\label{eq:oscillatory_rep}
    X_t=\int_{-\infty}^{\infty}\phi_t(\omega)\,dZ(\omega).
\end{equation}
\end{definition}

\section{Stationary Reference Process}\label{sec:stationary}

Let $\{S_t\}_{t\in\mathbb{R}}$ be a stationary Gaussian process with continuous spectral representation
\begin{equation}\label{eq:stationary_rep}
    S_t=\int_{-\infty}^{\infty}e^{i\omega t}\,dZ(\omega),
\end{equation}
where $Z(\omega)$ is an orthogonal-increment process with $E\lvert dZ(\omega)\rvert^2=d\mu(\omega)$ and $\mu$ is a finite measure on $\mathbb{R}$.

\section{Time-Changed Process}\label{sec:time_change}

\subsection{Definition and Unitary Operator}

\begin{definition}[Time-Changed Process]\label{def:time_changed_proc}
For $\theta\in\mathcal{F}$, define the time-changed process
\begin{equation}\label{eq:time_change}
    X_t \coloneqq \sqrt{\theta'(t)}\,S_{\theta(t)}, \qquad t\in\mathbb{R}.
\end{equation}
\end{definition}

\begin{definition}[Unitary Transformation Operator]\label{def:unitary_op}
For $\theta\in\mathcal{F}$, define the operator $M_\theta\colon L^2(\mathbb{R})\to L^2(\mathbb{R})$ by
\begin{equation}\label{eq:unitary_op}
    (M_\theta f)(t) = \sqrt{\theta'(t)}\,f\bigl(\theta(t)\bigr).
\end{equation}
\end{definition}

\begin{definition}[Inverse Unitary Transformation Operator]\label{def:inverse_unitary_op}
The inverse operator $M_\theta^{-1}\colon L^2(\mathbb{R})\to L^2(\mathbb{R})$ is defined by
\begin{equation}\label{eq:unitary_inverse}
    (M_\theta^{-1} g)(s) = \frac{g\bigl(\theta^{-1}(s)\bigr)}{\sqrt{\theta'\bigl(\theta^{-1}(s)\bigr)}}.
\end{equation}
\end{definition}

\begin{lemma}[Well-Definedness of Inverse Operator]\label{lem:inverse_well_defined}
The operator $M_\theta^{-1}$ in Definition~\ref{def:inverse_unitary_op} is well-defined for $\theta\in\mathcal{F}$.
\end{lemma}

\begin{proof}
Since $\theta'(t) = 0$ only on sets of measure zero by Definition~\ref{def:scaling}, and $\theta^{-1}$ maps sets of measure zero to sets of measure zero (as it preserves absolute continuity), the denominator $\sqrt{\theta'(\theta^{-1}(s))}$ is positive almost everywhere. The expression in equation~\eqref{eq:unitary_inverse} is therefore well-defined almost everywhere, which is sufficient for defining an element of $L^2(\mathbb{R})$.
\end{proof}

\begin{theorem}[Unitarity of Transformation Operator]\label{thm:unitary}
The operator $M_\theta$ defined in equation~\eqref{eq:unitary_op} is unitary, i.e.,
\begin{equation}\label{eq:L2_preserve}
    \int_{\mathbb{R}}\lvert (M_\theta f)(t)\rvert^2\,dt
    =\int_{\mathbb{R}}\lvert f(s)\rvert^2\,ds
\end{equation}
for all $f\in L^2(\mathbb{R})$.
\end{theorem}

\begin{proof}
Let $f\in L^2(\mathbb{R})$. The $L^2$-norm of $M_\theta f$ is computed as follows:
\begin{align}
    \int_{\mathbb{R}}\lvert (M_\theta f)(t)\rvert^2\,dt &= \int_{\mathbb{R}}\left|\sqrt{\theta'(t)}\,f\bigl(\theta(t)\bigr)\right|^2\,dt\\
    &= \int_{\mathbb{R}}\theta'(t)\,\lvert f\bigl(\theta(t)\bigr)\rvert^2\,dt.
\end{align}
Apply the change of variables $s=\theta(t)$. Since $\theta$ is absolutely continuous and strictly increasing, $ds=\theta'(t)\,dt$ almost everywhere. As $t$ ranges over $\mathbb{R}$, $s=\theta(t)$ ranges over $\mathbb{R}$ due to the bijectivity of $\theta$. Therefore:
\begin{align}
    \int_{\mathbb{R}}\theta'(t)\,\lvert f\bigl(\theta(t)\bigr)\rvert^2\,dt &= \int_{\mathbb{R}}\lvert f(s)\rvert^2\,ds.
\end{align}
This establishes equation~\eqref{eq:L2_preserve}.

To complete the proof of unitarity, it remains to show that $M_\theta^{-1}$ is indeed the inverse of $M_\theta$. For any $f\in L^2(\mathbb{R})$:
\begin{align}
    (M_\theta^{-1}M_\theta f)(s) &= (M_\theta^{-1})\left[\sqrt{\theta'(\cdot)}\,f\bigl(\theta(\cdot)\bigr)\right](s)\\
    &= \frac{\sqrt{\theta'\bigl(\theta^{-1}(s)\bigr)}\,f\bigl(\theta(\theta^{-1}(s))\bigr)}{\sqrt{\theta'\bigl(\theta^{-1}(s)\bigr)}}\\
    &= f(s),
\end{align}
where the last equality uses $\theta(\theta^{-1}(s)) = s$.

Similarly, for any $g\in L^2(\mathbb{R})$:
\begin{align}
    (M_\theta M_\theta^{-1} g)(t) &= \sqrt{\theta'(t)}\,(M_\theta^{-1} g)\bigl(\theta(t)\bigr)\\
    &= \sqrt{\theta'(t)}\,\frac{g\bigl(\theta^{-1}(\theta(t))\bigr)}{\sqrt{\theta'\bigl(\theta^{-1}(\theta(t))\bigr)}}\\
    &= \sqrt{\theta'(t)}\,\frac{g(t)}{\sqrt{\theta'(t)}}\\
    &= g(t).
\end{align}

Therefore, $M_\theta M_\theta^{-1} = M_\theta^{-1} M_\theta = I$, proving that $M_\theta$ is unitary.
\end{proof}

\begin{corollary}[Measure Preservation]\label{cor:measure_preserve}
The transformation $M_\theta$ preserves the $L^2$-measure in the sense that for any measurable set $A\subseteq\mathbb{R}$,
\begin{equation}\label{eq:measure_preserve_sets}
    \int_A \lvert (M_\theta f)(t)\rvert^2\,dt = \int_{\theta(A)} \lvert f(s)\rvert^2\,ds.
\end{equation}
\end{corollary}

\begin{proof}
The proof follows the same change of variables argument as in Theorem~\ref{thm:unitary}, applied to the characteristic function of the set $A$.
\end{proof}

\subsection{Oscillatory Representation}

\begin{theorem}[Oscillatory Form]\label{thm:osc_rep}
The process $\{X_t\}$ defined in equation~\eqref{eq:time_change} is oscillatory with oscillatory functions
\begin{equation}\label{eq:phi_def}
    \phi_t(\omega)=\sqrt{\theta'(t)}\,e^{i\omega\theta(t)}.
\end{equation}
\end{theorem}

\begin{proof}
From the spectral representation~\eqref{eq:stationary_rep} of the stationary process $S_t$:
\begin{align}
    X_t &= \sqrt{\theta'(t)}\,S_{\theta(t)}\\
    &= \sqrt{\theta'(t)}\int_{-\infty}^{\infty}e^{i\omega\theta(t)}\,dZ(\omega)\\
    &= \int_{-\infty}^{\infty}\sqrt{\theta'(t)}\,e^{i\omega\theta(t)}\,dZ(\omega)\\
    &= \int_{-\infty}^{\infty}\phi_t(\omega)\,dZ(\omega),
\end{align}
where $\phi_t(\omega) = \sqrt{\theta'(t)}\,e^{i\omega\theta(t)}$.

To verify this is an oscillatory representation according to Definition~\ref{def:oscillatory}, express $\phi_t(\omega)$ in the required form:
\begin{align}
    \phi_t(\omega) &= \sqrt{\theta'(t)}\,e^{i\omega\theta(t)}\\
    &= \sqrt{\theta'(t)}\,e^{i\omega(\theta(t)-t)}\,e^{i\omega t}\\
    &= A_t(\omega)e^{i\omega t},
\end{align}
where $A_t(\omega) = \sqrt{\theta'(t)}\,e^{i\omega(\theta(t)-t)}$.

Since $\theta'(t) \geq 0$ almost everywhere and $\theta'(t) = 0$ only on sets of measure zero, the function $A_t(\omega)$ is well-defined almost everywhere. Moreover, $A_t(\cdot)\in L^2(\mu)$ for each $t$ since:
\begin{align}
    \int_{-\infty}^{\infty}\lvert A_t(\omega)\rvert^2\,d\mu(\omega) &= \int_{-\infty}^{\infty}\theta'(t)\,d\mu(\omega)\\
    &= \theta'(t)\mu(\mathbb{R}) < \infty,
\end{align}
where the finiteness follows from $\mu$ being a finite measure and $\theta'(t)$ being finite almost everywhere.
\end{proof}

\subsection{Envelope and Evolutionary Spectrum}

\begin{corollary}[Envelope]\label{cor:envelope}
The oscillatory functions in equation~\eqref{eq:phi_def} admit the standard decomposition
\begin{equation}\label{eq:envelope}
    \phi_t(\omega)=A_t(\omega)e^{i\omega t},
    \quad\text{where}\quad
    A_t(\omega)=\sqrt{\theta'(t)}\,e^{i\omega(\theta(t)-t)}.
\end{equation}
\end{corollary}

\begin{proof}
This follows directly from the calculation in the proof of Theorem~\ref{thm:osc_rep}.
\end{proof}

\begin{corollary}[Evolutionary Spectrum]\label{cor:evolving_spec}
The evolutionary power spectrum is
\begin{equation}\label{eq:evolutionary_spec}
    dF_t(\omega)=\lvert A_t(\omega)\rvert^2\,d\mu(\omega)=\theta'(t)\,d\mu(\omega).
\end{equation}
\end{corollary}

\begin{proof}
By Definition~\ref{def:oscillatory} and the envelope from Corollary~\ref{cor:envelope}, the evolutionary power spectrum is:
\begin{align}
    dF_t(\omega) &= \lvert A_t(\omega)\rvert^2\,d\mu(\omega)\\
    &= \left|\sqrt{\theta'(t)}\,e^{i\omega(\theta(t)-t)}\right|^2\,d\mu(\omega)\\
    &= \theta'(t)\,\left|e^{i\omega(\theta(t)-t)}\right|^2\,d\mu(\omega)\\
    &= \theta'(t)\,d\mu(\omega),
\end{align}
since $|e^{i\alpha}| = 1$ for any real $\alpha$.
\end{proof}

\section{Operator Conjugation}\label{sec:conjugation}

\begin{theorem}[Operator Conjugation]\label{thm:operator_conjugation}
Let $T_K$ be the integral operator defined by
\begin{equation}\label{eq:integral_op_original}
    (T_K f)(t) = \int_{-\infty}^{\infty} K(|t-s|) f(s)\,ds
\end{equation}
for a stationary kernel $K$, and let $T_{K_\theta}$ be the integral operator defined by
\begin{equation}\label{eq:integral_op_transformed}
    (T_{K_\theta} g)(t) = \int_{-\infty}^{\infty} K(|\theta(t)-\theta(s)|) g(s)\,ds
\end{equation}
for the transformed kernel $K_\theta(s,t) = K(|\theta(t)-\theta(s)|)$. Then
\begin{equation}\label{eq:conjugation}
    T_{K_\theta} = M_\theta T_K M_\theta^{-1}.
\end{equation}
\end{theorem}

\begin{proof}
For any $g\in L^2(\mathbb{R})$, compute $(M_\theta T_K M_\theta^{-1} g)(t)$:
\begin{align}
    (M_\theta^{-1} g)(s) &= \frac{g(\theta^{-1}(s))}{\sqrt{\theta'(\theta^{-1}(s))}},\\
    (T_K M_\theta^{-1} g)(t) &= \int_{-\infty}^{\infty} K(|t-s|) \frac{g(\theta^{-1}(s))}{\sqrt{\theta'(\theta^{-1}(s))}} ds.
\end{align}

Apply the change of variables $u = \theta^{-1}(s)$, so $s = \theta(u)$ and $ds = \theta'(u)du$:
\begin{align}
    (T_K M_\theta^{-1} g)(t) &= \int_{-\infty}^{\infty} K(|t-\theta(u)|) \frac{g(u)}{\sqrt{\theta'(u)}} \theta'(u) du\\
    &= \int_{-\infty}^{\infty} K(|t-\theta(u)|) g(u) \sqrt{\theta'(u)} du.
\end{align}

Now apply $M_\theta$:
\begin{align}
    (M_\theta T_K M_\theta^{-1} g)(t) &= \sqrt{\theta'(t)} (T_K M_\theta^{-1} g)(\theta(t))\\
    &= \sqrt{\theta'(t)} \int_{-\infty}^{\infty} K(|\theta(t)-\theta(u)|) g(u) \sqrt{\theta'(u)} du.
\end{align}

Apply the change of variables $s = \theta(u)$ in the reverse direction:
\begin{align}
    (M_\theta T_K M_\theta^{-1} g)(t) &= \int_{-\infty}^{\infty} K(|\theta(t)-\theta(s)|) g(s) ds\\
    &= (T_{K_\theta} g)(t).
\end{align}

This establishes the conjugation relation~\eqref{eq:conjugation}.
\end{proof}

\section{Expected Zero Count}\label{sec:zero_count}

\begin{theorem}[Expected Zero-Counting Function]\label{thm:zero_count}
Let $\theta\in\mathcal{F}$ and let $K(\tau) = \operatorname{cov}(S_0, S_\tau)$ be twice differentiable at $\tau=0$. The expected number of zeros of the process $X_t$ in $[a,b]$ is
\begin{equation}\label{eq:zero_count}
    \mathbb{E}[N_{[a,b]}] = \sqrt{-\ddot{K}(0)}\,(\theta(b)-\theta(a)).
\end{equation}
\end{theorem}

\begin{proof}
The covariance function of the time-changed process is
\begin{equation}\label{eq:time_changed_cov}
    K_\theta(s,t) = \operatorname{cov}(X_s, X_t) = \sqrt{\theta'(s)\theta'(t)}\,K(|\theta(t)-\theta(s)|).
\end{equation}

For the zero-crossing analysis, consider the normalized process. By the Kac-Rice formula:
\begin{equation}\label{eq:kac_rice}
    \mathbb{E}[N_{[a,b]}] = \int_a^b \sqrt{-\lim_{s\to t}\frac{\partial^2}{\partial s\partial t}K_\theta(s,t)}\,dt.
\end{equation}

Computing the mixed partial derivative:
\begin{align}
    \frac{\partial}{\partial t}K_\theta(s,t) &= \frac{1}{2}\frac{\theta''(t)}{\sqrt{\theta'(t)}}\sqrt{\theta'(s)}K(|\theta(t)-\theta(s)|)\\
    &\quad + \sqrt{\theta'(s)\theta'(t)}K'(|\theta(t)-\theta(s)|)\operatorname{sgn}(\theta(t)-\theta(s))\theta'(t).
\end{align}

Taking the limit as $s\to t$ and using the fact that $K'(0) = 0$ for stationary processes:
\begin{align}
    \lim_{s\to t}\frac{\partial^2}{\partial s\partial t}K_\theta(s,t) &= \theta'(s)\theta'(t)K''(0)\\
    &= \theta'(t)^2\ddot{K}(0).
\end{align}

Substituting into the Kac-Rice formula:
\begin{align}
    \mathbb{E}[N_{[a,b]}] &= \int_a^b \sqrt{-\theta'(t)^2\ddot{K}(0)}\,dt\\
    &= \sqrt{-\ddot{K}(0)}\int_a^b \theta'(t)\,dt\\
    &= \sqrt{-\ddot{K}(0)}\,(\theta(b)-\theta(a)).
\end{align}

Here the second equality uses $\theta'(t) \geq 0$ almost everywhere.
\end{proof}

\section{Conclusion}\label{sec:conclusion}

This analysis establishes that Gaussian processes generated by measure-preserving bijective time changes of stationary processes form a well-defined subclass of oscillatory processes. The key contributions include:

\begin{enumerate}
\item The rigorous construction of the unitary operator $M_\theta$ and its inverse, with proper treatment of the case where $\theta'(t) = 0$ on sets of measure zero.
\item The explicit oscillatory representation with envelope function $A_t(\omega) = \sqrt{\theta'(t)}e^{i\omega(\theta(t)-t)}$.
\item The evolutionary power spectrum formula $dF_t(\omega) = \theta'(t)d\mu(\omega)$.
\item The operator conjugation relationship $T_{K_\theta} = M_\theta T_K M_\theta^{-1}$.
\item A closed-form expression for the expected zero count in terms of the range of the time transformation.
\end{enumerate}

\begin{thebibliography}{99}
\bibitem{priestley1965} M.B. Priestley. Evolutionary spectra and non-stationary processes. \emph{Journal of the Royal Statistical Society, Series B}, 27(2):204--237, 1965.

\bibitem{cramer1967} H. Cramér and M.R. Leadbetter. \emph{Stationary and Related Stochastic Processes}. Wiley, 1967.

\bibitem{kac1943} M. Kac. On the average number of real roots of a random algebraic equation. \emph{Bulletin of the American Mathematical Society}, 49(4):314--320, 1943.

\bibitem{rice1945} S.O. Rice. Mathematical analysis of random noise. \emph{Bell System Technical Journal}, 24(1):46--156, 1945.
\end{thebibliography}

\end{document}
