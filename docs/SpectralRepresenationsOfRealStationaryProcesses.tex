\documentclass{article}
\usepackage[english]{babel}
\usepackage{geometry,amsmath,amssymb,latexsym,theorem}
\geometry{letterpaper}

%%%%%%%%%% Start TeXmacs macros
\newcommand{\tmtextbf}[1]{\text{{\bfseries{#1}}}}
\newcommand{\tmtextit}[1]{\text{{\itshape{#1}}}}
\newenvironment{proof}{\noindent\textbf{Proof\ }}{\hspace*{\fill}$\Box$\medskip}
\newtheorem{corollary}{Corollary}
{\theorembodyfont{\rmfamily}\newtheorem{remark}{Remark}}
\newtheorem{theorem}{Theorem}
%%%%%%%%%% End TeXmacs macros

\begin{document}

\begin{theorem}[Real Spectral Representation for Stationary Processes]
  Let $\{\xi (t), t \in \mathbb{R}\}$ be a real-valued, zero-mean,
  second-order stationary process with covariance function $r (t) =\mathbb{E}
  [\xi (t) \xi (0)]$ and spectral distribution function $F (\omega)$. Then
  there exist real-valued random measures $\{U (\omega), \omega \geq 0\}$ and
  $\{V (\omega), \omega \geq 0\}$ with orthogonal increments such that:
  \begin{enumerate}
    \item \tmtextbf{Process Representation:}
    \begin{equation}
      \xi (t) = \int_0^{\infty} [\cos (\omega t) \hspace{0.17em} dU (\omega) +
      \sin (\omega t) \hspace{0.17em} dV (\omega)]
    \end{equation}
    \item \tmtextbf{Covariance Representation:}
    \begin{equation}
      r (t) = \int_0^{\infty} \cos (\omega t)  \hspace{0.17em} dF (\omega)
    \end{equation}
    \item \tmtextbf{Orthogonality Properties:}
    
    \begin{align}
      \mathbb{E} [U (\omega)] & =\mathbb{E} [V (\omega)] = 0 \\
      \mathbb{E} [dU (\omega_1) \hspace{0.17em} dU (\omega_2)] & =\mathbb{E}
      [dV (\omega_1) \hspace{0.17em} dV (\omega_2)] = \delta (\omega_1 -
      \omega_2) dF (\omega_1) \\
      \mathbb{E} [dU (\omega_1) \hspace{0.17em} dV (\omega_2)] & = 0 \quad
      \text{for all } \omega_1, \omega_2 \geq 0 
    \end{align}
  \end{enumerate}
\end{theorem}

\begin{proof}
  \\
  
  \begin{enumerate}
    \item \tmtextbf{Construction from Complex Representation:} From the
    complex spectral representation theorem, there holds
    \begin{equation}
      \xi (t) = \int_{- \infty}^{\infty} e^{i \omega t} d \zeta (\omega)
    \end{equation}
    where $\zeta (\omega)$ is a complex-valued random measure with orthogonal
    increments and $\mathbb{E} [|d \zeta (\omega) |^2] = \frac{1}{2} dF
    (\omega)$ for the two-sided representation.
    
    \item \tmtextbf{Reality Condition:} As $\xi (t)$ is real-valued,
    \begin{equation}
      \xi (t) = \overline{\xi (t)} = \int_{- \infty}^{\infty} e^{- i \omega t}
      d \overline{\zeta (\omega)}
    \end{equation}
    \item \tmtextbf{Symmetry Property:} This reality condition requires the
    spectral random measure to satisfy
    \begin{equation}
      d \zeta (- \omega) = d \overline{\zeta (\omega)}
    \end{equation}
    for all $\omega$.
    
    \item \tmtextbf{Factorization into Real Random Measures:} For $\omega >
    0$, define
    
    \begin{align}
      dU (\omega) & = 2 \hspace{0.17em} \Re [d \zeta (\omega)] \\
      dV (\omega) & = 2 \hspace{0.17em} \Im [d \zeta (\omega)] 
    \end{align}
    
    where $\Re$ and $\Im$ denote the real and imaginary parts.
    
    \item \tmtextbf{Derivation of Real Spectral Representation:}
    \begin{equation}
      \begin{array}{ll}
        \xi (t) & = \int_0^{\infty} e^{i \omega t} d \zeta (\omega) +
        \int_0^{\infty} e^{- i \omega t} d \zeta (- \omega)\\
        & = \int_0^{\infty} e^{i \omega t} d \zeta (\omega) + \int_0^{\infty}
        e^{- i \omega t} d \overline{\zeta (\omega)}\\
        & = \int_0^{\infty} [e^{i \omega t} + e^{- i \omega t}] \Re [d \zeta
        (\omega)] + \int_0^{\infty} i [e^{i \omega t} - e^{- i \omega t}] \Im
        [d \zeta (\omega)]\\
        & = \int_0^{\infty} 2 \cos (\omega t) \Re [d \zeta (\omega)] + 2 \sin
        (\omega t) \Im [d \zeta (\omega)]\\
        & = \int_0^{\infty} \cos (\omega t) dU (\omega) + \sin (\omega t) dV
        (\omega)
      \end{array}
    \end{equation}
    \item \tmtextbf{Orthogonality Verification:} For the two-sided complex
    representation,
    \begin{equation}
      \mathbb{E} [|d \zeta (\omega) |^2] = \frac{1}{2} dF (\omega)
    \end{equation}
    Since $|d \zeta (\omega) |^2 = [\Re [d \zeta (\omega)]]^2 + [\Im [d \zeta
    (\omega)]]^2$ and the real and imaginary parts are orthogonal with equal
    variances,
    \begin{equation}
      \mathbb{E} [[\Re [d \zeta (\omega)]]^2] =\mathbb{E} [[\Im [d \zeta
      (\omega)]]^2] = \frac{1}{4} dF (\omega)
    \end{equation}
    Therefore,
    \begin{equation}
      \mathbb{E} [dU (\omega)^2] =\mathbb{E} [dV (\omega)^2] = 4 \cdot
      \frac{1}{4} dF (\omega) = dF (\omega)
    \end{equation}
    \item \tmtextbf{Covariance Function:} The covariance is given by
    \begin{equation}
      \begin{array}{ll}
        r (t) & =\mathbb{E} [\xi (t) \xi (0)]\\
        & =\mathbb{E} \left[ \left( \int_0^{\infty} \cos (\omega t) dU
        (\omega) + \sin (\omega t) dV (\omega) \right)  \int_0^{\infty} dU
        (\omega') \right]\\
        & = \int_0^{\infty} \cos (\omega t) \mathbb{E} [dU (\omega)^2]
      \end{array}
    \end{equation}
    where cross-terms vanish by orthogonality and the sine term vanishes since
    $\mathbb{E} [dV (\omega)] = 0$. Using $\mathbb{E} [dU (\omega)^2] = dF
    (\omega)$:
    \begin{equation}
      r (t) = \int_0^{\infty} \cos (\omega t) dF (\omega)
    \end{equation}
  \end{enumerate}
\end{proof}

\begin{corollary}[Physical Interpretation]
  In the real spectral representation:
  \begin{enumerate}
    \item $\cos (\omega t) dU (\omega)$ represents the cosine component at
    frequency $\omega$ with random amplitude $dU (\omega)$.
    
    \item $\sin (\omega t) dV (\omega)$ represents the sine component at
    frequency $\omega$ with random amplitude $dV (\omega)$.
    
    \item $dF (\omega)$ represents the average power contributed by frequency
    components in $(\omega, \omega + d \omega)$.
    
    \item The random measures $U (\omega)$ and $V (\omega)$ are uncorrelated
    and have equal variance increments.
  \end{enumerate}
\end{corollary}

\begin{theorem}[U and V Random Measures]
  For a real-valued stationary process $\xi (t)$ with mean-square continuous
  sample paths and spectral representation
  \begin{equation}
    \xi (t) = \int_0^{\infty} [\cos (\omega t) \hspace{0.17em} dU (\omega) +
    \sin (\omega t) \hspace{0.17em} dV (\omega)]
  \end{equation}
  the random measures $U (\omega)$ and $V (\omega)$ are given explicitly by:
  \begin{enumerate}
    \item \tmtextbf{U-process formula:}
    \begin{equation}
      U (\omega) = \lim_{T \to \infty}  \frac{1}{\pi}  \int_{- T}^T \frac{1 -
      \cos (\omega t)}{t} \xi (t)  \hspace{0.17em} dt
    \end{equation}
    \item \tmtextbf{V-process formula:}
    \begin{equation}
      V (\omega) = \lim_{T \to \infty}  \frac{1}{\pi}  \int_{- T}^T \frac{\sin
      (\omega t)}{t} \xi (t)  \hspace{0.17em} dt
    \end{equation}
    \item \tmtextbf{Alternative forms using sine and cosine integrals:}
    
    \begin{align}
      U (\omega) & = \lim_{T \to \infty}  \frac{2}{\pi}  \int_0^T \frac{1 -
      \cos (\omega t)}{t} \xi (t)  \hspace{0.17em} dt \\
      V (\omega) & = \lim_{T \to \infty}  \frac{2}{\pi}  \int_0^T \frac{\sin
      (\omega t)}{t} \xi (t)  \hspace{0.17em} dt 
    \end{align}
    
    \item \tmtextbf{Incremental form:}
    
    \begin{align}
      U (\omega_2) - U (\omega_1) & = \lim_{T \to \infty}  \frac{1}{\pi} 
      \int_{- T}^T \frac{\cos (\omega_1 t) - \cos (\omega_2 t)}{t} \xi (t) 
      \hspace{0.17em} dt \\
      V (\omega_2) - V (\omega_1) & = \lim_{T \to \infty}  \frac{1}{\pi} 
      \int_{- T}^T \frac{\sin (\omega_2 t) - \sin (\omega_1 t)}{t} \xi (t) 
      \hspace{0.17em} dt 
    \end{align}
  \end{enumerate}
\end{theorem}

\begin{proof}
  \begin{enumerate}
    \\
    \item Starting from the complex inversion formula:
    \begin{equation}
      \zeta (\lambda) - \zeta (0) = \lim_{T \to \infty}  \frac{1}{2 \pi} 
      \int_{- T}^T \frac{1 - e^{- it \lambda}}{- it} \xi (t)  \hspace{0.17em}
      dt
    \end{equation}
    \item For real processes, the following relations hold from our
    definitions:
    
    \begin{align}
      d \zeta (\omega) & = \frac{1}{2}  [dU (\omega) - i \hspace{0.17em} dV
      (\omega)]  \quad \text{for } \omega > 0 \\
      d \zeta (- \omega) & = \frac{1}{2}  [dU (\omega) + i \hspace{0.17em} dV
      (\omega)]  \quad \text{for } \omega > 0 
    \end{align}
    
    \item Therefore,
    
    \begin{align}
      U (\omega) - U (0) & = 2 ([\zeta (\omega) - \zeta (0)] + [\zeta (-
      \omega) - \zeta (0)]) \\
      V (\omega) - V (0) & = 2 i ([\zeta (\omega) - \zeta (0)] - [\zeta (-
      \omega) - \zeta (0)]) 
    \end{align}
    
    \item Substituting the inversion formula and using $1 - e^{- it \lambda} =
    1 - \cos (\lambda t) + i \sin (\lambda t)$:
    
    \begin{align}
      U (\omega) & = \lim_{T \to \infty}  \frac{1}{\pi}  \int_{- T}^T \frac{1
      - \cos (\omega t)}{t} \xi (t)  \hspace{0.17em} dt \\
      V (\omega) & = \lim_{T \to \infty}  \frac{1}{\pi}  \int_{- T}^T
      \frac{\sin (\omega t)}{t} \xi (t)  \hspace{0.17em} dt 
    \end{align}
    
    where $U (0) = V (0) = 0$ is used.
    
    \item The alternative forms follow from the fact that $\xi (t)$ is real,
    making the integrands even for $U (\omega)$ and odd for $V (\omega)$.
  \end{enumerate}
\end{proof}

\begin{remark}
  The objects $U (\omega)$ and $V (\omega)$ appearing in the real spectral
  representation of a stationary process,
  \begin{equation}
    \xi (t) = \int_0^{\infty} \cos (\omega t)  \hspace{0.17em} dU (\omega) +
    \int_0^{\infty} \sin (\omega t)  \hspace{0.17em} dV (\omega)
  \end{equation}
  are \tmtextit{random measures} (or random set functions) on the frequency
  axis $[0, \infty)$. Their main property is that their increments over
  disjoint frequency intervals are orthogonal, i.e., uncorrelated (and
  independent if Gaussian). The notation $U (\omega)$ denotes the cumulative
  random measure up to frequency $\omega$:
  \begin{equation}
    U (\omega) = U ([0, \omega])  \qquad V (\omega) = V ([0, \omega])
  \end{equation}
  For a stationary process with mean-square continuous sample paths, each
  sample path uniquely determines the corresponding random measures through
  the inversion formulas given above.
\end{remark}

\end{document}
