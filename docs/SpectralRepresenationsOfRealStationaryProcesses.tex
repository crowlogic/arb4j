\documentclass{article}
\usepackage[english]{babel}
\usepackage{amsmath,amssymb,latexsym,theorem}

%%%%%%%%%% Start TeXmacs macros
\newcommand{\tmem}[1]{{\em #1\/}}
\newcommand{\tmtextbf}[1]{\text{{\bfseries{#1}}}}
\newenvironment{proof}{\noindent\textbf{Proof\ }}{\hspace*{\fill}$\Box$\medskip}
\newtheorem{corollary}{Corollary}
{\theorembodyfont{\rmfamily}\newtheorem{remark}{Remark}}
\newtheorem{theorem}{Theorem}
%%%%%%%%%% End TeXmacs macros

\begin{document}

\begin{theorem}
  [Real Spectral Representation for Stationary Processes] Let $\{\xi (t), t
  \in \mathbb{R}\}$ be a real-valued, zero-mean, second-order stationary
  process with covariance function $r (t) = E [\xi (t) \xi (0)]$ and spectral
  distribution function $F (\omega)$. Then there exist real-valued processes
  $\{U (\omega), \omega \geq 0\}$ and $\{V (\omega), \omega \geq 0\}$ with
  orthogonal increments such that:
  \begin{enumerate}
    \item \tmtextbf{Process Representation:}
    \begin{equation}
      \xi (t) = \int_0^{\infty} [\cos (\omega t) \hspace{0.17em} dU (\omega) +
      \sin (\omega t) \hspace{0.17em} dV (\omega)]
    \end{equation}
    \item \tmtextbf{Covariance Representation:}
    \begin{equation}
      r (t) = \int_0^{\infty} \cos (\omega t)  \hspace{0.17em} dF (\omega)
    \end{equation}
    \item \tmtextbf{Orthogonality Properties:}
    
    \begin{align}
      E [U (\omega)] & = E [V (\omega)] = 0 \\
      E [dU (\omega_1) dU (\omega_2)] & = E [dV (\omega_1) dV (\omega_2)] =
      \delta (\omega_1 - \omega_2) dF (\omega_1) \\
      E [dU (\omega_1) dV (\omega_2)] & = 0 \quad \text{for all } \omega_1,
      \omega_2 \geq 0 
    \end{align}
  \end{enumerate}
\end{theorem}

\begin{proof}
  {\linebreak}
  \begin{enumerate}
    \item \tmtextbf{Construction from Complex Representation:} From the
    complex spectral representation theorem, we have:
    \begin{equation}
      \xi (t) = \int_{- \infty}^{\infty} e^{i \omega t} d \zeta (\omega)
    \end{equation}
    where $\zeta (\omega)$ is a complex-valued process with orthogonal
    increments.
    
    \item \tmtextbf{Reality Condition:} Since $\xi (t)$ is real-valued, we
    have $\xi (t) = \overline{\xi (t)}$, which implies:
    \begin{equation}
      \int_{- \infty}^{\infty} e^{i \omega t} d \zeta (\omega) = \int_{-
      \infty}^{\infty} e^{- i \omega t} d \overline{\zeta (\omega)}
    \end{equation}
    \item \tmtextbf{Symmetry Property:} This reality condition forces the
    spectral process to satisfy:
    \begin{equation}
      d \zeta (- \omega) = d \overline{\zeta (\omega)}
    \end{equation}
    for all $\omega$.
    
    \item \tmtextbf{Decomposition into Real and Imaginary Parts:} For $\omega
    > 0$, write
    \begin{equation}
      d \zeta (\omega) = dA (\omega) + i \hspace{0.17em} dB (\omega)
    \end{equation}
    where $dA (\omega)$ and $dB (\omega)$ are real-valued processes, and thus
    \begin{equation}
      d \zeta (- \omega) = dA (\omega) - i \hspace{0.17em} dB (\omega)
    \end{equation}
    \item \tmtextbf{Derivation of Real Spectral Representation:}
    \begin{equation}
      \begin{array}{ll}
        \xi (t) & = \int_0^{\infty} e^{i \omega t} d \zeta (\omega) +
        \int_0^{\infty} e^{- i \omega t} d \zeta (- \omega)\\
        & = \int_0^{\infty} e^{i \omega t}  [dA (\omega) + i \hspace{0.17em}
        dB (\omega)] + e^{- i \omega t}  [dA (\omega) - i \hspace{0.17em} dB
        (\omega)]\\
        & = \int_0^{\infty} [(e^{i \omega t} + e^{- i \omega t}) dA (\omega)
        + i (e^{i \omega t} - e^{- i \omega t}) dB (\omega)]\\
        & = \int_0^{\infty} 2 \cos (\omega t) dA (\omega) + 2 \sin (\omega t)
        dB (\omega)
      \end{array}
    \end{equation}
    since
    \begin{equation}
      e^{i \omega t} + e^{- i \omega t} = 2 \cos (\omega t)
    \end{equation}
    and
    \begin{equation}
      i (e^{i \omega t} - e^{- i \omega t}) = 2 \sin (\omega t)
    \end{equation}
    \item \tmtextbf{Definition of U and V:} If we define
    \begin{equation}
      dU (\omega) = 2 \hspace{0.17em} dA (\omega)
    \end{equation}
    and
    \begin{equation}
      dV (\omega) = 2 \hspace{0.17em} dB (\omega)
    \end{equation}
    then
    \begin{equation}
      \xi (t) = \int_0^{\infty} \cos (\omega t) dU (\omega) + \sin (\omega t)
      dV (\omega)
    \end{equation}
    \item \tmtextbf{Orthogonality Verification:} We have
    \begin{equation}
      E [|d \zeta (\omega) |^2] = dF (\omega)
    \end{equation}
    therefore
    \begin{equation}
      E [dA (\omega)^2] = E [dB (\omega)^2] = \frac{1}{2}  \hspace{0.17em} dF
      (\omega)
    \end{equation}
    since
    \begin{equation}
      |d \zeta (\omega) |^2 = dA (\omega)^2 + dB (\omega)^2
    \end{equation}
    thus
    \begin{equation}
      E [dU (\omega)^2] = E [dV (\omega)^2] = 4 \cdot \frac{1}{2} dF (\omega)
      = dF (\omega)
    \end{equation}
    since $dA$ and $dB$ have orthogonal increments.
    
    \item \tmtextbf{Covariance Function:} Compute the covariance:
    \begin{equation}
      \begin{array}{ll}
        r (t) & = E [\xi (t) \xi (0)]\\
        & = E \left[ \int_0^{\infty} \cos (\omega t) dU (\omega) + \sin
        (\omega t) dV (\omega) \int_0^{\infty} dU (\omega') \right]\\
        & = \int_0^{\infty} \cos (\omega t) E [dU (\omega) dU (\omega)] +
        \sin (\omega t) E [dV (\omega) dU (\omega)]\\
        & \quad + \int_0^{\infty} \cos (\omega t) E [dU (\omega) dV (\omega)]
        + \sin (\omega t) E [dV (\omega) dV (\omega)]\\
        & = \int_0^{\infty} \cos (\omega t) E [dU (\omega)^2] + \sin (\omega
        t) E [dV (\omega)^2
      \end{array}
    \end{equation}
    where all cross-terms vanish by orthogonality. Recalling
    \begin{equation}
      E [dU (\omega)^2] = E [dV (\omega)^2] = dF (\omega)
    \end{equation}
    and noting that expectation of the sine term vanishes since the mean of
    $dV (\omega)$ is zero and sine is odd; thus,
    \begin{equation}
      r (t) = \int_0^{\infty} \cos (\omega t) dF (\omega)
    \end{equation}
    as required.
  \end{enumerate}
  
\end{proof}

\begin{corollary}
  [Physical Interpretation] In the real spectral representation:
  \begin{enumerate}
    \item $\cos (\omega t) dU (\omega)$ represents the cosine component at
    frequency $\omega$ with random amplitude $dU (\omega)$.
    
    \item $\sin (\omega t) dV (\omega)$ represents the sine component at
    frequency $\omega$ with random amplitude $dV (\omega)$.
    
    \item $dF (\omega)$ represents the average power contributed by frequency
    components in $(\omega, \omega + d \omega)$.
    
    \item The processes $U (\omega)$ and $V (\omega)$ are uncorrelated and
    have equal variance increments.
  \end{enumerate}
\end{corollary}

\

\begin{theorem}
  [U and V Processes] For a real-valued stationary process $\xi (t)$ with
  spectral representation
  \begin{equation}
    \xi (t) = \int_0^{\infty} [\cos (\omega t) \hspace{0.17em} dU (\omega) +
    \sin (\omega t) \hspace{0.17em} dV (\omega)]
  \end{equation}
  the processes $U (\omega)$ and $V (\omega)$ are given explicitly by:
  \begin{enumerate}
    \item \tmtextbf{U-process formula:}
    \begin{equation}
      U (\omega) = \lim_{T \to \infty}  \frac{1}{\pi}  \int_{- T}^T \frac{1 -
      \cos (\omega t)}{t} \xi (t)  \hspace{0.17em} dt
    \end{equation}
    \item \tmtextbf{V-process formula:}
    \begin{equation}
      V (\omega) = \lim_{T \to \infty}  \frac{1}{\pi}  \int_{- T}^T \frac{\sin
      (\omega t)}{t} \xi (t)  \hspace{0.17em} dt
    \end{equation}
    \item \tmtextbf{Alternative forms using sine and cosine integrals:}
    
    \begin{align}
      U (\omega) & = \lim_{T \to \infty}  \frac{2}{\pi}  \int_0^T \frac{1 -
      \cos (\omega t)}{t} \xi (t)  \hspace{0.17em} dt \\
      V (\omega) & = \lim_{T \to \infty}  \frac{2}{\pi}  \int_0^T \frac{\sin
      (\omega t)}{t} \xi (t)  \hspace{0.17em} dt 
    \end{align}
    
    \item \tmtextbf{Incremental form:}
    
    \begin{align}
      U (\omega_2) - U (\omega_1) & = \lim_{T \to \infty}  \frac{1}{\pi} 
      \int_{- T}^T \frac{\cos (\omega_1 t) - \cos (\omega_2 t)}{t} \xi (t) 
      \hspace{0.17em} dt \\
      V (\omega_2) - V (\omega_1) & = \lim_{T \to \infty}  \frac{1}{\pi} 
      \int_{- T}^T \frac{\sin (\omega_2 t) - \sin (\omega_1 t)}{t} \xi (t) 
      \hspace{0.17em} dt 
    \end{align}
  \end{enumerate}
\end{theorem}

\begin{proof}
  \begin{enumerate}
    \item Starting from the complex inversion formula:
    \begin{equation}
      \zeta (\lambda) - \zeta (0) = \lim_{T \to \infty}  \frac{1}{2 \pi} 
      \int_{- T}^T \frac{1 - e^{- it \lambda}}{- it} \xi (t)  \hspace{0.17em}
      dt
    \end{equation}
    \item For real processes, we have the relations:
    
    \begin{align}
      d \zeta (\omega) & = \frac{1}{2}  [dU (\omega) - i \hspace{0.17em} dV
      (\omega)]  \quad \text{for } \omega > 0 \\
      d \zeta (- \omega) & = \frac{1}{2}  [dU (\omega) + i \hspace{0.17em} dV
      (\omega)]  \quad \text{for } \omega > 0 
    \end{align}
    
    \item Therefore:
    
    \begin{align}
      U (\omega) - U (0) & = 2 [\zeta (\omega) - \zeta (0)] + 2 [\zeta (-
      \omega) - \zeta (0)] \\
      V (\omega) - V (0) & = 2 i [\zeta (\omega) - \zeta (0)] - 2 i [\zeta (-
      \omega) - \zeta (0)] 
    \end{align}
    
    \item Substituting the inversion formula:
    
    \begin{align}
      U (\omega) & = \lim_{T \to \infty}  \frac{1}{\pi}  \int_{- T}^T \frac{1
      - \cos (\omega t)}{t} \xi (t)  \hspace{0.17em} dt \\
      V (\omega) & = \lim_{T \to \infty}  \frac{1}{\pi}  \int_{- T}^T
      \frac{\sin (\omega t)}{t} \xi (t)  \hspace{0.17em} dt 
    \end{align}
    
    where we used $U (0) = V (0) = 0$.
    
    \item The alternative forms follow from the fact that $\xi (t)$ is real,
    making the integrands even for $U (\omega)$ and odd for $V (\omega)$.
  \end{enumerate}
\end{proof}

\begin{remark}
  The objects $U (\omega)$ and $V (\omega)$ appearing in the real spectral
  representation of a stationary process,
  \begin{equation}
    \xi (t) = \int_0^{\infty} \cos (\omega t)  \hspace{0.17em} dU (\omega) +
    \int_0^{\infty} \sin (\omega t)  \hspace{0.17em} dV (\omega)
  \end{equation}
  are not stochastic processes in the conventional sense (indexed by time or
  evolving in time), but are more properly understood as {\tmem{random
  measures}} (or random set functions) on the frequency axis $[0, \infty)$.
  Their main property is that their increments over disjoint frequency
  intervals are orthogonal, i.e., uncorrelated (and independent if Gaussian).
  The notation $U (\omega)$ denotes the cumulative random measure up to
  frequency $\omega$:
  \begin{equation}
    U (\omega) = U ([0, \omega]) \qquad V (\omega) = V ([0, \omega])
  \end{equation}
  Thus, while legacy literature (e.g., Cram{\'e}r, Leadbetter) sometimes
  refers to them as ``processes'', in modern probability theory they are
  correctly regarded as random orthogonal-increment measures determined by the
  spectral measure of the stationary process.
\end{remark}

\end{document}
