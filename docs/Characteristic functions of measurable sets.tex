\documentclass[12pt]{article}
\usepackage{amsmath, amsthm, amssymb, mathtools}
\usepackage[T1]{fontenc}
\usepackage[margin=1in]{geometry}

\title{Characteristic Functions of Measurable Sets: Definition and Equivalence Theorem}
\author{}
\date{}

% Theorem environments
\theoremstyle{plain}
\newtheorem{theorem}{Theorem}
\newtheorem{lemma}{Lemma}
\theoremstyle{definition}
\newtheorem{definition}{Definition}
\newtheorem{remark}{Remark}

% Convenience macros
\newcommand{\R}{\mathbb{R}}
\newcommand{\B}{\mathcal{B}}
\newcommand{\Sig}{\Sigma}
\newcommand{\one}{\mathbf{1}}

\begin{document}
\maketitle

\section*{Notions and Setup}
Throughout, let \((X,\Sig)\) denote a measurable space, where \(X\) is a set and \(\Sig\) is a \(\sigma\)-algebra of subsets of \(X\).\footnote{A \(\sigma\)-algebra \(\Sig \subseteq \mathcal{P}(X)\) is a nonempty family of subsets of \(X\) that is closed under complementation and countable unions. Equivalently: (i) \(X \in \Sig\), (ii) if \(A \in \Sig\) then \(X \setminus A \in \Sig\), and (iii) if \((A_n)_{n\in\mathbb{N}} \subseteq \Sig\), then \(\bigcup_{n=1}^\infty A_n \in \Sig\).} 

The \emph{Borel \(\sigma\)-algebra} on \(\R\), denoted \(\B(\R)\) or simply \(\B\), is the \(\sigma\)-algebra generated by the open subsets of \(\R\).

\begin{definition}[Measurable function]
Let \((X,\Sig)\) be a measurable space and let \((Y,\mathcal{T})\) be a measurable space. A function \(f:X \to Y\) is said to be \((\Sig,\mathcal{T})\)\emph{-measurable} (or simply \emph{measurable} when the \(\sigma\)-algebras are understood) if for every \(B \in \mathcal{T}\), the preimage \(f^{-1}(B) \in \Sig\).
\end{definition}

In particular, for real-valued functions \(f:X \to \R\), measurability means \(f^{-1}(B) \in \Sig\) for every Borel set \(B \in \B\). It is often sufficient (and equivalent) to check \(f^{-1}((-\infty,\alpha)) \in \Sig\) for all \(\alpha \in \R\), because the family \(\{(-\infty,\alpha):\alpha\in\R\}\) generates \(\B\).

\begin{definition}[Characteristic (indicator) function]
For a subset \(E \subseteq X\), the \emph{characteristic function} (also called the \emph{indicator function}) of \(E\) is the map \(\chi_E:X \to \R\) defined by
\[
\chi_E(x) \coloneqq
\begin{cases}
1, & x \in E,\\
0, & x \notin E.
\end{cases}
\]
\end{definition}

\section*{Main Result}

\begin{theorem}[Equivalence of measurability for characteristic functions and sets]
Let \((X,\Sig)\) be a measurable space and let \(E \subseteq X\). The following are equivalent:
\begin{enumerate}
    \item \(E \in \Sig\).
    \item The characteristic function \(\chi_E: (X,\Sig) \to (\R,\B)\) is measurable.
\end{enumerate}
\end{theorem}

\begin{proof}
The proof is divided into two implications.

\medskip
\noindent
\emph{(1) \(\Rightarrow\) (2):} Assume \(E \in \Sig\). To show that \(\chi_E\) is measurable, it suffices to verify that \(\chi_E^{-1}((-\infty,\alpha)) \in \Sig\) for all \(\alpha \in \R\). For any \(x \in X\), the only possible values are \(\chi_E(x) \in \{0,1\}\). Hence, for a fixed \(\alpha \in \R\), the set \(\{x \in X : \chi_E(x) < \alpha\}\) can be determined by comparing \(\alpha\) with \(0\) and \(1\). Consider the following exhaustive cases:

\begin{itemize}
    \item If \(\alpha \le 0\), then \(\chi_E(x) \ge 0\) for all \(x\), so \(\{x : \chi_E(x) < \alpha\} = \varnothing\), which belongs to \(\Sig\) because \(\varnothing \in \Sig\).
    \item If \(0 < \alpha \le 1\), then \(\chi_E(x) < \alpha\) holds if and only if \(\chi_E(x)=0\), equivalently \(x \notin E\). Thus \(\{x : \chi_E(x) < \alpha\} = X \setminus E\), which belongs to \(\Sig\) because \(\Sig\) is closed under complementation and \(E \in \Sig\).
    \item If \(\alpha > 1\), then \(\chi_E(x) < \alpha\) for all \(x \in X\), hence \(\{x : \chi_E(x) < \alpha\} = X\), which belongs to \(\Sig\) since \(X \in \Sig\).
\end{itemize}

Since in all cases \(\chi_E^{-1}((-\infty,\alpha)) \in \Sig\), the function \(\chi_E\) is measurable.

\medskip
\noindent
\emph{(2) \(\Rightarrow\) (1):} Assume \(\chi_E\) is measurable. Consider the Borel set \((0,\infty) \in \B\). Observing that \(\chi_E(x) > 0\) holds if and only if \(x \in E\), one has
\[
E \;=\; \{x \in X : \chi_E(x) > 0\} \;=\; \chi_E^{-1}((0,\infty)).
\]
By measurability of \(\chi_E\), the preimage \(\chi_E^{-1}((0,\infty))\) belongs to \(\Sig\). Therefore \(E \in \Sig\).

\medskip
Both implications have been established, completing the proof of the equivalence.
\end{proof}

\section*{Additional Remarks}
\begin{remark}
Equivalent characterizations can be obtained using other generating families of \(\B\), for example by verifying that \(\chi_E^{-1}((-\infty,\alpha]) \in \Sig\) for all \(\alpha\in\R\), which leads to the same case analysis with the sets \(\varnothing\), \(X\setminus E\), and \(X\).
\end{remark}

\begin{remark}
For a finite or countable collection of subsets \(\{E_k\}_{k\in I}\), linear combinations of characteristic functions produce simple functions, which are measurable if and only if each \(E_k\) is measurable. This underlies the construction of measurable simple functions and, subsequently, the development of the Lebesgue integral.
\end{remark}

\end{document}


