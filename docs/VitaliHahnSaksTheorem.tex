\documentclass{article}
\usepackage[english]{babel}
\usepackage{geometry,amsmath,amssymb,latexsym}
\geometry{letterpaper}

%%%%%%%%%% Start TeXmacs macros
\newcommand{\tmtextbf}[1]{\text{{\bfseries{#1}}}}
\newenvironment{proof}{\noindent\textbf{Proof\ }}{\hspace*{\fill}$\Box$\medskip}
\newtheorem{lemma}{Lemma}
\newtheorem{theorem}{Theorem}
%%%%%%%%%% End TeXmacs macros

\begin{document}

\title{Proof of the Vitali--Hahn--Saks Theorem}

\date{}

\maketitle

\begin{theorem}
  [Vitali--Hahn--Saks] Let $(X, \Sigma)$ be a measurable space, and let
  $\{\mu_n \}$ be a sequence of finite measures on $(X, \Sigma)$. Suppose that
  for every set $E$ in $\Sigma$, the limit $\lim_{n \to \infty} \mu_n (E)$
  exists (finite or infinite). Then:
  \begin{enumerate}
    \item There exists a measure $\mu$ on $(X, \Sigma)$ such that for every
    $E$ in $\Sigma$: $\mu (E) = \lim_{n \to \infty} \mu_n (E)$
    
    \item The sequence of measures $\{\mu_n \}$ is uniformly absolutely
    continuous with respect to $\mu$.
    
    \item The convergence of $\mu_n$ to $\mu$ is uniform on $\Sigma$.
  \end{enumerate}
\end{theorem}

\begin{proof}
  \tmtextbf{Step 1: Define the limit measure $\mu$}
  
  For each $E \in \Sigma$, define $\mu (E) = \lim_{n \to \infty} \mu_n (E)$.
  We need to show that $\mu$ is indeed a measure.
  
  a) Clearly, $\mu (\emptyset) = \lim_{n \to \infty} \mu_n (\emptyset) = 0$.
  
  b) Countable additivity: Let $\{E_k \}$ be a sequence of disjoint sets in
  $\Sigma$. We need to show that $\mu (\bigcup_k E_k) = \sum_k \mu (E_k)$.
  
  \begin{align*}
    \mu (\bigcup_k E_k) & = \lim_{n \to \infty} \mu_n (\bigcup_k E_k)\\
    & = \lim_{n \to \infty}  \sum_k \mu_n (E_k) \quad \text{(by countable
    additivity of $\mu_n$)}\\
    & = \sum_k \lim_{n \to \infty} \mu_n (E_k) \quad \text{(by the monotone
    convergence theorem)}\\
    & = \sum_k \mu (E_k)
  \end{align*}
  
  Thus, $\mu$ is a measure on $(X, \Sigma)$.
  
  \tmtextbf{Step 2: Prove uniform absolute continuity}
  
  We'll use the following lemma:
  
  \begin{lemma}
    For any $\varepsilon > 0$, there exists a $\delta > 0$ such that for all
    $n$ and all $E \in \Sigma$: If $\mu (E) < \delta$, then $\mu_n (E) <
    \varepsilon$.
  \end{lemma}
  
  Suppose the lemma is false. Then there exists an $\varepsilon > 0$ such that
  for every $k \in \mathbb{N}$, we can find $n_k$ and $E_k \in \Sigma$ with
  $\mu (E_k) < 1 / k$ and $\mu_{n_k} (E_k) \geq \varepsilon$.
  
  Define $F_k = \bigcup_{j \geq k} E_j$. Then $F_k \supseteq F_{k + 1}$ and
  $\mu (F_k) \leq \sum_{j \geq k} \mu (E_j) < \sum_{j \geq k} 1 / j \to 0$ as
  $k \to \infty$.
  
  But for any $k$, $\mu_{n_k} (F_k) \geq \mu_{n_k} (E_k) \geq \varepsilon$.
  
  This contradicts the fact that $\lim_{n \to \infty} \mu_n (F_k) = \mu (F_k)$
  for all $k$.
\end{proof}

\tmtextbf{Step 3: Prove uniform convergence}

We'll use Egoroff's theorem, which states that if a sequence of measurable
functions converges almost everywhere on a finite measure space, then it
converges uniformly except on a set of arbitrarily small measure.

For each $E \in \Sigma$, define $f_E (n) = \mu_n (E)$. The sequence $\{f_E
(n)\}$ converges for each $E$.

Let $\varepsilon > 0$. By the uniform absolute continuity proved in Step 2,
there exists a $\delta > 0$ such that $\mu (A) < \delta$ implies $\mu_n (A) <
\varepsilon / 3$ for all $n$.

Let $M = \mu (X)$. Choose a finite partition $\{P_1, ..., P_k \}$ of $X$ with
$\mu (P_i) < \delta$ for all $i$.

For each $P_i$, the sequence $f_{P_i} (n)$ converges. By Egoroff's theorem,
there exists $A_i \subset P_i$ with $\mu (A_i) < \delta / k$ such that
$f_{P_i} (n)$ converges uniformly on $P_i \setminus A_i$.

Let $A = \bigcup_i A_i$. Then $\mu (A) < \delta$, so $\mu_n (A) < \varepsilon
/ 3$ for all $n$.

For each $i$, choose $N_i$ such that for $n, m \geq N_i$, $|f_{P_i \setminus
A_i} (n) - f_{P_i \setminus A_i} (m) | < \varepsilon / 3 k$.

Let $N = \max \{N_i \}$. Then for $n, m \geq N$ and any $E \in \Sigma$:

\begin{align*}
  | \mu_n (E) - \mu_m (E) | & \leq | \mu_n (E \cap A) - \mu_m (E \cap A) | +
  \sum_i | \mu_n (E \cap (P_i \setminus A_i)) - \mu_m (E \cap (P_i \setminus
  A_i)) |\\
  & < \varepsilon / 3 + \varepsilon / 3 + \varepsilon / 3 = \varepsilon
\end{align*}

This proves uniform convergence.

Therefore, we have proved all three parts of the Vitali--Hahn--Saks theorem.

\end{document}
