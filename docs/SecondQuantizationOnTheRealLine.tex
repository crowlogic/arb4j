\documentclass[11pt]{article}
\usepackage{amsmath,amsthm,amssymb,amsfonts,mathtools}
\usepackage[a4paper,margin=1in]{geometry}
\usepackage[T1]{fontenc}
\usepackage{lmodern}
\usepackage{microtype}
\usepackage{hyperref}

\title{Second Quantization for a Free Particle on the Real Line: A Harmonic Oscillator Type Factorization}
\author{}
\date{}

\theoremstyle{definition}
\newtheorem{definition}{Definition}
\newtheorem{remark}{Remark}

\theoremstyle{plain}
\newtheorem{theorem}{Theorem}
\newtheorem{proposition}{Proposition}
\newtheorem{lemma}{Lemma}
\newtheorem{corollary}{Corollary}

\numberwithin{equation}{section}

\begin{document}
\maketitle

\begin{abstract}
A construction is presented in which the Hamiltonian of a non-relativistic free particle on the real line is realized on a bosonic Fock space as a factorized sum of commuting number operators, each associated with a one-particle momentum mode and acting as a harmonic oscillator energy observable without an additive ground-state shift. Definitions of the underlying Hilbert spaces, operators, domains, and commutation relations are given. The main statements identify the one-particle Hamiltonian as a multiplication operator in momentum representation and show that its many-body second quantization equals the sum of mode energies times number operators. Proofs rely on the Fourier transform, the spectral theorem for self-adjoint operators unitarily equivalent to multiplication operators, and the functorial construction of creation and annihilation operators on bosonic Fock space. Canonical references include standard treatments of Fock space and second quantization as well as expositions of the Fourier transform and the Stone--von Neumann uniqueness of the canonical commutation relations.
\end{abstract}

\tableofcontents

\section{One-Particle Structure on the Real Line}

\subsection{Hilbert space, Fourier transform, and momentum representation}

\begin{definition}[One-particle Hilbert space]
Let $\mathcal{H}_1 := L^2(\mathbb{R},dx)$ with inner product $\langle \psi,\varphi\rangle=\int_{\mathbb{R}}\overline{\psi(x)}\,\varphi(x)\,dx$.
\end{definition}

\begin{definition}[Fourier transform]
Let $\mathcal{F}:L^2(\mathbb{R})\to L^2(\mathbb{R})$ denote the unitary Fourier transform
\[
(\mathcal{F}\psi)(p)=\frac{1}{\sqrt{2\pi\hbar}}\int_{\mathbb{R}}e^{-ipx/\hbar}\psi(x)\,dx,
\]
defined first on $L^1(\mathbb{R})\cap L^2(\mathbb{R})$ and extended by density and unitarity.\footnote{See, e.g., standard spectral theory notes: the Fourier transform is unitary and diagonalizes translation operators, and it identifies $-\frac{\hbar^2}{2m}\Delta$ with multiplication by $\frac{p^2}{2m}$ \cite[Thm.~10]{harvardfourier}, \cite[Sec.~1.6]{kowalskispectral}, \cite[Sec.~2]{feldmanlaplacian}.}
\]
\end{definition}

\begin{definition}[Free one-particle Hamiltonian]
Let
\[
H_1:= -\frac{\hbar^2}{2m}\,\frac{d^2}{dx^2}
\]
with domain $\mathcal{D}(H_1):=H^2(\mathbb{R})\subset L^2(\mathbb{R})$.
\end{definition}

\begin{theorem}[Momentum representation of $H_1$]
Under $\mathcal{F}$, the operator $H_1$ is unitarily equivalent to multiplication by $\varepsilon(p):=\frac{p^2}{2m}$ on $L^2(\mathbb{R},dp)$ with domain
\[
\mathcal{D}(\varepsilon):=\Big\{ \widehat{\psi}\in L^2(\mathbb{R}) \ \big|\ \varepsilon\,\widehat{\psi}\in L^2(\mathbb{R})\Big\},
\]
so that $(\mathcal{F}H_1\mathcal{F}^{-1}\widehat{\psi})(p)=\varepsilon(p)\widehat{\psi}(p)$ for $\widehat{\psi}\in\mathcal{D}(\varepsilon)$.\footnote{Self-adjointness and identification as a multiplication operator via the Fourier transform can be found in standard sources \cite[Sec.~1.6]{kowalskispectral}, \cite[pp.~3--4]{feldmanlaplacian}.}
\end{theorem}

\begin{proof}
It is known that $H_1$ with domain $H^2(\mathbb{R})$ is self-adjoint on $L^2(\mathbb{R})$. The Fourier transform $\mathcal{F}$ is unitary on $L^2(\mathbb{R})$ and maps derivatives to multiplications by $(ip/\hbar)$. Hence $\mathcal{F}\circ \frac{d^2}{dx^2}\circ \mathcal{F}^{-1}$ equals multiplication by $-(p/\hbar)^2$. Therefore,
\[
\mathcal{F}H_1\mathcal{F}^{-1}=\mathcal{F}\left(-\frac{\hbar^2}{2m}\frac{d^2}{dx^2}\right)\mathcal{F}^{-1}=\frac{p^2}{2m}\,,
\]
as a multiplication operator on $L^2(\mathbb{R})$ with the stated domain. The domain identification follows from standard facts about multiplication operators: the domain consists of those $\widehat{\psi}$ such that $\varepsilon\,\widehat{\psi}\in L^2(\mathbb{R})$. Unitarity of $\mathcal{F}$ yields the claim.\end{proof}

\subsection{Spectral resolution}
\begin{proposition}
The spectral measure of $H_1$ is the push-forward of Lebesgue measure on $\mathbb{R}$ under $p\mapsto \varepsilon(p)=p^2/(2m)$, realized via $\mathcal{F}$ as the spectral measure of the multiplication operator $\varepsilon(p)$.\footnote{For the spectral theorem description of self-adjoint operators as multiplication operators, see standard references \cite[Ch.~2]{kowalskispectral}, \cite[Sec.~2]{taylorst}, \cite[Lecture~1]{danaspec}.}
\end{proposition}

\begin{proof}
Since $H_1$ is unitarily equivalent to the multiplication operator $M_{\varepsilon}$ on $L^2(\mathbb{R},dp)$, the projection-valued measure of $H_1$ is $\mathcal{F}^{-1}E_{\varepsilon}\mathcal{F}$, where $E_{\varepsilon}$ is the spectral measure of $M_{\varepsilon}$. The latter is determined by measurable subsets of $\mathbb{R}$ via multiplication by their indicator functions. The push-forward statement follows from the standard form of the spectral theorem for multiplication operators.\end{proof}

\section{Bosonic Fock Space and Mode Operators}

\subsection{Bosonic Fock space}

\begin{definition}[Bosonic Fock space]
Let $\mathfrak{h}:=L^2(\mathbb{R},dp)$ and define the bosonic Fock space
\[
\mathcal{F}_+(\mathfrak{h}):=\bigoplus_{n=0}^{\infty} \mathfrak{h}^{\otimes_s n},
\]
where $\otimes_s$ denotes the symmetric tensor product and the $0$-particle space is $\mathbb{C}\Omega$ with vacuum vector $\Omega$.\footnote{Standard constructions appear in many introductions to second quantization \cite[Sec.~1.4]{uiuc561}, \cite[Sec.~3]{usp-3}, \cite[Sec.~4.3]{libresecond}, \cite{scholarpedia}.}
\end{definition}

\begin{definition}[Creation and annihilation operators]
For $f\in\mathfrak{h}$ define $a^\dagger(f)$ and $a(f)$ on the algebraic finite particle subspace by the usual ladder operations:
\begin{align*}
a^\dagger(f)(\psi^{(n)})&:=\sqrt{n+1}\,\mathrm{Sym}(f\otimes \psi^{(n)}),\\
a(f)(\psi^{(n)})&:=\sqrt{n}\,\langle f,\cdot\rangle_{\mathfrak{h}}\ \text{contracted against one slot of } \psi^{(n)}.
\end{align*}
They extend to closed operators satisfying the canonical commutation relations
\[
[a(f),a^\dagger(g)]=\langle f,g\rangle_{\mathfrak{h}}\ \mathrm{id},\qquad [a(f),a(g)]=[a^\dagger(f),a^\dagger(g)]=0.
\]
\end{definition}

\begin{definition}[Field of mode operators in momentum representation]
For $f\in\mathfrak{h}$ define the operator-valued distributions by
\[
a(f)=\int_{\mathbb{R}} \overline{f(p)}\,a(p)\,dp,\qquad a^\dagger(f)=\int_{\mathbb{R}} f(p)\,a^\dagger(p)\,dp,
\]
in the sense that $[a(p),a^\dagger(q)]=\delta(p-q)\,\mathrm{id}$ when tested against functions in $\mathfrak{h}$.\footnote{Occupation-number representations are developed in many sources \cite[Sec.~1.4]{uiuc561}, \cite{scholarpedia}, \cite[Sec.~4.3]{libresecond}.}
\end{definition}

\subsection{Number operator and occupation numbers}

\begin{definition}[Number operator]
Define $N:=\int_{\mathbb{R}} a^\dagger(p)a(p)\,dp$ on its natural quadratic form domain. For any orthonormal set $\{e_k\}\subset\mathfrak{h}$ with $a_k:=a(e_k)$, $N=\sum_k a_k^\dagger a_k$ in the strong resolvent sense on the finite-particle subspace.
\end{definition}

\begin{proposition}[Occupation number basis]
Let $\{e_k\}$ be an orthonormal basis of $\mathfrak{h}$. Vectors
\[
|n_1,n_2,\dots\rangle:=\prod_{k} \frac{(a^\dagger(e_k))^{n_k}}{\sqrt{n_k!}}\ \Omega
\]
with finitely many nonzero $n_k$ span a dense subspace, and $N|n_1,n_2,\dots\rangle=\left(\sum_k n_k\right)|n_1,n_2,\dots\rangle$.\footnote{See, e.g., textbook expositions of Fock space and occupation numbers \cite[Sec.~1.4]{uiuc561}, \cite[Sec.~4.3]{libresecond}, \cite{scholarpedia}.}
\end{proposition}

\begin{proof}
On the algebraic Fock space generated by monomials in $a^\dagger(e_k)$ acting on $\Omega$, the stated vectors form the standard occupation number basis with orthonormality following from the CCR and the vacuum relations. The finite linear span is dense by construction of $\mathcal{F}_+(\mathfrak{h})$ as the Hilbert completion. The action of $N$ follows from $[N,a^\dagger(e_k)]=a^\dagger(e_k)$ and $N\Omega=0$.\end{proof}

\section{Second Quantization of the Free Hamiltonian}

\subsection{One-body operator as an energy multiplier}

\begin{definition}[One-body energy function]
Let $\varepsilon(p):=\frac{p^2}{2m}$ viewed as a measurable function on $\mathbb{R}$ and let $h:=M_{\varepsilon}$ be the multiplication operator on $\mathfrak{h}=L^2(\mathbb{R},dp)$ with domain
\[
\mathcal{D}(h):=\{f\in L^2(\mathbb{R}) \mid \varepsilon f\in L^2(\mathbb{R})\}.
\]
By the previous section, $h$ is unitarily equivalent to $H_1$ via $\mathcal{F}$.\footnote{See the Fourier-transform characterization of the Laplacian and its functional calculus \cite[Sec.~1.6]{kowalskispectral}, \cite[pp.~3--4]{feldmanlaplacian}.}
\end{definition}

\subsection{Many-body Hamiltonian as a factorized sum over modes}

\begin{definition}[Second quantization of $h$]
Define the operator $d\Gamma(h)$ on $\mathcal{F}_+(\mathfrak{h})$ by its action on the $n$-particle sector as
\[
d\Gamma(h)\big|_{\mathfrak{h}^{\otimes_s n}}:=\sum_{j=1}^n I\otimes\cdots\otimes \underbrace{h}_{j\text{th}}\otimes\cdots\otimes I,
\]
with the natural domain consisting of finite vectors with each component in the corresponding domain and closure taken in the usual way.\footnote{Standard definitions of second quantization of one-body operators can be found in many references \cite[Sec.~1.4]{uiuc561}, \cite[Sec.~3]{usp-3}, \cite{scholarpedia}.}
\end{definition}

\begin{theorem}[Mode factorization of the many-body free Hamiltonian]
Let $\mathfrak{h}=L^2(\mathbb{R},dp)$ and $h=M_{\varepsilon}$ with $\varepsilon(p)=\frac{p^2}{2m}$. Then on the finite-particle subspace,
\[
d\Gamma(h)=\int_{\mathbb{R}}\varepsilon(p)\,a^\dagger(p)a(p)\,dp,
\]
and for any orthonormal basis $\{e_k\}$ diagonalizing $h$ (in the sense of $\mathfrak{h}$-multiplication),
\[
d\Gamma(h)=\sum_k \lambda_k\, a_k^\dagger a_k,\qquad \text{with } a_k:=a(e_k),\ h e_k=\lambda_k e_k.
\]
In particular, in the momentum representation,
\[
H:=d\Gamma(h)=\int_{\mathbb{R}}\frac{p^2}{2m}\,a^\dagger(p)a(p)\,dp
\]
is a densely defined self-adjoint operator bounded from below by $0$.\footnote{The identity $d\Gamma(h)=\sum \lambda_k a_k^\dagger a_k$ is standard; see discussions of one-body operators in second quantization \cite[Sec.~1.4]{uiuc561}, \cite[Sec.~3]{usp-3}, \cite{scholarpedia}.}
\end{theorem}

\begin{proof}
Let $\mathcal{D}_{\mathrm{fin}}$ be the algebraic finite-particle subspace spanned by finite monomials in creation operators applied to $\Omega$, with single-particle vectors in $\mathcal{D}(h)$. On an $n$-particle vector $\Psi^{(n)}\in \mathfrak{h}^{\otimes_s n}$,
\[
d\Gamma(h)\Psi^{(n)}=\sum_{j=1}^n h_j\Psi^{(n)}.
\]
For simple tensors $\Psi^{(n)}=f_1\otimes_s\cdots\otimes_s f_n$,
\[
\langle \Psi^{(n)}, d\Gamma(h)\Psi^{(n)}\rangle=\sum_{j=1}^n \langle f_j, h f_j\rangle_{\mathfrak{h}}.
\]
Since $h$ is the multiplication by $\varepsilon$, $\langle f, h f\rangle=\int \varepsilon(p)|f(p)|^2\,dp$. Using the well-known identities
\[
\int a^\dagger(p)a(p)\,dp = N,\qquad a^\dagger(f)a(g)=\int \overline{g(p)}f(p)\,a^\dagger(p)a(p)\,dp,
\]
valid as quadratic forms on $\mathcal{D}_{\mathrm{fin}}$, one finds
\[
\sum_{j=1}^n h_j = \int \varepsilon(p)\,a^\dagger(p)a(p)\,dp
\]
on $\mathcal{D}_{\mathrm{fin}}$. If $h e_k=\lambda_k e_k$ with an orthonormal family $\{e_k\}$ (e.g., an $L^2$-basis diagonalizing multiplication), then by expanding $a^\dagger(f)=\sum_k \langle e_k,f\rangle a_k^\dagger$ and the bilinearity of quadratic forms,
\[
\int \varepsilon(p)\,a^\dagger(p)a(p)\,dp=\sum_k \lambda_k a_k^\dagger a_k
\]
on $\mathcal{D}_{\mathrm{fin}}$. Closability and self-adjointness follow from the Kato--Rellich framework adapted to second quantization of nonnegative $h$ together with the standard fact that $d\Gamma(h)$ is self-adjoint on the natural domain and satisfies $\langle \Psi, d\Gamma(h)\Psi\rangle\ge 0$ for $\Psi$ in its form domain. The closure is independent of the representing orthonormal system.\end{proof}

\subsection{Harmonic oscillator type structure in each mode}

\begin{proposition}[Single-mode oscillator structure]
Let $e\in\mathfrak{h}$ with $\|e\|=1$ and define $a:=a(e)$, $a^\dagger:=a^\dagger(e)$, and $\lambda:=\langle e, h e\rangle = \int \varepsilon(p)|e(p)|^2\,dp$. Then on the algebra generated by $a^\dagger$ on $\Omega$,
\[
[d\Gamma(h),a^\dagger]=\lambda\,a^\dagger,\qquad [d\Gamma(h),a]=-\lambda\,a,\qquad d\Gamma(h)=\lambda\,a^\dagger a + R,
\]
where $R$ commutes with $a$ and $a^\dagger$. Consequently the energies along the tower $\{(a^\dagger)^n\Omega\}_{n\ge 0}$ increase by $\lambda$ per quantum and the contribution of this mode to $H$ equals $\lambda\,a^\dagger a$.\footnote{See the standard ladder-commutator computation for one-body operators in Fock space \cite[Sec.~1.4]{uiuc561}, \cite{scholarpedia}, \cite[Sec.~4.3]{libresecond}.}
\end{proposition}

\begin{proof}
Using $[a(f),a^\dagger(g)]=\langle f,g\rangle$, one computes $[d\Gamma(h),a^\dagger(f)]=a^\dagger(hf)$ on $\mathcal{D}_{\mathrm{fin}}$. Taking $f=e$ with $he=\lambda e$ (e.g., if $e$ is an eigenvector, or in general by decomposing relative to a spectral resolution and restricting to an eigenspace component), yields $[d\Gamma(h),a^\dagger]=\lambda a^\dagger$ and by adjoint $[d\Gamma(h),a]=-\lambda a$. The decomposition $d\Gamma(h)=\lambda a^\dagger a + R$ follows by writing $d\Gamma(h)$ as the sum of its part along the one-dimensional subspace spanned by $e$ and its part on the orthogonal complement, which commutes with $a$ and $a^\dagger$ because $a$ and $a^\dagger$ only change occupation in the $e$-mode.\end{proof}

\begin{corollary}[Hamiltonian as a sum over oscillator number operators]
Let $\{e_k\}$ be an orthonormal basis diagonalizing $h$ with $h e_k=\lambda_k e_k$. Then
\[
H=d\Gamma(h)=\sum_k \lambda_k\,a_k^\dagger a_k,
\]
and each $k$-mode contributes $\lambda_k$ per particle with no additive constant.\footnote{Compare with the standard harmonic oscillator Hamiltonian written in terms of number operators; here the ground energy shift is absent because $\varepsilon(p)$ arises from a one-body kinetic energy rather than a single mechanical oscillator frequency \cite[Sec.~3]{usp-3}, \cite{scholarpedia}.}
\end{corollary}

\begin{proof}
This is the discrete-mode restatement of the theorem above.\end{proof}

\section{Free Field Representation on Fock Space}

\subsection{Plane-wave label and occupation numbers}

\begin{definition}[Plane-wave label in momentum representation]
Label single-particle modes by $p\in\mathbb{R}$, with $a^\dagger(p)$ creating a particle with momentum $p$ and $a(p)$ removing one, in the sense that for $f\in L^2(\mathbb{R})$,
\[
a^\dagger(f)=\int f(p)\,a^\dagger(p)\,dp,\qquad a(f)=\int \overline{f(p)}\,a(p)\,dp.
\]
\end{definition}

\begin{theorem}[Free many-body Hamiltonian as an energy-weighted number operator]
On the finite-particle subspace,
\[
H=\int_{\mathbb{R}}\varepsilon(p)\,a^\dagger(p)a(p)\,dp,\qquad \varepsilon(p)=\frac{p^2}{2m}.
\]
For vectors with finite occupation in measurable momentum sets, the energy is the integral of $\varepsilon$ weighted by the occupation density.\footnote{See \cite[Sec.~1.4]{uiuc561}, \cite{scholarpedia}, \cite[Sec.~4.3]{libresecond}.}
\end{theorem}

\begin{proof}
The statement is the continuous-mode version of $d\Gamma(h)=\sum_k \lambda_k a_k^\dagger a_k$, with $\lambda_k$ replaced by $\varepsilon(p)$ and the sum replaced by the integral against the spectral measure; the proof follows from the quadratic form computation in the proof of the earlier theorem, using the facts about $a^\dagger(f)a(g)$ and the bilinear pairing in $\mathfrak{h}$.\end{proof}

\subsection{Commutation structure and dynamics}

\begin{proposition}[Heisenberg evolution of modes]
Let $U(t):=e^{-itH/\hbar}$. Then for suitable domains,
\[
U(t)^\dagger a(p)\,U(t)=e^{-it\,\varepsilon(p)/\hbar} a(p),\qquad U(t)^\dagger a^\dagger(p)\,U(t)=e^{+it\,\varepsilon(p)/\hbar} a^\dagger(p).
\]
\end{proposition}

\begin{proof}
Using $H=\int \varepsilon(q)a^\dagger(q)a(q)\,dq$ and $[a(p),a^\dagger(q)]=\delta(p-q)$,
\[
[H,a(p)]=-\varepsilon(p)a(p),\qquad [H,a^\dagger(p)]=+\varepsilon(p)a^\dagger(p)
\]
as quadratic forms on $\mathcal{D}_{\mathrm{fin}}$. Stone’s theorem for the one-parameter unitary group $U(t)$ generated by $H$ gives the Heisenberg equations $\frac{d}{dt} a_t(p)=\frac{i}{\hbar}[H,a_t(p)]$, yielding the claimed exponentials.\end{proof}

\section{Field Representation in Position Space}

\subsection{Configuration representation via inverse Fourier transform}

\begin{definition}[Configuration-space field operators]
Define operator-valued distributions by
\[
\psi(x):=\frac{1}{\sqrt{2\pi\hbar}}\int_{\mathbb{R}}e^{ipx/\hbar}\,a(p)\,dp,\qquad
\psi^\dagger(x):=\frac{1}{\sqrt{2\pi\hbar}}\int_{\mathbb{R}}e^{-ipx/\hbar}\,a^\dagger(p)\,dp.
\]
Then $[\psi(x),\psi^\dagger(y)]=\delta(x-y)$ and $[\psi(x),\psi(y)]=[\psi^\dagger(x),\psi^\dagger(y)]=0$ in the sense of distributions.\footnote{The construction follows the usual passage between momentum and configuration labels \cite[Sec.~1.4]{uiuc561}, \cite{scholarpedia}, \cite[Sec.~4.3]{libresecond}.}
\end{definition}

\begin{theorem}[Hamiltonian in configuration variables]
Formally,
\[
H=\int_{\mathbb{R}} \psi^\dagger(x)\left(-\frac{\hbar^2}{2m}\frac{d^2}{dx^2}\right)\psi(x)\,dx,
\]
with quadratic form domain specified by the requirement that $x\mapsto \psi(x)\Psi$ lie in $H^1_{\mathrm{loc}}$ and $H\Psi$ be in the Fock space. On the dense subspace of finite-particle vectors with one-particle components in $H^2(\mathbb{R})$, the right-hand side equals $d\Gamma(H_1)$.\footnote{The identification of $H$ as the second quantization of $H_1$ in position variables is standard \cite[Sec.~1.4]{uiuc561}, \cite{scholarpedia}, \cite[Sec.~4.3]{libresecond}.}
\end{theorem}

\begin{proof}
Substitute the inverse Fourier expressions for $\psi$ and $\psi^\dagger$ into the momentum-space expression $H=\int \varepsilon(p)a^\dagger(p)a(p)\,dp$ and integrate against the exponential kernels to obtain the operator $-\frac{\hbar^2}{2m}\partial_x^2$ acting between $\psi^\dagger$ and $\psi$. The domain statement mirrors the one-particle identification and the second quantization of $H_1$.\end{proof}

\section{Stone--von Neumann Uniqueness and Representation Choice}

\begin{theorem}[Uniqueness of the CCR in one degree of freedom]
For a system with one degree of freedom and irreducibility and regularity assumptions, the representation of the Weyl canonical commutation relations is unique up to unitary equivalence. In particular, the Schr\"odinger representation on $L^2(\mathbb{R})$ and the momentum representation are unitarily equivalent.\footnote{See the Stone--von Neumann theorem: classical sources include \cite{svn-wiki} and expository mathematical treatments \cite{svnhistory}, \cite{summerssvn}.}
\end{theorem}

\begin{proof}
The Weyl relations $W(\xi)W(\eta)=e^{-i\sigma(\xi,\eta)/2}W(\xi+\eta)$ on a finite-dimensional symplectic space admit a unique (up to unitary equivalence) irreducible, strongly continuous representation. For one degree of freedom this identifies the Schr\"odinger representation on $L^2(\mathbb{R})$ as unique up to unitary equivalence. Therefore the momentum and configuration realizations are unitarily equivalent, and the corresponding Fock constructions based on $L^2(\mathbb{R})$ agree up to unitary equivalence.\end{proof}

\section{Summary of the Factorization Picture}

The operator $H_1$ on $L^2(\mathbb{R})$ corresponds under the Fourier transform to multiplication by $\varepsilon(p)=p^2/(2m)$. The bosonic Fock space over $L^2(\mathbb{R})$ is equipped with creation and annihilation operators obeying the canonical commutation relations. The second quantization $H=d\Gamma(h)$ acts as an integral of $\varepsilon(p)$ against the occupation number density $a^\dagger(p)a(p)$, hence as a factorized sum of independent harmonic-oscillator-type contributions for each momentum label, with energy per quantum $\varepsilon(p)$ and no additive ground-state term. The unitary dynamics multiplies each mode by a phase $e^{-it\varepsilon(p)/\hbar}$, and the position-space representation is obtained by inverse Fourier transform, giving $H=\int \psi^\dagger(x)\left(-\frac{\hbar^2}{2m}\frac{d^2}{dx^2}\right)\psi(x)\,dx$ on the natural domain.

\section*{References}
\vspace{-0.5em}
\begin{itemize}
\item E. Fradkin, lecture notes PHYS 561: Second Quantization, University of Illinois (one-body to many-body, Fock space, and number operators) \cite{uiuc561}.
\item Scholarpedia article: Second quantization (construction of Fock space and creation/annihilation operators) \cite{scholarpedia}.
\item Physics LibreTexts: Second Quantization (Fock space overview) \cite{libresecond}.
\item Expositions on Fourier transform and the free Laplacian as a multiplication operator: ETH notes by Kowalski \cite{kowalskispectral}, and UBC notes by Feldman \cite{feldmanlaplacian}.
\item Stone--von Neumann theorem: selective history (Lionel) \cite{svnhistory}; Wikipedia overview \cite{svn-wiki}; Summers' expository notes \cite{summerssvn}.
\end{itemize}

\bigskip

\begin{thebibliography}{9}

\bibitem{uiuc561}
Second Quantization (PHYS 561 notes). University of Illinois. Available as a PDF. 

\bibitem{usp-3}
Second Quantization (notes by G. T. Landi). Section on non-interacting systems and $H=\sum_k \frac{k^2}{2m} a_k^\dagger a_k$.

\bibitem{libresecond}
4.3: Second Quantization. Physics LibreTexts.

\bibitem{scholarpedia}
Second quantization. Scholarpedia.

\bibitem{harvardfourier}
Let's solve Schrödinger's equation for a particle on $\mathbb{R}$ with different boundary conditions (Fourier transform theorem and unitarity).

\bibitem{kowalskispectral}
E. Kowalski, Spectral theory in Hilbert spaces (ETH Z\"urich, FS 09).

\bibitem{feldmanlaplacian}
The Spectrum of Periodic Schr\"odinger Operators (Fourier transform unitarity and identification of $-\frac{1}{2m}\Delta$ with a multiplication operator).

\bibitem{svn-wiki}
Stone--von Neumann theorem (overview).

\bibitem{svnhistory}
A Selective History of the Stone--von Neumann Theorem (Lionel).

\bibitem{summerssvn}
Summers, On the Stone--von Neumann Uniqueness Theorem and Its Ramifications.

\bibitem{taylorst}
M. E. Taylor, The Spectral Theorem for Self-Adjoint and Unitary Operators.

\bibitem{danaspec}
Lecture Notes on the Spectral Theorem (Dana Williams).

\end{thebibliography}

\end{document}
