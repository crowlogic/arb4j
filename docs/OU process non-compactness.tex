\documentclass{article}
\usepackage{amsmath}
\usepackage{amssymb}
\usepackage{amsthm}

\newtheorem{theorem}{Theorem}

\title{Spectral Analysis of the Ornstein-Uhlenbeck Process on $\mathbb{R}$}
\author{}
\date{}

\begin{document}

\maketitle

\section{Covariance Function and Spectral Density}

The covariance function of the Ornstein-Uhlenbeck (OU) process is given by:

\[ C(x) = \sigma^2 e^{-\alpha|x|} \]

where $\sigma^2$ is the variance and $\alpha > 0$ is the mean reversion rate.

The spectral density $S(\omega)$ is the Fourier transform of $C(x)$:

\begin{align*}
S(\omega) &= \int_{-\infty}^{\infty} C(x) e^{-i\omega x} dx \\
&= \sigma^2 \int_{0}^{\infty} e^{-(
alpha+i\omega)x} dx + \sigma^2 \int_{0}^{\infty} e^{-(
alpha-i\omega)x} dx \\
&= \sigma^2 \left[\frac{1}{\alpha+i\omega} + \frac{1}{\alpha-i\omega}\right] \\
&= \frac{2\sigma^2\alpha}{\alpha^2+\omega^2}
\end{align*}

\section{Orthogonal Polynomials}

The polynomials orthogonal with respect to the spectral density $S(\omega)$ are related to the Routh-Romanovski polynomials. Let $x = \omega/\alpha$, then the weight function becomes:

\[ w(x) = \frac{1}{1+x^2} \]

The Routh-Romanovski polynomials $R_n(x)$ are defined by the recurrence relation:

\begin{align*}
R_0(x) &= 1 \\
R_1(x) &= x \\
R_{n+1}(x) &= x R_n(x) - n^2 R_{n-1}(x) \quad \text{for } n \geq 1
\end{align*}

These polynomials satisfy the orthogonality relation:

\[ \int_{-\infty}^{\infty} R_m(x) R_n(x) w(x) dx = k_n \delta_{mn} \]

where $k_n$ is a normalization constant and $\delta_{mn}$ is the Kronecker delta.

The first few polynomials are:

\begin{align*}
R_0(x) &= 1 \\
R_1(x) &= x \\
R_2(x) &= x^2 - 1 \\
R_3(x) &= x^3 - 3x \\
R_4(x) &= x^4 - 6x^2 + 3
\end{align*}

The polynomials orthogonal with respect to $S(\omega)$ are:

\[ P_n(\omega) = R_n(\omega/\alpha) \]

\section{Fourier Transforms of Orthogonal Polynomials}

The Fourier transforms of the first few Routh-Romanovski polynomials are:

\begin{align*}
r_0(t) &= \sqrt{2\pi} \delta(t) \\
r_1(t) &= i\sqrt{2\pi/\alpha} \frac{d}{dt} [e^{-\alpha|t|}] \\
r_2(t) &= -\sqrt{2\pi/\alpha^2} \frac{d^2}{dt^2} [e^{-\alpha|t|}] - \sqrt{2\pi} \delta(t)
\end{align*}

where $\delta(t)$ is the Dirac delta function.

\section{Entropy Integral and Non-Compactness}

To show that the covariance operator of the OU process is not compact on $L^2(\mathbb{R})$, we analyze the $\epsilon$-entropy integral.

The $\epsilon$-covering number $N(\epsilon)$ is related to the spectral density:

\[ N(\epsilon) \approx \int_{-\infty}^{\infty} \max\left(1, \sqrt{\frac{S(\omega)}{\epsilon^2}}\right) d\omega \]

For large $\omega$, $S(\omega) \sim 2\sigma^2\alpha/\omega^2$, so:

\[ N(\epsilon) \approx 2 \int_0^{\infty} \max\left(1, \frac{\sqrt{2\sigma^2\alpha}}{\epsilon\omega}\right) d\omega \]

Let $\omega_\epsilon = \sqrt{2\sigma^2\alpha}/\epsilon$. Then:

\begin{align*}
N(\epsilon) &\approx 2\left[\omega_\epsilon + \int_{\omega_\epsilon}^{\infty} \frac{\sqrt{2\sigma^2\alpha}}{\epsilon\omega} d\omega\right] \\
&= 2\left[\frac{\sqrt{2\sigma^2\alpha}}{\epsilon} + \frac{\sqrt{2\sigma^2\alpha}}{\epsilon} \log\left(\frac{\infty}{\omega_\epsilon}\right)\right] \\
&\approx \frac{C}{\epsilon} \log\left(\frac{1}{\epsilon}\right)
\end{align*}

where $C$ is a constant depending on $\sigma$ and $\alpha$.

The $\epsilon$-entropy $H(\epsilon)$ is defined as $\log(N(\epsilon))$, so:

\[ H(\epsilon) \approx \log\left(\frac{C}{\epsilon}\right) + \log\left(\log\left(\frac{1}{\epsilon}\right)\right) \]

The entropy integral is:

\[ \int_0^1 H(\epsilon) d\epsilon \approx \int_0^1 \left[\log\left(\frac{C}{\epsilon}\right) + \log\left(\log\left(\frac{1}{\epsilon}\right)\right)\right] d\epsilon \]

The second term causes the divergence. Using the change of variable $u = \log(1/\epsilon)$:

\[ \int_0^1 \log\left(\log\left(\frac{1}{\epsilon}\right)\right) d\epsilon = \int_{\infty}^0 \log(u) e^{-u} du = \infty \]

Therefore:

\[ \int_0^1 H(\epsilon) d\epsilon = \infty \]

This divergence of the entropy integral demonstrates that the covariance operator of the OU process is not compact on $L^2(\mathbb{R})$.

\end{document}