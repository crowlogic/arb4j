\documentclass[11pt]{article}
\usepackage{amsmath,amssymb,amsthm,mathtools}
\usepackage{enumitem}
\usepackage{hyperref}

\newtheorem{definition}{Definition}
\newtheorem{theorem}{Theorem}
\newtheorem{lemma}{Lemma}
\newtheorem{corollary}{Corollary}
\newtheorem{remark}{Remark}

\title{Measure-Preserving Bijective Time Changes of Stationary Gaussian Processes Generate Oscillatory Processes With Evolving Spectra}

\author{Stephen Crowley\thanks{Email: \texttt{stephencrowley214@gmail.com}}}

\date{August 1, 2025}

\begin{document}

\maketitle

\begin{abstract}
This article establishes that Gaussian processes obtained through measure-preserving bijective unitary time transformations of stationary processes constitute a subclass of oscillatory processes in the sense of Priestley. The transformation $X_t = \sqrt{\theta'(t)} S_{\theta(t)}$, where $S_t$ is a stationary Gaussian process and $\theta$ is a strictly monotonic function, yields an oscillatory process with evolutionary power spectrum $dF_t(\omega) = \theta'(t) d\mu(\omega)$. An explicit unitary transformation between the original stationary process and the transformed oscillatory process is established, preserving the $L^2$-norm and providing a complete spectral characterization.
\end{abstract}

\section{Scaling Functions}\label{sec:scaling}

\begin{definition}[Scaling Functions]\label{def:scaling}
Let $\mathcal{F}$ denote the set of functions $\theta\colon\mathbb{R}\to\mathbb{R}$ satisfying
\begin{enumerate}[label=(\alph*)]
    \item $\theta$ is continuously differentiable and $\theta'(t)>0$ for all $t\in\mathbb{R}$,
    \item $\theta$ is strictly increasing and bijective.
\end{enumerate}
\end{definition}

\begin{remark}
The conditions in Definition~\ref{def:scaling} ensure that $\theta^{-1}$ exists and is differentiable, enabling measure-preserving transformations.
\end{remark}

\section{Oscillatory Processes}\label{sec:oscillatory}

\begin{definition}[Oscillatory Process]\label{def:oscillatory}
A complex-valued, second-order process $\{X_t\}_{t\in\mathbb{R}}$ is called \emph{oscillatory} if there exist
\begin{enumerate}[label=(\roman*)]
    \item a family of functions $\{\phi_t(\omega)\}_{t\in\mathbb{R}}$ with $\phi_t(\omega)=A_t(\omega)e^{i\omega t}$ and $A_t(\cdot)\in L^2(\mu)$,
    \item a complex orthogonal-increment process $Z(\omega)$ with $E\lvert dZ(\omega)\rvert^2=d\mu(\omega)$,
\end{enumerate}
such that
\begin{equation}\label{eq:oscillatory_rep}
    X_t=\int_{-\infty}^{\infty}\phi_t(\omega)\,dZ(\omega).
\end{equation}
\end{definition}

\section{Stationary Reference Process}\label{sec:stationary}

Let $\{S_t\}_{t\in\mathbb{R}}$ be a stationary Gaussian process with continuous spectral representation
\begin{equation}\label{eq:stationary_rep}
    S_t=\int_{-\infty}^{\infty}e^{i\omega t}\,dZ(\omega),
\end{equation}
where $Z(\omega)$ is an orthogonal-increment process with $E\lvert dZ(\omega)\rvert^2=d\mu(\omega)$ and $\mu$ is a finite measure on $\mathbb{R}$.

\section{Time-Changed Process}\label{sec:time_change}

\subsection{Definition and Unitary Operator}

\begin{definition}[Time-Changed Process]\label{def:time_changed_proc}
For $\theta\in\mathcal{F}$, define the time-changed process
\begin{equation}\label{eq:time_change}
    X_t \coloneqq \sqrt{\theta'(t)}\,S_{\theta(t)}, \qquad t\in\mathbb{R}.
\end{equation}
\end{definition}

\begin{lemma}[Unitary Transformation]\label{lem:unitary}
Define the operator $M_\theta\colon L^2(\mathbb{R})\to L^2(\mathbb{R})$ by
\begin{equation}\label{eq:unitary_op}
    (M_\theta f)(t) = \sqrt{\theta'(t)}\,f\bigl(\theta(t)\bigr).
\end{equation}
Then $M_\theta$ is unitary, i.e.,
\begin{equation}\label{eq:L2_preserve}
    \int_{\mathbb{R}}\lvert (M_\theta f)(t)\rvert^2\,dt
    =\int_{\mathbb{R}}\lvert f(s)\rvert^2\,ds.
\end{equation}
\end{lemma}

\begin{proof}
Apply the change of variables $s=\theta(t)$; then $ds=\theta'(t)\,dt$, and equation~\eqref{eq:L2_preserve} follows immediately.
\end{proof}

\subsection{Oscillatory Representation}

\begin{theorem}[Oscillatory Form]\label{thm:osc_rep}
The process $\{X_t\}$ defined in equation~\eqref{eq:time_change} is oscillatory with
\begin{equation}\label{eq:phi_def}
    \phi_t(\omega)=\sqrt{\theta'(t)}\,e^{i\omega\theta(t)}.
\end{equation}
\end{theorem}

\begin{proof}
Substitute equation~\eqref{eq:stationary_rep} into equation~\eqref{eq:time_change}:
\[
    X_t=\sqrt{\theta'(t)}\int_{-\infty}^{\infty}e^{i\omega\theta(t)}\,dZ(\omega)
        =\int_{-\infty}^{\infty}\phi_t(\omega)\,dZ(\omega),
\]
which is of the form~\eqref{eq:oscillatory_rep} with $\phi_t$ given by equation~\eqref{eq:phi_def}.
\end{proof}

\subsection{Envelope and Evolutionary Spectrum}

\begin{corollary}[Envelope]\label{cor:envelope}
Equation~\eqref{eq:phi_def} admits the decomposition
\begin{equation}\label{eq:envelope}
    \phi_t(\omega)=A_t(\omega)e^{i\omega t},
    \quad\text{where}\quad
    A_t(\omega)=\sqrt{\theta'(t)}\,e^{i\omega(\theta(t)-t)}.
\end{equation}
\end{corollary}

\begin{corollary}[Evolutionary Spectrum]\label{cor:evolving_spec}
The evolutionary power spectrum is
\begin{equation}\label{eq:evolutionary_spec}
    dF_t(\omega)=\lvert A_t(\omega)\rvert^2\,d\mu(\omega)=\theta'(t)\,d\mu(\omega).
\end{equation}
\end{corollary}

\section{$L^2$-Norm Preservation}\label{sec:norm_preservation}

\begin{theorem}[Measure Preservation]\label{thm:measure_preserve}
The transformation defined in equation~\eqref{eq:time_change} preserves the $L^2$-norm in the sense that
\begin{equation}\label{eq:measure_preserve}
    \int_I \operatorname{var}(X_t)\,dt = \int_{\theta(I)} \operatorname{var}(S_s)\,ds
\end{equation}
for any measurable set $I\subseteq\mathbb{R}$.
\end{theorem}

\begin{proof}
Using the change of variables $s=\theta(t)$ with $ds=\theta'(t)\,dt$:
\begin{align}
    \int_I \operatorname{var}(X_t)\,dt &= \int_I \operatorname{var}\bigl(\sqrt{\theta'(t)}\,S_{\theta(t)}\bigr)\,dt\\
    &= \int_I \theta'(t)\,\operatorname{var}(S_{\theta(t)})\,dt\\
    &= \int_{\theta(I)} \operatorname{var}(S_s)\,ds.
\end{align}
\end{proof}

\section{Expected Zero Count}\label{sec:zero_count}

\begin{theorem}[Expected Zero-Counting Function]\label{thm:zero_count}
Let $\theta\in\mathcal{F}$ and let $K(\tau)$ be the covariance function of the stationary process $S_t$, twice differentiable at $\tau=0$. The expected number of zeros of the process $X_t$ in the interval $[a,b]$ is
\begin{equation}\label{eq:zero_count}
    \mathbb{E}[N_{[a,b]}] = \sqrt{-\ddot{K}(0)}\,(\theta(b)-\theta(a)).
\end{equation}
\end{theorem}

\begin{proof}
By the Kac-Rice formula, the expected zero count is
\begin{equation}\label{eq:kac_rice}
    \mathbb{E}[N_{[a,b]}] = \int_a^b \sqrt{-\lim_{s\to t}\frac{\partial^2}{\partial t\partial s}K_\theta(s,t)}\,dt,
\end{equation}
where $K_\theta(s,t)=K(|\theta(t)-\theta(s)|)$. Computing the mixed partial derivative:
\begin{equation}\label{eq:mixed_partial}
    \lim_{s\to t}\frac{\partial^2}{\partial t\partial s}K_\theta(s,t) = -\ddot{K}(0)\,\theta'(t)^2.
\end{equation}
Therefore:
\begin{equation}\label{eq:zero_count_integral}
    \mathbb{E}[N_{[a,b]}] = \sqrt{-\ddot{K}(0)}\int_a^b \theta'(t)\,dt = \sqrt{-\ddot{K}(0)}\,(\theta(b)-\theta(a)).
\end{equation}
\end{proof}

\begin{thebibliography}{99}
\bibitem{priestley1965} M.B. Priestley. Evolutionary spectra and non-stationary processes. \emph{Journal of the Royal Statistical Society, Series B}, 27(2):204--237, 1965.

\bibitem{cramer1967} H. Cramér and M.R. Leadbetter. \emph{Stationary and Related Stochastic Processes}. Wiley, 1967.
\end{thebibliography}

\end{document}
