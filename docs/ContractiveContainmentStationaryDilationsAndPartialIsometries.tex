\documentclass{article}
\usepackage{amsmath,amssymb,amsthm}
\usepackage{mathtools}

\newtheorem{theorem}{Theorem}
\newtheorem{lemma}[theorem]{Lemma}
\newtheorem{definition}[theorem]{Definition}
\newtheorem{corollary}[theorem]{Corollary}
\newtheorem{remark}[theorem]{Remark}
\newtheorem{proposition}[theorem]{Proposition}

\begin{document}
\title{Contractive Containment, Stationary Dilations, and Partial Isometries: Equivalence, Properties, and Geometric Intuition}
\author{}
\maketitle

\section{Preliminary Concepts}

\begin{definition}[Hilbert Space Contraction]
A bounded linear operator $T: H_1 \to H_2$ between Hilbert spaces is called a contraction if
\[ \|Tx\|_{H_2} \leq \|x\|_{H_1} \quad \forall x \in H_1 \]
Equivalently, $\|T\| \leq 1$.
\end{definition}

\begin{definition}[Stationary Process]
A stochastic process $\{Y(t)\}_{t \in \mathbb{R}}$ is stationary if for any finite set of time points $\{t_1,\ldots,t_n\}$ and any $h \in \mathbb{R}$, the joint distribution of
\[ \{Y(t_1+h),\ldots,Y(t_n+h)\} \]
is identical to that of $\{Y(t_1),\ldots,Y(t_n)\}$.
\end{definition}

\begin{definition}[Stationary Dilation]
Given a non-stationary process $X(t)$, a stationary dilation is a stationary process $Y(s)$ together with a family of bounded operators $\{\phi(t,\cdot)\}_{t \in \mathbb{R}}$ such that
\[ X(t) = \int_{\mathbb{R}} \phi(t,s)Y(s)ds \]
where $\phi(t,s)$ is a measurable function satisfying:
\begin{enumerate}
    \item $\|\phi(t,\cdot)\|_{\infty} \leq 1$ for all $t$
    \item The map $t \mapsto \phi(t,\cdot)$ is strongly continuous
\end{enumerate}
\end{definition}

\begin{remark}
The conditions on $\phi(t,s)$ ensure that the integral is well-defined and the resulting process $X(t)$ inherits appropriate regularity properties from $Y(s)$.
\end{remark}

\section{Main Results}

\begin{proposition}[Properties of Scaling Function]
The scaling function $\phi(t,s)$ in a stationary dilation satisfies:
\begin{enumerate}
    \item $\|\phi(t,s)\| \leq 1$ for all $t,s \in \mathbb{R}$
    \item For fixed $t$, $s \mapsto \phi(t,s)$ is measurable
    \item For fixed $s$, $t \mapsto \phi(t,s)$ is continuous
\end{enumerate}
\end{proposition}

\begin{theorem}[Equivalence of Containment]
For a non-stationary process $X(t)$ and a stationary process $Y(s)$, the following are equivalent:
\begin{enumerate}
    \item $Y(s)$ is a stationary dilation of $X(t)$
    \item There exists a contractive mapping $\Phi$ from the space generated by $Y$ to the space generated by $X$ such that $X(t) = (\Phi Y)(t)$ for all $t$
\end{enumerate}
\end{theorem}

\begin{proof}
($1 \Rightarrow 2$): Define $\Phi$ by
\[ (\Phi Y)(t) = \int_{\mathbb{R}} \phi(t,s)Y(s)ds \]

For any finite linear combination $\sum_i \alpha_i Y(t_i)$:
\begin{align*}
\|\Phi(\sum_i \alpha_i Y(t_i))\|^2 &= \|\sum_i \alpha_i \int_{\mathbb{R}} \phi(t_i,s)Y(s)ds\|^2 \\
&\leq \|\sum_i \alpha_i Y(t_i)\|^2
\end{align*}
where the inequality follows from the bound on $\|\phi(t,s)\|$ and the Cauchy-Schwarz inequality.

($2 \Rightarrow 1$): The contractive mapping $\Phi$ induces a family of operators $\phi(t,s)$ via the Kernel theorem for Hilbert spaces. The stationarity of $Y$ and the contractivity of $\Phi$ ensure that these operators satisfy the required properties.
\end{proof}

\begin{lemma}[Minimal Dilation Property]
If $Y(s)$ is a minimal stationary dilation of $X(t)$, then the scaling function $\phi(t,s)$ achieves the bound
\[ \sup_{t,s} \|\phi(t,s)\| = 1 \]
\end{lemma}

\begin{proof}
If $\sup_{t,s} \|\phi(t,s)\| < 1$, we could construct a smaller dilation by scaling $Y(s)$, contradicting minimality.
\end{proof}

\section{Structure Theory}

\begin{theorem}[Sz.-Nagy Dilation]
For any contraction $T$ on a Hilbert space $H$, there exists a minimal unitary dilation $U$ on a larger space $K \supseteq H$ such that:
\[ T^n = P_H U^n|_H \quad \forall n \geq 0 \]
where $P_H$ is the orthogonal projection onto $H$.
\end{theorem}

\begin{lemma}[Defect Operators]
For a contraction $T$, the defect operators defined by:
\[ D_T = (I - T^*T)^{1/2} \]
\[ D_{T^*} = (I - TT^*)^{1/2} \]
satisfy:
\begin{enumerate}
    \item $\|D_T\| \leq 1$ and $\|D_{T^*}\| \leq 1$
    \item $D_T = 0$ if and only if $T$ is an isometry
    \item $D_{T^*} = 0$ if and only if $T$ is a co-isometry
\end{enumerate}
\end{lemma}

\section{Convergence Properties}

\begin{theorem}[Strong Convergence]
For a contractive stationary dilation, the following limit exists in the strong operator topology:
\[ \lim_{n \to \infty} T^n = P_{ker(I-T^*T)} \]
where $P_{ker(I-T^*T)}$ is the orthogonal projection onto the kernel of $I-T^*T$.
\end{theorem}

\begin{proof}
For any $x$ in the Hilbert space:
\begin{enumerate}
    \item The sequence $\{\|T^n x\|\}$ is decreasing since $T$ is a contraction
    \item It is bounded below by 0
    \item Therefore, $\lim_{n \to \infty} \|T^n x\|$ exists
    \item The limit operator must be the projection onto the space of vectors $x$ satisfying $\|Tx\| = \|x\|$
    \item This space is precisely $ker(I-T^*T)$
\end{enumerate}
\end{proof}

\begin{corollary}[Asymptotic Behavior]
If $T$ is a strict contraction (i.e., $\|T\| < 1$), then
\[ \lim_{n \to \infty} T^n = 0 \]
in the strong operator topology.
\end{corollary}

\section{Partial Isometries: The Mathematical Scalpel}

\begin{definition}[Partial Isometry]
An operator $A$ on a Hilbert space $H$ is a partial isometry if $A^*A$ is an orthogonal projection.
\end{definition}

\begin{remark}[Geometric Intuition]
A partial isometry is like a mathematical scalpel that carves out a section of space:
\begin{itemize}
    \item It acts as a perfect rigid motion (isometry) on a specific subspace
    \item It completely annihilates the rest of the space
\end{itemize}
This property makes partial isometries powerful tools for selecting and transforming specific parts of a Hilbert space while cleanly disposing of the rest.
\end{remark}

\begin{proposition}[Key Properties of Partial Isometries]
Let $A$ be a partial isometry. Then:
\begin{enumerate}
    \item $A$ is an isometry when restricted to $(ker A)^\perp$
    \item $A(ker A)^\perp = ran A$
    \item $A^*$ is also a partial isometry
    \item $AA^*A = A$ and $A^*AA^* = A^*$
\end{enumerate}
\end{proposition}

\begin{theorem}[Geometric Characterization]
For a partial isometry $A$:
\[ A^*A = P_{(ker A)^\perp} \quad \text{and} \quad AA^* = P_{ran A} \]
where $P_S$ denotes the orthogonal projection onto subspace $S$.
\end{theorem}

\begin{proof}
The action of $A$ can be decomposed as:
\begin{enumerate}
    \item Project onto $(ker A)^\perp$ (this is $A^*A$)
    \item Apply a perfect rigid motion to the projected space
\end{enumerate}
This two-step process ensures $A^*A$ is the projection onto $(ker A)^\perp$.
\end{proof}

\begin{remark}[The "Not So Partial" Nature]
Despite the name, there's nothing incomplete about a partial isometry. It performs a complete operation:
\begin{itemize}
    \item It's a full isometry on its initial space ($(ker A)^\perp$)
    \item It perfectly maps this initial space onto its final space ($ran A$)
    \item It precisely annihilates everything else
\end{itemize}
This makes partial isometries fundamental building blocks in operator theory, crucial in polar decompositions, dimension theory of von Neumann algebras, and quantum mechanics.
\end{remark}

\end{document}
