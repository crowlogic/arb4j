\documentclass{article}
\usepackage[english]{babel}
\usepackage{amsmath}

%%%%%%%%%% Start TeXmacs macros
\newcommand{\tmaffiliation}[1]{\\ #1}
\newtheorem{definition}{Definition}
\newtheorem{theorem}{Theorem}
%%%%%%%%%% End TeXmacs macros

\begin{document}

\title{Fourier Transform Theory: Complex Exponentials and Generalized
Functions}

\author{
  Stephen Crowley
  \tmaffiliation{August 27, 2025}
}

\maketitle

\section{Introduction}

This document presents the fundamental relationships between the Fourier
transform of complex exponentials, the Dirac delta function, and the Heaviside
step function.

\section{Definitions}

\begin{definition}
  [Complex Exponential Function] The complex exponential function is defined
  as:
  \[ x (t) = e^{j \omega_0 t} \]
  where $j = \sqrt{- 1}$ is the imaginary unit and $\omega_0$ is the angular
  frequency.
\end{definition}

\begin{definition}
  [Dirac Delta Function] The Dirac delta function $\delta (t)$ is a
  generalized function (distribution) with the following properties:
  
  \begin{align}
    \delta (t) & = 0 \quad \text{for } t \neq 0 \\
    \int_{- \infty}^{\infty} \delta (t)  \hspace{0.17em} dt & = 1 \\
    \int_{- \infty}^{\infty} f (t) \delta (t - a)  \hspace{0.17em} dt & = f
    (a) \quad \text{(sifting property)} 
  \end{align}
\end{definition}

\begin{definition}
  [Heaviside Step Function] The Heaviside step function $H (t)$ is defined as:
  \[ H (t) = \left\{\begin{array}{ll}
       0, & t < 0\\
       1, & t \geq 0
     \end{array}\right. \]
\end{definition}

\begin{definition}
  [Fourier Transform] The Fourier transform of a function $f (t)$ is defined
  as:
  \[ F (\omega) =\mathcal{F} \{f (t)\} = \int_{- \infty}^{\infty} f (t) e^{- j
     \omega t}  \hspace{0.17em} dt \]
  with the inverse transform:
  \[ f (t) =\mathcal{F}^{- 1} \{F (\omega)\} = \frac{1}{2 \pi}  \int_{-
     \infty}^{\infty} F (\omega) e^{j \omega t}  \hspace{0.17em} d \omega \]
\end{definition}

\section{Main Theorems}

\begin{theorem}
  [Fourier Transform of Complex Exponential] The Fourier transform of the
  complex exponential $e^{j \omega_0 t}$ is:
  \[ \mathcal{F} \{e^{j \omega_0 t} \} = 2 \pi \delta (\omega - \omega_0) \]
\end{theorem}

Since $e^{j \omega_0 t}$ has infinite energy and is not absolutely integrable,
we use the inverse transform approach.

Starting with the frequency domain representation $2 \pi \delta (\omega -
\omega_0)$, we apply the inverse Fourier transform:

\begin{align}
  \mathcal{F}^{- 1}  \{2 \pi \delta (\omega - \omega_0)\} & = \frac{1}{2 \pi} 
  \int_{- \infty}^{\infty} 2 \pi \delta (\omega - \omega_0) e^{j \omega t} 
  \hspace{0.17em} d \omega \\
  & = \int_{- \infty}^{\infty} \delta (\omega - \omega_0) e^{j \omega t} 
  \hspace{0.17em} d \omega \\
  & = e^{j \omega_0 t} \quad \text{(by sifting property)} 
\end{align}

Therefore, by the uniqueness of the

\end{document}
