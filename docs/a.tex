\documentclass[11pt]{article}

\usepackage{amsmath}
\usepackage{amssymb}
\usepackage{amsthm}
\usepackage{mathtools}
\usepackage{geometry}
\usepackage{hyperref}
\usepackage{cleveref}

\usepackage[margin=0.5in]{geometry}

\newtheorem{theorem}{Theorem}[section]
\newtheorem{lemma}[theorem]{Lemma}
\newtheorem{corollary}[theorem]{Corollary}
\newtheorem{proposition}[theorem]{Proposition}
\newtheorem{definition}[theorem]{Definition}
\newtheorem{remark}[theorem]{Remark}

\newcommand{\assign}{:=}
\newcommand{\R}{\mathbb{R}}
\newcommand{\C}{\mathbb{C}}
\newcommand{\E}{\mathbb{E}}
\newcommand{\Prob}{\mathbb{P}}
\newcommand{\calB}{\mathcal{B}}
\newcommand{\calF}{\mathcal{F}}
\newcommand{\calH}{\mathcal{H}}

\title{The Spectral Representation of Oscillatory Processes}
\author{Stephen Crowley}
\date{November 25, 2025}

\begin{document}

\maketitle
\tableofcontents

\section{Definitions and Assumptions}

\begin{remark}
  Unless otherwise stated:
  \begin{enumerate}
    \item All parametric families $\{f_t(\omega)\}$ are jointly measurable with respect to $\calB(\R) \otimes \calB(\R)$.
    \item Dirac delta identities such as $\int e^{i(\mu - \lambda)u} \, du = 2\pi \delta(\mu - \lambda)$ are distributional.
    \item Integrals of the form $\int f(\omega) \, d\nu(\omega)$ denote Lebesgue--Stieltjes integration with respect to measure $\nu$, while $\int g(u) \, du$ denotes Lebesgue integration with respect to Lebesgue measure.
    \item Integrals with respect to orthogonal random measures $\Phi$, written $\int h(\omega) \, d\Phi(\omega)$, are Lebesgue--Stieltjes integrals in $L^2(\Omega)$ with variance $\E[|\int h \, d\Phi|^2] = \int |h|^2 \, d\mu$.
    \item Limit interchange is justified by dominated convergence under uniform $L^2$ bounds.
  \end{enumerate}
\end{remark}

Throughout, $(\Omega, \calF, \Prob)$ is a fixed probability space and $\mu$ is a finite Borel measure on $\R$.

\begin{definition}[Oscillatory Function Family]
  \label{def:oscillatory-function}
  A family $\{\phi_t(\omega)\}_{t \in \R} \subset L^2(\R, \mu)$ is \emph{oscillatory} if there exists a family $\{A_t(\omega)\}_{t \in \R, \omega \in \R}$ satisfying:
  \begin{enumerate}
    \item \emph{Factorization:} For all $t, \omega \in \R$,
    \begin{equation}
      \phi_t(\omega) = A_t(\omega) e^{i\omega t}
    \end{equation}
    
    \item \emph{Amplitude Representation:} For $\mu$-almost every $\omega$, there exists a probability measure $H_{\omega}$ on $\R$ with $\int |\lambda| \, dH_{\omega}(\lambda) < \infty$ such that
    \begin{equation}
      A_t(\omega) = \int_{-\infty}^{\infty} e^{i\lambda t} \, dH_{\omega}(\lambda) \quad \text{for all } t \in \R
    \end{equation}
    
    \item \emph{Concentration at Origin:} For $\mu$-almost every $\omega$, the measure $H_{\omega}$ concentrates at the origin in the sense that
    \begin{equation}
      H_{\omega}(\{0\}) > H_{\omega}(\R \setminus \{0\})
    \end{equation}
    equivalently, the point mass at zero exceeds the total remaining mass.
    
    \item \emph{Normalization:} For $\mu$-almost every $\omega$, $A_0(\omega) = 1$.
  \end{enumerate}
  The family $\{A_t(\omega)\}$ is called the \emph{amplitude family} associated to $\{\phi_t\}$.
\end{definition}

\begin{proposition}[Non-vanishing from Concentration]
  \label{prop:nonvanishing}
  Under the concentration condition in Definition~\ref{def:oscillatory-function}, the amplitude satisfies $A_t(\omega) \neq 0$ for all $t \in \R$ and $\mu$-almost every $\omega$.
\end{proposition}

\begin{proof}
  Decompose $H_{\omega} = h_0 \delta_0 + \tilde{H}_{\omega}$ where $h_0 = H_{\omega}(\{0\})$ and $\tilde{H}_{\omega}$ is supported on $\R \setminus \{0\}$ with total mass $m = 1 - h_0$. By the concentration condition, $h_0 > m$, which gives $h_0 > 1 - h_0$, hence $h_0 > 1/2$.
  
  The amplitude decomposes as
  \begin{equation}
    A_t(\omega) = h_0 + \int_{\R \setminus \{0\}} e^{i\lambda t} \, d\tilde{H}_{\omega}(\lambda)
  \end{equation}
  
  The integral term satisfies
  \begin{equation}
    \left|\int_{\R \setminus \{0\}} e^{i\lambda t} \, d\tilde{H}_{\omega}(\lambda)\right| \leq m < h_0
  \end{equation}
  
  Therefore $\mathrm{Re}\, A_t(\omega) \geq h_0 - m > 0$ for all $t$, which implies $A_t(\omega) \neq 0$ for all $t \in \R$.
\end{proof}

\begin{definition}[Oscillatory Stochastic Process]
  \label{def:oscillatory-process}
  A centered stochastic process $\{X_t\}_{t \in \R}$ is \emph{oscillatory} if there exists:
  \begin{enumerate}
    \item a finite Borel measure $\mu$ on $\R$,
    \item a complex orthogonal random measure $\Phi$ on $(\R, \calB(\R))$ with $\E[|\Phi(B)|^2] = \mu(B)$,
    \item an amplitude family $\{A_t(\omega)\}$ satisfying Definition~\ref{def:oscillatory-function},
  \end{enumerate}
  such that for each $t \in \R$,
  \begin{equation}
    X_t = \int_{\R} A_t(\omega) e^{i\omega t} \, d\Phi(\omega) \quad \text{in } L^2(\Omega).
  \end{equation}
\end{definition}

\section{Existence and Regularity}

\begin{theorem}[Existence and Regularity of Oscillatory Processes]
  \label{thm:oscillatory-existence}
  Let $\{X_t\}_{t \in \R}$ be a centered stochastic process with covariance function $R_{s,t} = \E[X_s \overline{X_t}]$. Suppose $R_{s,t}$ admits a representation
  \begin{equation}
    R_{s,t} = \int_{\R} \phi_s(\omega) \overline{\phi_t(\omega)} \, d\mu(\omega)
  \end{equation}
  for some finite Borel measure $\mu$ and an oscillatory function family $\{\phi_t\}$ in the sense of Definition~\ref{def:oscillatory-function} with associated amplitude family $\{A_t\}$. Then the process admits the spectral representation
  \begin{equation}
    X_t = \int_{\R} A_t(\omega) e^{i\omega t} \, d\Phi(\omega)
  \end{equation}
  where $\Phi$ is a complex orthogonal random measure with $\E[|\Phi(B)|^2] = \mu(B)$, and the amplitude family satisfies the following regularity conditions:
  \begin{enumerate}
    \item \label{regcond:nonvanishing} \emph{Non-vanishing:} For $\mu$-almost every $\omega \in \R$,
    \begin{equation}
      A_t(\omega) \neq 0 \quad \text{for all } t \in \R
    \end{equation}
    
    \item \label{regcond:fourier} \emph{Fourier--Stieltjes structure:} For $\mu$-almost every $\omega \in \R$, the representation
    \begin{equation}
      A_t(\omega) = \int_{-\infty}^{\infty} e^{i\lambda t} \, dH_{\omega}(\lambda)
    \end{equation}
    holds with $H_{\omega}(\R) = 1$, $H_{\omega}(\{0\}) > H_{\omega}(\R \setminus \{0\})$, and $\int |\lambda| \, dH_{\omega}(\lambda) < \infty$.
    
    \item \label{regcond:l2} \emph{Quadratic integrability:} For each fixed $t \in \R$,
    \begin{equation}
      \int_{\R} |A_t(\omega)|^2 \, d\mu(\omega) < \infty
    \end{equation}
    
    \item \label{regcond:normalize} \emph{Normalization:} For $\mu$-almost every $\omega \in \R$,
    \begin{equation}
      A_0(\omega) = 1
    \end{equation}
  \end{enumerate}
\end{theorem}

\begin{proof}
  The proof establishes factorization and orthogonal measure construction, verifies non-vanishing, confirms quadratic integrability, and validates the covariance structure.
  
  \paragraph{Factorization and Orthogonal Measure.}
  By assumption, $\{\phi_t\}$ is oscillatory, so the factorization $\phi_t(\omega) = A_t(\omega) e^{i\omega t}$ holds by Definition~\ref{def:oscillatory-function}. Since $R_{s,t}$ is non-negative definite and $\mu$ is finite, the spectral theorem for covariance kernels guarantees existence of a complex orthogonal random measure $\Phi$ with $\E[|\Phi(B)|^2] = \mu(B)$ and $\E[\Phi(B_1) \overline{\Phi(B_2)}] = 0$ for disjoint $B_1, B_2$.
  
  \paragraph{Non-vanishing Property.}
  The concentration condition in Definition~\ref{def:oscillatory-function} ensures that for $\mu$-a.e.\ $\omega$, $H_{\omega}(\{0\}) > H_{\omega}(\R \setminus \{0\})$. By Proposition~\ref{prop:nonvanishing}, this implies $A_t(\omega) \neq 0$ for all $t \in \R$, establishing condition~\ref{regcond:nonvanishing}.
  
  \paragraph{Quadratic Integrability.}
  For $\mu$-a.e.\ $\omega$,
  \begin{equation}
    |A_t(\omega)| \leq \int |e^{i\lambda t}| \, dH_{\omega}(\lambda) = H_{\omega}(\R) = 1
  \end{equation}
  Therefore,
  \begin{equation}
    \int_{\R} |A_t(\omega)|^2 \, d\mu(\omega) \leq \int_{\R} 1 \, d\mu(\omega) = \mu(\R) < \infty
  \end{equation}
  which proves condition~\ref{regcond:l2} and ensures the stochastic integral defining $X_t$ is well-defined in $L^2(\Omega)$.
  
  \paragraph{Covariance Verification.}
  Using the orthogonality of $\Phi$ and the isometry property of stochastic integrals:
  \begin{align}
    \E[X_s \overline{X_t}] &= \E\left[ \int_{\R} A_s(\omega) e^{i\omega s} \, d\Phi(\omega) \cdot \overline{\int_{\R} A_t(\nu) e^{i\nu t} \, d\Phi(\nu)} \right] \\
    &= \int_{\R} A_s(\omega) \overline{A_t(\omega)} e^{i\omega(s-t)} \, d\E[|\Phi(\omega)|^2] \\
    &= \int_{\R} A_s(\omega) \overline{A_t(\omega)} e^{i\omega(s-t)} \, d\mu(\omega) \\
    &= R_{s,t} \qedhere
  \end{align}
\end{proof}

\begin{corollary}[Derived Regularity Properties]
  \label{cor:derived-properties}
  Under the conditions of Theorem~\ref{thm:oscillatory-existence}, the following hold:
  \begin{enumerate}
    \item \label{der:local-bound} For any compact interval $[a,b] \subset \R$ and Borel set $E$ with $\mu(E) < \infty$,
    \begin{equation}
      \sup_{t \in [a,b]} \int_E |A_t(\omega)|^2 \, d\mu(\omega) \leq \mu(E)
    \end{equation}
    
    \item \label{der:ratio-bound} For compact $[a,b] \subset \R$ and compact $K \subset \R$ with $\mu(K) > 0$,
    \begin{equation}
      C_{a,b,K} \assign \sup_{\substack{t \in [a,b] \\ \omega, \nu \in K}} \left| \frac{A_t(\nu)}{A_t(\omega)} \right| < \infty
    \end{equation}
    
    \item \label{der:differentiable} For $\mu$-almost every $\omega \in \R$, the temporal derivative
    \begin{equation}
      \frac{\partial A_t(\omega)}{\partial t} = \int_{-\infty}^{\infty} i\lambda \, e^{i\lambda t} \, dH_{\omega}(\lambda)
    \end{equation}
    exists for all $t \in \R$ and is continuous in $t$.
  \end{enumerate}
\end{corollary}

\begin{proof}
  Property~\ref{der:local-bound} follows since $|A_t(\omega)| \leq 1$ for $\mu$-a.e.\ $\omega$ by Theorem~\ref{thm:oscillatory-existence}, giving $\sup_{t \in [a,b]} \int_E |A_t(\omega)|^2 \, d\mu(\omega) \leq \int_E 1 \, d\mu = \mu(E)$.
  
  For property~\ref{der:ratio-bound}, by condition~\ref{regcond:nonvanishing}, for each $\omega \in K$ there exists $\alpha(\omega) > 0$ such that $|A_t(\omega)| \geq \alpha(\omega)$ for all $t \in [a,b]$. Since $K$ is compact and $A_t(\omega)$ is jointly continuous in $(t,\omega)$ by the finite first moment condition and dominated convergence, the map $\omega \mapsto \alpha(\omega)$ is lower semicontinuous and attains its infimum $\alpha_K > 0$, thus $\left|\frac{A_t(\nu)}{A_t(\omega)}\right| \leq \alpha_K^{-1} < \infty$ since $|A_t(\nu)| \leq 1$.
  
  For property~\ref{der:differentiable}, differentiation under the integral is justified by dominated convergence and the finite first moment condition $\int |\lambda| \, dH_{\omega}(\lambda) < \infty$ from Definition~\ref{def:oscillatory-function}, thus a continuous derivative exists by standard results on Fourier--Stieltjes transforms.
\end{proof}

\section{Spectral Representation Theorems}

\begin{theorem}[Existence of Spectral Representation]
  \label{thm:existence}
  Let $\{X_t\}$ be a centered oscillatory process with covariance function $R_{s,t} = \E[X_s \overline{X_t}]$. Then there exist a finite Borel measure $\mu$ on $\R$, a family of oscillatory functions $\{A_t(\omega)\}$ satisfying Theorem~\ref{thm:oscillatory-existence}, and a complex orthogonal random measure $\Phi$ with $\E[|\Phi(B)|^2] = \mu(B)$, such that
  \begin{equation}
    X_t = \int_{\R} A_t(\omega) e^{i\omega t} \, d\Phi(\omega)
  \end{equation}
  and
  \begin{equation}
    R_{s,t} = \int_{\R} A_s(\omega) \overline{A_t(\omega)} e^{i\omega(s-t)} \, d\mu(\omega)
  \end{equation}
\end{theorem}

\begin{proof}
  By Definition~\ref{def:oscillatory-process}, an oscillatory process has a representation $X_t = \int_{\R} A_t(\omega) e^{i\omega t} \, d\Phi(\omega)$. The covariance formula follows from the covariance verification in Theorem~\ref{thm:oscillatory-existence}. The existence of such a representation is guaranteed by the construction in Theorem~\ref{thm:oscillatory-existence}.
\end{proof}

\begin{theorem}[Uniqueness of the Spectral Triple]
  \label{thm:uniqueness}
  The triple $(\mu, \Phi, A_t)$ is unique pathwise up to scalar multiple: if two triples $(\mu_1, \Phi_1, A_t^{(1)})$ and $(\mu_2, \Phi_2, A_t^{(2)})$ generate the same process $\{X_t\}$, then $\mu_1 = \mu_2$ (modulo null sets), $A_t^{(1)}(\omega) = A_t^{(2)}(\omega)$ for $\mu$-almost every $\omega$, and there exists $c \in \C$ with $|c| = 1$ such that $\Phi_2(B) = c\Phi_1(B)$ for all Borel sets $B$.
\end{theorem}

\begin{proof}
  Suppose
  \begin{equation}
    X_t = \int_{\R} A_t^{(1)}(\omega) e^{i\omega t} \, d\Phi_1(\omega) = \int_{\R} A_t^{(2)}(\omega) e^{i\omega t} \, d\Phi_2(\omega)
  \end{equation}
  
  Setting $s = t$ gives
  \begin{equation}
    \int_{\R} |A_t^{(1)}(\omega)|^2 \, d\mu_1(\omega) = \E[|X_t|^2] = \int_{\R} |A_t^{(2)}(\omega)|^2 \, d\mu_2(\omega)
  \end{equation}
  for all $t$. Taking $t = 0$ and using $A_0^{(1)} = A_0^{(2)} = 1$, we obtain $\mu_1(\R) = \mu_2(\R)$.
  
  For any Borel set $E$, define $Y_t^{(E)} = \int_E A_t^{(1)}(\omega) e^{i\omega t} \, d\Phi_1(\omega)$. By orthogonality,
  \begin{equation}
    \E[|Y_t^{(E)}|^2] = \int_E |A_t^{(1)}(\omega)|^2 \, d\mu_1(\omega)
  \end{equation}
  
  The same quantity computed using $(\mu_2, \Phi_2, A_t^{(2)})$ must be equal, so by polarization and the non-vanishing property, we obtain $\mu_1(E) = \mu_2(E)$. Hence $\mu_1 = \mu_2$ as measures.
  
  With $\mu_1 = \mu_2$, the equality of spectral representations implies that for each $t$,
  \begin{equation}
    \int_{\R} (A_t^{(1)}(\omega) - A_t^{(2)}(\omega)) e^{i\omega t} \, d\Phi_1(\omega) = 0 \quad \text{a.s.}
  \end{equation}
  
  By the non-vanishing property and the isometry, $A_t^{(1)}(\omega) = A_t^{(2)}(\omega)$ for $\mu$-a.e.\ $\omega$. Finally, since the amplitude family is unique, the random measures must satisfy $\Phi_2(B) = c\Phi_1(B)$ for some $c$ with $|c| = 1$ by the uniqueness of orthogonal random measures with prescribed variance.
\end{proof}

\section{The Shift Operator}

\begin{theorem}[Action of the Shift Operator]
  \label{thm:timeshift}
  Let $U_{\tau} : \calH_T \to \calH_T$ be the time-shift operator defined by $U_{\tau} X_t = X_{t+\tau}$. Under the spectral representation in Definition~\ref{def:oscillatory-process},
  \begin{align}
    U_{\tau} X_t &= X_{t+\tau} \\
    &= \int_{\R} A_{t+\tau}(\omega) e^{i\omega(t+\tau)} \, d\Phi(\omega) \\
    &= \int_{\R} A_{t+\tau}(\omega) e^{i\omega t} e^{i\omega\tau} \, d\Phi(\omega)
  \end{align}
\end{theorem}

\begin{proof}
  By Definition~\ref{def:oscillatory-process},
  \begin{equation}
    X_{t+\tau} = \int_{\R} A_{t+\tau}(\omega) e^{i\omega(t+\tau)} \, d\Phi(\omega)
  \end{equation}
  
  Factoring the exponential and applying the isometry:
  \begin{align}
    \E\left| \int_{\R} e^{i\omega\tau} A_{t+\tau}(\omega) e^{i\omega t} \, d\Phi(\omega) \right|^2 &= \int_{\R} |A_{t+\tau}(\omega)|^2 |e^{i\omega\tau}|^2 \, d\mu(\omega) \\
    &= \int_{\R} |A_{t+\tau}(\omega)|^2 \, d\mu(\omega) \\
    &= \E[|X_{t+\tau}|^2] < \infty
  \end{align}
  
  Thus the representation holds in $L^2(\Omega)$.
\end{proof}

\section{Time-Dependent Convolution Representation}

\begin{theorem}[Filter Representation]
  \label{thm:filter_representation}
  Let $X$ be a zero-mean stationary process
  \begin{equation}
    X(u) = \int e^{i\lambda u} \, d\Phi(\lambda)
  \end{equation}
  with spectral measure $F$, and let $Z$ be an oscillatory process with oscillatory function $\varphi_t(\lambda)$ and the same orthogonal random measure $\Phi$. Then
  \begin{equation}
    \label{eq:filter_representation}
    Z(t) = \int_{-\infty}^{\infty} h(t,u) X(u) \, du
  \end{equation}
  where
  \begin{equation}
    \label{eq:impulse_response_fourier}
    h(t,u) = \frac{1}{2\pi} \int_{-\infty}^{\infty} \varphi_t(\lambda) e^{-i\lambda u} \, d\lambda
  \end{equation}
  is the impulse response function.
\end{theorem}

\begin{proof}
  Substituting the definitions of $h(t,u)$ and $X(u)$ then applying Fubini and the sifting property:
  \begin{align}
    \int_{-\infty}^{\infty} h(t,u) X(u) \, du &= \int_{-\infty}^{\infty} \frac{1}{2\pi} \int_{-\infty}^{\infty} \varphi_t(\lambda) e^{-i\lambda u} \, d\lambda \int_{-\infty}^{\infty} e^{i\mu u} \, d\Phi(\mu) \, du \\
    &= \frac{1}{2\pi} \int_{-\infty}^{\infty} \int_{-\infty}^{\infty} \varphi_t(\lambda) \int_{-\infty}^{\infty} e^{i(\mu-\lambda)u} \, du \, d\lambda \, d\Phi(\mu) \\
    &= \frac{1}{2\pi} \int_{-\infty}^{\infty} \int_{-\infty}^{\infty} \varphi_t(\lambda) \cdot 2\pi\delta(\mu-\lambda) \, d\lambda \, d\Phi(\mu) \\
    &= \int_{-\infty}^{\infty} \int_{-\infty}^{\infty} \varphi_t(\lambda) \delta(\mu-\lambda) \, d\lambda \, d\Phi(\mu) \\
    &= \int_{-\infty}^{\infty} \varphi_t(\mu) \, d\Phi(\mu) = Z(t) \qedhere
  \end{align}
\end{proof}

\section{Isomorphism and Bidirectional Determination}

\begin{theorem}[Unitary Isomorphism]
  \label{thm:unitary}
  Define the map $U : \calH_T \to L^2(\R, \mu)$ by $U(X_t)(\omega) = A_t(\omega) e^{i\omega t}$. The map extends to a unitary isomorphism with
  \begin{equation}
    \langle X_s, X_t \rangle_{L^2(\Omega)} = \langle U(X_s), U(X_t) \rangle_{L^2(\R, \mu)}.
  \end{equation}
\end{theorem}

\begin{proof}
  The map $U$ is well-defined on the linear span $\calH_0 = \mathrm{span}\{X_t : t \in \R\}$. For $Y = \sum_{j=1}^n c_j X_{t_j}$, define
  \begin{equation}
    U(Y)(\omega) = \sum_{j=1}^n c_j A_{t_j}(\omega) e^{i\omega t_j}
  \end{equation}
  
  Linearity is immediate. The isometry property follows from the covariance calculation in Theorem~\ref{thm:oscillatory-existence}:
  \begin{equation}
    \langle X_s, X_t \rangle = \int_{\R} A_s(\omega) \overline{A_t(\omega)} e^{i\omega(s-t)} \, d\mu(\omega) = \langle U(X_s), U(X_t) \rangle
  \end{equation}
  
  Thus $\|U(Y)\|_{L^2(\mu)}^2 = \|Y\|_{L^2(\Omega)}^2$ for all $Y \in \calH_0$, so $U$ is an isometry. Since $\calH_0$ is dense in $\calH_T$ and the image $U(\calH_0)$ contains all functions of the form $\sum c_j A_{t_j}(\omega) e^{i\omega t_j}$, which are dense in $L^2(\R, \mu)$ by the non-vanishing property, $U$ extends uniquely to a unitary isomorphism.
\end{proof}

\begin{theorem}[Bidirectional Determination]
  \label{thm:bidirectional}
  The oscillatory process $\{X_t\}$ and the triple $(\mu, \Phi, \{A_t(\omega)\})$ are in one-to-one correspondence. Given $\{X_t\}$, the triple is uniquely determined (up to measure-theoretic equivalence) by Theorem~\ref{thm:uniqueness}. Conversely, given the triple satisfying Theorem~\ref{thm:oscillatory-existence}, the process is uniquely reconstructed via Definition~\ref{def:oscillatory-process}.
\end{theorem}

\begin{proof}
  Forward direction: Theorem~\ref{thm:uniqueness} establishes uniqueness.
  
  Reverse direction: Given $(\mu, \Phi, \{A_t\})$ satisfying Theorem~\ref{thm:oscillatory-existence}, the stochastic integral $X_t = \int_{\R} A_t(\omega) e^{i\omega t} \, d\Phi(\omega)$ is well-defined by condition~\ref{regcond:l2}. The covariance is given by $R_{s,t} = \int_{\R} A_s(\omega) \overline{A_t(\omega)} e^{i\omega(s-t)} \, d\mu(\omega)$. This pair $(X_t, R_{s,t})$ uniquely determines the law of the Gaussian process, establishing the one-to-one correspondence.
\end{proof}

\section{Gaussian Oscillatory Processes}

\begin{theorem}[Gaussian Structure]
  \label{thm:gaussian}
  If $\{X_t\}$ is a Gaussian oscillatory process satisfying Definition~\ref{def:oscillatory-process}, the orthogonal random measure $\Phi$ is Gaussian: for each Borel set $B$, $\Phi(B)$ is a complex Gaussian random variable with
  \begin{equation}
    \Phi(B) = \Phi_R(B) + i\Phi_I(B)
  \end{equation}
  where $\Phi_R, \Phi_I$ are independent real Gaussian orthogonal measures with $\E[\Phi_R(B)^2] = \E[\Phi_I(B)^2] = \mu(B)/2$.
\end{theorem}

\begin{proof}
  Since $X_t$ is Gaussian and any finite linear combination $\sum_{j=1}^n c_j X_{t_j}$ is Gaussian, the spectral representation yields
  \begin{equation}
    \sum_{j=1}^n c_j X_{t_j} = \int_{\R} \left[ \sum_{j=1}^n c_j A_{t_j}(\omega) e^{i\omega t_j} \right] d\Phi(\omega)
  \end{equation}
  is Gaussian for all choices of $c_j, t_j$. By the Cram\'er--Wold theorem, $\Phi(B)$ is Gaussian for all Borel $B$. 
  
  The complex orthogonality condition $\E[\Phi(B_1) \overline{\Phi(B_2)}] = 0$ for disjoint $B_1, B_2$ implies $\E[\Phi_R(B_1) \Phi_R(B_2)] = \E[\Phi_I(B_1) \Phi_I(B_2)] = 0$ and $\E[\Phi_R(B_1) \Phi_I(B_2)] = 0$. 
  
  The variance is computed as
  \begin{equation}
    \mu(B) = \E[|\Phi(B)|^2] = \E[\Phi_R(B)^2] + \E[\Phi_I(B)^2]
  \end{equation}
  and by symmetry $\E[\Phi_R(B)^2] = \E[\Phi_I(B)^2] = \mu(B)/2$.
\end{proof}

\subsection{The Envelope Spectrum}

\begin{definition}[Envelope Spectrum]
  \label{def:envelope-spectrum}
  The temporal Fourier transform of the amplitude,
  \begin{equation}
    \hat{A}_{\omega}(\lambda) = \int_{-\infty}^{\infty} A_t(\omega) e^{i\lambda t} \, dt
  \end{equation}
  is called the \emph{envelope spectrum} at carrier frequency $\omega$. It describes how the modulation $A_t(\omega)$ distributes across temporal frequencies $\lambda$.
\end{definition}

\section{Inverse Spectral Theorem}

\begin{theorem}[Inverse Spectral Theorem for Oscillatory Processes]
  \label{thm:inverse-spectral}
  Let $\{X_t\}_{t \in \R}$ be a Gaussian oscillatory process satisfying Definition~\ref{def:oscillatory-process} with amplitude family satisfying Theorem~\ref{thm:oscillatory-existence}. Given a sample path $\{X_t\}_{t \in \R}$, the orthogonal random measure $\Phi$ is recovered pathwise via
  \begin{equation}
    \Phi(B) = \lim_{T \to \infty} \frac{1}{2\pi T} \int_{-T}^T X_t \left[ \int_B \frac{e^{-i\omega t}}{A_t(\omega)} \, d\omega \right] dt
  \end{equation}
  where the division by $A_t(\omega)$ is well-defined by condition~\ref{regcond:nonvanishing} of Theorem~\ref{thm:oscillatory-existence}.
\end{theorem}

\begin{proof}
  Define the kernel
  \begin{equation}
    K_T(\lambda, \omega) = \frac{1}{2\pi T} \int_{-T}^T e^{i(\lambda-\omega)t} \, dt = \frac{\sin((\lambda-\omega)T)}{\pi(\lambda-\omega)T}
  \end{equation}
  
  For each fixed $\lambda$, $K_T(\lambda, \cdot)$ is an approximate identity: $\int K_T(\lambda, \omega) \, d\omega = 1$ and $K_T(\lambda, \omega) \to \delta_{\lambda}(\omega)$ as $T \to \infty$ in the sense of distributions.
  
  Substituting the spectral representation $X_t = \int_{\R} A_t(\lambda) e^{i\lambda t} \, d\Phi(\lambda)$ into the recovery formula:
  \begin{align}
    &\frac{1}{2\pi T} \int_{-T}^T X_t \left[ \int_B \frac{e^{-i\omega t}}{A_t(\omega)} \, d\omega \right] dt \notag \\
    &\quad= \frac{1}{2\pi T} \int_{-T}^T \left[\int_{\R} A_t(\lambda) e^{i\lambda t} \, d\Phi(\lambda)\right] \left[ \int_B \frac{e^{-i\omega t}}{A_t(\omega)} \, d\omega \right] dt \notag \\
    &\quad= \int_{\R} \left[ \int_B \frac{1}{2\pi T} \int_{-T}^T A_t(\lambda) e^{i(\lambda-\omega)t} \, dt \, d\omega \right] \frac{1}{A_t(\omega)} \, d\Phi(\lambda)
  \end{align}
  
  By Fubini's theorem (justified by absolute convergence from condition~\ref{regcond:l2} of Theorem~\ref{thm:oscillatory-existence}) and the definition of $K_T$:
  \begin{equation}
    = \int_{\R} \left[ \int_B K_T(\lambda, \omega) \, d\omega \right] d\Phi(\lambda)
  \end{equation}
  
  As $T \to \infty$, the inner integral $\int_B K_T(\lambda, \omega) \, d\omega \to \mathbf{1}_B(\lambda)$ for $\mu$-a.e.\ $\lambda$ by standard approximate identity theory. The dominated convergence theorem for orthogonal random measures yields:
  \begin{equation}
    \lim_{T \to \infty} \int_{\R} \left[ \int_B K_T(\lambda, \omega) \, d\omega \right] d\Phi(\lambda) = \int_{\R} \mathbf{1}_B(\lambda) \, d\Phi(\lambda) = \Phi(B)
  \end{equation}
  
  The exchange of limit and integral is justified by the uniform bound $|\int_B K_T(\lambda, \omega) \, d\omega| \leq 1$ and the finiteness of $\mu$.
\end{proof}

\end{document}


