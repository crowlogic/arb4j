\documentclass[12pt]{article}

\usepackage{amsmath}
\usepackage{amsfonts}
\usepackage{amssymb}

\title{Normalization and Fourier Analysis of Non-Stationary Processes}
\author{Technical Report}
\date{\today}

\begin{document}
\maketitle

\section{Zero-Crossing Rate Normalization}
Consider a process with zero crossing rates that increase as $|t| \to \infty$. Through normalization:

\begin{itemize}
\item Original process has increasing crossing rates
\item Normalize to maintain unit rate across whole domain
\item This enables Fourier transform analysis
\end{itemize}

\section{Process Transformation}
For a process with increasing zero crossing rates as $|t| \to \infty$:

The normalization procedure:
\begin{enumerate}
\item Apply normalization for unit zero crossing rate
\item Results in stationary process (constant crossing rate)
\item Enables Fourier transform analysis
\end{enumerate}

Key insights:
\begin{itemize}
\item Original: non-stationary (increasing crossings)
\item Post-normalization: stationary (unit crossing rate)
\item Makes Fourier transform well-defined
\item Spectral process exists for normalized version
\end{itemize}

The Fourier transform of the normalized process:
\[ \tilde{Y}(\omega) = \int_{-\infty}^{\infty} Y_{normalized}(t)e^{-i\omega t}dt \]

exists because normalization creates a well-behaved stationary process with constant crossing rate properties.

\section{Wigner-Ville Connection}
The Wigner-Ville distribution:
\[ W(t,\omega) = \int_{-\infty}^{\infty} x(t + \tau/2)x^*(t - \tau/2)e^{-i\omega\tau}d\tau \]

captures:
\begin{itemize}
\item Process that's "almost" stationary
\item Triangular/bilinear structure
\item Non-stationarity enters in controlled way
\item Maintains certain symmetries despite non-stationarity
\end{itemize}

\end{document}