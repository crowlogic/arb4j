\documentclass{article}
\usepackage[english]{babel}
\usepackage{geometry,amsmath,amssymb,latexsym}
\geometry{letterpaper}

%%%%%%%%%% Start TeXmacs macros
\newenvironment{proof}{\noindent\textbf{Proof\ }}{\hspace*{\fill}$\Box$\medskip}
\newtheorem{definition}{Definition}
\newtheorem{lemma}{Lemma}
\newtheorem{theorem}{Theorem}
%%%%%%%%%% End TeXmacs macros

\begin{document}

\title{Invertibility and Random Measure Formulas for Oscillatory Processes}

\author{Stephen Crowley}

\date{August 15, 2025}

\maketitle

{\tableofcontents}

\section{Oscillatory Gaussian Processes}

\begin{definition}
  [Orthogonal increment structure]\label{def:orthinc} Let $\mu$ be a positive
  Borel measure on $\mathbb{R}$. A complex-valued orthogonal increment process
  $Z$ is a set function on Borel subsets of $\mathbb{R}$ such that for
  disjoint $B_1, B_2 \subset \mathbb{R}$,
  \begin{equation}
    \mathbb{E} [Z (B_1) \hspace{0.17em} \overline{Z (B_2)}] = \mu (B_1 \cap
    B_2)
  \end{equation}
  and for bounded Borel $f : \mathbb{R} \to \mathbb{C}$ the stochastic
  integral
  \begin{equation}
    \int_{\mathbb{R}} f (\lambda)  \hspace{0.17em} dZ (\lambda)
  \end{equation}
  satisfies
  \begin{equation}
    \mathbb{E} \hspace{-0.17em} \left[ \left| \int_{\mathbb{R}} f (\lambda)
    \hspace{0.17em} dZ (\lambda) \right|^2 \right] = \int_{\mathbb{R}} |f
    (\lambda) |^2  \hspace{0.17em} \mu (d \lambda)
  \end{equation}
\end{definition}

\begin{definition}
  [White noise process]\label{def:whitenoise} A complex white noise process
  $W$ is an orthogonal increment process satisfying
  \begin{equation}
    \mathbb{E} [dW (u_1) \hspace{0.17em} \overline{dW (u_2)}] = \delta (u_1 -
    u_2)  \hspace{0.17em} du_1
  \end{equation}
\end{definition}

\begin{definition}
  [Stationary process]\label{def:stationary} The stationary process $X_s (t)$
  generated from white noise $W$ is
  \begin{equation}
    \label{eq:stationary-rep} X_s (t) = \int_{- \infty}^{\infty} e^{i \omega
    t}  \hspace{0.17em} dW (\omega)
  \end{equation}
  The process has covariance
  \begin{equation}
    \mathbb{E} [X_s (t_1) \hspace{0.17em} \overline{X_s (t_2)}] = \int_{-
    \infty}^{\infty} e^{i \omega (t_1 - t_2)}  \hspace{0.17em} d \omega = 2
    \pi \delta (t_1 - t_2)
  \end{equation}
\end{definition}

\begin{definition}
  [Time-dependent filter and gain]\label{def:filter-gain} The time-dependent
  filter $h (t, u)$ and gain function $A (t, \lambda)$ satisfy the Fourier
  transform pair
  \begin{equation}
    \label{eq:gain-from-filter} A (t, \lambda) = \int_{- \infty}^{\infty} h
    (t, u)  \hspace{0.17em} e^{- i \lambda (t - u)}  \hspace{0.17em} du
  \end{equation}
  \begin{equation}
    \label{eq:filter-from-gain} h (t, u) = \frac{1}{2 \pi}  \int_{-
    \infty}^{\infty} A (t, \lambda)  \hspace{0.17em} e^{i \lambda (t - u)} 
    \hspace{0.17em} d \lambda
  \end{equation}
  with square-integrability
  \begin{equation}
    \int_{- \infty}^{\infty} |h (t, u) |^2  \hspace{0.17em} du < \infty \quad
    \forall t \in \mathbb{R}.
  \end{equation}
\end{definition}

\begin{definition}
  [Oscillatory process]\label{def:oscproc} An oscillatory process is defined
  in three equivalent ways:
  
  \begin{align}
    X (t) & = \int_{\mathbb{R}} A (t, \lambda)  \hspace{0.17em} e^{i \lambda
    t}  \hspace{0.17em} dZ (\lambda)  \label{eq:osc-spectral}\\
    X (t) & = \int_{- \infty}^{\infty} h (t, u)  \hspace{0.17em} dW (u) 
    \label{eq:osc-filter}\\
    X (t) & = \int_{- \infty}^{\infty} h (t, u)  \hspace{0.17em} X_s  (t - u) 
    \hspace{0.17em} du  \label{eq:osc-convolution}
  \end{align}
  
  where $Z$, $W$, $X_s$, $h$, and $A$ are related by Definitions
  \ref{def:orthinc}--\ref{def:filter-gain}, and
  \begin{equation}
    \label{eq:Atlambda-L2-mu} \int_{\mathbb{R}} |A (t, \lambda) |^2 
    \hspace{0.17em} \mu (d \lambda) < \infty
  \end{equation}
  The covariance function is
  \begin{equation}
    \label{eq:covariance} \mathbb{E} [X (t_1) \hspace{0.17em} \overline{X
    (t_2)}] = \int_{\mathbb{R}} A (t_1, \lambda) \hspace{0.17em} \overline{A
    (t_2, \lambda)} \hspace{0.17em} e^{i \lambda (t_1 - t_2)}  \hspace{0.17em}
    \mu (d \lambda)
  \end{equation}
\end{definition}

\subsection{Amplitude and orthogonality}

\begin{definition}
  [Amplitude nondegeneracy]\label{def:nondeg} The amplitude $A$ satisfies
  \begin{equation}
    \label{eq:nonzero} A (t, \lambda) \neq 0 \quad \text{for all $(t,
    \lambda)$ in the domain.}
  \end{equation}
\end{definition}

\begin{definition}
  [Kernel orthonormality]\label{def:orthonormality} The amplitude satisfies
  \begin{equation}
    \label{eq:delta-ortho} \int_{- \infty}^{\infty} A (t, \lambda_1) 
    \hspace{0.17em} A (t, \lambda_2)  \hspace{0.17em} e^{i (\lambda_2 -
    \lambda_1) t}  \hspace{0.17em} dt = \delta (\lambda_1 - \lambda_2)
  \end{equation}
\end{definition}

\subsection{Inversion map}

\begin{definition}
  [Inversion operator]\label{def:invop} Define
  \begin{equation}
    \label{eq:invop} (\mathcal{I}X) (\lambda) = \int_{- \infty}^{\infty} A (t,
    \lambda)  \hspace{0.17em} e^{- i \lambda t}  \hspace{0.17em} X (t) 
    \hspace{0.17em} dt
  \end{equation}
\end{definition}

\section{Invertibility Conditions}

\begin{theorem}
  [Fundamental Invertibility]\label{thm:fund-inv} For $X$ as in Definition
  \ref{def:oscproc},
  \begin{equation}
    \label{eq:inv-identity} dZ (\lambda) = \int_{- \infty}^{\infty} A (t,
    \lambda)  \hspace{0.17em} e^{- i \lambda t}  \hspace{0.17em} X (t) 
    \hspace{0.17em} dt
  \end{equation}
  if and only if $A$ satisfies \eqref{eq:nonzero} and \eqref{eq:delta-ortho}.
\end{theorem}

\begin{proof}
  \begin{enumerate}
    \item From \eqref{eq:osc-spectral},
    \begin{equation}
      X (t) = \int_{\mathbb{R}} A (t, \lambda)  \hspace{0.17em} e^{i \lambda
      t}  \hspace{0.17em} dZ (\lambda)
    \end{equation}
    Multiply by $A (t, \lambda_0) e^{- i \lambda_0 t}$ and integrate over $t$:
    \begin{equation}
      \int_{- \infty}^{\infty} A (t, \lambda_0) e^{- i \lambda_0 t} X (t) 
      \hspace{0.17em} dt = \int_{- \infty}^{\infty} A (t, \lambda_0) e^{- i
      \lambda_0 t} \left[ \int_{\mathbb{R}} A (t, \lambda) e^{i \lambda t} 
      \hspace{0.17em} dZ (\lambda) \right] dt
    \end{equation}
    \item Swap order of integration:
    \begin{equation}
      = \int_{\mathbb{R}} \left[ \int_{- \infty}^{\infty} A (t, \lambda_0) A
      (t, \lambda) e^{i (\lambda - \lambda_0) t} dt \right] dZ (\lambda)
    \end{equation}
    \item Apply \eqref{eq:delta-ortho}:
    \begin{equation}
      = \int_{\mathbb{R}} \delta (\lambda - \lambda_0)  \hspace{0.17em} dZ
      (\lambda) = dZ (\lambda_0)
    \end{equation}
    \item Conversely, insert
    \begin{equation}
      X_{\lambda_0} (t) = A (t, \lambda_0) e^{i \lambda_0 t}
    \end{equation}
    into \eqref{eq:inv-identity}:
    \begin{equation}
      dZ_{\lambda_0} (\lambda) = \int_{- \infty}^{\infty} A (t, \lambda) e^{-
      i \lambda t} A (t, \lambda_0) e^{i \lambda_0 t} dt
    \end{equation}
    The left side equals $\delta (\lambda - \lambda_0)$, hence
    \eqref{eq:delta-ortho} holds. Nondegeneracy from linear independence
    follows by evaluating at $(t, \lambda)$ where $X (t) \neq 0$.
  \end{enumerate}
  
\end{proof}

\begin{lemma}
  [Uniqueness]\label{lem:unique} If $\mathcal{I}_1 X = dZ (\lambda)
  =\mathcal{I}_2 X$ for all $X$, then $\mathcal{I}_1 =\mathcal{I}_2$.
\end{lemma}

\begin{proof}
  \begin{enumerate}
    \item Let $\mathcal{L}=\mathcal{I}_1 -\mathcal{I}_2$. Choose
    \begin{equation}
      X_{\lambda_0} (t) = A (t, \lambda_0) e^{i \lambda_0 t}
    \end{equation}
    .
    
    \item Then $(\mathcal{L}X_{\lambda_0}) (\lambda)$ equals
    \begin{equation}
      \int_{- \infty}^{\infty} A (t, \lambda) e^{- i \lambda t} A (t,
      \lambda_0) e^{i \lambda_0 t} dt - \int_{- \infty}^{\infty} A (t,
      \lambda) e^{- i \lambda t} A (t, \lambda_0) e^{i \lambda_0 t} dt = 0
    \end{equation}
    \item Density of the span $\{X_{\lambda_0} \}$ implies $\mathcal{L}= 0$.
  \end{enumerate}
  
\end{proof}

\section{Real-Valuedness}

\begin{definition}
  [Real-valued oscillatory process]\label{def:real} An oscillatory process $X$
  given by \eqref{eq:osc-spectral} is real-valued when
  \begin{equation}
    \label{eq:real-cond} X (t) \in \mathbb{R} \quad \text{for all } t \in
    \mathbb{R}
  \end{equation}
  which requires the symmetry
  \begin{equation}
    \label{eq:hermitian} A (t, - \lambda)  \hspace{0.17em} dZ (- \lambda) =
    \overline{A (t, \lambda)  \hspace{0.17em} dZ (\lambda)}
  \end{equation}
\end{definition}

\section{Orthonormality Expanded}

\begin{theorem}
  [Triple integral expansion of orthonormality] The orthonormality condition
  \eqref{eq:delta-ortho} expands as
  \begin{equation}
    \begin{array}{ll}
      \int_{- \infty}^{\infty} \int_{- \infty}^{\infty} \int_{-
      \infty}^{\infty} h (t, u_1) h (t, u_2) e^{- i \lambda_1  (t - u_1)} e^{-
      i \lambda_2  (t - u_2)} e^{i (\lambda_2 - \lambda_1) t}  \hspace{0.17em}
      du_1  \hspace{0.17em} du_2  \hspace{0.17em} dt & = \delta (\lambda_1 -
      \lambda_2)
    \end{array} \label{eq:triple-integral}
  \end{equation}
\end{theorem}

\begin{proof}
  
  \begin{enumerate}
    \item Substitute \eqref{eq:gain-from-filter} into \eqref{eq:delta-ortho}
    and expand integrals to obtain the triple integral form.
    
    \item Correct simplification:
    \begin{equation}
      e^{- i \lambda_1  (t - u_1)} e^{- i \lambda_2  (t - u_2)} e^{i
      (\lambda_2 - \lambda_1) t} = e^{i \lambda_1 u_1} e^{i \lambda_2 u_2}
      e^{- 2 i \lambda_1 t}
    \end{equation}
    \item The $\delta (\lambda_1 - \lambda_2)$ factor arises only after
    integrating over all variables and invoking distributional Fourier
    inversion; it does not follow from the $t$-integral alone. This correction
    ensures rigor.
  \end{enumerate}
\end{proof}

\section{Random Measure Equivalences}

\begin{theorem}
  [Complete random measure formula] Define
  \begin{equation}
    \Phi (\lambda) = \int_{- \infty}^{\lambda} dZ (\nu)
  \end{equation}
  where $dZ (\nu)$ satisfies \eqref{eq:inv-identity}. Then, in the
  distributional sense,
  \begin{equation}
    \begin{array}{ll}
      \label{eq:phi-complete} \Phi (\lambda) & = \int_{- \infty}^{\infty}
      \frac{1 - e^{- i \lambda u}}{iu}  \hspace{0.17em} dW (u)\\
      & = \int_{- \infty}^{\infty} \frac{1 - e^{- i \lambda t}}{it} 
      \hspace{0.17em} X (t)  \hspace{0.17em} dt
    \end{array}
  \end{equation}
\end{theorem}

\begin{proof}
  \begin{enumerate}
    \item From the white noise representation,
    \begin{equation}
      dZ (\lambda) = \frac{1}{2 \pi}  \int_{- \infty}^{\infty} e^{- i \lambda
      u}  \hspace{0.17em} dW (u)
    \end{equation}
    \item Interpret $\int_{- \infty}^{\lambda} e^{- i \nu u}  \hspace{0.17em}
    d \nu$ in the tempered distribution sense:
    \begin{equation}
      \int_{- \infty}^{\lambda} e^{- i \nu u}  \hspace{0.17em} d \nu = \pi
      \delta (u) + \frac{1 - e^{- i \lambda u}}{iu}
    \end{equation}
    The Dirac term vanishes after pairing with $dW (u)$ for $u \neq 0$.
    
    \item Substitution yields the first equality in \eqref{eq:phi-complete}.
    
    \item The time-domain form follows by swapping the inversion formula into
    \eqref{eq:inv-identity} and applying the same distributional identity in
    $t$.
  \end{enumerate}
\end{proof}

\section{Remarks on Structure}

\subsection*{Summary of conditions}

\begin{equation}
  \label{eq:summary-1} X (t) = \int_{\mathbb{R}} A (t, \lambda) 
  \hspace{0.17em} e^{i \lambda t}  \hspace{0.17em} dZ (\lambda)
\end{equation}
\begin{equation}
  \label{eq:summary-2} \mathbb{E} [dZ (\lambda_1) \hspace{0.17em} \overline{dZ
  (\lambda_2)}] = \delta (\lambda_1 - \lambda_2)  \hspace{0.17em} \mu (d
  \lambda_1)
\end{equation}
\begin{equation}
  \label{eq:summary-3} \int_{- \infty}^{\infty} A (t, \lambda_1) 
  \hspace{0.17em} A (t, \lambda_2)  \hspace{0.17em} e^{i (\lambda_2 -
  \lambda_1) t}  \hspace{0.17em} dt = \delta (\lambda_2 - \lambda_1)
\end{equation}
\begin{equation}
  \label{eq:summary-4} dZ (\lambda) = \int_{- \infty}^{\infty} A (t, \lambda) 
  \hspace{0.17em} e^{- i \lambda t}  \hspace{0.17em} X (t)  \hspace{0.17em} dt
\end{equation}

\subsection*{Covariance identity}

From \eqref{eq:summary-1} and \eqref{eq:summary-2},
\begin{equation}
  \label{eq:cov-id} \mathbb{E} [X (t_1) \hspace{0.17em} \overline{X (t_2)}] =
  \int_{\mathbb{R}} A (t_1, \lambda) \hspace{0.17em} \overline{A (t_2,
  \lambda)} \hspace{0.17em} e^{i \lambda (t_1 - t_2)}  \hspace{0.17em} \mu (d
  \lambda)
\end{equation}

\subsection*{Necessity and sufficiency}

Equation \eqref{eq:summary-3} and nondegeneracy \eqref{eq:nonzero} are
necessary and sufficient for the inversion \eqref{eq:summary-4} by Theorem
\ref{thm:fund-inv}. Lemma \ref{lem:unique} gives uniqueness.

\section{References}

{\noindent}Priestley, M.B. (1965). Evolutionary spectra and non-stationary
processes. Journal of the Royal Statistical Society: Series B, 27(2),
204--237.\\
Priestley, M.B. (1981). Spectral Analysis and Time Series. Academic Press.

\end{document}
