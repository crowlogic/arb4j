\documentclass{article}
\usepackage[english]{babel}
\usepackage{geometry}
\geometry{letterpaper}

%%%%%%%%%% Start TeXmacs macros
\newcommand{\tmaffiliation}[1]{\\ #1}
%%%%%%%%%% End TeXmacs macros

\begin{document}

\title{Eigenfunction and Eigenvalue of the Sine Kernel}

\author{
  Stephen Crowley
  \tmaffiliation{April 25, 2024}
}

\maketitle

\section{Introduction}

The sine kernel is a significant element in the study of Gaussian processes
and random matrices. It is defined by the integral operator with kernel:
\begin{equation}
  K (x, y) = \frac{\sin (x - y)}{\pi (x - y)}
\end{equation}

\section{Eigenfunction and Eigenvalue}

An important eigenfunction of the sine kernel is $\sin (x)$, which satisfies
the integral equation:
\begin{equation}
  \int_{- \infty}^{\infty} \frac{\sin (x - y)}{\pi (x - y)} \sin (y) 
  \hspace{0.17em} dy = \sin (x)
\end{equation}
This shows that $\sin (x)$ is an eigenfunction of the integral operator with
the sine kernel, and its corresponding eigenvalue is 1.

\section{Identity Validation}

The identity involving the sine kernel can be verified through:
\begin{equation}
  \int_{- \infty}^{\infty} \frac{\sin (x - y) \sin y}{(x - y) \pi^2 y} 
  \hspace{0.17em} dy = \frac{\sin x}{x \pi}
\end{equation}
This integral represents a convolution of the sine function under the sine
kernel, emphasizing the role of the sine function and its spectral properties
in the context of the sine kernel.

\end{document}
