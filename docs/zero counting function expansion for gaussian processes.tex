

\documentclass{article}
\usepackage{amsmath, amssymb, amsthm}
\usepackage{mathtools}

\begin{document}

\section{Zero Counting Function via Regularized Transform}

\subsection{Forward Proof}
Starting with our counting function with KL expansion:
\begin{equation}
    N(T) = \lim_{\epsilon \to 0} \frac{1}{2\pi} \int_0^T \int_{-\infty}^{\infty} J_0(\epsilon r)|r| \exp\left(-ir\sum_{n=1}^{\infty} Z_n \sqrt{\lambda_n} \phi_n(t)\right) dr dt
\end{equation}

By Fubini's theorem:
\begin{equation}
    N(T) = \lim_{\epsilon \to 0} \frac{1}{2\pi} \int_{-\infty}^{\infty} J_0(\epsilon r)|r| \exp\left(-ir\sum_{n=1}^{\infty} Z_n \sqrt{\lambda_n} \int_0^T \phi_n(t) dt\right) dr
\end{equation}

Using the integral representation of $J_0$:
\begin{equation}
    J_0(\epsilon r) = \frac{1}{2\pi} \int_0^{2\pi} e^{i\epsilon r\cos\theta} d\theta
\end{equation}

Substituting:
\begin{equation}
    \lim_{\epsilon \to 0} \frac{1}{2\pi} \int_{-\infty}^{\infty} |r| \int_0^{2\pi} e^{i\epsilon r\cos\theta} \exp\left(-ir\sum_{n=1}^{\infty} Z_n \sqrt{\lambda_n} \int_0^T \phi_n(t) dt\right) d\theta dr
\end{equation}

By Fubini's theorem:
\begin{equation}
    \lim_{\epsilon \to 0} \frac{1}{2\pi} \int_0^{2\pi} \int_{-\infty}^{\infty} |r| e^{ir(\epsilon\cos\theta - \sum_{n=1}^{\infty} Z_n \sqrt{\lambda_n} \int_0^T \phi_n(t) dt)} dr d\theta
\end{equation}

The inner integral evaluates to:
\begin{equation}
    \int_{-\infty}^{\infty} |r| e^{ir(\epsilon\cos\theta - \sum_{n=1}^{\infty} Z_n \sqrt{\lambda_n} \int_0^T \phi_n(t) dt)} dr = \frac{2}{(\epsilon\cos\theta - \sum_{n=1}^{\infty} Z_n \sqrt{\lambda_n} \int_0^T \phi_n(t) dt)^2}
\end{equation}

Therefore:
\begin{equation}
    \frac{1}{\pi} \int_0^{2\pi} \frac{1}{(\epsilon\cos\theta - \sum_{n=1}^{\infty} Z_n \sqrt{\lambda_n} \int_0^T \phi_n(t) dt)^2} d\theta = \frac{1}{((\sum_{n=1}^{\infty} Z_n \sqrt{\lambda_n} \int_0^T \phi_n(t) dt)^2 + \epsilon^2)^{3/2}}
\end{equation}

Thus our counting function becomes:
\begin{equation}
    N(T) = \lim_{\epsilon \to 0} \frac{1}{2\pi} \frac{1}{((\sum_{n=1}^{\infty} Z_n \sqrt{\lambda_n} \int_0^T \phi_n(t) dt)^2 + \epsilon^2)^{3/2}}
\end{equation}

\subsection{Reverse Verification}
Starting from our final form:
\begin{equation}
    N(T) = \lim_{\epsilon \to 0} \frac{1}{2\pi} \frac{1}{((\sum_{n=1}^{\infty} Z_n \sqrt{\lambda_n} \int_0^T \phi_n(t) dt)^2 + \epsilon^2)^{3/2}}
\end{equation}

Using the integral representation:
\begin{equation}
    \frac{1}{((\sum_{n=1}^{\infty} Z_n \sqrt{\lambda_n} \int_0^T \phi_n(t) dt)^2 + \epsilon^2)^{3/2}} = \frac{1}{\pi} \int_0^{2\pi} \frac{1}{(\epsilon\cos\theta - \sum_{n=1}^{\infty} Z_n \sqrt{\lambda_n} \int_0^T \phi_n(t) dt)^2} d\theta
\end{equation}

This can be rewritten using:
\begin{equation}
    \frac{1}{(\epsilon\cos\theta - \sum_{n=1}^{\infty} Z_n \sqrt{\lambda_n} \int_0^T \phi_n(t) dt)^2} = \int_{-\infty}^{\infty} |r| e^{ir(\epsilon\cos\theta - \sum_{n=1}^{\infty} Z_n \sqrt{\lambda_n} \int_0^T \phi_n(t) dt)} dr
\end{equation}

The $\theta$ integral gives us back the Bessel function:
\begin{equation}
    \frac{1}{2\pi} \int_0^{2\pi} e^{ir\epsilon\cos\theta} d\theta = J_0(\epsilon r)
\end{equation}

Combining these steps:
\begin{equation}
    N(T) = \lim_{\epsilon \to 0} \frac{1}{2\pi} \int_{-\infty}^{\infty} J_0(\epsilon r)|r| \exp\left(-ir\sum_{n=1}^{\infty} Z_n \sqrt{\lambda_n} \int_0^T \phi_n(t) dt\right) dr
\end{equation}

By Fubini's theorem:
\begin{equation}
    N(T) = \lim_{\epsilon \to 0} \frac{1}{2\pi} \int_0^T \int_{-\infty}^{\infty} J_0(\epsilon r)|r| \exp\left(-ir\sum_{n=1}^{\infty} Z_n \sqrt{\lambda_n} \phi_n(t)\right) dr dt
\end{equation}

Which is our original counting function, verifying the equivalence.

\section*{Commentary on Mathematical Structure}
The appearance of the Bessel function $J_0(\epsilon r)$ in this regularization is not merely a computational convenience, but reveals deep connections to the Hankel transform and the underlying structure of phase space. This formulation naturally emerges from Fourier duality principles and the representation theory of the Heisenberg group. The radial nature of $J_0(\epsilon r)$ in frequency space, combined with the phase information in the exponential term, respects the fundamental symmetries of phase space. What initially appeared as a technical necessity for convergence (since the integral diverges without the Bessel function regularization) actually illuminates the profound mathematical structure governing the counting function. The regularization through $J_0(\epsilon r)$ provides exactly the right kind of "uncertainty principle" behavior that aligns with the natural symmetries of the Heisenberg group and the radial structure of phase space transformations.

\end{document}

