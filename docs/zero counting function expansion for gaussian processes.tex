\documentclass{article}
\usepackage{amsmath, amssymb, amsthm}
\usepackage{mathtools}

\begin{document}

\section{Zero Counting Function via Regularized Transform for Gaussian Processes on $[0,\infty)$}

\subsection{Theorem Statement and Preliminaries}
Let $\{X_t\}_{t\in[0,\infty)}$ be a real-valued centered Gaussian process with continuous sample paths and covariance operator $K$ which is compact relative to the canonical metric induced by its kernel. The operator $K$ is not required to be trace class. 

The canonical metric $d$ on $[0,\infty)$ is given by: \begin{equation} d(s,t) = \sqrt{K(s,s) + K(t,t) - 2K(s,t)} \end{equation}

Compactness relative to the canonical metric is characterized by the finiteness of Dudley's metric entropy integral:
\begin{equation}
    \int_0^1 \sqrt{\log N(\varepsilon,B_T,d)} \, d\varepsilon < \infty
\end{equation}
where the covering number $N(\varepsilon,B_T,d)$ can be expressed as:
\begin{equation}
    N(\varepsilon,B_T,d) = \sup_{t \in B_T} \inf\{n : \exists \{x_1,\ldots,x_n\} \subset B_T \text{ s.t. } \bigcup_{i=1}^n B_d(x_i,\varepsilon) \supseteq B_T\}
\end{equation}
This number admits an upper bound in terms of the eigenvalues (Dudley, 1999)\footnote{Dudley, R. M. (1999). Uniform Central Limit Theorems. Cambridge University Press.}:
\begin{equation}
    N(\varepsilon,B_T,d) \leq \min\{n \in \mathbb{N} : \lambda_n < \varepsilon^2\}
\end{equation}
where $\{\lambda_n\}$ are the eigenvalues of $K$. The upper limit of integration is 1 because functions in the reproducing kernel Hilbert space (RKHS) have norm bounded by 1, making this the maximum possible error in the canonical metric.

The process admits two equivalent representations through the Karhunen-Loève expansion:

\subsubsection*{Path to Coefficients}
Given a path $X_t$, its projection coefficients are:
\begin{equation}
    Z_n = \frac{1}{\sqrt{\lambda_n}} \int_0^\infty X_t \phi_n(t) dt
\end{equation}

\subsubsection*{Coefficients to Path}
Given the projection coefficients $\{Z_n\}$, the path is reconstructed as:
\begin{equation}
    X_t = \sum_{n=0}^{\infty} Z_n \sqrt{\lambda_n} \phi_n(t)
\end{equation}
where $\{\lambda_n, \phi_n(t)\}$ are the eigenvalue-eigenfunction pairs of $K$.

\subsection{Zero Counting Function Development}
Starting with the regularized Hankel transform representation:
\begin{equation}
\begin{split}
    N(T) &= \lim_{\epsilon \to 0} \frac{1}{2\pi} \int_0^T \int_{-\infty}^{\infty} J_0(\epsilon r)|r| \exp\left(-ir\sum_{n=0}^{\infty} Z_n \sqrt{\lambda_n} \phi_n(t)\right) dr dt \\
    &= \lim_{\epsilon \to 0} \frac{1}{2\pi} \int_0^T \mathcal{H}_{0,r\to\epsilon}\left[\exp\left(-ir\sum_{n=0}^{\infty} Z_n \sqrt{\lambda_n} \phi_n(t)\right)\right] dt
\end{split}
\end{equation}
where $\mathcal{H}_{0,r\to s}[f] = \int_0^\infty rf(r)J_0(sr)dr$

By Fubini's theorem:
\begin{equation}
\begin{split}
    N(T) &= \lim_{\epsilon \to 0} \frac{1}{2\pi} \mathcal{H}_{0,r\to\epsilon}\left[\exp\left(-ir\sum_{n=0}^{\infty} Z_n \sqrt{\lambda_n} \int_0^T \phi_n(t) dt\right)\right] \\
    &= \lim_{\epsilon \to 0} \frac{1}{2\pi} \int_{-\infty}^{\infty} J_0(\epsilon r)|r| \exp\left(-ir\sum_{n=0}^{\infty} Z_n \sqrt{\lambda_n} \int_0^T \phi_n(t) dt\right) dr
\end{split}
\end{equation}

Using the integral representation of $J_0$:
\begin{equation}
\begin{split}
    N(T) &= \lim_{\epsilon \to 0} \frac{1}{2\pi} \int_{-\infty}^{\infty} |r| \left(\frac{1}{2\pi} \int_0^{2\pi} e^{i\epsilon r\cos\theta} d\theta\right) \exp\left(-ir\sum_{n=0}^{\infty} Z_n \sqrt{\lambda_n} \int_0^T \phi_n(t) dt\right) dr \\
    &= \lim_{\epsilon \to 0} \frac{1}{2\pi} \int_0^{2\pi} \int_{-\infty}^{\infty} |r| e^{ir(\epsilon\cos\theta - \sum_{n=0}^{\infty} Z_n \sqrt{\lambda_n} \int_0^T \phi_n(t) dt)} dr d\theta
\end{split}
\end{equation}

The inner integral evaluates to:
\begin{equation}
    \int_{-\infty}^{\infty} |r| e^{ir(\epsilon\cos\theta - \sum_{n=0}^{\infty} Z_n \sqrt{\lambda_n} \int_0^T \phi_n(t) dt)} dr = \frac{2}{(\epsilon\cos\theta - \sum_{n=0}^{\infty} Z_n \sqrt{\lambda_n} \int_0^T \phi_n(t) dt)^2}
\end{equation}

Therefore:
\begin{equation}
\begin{split}
    N(T) &= \lim_{\epsilon \to 0} \frac{1}{\pi} \int_0^{2\pi} \frac{1}{(\epsilon\cos\theta - \sum_{n=0}^{\infty} Z_n \sqrt{\lambda_n} \int_0^T \phi_n(t) dt)^2} d\theta \\
    &= \lim_{\epsilon \to 0} \frac{1}{2\pi} \frac{1}{((\sum_{n=0}^{\infty} Z_n \sqrt{\lambda_n} \int_0^T \phi_n(t) dt)^2 + \epsilon^2)^{3/2}}
\end{split}
\end{equation}

\section*{Commentary on Mathematical Structure}
The appearance of $\epsilon$ within the Bessel function $J_0(\epsilon r)$ in this regularization reveals deep connections to the Hankel transform $\mathcal{H}_{0,r\to\epsilon}$ and the underlying structure of phase space. The scaling parameter $\epsilon$ acts as a resolution parameter in the dual space, providing the necessary regularization for the counting function while respecting the fundamental symmetries of the space.

\end{document}