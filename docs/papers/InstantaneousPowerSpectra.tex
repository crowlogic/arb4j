\documentclass{article}
\usepackage[english]{babel}
\usepackage{geometry,amsmath}
\geometry{letterpaper}

%%%%%%%%%% Start TeXmacs macros
\newcommand{\tmop}[1]{\ensuremath{\operatorname{#1}}}
\newcommand{\tmtextit}[1]{\text{{\itshape{#1}}}}
\newtheorem{definition}{Definition}
%%%%%%%%%% End TeXmacs macros

\begin{document}

\title{Instantaneous Power Spectra}

\author{Chester H. Page}

\date{}

\maketitle

\begin{center}
  National Bureau of Standards, Washington, D.C.\\
  (Received August 16, 1951)
\end{center}

{\tableofcontents}

\section{Introduction}

The intuitive concept of a changing spectrum is discussed. The instantaneous
power spectrum is defined mathematically and used to make the intuitive
concepts more precise. It depends upon the past history of a signal, but not
upon the future.

Integration of the instantaneous power spectrum over time yields the
conventional energy spectrum. The instantaneous power spectrum of a random
function may be averaged over the ensemble of functions, with a resulting
stochastic average instantaneous power spectrum that is equal to the
conventional time average power spectrum of a stochastic process.

The ordinary concept of the frequency spectrum of a signal is often confusing
in practice. The spectrum is expressed in terms of the Fourier transform of
the signal; the Fourier transform is determined by the complete history of the
signal from $t = - \infty$ to $t = + \infty$. In practice, we are interested
in what is happening now, and this is independent of what the signal may do in
the future. It is a familiar fact that in the Fourier solution of a circuit
problem, the future behavior of the applied signal does not affect the result,
even though it apparently enters the calculation via the transform.

For example, let a man and woman sing alternate verses of a song. It is a
common intuitive reaction to say that the spectrum keeps changing. We somehow
feel that a sample of sufficient duration defines a spectrum, yet the
mathematical definition of spectrum disallows this. The paradox becomes more
apparent when we decrease the time intervals for spectral comparison and ask
how the spectrum changes in successive milliseconds. This is ``obviously'' an
absurd query. The paradox is somewhat resolved by a sharper analysis. By the
spectrum of the male verses we really mean the power spectrum of this man's
singing, or mathematically the power spectrum of the stochastic process of
which the singing of one verse is a statistically large sample. This standard
concept of power spectrum will be discussed later. The role of duration is now
clarified; one verse is statistically sufficient to give a reasonably precise
determination of the process and its spectrum, one syllable is not a large
enough sample.

The paradox has not, however, been completely vanquished. There is certainly
nothing statistical about a pure sine wave. Let us play a simple melody on a
very crude electric organ, or more ideally, change the frequency of an audio
oscillator. We describe both these situations in terms of frequency changes in
time. Again, each tone must be of sufficient duration to define a frequency.
But how long does it take to establish a frequency? Indeed, what do we mean by
a frequency, since any signal of finite duration has a spectrum that is not
infinitely sharp? A unique frequency exists only for a sine wave of infinite
duration.

There must be some description of signals that asymptotically agrees with the
above intuitive properties. That is, a sine wave switched on at $t = 0$ has
some sort of spectral distribution that can be represented more and more
closely by a single frequency as time goes on. Similarly, a stochastic process
can be spectrally described in some manner that progressively agrees more
closely with the power spectrum of the process. We seek a description in terms
of an ``instantaneous power spectrum'' with the above properties.

\section{Mathematical Formulation}

\begin{definition}
  [Energy Spectrum] The (energy) spectrum of a signal is defined as the
  absolute square of the Fourier transform of the signal. Thus a signal $G
  (t)$ has the Fourier transform
  \begin{equation}
    g (f) = \int_{- \infty}^{\infty} e^{- 2 \pi ift} G (t)  \hspace{0.17em} dt
    \label{eq:fourier_transform}
  \end{equation}
  if the integral converges.
\end{definition}

Plancherel's theorem states that
\begin{equation}
  \int_{- \infty}^{\infty} |g (f) |^2 \hspace{0.17em} df = \int_{-
  \infty}^{\infty} |G (t) |^2  \hspace{0.17em} dt \label{eq:plancherel}
\end{equation}
Since $G^2 (t)$ is the instantaneous power (into unit resistance),
\eqref{eq:plancherel} represents the total energy of the signal, and $|g|^2$
is interpreted as the energy distribution in frequency, or the energy
spectrum. Thus the spectrum depends upon the entire history of the signal, by
virtue of the infinite limits in \eqref{eq:fourier_transform}. This property
of the spectrum is responsible for many apparent paradoxes and much confusion
in description.

Consider the energy of a signal to be distributed in time and frequency.
Denote the energy density in this $t - f$ plane by $\rho (t, f)$. The total
energy expended up to time $T$ will be
\[ \int_{- \infty}^T \int_{- \infty}^{\infty} \rho (t, f)  \hspace{0.17em} df
   \hspace{0.17em} dt \]
The rate of increase of total energy is the instantaneous power, and is simply
\begin{equation}
  \int_{- \infty}^{\infty} \rho (T, f)  \hspace{0.17em} df
  \label{eq:instantaneous_power}
\end{equation}
by differentiating the energy. Thus at any time $T$, the distribution $\rho
(T, f)$ can be considered as the instantaneous power spectrum.

Suppose the signal $G (t)$ to be switched off at $t = T$. This defines an
auxiliary signal
\begin{equation}
  G_a (t) = \left\{\begin{array}{ll}
    G (t) & t < T\\
    0 & t > T
  \end{array}\right. \label{eq:auxiliary_signal}
\end{equation}
which is identical with $G (t)$ up till time $T$, and therefore up till time
$T$ has delivered the same energy as $G (t)$. Since $G_a$ is defined for all
time, we can compute its Fourier transform and its energy spectrum, and the
latter should agree with the time integrated power spectrum of $G (t)$. That
is, we require $\rho (t, f)$ to satisfy the condition
\begin{equation}
  \int_{- \infty}^T \rho (t, f)  \hspace{0.17em} dt = | g_a (f) |^2
  \label{eq:integrated_spectrum}
\end{equation}
where $g_a (f)$ is the transform of $G_a (t)$. Since this condition must be
satisfied for all $T$, it is convenient to define the running transform of $G
(t)$ as the transform of a continually changing auxiliary function:
\begin{equation}
  g_t (f) = \int_{- \infty}^{\infty} G_t (x) e^{- 2 \pi ifx}  \hspace{0.17em}
  dx = \int_{- \infty}^t G (x) e^{- 2 \pi ifx}  \hspace{0.17em} dx
  \label{eq:running_transform}
\end{equation}
The instantaneous power spectrum must now satisfy
\begin{equation}
  \int_{- \infty}^t \rho (x, f)  \hspace{0.17em} dx = | g_t (f) |^2
  \label{eq:cumulative_spectrum}
\end{equation}
and this equation is sufficient to determine $\rho$. Differentiation with
respect to $t$ yields
\begin{equation}
  \rho (t, f) = \frac{\partial}{\partial t} | g_t (f) |^2
  \label{eq:instantaneous_spectrum}
\end{equation}
The expression for $\rho (t, f)$ can profitably be written in other forms.
First, substitute \eqref{eq:running_transform} into
\eqref{eq:instantaneous_spectrum}, so that
\begin{equation}
  \rho (t, f) = \frac{\partial}{\partial t} \left\{ \int_{- \infty}^t G (x)
  e^{- 2 \pi ifx}  \hspace{0.17em} dx \int_{- \infty}^t G (y) e^{+ 2 \pi ify} 
  \hspace{0.17em} dy \right\} \label{eq:instantaneous_spectrum_expanded}
\end{equation}
This is usually the most convenient form for calculation of $\rho (t, f)$. The
intermediate result \eqref{eq:instantaneous_spectrum_expanded} can be used as
follows. Integrating over frequency gives:

\begin{align}
  \int_{- \infty}^{\infty} \rho (t, f)  \hspace{0.17em} df & = G (t) \left(
  \left\{ \int_{- \infty}^{\infty} \int_{- \infty}^t G (y) e^{2 \pi if (y -
  t)}  \hspace{0.17em} dy \hspace{0.17em} df \right. + \int_{-
  \infty}^{\infty} \int_{- \infty}^t G (x) e^{- 2 \pi if (x - t)} 
  \hspace{0.17em} dx \hspace{0.17em} df \right) \nonumber\\
  & = 2 G (t)  \int_{- \infty}^{\infty} \int_{- \infty}^{\infty} G_t (x) \cos
  2 \pi f (x - t)  \hspace{0.17em} dx \hspace{0.17em} df 
\end{align}

By Fourier's theorem, the double integral is
\begin{equation}
  \frac{1}{2}  [G_t (t - 0) + G_t (t + 0)] = \frac{1}{2} G (t)
\end{equation}
by virtue of the discontinuity in $G_t (x)$ by definition
\eqref{eq:auxiliary_signal}. Therefore,
\begin{equation}
  \int_{- \infty}^{\infty} \rho (t, f)  \hspace{0.17em} df = G^2 (t) =
  \text{instantaneous power} \label{eq:energy_conservation}
\end{equation}
as expected. Equation \eqref{eq:instantaneous_spectrum_expanded} can also be
written as
\begin{equation}
  \rho (t, f) = 2 G (t)  \int_{- \infty}^t G (x) \cos 2 \pi f (x - t) 
  \hspace{0.17em} dx \label{eq:instantaneous_spectrum_cosine}
\end{equation}
and a change of variable yields
\begin{equation}
  \rho (t, f) = 2 \int_0^{\infty} G (t) G (t - \tau) \cos 2 \pi f \tau
  \hspace{0.17em} d \tau \label{eq:instantaneous_spectrum_delay}
\end{equation}
where $G (t)$ has been placed inside the integral purely for later
convenience.

\section{Examples}

\subsection{Step-Wave}

Consider the spectrum generated by a step-wave, i.e., direct current switched
on at $t = 0$. We find directly
\begin{equation}
  g_t (f) = \int_0^t e^{- 2 \pi ifx}  \hspace{0.17em} dx = \frac{e^{- 2 \pi
  ift} - 1}{- 2 \pi if}
\end{equation}
The real part gives
\begin{equation}
  \mathrm{Re} \{e^{2 \pi ift} g_t (f)\} = \mathrm{Re} \left\{ \frac{1 - e^{2
  \pi ift}}{2 \pi if} \right\} = \frac{\sin 2 \pi ft}{2 \pi f}
\end{equation}
and therefore
\begin{equation}
  \rho (t, f) = 2 \left( \frac{\sin 2 \pi ft}{2 \pi f} \right)
  \label{eq:step_spectrum}
\end{equation}
This spectrum is of the form $\sin x / x$ with its familiar oscillations. For
any $t$, it has its greatest maximum at $f = 0$, and this value $(\rho_{\max}
= 2 t)$ grows indefinitely. As time goes on, the distribution becomes more
concentrated around $f = 0$; the peaks of the instantaneous spectrum move in
toward $f = 0$ maintaining constant area under each ``hump'' of the curve. The
area
\begin{equation}
  \int_{- \infty}^{\infty} \rho \hspace{0.17em} df = 1
\end{equation}
for all $t$, and the spectrum approaches a $\delta$-function as $t \to
\infty$. Thus as $t$ increases, the spectrum is more and more precisely
described by stating that the signal is an oscillation of zero frequency.

\subsection{Stepped Sine Wave}

The stepped sine wave
\begin{equation}
  G (t) = \left\{\begin{array}{ll}
    0 & t < 0\\
    \sin 2 \pi f_0 t & t > 0
  \end{array}\right. \label{eq:stepped_sine}
\end{equation}
exhibits a similar behavior. Its spectrum is found by
\eqref{eq:instantaneous_spectrum_cosine} to be
\begin{equation}
  \rho (t, f) = \frac{- f_0 \sin 2 \pi f_0 t (\cos 2 \pi f_0 t - \cos 2 \pi
  ft)}{f^2 - f_0^2} \label{eq:stepped_sine_spectrum}
\end{equation}
which can also be expressed as
\begin{equation}
  \rho (t, f) = - f_0 \sin 2 \pi f_0 t \cdot \frac{\sin \pi (f + f_0) t}{f +
  f_0} \cdot \frac{\sin \pi (f - f_0) t}{f - f_0}
\end{equation}
As $t \to \infty$, the spectrum becomes a $\delta$-function distribution at $f
= f_0$ and $f = - f_0$. The appearance of negative frequencies is not
conceptually troublesome because the spectrum is always an even function of
$f$. Thus negative frequencies can be ignored in the interpretation.

Negative energy densities do occur, as is obvious in
\eqref{eq:stepped_sine_spectrum}. These seem physically unreal at first
glance, but must be present. For if either of these stepped signals is turned
off when $t$ is small, the total spectrum contains high frequencies. If turned
off later, the high frequency content is less, but the high frequency content
must have been present originally. This means that as time goes on, the
``virtual'' sidebands must be modified and some of the energy recalled.
Negative values of the density $\rho$ occur only in partial compensation of
earlier positive $\rho$; the total energy $\int_{- \infty}^t \rho
\hspace{0.17em} dt$ is always positive at all frequencies, by
\eqref{eq:cumulative_spectrum}.

\subsection{Isosceles Triangle}

Another interesting signal is the isosceles triangle, such as
\begin{equation}
  G (t) = \left\{\begin{array}{ll}
    0 & t < - 1\\
    1 + t & - 1 < t < 0\\
    1 - t & 0 < t < 1\\
    0 & t > 1
  \end{array}\right. \label{eq:triangle_signal}
\end{equation}
The instantaneous power spectrum is readily found to be
\begin{equation}
  \rho (t, f) = (1 + t)  \frac{1 - \cos 2 \pi f (1 + t)}{2 \pi^2 f^2} \quad
  \text{for } - 1 < t < 0 \label{eq:triangle_spectrum_rising}
\end{equation}
and
\begin{equation}
  \rho (t, f) = (1 - t)  \frac{2 \cos 2 \pi ft - 1 - \cos 2 \pi f (1 + t)}{2
  \pi^2 f^2}  \quad \text{for } 0 < t < 1 \label{eq:triangle_spectrum_falling}
\end{equation}
with $\rho (t, f) = 0$ elsewhere. During the rise of the signal, the energy
density is everywhere positive by \eqref{eq:triangle_spectrum_rising}. During
the latter half of the signal, there are regions of negative density. At all
times, the density is maximum at zero frequency, and this maximum is greatest
shortly after $t = 0$. The direct component of the signal is still increasing
when the signal peak is passed. A relief plot of $\rho$ is shown in Fig.~1.

\section{Extension to Reactive Circuits}

The definition of power spectrum in terms of pure (unit) resistance load was
made for simplicity. Consider the voltage $E (t)$ applied across a circuit
containing reactance, with the resulting current $I (t)$. The instantaneous
power delivered by the source is $EI$, so the total energy delivered is
\begin{equation}
  \int_{- \infty}^{\infty} E (t) I (t)  \hspace{0.17em} dt = \int_{-
  \infty}^{\infty} e (f) \bar{\imath} (f)  \hspace{0.17em} df = \int_{-
  \infty}^{\infty} \bar{e} (f) i (f)  \hspace{0.17em} df = \int_{-
  \infty}^{\infty} \mathrm{Re} \{e \bar{\imath} \}  \hspace{0.17em} df
  \label{eq:reactive_energy}
\end{equation}
where $e (f)$ and $i (f)$ are the Fourier transforms of $E (t)$ and $I (t)$,
with $\bar{e}$ the complex conjugate of $e$. The equation follows from
Plancherel's theorem, and the energy spectrum can be taken as any of the three
integrands on the right. The complex forms yield an imaginary component of
energy density. This component is an odd function of frequency, so its net
power is zero. Inclusion of this term does not apparently have any usefulness,
so we take the energy density in the third form: real part of $e
\bar{\imath}$.

The energy delivered by the source up until time $t$ is
\begin{equation}
  \mathcal{E} (t) = \int_{- \infty}^t E (x) I (x)  \hspace{0.17em} dx =
  \int_{- \infty}^{\infty} E_t (x) I_t (x)  \hspace{0.17em} dx
  \label{eq:reactive_cumulative_energy}
\end{equation}
where $I_t (x)$ is the current resulting from the auxiliary voltage $E_t (x)$
which vanishes for $x > t$. Note that $I_t (x)$ does not vanish identically
for $x > t$, and note also that any arbitrary definition of $I_t (x)$ for $x >
t$ would not affect \eqref{eq:reactive_cumulative_energy}. By Plancherel's
theorem, the right-hand side of \eqref{eq:reactive_cumulative_energy} is equal
to
\begin{equation}
  \mathcal{E} (t) = \int_{- \infty}^{\infty} \mathrm{Re} \{e_t (f)
  \bar{\imath}_t (f)\}  \hspace{0.17em} df \label{eq:reactive_spectrum}
\end{equation}
so that the cumulative energy spectrum can be taken as $\mathrm{Re} \{e_t (f)
\bar{\imath}_t (f)\}$. Its rate of increase is the instantaneous power
spectrum
\begin{equation}
  \rho (t, f) = \frac{\partial}{\partial t} \mathrm{Re} \{e_t (f)
  \bar{\imath}_t (f)\} \label{eq:reactive_instantaneous}
\end{equation}
The running transform of the current $i_t (f)$ can be found in terms of the
impedance of the circuit, or its reciprocal, the admittance:
\begin{equation}
  i_t (f) = \frac{e_t (f)}{z (f)} = e_t (f) A (f)
\end{equation}
This yields
\begin{equation}
  \rho (t, f) = \frac{\partial}{\partial t} \mathrm{Re} \{|e_t (f) |^2 A (f)\}
  = C (f) \frac{\partial}{\partial t} |e_t (f) |^2 \label{eq:reactive_final}
\end{equation}
using $A (f) = C (f) + iS (f)$. The result \eqref{eq:reactive_final} shows
that $\rho (t, f)$ has the original form \eqref{eq:instantaneous_spectrum}
multiplied by the conductance. This is the spectrum for the power delivered by
the source, not that dissipated by the resistance. These two are not in
continual balance because energy is stored in the reactance. The corresponding
total energy spectra for $t \to \infty$ must, however, be equal.

\section{Random Functions}

When the driving force $G (t)$ is a random function, a continuing process, the
total energy becomes infinite with time. The Fourier transform and the
conventional energy spectrum do not exist. An average energy spectrum per
second, or power spectrum, does exist and its customary definition can be
expressed in terms of our running transform:
\begin{equation}
  P (f) = \lim_{T \to \infty}  \frac{1}{T} |g_T (f) |^2
  \label{eq:random_power_spectrum}
\end{equation}
The usual procedure for finding $P (f)$ is to use the Wiener-Khintchine
theorem:
\begin{equation}
  P (f) = 2 \int_0^{\infty} \Psi (\tau) \cos 2 \pi f \tau \hspace{0.17em} d
  \tau \label{eq:wiener_khintchine}
\end{equation}
where
\begin{equation}
  \Psi (\tau) = \lim_{T \to \infty}  \frac{1}{T}  \int_0^T G (t) G (t - \tau) 
  \hspace{0.17em} dt \label{eq:autocorrelation}
\end{equation}
is, except for normalization, the auto-correlation function of the process $G
(t)$. Thus the standard form for the power spectrum of a stochastic process is
\begin{equation}
  P (f) = 2 \int_0^{\infty} \langle G (t) G (t - \tau) \rangle_{\tmop{time}}
  \cos 2 \pi f \tau \hspace{0.17em} d \tau
  \label{eq:stochastic_power_spectrum}
\end{equation}
with $\langle G (t) G (t - \tau) \rangle_{\tmop{time}}$ indicating the time
average of the product.

Now for a stationary ergodic process, time averages and stochastic averages
over the ensemble should be equivalent. Hence we expect that our instantaneous
power spectrum formula, applied to random functions, will yield a result
corresponding to \eqref{eq:stochastic_power_spectrum} after stochastic
averaging. Indeed, the stochastic average of
\eqref{eq:instantaneous_spectrum_delay} is
\begin{equation}
  \langle \rho (t, f) \rangle_{\tmop{stochastic}} = 2 \int_0^{\infty} \langle
  G (t) G (t - \tau) \rangle_{\tmop{stochastic}} \cos 2 \pi f \tau
  \hspace{0.17em} d \tau \label{eq:stochastic_instantaneous}
\end{equation}
which is the power spectrum $P (f)$ if $G$ is an ergodic process.

Returning to \eqref{eq:instantaneous_spectrum_delay}, if $G (t)$ is a function
switched on at $t = 0$, the integrand vanishes for $\tau > t$, and
\begin{equation}
  \rho (t, f) = 2 \int_0^t G (t) G (t - \tau) \cos 2 \pi f \tau
  \hspace{0.17em} d \tau \label{eq:finite_time_spectrum}
\end{equation}
Hence for a random function switched on at $t = 0$,

\begin{align}
  \langle \rho (t, f) \rangle_{\tmop{stochastic}} & = 2 \int_0^t \langle G (t)
  G (t - \tau) \rangle_{\tmop{stochastic}} \cos 2 \pi f \tau \hspace{0.17em} d
  \tau \nonumber\\
  & = 2 \int_0^t \Psi (\tau) \cos 2 \pi f \tau \hspace{0.17em} d \tau 
  \label{eq:developing_spectrum}
\end{align}

and the stochastic average of the instantaneous power spectrum asymptotically
approaches the function $P (f)$. The result \eqref{eq:developing_spectrum}
shows how the power spectrum of the process develops in time. At time $t$
after starting the process, if $t$ is sufficiently large that the process has
effectively ``forgotten'' its start, i.e., $\Psi (\tau) \approx 0$ for $\tau >
t$, the instantaneous power spectrum has effectively attained its ultimate
value. This means that a sufficient statistical sample of the process has been
taken for estimating the power spectrum.

\section{Conclusions}

It appears that an instantaneous power spectrum can be rigorously defined as a
useful concept, and that its properties not only fill an intuitive need, but
also serve to round out the mathematical structure of spectrum analysis.

\begin{thebibliography}{1}
  {\bibitem{Rice1944}}S.~O.~Rice, ``Mathematical Analysis of Random Noise,''
  \tmtextit{Bell System Technical Journal}, vol.~23, pp.~283--332, July 1944.
\end{thebibliography}

\end{document}
