\documentclass[11pt]{article}
\usepackage{amsmath}
\usepackage{amssymb}
\usepackage{amsthm}
\usepackage{enumerate}
\usepackage{enumitem}

\newtheorem{theorem}{Theorem}
\newtheorem{lemma}{Lemma}
\newtheorem{definition}{Definition}

\title{Hilbert Transforms and the Complex Representation of Real Signals}
\author{A. W. RIHACZEK\\
Aerospace Corp.\\
El Segundo, Calif.}
\date{}

\begin{document}

\maketitle

\section{Introduction}
\label{sec:introduction}

A considerable amount of work has been published on Hilbert transforms as related to the complex signal notation. A particularly useful result was derived by Bedrosian~\cite{bedrosian1963}, who found a product theorem for Hilbert transforms according to which $H\{f(t)g(t)\} = f(t)H\{g(t)\}$, provided that:
\begin{enumerate}
\item the spectral band of $g(t)$ lies entirely above that of $f(t)$, or
\item $f(t)$ and $g(t)$ are both analytic functions.
\end{enumerate}

However, in interpreting the practical implications of the theorem, Bedrosian errs and consequently is unable to reconcile his results and those of Urkowitz~\cite{urkowitz1962}, Lerner~\cite{lerner1960}, and Kelly, Reed, and Root~\cite{kelly1960}. He resolves these difficulties by stating that Lerner is in error. It is the purpose of this letter to show that with proper interpretation of Bedrosian's theorem all contradictions disappear, so that none of the results need be incorrect.

\section{Analysis of Bedrosian's Observation}
\label{sec:bedrosian_analysis}

At issue is the following observation by Bedrosian. He considers Urkowitz's result according to which $r(t) \sin(\omega_0 t + \phi)$ is the Hilbert transform of $r(t) \cos(\omega_0 t + \phi)$ only under the restriction that the highest frequency in $r(t)$ be smaller than the carrier frequency. He concludes, apparently from the results in footnote~\cite{bedrosian1963}, that no such spectral restriction applies when the signal is of the form $r(t) \cos[\omega_0 t + \phi(t)]$, on the grounds that this signal is the real part of the analytic signal $r(t)e^{i[\omega_0 t + \phi(t)]}$, whose real and imaginary parts are related by the Hilbert transform. From this he further concludes that Lerner must be incorrect in finding that the Hilbert transform relationship between $r(t) \cos[\omega_0 t + \phi(t)]$ and $r(t) \sin[\omega_0 t + \phi(t)]$ is valid only within a correction term that decreases in magnitude as the percentage bandwidth becomes smaller. However, since $\phi = \text{constant}$ is merely a special case of $\phi(t)$, it would be indeed surprising if a restriction applied for the case $\phi = \text{constant}$ and none in the more general case of an arbitrary $\phi(t)$. The answer lies in the interpretation of modulation and modulation functions.

\section{Correctness of Lerner's Result}
\label{sec:lerner_correctness}

It is easy to convince oneself that Lerner's general result of the only approximate Hilbert transform relation between the two quadrature signals must be correct. Bedrosian's first condition for the validity of the product theorem would require the spectrum of $\cos[\omega_0 t + \phi(t)]$ to lie entirely above the spectrum of $r(t)$. However, if an arbitrary phase modulation $\phi(t)$ is imposed on a carrier, then the spectrum of the modulated signal will, in general, extend from minus infinity to plus infinity, even though most of the signal energy will usually be contained within a relatively narrow band about the carrier frequency. Thus the first condition is violated. According to the second condition, both $r(t)$ and $\cos[\omega_0 t + \phi(t)]$ would have to be analytic. Again, since phase modulation of the carrier results generally in negative frequencies, the second condition is also violated. Thus we are forced to the conclusion that the two forms $r(t) \cos[\omega_0 t + \phi(t)]$ and $r(t) \sin[\omega_0 t + \phi(t)]$ cannot be strict Hilbert transforms of each other, which is Lerner's result.

\section{Reconciliation with Analytic Signal Theory}
\label{sec:reconciliation}

How can this finding be reconciled with the fact that the two quadrature signals can be considered the real and imaginary parts of the analytic signal $r(t)e^{i[\omega_0 t + \phi(t)]}$ and, hence, because of the analyticity, must be a Hilbert transform pair? From this point of view, it would indeed appear that Bedrosian is correct in claiming that Lerner's correction term should not exist. The reason for this seeming contradiction lies in the usage of the term \emph{modulation function} and the complex notation. In Lerner's derivation, and in the previous argument supporting his results, the terms amplitude and phase modulation are used to describe the type of modulation that was imposed on the carrier. Actually, the resulting modulated waveform will not have exactly the same modulation characteristics, since the spectrum foldover from the phase modulation will cause a ``distortion'' of the waveform. This distortion will be negligible when the modulation frequencies are small compared to the carrier frequency but, when this is not the case, then the modulated waveform will have little of the characteristics of the modulation waveforms. This usage of the term modulation function places no restrictions on the choice of $r(t)$ and $\phi(t)$. As a consequence, when to the real signal $r(t) \cos[\omega_0 t + \phi(t)]$ an imaginary part $jr(t) \sin[\omega_0 t + \phi(t)]$ is added to form the complex signal $r(t)e^{i[\omega_0 t + \phi(t)]}$, this signal will not be strictly analytic, and the Hilbert transform relation between real and imaginary part will apply only approximately.

All this changes when one starts from the form $r(t)e^{i[\omega_0 t + \phi(t)]}$. As soon as it is required that this signal be analytic, a restriction is imposed on the functions $r(t)$ and $\phi(t)$. Only those combinations are admissible that result in the cancellation of the negative frequencies as required to make the signal analytic. What is happening in a practical sense is that the functions $r(t)$ and $\phi(t)$ are redefined to include the effects of spectrum foldover, or signal distortion. These functions no longer describe the low-frequency signals used to modulate the carrier but, rather, the effects of modulation, including the distortions. For narrow-band signals, the differences between the two types of amplitude and phase modulation functions is small, as confirmed by Lerner's correction term. For broadband signals, however, the two sets of functions will bear little resemblance to each other. As a matter of fact, while we still may speak of an amplitude and a phase modulation function used to modulate a carrier in the broadband case, it will not be possible to define amplitude, phase, and carrier in the resulting ``modulated'' signal. The quantities of amplitude and phase have practical meaning only in the case of a narrow-band signal, for which the period and the amplitude of the RF oscillations change significantly only over many cycles of the signal. However, it is still possible, mathematically, to define quantities $r(t)$ and $\phi(t)$ such that the signal can be represented in the analytic form $r(t)e^{i[\omega_0 t + \phi(t)]}$, even though these ``modulation functions'' have no practical meaning. They are merely mathematical quantities with no relation to any physical modulation process, as the terms carrier, amplitude, and phase have lost their meaning.

\section{Amplitude Modulation Example}
\label{sec:amplitude_modulation}

The fact that an independent consideration of amplitude and phase modulation functions is possible only in the narrow-band case can be illustrated with the example of a purely amplitude modulated carrier, $r(t) \cos(\omega_0 t + \phi)$. If the restriction that the highest frequency in $r(t)$ be smaller than the carrier frequency is removed, the resulting spectrum foldover will give rise to a waveform that can no longer be described by pure amplitude modulation, even though generated by an amplitude modulation process. Two functions $r(t)$ and $\phi(t)$ are needed now if the waveform is to be represented in the form of a modulated carrier, but such a representation has no practical meaning. In summary, if amplitude and phase modulation functions are to be arbitrary in the sense that they can be specified independently of each other, $r(t) \cos[\omega_0 t + \phi(t)]$ and $r(t) \sin[\omega_0 t + \phi(t)]$ are only approximately a Hilbert transform pair, with a good approximation only in the case of narrow-band signals. When $r(t)$ and $\phi(t)$ are considered the mathematical functions obtained by simply omitting in the frequency spectrum of a real signal the negative frequencies, thus taking into account the spectrum foldover due to amplitude modulation with too high a frequency or due to phase modulation, the previous signals strictly form a Hilbert transform pair. However, the functions $r(t)$ and $\phi(t)$ then cannot be independently specified and, if the waveform is broadband, they have no practical meaning.

\section{Author's Reply}
\label{sec:author_reply}

\subsection{E. BEDROSIAN}
\label{subsec:bedrosian_reply}

Electronics Dept.\\
The RAND Corp.\\
Santa Monica, Calif.

Rihaczek's discussion of Lerner's result~\cite{lerner1960} in terms of imposed modulation and spectral distortion is very interesting, and lends credence to the belief, which I support, that $r(t) \sin[\omega_0 t + \phi(t)]$ and $r(t) \cos[\omega_0 t + \phi(t)]$ should approximate a Hilbert pair provided only that they are \emph{narrow band}. I should have been more explicit in my criticism of Lerner's result because I can see now that the context in which I refer to it creates the impression that I am disputing the existence of a correction term to such an approximation. Actually, my intention was only to point out that the analysis in Lerner's Appendix is defective. This can be seen by observing that the term-by-term integration of his equation~\eqref{eq:lerner51} is not justified. The resulting series clearly diverges, though Lerner appears to have overlooked that fact by incorrectly assigning finite values to divergent integrals in his equations~\eqref{eq:lerner53} and~\eqref{eq:lerner54}. Thus, though his result appears reasonable, it cannot be accepted as derived.

\begin{thebibliography}{9}
\bibitem{bedrosian1963}
E. Bedrosian, ``A product theorem for Hilbert transforms,'' \emph{Proc. IEEE} (Correspondence), vol. 51, pp. 868--869, May 1963.

\bibitem{urkowitz1962}
H. Urkowitz, ``Hilbert transforms of band-pass functions,'' \emph{Proc. IRE} (Correspondence), vol. 50, p. 2143, October 1962.

\bibitem{lerner1960}
R. M. Lerner, ``A matched filter detection system for complicated Doppler shifted signals,'' \emph{IEEE Trans. on Information Theory}, vol. IT-6, pp. 373--385, June 1960.

\bibitem{kelly1960}
E. J. Kelly, I. S. Reed, and W. L. Root, ``The detection of radar echoes in noise,'' \emph{J. SIAM}, vol. 8, pp. 309--341, June 1960.
\end{thebibliography}

\end{document}
