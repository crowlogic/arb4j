\documentclass{article}
\usepackage[english]{babel}
\usepackage{geometry,amsmath}
\geometry{letterpaper}

%%%%%%%%%% Start TeXmacs macros
\newtheorem{theorem}{Theorem}
%%%%%%%%%% End TeXmacs macros

%


\begin{document}

{\documentclassstar{11pt}{article}} {\usepackage{amsmath}}
{\usepackage{amssymb}} {\usepackage{amsthm}} {\usepackage{enumitem}}

{\title{Error Bounds for Asymptotic Expansions, with an Application to
Cylinder Functions of Large Argument}} {\author{Frank W. J. Olver}}

{\tableofcontents}

\section{Introduction}\label{sec:introduction}

In 1886, Poincar{\'e} {\cite{poincare1886}} introduced the notion of an
asymptotic expansion
\begin{equation}
  \label{eq:asymptotic_expansion} f (x) \sim t_0 + \frac{t_1}{x} +
  \frac{t_2}{x^2} + \cdots
\end{equation}
of an arbitrary function $f (x)$. According to his definition the coefficients
$t_s$ are independent of $x$, and
\begin{equation}
  \label{eq:remainder_definition} f (x) = \sum_{s = 0}^{m - 1} \frac{t_s}{x^s}
  + \epsilon_m (x)
\end{equation}
where, for each $m$, $\epsilon_m (x) = o (1 / x^{m - 1})$ as $x \to \infty$.
This concept admitted a new class of divergent series expansions to be useful
in analysis, enabling them to be manipulated in much the same way as
convergent power series. In turn, this has led to the development of a new
calculus, later called ``pure asymptotics'' by van der Corput
{\cite{van_der_corput}}. A feature of this development has been the
generalization of the original definition of Poincar{\'e}. Schmidt
{\cite{schmidt}} showed that the restrictive assumption in Poincar{\'e}'s
definition is not necessary. More recently, Erd{\'e}lyi {\cite{erdelyi}} has
extended the concept still further, and, with Wyman, applied it to the
asymptotic evaluation of certain integrals {\cite{erdelyi_wyman}}.
Erd{\'e}lyi's generalization is given in {\cite{erdelyi_generalization}}
below.

For some time, however, many numerical mathematicians have been aware that in
quite another way, the Poincar{\'e} definition is not restrictive enough. To
understand this point of view, consider for example the well-known asymptotic
expansion for the Hankel function $H_{\nu}^{(1)} (z)$ for large $|z|$ and
fixed $\nu$, given by
\begin{equation}
  \label{eq:hankel_expansion} H_{\nu}^{(1)} (z) \sim \left( \frac{2}{\pi z}
  \right)^{1 / 2} e^{i \phi}  \sum_{s = 0}^{\infty} \frac{i^s a_s}{z^s}
\end{equation}
in which
\begin{equation}
  \label{eq:hankel_coefficients} \phi = z - \frac{1}{2} \nu \pi - \frac{1}{4}
  \pi, \quad a_s = \frac{(\frac{1}{2} - \nu^2)  (\frac{9}{4} - \nu^2) \cdots
  ((\frac{s}{2})^2 - \nu^2)}{s! 8^s}
\end{equation}
This expansion holds in Poincar{\'e}'s sense {\cite{poincare_sense}} when $-
\pi < \arg z < 2 \pi$; in fact for arbitrary values of a positive number
$\delta$, it is uniformly valid in the sector $- \pi + \delta \leq \arg z \leq
2 \pi - \delta$ in the accepted sense that if the series is truncated after
$m$ terms, then the constant implied in the error term $O (|z|^{- m})$ can be
assigned independently of $\arg z$.

Now it will be seen later {\cite{later_reference}} that when $z$ is large and
if $\arg z$ is not too close to the constant in the uniform error bound
depends on $z$ and becomes increasingly large as $\delta \to 0$. As a
consequence, the concept of ``uniform validity'' can be quite misleading in
applications: an unsuspecting computer evaluating the series
\eqref{eq:hankel_expansion} by the usual computational procedure of truncation
at the smallest term may obtain inaccurate results in the sectors $- \pi <
\arg z < - \pi + \delta$ and $2 \pi - \delta < \arg z < 2 \pi$, grossly so in
the neighbourhoods of $\arg z = - \pi$ and $2 \pi$.

This situation, although perhaps not widely appreciated, is really a fairly
surprising, because \eqref{eq:hankel_expansion} is known to break down
completely on crossing the boundaries $\arg z = - \pi$ and $2 \pi$. It is more
natural {\cite{natural_reference}} to expect this failure to be gradual than
abrupt as the boundaries are approached. The behaviour of an ordinary
Taylor-series expansion of an analytic function is somewhat similar, inasmuch
as the circle of convergence is approached. The analogy is not complete,
however. A computer is warned of the inaccuracy of a truncated Taylor series
near the boundary of its region of validity by a diminution in the rate of
convergence. No similar warning is available for an asymptotic expansion.

Some way of excluding the direct use of an asymptotic expansion near the
boundaries of its region of validity is therefore desirable, and it is in this
sense that the Poincar{\'e} definition is insufficiently restrictive enough.
To understand this point of view, the restrictive definition should be
modified. The specific problem of Hankel and Bessel functions of large
arguments and derive some new results in connection with the expansion
\eqref{eq:hankel_expansion}.

\section{Complete asymptotic expansions}\label{sec:complete_expansions}

Practical dangers attending the use of asymptotic expansions have been
stressed previously by Miller {\cite{miller1}} {\cite{miller2}}. In the
expansions for the Weber functions given in these references, Miller
distinguishes between regions of validity in the sense of Poincar{\'e} and the
more restrictive ``complete sense of Watson.'' Essentially, the difference is
that in the former sense of contributions of an exponentially small character
are neglected (as they may be, according to the definition), whereas in the
latter sense they are retained if they have numerical significance.

For example, if $z$ is not an odd integer, then the definition, whereas in the
complete sense they are retained if they have numerical significance.

For example, if $z$ is not an odd integer then according to the definition
\eqref{eq:remainder_definition}, the quadrant $\frac{1}{2} \pi < \arg z \leq
\pi$; this is demonstrated in {\cite{watson_reference}} below. To achieve
complete validity in the quadrant $- \pi < \arg z \leq 2 \pi$ it is necessary
to add the series
\begin{equation}
  \label{eq:additional_series} (1 + e^{- 2 \nu i \pi}) \left( \frac{2}{\pi z}
  \right)^{1 / 2} e^{- i \phi}  \sum_{s = 0}^{\infty} (- i)^s \frac{a_s}{z^s}
\end{equation}
to the right of \eqref{eq:hankel_expansion}, whereas of course in the
Poincar{\'e} sense \eqref{eq:additional_series} is negligible compared with
the right of \eqref{eq:hankel_expansion} when $\frac{\pi}{2} < \arg z < 2
\pi$. Similarly, to achieve complete validity in the quadrant $- \pi < \arg z
< - \frac{\pi}{2}$, \eqref{eq:additional_series} is subtracted from the right
of \eqref{eq:hankel_expansion}.

By introducing exponentially small contributions of this type the numerical
difficulties can be overcome satisfactorily in many cases, and in other cases,
the precise is difficult to justify mathematically, however, without an
investigation of the remainder term of the given in {\cite{watson_reference}}
below. There is no readily applicable general definition of complete validity
available, nor is it easy to find what ``numerical significance'' is too vague
a criterion by itself. A drawback in practice of Watson's theory of the
uniqueness of asymptotic expansions {\cite{watson_uniqueness}} is the need to
assess properties of the remainder term which are not immediately available in
many applications. Furthermore, in cases where these properties are known it
quite possible that the statistic bound for the remainder term is also known,
thereby obviating the need for the theory for numerical purposes. This is
certainly true of the example given in Watson's paper {\cite{watson_paper}}.

The difficulty of recognizing when to include the numerical contribution of an
exponentially small term in an asymptotic expansion is illustrated by the
following example, which arose some years ago in computations at the National
Physical Laboratory, Teddington.

Let
\begin{equation}
  \label{eq:integral_example} I (n) = \int_0^{\pi} \frac{\cos nt}{t^2 + 1} dt
\end{equation}
By repeated integration by parts, one readily shows that for large positive
integer values of $n$, $I (n)$ has the Poincar{\'e} expansion
\begin{equation}
  \label{eq:poincare_expansion} I (n) \sim (- 1)^{n - 1}  \left(
  \frac{\lambda_1}{n} - \frac{\lambda_2}{n^2} + \frac{\lambda_3}{n^3} - \cdots
  \right)
\end{equation}
in which the coefficients $\lambda_s$ are given by
\begin{equation}
  \label{eq:lambda_coefficients} \lambda_s = (n^2 + 1)^{- 2 s} p_{2 s - 1} (n)
\end{equation}
the $p_s (t)$ being polynomials in $t$ of degree $s$, defined recursively by
$p_0 (t) = 1$, and
\begin{equation}
  \label{eq:recursive_polynomials} p_s (t) = 2 at \cdot p_{s - 1} (t) - (t^2 +
  1) p'_{s - 1} (t) \quad (s = 1, 2, \ldots) .
\end{equation}
Explicit expressions for the first six polynomials are

\begin{align}
  p_0 (t) & = 1 \quad p_1 (t) = 2 t \quad p_2 (t) = 2 (3 t^2 - 1) \quad p_3
  (t) = 24 (t^3 - t)  \label{eq:explicit_polynomials}\\
  p_4 (t) & = 24 (5 t^4 - 10 t^2 + 1) \quad p_5 (t) = 240 (3 t^5 - 10 t^3 + 3
  t) 
\end{align}

and on numerical evaluation, one obtains to five decimals
\begin{equation}
  \label{eq:numerical_values} \lambda_1 = 0.05318 \quad \lambda_2 = 0.04791
  \quad \lambda_3 = 0.08985
\end{equation}
Thus for $n = 10$, the series \eqref{eq:poincare_expansion} gives
\begin{equation}
  \label{eq:series_result} I (10) \approx - (0.0005318 - 0.0000048 + 0.0000001
  - \cdots) = - 0.0005271.
\end{equation}
This answer is quite incorrect however, because direct numerical quadrature of
the expression \eqref{eq:integral_example} yields, to seven decimals,
\begin{equation}
  \label{eq:quadrature_result} I (10) = - 0.0004558
\end{equation}
The inclusion of additional terms in the expansion would not help matters, and
a partial explanation of the discrepancy is as follows. One may write
\begin{equation}
  \label{eq:integral_split} I (n) = \int_0^{\infty} \frac{\cos nt}{t^2 + 1} dt
  - \int_{\pi}^{\infty} \frac{\cos nt}{t^2 + 1} dt
\end{equation}
The first of these integrals equals $\frac{\pi}{2} e^{- n}$; the second may
again be expanded by repeated partial integration. In this way, one finds that
\begin{equation}
  \label{eq:alternative_expansion} I (n) \sim \frac{1}{2} \pi e^{- n} + (-
  1)^{n - 1}  \left( \frac{\lambda_1}{n^2} - \frac{\lambda_2}{n^4} +
  \frac{\lambda_3}{n^6} - \cdots \right)
\end{equation}
where the $\lambda_s$ are the same as in \eqref{eq:lambda_coefficients}. From
this result one obtains the correct numerical value at $n = 10$, because
\begin{equation}
  \label{eq:correct_value} \frac{1}{2} \pi e^{- 10} = 0.0000713
\end{equation}
which is exactly the discrepancy between the values \eqref{eq:series_result}
and \eqref{eq:quadrature_result}.

An alternative way of deriving \eqref{eq:alternative_expansion} is to apply
the Residue theorem and Watson's lemma {\cite{watson_lemma}}, page 236, to the
contour integral
\begin{equation}
  \label{eq:contour_integral} \int_C \frac{e^{int}}{t^2 + 1} dt
\end{equation}
where $C$ is the rectangle having vertices $\pm \pi$ and $\pm \pi + i$.

In the sense of Miller and Watson, \eqref{eq:alternative_expansion} is
apparently a complete asymptotic expansion for positive integer $n$, whereas
\eqref{eq:poincare_expansion} is incomplete. There is, however, no
self-evident conclusive mathematical reason why this should be so, in fact
\eqref{eq:alternative_expansion} was obtained by less obvious procedures. It
is even possible that \eqref{eq:alternative_expansion} is itself incomplete,
for without further investigation one does not know whether or not there have
been neglected other exponential terms, for example $e^{- 2 n}$, which make
numerically significant contributions for smaller values of $n$.

\section{The need for error bounds}\label{sec:need_error_bounds}

The difficulty illustrated by the examples of the previous sections is linked
to a fundamental weakness of the Poincar{\'e} definition: it provides no
direct answer to the question ``What is the precise relation between an
asymptotic expansion and the function from which it is derived?'' Thus
strictly speaking there is no connection between pure asymptotics and applied
mathematics, except in the limit $|z| = \infty$. To establish a connection,
two courses are open. Either one can seek upper bounds for the differences
between the partial sums of an asymptotic expansion and the function from
which it was obtained, or one can endeavour to transform the expansion into a
convergent form, as, for example, in {\cite{reference17}} and
{\cite{reference19}}. In the present paper the authors confine themselves to
the former possibility.

Although the theory of pure asymptotics has been extensively developed and
applied, the corresponding theory of error bounds has been comparatively
neglected. The literature on this aspect consists mainly of scattered results
applicable to special functions.

The few theorems of a general nature which have been discovered
{\cite{reference3}}, {\cite{reference5}}, are concerned with asymptotic
expansions of integral representations with real variables. (The example of
\eqref{eq:integral_example}, incidentally, is not covered by these theorems.)
A possible reason for this neglect is the belief {\cite{reference1}},
{\cite{reference17}}, that when error bounds are needed they can be obtained
merely by retracing the steps of the asymptotic proof. This is frequently a
difficult and tedious undertaking, and the bounds it yields are often quite
unrealistic. There is no readily applicable general theorem, and the writer's
experience with expansions arising from differential equations indicates that
it may often be necessary to develop entirely new proofs of the theorems of
pure asymptotics before attempting to follow through explicit treatment of the
error terms.

In this connection, attention may be drawn to the suggestion of Wyman
{\cite{wyman}} that the main direction in which the modern theory of
asymptotics will move is towards the use of more general concepts in the
theory of pure asymptotics. The importance of investigating such
generalizations is indisputable, but perhaps there is a need to stress that
the bridging of a gap between pure and applied mathematics in this branch of
analysis by the development of satisfactory theories of error bounds is also
of importance. Moreover, such theories may sometimes provide an alternative
way of overcoming one of the difficulties which has helped stimulate the
recent further generalizations {\cite{reference6}} in the definition of an
asymptotic expansion: the need to avoid narrow concepts concerning both the
choice of asymptotic variable for a given expansion, and the nature of the
uniformity of the expansion with respect to other variables.

This observation may be illustrated briefly by the following example. In
{\cite{reference17}}, \eqref{eq:reference_6_31}, Erd{\'e}lyi and Wyman have
published a generalized series expansion in terms of Airy functions for the
Hankel function $H_{\nu}^{(1)} (x)$ when $\nu$ and $x$ are real and positive,
having a ``scale''
\begin{equation}
  \label{eq:scale_function} \tau^{- 1}  (2 m / 3)  \quad \text{as} \quad \tau
  \to \infty, \quad \text{where}
\end{equation}
\begin{equation}
  \label{eq:tau_definition} \tau = | \frac{1}{4} (\nu^2 - x^2) |^{1 / 2} +
  (\frac{1}{2} \nu)^{3 / 2} .
\end{equation}
This means that for each fixed integer $m$, the $(m + 1)$th partial sum of the
series differs from $H_{\nu}^{(1)} (x)$ by $o (\tau^{- 1 - (2 m / 3)})$ as
$\tau \to \infty$. Thus these authors have succeeded in describing the
behaviour of $H_{\nu}^{(1)} (x)$ when either $x$ or $\nu$ is large by means of
a single asymptotic expansion. Other investigators, using Poincar{\'e}
expansions, have had to distinguish between the two cases. Recently however,
the present writer {\cite{present_writer}} has derived sharp error bounds for
the most

powerful of the existing Poincar{\'e}-type expansions for $H_{\nu}^{(1)} (x)$
for large $x$, namely the uniform expansion in terms of Airy functions. From
these bounds it can be seen that although the expansion was derived on the
assumption that $\nu$ is large, it also has an asymptotic property for large
$x$. Indeed, without going into detailed proof, it can be stated that the
uniform expansion in terms of Airy functions for large $x$ is also a
generalized expansion, in the sense of Erd{\'e}lyi, this time with respect to
the scale $| (\nu t)^{- m (x - \nu)} |$ as $x \to \infty$, for any
non-negative number $\delta$. This is, in fact, a considerably more powerful
scale than that of the new expansion.

(Notwithstanding the generally and greater power of the Airy function
expansion, the Poincar{\'e}-type expansion of $H_{\nu}^{(1)} (x)$ for fixed
$\nu$ and large $x$ remains important, owing to its simplicity, and a further
study of it is made in \eqref{eq:sections_5_7}.)

\section{Nature of the error bounds}\label{sec:nature_error_bounds}

In seeking bounds for the error term of the partial expansion of the form
\eqref{eq:remainder_definition}, what kind of success can be hoped for? If
$t_m$ is non-zero, then on replacing $x$ by the complex variable $z$, the
bound $| \epsilon_m (z) | \sim |t_m z^{- m} |$. Hence the most that can be
established, in general, is that $| \epsilon_m (z) |$ is bounded by the
modulus of the first (non-vanishing) neglected term of the series. This bound
cannot apply when $| \arg t_{m + 1} / (zt_m) | < \frac{1}{2} \pi$, however,
for the modulus of the right side of the equation
\begin{equation}
  \label{eq:error_bound_equation} \epsilon_m (z) = t_m z^{- m} + t_{m + 1}
  z^{- m - 1} + o (|z|^{- m - 1})
\end{equation}
would exceed that of its first term for all sufficiently large $|z|$. In
particular, this happens when $t_m$ and $t_{m + 1}$ are real and of the same
sign, and $z$ is real and positive.

A modest error bound which would always be feasible is a multiple $(> 1)$ of
the modulus of the first non-vanishing neglected term. This multiple itself
could depend on $z$, and then ideally it would tend to unity as $|z| \to
\infty$. The last condition is not essential from the standpoint of most
applications however, and a bound of this kind if likely to be quite
satisfactory with any value of $|z|$ not too large. Perhaps this can be
appreciated best by observing that for a specified precision in $(z)$, the
difference between having $|p_m t_m z^{- m} |$ and $|t_m z^{- m} |$ as bounds
for $\epsilon_m (z)$ only affects the minimum allowable value of $|z|$ by the
factor $(p_m)^{1 / m}$. Certainly, for example, if $p_m < 10$ there are few
situations in which such a reduction in the region of applicability is likely
to be of importance. From this point of view, the expenditure of heavy
analytical effort to achieve a slight reduction in the value of $p_m$ is
unjustified, except possibly in the case of the dominant term $(m = 1)$ of the
expansion.

\section{Hankel functions of large argument}\label{sec:hankel_functions}

The following theorem is obtained from Theorem 7 of {\cite{reference12}} by
taking the parameter $u$ occurring there to be unity, and making minor
changes:

\begin{theorem}
  \label{thm:hankel_functions}Let $f (z)$ be regular in a simply-connected
  complex domain $D$, and a sequence of functions $A_s (z)$ be defined by $A_0
  (z) = 1$ and
  \begin{equation}
    \label{eq:hankel_recurrence} A_{s + 1} (z) = - \frac{1}{2} A_s' (z) +
    \frac{1}{4}  \int f (z) A_s (z) dz \quad (s = 0, 1, \ldots)
  \end{equation}
\end{theorem}

Then the differential equation
\begin{equation}
  \label{eq:hankel_differential} \frac{d^2 w}{dz^2} = \{1 + f (z)\} w
\end{equation}
has a solution $w_m (z)$, depending on an arbitrary point $a$ of $D$ and an
arbitrary positive integer $m$, such that
\begin{equation}
  \label{eq:hankel_solution} w_m (z) = e^z \left[ \sum_{s = 0}^{m - 1}
  \frac{A_s (z) + \epsilon_m (z)}{z^s} \right]
\end{equation}
and
\begin{equation}
  \label{eq:hankel_solution_prime} w_m' (z) = e^z  \left[ \sum_{s = 0}^{m - 1}
  \{A_s (z) + A_s' (z)\}+ \eta_m (z) \right]
\end{equation}
where

A form of error bound which has emerged from recent investigations by the
writer {\cite{reference12}}, {\cite{reference13}} of the asymptotic solutions
of certain second-order differential equations with respect to a parameter
effectively consists of the variation, that is, the integral of the modulus of
the derivative, of the first neglected term of the series taken over a
suitable contour. More recent work, in preparation for publication, has shown
that bounds of this type are also applicable, in certain cases, to asymptotic
expansions of solutions for differential equations without a parameter $z \to
\infty$ and $(1.01)$, and from the observations made in
\eqref{eq:sections_1_1} might expect that as $z$ approaches the boundaries of
the regions of validity, the contour of integration would naturally be subject
to certain restrictions, and from the observations made in
\eqref{eq:sections_1_1} we might expect that the bound it decreases in $\Re
z$.

In the remaining part of this paper, it is shown that this variational form of
error bound is applicable to the standard cylinder functions of large
argument, and that it does indeed have the feature just described.

\end{document}
