\documentclass{article}
\usepackage[english]{babel}
\usepackage{amsmath,latexsym}

%%%%%%%%%% Start TeXmacs macros
\newcommand{\tmtextit}[1]{\text{{\itshape{#1}}}}
\newenvironment{proof}{\noindent\textbf{Proof\ }}{\hspace*{\fill}$\Box$\medskip}
\newtheorem{theorem}{Theorem}
%%%%%%%%%% End TeXmacs macros

%


\begin{document}

\title{Comments on ``A Note on the Predictability of Band-Limited Processes''}

\author{Farokh Marvasti}

\date{}

\maketitle

\begin{abstract}
  A simpler proof is provided for the problem presented by Papoulis regarding
  past uniform sampling of random signals, extending the result to
  deterministic signals and nonuniform sampling.
\end{abstract}

\section{Introduction}

The problem of past uniform sampling of random signals, as presented by
Papoulis{\cite{papoulis1985}}, is shown to have been previously solved by
others using different methods. This note provides a simpler proof that covers
both random and deterministic signals with uniform or nonuniform sampling.

\section{Proof of Uniqueness for Band-Limited Signals}

\begin{theorem}
  \label{thm:uniqueness}If the past samples have a rate slightly greater than
  the Nyquist rate, then no band-limited signal (having a bandwidth equal to
  or smaller than half the Nyquist rate) can be found to have zero crossings
  at the past instances.
\end{theorem}

\begin{proof}
  The average zero crossings per interval of a band-limited signal is equal to
  the Nyquist rate. Therefore, the past samples represent uniquely the
  band-limited signal {\cite{requicha1980}}. This argument is valid for
  deterministic and random signals with uniform or nonuniform samples.
\end{proof}

\section{Discussion}

The proof presented here is implied in {\cite{requicha1980}}. Previous
solutions for this problem include:
\begin{itemize}
  \item Wainstein and Zubakov {\cite{wainstein1962}} at three times the
  Nyquist rate,
  
  \item Brown {\cite{brown1972}} at two times the Nyquist rate,
  
  \item Beutler {\cite{beutler1966}} for deterministic and random signals with
  general nonuniform samples,
  
  \item Fjallbrandt {\cite{fjallbrandt1975}} for past uniform samples.
\end{itemize}

\section{Conclusion}

The proof by Papoulis, while different, can be simplified and generalized to
cover both deterministic and random signals with uniform or nonuniform
sampling, as shown above.

\section*{Acknowledgments}

The author was with the University of California at Davis, on leave from
Sharif University of Technology, Tehran, Iran, and is now at AT\&T Bell
Laboratories, Naperville, IL.

\section*{References}

\begin{thebibliography}{9}
  {\bibitem{papoulis1985}}A. Papoulis, ``A note on the predictability of
  band-limited processes,'' \tmtextit{Proc. IEEE}, vol. 73, no. 8, pp.
  1332--1333, Aug. 1985.
  
  {\bibitem{wainstein1962}}L. A. Wainstein and V. D. Zubakov,
  \tmtextit{Extraction of Signals from Noise}. Englewood Cliffs, NJ:
  Prentice-Hall, 1962.
  
  {\bibitem{brown1972}}J. L. Brown, Jr., ``Uniform linear prediction of
  bandlimited processes from samples,'' \tmtextit{IEEE Trans. Inform. Theory},
  vol. IT-18, pp. 662--664, Sept. 1972.
  
  {\bibitem{beutler1966}}F. Beutler, ``Error free recovery of signals from
  irregular samples,'' \tmtextit{SIAM Rev.}, vol. 8, pp. 322--335, July 1966.
  
  {\bibitem{fjallbrandt1975}}T. T. Fjallbrandt, ``Interpolation and
  extrapolation in non-uniform sampling sequences with average sampling rate
  below the Nyquist rate,'' \tmtextit{Electron. Lett.}, vol. 11, no. 12, pp.
  264--266, 12th June 1975.
  
  {\bibitem{requicha1980}}A. Requicha, ``The zeros of entire functions: Theory
  and engineering applications,'' \tmtextit{Proc. IEEE}, vol. 68, no. 3, pp.
  308--328, Mar. 1980.
  
  {\bibitem{knab1981}}J. J. Knab, ``Noncentral interpolation of band-limited
  signals,'' \tmtextit{IEEE Trans. Aerosp. Electron. Syst.}, vol. AES-17, no.
  4, pp. 586--590, July 1981.
\end{thebibliography}

\end{document}
