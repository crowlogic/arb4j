\documentclass{article}
\usepackage[english]{babel}
\usepackage{amsmath,latexsym}

%%%%%%%%%% Start TeXmacs macros
\newcommand{\tmem}[1]{{\em #1\/}}
\newcommand{\tmtextbf}[1]{\text{{\bfseries{#1}}}}
\newenvironment{proof}{\noindent\textbf{Proof\ }}{\hspace*{\fill}$\Box$\medskip}
%%%%%%%%%% End TeXmacs macros

%


\begin{document}

\title{Comments on Signal Processing and Prediction}

\author{
  Farokh Marvasti
  \and
  J. L. Brown
  \and
  Jr.
}

\date{}

\maketitle

\section{Comments on ``A Note on the Predictability of Band-Limited
Processes''}\label{sec:comments}

\subsection*{Author Information}

\tmtextbf{FAROKH MARVASTI}

We show that the problem presented in the above letter~{\cite{papoulis1985}}
by Papoulis has been proved by
others~{\cite{beutler1966,fjallbrandt1975,requicha1980,knab1981}} using
different methods. We present a much simpler proof which covers random and
deterministic signals with uniform or nonuniform sampling.

I just came across a recent letter by Papoulis~{\cite{papoulis1985}} on past
uniform sampling of random signals. This problem was solved
in~{\cite{wainstein1962}} at three times the Nyquist rate and
in~{\cite{brown1972}} at two times the Nyquist rate.
Beutler~{\cite{beutler1966}} solved this problem both for deterministic and
random signals for general nonuniform samples at any rate greater than that of
Nyquist. Fjallbrandt~{\cite{fjallbrandt1975}} has
verified~{\cite{beutler1966}} for past uniform samples. Although the proof by
Papoulis~{\cite{papoulis1985}} is different from the others, there is a much
simpler way to prove past sampling for uniform or nonuniform samples for
deterministic or random signals. The proof is implied
in~{\cite{requicha1980}}.

\begin{proof}
  \label{proof:main}If the past samples have a rate slightly greater than the
  Nyquist, then no band-limited signal (having a bandwidth equal to or smaller
  than half the Nyquist rate) can be found to have zero crossings at the past
  instances. This is because the average zero crossings per interval of a
  band-limited signal is equal to the Nyquist rate. Therefore, the past
  samples represent uniquely the band-limited signal~{\cite{requicha1980}}.
  This argument is valid for deterministic and random signals with uniform (or
  random) samples.
\end{proof}

\subsection*{Manuscript Information}

Manuscript received January 21, 1986.

The author was with the University of California at Davis, Davis, CA, on leave
from Sharif University of Technology, Tehran, Iran. He is now at AT\&T Bell
Laboratories, Naparville, IL 60566, USA.

\begin{thebibliography}{9}
  {\bibitem{wainstein1962}}L. A. Wainstein and V. D. Zubakov,
  {\tmem{Extraction of Signals from Noise}}. Englewood Cliffs, NJ:
  Prentice-Hall, 1962.
  
  {\bibitem{brown1972}}J. L. Brown, Jr., ``Uniform linear prediction of
  bandlimited processes from samples,'' {\tmem{IEEE Trans. Inform. Theory}},
  vol. IT-18, pp. 662-664, Sept. 1972.
  
  {\bibitem{beutler1966}}F. Beutler, ``Error free recovery of signals from
  irregular samples,'' {\tmem{SIAM Rev.}}, vol. 8, pp. 322-335, July 1966.
  
  {\bibitem{fjallbrandt1975}}T. T. Fjallbrandt, ``Interpolation and
  extrapolation in non-uniform sampling sequences with average sampling rate
  below the Nyquist rate,'' {\tmem{Electron. Lett.}}, vol. 11, no. 12, pp.
  264-266, 12th June 1975.
  
  {\bibitem{requicha1980}}A. Requicha, ``The zeros of entire functions: Theory
  and engineering applications,'' {\tmem{Proc. IEEE}}, vol. 68, no. 3, pp.
  308-328, Mar. 1980.
  
  {\bibitem{knab1981}}J. J. Knab, ``Noncentral interpolation of band-limited
  signals,'' {\tmem{IEEE Trans. Aerosp. Electron. Syst.}}, vol. AES-17, no. 4,
  pp. 586-590, July 1981.
  
  {\bibitem{papoulis1985}}A. Papoulis, ``A Note on the Predictability of
  Band-Limited Processes,'' {\tmem{Proc. IEEE}}, vol. 73, no. 8, pp.
  1332-1333, Aug. 1985.
\end{thebibliography}

\end{document}
