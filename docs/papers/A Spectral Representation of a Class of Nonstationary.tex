\documentclass{article}
\usepackage{amsmath,amsthm,amssymb}
\usepackage{enumitem}
\usepackage[margin=0.5in]{geometry}
\usepackage{hyperref}

% Theorem styles
\newtheorem{theorem}{Theorem}[section]
\newtheorem{definition}[theorem]{Definition}
\newtheorem{proposition}[theorem]{Proposition}

% Equation numbering within sections
\numberwithin{equation}{section}

% For the proof environment
\providecommand\qed{\hfill \ensuremath{\square}}

%% Document metadata from PDF
%% This article was downloaded by: [Mount Royal University]
%% On: 11 May 2013, At: 07:37
%% Publisher: Taylor & Francis
%% Journal: Journal of Statistical Theory and Practice
%% Copyright: Grace Scientific Publishing, LLC

\begin{document}

\title{A Spectral Representation of a Class of Nonstationary Processes}
\author{Randall J. Swift\\
Department of Mathematics and Statistics\\
California State Polytechnic University\\
Pomona, CA 91768, USA\\
\texttt{rjswift@csupomona.edu}}
\date{Dedicated to Professor M.M. Rao, advisor and friend, on the occasion of his 80th birthday.}
\maketitle

\begin{abstract}
Received: November 28, 2009; Revised: January 10, 2010

In this paper, an extension of the weakly harmonizable class of processes is considered. This class, termed almost periodic contractive harmonizable, is based upon the natural contractive operator associated with harmonizable processes. A spectral representation of these processes is obtained. A relation between the almost periodic contractive harmonizable and the oscillatory harmonizable classes is considered. The paper concludes with a series representation for the almost periodic contractive harmonizable class.

\textit{AMS Subject Classification:} 60G12; 60G35.

\textit{Key-words:} Harmonizable processes; Oscillatory processes; Semi-groups of contractive operators; Almost periodic functions.
\end{abstract}

\section{Introduction}\label{sec:intro}

To introduce the desired class of stochastic processes, recall that a stationary process $X:\mathbb{R}\to L_{0}^{2}(P)$ can be expressed as:
\begin{equation}\label{eq:stationary_rep}
X(t)=\int_{\mathbb{R}}e^{i\lambda t}\,dZ(\lambda),
\end{equation}
where $Z(\cdot)$ is a $\sigma$-additive stochastic measure on the Borel $\sigma$-algebra $\mathcal{B}$ of the real line $\mathbb{R}$, with orthogonal values in the complex Hilbert space $L_{0}^{2}(P)$ of centered random variables. The covariance $r(\cdot,\cdot)$ of the process is
\begin{equation}\label{eq:stationary_cov}
r(s,t)=\int_{\mathbb{R}}e^{i(s-t)\lambda}\,dF(\lambda),
\end{equation}
where $E(Z(A)Z(B))=F(A\cap B)$, $F$ a positive finite Borel measure on $\mathbb{R}$. Here $E(\cdot)$ denotes the expectation. One notes that the covariance is a function of the difference, $s-t$.

A natural generalization of the concept of stationarity is given by processes $X:\mathbb{R}\to L_{0}^{2}(P)$ with covariance $r(\cdot,\cdot)$ expressible as
\begin{equation}\label{eq:harmonizable_cov}
r(s,t)=\int_{\mathbb{R}}\int_{\mathbb{R}}e^{i\lambda s-i\lambda^{\prime}t}\,dF(\lambda,\lambda^{\prime}),
\end{equation}
where $F(\cdot,\cdot)$ is a complex bimeasure, called the spectral bimeasure of the process, of bounded variation in the Vitali's sense or more inclusively in Fréchet's sense; in which case the integrals are strict Morse-Transue (cf.~\cite{Rao1984}). The covariance as well as the process are termed strongly or weakly harmonizable respectively. The covariance of a harmonizable process is not a function of the difference, $s-t$.

Every weakly or strongly harmonizable process $X:\mathbb{R}\to L^{2}(P)$ has an integral representation given by \eqref{eq:stationary_rep}, where $Z:\mathcal{B}\to L^{2}(P)$ is a stochastic measure (not necessarily with orthogonal values) and is called the spectral measure of the process. Both of these concepts reduce to the stationary case if $F$ concentrates on the diagonal $\lambda=\lambda^{\prime}$ of $\mathbb{R}\times\mathbb{R}$.

The theory of harmonizable processes is being developed by Rao and others. Central to this theory is the 1982 and 1986 papers of Rao~\cite{Rao1984}, and Chang and Rao~\cite{ChangRao1986}. In these papers the required bimeasure theory is developed and its relationship with the harmonizable and other nonstationary classes of processes is considered. The article by Swift~\cite{Swift1997b} details much of the advances on harmonizable processes.

\section{Spectral Representation}\label{sec:spectral}

In this work, a class of nonstationary processes related to the harmonizable class is introduced. Specifically, if one recalls from the classical theory of stationary processes (cf.~\cite{Rozanov1959}) every such process $X(\cdot)$ can be represented as
\begin{equation}\label{eq:unitary_rep}
X(t)=U(t)X(0),
\end{equation}
where $\{U(t),t\in\mathbb{R}\}$ is a group of unitary operators acting on $L_{0}^{2}(P)$. Rao~\cite{Rao1984} showed that the corresponding operator representation for harmonizable processes is not as simple. The representation for a weakly harmonizable process $X(\cdot)$ is
\begin{equation}\label{eq:harmonizable_op_rep}
X(t)=T(t)X(0),
\end{equation}
where $\{T(t),t\in\mathbb{R}\}$ is a weakly continuous family of positive definite contractions. This provides the motivation for the following extension of the weakly harmonizable class of processes.

\begin{definition}\label{def:apc_harmonizable}
An $L^{2}$-continuous process $X:\mathbb{R}\to L_{0}^{2}(P)$ is \emph{almost periodic contractive weakly harmonizable} if
\begin{equation}\label{eq:def_rep}
X(t)=T(t)Q(t)
\end{equation}
where $\{T(t),t\in\mathbb{R}\}$ is a weakly continuous semi-group of positive definite contractive operators on $H(\mathbb{R})$, with
\begin{equation}\label{eq:H_R_def}
H(A)=\operatorname{cl}\{\sum_{j}\alpha_{j}Z(A_{j},t_{j}),t_{j}\in\mathbb{R},A_{j}\in\mathcal{B}(A)\}
\end{equation}
and $Q(\cdot)$ a uniformly almost periodic function taking values in $L^{2}(P)$.
\end{definition}

From this Definition, it follows that every weakly harmonizable process is an almost periodic contractive weakly harmonizable process. Since if $Q(t)=X(0)$, a (trivially) uniformly almost periodic function taking values in $L^{2}(P)$, then
\begin{equation}\label{eq:harmonizable_special}
X(t)=T(t)Q(t)=T(t)X(0).
\end{equation}

Using the Definition, a spectral representation of an almost periodic contractive weakly harmonizable process can be obtained.

\begin{theorem}\label{thm:spectral_rep}
An $L^{2}$-continuous process $X:\mathbb{R}\to L_{0}^{2}(P)$ is almost periodic contractive weakly harmonizable if and only if $X(\cdot)$ has spectral representation
\begin{equation}\label{eq:thm_rep}
X(t)=\int_{\mathbb{R}}e^{i\lambda t}Z(d\lambda,t)
\end{equation}
where for each $\lambda$, the time dependent vector measure $Z(\cdot,\cdot)$ is uniformly almost periodic in $t$ and $Z(\cdot,t)$ is a stochastic measure satisfying
\begin{equation}\label{eq:Z_cov}
E(Z(A,s)\overline{Z(B,t)})=F(A,B,s,t)\quad\text{for }(A,B)\in\mathcal{B}\times\mathcal{B}\text{ and }(s,t)\in\mathbb{R}\times\mathbb{R},
\end{equation}
where $F(\cdot,\cdot,s,t)$ is a function of bounded Fréchet variation for each $s,t$ and for each $\lambda,\lambda^{\prime}$, $F(\lambda,\lambda^{\prime},\cdots)$ is a uniformly almost periodic function of two variables.
\end{theorem}

\begin{proof}
Let
\begin{equation}\label{eq:Q_def}
Q(t)=Z(\mathbb{R},t)
\end{equation}
so that $Q(\cdot)$ is a uniformly almost periodic function. Define
\begin{equation}\label{eq:H_A_def2}
H(A)=\operatorname{cl}\{\sum_{j}\alpha_{j}Z(A_{j},t_{j}),t_{j}\in\mathbb{R},A_{j}\in\mathcal{B}(A)\},
\end{equation}
and let $\tilde{E}_{\lambda}$ be the projection of $X$ onto $H((-\infty,\lambda])$ so that
\begin{equation}\label{eq:Z_projection}
Z((a,b],t)=\tilde{E}_{(a,b]}Q(t)\quad\text{for }(a,b]\in\mathcal{B}.
\end{equation}
Thus
\begin{align}
X(t) &= \int_{\mathbb{R}}e^{i\lambda t}Z(d\lambda,t)\\
&= \int_{\mathbb{R}}e^{i\lambda t}\,d\tilde{E}_{\lambda}(Q(t))\\
&= U(t)Q(t).\label{eq:U_rep}
\end{align}
The last equality follows from Stone's Theorem (cf.~\cite{RieszNagy}) for the representation of unitary operators. Here, $U(t)$ is the group of unitary operators on $H(\mathbb{R})$ generated by the family of projections $\{\tilde{E}_{\lambda},\lambda\in\mathbb{R}\}$, but unitary operators $U(t)$ are contractive.

For the other direction, if $X(\cdot)$ is an almost periodic contractive weakly harmonizable process then $X(\cdot)$ has representation
\begin{equation}\label{eq:X_TQ}
X(t)=T(t)Q(t).
\end{equation}
Define a measure $Z(\cdot,t)$ by the action of the projections $\tilde{E}_{\lambda}$ on $Q(t)$:
\begin{equation}\label{eq:Z_def}
Z(A,t)=\tilde{E}_{A}Q(t)\quad\text{for }A\in\mathcal{B}.
\end{equation}
Now, by the definition of $H(A)$, $Z(A,t)$ satisfies
\begin{equation}\label{eq:Z_cov2}
E(Z(A,s)\overline{Z(B,t)})=F(A,B,s,t)\quad\text{for }(A,B)\in\mathcal{B}\times\mathcal{B}\text{ and }(s,t)\in\mathbb{R}\times\mathbb{R},
\end{equation}
with $F(\cdot,\cdot,s,t)$ a function of bounded Fréchet variation. Further, observe
\begin{align}
\|Z(A,t+\tau)-Z(A,t)\| &= \|\tilde{E}_{A}Q(t+\tau)-\tilde{E}_{A}Q(t)\|\\
&\leq \|Q(t+\tau)-Q(t)\|\label{eq:Z_continuity}
\end{align}
so that since $Q(t)$ is uniformly almost periodic, it follows that $Z(A,\cdot)$ is also uniformly almost periodic.

Now by Theorem 6.3 of Chang and Rao~\cite{ChangRao1986}, there is a group of unitary operators $\{U(t),t\in\mathbb{R}\}$ on $H(X)=\overline{sp}\{X(t),t\in\mathbb{R}\}$ so that $\tilde{T}(t)=\Pi\circ U(t)$ where $\Pi$ is the orthogonal projection of $H(X)$ on $L_{0}^{2}(P)$ and $\tilde{T}(\cdot)=T(\cdot)\big|_{L_{0}^{2}(P)}$. Thus
\begin{equation}\label{eq:X_tilde}
X(t)=T(t)Q(t)=\Pi\circ U(t)Q(t)
\end{equation}
so that applying Stone's Theorem, one obtains
\begin{align}
X(t) &= \Pi\circ\int_{\mathbb{R}}e^{i\lambda t}\,dE_{\lambda}(Q(t))\\
&= \int_{\mathbb{R}}e^{i\lambda t}\Pi\circ Z(d\lambda,t)\label{eq:X_final}
\end{align}
where $\tilde{Z}(\cdot,t)=\Pi\circ Z(\cdot,t)$ is a u.a.p. function and $Z(\cdot,t)$ a measure satisfying \eqref{eq:Z_cov}.

This Theorem provides an extension to the representation of the class of processes obtained by Hurd~\cite{Hurd1992}. Hurd showed that the almost periodic unitary class of processes have spectral representation
\begin{equation}\label{eq:hurd_rep}
X(t)=\int_{\mathbb{R}}e^{i\lambda t}Z(d\lambda,t),
\end{equation}
where for each $\lambda$, the time dependent vector measure $Z(\cdot,t)$ is almost periodic in the variable $t$ and $Z(\cdot,t)$ is orthogonally scattered in that
\begin{equation}\label{eq:hurd_orth}
E(Z(A,s)\overline{Z(B,t)})=0
\end{equation}
for $(A,B)\in\mathcal{B}\times\mathcal{B}$, $A\cap B=\emptyset$ and $(s,t)\in\mathbb{R}\times\mathbb{R}$. This observation is summarized in

\begin{proposition}\label{prop:unitary_subset}
The class of almost periodic unitary processes is contained in the class of almost periodic contractive weakly harmonizable processes.
\end{proposition}
\end{proof}

\section{Amplitude Modulation and the Oscillatory Class}\label{sec:oscillatory}

The class of almost periodic contractive weakly harmonizable processes arise naturally from applications of modulated processes. If $X(t)$ is a continuous weakly harmonizable process and if $f:\mathbb{R}\to\mathbb{C}$ is a uniformly almost periodic function, then $Y(t)=f(t)X(t)$ will be an almost periodic contractive weakly harmonizable process, since
\begin{equation}\label{eq:modulated}
Y(t)=f(t)X(t)=T(t)f(t)X(0),
\end{equation}
where $f(t)X(0)$ is a uniformly almost periodic function taking values in $L^{2}(P)$.

A class of modulated stationary stochastic processes introduced by Priestley~\cite{Priestley1981}, are obtained when a stationary process $X_{0}(t)$ is multiplied by some nonrandom modulating function $A(t)$:
\begin{equation}\label{eq:priestley_mod}
X(t)=A(t)X_{0}(t).
\end{equation}
This class of processes has been investigated by Joyeux~\cite{Joyeux1987} and Priestley~\cite{Priestley1981}. The book by Yaglom~\cite{Yaglom1987} provides a nice treatment of these processes. Priestley defined this class of processes $X:\mathbb{R}\to L_{0}^{2}(P)$ as oscillatory provided they have representation
\begin{equation}\label{eq:priestley_osc}
X(t)=\int_{\mathbb{R}}A(t,\lambda)e^{i\lambda t}\,dZ(\lambda)
\end{equation}
where $Z(\cdot)$ is a stochastic measure with orthogonal increments and
\begin{equation}\label{eq:A_rep}
A(t,\lambda)=\int_{\mathbb{R}}e^{itx}H(\lambda,dx)
\end{equation}
with $\tilde{H}(\cdot,B)$ a Borel function on $\mathbb{R}$, $H(\lambda,\cdot)$ a signed measure and $A(t,\lambda)$ having an absolute maximum at $\lambda=0$ independent of $t$.

Swift~\cite{Swift1997a}, extended the class of oscillatory processes and obtained a spectral representation for the class of oscillatory harmonizable process $X(\cdot)$ as
\begin{equation}\label{eq:osc_harm_rep}
X(t)=\int_{\mathbb{R}}A(t,\lambda)e^{i\lambda t}\,dZ(\lambda)
\end{equation}
where $Z(\cdot)$ is a stochastic measure satisfying
\begin{equation}\label{eq:Z_cov_osc}
E(Z(B_{1})\overline{Z(B_{2})})=F(B_{1},B_{2})
\end{equation}
with $F(\cdot,\cdot)$ a function of bounded Fréchet variation.

Note that if $A(t,\lambda)=1$, this class coincides with the weakly harmonizable class. Observe, further, that if $F(\cdot,\cdot)$ concentrates on the diagonal $\lambda=\lambda^{\prime}$, the oscillatory processes introduced by Priestley are obtained. In light of this, Priestley's class of processes will be termed oscillatory stationary.

A relation between the almost periodic contractive class and the oscillatory harmonizable class is implied by the following operator characterization, Swift~\cite{Swift1997a}.

\begin{theorem}\label{thm:osc_char}
$X(\cdot)$ is an oscillatory weakly harmonizable process if and only if it is representable as
\begin{equation}\label{eq:osc_rep}
X(t)=a(t)T(t)Y(0),\quad t\in\mathbb{R},
\end{equation}
where $Y_{0}=Y(0)\in H(X)=\overline{sp}\{X(t),t\in\mathbb{R}\}$ with $a(t)$ a densely defined closed operator in $H(X)$ for each $t\in\mathbb{R}$ and $\{T(s),s\in\mathbb{R}\}$ a weakly continuous family of positive definite contractive operators in $H(X)$ which commutes with each $a(t)$, $\bar{t}\in\mathbb{R}$.
\end{theorem}

Let $X:\mathbb{R}\to L_{0}^{2}(P)$ be an almost periodic contractive weakly harmonizable process, then $X(t)$ has representation
\begin{equation}\label{eq:apc_rep}
X(t)=T(t)Q(t)
\end{equation}
where $Q(t)=\bar{Z}(\mathbb{R},t)$ (see equation~\eqref{eq:Q_def}), is a uniformly almost periodic $L^{2}$-function and $\{T(t),t\in\mathbb{R}\}$ is a semi-group of positive definite contractive operators on $H(\mathbb{R})$ generated by $\{\tilde{E}_{\lambda},\lambda\in\mathbb{R}\}$. Here
\begin{equation}\label{eq:H_A_def3}
H(A)=\operatorname{cl}\{\sum_{j}\alpha_{j}Z(A_{j},t_{j}),t_{j}\in\mathbb{R},A_{j}\subset A\}
\end{equation}
and $\tilde{E}_{\lambda}X$ is the projection of $X$ onto $H((-\infty,\lambda])$ so that $Z((a,b],t)=\tilde{E}_{(a,b]}Q(t)$. Now for $X(t)$ to be oscillatory weakly harmonizable, $Q(t)$ must be a closed densely defined operator and $T(t)$ and $Q(t)$ must commute. But, since $T(t)$ and $\tilde{E}_{\lambda}(B)$ commute for all $t\in\mathbb{R}$ and $B\in\mathcal{B}$, it follows that $Q(t)$ and $\tilde{E}_{\lambda}(B)$ commute for all $t\in\mathbb{R}$ and $B\in\mathcal{B}$. Hence, $Q(t)$ and $T(t)$ commute for all $t\in\mathbb{R}$. Further, since
\begin{equation}\label{eq:Q_integral}
Q(t)=\int_{\mathbb{R}}Z(d\lambda,t)
\end{equation}
and $\hat{F}(\cdot,\cdot,s,t)$ is of finite Fréchet variation for $(s,t)\in\mathbb{R}\times\mathbb{R}$, it follows from a standard Hilbert space argument, that $Q(t)$ is a closed densely defined operator. But,
\begin{equation}\label{eq:apc_osc2}
X(t)=\bar{T}(t)Q(t)
\end{equation}
is then the form of the representation \eqref{eq:osc_rep} of an oscillatory weakly harmonizable process. This observation is summarized in the following proposition.

\begin{proposition}\label{prop:proper_containment}
The class of almost periodic contractive weakly harmonizable processes is properly contained in the class of oscillatory weakly harmonizable processes.
\end{proposition}

It is not difficult to see that this containment is proper. If $f:\mathbb{R}\to\mathbb{C}$ is a function (not necessarily uniformly almost periodic) which admits a generalized Fourier transform, then the process
\begin{equation}\label{eq:counterexample}
Y(t)=f(t)X(t)
\end{equation}
with $X(t)$ weakly harmonizable, is oscillatory weakly harmonizable but is not an almost periodic contractive weakly harmonizable process.

\section{A Series Approximation}\label{sec:series}

The approximation theorem for uniformly almost periodic functions yields the following series representation for the class of almost periodic contractive harmonizable processes.

\begin{theorem}\label{thm:series_rep}
A process $X:\mathbb{R}\to L_{0}^{2}(P)$ is almost periodic contractive weakly harmonizable if and only if the sequence of processes $\{\widetilde{X}_{n}(t),n\in\mathbb{N}\}$ converges uniformly to $X(t)$ with respect to $t$, where
\begin{equation}\label{eq:approx_series}
X_{N}(t)=\sum_{j=1}^{K(N)}Y_{j,K(N)}(t)e^{i\lambda_{j}t},
\end{equation}
and $\{Y_{j,K(N)}(t),j=1,\ldots,K(N),N\in\mathbb{N}\}$ is a collection of weakly harmonizable processes with $\{\lambda_{j},j\in\mathbb{N}\}$ a countable set of real numbers.
\end{theorem}

\begin{proof}
Suppose $X(\cdot)$ is an almost periodic contractive weakly harmonizable process, then it has representation
\begin{equation}\label{eq:X_TQ2}
X(t)=T(t)Q(t).
\end{equation}
Now consider $Q(t)$ a uniformly almost periodic $L^{2}$-function. By the classical approximation theorem, there is a sequence of trigonometric polynomials
\begin{equation}\label{eq:Q_approx}
Q_{N}(t)=\sum_{j=1}^{K(N)}Q_{j,K(N)}e^{i\lambda_{j}t}
\end{equation}
where for each $t\in\mathbb{R}$
\begin{equation}\label{eq:Q_limit}
\lim_{N\to\infty}Q_{N}(t)=Q(t).
\end{equation}
Let $\bar{X}_{N}(t)=T(t)\bar{Q}_{N}(t)$ so that
\begin{align}
\|X_{N}(t)-X(t)\| &= \|T(t)Q_{N}(t)-T(t)Q(t)\|\\
&\leq \|Q_{N}(t)-Q(t)\|\label{eq:X_approx_bound}
\end{align}
hence $X_{N}(t)$ converges to $X(t)$ uniformly in $t$ for compact sets. The
\begin{align}
X_{N}(t) &= T(t)Q(t)\\
&= T(t)\sum_{j=1}^{K(N)}Q_{j,K(N)}e^{i\lambda_{j}t}\\
&= \sum_{j=1}^{K(N)}T(t)Q_{j,K(N)}e^{i\lambda_{j}t}.\label{eq:X_N_series}
\end{align}
Now let $Y_{j,K(N)}(t)=T(t)Q_{j,K(N)}e^{i\lambda_{j}t}$, by a theorem of Rao~\cite{Rao1984}, $Y_{j,K(N)}(t)$ is weakly harmonizable. Thus if $X(t)$ is an almost periodic (in $t$) contractive weakly harmonizable process, then
\begin{equation}\label{eq:X_N_final}
X_{N}(t)=\sum_{j=1}^{K(N)}Y_{j,K(N)}(t)e^{i\lambda_{j}t},
\end{equation}
converges uniformly to $X(t)$ with respect to $t$.

For the converse, suppose $\{Y_{j,K(N)}(t),j=1,\ldots,K(N),N\in\mathbb{N}\}$ is a collection of weakly harmonizable processes so that
\begin{equation}\label{eq:X_N_assumed}
X_{N}(t)=\sum_{j=1}^{K(N)}Y_{j,K(N)}(t)e^{i\lambda_{j}t},
\end{equation}
converges uniformly with respect to $t$ to $X(t)$. Now $Y_{j,K(N)}(t)$ weakly harmonizable implies that there is a family of positive definite contractive operators so that
\begin{equation}\label{eq:Y_weakly_harm}
Y_{j,K(N)}(t)=T(t)Y_{j,K(N)}(0),
\end{equation}
but this implies that
\begin{equation}\label{eq:X_N_conclusion}
X_{N}(t)=\sum_{j=1}^{K(N)}Y_{j,K(N)}(t)e^{i\lambda_{j}t},
\end{equation}
is almost periodic contractive weakly harmonizable. Thus since $X_{N}(t)$ converges to $X(t)$ uniformly in $t$, $X(t)$ is almost periodic contractive weakly harmonizable.~\qed
\end{proof}

It is not difficult to show that $\lim_{N\to\infty}Y_{j,K(N)}(t)$ exists in the $L^{2}$-sense. If we let
\begin{equation}\label{eq:Y_limit}
Y_{j}(t)=\lim_{N\to\infty}Y_{j,K(N)}(t)
\end{equation}
then $X(t)$ can be considered to have a Fourier series whose coefficients are stochastic processes
\begin{equation}\label{eq:fourier_series}
X(t)\sim\sum_{j=1}^{\infty}Y_{j}(t)e^{i\lambda_{j}t},
\end{equation}
where $\{\lambda_{j},j=1,2,\ldots\}$ and $\{Y_{j}(t)\}$ are uniquely determined from $T(t)$ and $Q(t)$. The representation \eqref{eq:fourier_series} includes Ogura's representation~\cite{Ogura1971} for a harmonizable periodically correlated process, as Ogura's representation requires $Y_{j}(t)$ to be bandlimited stationary processes.

\begin{thebibliography}{99}

\bibitem{Besicovitch1954} Besicovitch, A.S., 1954. \textit{Almost Periodic Functions}. Dover Publications, Inc., New York.

\bibitem{ChangRao1986} Chang, D.K., Rao, M.M., 1986. Bimeasures and nonstationary processes. In: \textit{Real and Stochastic Analysis}, 7--118. John Wiley and Sons, New York.

\bibitem{Hurd1992} Hurd, H.L., 1992. Almost periodically unitary stochastic processes. \textit{Stochastic Processes and Applications}, 43(1), 99--113.

\bibitem{Joyeux1987} Joyeux, R., 1987. Slowly changing processes and harmonizability. \textit{Journal of Time Series Analysis}, 8(4).

\bibitem{Ogura1971} Ogura, H., 1971. Spectral representation of a periodic nonstationary random process. \textit{IEEE Trans.}, IT-17(2), 143--149.

\bibitem{Priestley1981} Priestley, M.B., 1981. \textit{Spectral Analysis and Time Series}, Volumes 1 and 2. Academic Press, London.

\bibitem{Rao1984} Rao, M.M., 1984. Harmonizable processes: structure theory. \textit{L'Enseign Math.}, 28, 295--351.

\bibitem{RieszNagy} Riesz, F., Sz-Nagy, B., \textit{Functional Analysis}. Dover, New York.

\bibitem{Rozanov1959} Rozanov, Yu., 1959. Spectral analysis of abstract functions. \textit{Theor. Prob. Appl.}, 4, 271--287.

\bibitem{Rozanov1967} Rozanov, Yu., 1967. \textit{Stationary Random Processes}. Holden-Day, San Francisco.

\bibitem{Swift1997a} Swift, R.J., 1997a. An operator characterization of oscillatory harmonizable processes. In: Goldstein, J., Gretsky, N., Uhl, J.J. (Editors), \textit{Stochastic Processes and Functional Analysis}, 235--243. Marcel Dekker, New York.

\bibitem{Swift1997b} Swift, R.J., 1997b. Some aspects of harmonizable processes and fields. In: \textit{Real and Stochastic Analysis: Recent Advances}, 303--365. CRC Press, New York.

\bibitem{Yaglom1987} Yaglom, A.M., 1987. \textit{Correlation Theory of Stationary and Related Random Functions}, Volumes 1 and 2. Springer-Verlag, New York.

\end{thebibliography}

\end{document}
