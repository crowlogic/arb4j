\documentclass[11pt]{article}
\usepackage{amsmath}
\usepackage{amssymb}
\usepackage{amsthm}
\usepackage{enumitem}

\newtheorem{theorem}{Theorem}[section]
\newtheorem{lemma}[theorem]{Lemma}
\newtheorem{corollary}[theorem]{Corollary}
\newtheorem{definition}[theorem]{Definition}
\newtheorem{remark}[theorem]{Remark}

\title{Definition of $\zeta(s)$, $Z(t)$ and Basic Notions}
\author{}
\date{}

\begin{document}

\maketitle

\section{The basic notions}\label{sec:basic}

The classical Riemann zeta-function
\begin{equation}\label{eq:zeta_def}
\zeta(s) = \sum_{n=1}^{\infty} n^{-s} = \prod_p (1 - p^{-s})^{-1} \quad (s = \sigma + it, \sigma > 1)
\end{equation}
admits analytic continuation to $\mathbb{C}$. It is regular on $\mathbb{C}$ except for a simple pole at $s = 1$. The product representation in equation~\eqref{eq:zeta_def} shows that $\zeta(s)$ does not vanish for $\sigma > 1$. The Laurent expansion of $\zeta(s)$ at $s = 1$ reads
\begin{equation}\label{eq:laurent}
\zeta(s) = \frac{1}{s-1} + \gamma_0 + \gamma_1(s-1) + \gamma_2(s-1)^2 + \cdots,
\end{equation}
where the so-called Stieltjes constants $\gamma_k$ are given by
\begin{equation}\label{eq:stieltjes}
\gamma_k = \frac{(-1)^k}{k!} \lim_{N \to \infty} \left( \sum_{m \leq N} \frac{\log^k m}{m} - \frac{\log^{k+1} N}{k+1} \right) \quad (k = 0, 1, 2, \ldots).
\end{equation}
In particular
\begin{equation}\label{eq:euler_constant}
\gamma \equiv \gamma_0 = \lim_{N \to \infty} \left( 1 + \frac{1}{2} + \cdots + \frac{1}{N} - \log N \right) = -\Gamma'(1) = 0.5772157\ldots
\end{equation}
is the Euler constant and
\begin{equation}\label{eq:gamma_func}
\Gamma(s) = \int_0^{\infty} x^{s-1} e^{-x} dx \quad (\text{Re } s > 0)
\end{equation}
is the familiar Euler gamma-function.

The product in equation~\eqref{eq:zeta_def} is called the Euler product. As usual, $p$ denotes prime numbers, so that by its very essence $\zeta(s)$ represents an important tool for the investigation of prime numbers. This is even more evident from the relation
\begin{equation}\label{eq:log_deriv}
-\frac{\zeta'(s)}{\zeta(s)} = \sum_{n=1}^{\infty} \Lambda(n) n^{-s} \quad (\sigma > 1),
\end{equation}
which follows by logarithmic differentiation of equation~\eqref{eq:zeta_def}, where the von Mangoldt function $\Lambda(n)$ is defined as
\begin{equation}\label{eq:mangoldt}
\Lambda(n) = \begin{cases}
\log p & \text{if } n = p^{\alpha}, \\
0 & \text{if } n \neq p^{\alpha}
\end{cases} \quad (\alpha \in \mathbb{N}).
\end{equation}

The zeta-function can be also used to generate many other important arithmetic functions; for example,
\begin{equation}\label{eq:squarefree}
\frac{\zeta(s)}{\zeta(2s)} \quad (\sigma > 1),
\end{equation}
\begin{equation}\label{eq:squarefull}
\frac{\zeta(2s)\zeta(3s)}{\zeta(6s)} \quad (\sigma > \frac{1}{2})
\end{equation}
generate the characteristic functions of squarefree and squarefull numbers, respectively. One also has, for a given $k \in \mathbb{N}$,
\begin{equation}\label{eq:divisor_gen}
\zeta^k(s) = \sum_{n=1}^{\infty} d_k(n) n^{-s} \quad (\sigma > 1),
\end{equation}
where the (general) divisor function $d_k(n)$ represents the number of ways $n$ can be written as a product of $k$ factors, so that in particular $d_1(n) \equiv 1$ and $d_2(n) = \sum_{\delta|n} 1$ is the number of positive divisors of $n$. The function $d_k(n)$ is a multiplicative function of $n$ (meaning $d_k(mn) = d_k(m)d_k(n)$ if $m$ and $n$ are coprime), and
\begin{equation}\label{eq:divisor_prime}
d_k(p^{\alpha}) = (-1)^{\alpha} \binom{-k}{\alpha} = \frac{k(k+1) \cdots (k+\alpha-1)}{\alpha!}
\end{equation}
for primes $p$ and $\alpha \in \mathbb{N}$.

Another significant aspect of $\zeta(s)$ is that it can be generalized to many other similar Dirichlet series (notably to the Selberg class $\mathcal{S}$, which will be discussed in Chapter 3). A vast body of literature exists on many facets of zeta-function theory, such as the distribution of its zeros and power moments of $|\zeta(\frac{1}{2} + it)|$ (see, e.g., the monographs \cite{Iv1}, \cite{Iv4}, \cite{Mot1}, \cite{Ram} and \cite{Tit3}). It is within this framework that the classical Hardy function (see, e.g., \cite{Iv1}) $Z(t)$ ($t \in \mathbb{R}$) arises, and plays an important role in the theory of $\zeta(s)$. It is defined as
\begin{equation}\label{eq:hardy_def}
Z(t) := \zeta\left(\frac{1}{2} + it\right) \left| \chi\left(\frac{1}{2} + it\right) \right|^{-1/2},
\end{equation}
where $\chi(s)$ comes from the well-known functional equation for $\zeta(s)$; see equations~\eqref{eq:func_eq} and~\eqref{eq:func_eq_alt} below. The basic properties of $Z(t)$ will be discussed in Section~\ref{sec:hardy}.

\section{The functional equation for $\zeta(s)$}\label{sec:func_eq}

The functional equation is one of the most fundamental tools of zeta-function theory. Therefore the following provides a proof which incidentally originated with the great German mathematician B. Riemann (1826-1866), who founded the theory of $\zeta(s)$ in his epoch-making memoir \cite{Rie}.

\begin{theorem}\label{thm:func_eq}
The function $\zeta(s)$ admits analytic continuation to $\mathbb{C}$, where it satisfies the functional equation
\begin{equation}\label{eq:func_eq}
\pi^{-s/2} \zeta(s) \Gamma\left(\frac{s}{2}\right) = \pi^{-(1-s)/2} \zeta(1-s) \Gamma\left(\frac{1-s}{2}\right).
\end{equation}
\end{theorem}

\begin{remark}\label{rem:func_eq_alt}
The functional equation~\eqref{eq:func_eq} is in a symmetric form. Alternatively one can write equation~\eqref{eq:func_eq} as
\begin{equation}\label{eq:func_eq_alt}
\zeta(s) = \chi(s) \zeta(1-s),
\end{equation}
where
\begin{equation}\label{eq:chi_def}
\chi(s) = \frac{\Gamma\left(\frac{1-s}{2}\right)}{\Gamma\left(\frac{s}{2}\right)} \pi^{s-1/2}.
\end{equation}
This expression can be put into other equivalent forms. For example,
\begin{equation}\label{eq:chi_alt}
\chi(s) = 2^s \pi^{s-1} \sin\left(\frac{\pi s}{2}\right) \Gamma(1-s) = \frac{(2\pi)^s}{2\Gamma(s)} \cos(\pi s/2),
\end{equation}
where the well-known identities
\begin{equation}\label{eq:gamma_identities}
\Gamma(s)\Gamma(1-s) = \frac{\pi}{\sin(\pi s)}, \quad \Gamma(s)\Gamma\left(s + \frac{1}{2}\right) = 2^{1-2s}\sqrt{\pi}\Gamma(2s)
\end{equation}
are used.
\end{remark}

\begin{remark}\label{rem:chi_property}
Note that equation~\eqref{eq:func_eq_alt} gives the identity
\begin{equation}\label{eq:chi_inverse}
\chi(s)\chi(1-s) = 1.
\end{equation}
All identities~\eqref{eq:func_eq}--\eqref{eq:chi_inverse} hold for $s \in \mathbb{C}$.
\end{remark}

Before proceeding to the proof of the functional equation~\eqref{eq:func_eq}, the following result on a transformation formula for the theta-function (see equation~\eqref{eq:theta_def}) is needed, embodied in the following lemma.

\begin{lemma}\label{lem:theta_transform}
The following holds:
\begin{equation}\label{eq:theta_transform}
\sum_{n=-\infty}^{\infty} e^{-\pi n^2 t} = \frac{1}{\sqrt{t}} \sum_{n=-\infty}^{\infty} e^{-\pi n^2/t} \quad (t > 0).
\end{equation}
\end{lemma}

\begin{proof}[Proof of Lemma~\ref{lem:theta_transform}]
For $v \in \mathbb{R}$, $\tau = iy$, $y > 0$, the Fourier expansion
\begin{equation}\label{eq:fourier}
f(v) := \sum_{n=-\infty}^{\infty} e^{\pi i \tau (n+v)^2} = \sum_{k=-\infty}^{\infty} c_k e^{2\pi ikv}
\end{equation}
holds, since $f(v)$ is periodic with period 1 and $f(v) \in C^1[0,1]$. Hence with $A = -2\pi ik$, $B = \pi y$, for the Fourier coefficients $c_k$ the expression
\begin{align}\label{eq:fourier_coeff}
c_k &= \int_0^1 \sum_{n=-\infty}^{\infty} e^{\pi i \tau (n+v)^2 - 2\pi ikv} dv \\
&= \sum_{n=-\infty}^{\infty} \int_0^1 e^{\pi i \tau (n+v)^2 - 2\pi ik(n+v)} dv \nonumber \\
&= \int_{-\infty}^{\infty} e^{-\pi yv^2 - 2\pi ikv} dv = \int_{-\infty}^{\infty} e^{Av - Bv^2} dv \nonumber \\
&= \sqrt{\frac{\pi}{B}} e^{A^2/(4B)} = \frac{1}{\sqrt{y}} e^{-\pi k^2/y}. \nonumber
\end{align}
The change of order of summation and integration in equation~\eqref{eq:fourier_coeff} is justified by absolute convergence. Here the classical integral
\begin{equation}\label{eq:gaussian_int}
\int_{-\infty}^{\infty} \exp(At - Bt^2) dt = \sqrt{\frac{\pi}{B}} \exp\left(\frac{A^2}{4B}\right) \quad (\text{Re } B > 0)
\end{equation}
is used.

Setting $v = 0$, $i\tau = i^2y = -t$, $y = t$ in equations~\eqref{eq:fourier} and~\eqref{eq:fourier_coeff}, equation~\eqref{eq:theta_transform} of Lemma~\ref{lem:theta_transform} is obtained. By analytic continuation it is seen that equation~\eqref{eq:theta_transform} remains valid for $\text{Re } t > 0$. If the theta-function is defined as
\begin{equation}\label{eq:theta_def}
\vartheta(t) = \sum_{n=1}^{\infty} e^{-\pi n^2 t} \quad (\text{Re } t > 0),
\end{equation}
then equation~\eqref{eq:theta_transform} yields the transformation formula
\begin{equation}\label{eq:theta_transform_alt}
\vartheta(t) = \frac{1}{2\sqrt{t}} \left( 2\vartheta\left(\frac{1}{t}\right) + 1 \right) - \frac{1}{2} \quad (\text{Re } t > 0).
\end{equation}
\end{proof}

\begin{proof}[Proof of Theorem~\ref{thm:func_eq}]
Starting from
\begin{equation}\label{eq:gamma_integral}
\Gamma\left(\frac{s}{2}\right) = \int_0^{\infty} e^{-x} x^{s/2-1} dx \quad (\sigma > 0),
\end{equation}
which is just equation~\eqref{eq:gamma_func} with $s/2$ in place of $s$. If $n \in \mathbb{N}$, writing $\pi n^2 x$ in place of $x$ gives
\begin{equation}\label{eq:gamma_scaled}
\Gamma\left(\frac{s}{2}\right) = \pi^{s/2} n^s \int_0^{\infty} e^{-\pi n^2 x} x^{s/2-1} dx \quad (\sigma > 0),
\end{equation}
or
\begin{equation}\label{eq:zeta_integral_term}
n^{-s} = \frac{\pi^{s/2}}{\Gamma(s/2)} \int_0^{\infty} e^{-\pi n^2 x} x^{s/2-1} dx \quad (\sigma > 0).
\end{equation}
Summation over $n$ gives, for $\sigma > 1$,
\begin{equation}\label{eq:zeta_integral}
\zeta(s) = \sum_{n=1}^{\infty} n^{-s} = \frac{\pi^{s/2}}{\Gamma(s/2)} \sum_{n=1}^{\infty} \int_0^{\infty} e^{-\pi n^2 x} x^{s/2-1} dx.
\end{equation}
Since the series
\begin{equation}\label{eq:convergence}
\sum_{n=1}^{\infty} \int_0^{\infty} \left| e^{-\pi n^2 x} x^{s/2-1} \right| dx = \sum_{n=1}^{\infty} \int_0^{\infty} e^{-\pi n^2 x} x^{\sigma/2-1} dx = \sum_{n=1}^{\infty} \frac{\Gamma(\sigma/2) \pi^{-\sigma/2}}{n^{\sigma}}
\end{equation}
converges for $\sigma > 1$, the order of summation and integration can be changed to obtain
\begin{equation}\label{eq:zeta_theta}
\zeta(s) = \frac{\pi^{s/2}}{\Gamma(s/2)} \int_0^{\infty} \vartheta(x) x^{s/2-1} dx.
\end{equation}

In view of equation~\eqref{eq:theta_transform_alt}, equation~\eqref{eq:zeta_theta} may be written as
\begin{align}\label{eq:zeta_functional}
\pi^{-s/2} \Gamma(s/2) \zeta(s) &= \int_0^1 x^{s/2-1} \vartheta(x) dx + \int_1^{\infty} x^{s/2-1} \vartheta(x) dx \\
&= \int_0^1 x^{s/2-1} \left[ x^{-1/2} \vartheta\left(\frac{1}{x}\right) + \frac{1}{2} x^{-1/2} - \frac{1}{2} \right] dx + \int_1^{\infty} x^{s/2-1} \vartheta(x) dx \nonumber \\
&= \frac{1}{s-1} - \frac{1}{s} + \int_0^1 x^{s/2-3/2} \vartheta(1/x) dx + \int_1^{\infty} x^{s/2-1} \vartheta(x) dx \nonumber \\
&= \frac{1}{s(s-1)} + \int_1^{\infty} \left( x^{-s/2-1/2} + x^{s/2-1} \right) \vartheta(x) dx. \nonumber
\end{align}

Note first that the last expression in equation~\eqref{eq:zeta_functional} remains invariant if $s$ is replaced by $1-s$. Secondly, the last integral in equation~\eqref{eq:zeta_functional} converges uniformly (since $x \geq 1$ in the integrand) in any strip
\begin{equation}\label{eq:strip}
-\infty < a \leq \sigma = \text{Re } s \leq b < +\infty.
\end{equation}
Consequently the last integral in equation~\eqref{eq:zeta_functional} represents an entire function of $s$. Therefore
\begin{equation}\label{eq:entire_part}
\pi^{-s/2} \Gamma(s/2) \zeta(s) - \frac{1}{s(s-1)}
\end{equation}
is an entire function of $s$. Since $\pi^{s/2}/\Gamma(s/2)$ is an entire function (because $\Gamma(s)$ has no zeros),
\begin{equation}\label{eq:zeta_entire}
\zeta(s) - \frac{1}{s(s-1)} \frac{\pi^{s/2}}{2\Gamma(s/2)}
\end{equation}
is also an entire function. Further, since $s\Gamma(s/2) = 2\Gamma(s/2 + 1)$, it follows that
\begin{equation}\label{eq:zeta_pole}
\zeta(s) - \frac{1}{s-1} \frac{\pi^{s/2}}{2\Gamma(s/2 + 1)}
\end{equation}
is an entire function. Since $\sqrt{\pi}/(2\Gamma(3/2)) = 1$, $\zeta(s) - 1/(s-1)$ is an entire function, thus $\zeta(s)$ is regular in $\mathbb{C}$ except for a simple pole at $s = 1$ with residue 1.

This discussion shows that equation~\eqref{eq:zeta_functional} provides analytic continuation of $\zeta(s)$ to $\mathbb{C}$, as well as the functional equation~\eqref{eq:func_eq}.
\end{proof}

\begin{corollary}\label{cor:xi_function}
If
\begin{equation}\label{eq:eta_xi_def}
\eta(s) = \pi^{-s/2} \Gamma(s/2) \zeta(s), \quad \xi(s) = \frac{1}{2} s(s-1) \eta(s)
\end{equation}
are defined, then $\xi(s)$ is an entire function of $s$ satisfying the functional equation $\xi(s) = \xi(1-s)$. It is real for $t = 0$ and $\sigma = 1/2$ and $\xi(0) = \xi(1) = 1/2$.
\end{corollary}

\section{Properties of Hardy's function}\label{sec:hardy}

Continuing with the discussion of $Z(t)$. Recall that the zeta, sine and the gamma-function take conjugate values at conjugate points. Hence it follows from equations~\eqref{eq:chi_alt} and~\eqref{eq:chi_inverse} that
\begin{equation}\label{eq:chi_real}
\chi\left(\frac{1}{2} + it\right) = \chi\left(\frac{1}{2} - it\right) = \chi^{-1}\left(\frac{1}{2} + it\right),
\end{equation}
so that equation~\eqref{eq:hardy_def} gives $Z(t) \in \mathbb{R}$ when $t \in \mathbb{R}$, and $|Z(t)| = |\zeta(\frac{1}{2} + it)|$. Thus the zeros of $\zeta(s)$ on the "critical line" $\text{Re } s = 1/2$ are in one-to-one correspondence with the real zeros of $Z(t)$. This property makes $Z(t)$ an invaluable tool in the study of the zeros of the zeta-function on the critical line. If equations~\eqref{eq:func_eq} and~\eqref{eq:func_eq_alt} are used, then
\begin{equation}\label{eq:chi_phase}
\left| \chi\left(\frac{1}{2} + it\right) \right|^{-1/2} = \pi^{-it/2} \frac{\left| \Gamma\left(\frac{1}{4} + \frac{it}{2}\right) \right|}{\Gamma\left(\frac{1}{4} + \frac{it}{2}\right)} := e^{i\theta(t)},
\end{equation}
say, where $\theta(t)$ is a smooth function for which
\begin{equation}\label{eq:theta_def_alt}
\theta(t) = -\frac{1}{2i} \log \chi\left(\frac{1}{2} + it\right), \quad \theta'(t) = -\frac{1}{2} \frac{\chi'\left(\frac{1}{2} + it\right)}{\chi\left(\frac{1}{2} + it\right)}.
\end{equation}

Note that also
\begin{equation}\label{eq:theta_imag}
\theta(t) = \text{Im} \left[ \log \Gamma\left(\frac{1}{4} + \frac{it}{2}\right) \right] - \frac{t}{2} \log \pi \in \mathbb{R}
\end{equation}
if $t \in \mathbb{R}$, thus $\theta(0) = 0$. The function $\theta(t)$ is odd, since in view of $\chi(s)\chi(1-s) = 1$,
\begin{equation}\label{eq:theta_odd}
\theta(-t) = -\frac{1}{2i} \log \chi\left(\frac{1}{2} - it\right) = -\frac{1}{2i} \log \frac{1}{\chi\left(\frac{1}{2} + it\right)} = \frac{1}{2i} \log \chi\left(\frac{1}{2} + it\right) = -\theta(t).
\end{equation}
It is also monotonic increasing for $t \geq 7$, which follows from formulas~\eqref{eq:theta_explicit}--\eqref{eq:r_def} below. $Z(t)$ may be written alternatively as
\begin{equation}\label{eq:hardy_alt}
Z(t) = e^{i\theta(t)} \zeta\left(\frac{1}{2} + it\right), \quad e^{i\theta(t)} := \pi^{-it/2} \frac{\Gamma\left(\frac{1}{4} + \frac{it}{2}\right)}{\left| \Gamma\left(\frac{1}{4} + \frac{it}{2}\right) \right|} \quad (\theta(t) \in \mathbb{R}).
\end{equation}

It is also useful to note that $Z(t)$ is an even function of $t$, because
\begin{align}\label{eq:hardy_even}
Z(-t) &= \zeta\left(\frac{1}{2} - it\right) \left| \chi\left(\frac{1}{2} - it\right) \right|^{-1/2} \\
&= \zeta\left(\frac{1}{2} + it\right) \left| \chi\left(\frac{1}{2} - it\right) \right|^{1/2} = \zeta\left(\frac{1}{2} + it\right) \left| \chi\left(\frac{1}{2} + it\right) \right|^{-1/2} = Z(t). \nonumber
\end{align}

The explicit representation
\begin{equation}\label{eq:theta_explicit}
\theta(t) = \frac{t}{2} \log \frac{t}{2\pi} - \frac{t}{2} - \frac{\pi}{8} + R(t)
\end{equation}
holds. Here (see Lemma 5.1 for a proof; here $R(t)$ is not to be confused with the error term in the Dirichlet divisor problem)
\begin{equation}\label{eq:r_def}
R(t) := \frac{t}{4} \log\left(1 + \frac{1}{4t^2}\right) + \frac{1}{4} \arctan \frac{1}{2t} + \frac{t}{2} \int_0^{\infty} \frac{\psi(u)}{(u + \frac{1}{4})^2 + (\frac{t}{2})^2} du
\end{equation}
with
\begin{equation}\label{eq:psi_def}
\psi(x) = x - [x] - \frac{1}{2} = -\sum_{n=1}^{\infty} \frac{\sin(2n\pi x)}{n\pi} \quad (x \notin \mathbb{Z}).
\end{equation}
The representation~\eqref{eq:theta_explicit}--\eqref{eq:r_def} follows from Stirling's formula (see equation~\eqref{eq:stirling}) for the gamma-function in the form
\begin{equation}\label{eq:stirling_log}
\log \Gamma(s) = (s - 1/2) \log s - s + \log \sqrt{2\pi} - \int_0^{\infty} \frac{\psi(u)}{u + s} du,
\end{equation}
which in turn is a consequence of the product formula
\begin{equation}\label{eq:gamma_product}
\frac{1}{\Gamma(s)} = s \exp(\gamma s) \prod_{n=1}^{\infty} \left(1 + \frac{s}{n}\right) e^{-s/n}.
\end{equation}

Note that equation~\eqref{eq:gamma_product} valid for $s \in \mathbb{C}$, and can serve as a definition of $\Gamma(s)$ equivalent to equation~\eqref{eq:gamma_func}.

The expression~\eqref{eq:theta_explicit} is very useful, since it allows one to evaluate explicitly all the derivatives of $\theta(t)$. For $t \to \infty$ it is seen that $R(t)$ admits an asymptotic expansion in terms of negative powers of $t$, and from equation~\eqref{eq:theta_imag} and Stirling's formula it is found that ($B_k$ is the $k$th Bernoulli number)
\begin{equation}\label{eq:r_asymptotic}
R(t) \sim \sum_{n=1}^{\infty} \frac{(2^{2n} - 1)|B_{2n}|}{2^{2n}(2n-1) 2n t^{2n-1}}.
\end{equation}
The meaning of $\sim$ in equation~\eqref{eq:r_asymptotic} is that, for an arbitrary integer $N \geq 1$, $R(t)$ equals the sum of the first $N$ terms of the series in equation~\eqref{eq:r_asymptotic}, plus the error term, which is $O_N(t^{-2N-1})$. In general there will be, for $k \geq 0$ and suitable constants $c_{k,n}$,
\begin{equation}\label{eq:r_derivatives}
R^{(k)}(t) \sim \sum_{n=1}^{\infty} c_{k,n} t^{1-2n-k}.
\end{equation}

Thus equations~\eqref{eq:theta_explicit} and~\eqref{eq:r_asymptotic} give
\begin{equation}\label{eq:theta_asymptotic}
\theta(t) \sim \frac{t}{2} \log \frac{t}{2\pi} - \frac{t}{2} - \frac{\pi}{8} + \sum_{n=1}^{\infty} \frac{(2^{2n-1} - 1)|B_{2n}|}{2^{2n}(2n-1) 2n t^{2n-1}},
\end{equation}
and there are also asymptotic expansions for the derivatives of $\theta(t)$. In particular, the approximations
\begin{align}\label{eq:theta_approx}
\theta(t) &= \frac{t}{2} \log \frac{t}{2\pi} - \frac{t}{2} - \frac{\pi}{8} + \frac{1}{48t} + \frac{7}{5760t^3} + O\left(\frac{1}{t^5}\right), \\
\theta'(t) &= \frac{1}{2} \log \frac{t}{2\pi} + O\left(\frac{1}{t^2}\right), \nonumber \\
\theta''(t) &= \frac{1}{2t} + O\left(\frac{1}{t^3}\right) \nonumber
\end{align}
hold, which are sufficiently sharp for many applications.

\section{The distribution of zeta-zeros}\label{sec:zeros}

In what concerns the distribution of zeros of $\zeta(s)$, it is known that $\zeta(s)$ has no zeros in the region
\begin{equation}\label{eq:zero_free}
\sigma \geq 1 - C(\log t)^{-2/3}(\log \log t)^{-1/3} \quad (C > 0, t \geq t_0 > 0).
\end{equation}

This result, the strongest so-called zero-free region for $\zeta(s)$ even today, was obtained by an application of I. M. Vinogradov's method of exponential sums. In a modern form, the crucial bound which implies equation~\eqref{eq:zero_free} states (see, e.g., \cite{Iv1}, chapter 6) that
\begin{equation}\label{eq:vinogradov}
\sum_{N < n \leq N_1 \leq 2N} n^{it} \ll N \exp\left(-C \frac{\log^3 N}{\log^2 t}\right) \quad (C > 0)
\end{equation}
for $N_0 \leq N \leq \frac{1}{2}t$, $t \geq t_0$. From equation~\eqref{eq:func_eq} it follows that $\zeta(-2n) = 0$ for $n \in \mathbb{N}$. These zeros are the only real zeros of $\zeta(s)$, and are called the trivial zeros of $\zeta(s)$. In 1859, B. Riemann \cite{Rie} calculated a few complex zeros of $\zeta(s)$ and found that they lie on the line $\text{Re } s = \frac{1}{2}$, which is called the critical line in the theory of $\zeta(s)$. The first ten pairs of complex zeros (arranged in size according to their absolute value) are (see, e.g., C. B. Haselgrove \cite{Has})
\begin{align}\label{eq:first_zeros}
&\frac{1}{2} \pm i14.134725\ldots, \quad \frac{1}{2} \pm i21.022039\ldots, \quad \frac{1}{2} \pm i25.010857\ldots, \\
&\frac{1}{2} \pm i30.424876\ldots, \quad \frac{1}{2} \pm i32.935061\ldots, \quad \frac{1}{2} \pm i37.586178\ldots, \nonumber \\
&\frac{1}{2} \pm i40.918719\ldots, \quad \frac{1}{2} \pm i43.327073\ldots, \quad \frac{1}{2} \pm i48.005150\ldots, \nonumber \\
&\frac{1}{2} \pm i49.773832\ldots. \nonumber
\end{align}

The number of complex zeros $\rho = \beta + i\gamma$ of $\zeta(s)$ with $0 < \gamma \leq T$ (multiplicities included) is denoted by $N(T)$. The asymptotic formula for $N(T)$ is the famous Riemann-von Mangoldt formula. It was enunciated by B. Riemann \cite{Rie} in 1859, but proved by H. von Mangoldt \cite{Man} in 1895. It is stated here as follows.

\begin{theorem}\label{thm:riemann_mangoldt}
Let
\begin{equation}\label{eq:s_def}
S(T) := \frac{1}{\pi} \arg \zeta\left(\frac{1}{2} + iT\right).
\end{equation}
Then
\begin{equation}\label{eq:n_formula}
N(T) = \frac{T}{2\pi} \log \frac{T}{2\pi} - \frac{T}{2\pi} + \frac{7}{8} + S(T) + O\left(\frac{1}{T}\right),
\end{equation}
where the $O$-term is a continuous function of $T$, and
\begin{equation}\label{eq:s_bound}
S(T) = O(\log T).
\end{equation}
Here $\arg \zeta(\frac{1}{2} + iT)$ is evaluated by continuous variation starting from $\arg \zeta(2) = 0$ and proceeding along straight lines, first up to $2 + iT$ and then to $1/2 + iT$, assuming that $T$ is not an ordinate of a zeta zero. If $T$ is an ordinate of a zero, then $S(T) = S(T + 0)$ is set.
\end{theorem}

\begin{remark}\label{rem:s_rh}
On the RH (the Riemann hypothesis, that all complex zeros of $\zeta(s)$ have real parts equal to $1/2$) one can slightly improve equation~\eqref{eq:s_bound} and obtain that (see \cite{Tit3})
\begin{equation}\label{eq:s_rh_bound}
S(T) = O\left(\frac{\log T}{\log \log T}\right).
\end{equation}
\end{remark}

\begin{proof}[Proof of Theorem~\ref{thm:riemann_mangoldt}]
Let $D$ be the rectangle with vertices $2 \pm iT$, $-1 \pm iT$, where $T > 3$ is not an ordinate of a zero. The function $\xi(s)$, defined by equation~\eqref{eq:eta_xi_def}, has $2N(T)$ zeros in the interior of $D$, and none on the boundary. Therefore
\begin{equation}\label{eq:contour_integral}
N(T) = \frac{1}{4\pi} \text{Im} \left[ \oint_D \frac{\xi'(s)}{\xi(s)} ds \right].
\end{equation}

Logarithmic differentiation of equation~\eqref{eq:eta_xi_def} gives
\begin{equation}\label{eq:xi_log_deriv}
\frac{\xi'(s)}{\xi(s)} = \frac{1}{s} + \frac{1}{s-1} + \frac{\eta'(s)}{\eta(s)},
\end{equation}
where $\eta(s)$ is also given by equation~\eqref{eq:eta_xi_def}. Observe first that
\begin{equation}\label{eq:pole_contribution}
\text{Im} \left[ \oint_D \left( \frac{1}{s} + \frac{1}{s-1} \right) ds \right] = 4\pi.
\end{equation}
Next, note that $\eta(s) = \eta(1-s)$ and $\eta(\sigma \pm it)$ are conjugates, so that
\begin{equation}\label{eq:eta_symmetry}
\oint_D \frac{\eta'(s)}{\eta(s)} ds = 4 \text{Im} \left[ \int_L \frac{\eta'(s)}{\eta(s)} ds \right],
\end{equation}
where $L$ consists of the segments $[2, 2+iT]$ and $[2+iT, 1/2+iT]$. Therefore
\begin{align}\label{eq:eta_integral}
\text{Im} \left[ \int_L \frac{\eta'(s)}{\eta(s)} ds \right] &= \text{Im} \left[ \int_L \left( -\frac{1}{2} \log \pi + \frac{1}{2} \frac{\Gamma'(s/2)}{\Gamma(s/2)} + \frac{\zeta'(s)}{\zeta(s)} \right) ds \right] \\
&= -\frac{1}{2}(\log \pi) T + \text{Im} \left[ \int_L \frac{\Gamma'(s/2)}{2\Gamma(s/2)} ds + \int_L \frac{\zeta'(s)}{\zeta(s)} ds \right]. \nonumber
\end{align}

Note that
\begin{equation}\label{eq:gamma_integral_eval}
\text{Im} \left[ \int_L \frac{\Gamma'(s/2)}{2\Gamma(s/2)} ds \right] = \text{Im} \log \Gamma\left(\frac{1}{4} + \frac{iT}{2}\right),
\end{equation}
and using Stirling's formula in the form~\eqref{eq:stirling} there is
\begin{equation}\label{eq:stirling_eval}
\text{Im} \left[ \int_L \frac{\Gamma'(s/2)}{2\Gamma(s/2)} ds \right] = \frac{T}{2} \log \frac{T}{2} - \frac{T}{2} - \frac{\pi}{8} + O\left(\frac{1}{T}\right),
\end{equation}
and to prove Theorem~\ref{thm:riemann_mangoldt} it remains to show that
\begin{equation}\label{eq:zeta_integral_bound}
\text{Im} \left[ \int_{2+iT}^{1/2+iT} \frac{\zeta'(s)}{\zeta(s)} ds \right] = O(\log T),
\end{equation}
since the integral over the other segment of $L$ is clearly bounded. To prove equation~\eqref{eq:zeta_integral_bound} some estimates involving $\zeta'/\zeta$ are needed. The expression
\begin{equation}\label{eq:zeta_log_deriv}
\frac{\zeta'(s)}{\zeta(s)} = B - \frac{1}{s-1} + \frac{1}{2} \log \pi - \frac{\Gamma'(s/2+1)}{2\Gamma(s/2+1)} + \sum_{\rho} \left( \frac{1}{s-\rho} + \frac{1}{\rho} \right)
\end{equation}
is used. This formula, where $\rho$ denotes complex (non-trivial) zeros of $\zeta(s)$, and $B = \log 2 + \frac{1}{2} \log \pi - 1 - \frac{1}{2}\gamma$, follows by logarithmic differentiation of the product formula
\begin{equation}\label{eq:hadamard_product}
f(s) = e^{A+Bs} \prod_{n=1}^{\infty} (1 - s/\rho_n) e^{s/\rho_n}
\end{equation}
for suitable constants $A$, $B$ if $f(s)$ is an integral function of order 1 with zeros $\rho_1, \rho_2, \ldots$. Taking in equation~\eqref{eq:hadamard_product} $f(s) = \xi(s)$, which is an integral function of order 1, equation~\eqref{eq:zeta_log_deriv} is obtained.

Now suppose that $t \geq 2$, $1 \leq \sigma \leq 2$. The gamma-term in equation~\eqref{eq:zeta_log_deriv} is $\ll \log t$ and consequently
\begin{equation}\label{eq:zeta_real_part}
-\text{Re } \frac{\zeta'(s)}{\zeta(s)} < C \log t - \sum_{\rho} \text{Re } \left( \frac{1}{s-\rho} + \frac{1}{\rho} \right) \quad (1 \leq \sigma \leq 2, t \geq 2).
\end{equation}

In equation~\eqref{eq:zeta_real_part} take $s = 2 + iT$. Since $\frac{\zeta'}{\zeta}(2 + iT) \ll 1$, 
\begin{equation}\label{eq:zero_sum_bound}
\sum_{\rho} \text{Re } \left( \frac{1}{s-\rho} + \frac{1}{\rho} \right) < C \log T
\end{equation}
is obtained.

If $\rho = \beta + i\gamma$ is a non-trivial zero of $\zeta(s)$, then
\begin{equation}\label{eq:rho_real_part}
\text{Re } \frac{1}{\rho} = \frac{\beta}{\beta^2 + \gamma^2} > 0
\end{equation}
and
\begin{equation}\label{eq:s_rho_real_part}
\text{Re } \frac{1}{s-\rho} = \frac{2-\beta}{(2-\beta)^2 + (T-\gamma)^2} \geq \frac{1}{4 + (T-\gamma)^2},
\end{equation}
hence equation~\eqref{eq:zero_sum_bound} gives
\begin{equation}\label{eq:zero_density}
\sum_{\rho} \frac{1}{1 + (T-\gamma)^2} \ll \log T,
\end{equation}
where summation is over all non-trivial zeros $\rho$ of $\zeta(s)$. The bound~\eqref{eq:zero_density} immediately gives
\begin{equation}\label{eq:zero_count}
N(T+1) - N(T) \ll \sum_{\rho} \frac{1}{1 + (T-\gamma)^2} \ll \log T,
\end{equation}
that is, each strip $T < t \leq T+1$ contains fewer than $C \log T$ zeros of $\zeta(s)$ for some absolute constant $C > 0$. Again using equation~\eqref{eq:zeta_log_deriv} with $s = \sigma + it$, $-1 \leq \sigma \leq 2$ and $2 + it$, where $t > 2$ is not an ordinate of any $\rho$, and subtracting, 
\begin{equation}\label{eq:zeta_difference}
\frac{\zeta'(s)}{\zeta(s)} = \sum_{\rho} \left( \frac{1}{s-\rho} - \frac{1}{2+it-\rho} \right) + O(\log t)
\end{equation}
is obtained.

In equation~\eqref{eq:zeta_difference} for the terms with $|\gamma - t| \geq 1$ there is
\begin{equation}\label{eq:term_estimate}
\left| (s-\rho)^{-1} - (2+it-\rho)^{-1} \right| = \frac{2-\sigma}{|(s-\rho)(2+it-\rho)|} \leq \frac{3}{|\gamma - t|^2}
\end{equation}
if $-1 \leq \sigma \leq 2$. Hence by equation~\eqref{eq:zeta_log_deriv} the portion of the sum in equation~\eqref{eq:zeta_difference} for which $|\gamma - t| \geq 1$ is $\ll \log t$, and for $|\gamma - t| < 1$ there is $|2 + it - \rho| \geq 1$, and the number of such $\rho$ is $\ll \log t$ by equation~\eqref{eq:zero_count}. Thus
\begin{equation}\label{eq:zeta_local}
\frac{\zeta'(s)}{\zeta(s)} = \sum_{\rho, |\gamma - t| < 1} \frac{1}{s-\rho} + O(\log t) \quad (-1 \leq \sigma \leq 2)
\end{equation}
is obtained.

Now the proof of equation~\eqref{eq:zeta_integral_bound} easily follows, since by equation~\eqref{eq:zeta_local} (here $\Delta$ denotes the variation of the argument)
\begin{align}\label{eq:arg_variation}
\text{Im} \left[ \int_{2+iT}^{1/2+iT} \frac{\zeta'(s)}{\zeta(s)} ds \right] &= \text{Im} \left[ \int_{2+iT}^{1/2+iT} \sum_{\rho, |\gamma - t| < 1} \frac{1}{s-\rho} ds \right] + O(\log T) \\
&= \sum_{\rho, |\gamma - t| < 1} \Delta \arg(s-\rho) + O(\log T) = O(\log T), \nonumber
\end{align}
since $|\Delta \arg(s-\rho)| < \pi$ on $[1/2 + iT, 2 + iT]$ and equation~\eqref{eq:zero_count} holds. This completes the proof of Theorem~\ref{thm:riemann_mangoldt}. Note, however, that by following the preceding proof, in view of equation~\eqref{eq:gamma_integral_eval}, the following corollary is obtained.
\end{proof}

\begin{corollary}\label{cor:n_theta}
\begin{equation}\label{eq:n_theta_relation}
N(T) = \frac{1}{\pi} \theta(T) + 1 + S(T).
\end{equation}
\end{corollary}

Formula~\eqref{eq:n_theta_relation} shows the important connection between the functions $N(T)$, $\theta(T)$ and $S(T)$. Recall that $\theta(T)$ is a very smooth function, so that the jumps of $S(T)$ come at the zeros of $\zeta(s)$, which is of course also evident from its definition~\eqref{eq:s_def}. Since (see equation~\eqref{eq:theta_imag}) $\theta(t)$ plays a fundamental role in the theory of $Z(t)$, Corollary~\ref{cor:n_theta} shows an intrinsic connection between $N(T)$, $S(T)$ and $Z(T)$. This is also evident from the relation
\begin{equation}\label{eq:log_zeta_relation}
\log \zeta\left(\frac{1}{2} + it\right) = \log |Z(t)| + \pi i S(t).
\end{equation}

\begin{corollary}\label{cor:gamma_sums}
The following hold:
\begin{align}\label{eq:gamma_bounds}
\sum_{|\gamma| \leq T} \frac{1}{|\gamma|} &\ll \log^2 T, \\
\sum_{|\gamma| > T} \frac{1}{\gamma^2} &\ll \frac{\log T}{T}, \nonumber \\
\gamma_n &\sim \frac{2\pi n}{\log n} \quad (n \to \infty). \nonumber
\end{align}
In equation~\eqref{eq:gamma_bounds} $0 < \gamma_1 \leq \gamma_2 \leq \cdots$ denote the consecutive ordinates of non-trivial zeros $\rho = \beta + i\gamma$ of $\zeta(s)$. They should not be confused with the Stieltjes constants, defined before equation~\eqref{eq:gamma_func} as the Laurent coefficients (Pierre Alphonse Laurent, July 18, 1813-September 2, 1854, French mathematician) of $\zeta(s)$ at $s = 1$. Both bounds in equation~\eqref{eq:gamma_bounds} follow by partial summation from equations~\eqref{eq:s_def} and~\eqref{eq:n_formula}, as well as the asymptotic formula for $\gamma_n$ in view of the obvious inequality $N(\gamma_n - 1) < n \leq N(\gamma_n + 1)$.
\end{corollary}

Finally a few words about the Riemann hypothesis (RH for short), still unsettled at the time of the writing of this text. In his celebrated work \cite{Rie} B. Riemann conjectured that all complex zeros of the zeta-function lie on the critical line. This statement is called the Riemann hypothesis, and is probably the most famous open problem in Mathematics.

The RH implies (see, e.g., \cite{Iv1}, \cite{Tit3}) that
\begin{equation}\label{eq:rh_bound}
\zeta\left(\frac{1}{2} + it\right) \ll \exp\left(C \frac{\log t}{\log \log t}\right) \quad (C > 0).
\end{equation}

A slightly weaker bound than equation~\eqref{eq:rh_bound}, which in practice can often replace the RH, is the bound
\begin{equation}\label{eq:lindelof}
\zeta\left(\frac{1}{2} + it\right) \ll_{\varepsilon} (|t| + 1)^{\varepsilon},
\end{equation}
which is known as the Lindelöf hypothesis (LH for short). It is also unproved, and it is not known whether equation~\eqref{eq:lindelof} implies the RH, although this is not very likely. Namely the LH is equivalent (see, \cite{Tit3}, theorem 13.5), to the statement that, for every $\sigma > 1/2$,
\begin{equation}\label{eq:lh_equivalent}
N(\sigma, T+1) - N(\sigma, T) = o(\log T) \quad (T \to \infty),
\end{equation}
where $N(\sigma, T)$ denotes the number of complex zeros $\rho = \beta + i\gamma$ for which $\beta \geq \sigma$, $|\gamma| \leq T$. For the best unconditional bounds for $\zeta(\frac{1}{2} + it)$ and related topics, see the survey paper of M. N. Huxley and the present author \cite{HuIv}.

Hardy's original application of $Z(t)$ was to show that $\zeta(s)$ has infinitely many zeros on the critical line $\text{Re } s = 1/2$ (see, e.g., E. C. Titchmarsh \cite{Tit3}). This will be discussed in the next chapter. Later A. Selberg (see \cite{Sel} and \cite{Tit3}) obtained that a positive proportion of zeros of $\zeta(s)$ lies on the critical line. This can be stated as
\begin{equation}\label{eq:selberg}
N_0(T) \geq C N(T) \quad (C > 0, T \geq T_0),
\end{equation}
where $N_0(T)$ denotes the number of zeros of $Z(t)$ in $(0, T]$, or equivalently the number of complex zeros of $\zeta(s)$ on the critical line $\text{Re } s = 1/2$ whose imaginary parts lie in $(0, T]$. Selberg's bound~\eqref{eq:selberg} improves on $N_0(T) \geq CT$, which is a result of G. H. Hardy and J. E. Littlewood \cite{HaLi2} of 1921. Selberg's bound is one of the most important results of analytic number theory of all time, as it reveals the true order of magnitude of $N_0(T)$. Later work by various scholars led to explicit values of $C$ in equation~\eqref{eq:selberg}.

\section*{Notes}

It was actually the great Swiss mathematician Leonhard Euler (April 15, 1707-September 18, 1783) who first used the zeta-function, albeit only for real values of the variable. Besides equation~\eqref{eq:zeta_def}, Euler discovered several other identities from zeta-function theory. His paper \cite{Eul} from 1768 contains the assertion
\begin{equation}\label{eq:euler_identity}
\frac{1 - 2^{n-1} + 3^{n-1} - 4^{n-1} + 5^{n-1} - \cdots}{1 - 2^{-n} + 3^{-n} - 4^{-n} + 5^{-n} - \cdots} = -1 \times 2 \times 3 \times \cdots (n-1) \frac{(2^{n-1} - 1)}{(2^{n-1} - 1)\pi^n} \cos\left(\frac{\pi n}{2}\right),
\end{equation}
which he verified for $n = 1$ and $n = 2k$. E. Landau \cite{Lan} wrote Euler's identity as
\begin{equation}\label{eq:landau_euler}
\frac{\lim_{x \to 1} \sum_{n=1}^{\infty} (-1)^{n+1} n^{s-1} x^{n-1}}{\lim_{x \to 1} \sum_{n=1}^{\infty} (-1)^{n+1} n^{-s} x^{n-1}} = -\Gamma(s) \frac{(2^s - 1)}{(2^{s-1} - 1)\pi^s} \cos\left(\frac{\pi s}{2}\right),
\end{equation}
proved its validity, and showed its equivalence with the functional equation~\eqref{eq:func_eq}.

There are many ways to obtain the analytic continuation of $\zeta(s)$ outside the region $\sigma > 1$. One simple way (see T. Estermann \cite{Est2}) is to write, for $\sigma > 1$,
\begin{equation}\label{eq:estermann}
\zeta(s) = \sum_{n=1}^{\infty} n^{-s} = \sum_{n=1}^{\infty} \left( n^{-s} - \int_n^{n+1} u^{-s} du \right) + \frac{1}{s-1}
\end{equation}
and to observe that
\begin{equation}\label{eq:estermann_bound}
\left| n^{-s} - \int_n^{n+1} u^{-s} du \right| = \left| s \int_n^{n+1} \int_n^u z^{-s-1} dz du \right| \leq |s| n^{-\sigma-1}.
\end{equation}
Hence the second series above converges absolutely for $\sigma > 0$, and by the principle of analytic continuation
\begin{equation}\label{eq:analytic_cont}
\zeta(s) = \sum_{n=1}^{\infty} \left( n^{-s} - \int_n^{n+1} u^{-s} du \right) + \frac{1}{s-1} \quad (\sigma > 0)
\end{equation}
is obtained, showing incidentally that $\zeta(s)$ is regular for $\sigma > 0$, except for a simple pole at $s = 1$ with residue equal to 1.

Analytic continuation of $\zeta(s)$ for $\sigma > 0$ is also given by
\begin{equation}\label{eq:dirichlet_eta}
\zeta(s) = (1 - 2^{1-s})^{-1} \sum_{n=1}^{\infty} (-1)^n n^{-s}
\end{equation}
since the series in equation~\eqref{eq:dirichlet_eta} converges for $\sigma > 0$.

For $x > 1$ one has
\begin{align}\label{eq:partial_sum}
\sum_{n \leq x} n^{-s} &= \int_{1-0}^x u^{-s} d[u] = [x] x^{-s} + s \int_1^x [u] u^{-s-1} du \\
&= O(x^{1-\sigma}) + s \int_1^x ([u] - u) u^{-s-1} du + \frac{s}{s-1} - s \frac{x^{1-s}}{s-1}. \nonumber
\end{align}
If $\sigma > 1$ and $x \to \infty$, it follows that
\begin{equation}\label{eq:integral_form}
\zeta(s) = \frac{s}{s-1} + s \int_1^{\infty} ([u] - u) u^{-s-1} du.
\end{equation}
By using the customary notation $\psi(x) = x - [x] - 1/2$, this relation can be written as
\begin{equation}\label{eq:psi_integral}
\zeta(s) = \frac{1}{s-1} + \frac{1}{2} - s \int_1^{\infty} \psi(u) u^{-s-1} du.
\end{equation}

Since $\int_y^{y+1} \psi(u) du = 0$ for any real $y$, integration by parts shows that equation~\eqref{eq:psi_integral} provides the analytic continuation of $\zeta(s)$ to the half-plane $\sigma > -1$, and in particular it shows that $\zeta(0) = -1/2$. It also follows that the Laurent expansion of $\zeta(s)$ at its pole $s = 1$ has the form
\begin{equation}\label{eq:laurent_stieltjes}
\zeta(s) = \frac{1}{s-1} + \gamma_0 + \gamma_1(s-1) + \gamma_2(s-1)^2 + \cdots,
\end{equation}
where the so-called Stieltjes constants $\gamma_k$ are given by
\begin{align}\label{eq:stieltjes_integral}
\gamma_k &= \frac{(-1)^{k+1}}{k!} \int_{1-0}^{\infty} x^{-1} (\log x)^k d\psi(x) \\
&= \frac{(-1)^k}{k!} \lim_{N \to \infty} \left( \sum_{m \leq N} \frac{\log^k m}{m} - \frac{\log^{k+1} N}{k+1} \right), \nonumber
\end{align}
as already mentioned in the text. The formula~\eqref{eq:stieltjes_integral} was proved first by T. J. Stieltjes \cite{Sti} in 1905 (Thomas Joannes Stieltjes, December 29, 1856-December 31, 1894, Dutch mathematician). On successive integrations by parts of the integral in equation~\eqref{eq:psi_integral} one can obtain the analytic continuation of $\zeta(s)$ to $\mathbb{C}$.

A detailed discussion of the $\gamma_k$s is given by I. M. Israilov \cite{Isr1}, \cite{Isr2}. The first three values, to five decimal places, are
\begin{align}\label{eq:stieltjes_values}
\gamma_1 &= 0.07281\ldots, \\
\gamma_2 &= -0.00485\ldots, \nonumber \\
\gamma_3 &= -0.00034\ldots. \nonumber
\end{align}
Not much is known about the properties of the Stieltjes constants $\gamma_k$. It is widely believed that they are all irrational, but this has not been proved even for $\gamma = \gamma_0$ (Euler's constant).

A number $n$ is squarefree if $n = 1$ or $n = p_1 \ldots p_r$, where $p_1, \ldots, p_r$ are different primes. A number $n$ is squarefull if $n = 1$ or $n = p_1^{\alpha_1} \ldots p_r^{\alpha_r}$, where $\alpha_1 \geq 2, \ldots, \alpha_r \geq 2$. Various generating functions involving $\zeta(s)$ are thoroughly discussed in chapter 1 of \cite{Iv1}, and various divisor problems are discussed in chapters 13 and 14.

The equivalence of the definitions~\eqref{eq:gamma_func} and~\eqref{eq:gamma_product} for $\Gamma(s)$ is standard. For example, K. Ramachandra \cite{Ram} in the appendix starts from equation~\eqref{eq:gamma_func} and derives several properties of $\Gamma(s)$, including equation~\eqref{eq:gamma_product} and Stirling's formula~\eqref{eq:stirling}. On the other hand, Karatsuba–Voronin \cite{KaVo} (also Montgomery–Vaughan \cite{MoVa}, appendix C) in the appendix start from equation~\eqref{eq:gamma_product} and show that equation~\eqref{eq:gamma_func} holds. These works contain all the facts about $\Gamma(s)$ needed in this text.

There are many proofs in the literature of the fundamental functional equation~\eqref{eq:func_eq} for $\zeta(s)$. For example, E. C. Titchmarsh \cite{Tit3} in chapter 2 of his well-known monograph presents seven different proofs of equation~\eqref{eq:func_eq}.

A quick proof of equation~\eqref{eq:gaussian_int} is as follows (see (A.38) of \cite{Iv1}). By the principle of analytic continuation it suffices to prove equation~\eqref{eq:gaussian_int} for $B$ real and positive, when the change of variable
\begin{equation}\label{eq:gaussian_substitution}
t = \frac{A}{2B} + \frac{x}{\sqrt{B}}
\end{equation}
gives
\begin{equation}\label{eq:gaussian_proof}
\int_{-\infty}^{\infty} \exp(At - Bt^2) dt = B^{-1/2} \exp(A^2/(4B)) \int_{-\infty}^{\infty} e^{-x^2} dx = (\pi/B)^{1/2} \exp(A^2/(4B)).
\end{equation}

Recall that the Bernoulli numbers $B_k$ are defined by the series expansion
\begin{equation}\label{eq:bernoulli_def}
\frac{z}{e^z - 1} = \sum_{k=0}^{\infty} B_k \frac{z^k}{k!} \quad (|z| < 2\pi),
\end{equation}
so that $B_0 = 1$, $B_1 = -1/2$, $B_2 = 1/6$, $B_4 = -1/30$, $B_6 = 1/42$ etc., and $B_{2k+1} = 0$ for $k \geq 1$. A classical formula (see \cite{Iv1}, theorem 1.4 for a proof) is that
\begin{equation}\label{eq:zeta_even}
\zeta(2k) = (-1)^{k+1} \frac{(2\pi)^{2k} B_{2k}}{2(2k)!} \quad (k \in \mathbb{N}),
\end{equation}
so that in particular
\begin{align}\label{eq:zeta_examples}
\zeta(2) &= \sum_{n=1}^{\infty} n^{-2} = \frac{\pi^2}{6}, \\
\zeta(4) &= \sum_{n=1}^{\infty} n^{-4} = \frac{\pi^4}{90}. \nonumber
\end{align}

Not much is known about the numbers $\zeta(2n+1)$, $n \in \mathbb{N}$. R. Apéry \cite{Ape} (Roger Apéry, November 14, 1916-December 18, 1994, a Greek-French mathematician) proved in 1978 that $\zeta(3)$ is irrational. This result has incited much subsequent research. For example, T. Rivoal \cite{Riv1} proved that infinitely many of the numbers $\zeta(2n+1)$ are irrational. In \cite{Riv2} he proved that one of the nine numbers $\zeta(2n+1)$ for $2 \leq n \leq 10$ is irrational. Rivoal's method was generalized by V. Zudilin \cite{Zud}, who obtained a number of irrationality results for $\zeta(2n+1)$.

Hardy's function $Z(t)$ was named after Godfrey Harold "G. H." Hardy FRS (February 7, 1877-December 1, 1947), one of the greatest mathematicians of his time. He is best known for his numerous achievements in number theory and mathematical analysis, often obtained in joint works with J. E. Littlewood (June 9, 1885-September 6, 1977). Starting in 1914, he was the mentor of the famous Indian mathematician Srinivasa Ramanujan (December 22, 1887-April 26, 1920), a relationship that has become celebrated. He was a lecturer at Cambridge from 1906, which he left in 1919 to take the Savilian Chair of Geometry at Oxford. He returned to Cambridge in 1931, where he was Sadleirian Professor until 1942. Besides writing numerous research articles of highest quality (these may be found in \cite{Har7}), he wrote several well-known books, such as \cite{Har4}, \cite{Har5} and \cite{Har6}. His essay from 1940 on the
aesthetics of mathematics, A Mathematician's Apology \cite{Har3}, is often considered as one of the best insights into the mind of a working mathematician written for the layman. His textbook An Introduction to the Theory of Numbers \cite{HaWr}, written jointly with E. M. Wright (Sir Edward Maitland Wright, February 13, 1906-February 2, 2005), is one of the best introductory texts on number theory ever written.

Stirling's formula for $\Gamma(s)$ exists in many forms. See equation~\eqref{eq:stirling} for a sharp version of this result. It is named after the Scottish mathematician James Stirling (May 1692-December 5, 1770).

The functional equation of $\zeta(s)$ in a certain sense characterizes it completely. This was established long ago by H. Hamburger \cite{Ham} (Hans Ludwig Hamburger, August 5, 1889, Berlin-August 14, 1956, a German mathematician), who proved the following result (see also chapter 2 of \cite{Cha}). Let $G$ be an integral function of finite order, $P$ a polynomial, and let the series
\begin{equation}\label{eq:hamburger_f}
f(s) = \frac{G(s)}{P(s)} = \sum_{n=1}^{\infty} a_n n^{-s}
\end{equation}
converge absolutely for $\sigma > 1$. If
\begin{equation}\label{eq:hamburger_functional}
\pi^{-s/2}\Gamma(s/2)f(s) = \pi^{-(1-s)/2}\Gamma((1-s)/2)g(s),
\end{equation}
where $g(1-s) = \sum_{n=1}^{\infty} b_n n^{-s}$ converges absolutely for $\sigma < -\alpha < 0$, then
\begin{equation}\label{eq:hamburger_result}
f(s) = a_1\zeta(s) = g(s).
\end{equation}

Ivan Matveevich Vinogradov (September 14, 1891-March 20, 1983) was a leading Soviet mathematician. He served as director of the Steklov Mathematical Institute for 49 years. For his method of exponential sums see \cite{Vin1} and \cite{Vin2}. For the sharpest value of the constant $C$ in equation~\eqref{eq:vinogradov} and related bounds, see the papers of K. Ford \cite{For1}, \cite{For2}.

An extensive account on the RH is to be found in the monograph of P. Borwein et al. \cite{BCRW}. There is much numerical evidence favoring the RH, and for the calculations involving the zeros of $\zeta(s)$ see e.g., the works of A. M. Odlyzko \cite{Odl1}, \cite{Odl2}, H. te Riele and J. van de Lune \cite{RiLu}, and van de Lune et al \cite{LRW}. An excellent account of the RH and LH is to be found in the classical book of E. C. Titchmarsh \cite{Tit3}. Edward Charles "Ted" Titchmarsh (June 1, 1899-January 18, 1963 Oxford) was a leading British mathematician. He was a student of G. H. Hardy, and is known for work in analytic number theory, Fourier analysis and other parts of mathematical analysis. He was Savilian Professor of Geometry at the University of Oxford from 1932 to 1963.

A function $f(s)$, regular over $\mathbb{C}$, is called an integral (or entire) function of finite order if
\begin{equation}\label{eq:finite_order}
|f(s)| \ll \exp(B|s|^A)
\end{equation}
for some constants $A$, $B$ ($\geq 0$) as $|s| \to \infty$. The order of $f(s)$ is the lower bound of $A$ for which equation~\eqref{eq:finite_order} holds. The study of integral functions of finite order was developed at the end of the nineteenth century by J. Hadamard, who showed that these functions can be written as an infinite product containing factors of the form $s - s_0$ corresponding to the zero $s_0$ of the function in question. The integral function often used in the theory of $\zeta(s)$ is $\xi(s)$, which is an integral function of order one (see equation~\eqref{eq:eta_xi_def}). To see this, note that from equation~\eqref{eq:psi_integral} there is
\begin{equation}\label{eq:xi_bound_prep}
\log |(1-s)\zeta(s)| \ll \log |s| + 1
\end{equation}
uniformly for $\sigma \geq \frac{1}{2}$, and by Stirling's formula it follows that
\begin{equation}\label{eq:xi_order_bound}
\log |\xi(s)| \ll |s|(\log |s| + 1)
\end{equation}
uniformly for $\sigma \geq \frac{1}{2}$. Since $\xi(1-s) = \xi(s)$, this bound holds uniformly for $\sigma \leq \frac{1}{2}$, too. Further, by Stirling's formula for real $s$ there is $\log \xi(s) \sim \frac{1}{2}s \log s$ as $s \to \infty$, which shows that the order of $\xi(s)$ is exactly unity.

The $\ll$-constant in equation~\eqref{eq:zero_count} was made explicit by G. Csordas et al. \cite{COSV}, who proved that
\begin{equation}\label{eq:explicit_zero_bound}
N(T+1) - N(T) \leq \log T \quad (T \geq 3 \times 10^8),
\end{equation}
which is an inequality that is easy to remember.

For an explicit value of $C$ in equation~\eqref{eq:rh_bound}, see K. Soundararajan \cite{Sou4}. He proved that one can take $C = 3/8$. A sharper result, also on the RH, was obtained by V. Chandee and K. Soundararajan \cite{ChSo}, namely
\begin{equation}\label{eq:chandee_sound}
\left|\zeta\left(\frac{1}{2} + it\right)\right| \ll \exp\left(\frac{\log 2}{2} \frac{\log t}{\log \log t}\left(1 + O\left(\frac{\log \log \log t}{\log \log t}\right)\right)\right).
\end{equation}

On the other hand, it is known that there are arbitrarily large values of $t$ for which one has unconditionally
\begin{equation}\label{eq:lower_bound_zeta}
\left|\zeta\left(\frac{1}{2} + it\right)\right| > \exp\left((1 + o(1))\sqrt{\frac{\log t}{\log \log t}}\right) \quad (t \to \infty),
\end{equation}
as shown by K. Soundararajan \cite{Sou3}.

The function $S(T)$ is relatively small, and one can make the bounds in equations~\eqref{eq:s_bound} and~\eqref{eq:s_rh_bound} explicit. Namely it was proved by T. S. Trudgian \cite{Tru3} that, for $T > e$,
\begin{equation}\label{eq:trudgian_s_bound}
|S(T)| \leq 1.998 + 0.17 \log T.
\end{equation}

This is an unconditional result. On the RH, K. Ramachandra and A. Sankaranarayanan \cite{RaSa2} showed that
\begin{equation}\label{eq:rh_s_bound}
|S(T)| \leq 1.2 \frac{\log T}{\log \log T} \quad (T > T_0).
\end{equation}

This was improved, again on the RH, by D. A. Goldston and S. M. Gonek \cite{GoGo}. They proved that, for $0 < h \leq \sqrt{t}$,
\begin{equation}\label{eq:goldston_gonek}
|S(t+h) - S(t)| \leq \left(\frac{1}{2} + o(1)\right) \frac{\log T}{\log \log T} \quad (T \to \infty),
\end{equation}
and deduced, via $\int_0^T S(t) dt \ll \log T$ (this is an unconditional result of J. E. Littlewood \cite{Lit}), that under the RH,
\begin{equation}\label{eq:rh_s_improved}
|S(T)| \leq \left(\frac{1}{2} + o(1)\right) \frac{\log T}{\log \log T} \quad (T \to \infty),
\end{equation}
and a bound analogous to equation~\eqref{eq:rh_s_improved} for $m(\frac{1}{2} + i\gamma)$, the multiplicity of the zero $\frac{1}{2} + i\gamma$ ($\gamma > 0$), also under the RH. Related results are to be found in the works of M. A. Korolev \cite{Kor1}, \cite{Kor2} and Karatsuba-Korolev \cite{KaKo}. For the moments of $S(t)$ one has the classical unconditional result of A. Selberg \cite{Sel} (Atle Selberg, June 14, 1917-August 6, 2007, Norwegian mathematician known for his work in analytic number theory, and in the theory of automorphic forms, in particular bringing them into relation with spectral theory) that, for fixed $k \in \mathbb{N}$,
\begin{equation}\label{eq:selberg_moments}
\int_0^T |S(t)|^{2k} dt = \frac{(2k)!}{k!(2\pi)^{2k}} T (\log \log T)^k + O(T (\log \log T)^{k-1/2}).
\end{equation}

This suggests that $S(t)/\sqrt{\log \log t}$ resembles a Gaussian random variable with mean 0 and variance $2\pi^2$. A. Ghosh \cite{Gho} established this in 1983.

Several omega results involving the functions $S(T)$ and $S_1(T) = \int_0^T S(t) dt$ are proved by K.-M. Tsang \cite{Tsa1}. As is well-known (see Chapter IX of \cite{Tit3}), these functions are closely related to the distribution of the imaginary parts of the zeros of $\zeta(s)$. A. Selberg \cite{Sel} has proved that unconditionally one has
\begin{align}\label{eq:selberg_omega}
S(T) &= \Omega_{\pm}((\log T)^{1/3}(\log \log T)^{-7/3}), \\
S_1(T) &= \Omega_{+}((\log T)^{1/2}(\log \log T)^{-4}), \nonumber \\
S_1(T) &= \Omega_{-}((\log T)^{1/3}(\log \log T)^{-10/3}). \nonumber
\end{align}

Tsang refined Selberg's arguments, based on the evaluation of high moments of $S(T)$ (see equation~\eqref{eq:selberg_moments}), and proved that
\begin{align}\label{eq:tsang_omega}
S(T) &= \Omega_{\pm}((\log T/ \log \log T)^{1/3}), \\
S_1(T) &= \Omega_{+}((\log T)^{1/2}(\log \log T)^{-9/4}), \nonumber \\
S_1(T) &= \Omega_{-}((\log T)^{1/3}(\log \log T)^{-4/3}), \nonumber
\end{align}
and if one assumes the Riemann hypothesis, then
\begin{equation}\label{eq:tsang_rh}
S_1(T) = \Omega_{\pm}((\log T)^{1/2}(\log \log T)^{-3/2}).
\end{equation}

It may be conjectured that both $S(T)$ and $S_1(T)$ are of the order $(\log T)^{1/2+o(1)}$ as $T \to \infty$, although it is known, for example, only that $S(T) = O(\log T)$ (and $O(\log T/ \log \log T)$ if the RH holds), so that there is still a considerable gap between $O$- and $\Omega$-results.

K.-M. Tsang \cite{Tsa2} has shown that unconditionally
\begin{equation}\label{eq:tsang_sup}
\left(\sup_{T < t \leq 2T} \log \left|\zeta\left(\frac{1}{2} + it\right)\right|\right) \left(\sup_{T < t \leq 2T} \pm S(t)\right) \gg \frac{\log T}{\log \log T},
\end{equation}
which means that the result holds once with $+S(t)$ and once with $-S(t)$. He also proved that
\begin{equation}\label{eq:tsang_s1}
S_1(t) = \int_0^t S(u) du = \Omega_{+}((\log t)^{1/2}(\log \log t)^{-3/2}).
\end{equation}

These results supplement that of H. L. Montgomery \cite{Mon2}; they are good when $\sigma > 1/2$ is fixed. Montgomery showed that, for $\frac{1}{2} < \sigma < 1$ fixed, one has
\begin{equation}\label{eq:montgomery_log}
\log |\zeta(\sigma + it)| = \Omega_{+}(\log^{1-\sigma} t(\log \log t)^{-\sigma}),
\end{equation}
and under the RH that
\begin{equation}\label{eq:montgomery_rh}
\log \left|\zeta\left(\frac{1}{2} + it\right)\right| = \Omega\left(\frac{1}{20} \sqrt{\frac{\log t}{\log \log t}}\right).
\end{equation}

K. Ramachandra and A. Sankaranarayanan \cite{RaSa1} used Montgomery's method, made optimal use of the parameter $\alpha$ in his proof and obtained a result which is perhaps the limit of Montgomery's method, containing explicit evaluation of the constant implied by the $\Omega$-symbol in equation~\eqref{eq:montgomery_log}. As to the true order of $\zeta(\frac{1}{2} + it)$, it is a deep open question, and one can only make guesses. For example, D. W. Farmer et al. \cite{FGH} conjecture that, as $T \to \infty$,
\begin{equation}\label{eq:farmer_conj1}
\max_{t \in [0,T]} |Z(t)| = \max_{t \in [0,T]} \left|\zeta\left(\frac{1}{2} + it\right)\right| = \exp\left((1 + o(1))\sqrt{\frac{1}{2} \log T \log \log T}\right)
\end{equation}
and
\begin{equation}\label{eq:farmer_conj2}
\limsup_{t \to \infty} \frac{S(t)}{\sqrt{\log t \log \log t}} = \frac{1}{\pi\sqrt{2}}.
\end{equation}

Their arguments are based on methods involving random matrix theory (see, for example, the books of G. W. Anderson et al. \cite{AGZ} and M.L. Mehta \cite{Meh} for an account on random matrices). In analytic number theory, the distribution of zeros of the Riemann zeta-function (and other L-functions) is often modeled by the distribution of eigenvalues of certain random matrices (see J. Keating \cite{Kea}). The connection was first discovered by Hugh Montgomery (see \cite{Mon1}) and Freeman J. Dyson. It is connected to the Hilbert-Pólya conjecture (David Hilbert, January 23, 1862-February 14, 1943, a German mathematician, and George Pólya, December 13, 1887-September 7, 1985, a Hungarian mathematician) that the imaginary parts of the zeros of $\zeta(s)$ correspond to the eigenvalues of an unbounded self-adjoint operator.

For a thorough discussion on upper bounds for $N(\sigma, T)$, see \cite{Iv1}, chapter 11.

Selberg's method for detecting zeros on the critical line is expounded in many texts, such as \cite{Tit3}, \cite{Iv1} and \cite{KaVo}. After Selberg's pioneering work, it was later used and refined by several mathematicians, most notably by N. Levinson \cite{Lev} (Norman Levinson, August 11, 1912-October 10, 1975, American mathematician, made major contributions in the study of Fourier transforms, complex analysis, non-linear differential equations, number theory, and signal processing). Levinson showed that more than a third of zeros lie on the critical line (i.e. $C = 1/3$ in equation~\eqref{eq:selberg}). A simplification of Levinson's method is to be found in the work \cite{CoGh2} of J. B. Conrey and A. Ghosh, and in the recent work \cite{You} of M. P. Young. The research was carried on by other mathematicians and, for example, J. B. Conrey \cite{Con2}, showed that at least two fifth of the zeros of $\zeta(s)$ are on the critical line. In \cite{CGG3}, Conrey et al. investigated the occurrence of simple zeros on the critical line. The starting point is the obvious observation that $\rho$ is a simple zero of $\zeta(s)$ if and only if $\zeta'(\rho) \neq 0$, and thus by Cauchy's inequality
\begin{equation}\label{eq:cauchy_simple}
\left|\sum_{0 < \gamma \leq T} B(\rho)\zeta'(\rho)\right|^2 \leq N_s(T) \sum_{0 < \gamma \leq T} |B(\rho)\zeta'(\rho)|^2,
\end{equation}
where $N_s(T)$ is the number of simple zeros $\rho = \beta + i\gamma$ of $\zeta(s)$ with ordinates in $(0, T]$, and $B(s)$ is any analytic function. To minimize the loss in applying Cauchy's inequality one wants to choose $B$ so that $B\zeta'$ is close to constant, and therefore $B$ is chosen as a suitable Dirichlet series mollifier for $\zeta'(s)$. The choice of the mollifier is incidentally the most complicated task in this problem. Assuming the RH, they show that $N_s(T)/N(T) \geq 19/27 = 0.703703\ldots$ as $T \to \infty$.

The latest development concerning Selberg's method involve the works of Bui et al. \cite{BCY}, and by S. Feng \cite{Fen}. The former proved that at least 41.05\% of the zeros of $\zeta(s)$ are on the critical line (i.e. one can take $C = 0.4105$ in equation~\eqref{eq:selberg}) and at least 40.58\% of the zeros of $\zeta(s)$ are simple and on the critical line. At about the same time Feng independently claimed that at least 41.73\% of the zeros of $\zeta(s)$ are on the critical line and at least 40.75\% of them are simple and on the critical line. Bui et al. found some mistakes in Feng's work and in the second version of his paper in the ArXiv (http://arxiv.org/abs/1003.0059), he removed the result on simple zeros, and now claimed only that $C = 0.4128$. Anyway, as it stands at the moment, Feng seems to be the record holder for the proportion of zeros, while Bui et al. are the ones for simple zeros.

So far it is not known whether the simplicity of zeros ($\zeta'(\rho) \neq 0$ if $\zeta(\rho) = 0$) and the RH imply one another – it could happen, as far as it is known, that both statements are true, both are false, or that one is true and one is false!

\begin{thebibliography}{99}
\bibitem{AGZ} G. W. Anderson, A. Guionnet and O. Zeitouni, \textit{An Introduction to Random Matrices}, Cambridge University Press, Cambridge, 2010.

\bibitem{Ape} R. Apéry, Irrationalité de $\zeta(2)$ et $\zeta(3)$. Astérisque, 61, 11-13 (1979).

\bibitem{BCRW} P. Borwein, S. Choi, B. Rooney and A. Weirathmueller (eds.), \textit{The Riemann Hypothesis: A Resource for the Afficionado and Virtuoso Alike}, CMS Books in Mathematics, Springer, New York, 2008.

\bibitem{BCY} H. M. Bui, J. B. Conrey and M. P. Young, More than 41\% of the zeros of the zeta function are on the critical line. \textit{Acta Arith.} 150, 35-64 (2011).

\bibitem{Cha} S. J. Chowla, \textit{The Riemann Hypothesis and Hilbert's Tenth Problem}, Gordon and Breach, New York, 1965.

\bibitem{ChSo} V. Chandee and K. Soundararajan, Bounding $|\zeta(1/2+it)|$ on the Riemann hypothesis. \textit{Bull. London Math. Soc.} 43, 243-250 (2011).

\bibitem{CGG3} J. B. Conrey, A. Ghosh and S. M. Gonek, Simple zeros of the Riemann zeta-function. \textit{Proc. London Math. Soc.} (3) 76, 497-522 (1998).

\bibitem{Con2} J. B. Conrey, More than two fifths of the zeros of the Riemann zeta function are on the critical line. \textit{J. Reine Angew. Math.} 399, 1-26 (1989).

\bibitem{CoGh2} J. B. Conrey and A. Ghosh, A simpler proof of Levinson's theorem. \textit{Math. Proc. Cambridge Philos. Soc.} 97, 385-395 (1985).

\bibitem{COSV} G. Csordas, A. M. Odlyzko, W. Smith and R. S. Varga, A new Lehmer pair of zeros and a new lower bound for the de Bruijn-Newman constant $\Lambda$. \textit{Electron. Trans. Numer. Anal.} 1, 104-111 (1993).

\bibitem{Est2} T. Estermann, \textit{Introduction to Modern Prime Number Theory}, Cambridge University Press, Cambridge, 1952.

\bibitem{Eul} L. Euler, Remarques sur un beau rapport entre les séries des puissances tant directes que réciproques. \textit{Mém. Acad. Sci. Berlin} 17, 83-106 (1768).

\bibitem{FGH} D. W. Farmer, S. M. Gonek and C. P. Hughes, The maximum size of L-functions. \textit{J. Reine Angew. Math.} 609, 215-236 (2007).

\bibitem{Fen} S. Feng, Zeros of the Riemann zeta function on the critical line. \textit{J. Number Theory} 132, 511-542 (2012).

\bibitem{For1} K. Ford, Vinogradov's integral and bounds for the Riemann zeta function. \textit{Proc. London Math. Soc.} (3) 85, 565-633 (2002).

\bibitem{For2} K. Ford, Zero-free regions for the Riemann zeta function. In \textit{Number Theory for the Millennium II} (Urbana, IL, 2000), pp. 25-56, A K Peters, Natick, MA, 2002.

\bibitem{Gho} A. Ghosh, The distribution of $\zeta(1/2+it)$. In \textit{Number Theory, Trace Formulas and Discrete Groups} (Oslo, 1987), pp. 374-383, Academic Press, Boston, MA, 1989.

\bibitem{GoGo} D. A. Goldston and S. M. Gonek, A note on S(t) and the zeros of the Riemann zeta-function. \textit{Bull. London Math. Soc.} 39, 482-486 (2007).

\bibitem{Ham} H. Hamburger, Über die Riemannsche Funktionalgleichung der $\zeta$-Funktion. \textit{Math. Z.} 10, 240-254 (1921); 11, 224-245 (1921); 13, 283-311 (1922).

\bibitem{Har3} G. H. Hardy, \textit{A Mathematician's Apology}, Cambridge University Press, Cambridge, 1940.

\bibitem{Har4} G. H. Hardy, \textit{Divergent Series}, Oxford University Press, London, 1949.

\bibitem{Har5} G. H. Hardy, \textit{Pure Mathematics}, 10th ed., Cambridge University Press, Cambridge, 1952.

\bibitem{Har6} G. H. Hardy, \textit{A Course of Pure Mathematics}, 10th ed., Cambridge University Press, Cambridge, 1952.

\bibitem{Har7} G. H. Hardy, \textit{Collected Papers of G. H. Hardy} (7 volumes), Oxford University Press, Oxford, 1966-1979.

\bibitem{HaLi2} G. H. Hardy and J. E. Littlewood, The zeros of Riemann's zeta-function on the critical line. \textit{Math. Z.} 10, 283-317 (1921).

\bibitem{Has} C. B. Haselgrove, Tables of the Riemann zeta function. \textit{Roy. Soc. Math. Tables} 6, Cambridge University Press, Cambridge, 1960.

\bibitem{HaWr} G. H. Hardy and E. M. Wright, \textit{An Introduction to the Theory of Numbers}, 6th ed., Oxford University Press, Oxford, 2008.

\bibitem{HuIv} M. N. Huxley and A. Ivić, Subconvexity for the Riemann zeta function and the divisor problem. \textit{Bull. Cl. Sci. Math. Nat. Sci. Math.} 32, 13-32 (2007).

\bibitem{Isr1} I. M. Israilov, On the Laurent expansion of the Riemann zeta function. \textit{Trudy Mat. Inst. Steklov.} 158, 98-104 (1981) [Russian]; English transl. in \textit{Proc. Steklov Inst. Math.} 158, 105-112 (1983).

\bibitem{Isr2} I. M. Israilov, The Stieltjes constants. \textit{Russian Math. Surveys} 40, 217-218 (1985).

\bibitem{Iv1} A. Ivić, \textit{The Riemann Zeta-Function: Theory and Applications}, 2nd ed., Dover Publications, Mineola, NY, 2003.

\bibitem{Iv4} A. Ivić, \textit{Mean Values of the Riemann Zeta Function}, Lectures on Mathematics ETH Zürich, Birkhäuser, Boston, 1991.

\bibitem{KaKo} A. A. Karatsuba and M. A. Korolev, The argument of the Riemann zeta function. \textit{Russian Math. Surveys} 60, 433-488 (2005).

\bibitem{KaVo} A. A. Karatsuba and S. M. Voronin, \textit{The Riemann Zeta-Function}, de Gruyter, Berlin, 1992.

\bibitem{Kea} J. Keating, Random matrix theory and $\zeta(1/2+it)$. \textit{Commun. Math. Phys.} 281, 499-522 (2008).

\bibitem{Kor1} M. A. Korolev, On the argument of the Riemann zeta function on the critical line. \textit{Izv. Ross. Akad. Nauk Ser. Mat.} 67, 21-60 (2003) [Russian]; English transl. in \textit{Izv. Math.} 67, 225-264 (2003).

\bibitem{Kor2} M. A. Korolev, Gram's law and the argument of the Riemann zeta function. \textit{Publ. Inst. Math. (Beograd) (N.S.)} 76, 11-23 (2004).

\bibitem{Lan} E. Landau, Über die Anzahl der Gitterpunkte in gewissen Bereichen. \textit{Gött. Nachr.} 1912, 687-771.

\bibitem{Lev} N. Levinson, More than one-third of zeros of Riemann's zeta-function are on $\sigma = 1/2$. \textit{Adv. Math.} 13, 383-436 (1974).

\bibitem{Lit} J. E. Littlewood, On the zeros of the Riemann zeta-function. \textit{Proc. Cambridge Philos. Soc.} 22, 295-318 (1924).

\bibitem{LRW} J. van de Lune, H. J. J. te Riele and D. T. Winter, On the zeros of the Riemann zeta function in the critical strip. IV. \textit{Math. Comp.} 46, 667-681 (1986).

\bibitem{Man} H. von Mangoldt, Zu Riemanns Abhandlung "Über die Anzahl der Primzahlen unter einer gegebenen Grösse". \textit{J. Reine Angew. Math.} 114, 255-305 (1895).

\bibitem{Meh} M. L. Mehta, \textit{Random Matrices}, 3rd ed., Pure and Applied Mathematics 142, Elsevier/Academic Press, Amsterdam, 2004.

\bibitem{Mon1} H. L. Montgomery, The pair correlation of zeros of the zeta function. \textit{Proc. Sympos. Pure Math.} 24, 181-193 (1973).

\bibitem{Mon2} H. L. Montgomery, Extreme values of the Riemann zeta function. \textit{Comment. Math. Helv.} 52, 511-518 (1977).

\bibitem{Mot1} Y. Motohashi, \textit{Spectral Theory of the Riemann Zeta-Function}, Cambridge University Press, Cambridge, 1997.

\bibitem{MoVa} H. L. Montgomery and R. C. Vaughan, \textit{Multiplicative Number Theory I: Classical Theory}, Cambridge Studies in Advanced Mathematics 97, Cambridge University Press, Cambridge, 2007.

\bibitem{Odl1} A. M. Odlyzko, On the distribution of spacings between zeros of the zeta function. \textit{Math. Comp.} 48, 273-308 (1987).

\bibitem{Odl2} A. M. Odlyzko, The $10^{20}$-th zero of the Riemann zeta function and 175 million of its neighbors. \textit{AT\&T Bell Lab. preprint} (1992).

\bibitem{Ram} K. Ramachandra, \textit{Lectures on Transcendental Numbers}, Ramanujan Institute, University of Madras, Madras, 1969.

\bibitem{RaSa1} K. Ramachandra and A. Sankaranarayanan, On some theorems of Littlewood and Selberg. I. \textit{J. Number Theory} 44, 281-291 (1993).

\bibitem{RaSa2} K. Ramachandra and A. Sankaranarayanan, Notes on the Riemann zeta-function. II. \textit{Acta Arith.} 75, 1-22 (1996).

\bibitem{Rie} B. Riemann, Über die Anzahl der Primzahlen unter einer gegebenen Grösse. \textit{Ber. Königl. Preuss. Akad. Wiss. Berlin} 1859, 671-680.

\bibitem{RiLu} H. te Riele and J. van de Lune, Rigorous high speed separation of zeros of Riemann's zeta function. \textit{Analytic Number Theory} (Allerton Park, IL, 1995), pp. 183-197, Progr. Math. 138, Birkhäuser, Boston, MA, 1996.

\bibitem{Riv1} T. Rivoal, La fonction zêta de Riemann prend une infinité de valeurs irrationnelles aux entiers impairs. \textit{C. R. Acad. Sci. Paris Sér. I Math.} 331, 267-270 (2000).

\bibitem{Riv2} T. Rivoal, At least one of the nine numbers $\zeta(5), \zeta(7), \ldots, \zeta(21)$ is irrational. \textit{Acta Arith.} 101, 305-322 (2002).

\bibitem{Sel} A. Selberg, On the zeros of Riemann's zeta-function. \textit{Skr. Norske Vid. Akad. Oslo} 10, 1-59 (1942).

\bibitem{Sou3} K. Soundararajan, Extreme values of zeta and L-functions. \textit{Math. Ann.} 342, 467-486 (2008).

\bibitem{Sou4} K. Soundararajan, Weak subconvexity for central values of L-functions. \textit{Ann. of Math.} (2) 172, 1469-1498 (2010).

\bibitem{Sti} T. J. Stieltjes, Recherches sur les fractions continues. \textit{Ann. Fac. Sci. Toulouse} 8, J76-J122; 9, A5-A47 (1894-1895).

\bibitem{Tit3} E. C. Titchmarsh, \textit{The Theory of the Riemann Zeta-Function}, 2nd ed., revised by D. R. Heath-Brown, Oxford University Press, New York, 1986.

\bibitem{Tru3} T. S. Trudgian, On the success and failure of Gram's law for the Riemann zeta function. \textit{Acta Arith.} 125, 225-256 (2006).

\bibitem{Tsa1} K.-M. Tsang, The distribution of the values of the Riemann zeta function. PhD thesis, Princeton University, 1984.

\bibitem{Tsa2} K.-M. Tsang, Higher-power moments of $\Delta(x)$, $E(t)$ and $P(x)$. \textit{J. London Math. Soc.} (2) 65, 65-84 (2002).

\bibitem{Vin1} I. M. Vinogradov, \textit{The Method of Trigonometrical Sums in the Theory of Numbers}, translated from the Russian, revised and annotated by K. F. Roth and A. Davenport. Interscience Publishers, London-New York, 1954.

\bibitem{Vin2} I. M. Vinogradov, \textit{Special Variants of the Method of Trigonometric Sums}, translated from the Russian by K. F. Roth and A. Davenport, Pergamon Press, Oxford-London-New York-Paris, 1982.

\bibitem{You} M. P. Young, The first moment of L-functions of primitive Dirichlet characters with a large modulus. \textit{J. Number Theory} 114, 100-119 (2005).

\bibitem{Zud} V. Zudilin, One of the numbers $\zeta(5), \zeta(7), \zeta(9), \zeta(11)$ is irrational. \textit{Russian Math. Surveys} 56, 774-776 (2001).

\end{thebibliography}

\end{document}
