\documentclass[12pt]{article}
\usepackage{amsmath,amsthm,amssymb,amsfonts}
\usepackage{enumitem}
\usepackage{hyperref}
\usepackage{geometry}
\geometry{margin=1in}
\usepackage{setspace}
\setstretch{1.1}

% Theorem environments
\newtheorem{theorem}{Theorem}[section]
\newtheorem{lemma}[theorem]{Lemma}
\newtheorem{corollary}[theorem]{Corollary}
\theoremstyle{definition}
\newtheorem{assumption}[theorem]{Assumption}
\newtheorem{remark}[theorem]{Remark}

\title{The Normality of Time-Invariant, Subordinative Operators in a Hilbert Space}
\author{P. Masani\thanks{This work was supported by the Office of Naval Research, and was begun at the Bell Telephone Laboratories, Murray Hill, New Jersey.}}
\date{}

\begin{document}

\maketitle

\section{Introduction}

Let $(U_t,\, t\in\mathbb{R})$ be a strongly continuous group of unitary operators on a (complex) Hilbert space $X$ onto $H$, and let $E$ be its spectral measure on the family $\mathcal{B}$ of Borel subsets of the real number field $\mathbb{R}$, so that
\begin{equation}
U_t = \int e^{it\lambda} E(d\lambda)
\label{eq:stone}
\end{equation}
by Stone's Theorem~\cite{stone1951}. Let $S$ be the cyclic subspace generated by $x$ under the action of the $U_t$, $t\in\mathbb{R}$, or, equivalently, under that of $E(B)$, $B\in\mathcal{B}$. Let $L$ be the cyclic projection of $x$, i.e., the projection on $C$ with range $S_x$. Using a term due to Kolmogorov~\cite{kolmogorov1941}, we shall say that $x$ is subordinate to $y$ if $S_x \subset S_y$.

Our purpose is to assert the following theorem and deduce some corollaries which generalize known results:

\begin{theorem}\label{thm:main}
Given a strongly continuous unitary group $(U_t,\, t\in\mathbb{R})$ with spectral measure $E$ on the family $\mathcal{B}$ of Borel subsets of $\mathbb{R}$, let $T$ be any (single-valued, unbounded) linear operator from $X$ to $C$ such that
\begin{enumerate}[label=(\roman*)]
    \item \label{itm:timeinv} $U_t T = T U_t$ for all $t\in\mathbb{R}$ (``time-invariant''),
    \item \label{itm:subord} $T(x) \in S_x$ for all $x\in D_r$ (``subordinative''),
    \item \label{itm:closedsep} $T$ is closed and $H_0 = \overline{D_r}$ is separable,
\end{enumerate}
then there exists a complex-valued Borel-measurable function $\varphi$ on $\mathbb{R}$ such that
\begin{equation}
T = \int \varphi(\lambda) E_0(d\lambda)
\label{eq:spectralint}
\end{equation}
where $E_0$ is the restriction of $E$ to $H_0$, and $\varphi$ is unique up to sets of zero $E_0$-measure.
\end{theorem}

\section{Statistical Theory of Linear Filters}

Theorem~\ref{thm:main} has its genesis in the statistical theory of linear filters as conceived by N. Wiener. In this theory, the signals are realizations of strictly stationary stochastic processes (S.P.). It is assumed that these processes are governed by a single measure-preserving, ergodic flow over a probability space $(\Omega, \mathcal{B}, P)$. This flow induces the unitary group $(U_t,\, t\in\mathbb{R})$ on the Hilbert space $H = L^2(\Omega, \mathcal{B}, P)$. $T$ is the filter transformation; it converts the random function $f$ of the input S.P. $(U_t(f), t\in\mathbb{R})$ into the random function $g$ of the output S.P. $(U_t(g), t\in\mathbb{R})$. 

Condition~\ref{itm:timeinv} states that the flow of time does not affect the filter operation; thus it is the mathematical expression of what is termed \emph{time-invariance} in the engineering literature. Condition~\ref{itm:subord}, called the \emph{subordination property} of $T$ (cf. Section~\ref{sec:commutant}), subsumes what is referred to as the \emph{causal} or \emph{non-anticipative} property of the filter, viz., the filter response at any moment depends on inputs fed into it in the past alone. In particular, the response of a linear filter is a linear combination of past inputs or a limit thereof; thus
\begin{equation}
\forall x\in D_T, \quad T(x) \in S\big(U_t(x),\, t \leq 0\big) \subset S_x.
\label{eq:causal}
\end{equation}

In short, the requirements~\ref{itm:timeinv} and~\ref{itm:subord} are satisfied by all time-invariant, causal, linear filters. As for~\ref{itm:closedsep}, it is the natural assumption to make in order to get a decent mathematical theory.

In the engineering literature, non-rigorous methods are used to obtain a function $\varphi$ such that $G'(\lambda) = |\varphi(\lambda)|^2 F'(\lambda)$, where $F', G'$ are the spectral densities of the input and output S.P.'s. $\varphi$ is called the \emph{frequency-response function}. But to assert that such a $\varphi$ exists is tantamount to asserting that $T = \int \varphi(\lambda) E(d\lambda)$. Thus, Theorem~\ref{thm:main} provides a rigorous basis for the introduction of the frequency-response function into the theory. We thereby extend the work of Youla, Castriota, and Carlin~\cite{youla1959} on the rigorous development of the classical (nonstatistical) theory of passive filters.

\section{The Commutant; Logical Order of Results}
\label{sec:commutant}

From conditions~\ref{itm:timeinv} and~\ref{itm:subord}, one can show that $H_0$ reduces $E$ and that $T$ lives on $H_0$, i.e., $D_{U_r} \subset H_0$. Hence, without loss of generality, one can treat $H_0$ as the overall Hilbert space, and suppose that $U$ and $E(B)$ are defined merely on $H_0$. In short, we can take $H = H_0$. From here on, we therefore remove hypothesis~\ref{itm:closedsep} in favor of the following:

\begin{assumption}\label{ass:closedense}
$T$ is closed, $D_r$ is everywhere dense in $H$, and $H$ is separable.
\end{assumption}

Given Assumption~\ref{ass:closedense}, the well-known necessary and sufficient condition that $T = \int \varphi(\lambda) E(d\lambda)$ is in terms of the operator $H = E(d\lambda)$:
\begin{equation}
B \text{ is bounded and } BH \subset HB \implies BT \subset TB,
\label{eq:commutant}
\end{equation}
where $S(A)$ denotes the (closed) subspace spanned by $A$.

Now $U_t$ and $H$ commute (being spectral integrals with respect to the same $E$), therefore by~\eqref{eq:commutant}, $U_t T = T U_t$ for all $t\in\mathbb{R}$. Next, the cyclic subspaces reduce all spectral integrals, therefore $L_x H \subset H L_x$ for $x\in H$, whence condition~\ref{itm:subord} follows easily. Thus, the implication
\begin{equation}
\eqref{eq:commutant} \implies \text{conditions~\ref{itm:timeinv},~\ref{itm:subord}}
\label{eq:commutant_implies}
\end{equation}
is trivial, and the well-known theorem involving~\eqref{eq:commutant} follows easily from Theorem~\ref{thm:main}. To establish Theorem~\ref{thm:main}, one has to prove, in effect, the converse implication
\begin{equation}
\text{conditions~\ref{itm:timeinv},~\ref{itm:subord}} \implies \eqref{eq:commutant}.
\label{eq:converse}
\end{equation}
This is much harder, since in condition~\ref{itm:timeinv} $T$ is required to commute with a much smaller class than in~\eqref{eq:commutant}, and condition~\ref{itm:subord} is not a commutation relation at all.

Actually, the use of cyclic projections in the usual proof of the theorem involving the commutant condition~\eqref{eq:commutant} (cf.~\cite{riesz1959}, p.~351), renders the latter somewhat superfluous for our purposes. The most logical and economical order of proving propositions seems to be the following, in which the theorem involving~\eqref{eq:commutant} comes last:
\begin{enumerate}[label=(\arabic*)]
    \item\label{enum:LtTLt} Conditions~\ref{itm:timeinv},~\ref{itm:subord} $\implies$ for each $x\in M_r$, $L_x T \subset T L_x$,
    \item\label{enum:LxTLx} Conditions~\ref{itm:timeinv},~\ref{itm:subord} for each $x\in M_r$, $L_x T \subset T L_x$,
    \item\label{enum:LxTiff} For each $x\in M_r$, $L_x T \subset T L_x \iff T = \int \varphi(\lambda) E(d\lambda)$,
    \item\label{enum:commutantiff} $\{H\}' \subset \{T\}' \iff T = \int \varphi(\lambda) E(d\lambda)$.
\end{enumerate}

\section{Outline of the Proof}

The following three lemmas on cyclic subspaces and projections play an important role in our proof:

\begin{lemma}\label{lem:cyclicproj}
If $x \in S_\alpha$, then there exists $C_{x,\alpha} \in \mathcal{B}$ such that $L_x = E(C_{x,\alpha}) L_x$.
\end{lemma}

\begin{lemma}\label{lem:cyclicintersection}
The intersection of two cyclic subspaces is cyclic; in fact,
\begin{equation}
S_{L_{\alpha\beta}(\alpha)} = S_\alpha \cap S_\beta = S_{L_{\alpha\beta}(\beta)},
\label{eq:cyclicintersection}
\end{equation}
where $L_{\alpha\beta}$ is the projection onto $S_\alpha \cap S_\beta$.
\end{lemma}

\begin{lemma}\label{lem:cyclicsum}
For $\alpha, \beta \in C$, if $\beta' = \beta - L_{\alpha\beta}(\beta)$, then
\begin{align}
S_\alpha + S_\beta &= S_\alpha + S_{\beta'}, \label{eq:cyclicsum1}\\
S_\alpha \cap S_{\beta'} &= \{0\}. \label{eq:cyclicsum2}
\end{align}
\end{lemma}

We also need several lemmas concerning the operator $T$:

\begin{lemma}\label{lem:TLTL}
The condition $T L_x \subset L_x T$ for $x\in M_r$ (cf.~\ref{enum:LtTLt}) is equivalent to the condition
\begin{equation}
T(x) \in S_x \text{ for } x\in D_r \quad \text{and} \quad L_x(D_r) \subset D_r \text{ for } x\in M_r.
\label{eq:TLTL}
\end{equation}
\end{lemma}

\begin{lemma}\label{lem:UTTU}
For a closed operator $T$, $U_t T = T U_t$ for $t\in\mathbb{R}$ if and only if $E(B) T = T E(B)$ for $B\in\mathcal{B}$.
\end{lemma}

\begin{lemma}\label{lem:spectralDr}
If $T$ is closed, $U_t T = T U_t$ for $t\in\mathbb{R}$, and $\varphi$ is bounded, then
\begin{equation}
\int \varphi(\lambda) E(d\lambda) (D_r) \subset D_r.
\label{eq:spectralDr}
\end{equation}
\end{lemma}

\begin{lemma}\label{lem:LbetaDr}
If $T$ is as in Lemma~\ref{lem:spectralDr}, then
\begin{align}
L_\alpha(\beta) &\in D_r \quad \text{for } \beta \in D_r,\, \alpha \in S_\beta, \label{eq:LalphaDr}\\
L_{\alpha\beta}(\beta) &\in D_r \quad \text{for } \alpha \in H,\, \beta \in D_r. \label{eq:LalphabetaDr}
\end{align}
\end{lemma}

\begin{lemma}\label{lem:spectralRa}
If $T$ is closed and satisfies conditions~\ref{itm:timeinv},~\ref{itm:subord}, then, for each $\alpha \in D_r$, there exists a spectral integral $R_\alpha$ such that
\begin{equation}
R|_{D_T} = R|_{S_{\alpha, R_\alpha}}.
\label{eq:spectralRa}
\end{equation}
\end{lemma}

Finally, we need the following property of spectral integrals:

\begin{lemma}\label{lem:spectralcluster}
If $S = \int \varphi(\lambda) E(d\lambda)$ is any spectral integral, $a$ is any vector, and $[a] = \{x: x\in X \text{ and } S_x = S_a\}$, then $D_S[a]$ is an infinite set with cluster point $a$; in fact,
\begin{align}
a_n &= \int \min\{1, n/|\varphi(\lambda)|\} E(d\lambda) a \in D_S[a], \label{eq:anDs}\\
a_n &\to a. \label{eq:anconverge}
\end{align}
\end{lemma}

Space does not permit us to say more concerning these lemmas, nor even to indicate the main steps in the proofs of \ref{enum:LtTLt}--\ref{enum:LxTiff}. \ref{enum:LtTLt} and \ref{enum:LxTiff}, of course, yield Theorem~\ref{thm:main}, from which \ref{enum:commutantiff} follows at once as indicated in Section~\ref{sec:commutant}.

\section{Corollaries}

Let $\mathrm{mult}\, E = 1$. Then, taking an $\alpha$ in Lemma~\ref{lem:cyclicproj} for which $S_\alpha = C$, we see that every cyclic projection $L_x$ is a spectral projection $E(C)$. Hence, by Lemma~\ref{lem:UTTU} and Lemma~\ref{lem:TLTL}, every time-invariant operator is subordinative. In this case, Theorem~\ref{thm:main} reduces to the following:

\begin{corollary}\label{cor:mult1}
When $\mathrm{mult}\, E = 1$, every closed, time-invariant, linear operator with an everywhere dense domain is a spectral integral.
\end{corollary}

In the special case $H = L^2(-\infty, \infty)$, $\{U(f)\}(x) = f(t+1)$, this reduces to a result first proved by Bochner with the restriction (now seen to be unnecessary) that $T$ is bounded~\cite{bochner1929}.

Next, let $\mathrm{mult}\, E \geq \aleph_0$, but $T$ be bounded on $H$. Then trivially $L_x(D_r) \subset D_r$, $x \in H$. Hence, by Lemma~\ref{lem:TLTL} and \ref{enum:LxTiff}, every bounded subordinative operator is time-invariant. In this case, Theorem~\ref{thm:main} yields the following:

\begin{corollary}\label{cor:boundedsub}
Every linear, subordinative operator, which is bounded on $C$, is a spectral integral.
\end{corollary}

When $\mathrm{mult}\, E > 1$, Corollary~\ref{cor:mult1} fails even for a bounded $T$: just take $T = L_\alpha$, where $L_\alpha$ does not commute with some $L_\beta$. Next, when $T$ is closed but unbounded, Corollary~\ref{cor:boundedsub} fails even when $\mathrm{mult}\, E = 1$. Take $T = iD$, where $D$ is the differentiation operator restricted to the absolutely continuous functions $f$ in $L^2(C)$, such that $f(1) = 0$, $C = \text{unit circle}$, and $U_t$ is translation: $\{U_t(f)\}(e^{i\theta}) = f\{e^{i(\theta + t)}\}$. $T$ is closed, subordinative and symmetric with an everywhere dense domain, but it is not self-adjoint and not therefore a spectral integral. These examples show that the hypotheses of Theorem~\ref{thm:main} are in a sense the best possible.

\section*{Acknowledgment}
The author would like to thank Professor J. Feldman for his useful suggestion that time-invariance might follow from the requirement $L_x T L_x$ (cf.~\ref{enum:LxTLx}). He would also like to thank Drs. I. W. Sandburg and V. E. Benes of the Bell Telephone Laboratories for stimulating conversations on the subject.

\begin{thebibliography}{9}

\bibitem{bochner1929}
S. Bochner, \emph{Ein Satz über lineare Operationen}, Math. Z. \textbf{29} (1929), 737--743.

\bibitem{kolmogorov1941}
A. N. Kolmogorov, \emph{Stationary sequences in Hilbert space}, Byull. Moskov. Gos. Univ. Mat. \textbf{2} (1941), no. 6. (Russian)

\bibitem{riesz1959}
F. Riesz and B. Sz.-Nagy, \emph{Functional analysis}, Ungar, New York, 1959.

\bibitem{stone1951}
M. H. Stone, \emph{On unbounded operators in Hilbert space}, J. Indian Math. Soc. \textbf{16} (1951), 155--192.

\bibitem{youla1959}
D. C. Youla, L. J. Castriota and H. J. Carlin, \emph{Bounded real scattering matrices and the foundations of linear passive network theory}, IRE Trans. Circuit Theory CT-6 (1959), 102--124.

\end{thebibliography}

\end{document}
