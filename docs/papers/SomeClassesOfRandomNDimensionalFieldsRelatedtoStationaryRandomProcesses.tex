\documentclass[12pt]{article}
\usepackage{amsmath, amsthm, amssymb, enumitem}
\usepackage[margin=0.5in]{geometry}

% Theorem and related environments
\newtheorem{theorem}{Theorem}[section]
\newtheorem{lemma}[theorem]{Lemma}
\newtheorem{definition}[theorem]{Definition}
\newtheorem{remark}[theorem]{Remark}

\title{Some Classes of Random Fields in n-Dimensional Space, Related to Stationary Random Processes}
\author{A.M. Yaglom\\(Translated by R.A. Silverman)}
\date{1957}

\begin{document}

\maketitle

\noindent\textit{Theory of Probability and Its Applications, 1957, Volume II, Number 3}

\section{Introduction}

In recent decades there has been a great development of the theory of random processes, i.e., of random functions of one real variable, usually regarded as the time. At present there is a whole series of special monographs and review articles devoted to this theory (see e.g. \cite{Levy1948, Doob1956, Blanc-Lapierre1953, Bartlett1955, Yaglom1952}). The theory of random functions of several variables has been much less studied; the present article is devoted to certain topics in this theory.

We consider complex random functions $\xi(\mathbf{x}) = \xi(x_1, \cdots, x_n)$ of $n$ real variables, i.e., of a point $\mathbf{x} = (x_1, \cdots, x_n)$ of the $n$-dimensional space $R_n$; we call such functions random \textit{fields} in $R_n$. Together with ordinary fields $\xi(\mathbf{x})$ we also consider generalized random fields (in the sense of Ito \cite{Ito1954}--Gelfand \cite{Gelfand1955}), i.e., random linear functionals $\xi(\varphi)$ with respect to $\varphi \in D$, where $D$ is the class of infinitely differentiable complex functions of $n$ variables, and each $\varphi \in D$ vanishes outside some compact set. In this study we confine ourselves to the correlation theory of random fields, i.e., random variables which are equal to each other with probability one are regarded as identical and we consider only the first and second moments

\begin{equation}\label{eq:ordinary-moments}
m(\mathbf{x}) = M\xi(\mathbf{x}), \quad B(\mathbf{x}_1, \mathbf{x}_2) = M\xi(\mathbf{x}_1)\overline{\xi(\mathbf{x}_2)}
\end{equation}

or

\begin{equation}\label{eq:generalized-moments}
m(\varphi) = M\xi(\varphi), \quad B(\varphi_1, \varphi_2) = M\xi(\varphi_1)\overline{\xi(\varphi_2)}
\end{equation}

of the random variables $\xi(\mathbf{x}), \mathbf{x} \in R_n$ or $\xi(\varphi), \varphi \in D$, respectively. In what follows the mathematical expectations \eqref{eq:ordinary-moments} and \eqref{eq:generalized-moments} will always be assumed to be finite; moreover, the fields considered will always be assumed to be continuous in the sense that

\begin{equation}\label{eq:continuity-ordinary}
\lim_{\mathbf{x}_1 \to \mathbf{x}} M|\xi(\mathbf{x}_1) - \xi(\mathbf{x})|^2 = 0
\end{equation}

or

\begin{equation}\label{eq:continuity-generalized}
\lim_{\varphi_1 \to \varphi} M|\xi(\varphi_1) - \xi(\varphi)|^2 = 0,
\end{equation}

respectively, where in \eqref{eq:continuity-generalized} $\varphi_1 \to \varphi$ is understood in the sense of the usual Schwartz topology \cite{Schwartz1950} in the space $D$ (i.e., $\lim \varphi_n = \varphi$ means that the functions $\varphi_n$ and $\varphi$ all vanish outside the same compact set and that as $n \to \infty$ all the partial derivatives of $\varphi_n$ approach the corresponding partial derivatives of the function $\varphi$ on this compact set). It is clear that using the formula

\begin{equation}\label{eq:field-representation}
\xi(\varphi) = \int_{R_n} \xi(\mathbf{x})\varphi(\mathbf{x}) \, d\mathbf{x},
\end{equation}

where $d\mathbf{x} = dx_1 \cdots dx_n$, we can associate a certain continuous generalized field $\xi(\varphi)$ with every continuous random field; in this sense ordinary random fields are a special case of generalized random fields. At the same time, the transition to the more general case of generalized random fields actually simplifies proofs in some cases; therefore, in what follows, we shall always consider generalized fields first, and then derive as consequences the corresponding results for ordinary fields.

In the applications, random fields $\xi(\mathbf{x})$ were encountered first in the statistical theory of turbulence; in keeping with the particularly important role played in this theory by the special cases of homogeneous turbulence and homogeneous isotropic turbulence, the first concern here was with the notions of homogeneous and homogeneous isotropic random fields, respectively, which generalize the notion of a stationary random process. In this regard, many important results concerning such fields $\xi(\mathbf{x})$ were pointed out in papers on turbulence theory, published in physical and geophysical journals (see \cite{Taylor1935, VonKarman1941, Obukhov1941}); however, these results usually lacked complete mathematical proofs. The further development of the statistical theory of turbulence, associated with the important concept of locally homogeneous and locally isotropic turbulence, introduced by A.N. Kolmogorov \cite{Kolmogorov1941a, Kolmogorov1941b}, likewise required the consideration of more general locally homogeneous and locally isotropic random fields, which generalize the notion of a process with stationary increments; in this connection, such fields are frequently mentioned in the literature (see e.g. \cite{Batchelor1947, Obukhov1951}), but it appears that until now they have hardly been studied mathematically. Thus, the purpose of this paper is to give a mathematical theory, which is as complete as possible, of all the classes of random fields just enumerated, a theory analogous to the well known theory of stationary random processes and processes with stationary increments.

While working on the formulation of this paper, the author became acquainted with the paper of Ito \cite{Ito1956}, devoted to the mathematical theory of a class of multidimensional generalized random fields, which contain as a special case all the fields studied in Section~4 below. However, Ito's results (accompanied only by a sketch of the proofs) encompass only a comparatively small part of the results of our Section~4.

\section{Homogeneous Random Fields}

We introduce the operator $\tau_{\mathbf{y}}$ of the space $D$ by the vector $\mathbf{y}$ which shifts the argument of the function

\begin{equation}\label{eq:shift-operator}
\tau_{\mathbf{y}}\varphi(\mathbf{x}) = \varphi(\mathbf{x} + \mathbf{y}).
\end{equation}

A random field $\xi(\varphi)$ (in general, a generalized random field) is called \textit{homogeneous} if the first and second moments of the field, the mean value and correlation functional $B(\varphi_{1},\varphi_{2})$, are invariant with respect to any shift $\tau_{\mathbf{y}}$ of their functional arguments:

\begin{equation}\label{eq:homogeneous-mean}
m(\varphi) = m(\tau_{\mathbf{y}}\varphi),
\end{equation}

\begin{equation}\label{eq:homogeneous-correlation}
B(\varphi_{1},\varphi_{2}) = B(\tau_{\mathbf{y}}\varphi_{1},\tau_{\mathbf{y}}\varphi_{2}).
\end{equation}

These conditions are obviously met, in particular, in the case where $\xi(\varphi)$ is given by equation \eqref{eq:field-representation}, and where $\xi(\mathbf{x})$ is an ordinary random field such that for any $\mathbf{y} \in R_{n}$ the mean value $m(\mathbf{x})$ and the correlation function $B(\mathbf{x}_{1},\mathbf{x}_{2})$ satisfy the relations

\begin{equation}\label{eq:ordinary-homogeneous-mean}
m(\mathbf{x}+\mathbf{y}) = m(\mathbf{x}),\; \text{i.e., } m(\mathbf{x}) = m = \text{const.},
\end{equation}

\begin{equation}\label{eq:ordinary-homogeneous-correlation}
B(\mathbf{x}_{1}+\mathbf{y},\mathbf{x}_{2}+\mathbf{y}) = B(\mathbf{x}_{1},\mathbf{x}_{2}),\; \text{i.e., } B(\mathbf{x}_{1},\mathbf{x}_{2}) = B(\mathbf{x}_{1}-\mathbf{x}_{2}).
\end{equation}

The conditions \eqref{eq:ordinary-homogeneous-mean} and \eqref{eq:ordinary-homogeneous-correlation} are the conditions for an ordinary (non-generalized) homogeneous field.

\begin{theorem}\label{thm:homogeneous-representation}
For any (generalized) homogeneous random field $\xi(\varphi)$ in $R_{n}$ the mean value $m(\varphi)$ and the correlation functional $B(\varphi_{1},\varphi_{2})$ can be represented in the form

\begin{equation}\label{eq:homogeneous-mean-form}
m(\varphi) = m \int_{R_{n}} \varphi(\mathbf{x}) \, d\mathbf{x},
\end{equation}

where $m$ is a constant,

\begin{equation}\label{eq:homogeneous-correlation-form}
B(\varphi_{1},\varphi_{2}) = \int_{P_{n}} \tilde{\varphi}_{1}(\boldsymbol{\lambda}) \overline{\tilde{\varphi}_{2}(\boldsymbol{\lambda})} \, F(d\boldsymbol{\lambda}),
\end{equation}

where

\begin{equation}\label{eq:fourier-transform}
\tilde{\varphi}(\boldsymbol{\lambda}) = \int_{R_{n}} e^{i\boldsymbol{\lambda} \cdot \mathbf{x}} \varphi(\mathbf{x}) \, d\mathbf{x}
\end{equation}

is the Fourier transform of the function $\varphi(\mathbf{x})$, and $F(S)$ is a measure on the $n$-dimensional space of ``wave vectors'' such that the inequality

\begin{equation}\label{eq:measure-condition}
\int_{P_{n}} \frac{F(d\boldsymbol{\lambda})}{(1+\lambda^{2})^{p}} < \infty,\quad \lambda = |\boldsymbol{\lambda}| = \sqrt{\lambda_{1}^{2} + \cdots + \lambda_{n}^{2}}
\end{equation}

is satisfied for some non-negative number $p$ (a ``slowly growing measure'').

Conversely, every linear functional of the form \eqref{eq:homogeneous-mean-form}, where $m$ is a constant, is the mean value of some generalized homogeneous random field, and every bilinear functional of the form \eqref{eq:homogeneous-correlation-form}, where $F(S)$ is a slowly growing measure in $P_{n}$, is the correlation functional of some such field.
\end{theorem}

We shall henceforth call the measure $F(S)$ the \textit{spectral measure} of the field.

\begin{proof}
We use the well known fact (see \cite{Schwartz1950, Gelfand1955}) that every linear functional $m(\varphi)$ on the space $D(T)$ of infinitely differentiable functions defined on a given finite $n$-dimensional parallelopiped $T$ (where $D(T)$ is regarded as a subspace of the space $D$) can be represented in the form

\begin{equation}\label{eq:linear-functional-representation}
m(\varphi) = \int_{R_{n}} g(\mathbf{x}) \frac{\partial^{N}\varphi(\mathbf{x})}{\partial x_{1}^{k_{1}} \cdots \partial x_{n}^{k_{n}}} \, d\mathbf{x},\quad N = k_{1} + \cdots + k_{n},
\end{equation}

where $g(\mathbf{x})$ is a continuous function of $\mathbf{x}$. But in this case it follows from \eqref{eq:homogeneous-mean} that the mean value $m(\varphi)$ of the generalized homogeneous random field can also be represented in the same way for all $\varphi \in D_{1}$, where here the function $g(\mathbf{x})$ must satisfy the condition

\begin{equation}\label{eq:homogeneity-condition-d1}
\int_{R_{n}} [g(\mathbf{x}+\mathbf{y}) - g(\mathbf{x})] \frac{\partial^{N} \varphi(\mathbf{x})}{\partial x_{1}^{k_{1}} \cdots \partial x_{n}^{k_{n}}} \, d\mathbf{x} = 0
\end{equation}

for any $\mathbf{y} \in R_{n}$ and $\varphi \in D_{1}$. The last equation shows that $g(\mathbf{x}+\mathbf{y}) - g(\mathbf{x})$ is a polynomial in $x_{1}, \cdots, x_{n}$ (with coefficients which can depend on $y_{1}, \cdots, y_{n}$) of degree no higher than $k_{j}$ in $x_{j}$, $j=1,\cdots,n$. Therefore $g(\mathbf{x})$ is a polynomial in $x_{1}, \cdots, x_{n}$ whose degree with respect to $x_{j}$ does not exceed $k_{j}+1$, and after some partial differentiations we obtain from \eqref{eq:linear-functional-representation} the formula

\begin{equation}\label{eq:locally-homogeneous-mean-form}
m(\varphi) = \sum_{j=1}^{n} m_{j} \int_{R_{n}} x_{j} \varphi(\mathbf{x}) \, d\mathbf{x},
\end{equation}

which is equivalent to \eqref{eq:locally-homogeneous-mean}.

To prove that every linear functional of the form \eqref{eq:locally-homogeneous-mean} is the mean value of some generalized locally homogeneous field, it is sufficient to consider the field $\xi(\varphi) = \mathbf{m} \cdot \nabla\tilde{\varphi}(\mathbf{0})$ which is obviously locally homogeneous.

We now turn to the proof of equation \eqref{eq:locally-homogeneous-correlation}. For this we use the fact that if $\xi(\varphi)$ is a locally homogeneous random field in $R_{n}$ and $\mathbf{r}$ is a fixed vector, then the linear functional

\begin{equation}\label{eq:increment-field-functional}
\xi_{\mathbf{r}}(\varphi) = \xi(\varphi - \tau_{\mathbf{r}}\varphi), \quad \varphi \in D, \; \varphi - \tau_{\mathbf{r}}\varphi \in D_{1},
\end{equation}

will be a homogeneous random field in $R_{n}$, depending on the vector parameter $\mathbf{r}$, where the $N$ random fields $\{\xi_{\mathbf{r}_{1}}(\varphi), \cdots, \xi_{\mathbf{r}_{N}}(\varphi)\}$ will constitute an $N$-dimensional homogeneous random field for any $\mathbf{r}_{1}, \cdots, \mathbf{r}_{N} \in R_{n}$. Therefore, by Theorem~\ref{thm:multidimensional-homogeneous}

\begin{equation}\label{eq:increment-correlation-proof}
M \xi_{\mathbf{r}_{1}}(\varphi_{1}) \overline{\xi_{\mathbf{r}_{2}}(\varphi_{2})} = \int_{P_{n}} \tilde{\varphi}_{1}(\boldsymbol{\lambda}) \overline{\tilde{\varphi}_{2}(\boldsymbol{\lambda})} \, F(d\boldsymbol{\lambda}; \mathbf{r}_{1},\mathbf{r}_{2}),
\end{equation}

where $F(S; \mathbf{r}_{1},\mathbf{r}_{2})$ is a complex function of the set $S$ of the space $P_{n}$, a function which is slowly growing in modulus, depends on the vectors $\mathbf{r}_{1},\mathbf{r}_{2}$, and is such that the inequality

\begin{equation}\label{eq:increment-positivity}
\sum_{k,l=1}^{N} F(S; \mathbf{r}_{k},\mathbf{r}_{l}) \alpha_{k} \bar{\alpha}_{l} \ge 0
\end{equation}

is valid for any choice of $S$, the positive integer $N$, the vectors $\mathbf{r}_{1}, \cdots, \mathbf{r}_{N}$ and the complex numbers $\alpha_{1}, \cdots, \alpha_{N}$ (so that in particular $F(S; \mathbf{r},\mathbf{r})$ is always a non-negative measure).

We observe now that, because of the linearity of $\xi_{\mathbf{r}}(\varphi)$ and the obvious equality $\tau_{\mathbf{r}_{1}'}\tau_{\mathbf{r}_{1}''} = \tau_{\mathbf{r}_{1}'+\mathbf{r}_{1}''}$, we have

\begin{equation}\label{eq:increment-additivity}
\xi_{\mathbf{r}_{1}'+\mathbf{r}_{1}''}(\varphi) = \xi_{\mathbf{r}_{1}'}(\varphi) + \xi_{\mathbf{r}_{1}'}(\tau_{\mathbf{r}_{1}''}\varphi) = \xi_{\mathbf{r}_{1}'}(\varphi) + \xi_{\mathbf{r}_{1}''}(\tau_{\mathbf{r}_{1}'}\varphi).
\end{equation}

Bearing in mind that the Fourier transforms of the functions $\varphi$ and $\tau_{\mathbf{r}}\varphi$ are connected by the relation $\widetilde{\tau_{\mathbf{r}}\varphi}(\boldsymbol{\lambda}) = e^{-i\boldsymbol{\lambda} \cdot \mathbf{r}} \tilde{\varphi}(\boldsymbol{\lambda})$, we immediately obtain from \eqref{eq:increment-additivity} and \eqref{eq:increment-correlation-proof}

\begin{equation}\label{eq:increment-correlation-relation}
\int_{P_{n}} \tilde{\varphi}_{1}(\boldsymbol{\lambda}) \overline{\tilde{\varphi}_{2}(\boldsymbol{\lambda})} \, (1-e^{-i\boldsymbol{\lambda} \cdot \mathbf{r}_{1}}) \, F(d\boldsymbol{\lambda}; \mathbf{r}_{1}',\mathbf{r}_{2}) = \int_{P_{n}} \tilde{\varphi}_{1}(\boldsymbol{\lambda}) \overline{\tilde{\varphi}_{2}(\boldsymbol{\lambda})} \, (1-e^{-i\boldsymbol{\lambda} \cdot \mathbf{r}_{1}'}) \, F(d\boldsymbol{\lambda}; \mathbf{r}_{1}'',\mathbf{r}_{2}).
\end{equation}

Since here $\tilde{\varphi}_{1}(\boldsymbol{\lambda})$ and $\tilde{\varphi}_{2}(\boldsymbol{\lambda})$ are Fourier transforms of two arbitrary functions of $D$, it follows from the equality of the middle and right hand terms of \eqref{eq:increment-correlation-relation} that

\begin{equation}\label{eq:increment-measure-relation}
\int_{S} (1-e^{-i\boldsymbol{\lambda} \cdot \mathbf{r}_{1}}) \, F(d\boldsymbol{\lambda}; \mathbf{r}_{1}',\mathbf{r}_{2}) = \int_{S} (1-e^{-i\boldsymbol{\lambda} \cdot \mathbf{r}_{1}'}) \, F(d\boldsymbol{\lambda}; \mathbf{r}_{1}'',\mathbf{r}_{2})
\end{equation}

for all $S$ for which $F(S; \mathbf{r}_{1}'+\mathbf{r}_{1}'',\mathbf{r}_{2})$ is defined.

Now applying equation \eqref{eq:increment-additivity} to the field $\xi_{\mathbf{r}_{2}}(\varphi_{2})$, we obtain similarly

\begin{equation}\label{eq:increment-measure-relation-2}
\int_{S} (1-e^{-i\boldsymbol{\lambda} \cdot \mathbf{r}_{1}})(1-e^{i\boldsymbol{\lambda} \cdot \mathbf{r}_{2}}) \, F(d\boldsymbol{\lambda}; \mathbf{r}_{1},\mathbf{r}_{2}) = \int_{S} (1-e^{-i\boldsymbol{\lambda} \cdot \mathbf{r}_{1}'})(1-e^{i\boldsymbol{\lambda} \cdot \mathbf{r}_{2}}) \, F(d\boldsymbol{\lambda}; \mathbf{r}_{1}'',\mathbf{r}_{2}).
\end{equation}

By now it is not hard to obtain equation \eqref{eq:locally-homogeneous-correlation}. Let $S$ be a set of $P_{n}^{-}$ and let

\begin{equation}\label{eq:spectral-measure-definition}
F(S) = \int_{S} \frac{F(d\boldsymbol{\lambda}; \mathbf{r}_{1},\mathbf{r}_{2})}{(1-e^{-i\boldsymbol{\lambda} \cdot \mathbf{r}_{1}})(1-e^{i\boldsymbol{\lambda} \cdot \mathbf{r}_{2}})}.
\end{equation}

According to \eqref{eq:increment-measure-relation-2}, the function $F(S)$ does not depend on the choice of the vectors $\mathbf{r}_{1},\mathbf{r}_{2}$, and consequently can be defined for any set $S$ of $P_{n}^{-}$ (since $S$ can always be decomposed into parts, for each of which the product $(1-e^{-i\boldsymbol{\lambda} \cdot \mathbf{r}_{1}})(1-e^{i\boldsymbol{\lambda} \cdot \mathbf{r}_{2}})$ does not vanish for any choice of $\mathbf{r}_{1},\mathbf{r}_{2}$). It follows from the non-negativity of $F(d\boldsymbol{\lambda}; \mathbf{r},\mathbf{r})$ that the function $F(S)$ is non-negative; as the set $S$ approaches infinity unboundedly or the point $\mathbf{0}$, the value of $F(S)$ will approach infinity, but only in such a way that the inequality \eqref{eq:locally-homogeneous-measure-condition} holds. Thus equation \eqref{eq:increment-correlation-proof} can be rewritten in the form

\begin{equation}\label{eq:increment-correlation-final}
M \xi_{\mathbf{r}_{1}}(\varphi_{1}) \overline{\xi_{\mathbf{r}_{2}}(\varphi_{2})} = \int_{P_{n}^{-}} \tilde{\varphi}_{1}(\boldsymbol{\lambda}) \overline{\tilde{\varphi}_{2}(\boldsymbol{\lambda})} \, (1-e^{-i\boldsymbol{\lambda} \cdot \mathbf{r}_{1}})(1-e^{i\boldsymbol{\lambda} \cdot \mathbf{r}_{2}}) \, F(d\boldsymbol{\lambda}) + \mathcal{A} \mathbf{r}_{1} \cdot \mathbf{r}_{2},
\end{equation}

where $F(S)$ is a measure in $P_{n}^{-}$ which meets the conditions stated in the formulation of Theorem~\ref{thm:locally-homogeneous-representation}, and $\mathcal{A}$ is a numerical function of the vectors $\mathbf{r}_{1}$ and $\mathbf{r}_{2}$.

Let us further specify the form of this last function. It easily follows from the equality of the first and middle terms of \eqref{eq:increment-correlation-relation} that the function $\mathcal{A}(\mathbf{r}_{1},\mathbf{r}_{2})$ depends linearly on $\mathbf{r}_{1}$; in just the same way it is shown that it depends linearly on $\mathbf{r}_{2}$. Thus

\begin{equation}\label{eq:linear-term}
\mathcal{A}(\mathbf{r}_{1},\mathbf{r}_{2}) = \mathcal{A} \mathbf{r}_{1} \cdot \mathbf{r}_{2},
\end{equation}

where as a consequence of inequality \eqref{eq:increment-positivity} the matrix $\mathcal{A}$ must be Hermitean non-negative.

If now we write

\begin{equation}\label{eq:difference-definition}
\varphi - \tau_{\mathbf{r}}\varphi = \varphi^{(\mathbf{r})},
\end{equation}

then equations \eqref{eq:increment-correlation-final} and \eqref{eq:linear-term} can be rewritten in the

\begin{equation}\label{eq:increment-correlation-difference}
M \xi(\varphi_{1}^{(\mathbf{r}_{1})}) \overline{\xi(\varphi_{2}^{(\mathbf{r}_{2})})} = \int_{P_{n}^{-}} \tilde{\varphi}_{1}^{(\mathbf{r}_{1})}(\boldsymbol{\lambda}) \overline{\tilde{\varphi}_{2}^{(\mathbf{r}_{2})}(\boldsymbol{\lambda})} \, F(d\boldsymbol{\lambda}) + \mathcal{A} \nabla\tilde{\varphi}_{1}^{(\mathbf{r}_{1})}(\mathbf{0}) \cdot \overline{\nabla\tilde{\varphi}_{2}^{(\mathbf{r}_{2})}(\mathbf{0})},
\end{equation}

which proves equation \eqref{eq:locally-homogeneous-correlation} for functions $\varphi_{1}$ and $\varphi_{2}$ of the form \eqref{eq:difference-definition}. It follows immediately from this that this formula is true also for the case where $\varphi_{1}$ and $\varphi_{2}$ are linear combinations or limits of linear combinations of functions of the form \eqref{eq:difference-definition}; but since functions of this form and their linear combinations are dense in $D_{1}$, this completely proves equation \eqref{eq:locally-homogeneous-correlation}.

The proof that every bilinear functional of the form \eqref{eq:locally-homogeneous-correlation} is the correlation functional of a locally homogeneous field can be carried out just like the corresponding proof for generalized stationary processes (see \cite{Ito1954}). Namely, if $\varphi_{1}, \cdots, \varphi_{N} \in D_{1}$ and $B(\varphi_{k},\varphi_{l})$ is given by equation \eqref{eq:locally-homogeneous-correlation}, then

\begin{equation}\label{eq:positive-definiteness}
\sum_{k,l=1}^{N} B(\varphi_{k},\varphi_{l}) \alpha_{k} \bar{\alpha}_{l} \ge 0
\end{equation}

for any complex numbers $\alpha_{1}, \cdots, \alpha_{N}$; therefore we can define a set of multidimensional normal distributions for the complex random variables $\xi(\varphi)$, $\varphi \in D_{1}$, such that $M\xi(\varphi_{1}) \overline{\xi(\varphi_{2})} = B(\varphi_{1},\varphi_{2})$ and, for example, $M\xi(\varphi) = 0$. Moreover, using the bilinearity of $B(\varphi_{1},\varphi_{2})$ it can immediately be verified that $M|c\xi(\varphi) - \xi(c\varphi)|^{2} = 0$ for any $\varphi \in D_{1}$ and any complex number $c$, and that $M|\xi(\varphi_{1}+\varphi_{2}) - \xi(\varphi_{1}) - \xi(\varphi_{2})|^{2} = 0$ for any $\varphi_{1},\varphi_{2} \in D_{1}$, and it can also be proved that the continuity condition \eqref{eq:continuity-generalized} holds. Consequently, $\xi(\varphi)$ is a generalized locally homogeneous field with correlation functional \eqref{eq:locally-homogeneous-correlation}, QED.

Another proof of the converse of Theorem~\ref{thm:locally-homogeneous-representation} can be given by using Theorem~\ref{thm:locally-homogeneous-spectral} on the spectral decomposition of the locally homogeneous field itself.
\end{proof}


\begin{theorem}\label{thm:multidimensional-homogeneous}
For any collection of $N$ generalized homogeneous random fields $\{\xi_{1}(\varphi), \cdots, \xi_{N}(\varphi)\}$ in $R_{n}$, the joint correlation functional can be represented in the form

\begin{equation}\label{eq:multidimensional-correlation}
M \xi_{k}(\varphi_{1}) \overline{\xi_{l}(\varphi_{2})} = \int_{P_{n}} \tilde{\varphi}_{1}(\boldsymbol{\lambda}) \overline{\tilde{\varphi}_{2}(\boldsymbol{\lambda})} \, F_{kl}(d\boldsymbol{\lambda}),
\end{equation}

where $F_{kl}(S)$ is a complex matrix-valued measure such that the matrix $[F_{kl}(S)]$ is non-negative Hermitean for each set $S$, and each $F_{kl}(S)$ satisfies the slow growth condition \eqref{eq:measure-condition}.
\end{theorem}

\begin{proof}
This follows directly from Theorem~\ref{thm:homogeneous-representation} applied to each field individually and using the positive definiteness condition for the joint correlation structure.
\end{proof}

\section{Locally Homogeneous Random Fields (Fields with Homogeneous Increments)}

By a generalized locally homogeneous random field (or, alternatively, a random field with homogeneous increments) we mean a random linear functional $\xi(\varphi)$, continuous in the sense of \eqref{eq:continuity-generalized}, defined on the set $D_{1}$ of functions $\varphi \in D$ satisfying the condition

\begin{equation}\label{eq:zero-mean-condition}
\int_{R_{n}} \varphi(\mathbf{x}) \, d\mathbf{x} = 0,
\end{equation}

and such that the mathematical expectations \eqref{eq:generalized-moments} satisfy the conditions \eqref{eq:homogeneous-mean} and \eqref{eq:homogeneous-correlation}, for any $\varphi, \varphi_{1}, \varphi_{2} \in D_{1}$ and any $\mathbf{y} \in R_{n}$.

We observe that if we assume that the functional $\xi(\varphi)$ is given by equation \eqref{eq:field-representation}, where $\xi(\mathbf{x})$ is a continuous random point function, then if we confine ourselves to functions $\varphi$ of the class $D_{1}$, we can only define the values of $\xi(\mathbf{x})$ up to an additive constant. A random field $\xi(\mathbf{x})$ which is defined only up to an additive constant can be regarded as a collection of random variables $\xi(\mathbf{x}) - \xi(\mathbf{0})$, or (which will be more convenient here) as a collection of quantities

\begin{equation}\label{eq:increment-field}
A_{\mathbf{r}}\xi(\mathbf{x}) = \xi(\mathbf{x}) - \xi(\mathbf{x} - \mathbf{r}),
\end{equation}

defined for all possible $\mathbf{x}, \mathbf{r} \in R_{n}$ (i.e., as a field in $R_{2n}$) and satisfying the condition

\begin{equation}\label{eq:increment-condition}
A_{\mathbf{r}_{1}+\mathbf{r}_{2}}\xi(\mathbf{x}) = A_{\mathbf{r}_{1}}\xi(\mathbf{x}) + A_{\mathbf{r}_{2}}(\mathbf{x} - \mathbf{r}_{1}).
\end{equation}

Now taking into consideration the conditions \eqref{eq:homogeneous-mean} and \eqref{eq:homogeneous-correlation}, we arrive at the following definition: a random field $\xi(\mathbf{x})$, defined only up to a constant, is called an ordinary (non-generalized) locally homogeneous field, if the mathematical expectations

\begin{equation}\label{eq:increment-mean}
M A_{\mathbf{y}}\xi(\mathbf{x}) = m(\mathbf{r}),
\end{equation}

and

\begin{equation}\label{eq:increment-correlation}
M A_{\mathbf{y}}\xi(\mathbf{x}+\mathbf{r}_{1}) \overline{A_{\mathbf{y}}\xi(\mathbf{x}+\mathbf{r}_{2})} = D(\mathbf{x}; \mathbf{r}_{1}, \mathbf{r}_{2})
\end{equation}

do not depend on $\mathbf{y}$ for all $\mathbf{x}, \mathbf{y}, \mathbf{r}, \mathbf{r}_{1}, \mathbf{r}_{2} \in R_{n}$. The mathematical expectations \eqref{eq:increment-mean} and \eqref{eq:increment-correlation} are clearly the mean value and the correlation functional of the field $A_{\mathbf{r}}\xi(\mathbf{x})$; alternatively, they are also called the mean value of the increment of the field $\xi(\mathbf{x})$ and the structure function of the field $\xi(\mathbf{x})$, respectively. We also note that in the case of a real field $A_{\mathbf{r}}\xi(\mathbf{x})$, using the identity

\begin{equation}\label{eq:structure-function-identity}
(a - b)(c - d) = \frac{1}{2}[(a - d)^{2} + (b - c)^{2} - (a - c)^{2} - (b - d)^{2}],
\end{equation}

we can always express $D(\mathbf{x}; \mathbf{r}_{1}, \mathbf{r}_{2})$ in terms of the simpler function of one variable

\begin{equation}\label{eq:simple-structure-function}
D(\mathbf{r}) = M|A_{\mathbf{r}}\xi(\mathbf{x})|^{2} = D(\mathbf{0}; \mathbf{r}, \mathbf{r}),
\end{equation}

also called the structure function of the field $\xi(\mathbf{x})$; it is even sufficient for the field $A_{\mathbf{r}}\xi(\mathbf{x})$ to be real only in the wide sense (see Remark~\ref{rem:real-locally-homogeneous} to Theorem~\ref{thm:locally-homogeneous-representation}).

The definition of an ordinary locally homogeneous field was first given in the work of A.N. Kolmogorov \cite{Kolmogorov1941b}, devoted to a study of the local properties of developed turbulence; since that time such fields are continually used in the statistical theory of turbulence (see e.g. \cite{Kolmogorov1941a, Batchelor1947, Obukhov1951}). However, such fields have not been considered so far in the mathematical literature, despite the fact that the corresponding concept for the case of one variable (namely, the concept of a random process with stationary increments) had already been studied in 1940 in the work of A.N. Kolmogorov \cite{Kolmogorov1940} (also \cite{Cramer1947, Doob1956, Yaglom1952}).

\begin{theorem}\label{thm:locally-homogeneous-representation}
Given any generalized locally homogeneous random field $\xi(\varphi)$ in $R_{n}$, the mean value $m(\varphi)$ and correlation functional $B(\varphi_{1},\varphi_{2})$ can be represented in the form

\begin{equation}\label{eq:locally-homogeneous-mean}
m(\varphi) = \mathbf{m} \cdot \nabla\tilde{\varphi}(\mathbf{0}),
\end{equation}

where $\mathbf{m} = (m_{1}, \ldots, m_{n})$ is a constant vector, $\nabla$ is the gradient operator,

\begin{equation}\label{eq:locally-homogeneous-correlation}
B(\varphi_{1},\varphi_{2}) = \int_{P_{n}^{-}} \tilde{\varphi}_{1}(\boldsymbol{\lambda}) \overline{\tilde{\varphi}_{2}(\boldsymbol{\lambda})} \, F(d\boldsymbol{\lambda}) + \mathcal{A} \nabla\tilde{\varphi}_{1}(\mathbf{0}) \cdot \overline{\nabla\tilde{\varphi}_{2}(\mathbf{0})},
\end{equation}

$P_{n}^{-}$ is the space $P_{n}$ of vectors $\boldsymbol{\lambda}$ minus the ``zero point'' $\mathbf{0}$, $\tilde{\varphi}(\boldsymbol{\lambda})$ is the Fourier transform of the function $\varphi(\mathbf{x})$, and finally $\mathcal{A}$ is a non-negative Hermitean matrix, $F(S)$ is a measure on the space $P_{n}^{-}$ such that

\begin{equation}\label{eq:locally-homogeneous-measure-condition}
\int_{P_{n}^{-}} \frac{F(d\boldsymbol{\lambda})}{(1+\lambda^{2})^{p+1}} < \infty
\end{equation}

for some non-negative $p$.

Conversely, every linear functional of the form \eqref{eq:locally-homogeneous-mean} is the mean value of some generalized locally homogeneous random field, and every bilinear functional of the form \eqref{eq:locally-homogeneous-correlation} is the correlation functional of some such field.
\end{theorem}

We shall call the measure $F(S)$ the \textit{spectral measure} of the locally homogeneous field $\xi(\varphi)$. In general this measure will be infinite. However, for all $F$-measurable interior sets of $P_{n}^{-}$ (i.e., for all bounded $F$-measurable sets $S$ which are a finite distance from the origin of coordinates $\mathbf{0}$) the values of $F(S)$ will always be finite; hereafter by sets $S$ of $P_{n}^{-}$ we shall always understand only such sets.

\begin{proof}
First of all we note that the usual proof that every linear functional $m(\varphi)$ on the space $D(T)$ of infinitely differentiable functions on a finite $n$-dimensional interval $T$ can be represented in the form \eqref{eq:linear-functional-representation} is completely applicable to functionals on the subspace $D_{1}(T)$ of infinitely differentiable functions on the same interval which satisfy the condition \eqref{eq:zero-mean-condition}. But in this case it follows from \eqref{eq:homogeneous-mean} that the functional $m(\varphi)$ can be represented in the same way for all $\varphi \in D_{1}$, where here the function $g(\mathbf{x})$ must satisfy the condition

\begin{equation}\label{eq:homogeneity-condition-d1-2}
\int_{R_{n}} [g(\mathbf{x}+\mathbf{y}) - g(\mathbf{x})] \frac{\partial^{N} \varphi(\mathbf{x})}{\partial x_{1}^{k_{1}} \cdots \partial x_{n}^{k_{n}}} \, d\mathbf{x} = 0
\end{equation}

for any $\mathbf{y} \in R_{n}$ and $\varphi \in D_{1}$. The last equation shows that $g(\mathbf{x}+\mathbf{y}) - g(\mathbf{x})$ is a polynomial in $x_{1}, \cdots, x_{n}$ (with coefficients which can depend on $y_{1}, \cdots, y_{n}$) of degree no higher than $k_{j}$ in $x_{j}$, $j=1,\cdots,n$. Therefore $g(\mathbf{x})$ is a polynomial in $x_{1}, \cdots, x_{n}$ whose degree with respect to $x_{j}$ does not exceed $k_{j}+1$, and after some partial differentiations we obtain from \eqref{eq:linear-functional-representation} the formula

\begin{equation}\label{eq:locally-homogeneous-mean-form-2}
m(\varphi) = \sum_{j=1}^{n} m_{j} \int_{R_{n}} x_{j} \varphi(\mathbf{x}) \, d\mathbf{x},
\end{equation}

which is equivalent to \eqref{eq:locally-homogeneous-mean}.

To prove that every linear functional of the form \eqref{eq:locally-homogeneous-mean} is the mean value of some generalized locally homogeneous field, it is sufficient to consider the field $\xi(\varphi) = \mathbf{m} \cdot \nabla\tilde{\varphi}(\mathbf{0})$ which is obviously locally homogeneous.

We now turn to the proof of equation \eqref{eq:locally-homogeneous-correlation}. For this we use the fact that if $\xi(\varphi)$ is a locally homogeneous random field in $R_{n}$ and $\mathbf{r}$ is a fixed vector, then the linear functional

\begin{equation}\label{eq:increment-field-functional-2}
\xi_{\mathbf{r}}(\varphi) = \xi(\varphi - \tau_{\mathbf{r}}\varphi), \quad \varphi \in D, \; \varphi - \tau_{\mathbf{r}}\varphi \in D_{1},
\end{equation}

will be a homogeneous random field in $R_{n}$, depending on the vector parameter $\mathbf{r}$, where the $N$ random fields $\{\xi_{\mathbf{r}_{1}}(\varphi), \cdots, \xi_{\mathbf{r}_{N}}(\varphi)\}$ will constitute an $N$-dimensional homogeneous random field for any $\mathbf{r}_{1}, \cdots, \mathbf{r}_{N} \in R_{n}$. Therefore, by Theorem~\ref{thm:multidimensional-homogeneous}

\begin{equation}\label{eq:increment-correlation-2}
M \xi_{\mathbf{r}_{1}}(\varphi_{1}) \overline{\xi_{\mathbf{r}_{2}}(\varphi_{2})} = \int_{P_{n}} \tilde{\varphi}_{1}(\boldsymbol{\lambda}) \overline{\tilde{\varphi}_{2}(\boldsymbol{\lambda})} \, F(d\boldsymbol{\lambda}; \mathbf{r}_{1},\mathbf{r}_{2}),
\end{equation}

where $F(S; \mathbf{r}_{1},\mathbf{r}_{2})$ is a complex function of the set $S$ of the space $P_{n}$, a function which is slowly growing in modulus, depends on the vectors $\mathbf{r}_{1},\mathbf{r}_{2}$, and is such that the inequality

\begin{equation}\label{eq:increment-positivity-2}
\sum_{k,l=1}^{N} F(S; \mathbf{r}_{k},\mathbf{r}_{l}) \alpha_{k} \bar{\alpha}_{l} \ge 0
\end{equation}

is valid for any choice of $S$, the positive integer $N$, the vectors $\mathbf{r}_{1}, \cdots, \mathbf{r}_{N}$ and the complex numbers $\alpha_{1}, \cdots, \alpha_{N}$ (so that in particular $F(S; \mathbf{r},\mathbf{r})$ is always a non-negative measure).

We observe now that, because of the linearity of $\xi_{\mathbf{r}}(\varphi)$ and the obvious equality $\tau_{\mathbf{r}_{1}'}\tau_{\mathbf{r}_{1}''} = \tau_{\mathbf{r}_{1}'+\mathbf{r}_{1}''}$, we have

\begin{equation}\label{eq:increment-additivity-2}
\xi_{\mathbf{r}_{1}'+\mathbf{r}_{1}''}(\varphi) = \xi_{\mathbf{r}_{1}'}(\varphi) + \xi_{\mathbf{r}_{1}'}(\tau_{\mathbf{r}_{1}''}\varphi) = \xi_{\mathbf{r}_{1}'}(\varphi) + \xi_{\mathbf{r}_{1}''}(\tau_{\mathbf{r}_{1}'}\varphi).
\end{equation}

Bearing in mind that the Fourier transforms of the functions $\varphi$ and $\tau_{\mathbf{r}}\varphi$ are connected by the relation $\widetilde{\tau_{\mathbf{r}}\varphi}(\boldsymbol{\lambda}) = e^{-i\boldsymbol{\lambda} \cdot \mathbf{r}} \tilde{\varphi}(\boldsymbol{\lambda})$, we immediately obtain from \eqref{eq:increment-additivity-2} and \eqref{eq:increment-correlation-2}

\begin{equation}\label{eq:increment-correlation-relation-2}
\int_{P_{n}} \tilde{\varphi}_{1}(\boldsymbol{\lambda}) \overline{\tilde{\varphi}_{2}(\boldsymbol{\lambda})} \, (1-e^{-i\boldsymbol{\lambda} \cdot \mathbf{r}_{1}}) \, F(d\boldsymbol{\lambda}; \mathbf{r}_{1}',\mathbf{r}_{2}) = \int_{P_{n}} \tilde{\varphi}_{1}(\boldsymbol{\lambda}) \overline{\tilde{\varphi}_{2}(\boldsymbol{\lambda})} \, (1-e^{-i\boldsymbol{\lambda} \cdot \mathbf{r}_{1}'}) \, F(d\boldsymbol{\lambda}; \mathbf{r}_{1}'',\mathbf{r}_{2}).
\end{equation}

Since here $\tilde{\varphi}_{1}(\boldsymbol{\lambda})$ and $\tilde{\varphi}_{2}(\boldsymbol{\lambda})$ are Fourier transforms of two arbitrary functions of $D$, it follows from the equality of the middle and right hand terms of \eqref{eq:increment-correlation-relation-2} that

\begin{equation}\label{eq:increment-measure-relation-3}
\int_{S} (1-e^{-i\boldsymbol{\lambda} \cdot \mathbf{r}_{1}}) \, F(d\boldsymbol{\lambda}; \mathbf{r}_{1}',\mathbf{r}_{2}) = \int_{S} (1-e^{-i\boldsymbol{\lambda} \cdot \mathbf{r}_{1}'}) \, F(d\boldsymbol{\lambda}; \mathbf{r}_{1}'',\mathbf{r}_{2})
\end{equation}

for all $S$ for which $F(S; \mathbf{r}_{1}'+\mathbf{r}_{1}'',\mathbf{r}_{2})$ is defined.

Now applying equation \eqref{eq:increment-additivity-2} to the field $\xi_{\mathbf{r}_{2}}(\varphi_{2})$, we obtain similarly

\begin{equation}\label{eq:increment-measure-relation-4}
\int_{S} (1-e^{-i\boldsymbol{\lambda} \cdot \mathbf{r}_{1}})(1-e^{i\boldsymbol{\lambda} \cdot \mathbf{r}_{2}}) \, F(d\boldsymbol{\lambda}; \mathbf{r}_{1},\mathbf{r}_{2}) = \int_{S} (1-e^{-i\boldsymbol{\lambda} \cdot \mathbf{r}_{1}'})(1-e^{i\boldsymbol{\lambda} \cdot \mathbf{r}_{2}}) \, F(d\boldsymbol{\lambda}; \mathbf{r}_{1}'',\mathbf{r}_{2}).
\end{equation}

By now it is not hard to obtain equation \eqref{eq:locally-homogeneous-correlation}. Let $S$ be a set of $P_{n}^{-}$ and let

\begin{equation}\label{eq:spectral-measure-definition-2}
F(S) = \int_{S} \frac{F(d\boldsymbol{\lambda}; \mathbf{r}_{1},\mathbf{r}_{2})}{(1-e^{-i\boldsymbol{\lambda} \cdot \mathbf{r}_{1}})(1-e^{i\boldsymbol{\lambda} \cdot \mathbf{r}_{2}})}.
\end{equation}

According to \eqref{eq:increment-measure-relation-4}, the function $F(S)$ does not depend on the choice of the vectors $\mathbf{r}_{1},\mathbf{r}_{2}$, and consequently can be defined for any set $S$ of $P_{n}^{-}$ (since $S$ can always be decomposed into parts, for each of which the product $(1-e^{-i\boldsymbol{\lambda} \cdot \mathbf{r}_{1}})(1-e^{i\boldsymbol{\lambda} \cdot \mathbf{r}_{2}})$ does not vanish for any choice of $\mathbf{r}_{1},\mathbf{r}_{2}$). It follows from the non-negativity of $F(d\boldsymbol{\lambda}; \mathbf{r},\mathbf{r})$ that the function $F(S)$ is non-negative; as the set $S$ approaches infinity unboundedly or the point $\mathbf{0}$, the value of $F(S)$ will approach infinity, but only in such a way that the inequality \eqref{eq:locally-homogeneous-measure-condition} holds. Thus equation \eqref{eq:increment-correlation-2} can be rewritten in the form

\begin{equation}\label{eq:increment-correlation-final-2}
M \xi_{\mathbf{r}_{1}}(\varphi_{1}) \overline{\xi_{\mathbf{r}_{2}}(\varphi_{2})} = \int_{P_{n}^{-}} \tilde{\varphi}_{1}(\boldsymbol{\lambda}) \overline{\tilde{\varphi}_{2}(\boldsymbol{\lambda})} \, (1-e^{-i\boldsymbol{\lambda} \cdot \mathbf{r}_{1}})(1-e^{i\boldsymbol{\lambda} \cdot \mathbf{r}_{2}}) \, F(d\boldsymbol{\lambda}) + \mathcal{A} \mathbf{r}_{1} \cdot \mathbf{r}_{2},
\end{equation}

where $F(S)$ is a measure in $P_{n}^{-}$ which meets the conditions stated in the formulation of Theorem~\ref{thm:locally-homogeneous-representation}, and $\mathcal{A}$ is a numerical function of the vectors $\mathbf{r}_{1}$ and $\mathbf{r}_{2}$.

Let us further specify the form of this last function. It easily follows from the equality of the first and middle terms of \eqref{eq:increment-correlation-relation-2} that the function $\mathcal{A}(\mathbf{r}_{1},\mathbf{r}_{2})$ depends linearly on $\mathbf{r}_{1}$; in just the same way it is shown that it depends linearly on $\mathbf{r}_{2}$. Thus

\begin{equation}\label{eq:linear-term-2}
\mathcal{A}(\mathbf{r}_{1},\mathbf{r}_{2}) = \mathcal{A} \mathbf{r}_{1} \cdot \mathbf{r}_{2},
\end{equation}

where as a consequence of inequality \eqref{eq:increment-positivity-2} the matrix $\mathcal{A}$ must be Hermitean non-negative.

If now we write

\begin{equation}\label{eq:difference-definition-2}
\varphi - \tau_{\mathbf{r}}\varphi = \varphi^{(\mathbf{r})},
\end{equation}

then equations \eqref{eq:increment-correlation-final-2} and \eqref{eq:linear-term-2} can be rewritten in the

\begin{equation}\label{eq:increment-correlation-difference-2}
M \xi(\varphi_{1}^{(\mathbf{r}_{1})}) \overline{\xi(\varphi_{2}^{(\mathbf{r}_{2})})} = \int_{P_{n}^{-}} \tilde{\varphi}_{1}^{(\mathbf{r}_{1})}(\boldsymbol{\lambda}) \overline{\tilde{\varphi}_{2}^{(\mathbf{r}_{2})}(\boldsymbol{\lambda})} \, F(d\boldsymbol{\lambda}) + \mathcal{A} \nabla\tilde{\varphi}_{1}^{(\mathbf{r}_{1})}(\mathbf{0}) \cdot \overline{\nabla\tilde{\varphi}_{2}^{(\mathbf{r}_{2})}(\mathbf{0})},
\end{equation}

which proves equation \eqref{eq:locally-homogeneous-correlation} for functions $\varphi_{1}$ and $\varphi_{2}$ of the form \eqref{eq:difference-definition-2}. It follows immediately from this that this formula is true also for the case where $\varphi_{1}$ and $\varphi_{2}$ are linear combinations or limits of linear combinations of functions of the form \eqref{eq:difference-definition-2}; but since functions of this form and their linear combinations are dense in $D_{1}$, this completely proves equation \eqref{eq:locally-homogeneous-correlation}.

The proof that every bilinear functional of the form \eqref{eq:locally-homogeneous-correlation} is the correlation functional of a locally homogeneous field can be carried out just like the corresponding proof for generalized stationary processes (see \cite{Ito1954}). Namely, if $\varphi_{1}, \cdots, \varphi_{N} \in D_{1}$ and $B(\varphi_{k},\varphi_{l})$ is given by equation \eqref{eq:locally-homogeneous-correlation}, then

\begin{equation}\label{eq:positive-definiteness-2}
\sum_{k,l=1}^{N} B(\varphi_{k},\varphi_{l}) \alpha_{k} \bar{\alpha}_{l} \ge 0
\end{equation}

for any complex numbers $\alpha_{1}, \cdots, \alpha_{N}$; therefore we can define a set of multidimensional normal distributions for the complex random variables $\xi(\varphi)$, $\varphi \in D_{1}$, such that $M\xi(\varphi_{1}) \overline{\xi(\varphi_{2})} = B(\varphi_{1},\varphi_{2})$ and, for example, $M\xi(\varphi) = 0$. Moreover, using the bilinearity of $B(\varphi_{1},\varphi_{2})$ it can immediately be verified that $M|c\xi(\varphi) - \xi(c\varphi)|^{2} = 0$ for any $\varphi \in D_{1}$ and any complex number $c$, and that $M|\xi(\varphi_{1}+\varphi_{2}) - \xi(\varphi_{1}) - \xi(\varphi_{2})|^{2} = 0$ for any $\varphi_{1},\varphi_{2} \in D_{1}$, and it can also be proved that the continuity condition \eqref{eq:continuity-generalized} holds. Consequently, $\xi(\varphi)$ is a generalized locally homogeneous field with correlation functional \eqref{eq:locally-homogeneous-correlation}, QED.

Another proof of the converse of Theorem~\ref{thm:locally-homogeneous-representation} can be given by using Theorem~\ref{thm:locally-homogeneous-spectral} on the spectral decomposition of the locally homogeneous field itself.
\end{proof}


\begin{theorem}\label{thm:locally-homogeneous-spectral}
Any generalized locally homogeneous random field $\xi(\varphi)$ in $R_{n}$ admits a spectral representation of the form

\begin{equation}\label{eq:spectral-representation}
\xi(\varphi) = \int_{P_{n}^{-}} \tilde{\varphi}(\boldsymbol{\lambda}) \, Z(d\boldsymbol{\lambda}) + \mathbf{m} \cdot \nabla\tilde{\varphi}(\mathbf{0}),
\end{equation}

where $Z(S)$ is a random measure with orthogonal increments such that $M|Z(S)|^{2} = F(S)$, and $\mathbf{m}$ is the constant vector from \eqref{eq:locally-homogeneous-mean}.
\end{theorem}

\begin{proof}
The proof follows from the standard construction of the stochastic integral with respect to the spectral measure, using the representation \eqref{eq:locally-homogeneous-correlation} and the theory of random measures with orthogonal increments.
\end{proof}

\begin{remark}\label{rem:real-locally-homogeneous}
In the case of real locally homogeneous random fields, it follows at once from the conditions $m(\bar{\varphi}) = \overline{m(\varphi)}$ and $B(\bar{\varphi}_{1},\bar{\varphi}_{2}) = \overline{B(\varphi_{1},\varphi_{2})}$ that the vector $\mathbf{m}$ must be real, the measure $F(S)$ such that $F(-S) = F(S)$ and the matrix $\mathcal{A}$ real, symmetric and non-negative. The stated conditions regarding $\mathbf{m}$, $F(S)$ and $\mathcal{A}$ are clearly necessary and sufficient for the equalities $m(\bar{\varphi}) = \overline{m(\varphi)}$ and $B(\bar{\varphi}_{1},\bar{\varphi}_{2}) = \overline{B(\varphi_{1},\varphi_{2})}$ to be true; in this case it is natural to call the field itself real in the wide sense.
\end{remark}

\begin{remark}\label{rem:matrix-vector-relation}
Obviously the matrix $\mathcal{A}$ is related to the vector $\mathbf{m}$ by the condition that the matrix $\mathcal{A} - \mathbf{m}\mathbf{m}^{*}$ be non-negative; for a given $\mathbf{m}$ the measure $F$ can be completely arbitrary.
\end{remark}

\begin{remark}\label{rem:ordinary-locally-homogeneous}
In the case of an ordinary (non-generalized) locally homogeneous field $\xi(\mathbf{x})$, the quantities $m(\varphi)$ and $B(\varphi_{1}^{(\mathbf{r}_{1})},\varphi_{2}^{(\mathbf{r}_{2})})$ (see \eqref{eq:difference-definition-2}) can clearly be represented in the following way in terms of the mean value $m(\mathbf{r})$ of the increment of the field $\xi(\mathbf{x})$ and the corresponding structure function $D(\mathbf{x};\mathbf{r}_{1},\mathbf{r}_{2})$:

\begin{equation}\label{eq:ordinary-locally-homogeneous-mean}
m(\varphi) = \int_{R_{n}} m(\mathbf{x}) \varphi(\mathbf{x}) \, d\mathbf{x},
\end{equation}

\begin{equation}\label{eq:ordinary-locally-homogeneous-correlation}
B(\varphi_{1}^{(\mathbf{r}_{1})},\varphi_{2}^{(\mathbf{r}_{2})}) = \int_{R_{n}} \int_{R_{n}} D(\mathbf{x}_{1}-\mathbf{x}_{2};\mathbf{r}_{1},\mathbf{r}_{2}) \varphi_{1}(\mathbf{x}_{1}) \overline{\varphi_{2}(\mathbf{x}_{2})} \, d\mathbf{x}_{1} d\mathbf{x}_{2}.
\end{equation}

Here equation \eqref{eq:locally-homogeneous-mean} means that

\begin{equation}\label{eq:linear-mean-increment}
m(\mathbf{r}) = \mathbf{m} \cdot \mathbf{r},
\end{equation}

and \eqref{eq:locally-homogeneous-correlation}, \eqref{eq:linear-term-2} that

\begin{equation}\label{eq:structure-function-spectral}
D(\mathbf{x};\mathbf{r}_{1},\mathbf{r}_{2}) = \int_{P_{n}^{-}} e^{i\boldsymbol{\lambda} \cdot \mathbf{x}} (1-e^{-i\boldsymbol{\lambda} \cdot \mathbf{r}_{1}})(1-e^{i\boldsymbol{\lambda} \cdot \mathbf{r}_{2}}) \, F(d\boldsymbol{\lambda}) + \mathcal{A}\mathbf{r}_{1} \cdot \mathbf{r}_{2},
\end{equation}

and in particular

\begin{equation}\label{eq:simple-structure-function-spectral}
D(\mathbf{r}) = D(\mathbf{0};\mathbf{r},\mathbf{r}) = 2 \int_{P_{n}^{-}} (1-\cos(\boldsymbol{\lambda} \cdot \mathbf{r})) \, F(d\boldsymbol{\lambda}) + \mathcal{A}\mathbf{r} \cdot \mathbf{r},
\end{equation}

(in the real case the function \eqref{eq:structure-function-spectral} can always be expressed in terms of \eqref{eq:simple-structure-function-spectral} by using the identity \eqref{eq:structure-function-identity}). It is not hard to show that for equations \eqref{eq:linear-mean-increment} and \eqref{eq:simple-structure-function-spectral} to make sense, the inequality \eqref{eq:locally-homogeneous-measure-condition} must hold for $p = 0$. In fact, it follows from \eqref{eq:simple-structure-function-spectral} that

\begin{equation}\label{eq:structure-function-sphere}
\int_{|\mathbf{r}|=1} D(\mathbf{r}) \, d\sigma(\mathbf{r}) = \int_{P_{n}^{-}} \Phi(\lambda) \frac{\lambda^{2} \, F(d\boldsymbol{\lambda})}{1+\lambda^{2}} + A_{0},
\end{equation}

($d\sigma(\mathbf{r})$ is an element of area of the sphere $|\mathbf{r}|=1$), where $\Phi(\lambda)$ is an everywhere positive function, continuous on the half-line $0 \le \lambda < \infty$, and approaching a positive limit as $\lambda \to \infty$ (this is the integral over the $(n-1)$-dimensional unit sphere of the function $2(1-\cos(\boldsymbol{\lambda} \cdot \mathbf{r}))(1+\lambda^{2})/\lambda^{2}$); $A_{0}$ is a non-negative constant (the integral of $\mathbf{r} \cdot \mathcal{A}\mathbf{r}$ over the same sphere). From this it follows at once that $\Phi(\lambda) > b > 0$ (where $b$ is some constant), which means that

\begin{equation}\label{eq:finite-structure-condition}
\int_{P_{n}^{-}} \frac{\lambda^{2} \, F(d\boldsymbol{\lambda})}{1+\lambda^{2}} < \infty,
\end{equation}

i.e., we can really put $p = 0$ in \eqref{eq:locally-homogeneous-measure-condition}. A little later (see Remark~\ref{rem:ordinary-from-generalized} to Theorem~\ref{thm:locally-homogeneous-spectral}) we shall show that conversely, if we can put $p = 0$ in \eqref{eq:locally-homogeneous-measure-condition}, then $\xi(\varphi)$ must be representable in the form \eqref{eq:field-representation}, where $\xi(\mathbf{x})$ (here instead of $\xi(\mathbf{x})$ we can clearly write $A_{\mathbf{r}}\xi(\mathbf{x})$, where $\mathbf{r}$ is arbitrary) is an ordinary (non-generalized) locally homogeneous field, with the structure function \eqref{eq:simple-structure-function-spectral}.
\end{remark}

% Bibliography

\begin{remark}\label{rem:ordinary-from-generalized}
If the spectral measure $F(S)$ in \eqref{eq:locally-homogeneous-correlation} satisfies condition \eqref{eq:locally-homogeneous-measure-condition} with $p = 0$, then the generalized locally homogeneous field $\xi(\varphi)$ can be represented in the form \eqref{eq:field-representation}, where $\xi(\mathbf{x})$ is an ordinary (non-generalized) locally homogeneous field. This follows from the fact that the integral \eqref{eq:spectral-representation} converges for ordinary test functions when $p = 0$, allowing pointwise evaluation of the field.
\end{remark}


\begin{thebibliography}{99}

\bibitem{Levy1948} P. Lévy, \textit{Processus Stochastiques et Mouvement Brownien}, Paris, 1948.

\bibitem{Doob1956} J. Doob, \textit{Stochastic Processes}, Moscow, 1956 (Translated to Russian).

\bibitem{Blanc-Lapierre1953} A. Blanc-Lapierre and R. Fortet, \textit{Théorie des Fonctions Aléatoires}, Paris, 1953.

\bibitem{Bartlett1955} M.S. Bartlett, \textit{An Introduction to Stochastic Processes}, Cambridge, 1955.

\bibitem{Yaglom1952} A.M. Yaglom, \textit{An introduction to the theory of stationary random functions}, Usp. Mat. Nauk, Vol. 7, No. 5(51), 1952, pp. 3-168 (In Russian).

\bibitem{Ito1954} I. Ito, \textit{Stationary random distributions}, Mem. College Sci., Univ. Kyoto, Ser. A, 28, 1954, pp. 209-223.

\bibitem{Gelfand1955} I.M. Gel'fand, \textit{Generalized random processes}, Dok. Akad. Nauk SSSR, 100, 1955, pp. 853-856.

\bibitem{Schwartz1950} L. Schwartz, \textit{Théorie des Distributions}, Vols. I, II, Paris, 1950-1.

\bibitem{Taylor1935} G.I. Taylor, \textit{Statistical theory of turbulence}, Proc. Roy. Soc. A, 151, 1935, pp. 421-478.

\bibitem{VonKarman1941} Th. von Kármán and L. Howarth, \textit{On the statistical theory of isotropic turbulence}, Proc. Roy. Soc. A, 164, 1941, pp. 192-215.

\bibitem{Obukhov1941} A.M. Obukhov, \textit{On the distribution of energy in the spectrum of turbulent flow}, Izv. Akad. Nauk SSSR, Ser. Geograf. Geofiz., Nos. 4-5, 1941, pp. 453-466.

\bibitem{Kolmogorov1941a} A.N. Kolmogorov, \textit{The local structure of turbulence in an incompressible fluid at very large Reynolds numbers}, Dok. Akad. Nauk SSSR, 30, 1941, pp. 229-303.

\bibitem{Kolmogorov1941b} A.N. Kolmogorov, \textit{The distribution of energy in locally isotropic turbulence}, Dok. Akad. Nauk SSSR, 32, 1941, pp. 19-21.

\bibitem{Batchelor1947} G.K. Batchelor, \textit{Kolmogorov's theory of locally isotropic turbulence}, Proc. Camb. Phil. Soc., 43, 1947, pp. 553-559.

\bibitem{Obukhov1951} A.M. Obukhov, \textit{The statistical description of continuous fields}, Trudy Geofiz. Inst. Akad. Nauk SSSR, No. 24(151), 1954, pp. 3-42 (In Russian).

\bibitem{Ito1956} I. Ito, \textit{Isotropic random current}, Proceedings Third Berkeley Symposium on Math. Stat. and Prob., Berkeley and Los Angeles, Vol. II, 1956, pp. 125-132.

\bibitem{Ryzhik1951} I.M. Ryzhik and I.S. Gradshtein, \textit{Tables of Integrals, Sums, Series, and Products}, Moscow, 1951 (In Russian).

\bibitem{Kolmogorov1940} A.N. Kolmogorov, \textit{Curves in Hilbert space, invariant under a one-parameter group of translations}, Dok. Akad. Nauk SSSR, 26, 1940, pp. 6-9.

\bibitem{Cramer1947} H. Cramér, \textit{Random Variables and Probability Distributions}, Moscow, 1947 (Translated to Russian).

\end{thebibliography}

\end{document}

