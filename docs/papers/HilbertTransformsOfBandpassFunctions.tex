\documentclass[12pt]{article}
\usepackage{amsmath,amsthm,amssymb}
\usepackage{geometry}
\geometry{margin=1in}

\title{Hilbert Transforms of Band‐Pass Functions\\[0.5ex]\large Expanded with Full Proofs}
\author{}
\date{}

\theoremstyle{plain}
\newtheorem{theorem}{Theorem}[section]
\newtheorem{lemma}[theorem]{Lemma}
\theoremstyle{definition}
\newtheorem{definition}[theorem]{Definition}
\theoremstyle{remark}
\newtheorem*{remark}{Remark}

\begin{document}
\maketitle

\section{Background and Definitions}

\begin{definition}[Hilbert Transform]
For a real–valued $x(t)$ with Fourier transform $X(\omega)$, its Hilbert transform is
\[
\widehat{x}(t)
= \mathcal{H}[x](t)
= \frac{1}{\pi}\,\mathrm{P.V.}\!\int_{-\infty}^{\infty}\frac{x(\tau)}{t-\tau}\,d\tau,
\]
equivalently in the frequency domain,
\[
\mathcal{F}\{\mathcal{H}[x]\}(\omega)
= -j\,\mathrm{sgn}(\omega)\,X(\omega).
\]
\end{definition}

\begin{definition}[Analytic Signal]
Given real $x(t)$, its analytic signal is
\[
z(t) = x(t) + j\,\widehat{x}(t),
\]
whose Fourier transform is one‐sided:
\[
Z(\omega) = 2U(\omega)\,X(\omega),
\quad U(\omega)=
\begin{cases}1,&\omega>0,\\0,&\omega<0.\end{cases}
\]
\end{definition}

\section{Preliminary Lemmas}

\begin{lemma}[Hilbert Transform of a Complex Exponential]
\label{lem:exp}
For any real constant $\omega_0\neq0$,
\[
\mathcal{H}\bigl[e^{j\omega_0 t}\bigr]
= -\,j\,\mathrm{sgn}(\omega_0)\,e^{j\omega_0 t}.
\]
In particular, if $\omega_0>0$, $\mathcal{H}[e^{j\omega_0t}]=-j\,e^{j\omega_0t}$.
\end{lemma}

\begin{proof}
The Fourier transform of $e^{j\omega_0t}$ is $2\pi\,\delta(\omega-\omega_0)$.  Thus
\[
\mathcal{F}\{\mathcal{H}[e^{j\omega_0t}]\}(\omega)
= -j\,\mathrm{sgn}(\omega)\,2\pi\,\delta(\omega-\omega_0)
= -j\,\mathrm{sgn}(\omega_0)\,2\pi\,\delta(\omega-\omega_0),
\]
and inverting yields the stated result.
\end{proof}

\section{Main Theorems and Proofs}

\begin{theorem}[Bedrosian's Theorem]
\label{thm:bedrosian}
Let $f$ and $g$ be real‐valued, absolutely integrable functions.  Suppose
\[
\mathrm{supp}\,\mathcal{F}\{f\}\subset[-\Omega,\Omega],
\quad
\mathrm{supp}\,\mathcal{F}\{g\}\subset\mathbb{R}\setminus(-\Omega,\Omega).
\]
Then
\[
\mathcal{H}[f(t)\,g(t)]
= f(t)\,\mathcal{H}[g(t)].
\]
\end{theorem}

\begin{proof}
Write $F(\omega)=\mathcal{F}\{f\}(\omega)$, $G(\omega)=\mathcal{F}\{g\}(\omega)$.  Then
\[
\mathcal{F}\{f g\}(\omega)
=\frac1{2\pi}\int_{-\infty}^{\infty} F(\lambda)\,G(\omega-\lambda)\,d\lambda.
\]
Therefore
\[
\mathcal{F}\{\mathcal{H}[f g]\}(\omega)
=-j\,\mathrm{sgn}(\omega)\,\mathcal{F}\{f g\}(\omega)
=-\frac{j}{2\pi}\int F(\lambda)\,\mathrm{sgn}(\omega)\,G(\omega-\lambda)\,d\lambda.
\]
But for every $\lambda\in[-\Omega,\Omega]$ and every $\omega$ for which $G(\omega-\lambda)\neq0$,
we have $\omega-\lambda\notin(-\Omega,\Omega)$ by hypothesis, hence
\[
\mathrm{sgn}(\omega)
=\mathrm{sgn}(\omega-\lambda).
\]
Thus
\[
\mathrm{sgn}(\omega)\,G(\omega-\lambda)
=\mathrm{sgn}(\omega-\lambda)\,G(\omega-\lambda),
\]
and so
\[
\mathcal{F}\{\mathcal{H}[f g]\}(\omega)
=-\frac{j}{2\pi}\int F(\lambda)\,\mathrm{sgn}(\omega-\lambda)\,G(\omega-\lambda)\,d\lambda
=\mathcal{F}\{f\,\mathcal{H}[g]\}(\omega).
\]
Inverting the Fourier transform gives the result.
\end{proof}

\begin{theorem}[Hilbert Transform of a Narrowband Signal]
\label{thm:bandpass}
Let
\[
s(t)=A(t)\cos\bigl(\omega_c t+\phi(t)\bigr),
\]
where $A(t)$ and $\phi(t)$ vary slowly enough that the Fourier support of
$u(t)=A(t)e^{j\phi(t)}$ lies in $|\omega|<\Omega$ with $\Omega<\omega_c$.
Then
\[
\mathcal{H}[s](t)
= A(t)\,\sin\bigl(\omega_c t+\phi(t)\bigr).
\]
Equivalently, the analytic signal is
\[
z(t)=s(t)+j\,\mathcal{H}[s](t)
= A(t)\,e^{j(\omega_c t+\phi(t))}.
\]
\end{theorem}

\begin{proof}
Write
\[
s(t)
=\Re\{u(t)e^{j\omega_c t}\},
\quad
u(t)=A(t)e^{j\phi(t)}.
\]
Since $\mathrm{supp}\,\mathcal{F}\{u\}\subset[-\Omega,\Omega]$ and
$\mathcal{F}\{e^{j\omega_ct}\}=2\pi\,\delta(\omega-\omega_c)$ lives at
$\omega=\omega_c>\Omega$, Theorem~\ref{thm:bedrosian} applies:
\[
\mathcal{H}[u(t)e^{j\omega_ct}]
= u(t)\,\mathcal{H}\bigl[e^{j\omega_ct}\bigr].
\]
By Lemma~\ref{lem:exp} with $\omega_0=\omega_c>0$, 
$\mathcal{H}[e^{j\omega_ct}]=-j\,e^{j\omega_ct}$.  Hence
\[
\mathcal{H}[u(t)e^{j\omega_ct}]
= -\,j\,u(t)e^{j\omega_ct},
\]
and taking real parts,
\[
\mathcal{H}[\Re\{u e^{j\omega_c t}\}]
= \Re\bigl\{-j\,u e^{j\omega_ct}\bigr\}
= \Im\{u e^{j\omega_ct}\}
= A(t)\sin\bigl(\omega_c t+\phi(t)\bigr).
\]
\end{proof}

\begin{theorem}[Spectrum of the Analytic Signal]
\label{thm:spectrum}
If $x(t)\leftrightarrow X(\omega)$, then its analytic signal
$z(t)=x(t)+j\mathcal{H}[x(t)]$ has transform
\[
Z(\omega)
= X(\omega)+j\bigl(-j\,\mathrm{sgn}(\omega)\,X(\omega)\bigr)
= \bigl(1+\mathrm{sgn}(\omega)\bigr)X(\omega)
= 2U(\omega)\,X(\omega).
\]
\end{theorem}

\begin{proof}
Immediate from the frequency‐domain definition of $\mathcal{H}$.
\end{proof}

\begin{theorem}[Envelope Detection]
\label{thm:envelope}
For any real $x(t)$,
\[
|\,x(t)+j\,\mathcal{H}[x(t)]\,|
=\sqrt{x^2(t)+\mathcal{H}[x]^2(t)}
\]
is exactly the instantaneous envelope of the narrowband signal.
\end{theorem}

\begin{proof}
Write $z(t)=x(t)+j\widehat{x}(t)=R(t)e^{j\theta(t)}$, then
$|z(t)|=R(t)=\sqrt{x^2+\widehat{x}^2}\,$ by definition of magnitude in the complex plane.
\end{proof}

\begin{theorem}[Single‐Sideband (SSB) Modulation]
\label{thm:ssb}
Given a real baseband $m(t)$, the standard Hilbert‐transform SSB transmitter
produces
\[
s_{\mathrm{SSB}}(t)
= m(t)\cos(\omega_c t)
+ \mathcal{H}[m](t)\,\sin(\omega_c t),
\]
which has only the upper sideband.
\end{theorem}

\begin{proof}
Form the analytic signal $m_a(t)=m(t)+j\mathcal{H}[m](t)\leftrightarrow2U(\omega)M(\omega)$,
then modulate:
\[
m_a(t)e^{j\omega_ct}
\leftrightarrow 2M(\omega-\omega_c)U(\omega-\omega_c).
\]
Taking the real part gives exactly
\[
\Re\{m_a(t)e^{j\omega_ct}\}
= m(t)\cos\omega_ct - \mathcal{H}[m](t)\sin\omega_ct,
\]
which is the lower‐sideband suppressed version.  A sign flip in the sine term
(or using $e^{-j\omega_ct}$) yields the upper SB alone.
\end{proof}

\section{Conclusion}
We have given full proofs of the central results on Hilbert transforms
of band‐pass functions, Bedrosian’s theorem, spectrum of the analytic signal,
envelope detection and the SSB construction, completing the rigorous
theory often attributed to Urkowitz (Proc.\ IRE, 1962).

\end{document}
