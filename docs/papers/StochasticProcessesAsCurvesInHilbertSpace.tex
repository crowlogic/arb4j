\documentclass[11pt]{article}
\usepackage{amsmath,amssymb,amsthm}
\usepackage{enumerate}
\usepackage{cite}

\newtheorem{theorem}{Theorem}
\newtheorem{definition}{Definition}
\newtheorem{lemma}{Lemma}

\title{STOCHASTIC PROCESSES AS CURVES IN HILBERT SPACE}
\author{HARALD CRAM\'ER}
\date{}

\begin{document}

\maketitle

\begin{abstract}
Regular complex-valued random processes $x(t)$ with finite moments of second order are studied by methods of Hilbert space geometry. A representation formula is given for the process $x(t)$ in terms of "past and present innovations". The number $N$ is called the complete spectral multiplicity of the process $x(t)$ and is the smallest number for which such a representation exists. It is shown that the multiplicity of $x(t)$ is uniquely determined by the corresponding correlation function and that one can always find a harmonizing process $x(t)$ which has the multiplicity prescribed in advance.
\end{abstract}

\section{Introduction}

In this paper the theory of spectral multiplicity in a separable Hilbert space will be applied to the study of stochastic processes $z(t)$, where $z(t)$ is a complex-valued random variable with a finite second-order moment, while the parameter $t$ may take any real values.

For an account of multiplicity theory one may refer to Chapter 7 of Stone's book \cite{stone1932} which deals with the case of a separable space. The treatment of the subject found e.g. in the books by Halmos \cite{halmos1957} and Nakano \cite{nakano1953} is mainly concerned with the more general and considerably more intricate case of a non-separable space.

The considerations will apply to certain classes of curves in a purely abstract Hilbert space, and it is only a question of terminology when, throughout this paper, the discussion is confined to that particular realization of Hilbert space which has proved useful in probability theory.

Finally, all statements may be directly generalized to the case of vector processes of the form $(z_1(t), \ldots, z_k(t))$.

\section{Basic Definitions and Auxiliary Concepts}

Consider the set of all complex-valued random variables $x$ defined on a fixed probability space and satisfying the relations
\begin{equation}
E\{|x|^2\} < \infty.
\label{eq:finite_moment}
\end{equation}

Two variables $x$ and $y$ will be regarded as identical if
\begin{equation}
E\{|x - y|^2\} = 0.
\label{eq:identical_vars}
\end{equation}

The set of all these variables forms a Hilbert space $H$, if the inner product is defined in the usual way:
\begin{equation}
(x, y) = E\{x\overline{y}\}.
\label{eq:inner_product}
\end{equation}

Convergence of sequences of random variables will always be understood as strong convergence in the topology thus introduced, i.e. as convergence in quadratic mean according to probability terminology.

If for every real $t$ a random variable $x(t) \in H$ is given, the set of variables $x(t)$ may be regarded as a stochastic process with continuous time $t$, or, alternatively, as a curve $C$ in the Hilbert space $H$. It is well known that various properties of stochastic processes have been studied by regarding them as curves in $H$ (cf., e.g., \cite{kolmogorov1940a}, \cite{kolmogorov1940b}, \cite{neumann1941}, \cite{pinsker1955}, \cite{cramer1951}, \cite{cramer1961}).

Certain subspaces of $H$ associated with the $x(t)$ curve or process are defined by writing
\begin{align}
H(x) &= S\{x(u), -\infty < u < \infty\}, \label{eq:subspace_H_x}\\
H(x, t) &= S\{x(u), u \leq t\}, \label{eq:subspace_H_x_t}\\
H(x, -\infty) &= \bigcap_{t} H(x, t). \label{eq:subspace_H_x_minus_infty}
\end{align}

Here $S\{\cdot\}$ denotes the subspace of $H$ spanned by the random variables indicated between the brackets.

Evidently $H(x)$ is the smallest subspace of $H$ which contains the whole curve $C$ generated by the $x(t)$ process, while $H(x, t)$ is the smallest subspace containing the "arc" of $C$ formed by all points $x(u)$ with $u \leq t$. If the parameter $t$ is interpreted as time, $H(x, t)$ will represent the "past and present" of the $x(t)$ process from the point of view of the instant $t$, while $H(x, -\infty)$ represents the "infinitely remote past" of the process.

For the processes $y(t)$, $z(t)$, etc., the analogous notations $H(y, t)$, $H(z, t)$, etc., are used for subspaces defined in the corresponding way.

In the sequel only stochastic processes $x(t) \in H$ which are assumed to satisfy the following conditions (A) and (B) will be considered:

\begin{enumerate}[(A)]
\item The subspace $H(x, -\infty)$ contains only the zero element of $H$.
\item For all $t$ the limits $x(t + 0)$ exist and $x(t - 0) = x(t)$.
\end{enumerate}

The condition (A) implies that $x(t)$ is a regular, or purely non-deterministic process (cf. \cite{cramer1961a}). From (B) it follows, as shown in \cite{cramer1961a}, that the space $H(x)$ is separable, and that the $x(t)$ curve has at most an enumerable number of discontinuities. The condition $x(t - 0) = x(t)$ is not essential, and is introduced here only in order to avoid some trivial complications.

As $t$ increases through real values, the subspaces $H(x, t)$ will obviously form a never decreasing family. For a fixed finite $t$ the union of all $H(x, u)$ with $u < t$ may not be a closed set, but if $H(x, t - 0)$ is defined as the closure of this union, it is easily proved that $H(x, t - 0) = H(x, t)$ always holds.

Similarly, the union of the $H(x, u)$ for all real $u$ may not be closed, but if $H(x, +\infty)$ is defined as its closure, then $H(x, +\infty) = H(x)$.

Suppose that a certain time point $t$ is such that for all $h > 0$
\begin{equation}
H(x, t - h) \neq H(x, t + h).
\label{eq:innovation_condition}
\end{equation}

Every time interval $t - h < u < t + h$ will then contain at least one $x(u)$ not included in $H(x, t - h)$. This may be expressed by saying that the process receives a new impulse, or an innovation, during the interval $(t - h, t + h)$ for every $h > 0$. The set of all time points $t$ with this property will be called the innovation spectrum of the $x(t)$ process.

\section{Projection Operators and Spectral Families}

Suppose that a stochastic process $x(t)$ satisfying the conditions (A) and (B) is given. By $P_t$ the projection operator in the Hilbert space $H(x)$ with range $H(x, t)$ is denoted. It then follows from the properties of the $H(x, t)$ family given above that
\begin{align}
P_t &\leq P_u \text{ for } t < u, \label{eq:proj_monotone}\\
P_{t-0} &= P_t \text{ for all } t, \label{eq:proj_left_cont}\\
P_{-\infty} &= 0, \quad P_{+\infty} = I, \label{eq:proj_limits}
\end{align}
where $0$ and $I$ denote respectively the zero and the identity operator in $H(x)$.

It follows that the $P_t$ form a spectral family of projections or a resolution of the identity according to Hilbert space terminology. As has been seen, the projections $P_t$ are, for all real $t$, uniquely determined by the given $x(t)$ process.

For any random variable $z \in H(x)$ with $E\{|z|^2\} < \infty$ a stochastic process is now defined by writing
\begin{equation}
z(t) = P_t z.
\label{eq:partial_innovation}
\end{equation}

It is then readily seen that $z(t)$ will be a process with orthogonal increments satisfying (A) and (B). Writing
\begin{equation}
F_z(t) = E\{|z(t)|^2\},
\label{eq:distribution_function}
\end{equation}
it follows that $F_z(t)$ is, for any fixed $z$, a distribution function of $t$ such that $F_z(t - 0) = F_z(t)$ for all $t$.

The points of increase of $z(t)$ coincide with the points of increase of $F_z(t)$ and form a subset of the innovation spectrum of $x(t)$, and, accordingly, $z(t)$ is denoted as a partial innovation process associated with $x(t)$. The space $H(z, t)$ is spanned by the random variables $z(u)$ with $u \leq t$, and it is known (cf., e.g., \cite{doob1953}, p. 425--428) that $H(z, t)$ is identical with the set of all random variables of the form
\begin{equation}
\int_{-\infty}^t g(t, u) dz(u),
\label{eq:stochastic_integral}
\end{equation}
where $g(t, u)$ is a non-random function such that the integral
\begin{equation}
\int_{-\infty}^t |g(t, u)|^2 dF_z(u)
\label{eq:integral_convergence}
\end{equation}
is convergent.

\section{Equivalence Classes and Multiplicity}

Consider now the class $Q$ of all distribution functions $F(t)$ determined such as to be continuous to the left for all $t$. A partial ordering in $Q$ is introduced by saying that $F_1$ is superior to $F_2$, and writing $F_1 \succeq F_2$, whenever $F_1$ is absolutely continuous with respect to $F_2$. If $F_1 \succeq F_2$ and $F_2 \succeq F_1$, then $F_1$ and $F_2$ are said to be equivalent.

The set of all distribution functions equivalent to a given $F(t)$ forms an equivalence class $R$. A partial ordering is introduced in the set of all equivalence classes in the obvious way by writing $R_1 \succeq R_2$ when the corresponding relation holds for any $F_1 \in R_1$ and $F_2 \in R_2$. A point $t$ is called a point of increase for the equivalence class $R$ whenever $t$ is a point of increase for any $F \in R$.

In the sequel consideration will be given to never increasing sequences of equivalence classes:
\begin{equation}
R_1 \succeq R_2 \succeq \cdots \succeq R_N.
\label{eq:equivalence_sequence}
\end{equation}

The number $N$ of elements in a sequence of this form, which may be finite or infinite, will be called the total multiplicity of the sequence. Also defined is a multiplicity function $N(t)$ of the sequence~\eqref{eq:equivalence_sequence} by writing for any real $t$
\begin{equation}
N(t) = \text{the number of those } R_n \text{ in } \eqref{eq:equivalence_sequence} \text{ for which } t \text{ is a point of increase}.
\label{eq:multiplicity_function}
\end{equation}

$N(t)$, like $N$, may be finite or infinite, and the total multiplicity $N$ will obviously satisfy the relation
\begin{equation}
N = \sup_t N(t),
\label{eq:total_multiplicity}
\end{equation}
where $t$ runs through all real values.

If, in particular, $N(t) = 0$ for all $t$ in some closed interval $[a, b]$, all functions $F(t)$ belonging to any of the equivalence classes in the sequence~\eqref{eq:equivalence_sequence} will be constant throughout $[a, b]$.

\section{Spectral Representation}

To any $x(t)$ satisfying the conditions (A) and (B) there corresponds according to Section~3 a uniquely determined spectral family of projections $P_t$ satisfying~\eqref{eq:proj_monotone}--\eqref{eq:proj_limits}. It then follows from the theory of spectral multiplicity in a separable Hilbert space (\cite{stone1932}, Chapter 7) that to the same $x(t)$ there corresponds a uniquely determined, never increasing sequence~\eqref{eq:equivalence_sequence} of equivalence classes, having the following properties:

If $N$ is the total multiplicity of the sequence~\eqref{eq:equivalence_sequence}, it is possible to find $N$ orthonormal random variables $z_1, \ldots, z_N \in H(x)$ such that the corresponding processes with orthogonal increments defined in Section~3 satisfy the relations
\begin{align}
F_{z_n}(t) &\in R_n, \label{eq:distribution_in_class}\\
H(z_m, t) &\perp H(z_n, t), \quad m \neq n, \label{eq:orthogonal_subspaces}\\
H(x, t) &= \bigoplus_{n=1}^N H(z_n, t), \label{eq:direct_sum}
\end{align}
where the last sum denotes the vector sum of the mutually orthogonal subspaces involved.

Now $x(t)$ is always an element of $H(x, t)$ and from Section~3 the following theorem is obtained, previously given in somewhat less precise form in \cite{cramer1951} and \cite{cramer1961}:

\begin{theorem}\label{thm:spectral_representation}
To any stochastic process $x(t)$ satisfying (A) and (B) there corresponds a uniquely determined sequence~\eqref{eq:equivalence_sequence} of equivalence classes such that $x(t)$ can be represented in the form
\begin{equation}
x(t) = \sum_{n=1}^N \int_{-\infty}^t g_n(t, u) dz_n(u),
\label{eq:spectral_representation}
\end{equation}
where the $z_n(u)$ are mutually orthogonal processes with orthogonal increments satisfying~\eqref{eq:distribution_in_class}--\eqref{eq:direct_sum}. The $g_n(t, u)$ are non-random functions such that
\begin{equation}
\sum_{n=1}^N \int_{-\infty}^t |g_n(t, u)|^2 dF_{z_n}(u) < \infty.
\label{eq:convergence_condition}
\end{equation}
The number $N$, which is called the total spectral multiplicity of the $x(t)$ process, is the uniquely determined number of elements in~\eqref{eq:equivalence_sequence} and may be finite or infinite. No representation of the form~\eqref{eq:spectral_representation} with these properties exists for any smaller value of $N$.
\end{theorem}

The sequence~\eqref{eq:equivalence_sequence} corresponding to a given $x(t)$ process will be said to determine the spectral type of the process.

The relation~\eqref{eq:spectral_representation} gives a linear representation of $x(t)$ in terms of past and present innovation elements $dz_n(u)$. The total innovation process associated with $x(t)$ is an $N$-dimensional vector process $\{z_1(t), \ldots, z_N(t)\}$ where, as before, $N$ may be finite or infinite.

It is interesting to compare this with the situation in the case of a regular process with discrete time (\cite{cramer1951}, Theorem 1) where a similar representation always holds with $N = 1$.

Also in the particular case of a stationary process with continuous time, satisfying (A) and (B), it follows from well-known theorems that $N = N(t) = 1$ for all $t$, and that the only element in the corresponding sequence~\eqref{eq:equivalence_sequence} may be represented by any absolutely continuous distribution function $F(t)$ having an everywhere positive density function.

\section{Prediction and Innovation}

The best linear least squares prediction of $x(t + h)$ in terms of all $x(u)$ with $u \leq t$ is obtained from~\eqref{eq:spectral_representation} in the form
\begin{equation}
P_t x(t + h) = \sum_{n=1}^N \int_{-\infty}^t g_n(t + h, u) dz_n(u).
\label{eq:prediction}
\end{equation}

The error involved in this prediction is
\begin{equation}
x(t + h) - P_t x(t + h) = \sum_{n=1}^N \int_t^{t+h} g_n(t + h, u) dz_n(u).
\label{eq:prediction_error}
\end{equation}

Now consider the multiplicity function $N(t)$ associated with the sequence~\eqref{eq:equivalence_sequence}, as defined in Section~4. Suppose that in the closed interval $t \leq u \leq t + h$ there is $N(u) \leq N_1 < N$. Then all terms with $n > N_1$ in the second member of~\eqref{eq:prediction_error} will reduce to zero, so that the innovation entering into the process during $[t, t + h]$ will only be of dimensionality $N_1$. Speaking somewhat loosely, the multiplicity function $N(t)$ determines for every $t$ the dimensionality of the innovation element $\{dz_1(t), dz_2(t), \ldots\}$.

If, in particular, $N(u) = 0$ for $t \leq u \leq t + h$, it follows that the process does not receive any innovation at all during this interval. Accordingly in this case the whole second member of~\eqref{eq:prediction_error} reduces to zero, so that exact prediction is possible over the interval considered.

\section{Correlation Functions and Spectral Type}

The correlation function of the $x(t)$ process is introduced:
\begin{equation}
r(s, t) = E\{x(s)\overline{x(t)}\}.
\label{eq:correlation_function}
\end{equation}

As before it is assumed that all stochastic processes considered satisfy the conditions (A) and (B). The following theorem is proved, which shows that the spectral type of a process is uniquely determined by the correlation function.

\begin{theorem}\label{thm:correlation_determines_type}
Let $x(t)$ and $y(t)$ be two processes satisfying (A) and (B) and having the same correlation function $r(s, t)$. The sequences of equivalence classes, which correspond to $x(t)$ and $y(t)$ in the way described in Theorem~\ref{thm:spectral_representation}, are then identical.
\end{theorem}

\begin{proof}
$x(t)$ and $y(t)$ define two curves situated, respectively, in the spaces $H(x)$ and $H(y)$. Define a transformation $V$ from the $x$-curve to the $y$-curve by writing
\begin{equation}
Vx(t) = y(t),
\label{eq:transformation}
\end{equation}
and extend this definition by linearity to the linear manifold in $H(x)$ determined by all points $x(t)$. It is readily seen that this definition is unique, and that the transformation is isometric. It follows, in fact, from the equality of the correlation functions that any linear relation $\sum \alpha_k x(t_k) = 0$ implies and is implied by the corresponding relation $\sum \alpha_k y(t_k) = 0$, which shows that the transformation is unique, while the isometry follows from the identity
\begin{equation}
r(s, t) = (x(s), x(t)) = (Vx(s), Vx(t)).
\label{eq:isometry}
\end{equation}

The transformation can now be extended to an isometric transformation $V$ defined in the whole space $H(x)$. If the restriction of $V$ to $H(x, t)$ is considered, it is immediately seen that for all $t$
\begin{equation}
VH(x, t) = H(y, t).
\label{eq:subspace_mapping}
\end{equation}

Denoting by $P_t^{(x)}$ and $P_t^{(y)}$ the spectral families of projections corresponding, respectively, to $x(t)$ and $y(t)$, then
\begin{equation}
VP_t^{(x)}V^{-1} = P_t^{(y)}.
\label{eq:spectral_equivalence}
\end{equation}

Thus the two spectral families are isometrically equivalent, and the assertion of the theorem now follows directly from Hilbert space theory. In the particular case when $H(x) = H(y)$, the transformation $V$ will be unitary.
\end{proof}

On the other hand, two processes with isometrically equivalent spectral families do not necessarily have the same correlation function. In other words, the correlation function is not uniquely determined by the spectral type.

In order to see this, it is enough to consider the two processes $x(t)$ and $y(t) = f(t)x(t)$, where $f(t)$ is a non-random function such that $0 < m \leq |f(t)| \leq M$ for all $t$. It is clear that $H(x, t) = H(y, t)$ for all $t$, while the correlation functions differ by the factor $f(s)\overline{f(t)}$.

\section{Construction of Processes with Given Spectral Type}

In this section it will be shown that a stochastic process possessing any given spectral type can always be found. The more precise statement is contained in the following theorem.

\begin{theorem}\label{thm:construction}
Suppose that a sequence of equivalence classes of the form~\eqref{eq:equivalence_sequence} is given. Then there exists a harmonizable process $x(t)$ which has the spectral type defined by the given sequence.
\end{theorem}

Comparing this statement with the final remark in Section~5, it will be seen how restricted the class of stationary processes is in comparison with the class of harmonizable processes.

\begin{proof}
Denote by $A_1, A_2, \ldots$ a sequence of disjoint sets of real points such that the measure of every $A_n$ is positive in any non-vanishing interval.\footnote{The use of the sets $A_n$ for the construction of processes with given multiplicity properties goes back to a correspondence between Professor Kolmogorov and the present author (cf. \cite{cramer1962}). A simple way of constructing the $A_n$ is the following: Let $1 < n_1 < n_2 < \cdots$ be positive integers such that $\sum 1/n_k$ converges. Almost every real $x$ then has a unique expansion $x = r_0 + \sum r_k/(n_1 \cdots n_k)$, where the $r_k$ are integers and $0 \leq r_k < n_k$ for $k \geq 1$. If $A_n$ is the set of those $x$ for which the number of zeros among the $r_k$ with $k \geq 1$ is finite and of the form $2^{2p+1}$ where $p$ is a non-negative integer, then the sequence $A_1, A_2, \ldots$ has the required properties.} If $\alpha_n(v)$ is the characteristic function of $A_n$ then
\begin{equation}
\int_a^b \alpha_n(v) dv > 0
\label{eq:positive_measure}
\end{equation}
for all $n$ and for any real $a < b$.

Take in each equivalence class $R_n$ appearing in the given sequence~\eqref{eq:equivalence_sequence} a distribution function $F_n(t) \in R_n$. Obviously the functions $F_1, F_2, \ldots, F_N$ can be chosen so that the integrals
\begin{equation}
k_n = \int_{-\infty}^{\infty} e^{t^2} dF_n(t)
\label{eq:moment_integral}
\end{equation}
converge for all $n$. Then $1 \leq k_n < \infty$. Assuming that the basic probability field is not too restricted, $N$ mutually orthogonal stochastic processes $z_1(t), \ldots, z_N(t)$ with orthogonal increments can be found such that
\begin{equation}
F_{z_n}(t) = E\{|z_n(t)|^2\} = F_n(t).
\label{eq:process_distribution}
\end{equation}

The following definition is now introduced:
\begin{equation}
g_n(t, u) = \begin{cases}
\frac{1}{nk_n} \int_u^t e^{-(t-v)} \alpha_n(v) dv, & u < t, \\
0, & u \geq t,
\end{cases}
\label{eq:kernel_function}
\end{equation}
and
\begin{align}
x_n(t) &= \int_{-\infty}^t g_n(t, u) dz_n(u), \label{eq:component_process}\\
x(t) &= \sum_{n=1}^N x_n(t). \label{eq:total_process}
\end{align}

For $u < t$ there is
\begin{equation}
0 < g_n(t, u) < \frac{1}{nk_n} e^{-(t-u)},
\label{eq:kernel_bound}
\end{equation}
and hence by~\eqref{eq:moment_integral},
\begin{align}
E\{|x_n(t)|^2\} &< \frac{1}{n^2k_n^2} \int_{-\infty}^{\infty} e^{-2t} \int_{-\infty}^{\infty} (t-u)^2_+ dF_n(u) \nonumber \\
&\leq \frac{8}{n^2k_n^2} \int_{-\infty}^{\infty} e^{-2t} (t^4 + u^4) dF_n(u) \leq \frac{8(t^4 + k_n)}{n^2k_n}, \label{eq:moment_bound}
\end{align}
so that the series for $x(t)$ converges in quadratic mean if $N = \infty$. (Note that the $x_n(t)$, like the $z_n(t)$, are mutually orthogonal.)

It will now be proved that (a) the $x(t)$ process defined by~\eqref{eq:total_process} has the given spectral type, and (b) that it is harmonizable.

It follows from the construction of the $z_n(t)$ and from~\eqref{eq:total_process} that
\begin{align}
F_{z_n}(t) &\in R_n, \label{eq:type_condition_a}\\
H(z_m, t) &\perp H(z_n, t), \quad m \neq n, \label{eq:type_condition_b}\\
H(x, t) &\subseteq \bigoplus_{n=1}^N H(z_n, t). \label{eq:type_condition_c}
\end{align}

If it can be shown that the sign of equality holds in the last relation, the relations~\eqref{eq:distribution_in_class}--\eqref{eq:direct_sum} will be satisfied and it then follows that the $x(t)$ process defined by~\eqref{eq:total_process} has the given spectral type. In order to prove this it is sufficient to show that
\begin{equation}
z_n(t) \in H(x, t)
\label{eq:innovation_in_space}
\end{equation}
for all $n$ and $t$.

There is
\begin{equation}
e^t x(t) = \sum_{n=1}^N \int_{-\infty}^t g_n(t, u) e^t dz_n(u) = \sum_{n=1}^N \frac{1}{nk_n} \int_{-\infty}^t e^u \alpha_n(u) dz_n(u).
\label{eq:exponential_process}
\end{equation}

It is shown without difficulty that the derivative in q.m. of this random function exists for all $t$ and has the expression
\begin{equation}
\frac{d}{dt}(e^t x(t)) = \sum_{n=1}^N \frac{1}{nk_n} \int_{-\infty}^{\infty} \alpha_n(u) z_n(u) du,
\label{eq:first_derivative}
\end{equation}
where the last sum converges in q.m. It is desired to show that for almost all $t$ (Lebesgue measure) differentiation once more in q.m. is possible, and so obtain
\begin{equation}
\frac{d^2}{dt^2}(e^t x(t)) = \sum_{n=1}^N \frac{1}{nk_n} \alpha_n(t) z_n(t).
\label{eq:second_derivative}
\end{equation}

In order to prove this it must be shown that the random variable
\begin{equation}
W = \sum_{n=1}^N \frac{1}{nk_n h} \int_t^{t+h} \alpha_n(u) (z_n(u) - z_n(t)) du - \alpha_n(t) z_n(t)
\label{eq:difference_quotient}
\end{equation}
converges to zero in q.m. for almost all $t$ as $h \to 0$. There is $W = W_1 + W_2$, where
\begin{align}
W_1 &= \sum_{n=1}^N \frac{1}{nk_n h} \int_t^{t+h} \alpha_n(u) (z_n(u) - z_n(t)) du, \label{eq:W1}\\
W_2 &= \sum_{n=1}^N \frac{1}{nk_n} \alpha_n(t) \left(\frac{1}{h} \int_t^{t+h} \alpha_n(u) du - \alpha_n(t)\right). \label{eq:W2}
\end{align}

Now both $W_1$ and $W_2$ are sums of mutually orthogonal random variables and there is
\begin{align}
E|W_1|^2 &= \sum_{n=1}^N \frac{1}{n^2 k_n^2 h^2} \int_t^{t+h} \int_t^{t+h} \alpha_n(u) \alpha_n(v) [F_n(\min(u,v)) - F_n(t)] du dv \nonumber \\
&\leq 2 \sum_{n=1}^N \frac{1}{n^2 h^2} \int_t^{t+h} (t+h-u)[F_n(u) - F_n(t)] du \leq 2 \sum_{n=1}^N \frac{F_n(t+h) - F_n(t)}{n^2} \label{eq:W1_bound}
\end{align}
and
\begin{equation}
E|W_2|^2 = \sum_{n=1}^N \frac{1}{n^2} F_n(t) \left|\frac{1}{h} \int_t^{t+h} \alpha_n(u) du - \alpha_n(t)\right|^2.
\label{eq:W2_bound}
\end{equation}

However, all the $F_n$ are continuous almost everywhere, and it follows that $W_1$ tends to zero in q.m. for almost all $t$. On the other hand, the metric density of any $A_n$ exists almost everywhere and is equal to $\alpha_n(t)$ so that $W_2$ tends to zero in q.m. almost everywhere. Thus it has been shown that~\eqref{eq:second_derivative} holds for almost all $t$.

Let now $m$ be a given integer, $1 \leq m \leq N$. The sets $A_n$ being disjoint, it then follows from~\eqref{eq:second_derivative} that for almost all $t \in A_m$
\begin{equation}
\frac{d^2}{dt^2}(e^t x(t)) = \frac{1}{mk_m} z_m(t).
\label{eq:component_derivative}
\end{equation}

The first member of the last relation is evidently an element of $H(x, t)$, so that $z_m(t) \in H(x, t)$ for almost all $t \in A_m$. Now $A_m$ is of positive measure in every non-vanishing interval, while $z_m(t)$ is by definition everywhere continuous to the left in q.m. Thus $z_m(t) \in H(x, t)$ for all $t$ and all $m = 1, \ldots, N$, and according to the above this proves that $x(t)$ has the given spectral type.

In order to prove also that $x(t)$ is harmonizable the Fourier transform $h_n(\lambda, u)$ of $g_n(t, u)$ with respect to $t$ is introduced. From~\eqref{eq:kernel_function} there is obtained
\begin{align}
h_n(\lambda, u) &= \int_{-\infty}^{\infty} e^{-it\lambda} g_n(t, u) dt \nonumber \\
&= \frac{1}{nk_n} \int_u^{\infty} e^{-t(1+i\lambda)} dt \int_u^t (t-v) \alpha_n(v) dv. \label{eq:fourier_transform_step}
\end{align}

The double integral is absolutely convergent and there is
\begin{align}
h_n(\lambda, u) &= \frac{1}{nk_n} \int_u^{\infty} \alpha_n(v) dv \int_v^{\infty} (t-v) e^{-t(1+i\lambda)} dt \nonumber \\
&= \frac{1}{nk_n(1+i\lambda)^2} \int_u^{\infty} \alpha_n(v) e^{-v(1+i\lambda)} dv, \label{eq:fourier_transform}
\end{align}
\begin{equation}
|h_n(\lambda, u)| < \frac{e^{-u}}{nk_n(1+\lambda^2)}.
\label{eq:fourier_bound}
\end{equation}

Thus $h_n(\lambda, u)$ is, for any fixed $u$, absolutely integrable with respect to $\lambda$. On the other hand, it follows from~\eqref{eq:kernel_function} that $g_n(t, u)$ is everywhere continuous, so that the inverse Fourier formula holds:
\begin{equation}
g_n(t, u) = \frac{1}{2\pi} \int_{-\infty}^{\infty} h_n(\lambda, u) e^{it\lambda} d\lambda.
\label{eq:inverse_fourier}
\end{equation}

Now the correlation functions of $x_n(t)$ and $x(t)$ are, by~\eqref{eq:kernel_function} and~\eqref{eq:total_process},
\begin{align}
r_n(s, t) &= E\{x_n(s) \overline{x_n(t)}\} = \int_{-\infty}^{\infty} g_n(s, u) g_n(t, u) dF_n(u), \label{eq:component_correlation}\\
r(s, t) &= E\{x(s) \overline{x(t)}\} = \sum_{n=1}^N r_n(s, t). \label{eq:total_correlation}
\end{align}

Replacing the $g_n$ by their expressions according to~\eqref{eq:inverse_fourier} there is obtained
\begin{equation}
r_n(s, t) = \frac{1}{(2\pi)^2} \int_{-\infty}^{\infty} \int_{-\infty}^{\infty} e^{i(s\lambda - t\mu)} d\lambda d\mu \int_{-\infty}^{\infty} h_n(\lambda, u) \overline{h_n(\mu, u)} dF_n(u),
\label{eq:fourier_correlation}
\end{equation}
the inversion of the order of integration being justified by absolute convergence according to~\eqref{eq:moment_integral} and~\eqref{eq:fourier_bound}.

If written
\begin{align}
C_n(\lambda, \mu) &= \frac{1}{(2\pi)^2} \int_{-\infty}^{\infty} h_n(\lambda, u) \overline{h_n(\mu, u)} dF_n(u), \label{eq:spectral_function_n}\\
C_n(\lambda, \mu) &= \int_{-\infty}^{\lambda} \int_{-\infty}^{\mu} c_n(\rho, \sigma) d\rho d\sigma, \label{eq:spectral_density_n}
\end{align}
it follows from well-known criteria (cf., e.g., \cite{loeve1963}, p. 466--469) that $C_n(\lambda, \mu)$ is a correlation function. Further $C_n(\lambda, \mu)$ is of bounded variation over the whole $(\lambda, \mu)$-plane, its variation being bounded by the expression
\begin{align}
&\int_{-\infty}^{\infty} \int_{-\infty}^{\infty} |C_n(\lambda, \mu)| d\lambda d\mu \nonumber \\
&< \frac{1}{\pi} \int_{-\infty}^{\infty} \frac{d\lambda}{1+\lambda^2} \int_{-\infty}^{\infty} \frac{d\mu}{1+\mu^2} \int_{-\infty}^{\infty} e^{1+u^2} dF_n(u) < \frac{1}{4n^2k_n^2} \label{eq:variation_bound}
\end{align}
obtained from~\eqref{eq:moment_integral} and~\eqref{eq:fourier_bound}.

It now follows that
\begin{align}
r_n(s, t) &= \int_{-\infty}^{\infty} \int_{-\infty}^{\infty} e^{i(s\lambda - t\mu)} d\lambda d\mu C_n(\lambda, \mu), \label{eq:harmonizable_n}\\
r(s, t) &= \int_{-\infty}^{\infty} \int_{-\infty}^{\infty} e^{i(s\lambda - t\mu)} d\lambda d\mu C(\lambda, \mu), \label{eq:harmonizable}
\end{align}
where
\begin{equation}
C(\lambda, \mu) = \sum_{n=1}^N C_n(\lambda, \mu)
\label{eq:total_spectral_function}
\end{equation}
is a correlation function which, according to the above, is of bounded variation over the whole $(\lambda, \mu)$-plane. Hence it may be concluded (cf. \cite{cramer1951}, and \cite{loeve1963}, p. 476) that $x(t)$ is a harmonizable process
\begin{equation}
x(t) = \int_{-\infty}^{\infty} e^{it\lambda} dy(\lambda),
\label{eq:harmonizable_representation}
\end{equation}
where $y(\lambda)$ has the correlation function $C(\lambda, \mu)$. Note that $x(t)$ is a regular process and is everywhere continuous in quadratic mean. The proof is completed.
\end{proof}

\begin{thebibliography}{99}
\bibitem{cramer1951} H. Cram\'er, \emph{A contribution to the theory of stochastic processes}, Proc. Second Berkeley Sympos. Math. Statist. and Prob., 1951, pp. 329--339.

\bibitem{cramer1961} H. Cram\'er, \emph{On some classes of non-stationary stochastic processes}, Proc. 4th Berkeley Sympos. Math. Statist. and Prob., II, 1961, pp. 57--77.

\bibitem{cramer1961a} H. Cram\'er, \emph{On the structure of purely non-deterministic stochastic processes}, Arkiv Math., 4, 1961, pp. 249--266.

\bibitem{cramer1962} H. Cram\'er, \emph{D\'ecompositions orthogonales de certains proc\`es stochastiques}, Ann. Fac. Sciences Clermont, 11, 1962, pp. 15--21.

\bibitem{doob1953} J. L. Doob, \emph{Stochastic Processes}, Wiley, N.Y., 1953.

\bibitem{halmos1957} P. R. Halmos, \emph{Introduction to Hilbert Space and the Theory of Spectral Multiplicity}, Chelsea, N.Y., 2nd ed. 1957.

\bibitem{kolmogorov1940a} A. N. Kolmogorov, \emph{Curves in Hilbert space which are invariant with respect to a one-parameter group of motions}, DAN SSSR, 26, 1940, pp. 6--9. (In Russian.)

\bibitem{kolmogorov1940b} A. N. Kolmogorov, \emph{Wiener's spiral and some other interesting curves in Hilbert space}, DAN SSSR, 26, 1940, pp. 115--118. (In Russian.)

\bibitem{loeve1963} M. Lo\`eve, \emph{Probability Theory}, 3rd ed., Van Nostrand, Princeton, N.J., 1963.

\bibitem{nakano1953} H. Nakano, \emph{Spectral Theory in the Hilbert Space}, Jap. Soc. for the Promotion of Science, Tokyo, 1953.

\bibitem{neumann1941} J. von Neumann and I. J. Schoenberg, \emph{Fourier integrals and metric geometry}, Trans. Amer. Math. Soc., 50, 1941, pp. 226--251.

\bibitem{pinsker1955} M. S. Pinsker, \emph{Theory of curves in Hilbert space with stationary $n$-th increments}, Izv. AN SSSR, Ser. Mat., 19, 1955, pp. 319--344. (In Russian.)

\bibitem{stone1932} M. H. Stone, \emph{Linear Transformations in Hilbert Space}, American Math. Soc., N.Y., 1932.
\end{thebibliography}

\end{document}


