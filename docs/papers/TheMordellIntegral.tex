
\documentclass[12pt]{article}
\usepackage{amsmath,amssymb,amsthm}
\usepackage{enumitem}
\usepackage{hyperref}
\usepackage{geometry}
\geometry{a4paper, margin=1in}

\newtheorem{theorem}{Theorem}[section]
\newtheorem{lemma}[theorem]{Lemma}
\newtheorem{definition}[theorem]{Definition}
\newtheorem{corollary}[theorem]{Corollary}
\newtheorem{proposition}[theorem]{Proposition}

\theoremstyle{remark}
\newtheorem*{remark}{Remark}

\begin{document}

\title{The Definite Gaussian Integral and the Analytic Theory of Numbers}
\author{L. J. Mordell\\
University of Manchester}
\date{}
\maketitle

\section{Introduction}

Professor Siegel~\cite{Siegel1932} in a memoir recently published dealing with the manuscripts left by Riemann has pointed out that Riemann dealt with some integrals of the type
\begin{equation}\label{eq:riemann_integral}
I = \int \frac{e^{at + bt^2}}{e^{ct} + d} \, dt
\end{equation}
in his researches on the zeta function. Not only can the usual functional equation be thus found, but also an asymptotic formula is obtained for the zeta-function of which the first term gives the well known approximate functional equation due to Hardy and Littlewood~\cite{HardyLittlewood1914}.

Kronecker's evaluation of the Gauss's sums by special integrals of this kind is classic. Not so well known is his evaluation of the integral
\begin{equation}\label{eq:kronecker_integral}
\int_0^\infty \frac{e^{\pi i t^2/n}}{\cosh at} \, dt
\end{equation}
in finite terms when $n$ is an integer. The general integral or particular cases have also been considered by Lerch~\cite{Lerch1892}, Hardy~\cite{Hardy1904}, Ramanujan~\cite{Ramanujan1915a}, van der Corput~\cite{vanderCorput1922} and myself. My results which included the complete evaluation of the general integral, were found in September 1918 and published in 1920 in volume 48 of the Quarterly Journal. The paper is not well known and has even escaped the notice of the editors of the Fortschritte. Further, it is not easily accessible outside of Great Britain. It seems in view of the interest aroused by Siegel's paper that it might be desirable to give a more accessible and fuller account of the integral, and the considerations leading to it and the results deduced from it. These are concerned chiefly with formulae involving the class number of definite binary quadratics, many of which are also not easily accessible and suggest interesting problems for research. I also include some new and allied results dealing with the approximate functional equation of the theta function.

The starting point of my investigations was the theory of the positive, definite binary quadratic form
\begin{equation}\label{eq:binary_quadratic_form}
ax^2 + 2hxy + by^2,
\end{equation}
where $a, h, b$ are integers, so that the determinant of the form is
\begin{equation}\label{eq:determinant}
h^2 - ab = -D < 0.
\end{equation}

Let $F(D)$ be the number of uneven classes of forms of given determinant $-D$, that is, classes of forms in which $a$ and $b$ are not both even, and let $G(D)$ be the total number of classes of forms of determinant $-D$. It proves convenient to assume that weights $\frac{1}{2}, \frac{1}{3}$ are attached to the forms $(a, 0, a)$, $(2a, a, 2a)$ respectively, and also that
\begin{equation}\label{eq:F0_G0}
F(0) = 0, \quad G(0) = -\frac{1}{12}.
\end{equation}

The formulae for the class number are nearly a century old. It is well known that Dirichlet proved that when $-D$ is negative and has no squared factors $> 1$,
\begin{equation}\label{eq:dirichlet_formula}
F(D) = \frac{\sqrt{D}}{\pi} \left(\left(\frac{1}{1}\right) + \left(\frac{1}{3}\right) + \left(\frac{1}{5}\right) + \cdots\right)
\end{equation}

Another formula published by me~\cite{Mordell1916} some years ago states that for all $-D < 0$,
\begin{equation}\label{eq:mordell_formula}
F(D) = \frac{\sqrt{D}}{\pi} \left( \frac{N(1)}{1} - \frac{N(3)}{3} + \frac{N(5)}{5} - \cdots \right),
\end{equation}
where $N(n)$ is the number of solutions mod $n$ of the congruence
\begin{equation}\label{eq:congruence}
x^2 \equiv D \pmod{n}.
\end{equation}

These series can be summed in finite terms, but none of these formulae would suggest the existence of the so-called class relation formulae originally discovered by Kronecker. One is
\begin{equation}\label{eq:class_relation}
F(n) + 2F(n-1^2) + 2F(n-2^2) + \cdots = I(n) - \Gamma(n),
\end{equation}
where the summation on the left is continued so long as the argument of the function is not negative; $\Gamma(n)$ denotes the sum of those divisors of $n$ which are $< \sqrt{n}$ and of the same parity as their conjugate divisors, a weight $\frac{1}{2}$ being attached to $\sqrt{n}$ if this is a divisor; $I(n)$ represents the sum of those divisors of $n$ whose conjugates are uneven.

Kronecker proved his formula originally by comparing two expressions for the degree of the modular equations in the complex multiplication of elliptic functions. He and Hermite found other proofs from the expansion as Fourier series of products and quotients of thetafunctions. The real difficulty here of course is the selection of the appropriate functions. Many very important results were found by other writers especially Gierster, Hurwitz, who developed his theory of modular correspondences, Petr and Humbert. A direct method was developed by myself~\cite{Mordell1916} depending upon expansions involving certain integral functions connected with the thetafunctions. A more detailed account of the whole subject and references will be found in Chapter 6 of the third volume of Dickson's History of the Theory of Numbers.

The plan of this paper is as follows. In Section~\ref{sec:integral_function}, I introduce an integral function, $f(x)$, the study of which led me to consider these integrals. It is shown in Section~\ref{sec:standard_forms} that the general integral can be reduced to three standard forms corresponding to the cases $\Re(a/c^2) < 0$, $> 0$, $= 0$. The first is evaluated in Section~\ref{sec:first_standard_form} by means of $f(x)$, and the second in Section~\ref{sec:second_standard_form} by means of an incomplete theta series. These two forms coalesce when $\Re(a/c^2) = 0$ and this is dealt with in Section~\ref{sec:coalesce_case} which also includes the evaluation of the integral in finite terms when $ia/c^2$ is a rational number. A few miscellaneous results are given in Section~\ref{sec:miscellaneous_results}. Section~\ref{sec:gauss_sums} deals again with the case when $ia/c^2$ is rational and includes the method of evaluation of the Gauss's sums, to which I was led and which I published in 1918. The case when $ia/c^2$ is irrational is resumed in Section~\ref{sec:approximate_functional_equation} and it is shown to contain, in particular, the result for the approximate functional equation of the thetafunction. Finally in Section~\ref{sec:problems}, I mention some problems awaiting solution.

A variation of the first standard form had been considered by Lerch~\cite{Lerch1892} nearly forty years ago, but I was not aware of this when I wrote my Quarterly Journal paper in 1918. He evaluated his integral in two entirely different ways. In one, it follows as an obvious consequence of some expansions involving functions similar to $f(x)$, which of course are well known in connection with the expansions of doubly periodic and allied functions. In the other, it is deduced by an application of Poisson's summation formula to a series of the type $f(x)$. His results require the application of contour integration for the transformation of various integrals. He finds from the value of his integrals, the functional equations corresponding to equations~\eqref{eq:functional_eq_1} and~\eqref{eq:functional_eq_2} and proves that their solution is unique. Some of his results are given in Section~\ref{sec:miscellaneous_results}.

My procedure is fundamentally different. After showing how I was led to consider the first standard form, I prove by simple contour integration that the integral satisfies two functional equations which define the integral uniquely. It is now a simple matter to solve these equations. The whole procedure makes comparatively little use of detailed results or expansions and is a useful addition to the known standard methods of evaluating contour integrals, especially when the results permit of evaluation in finite terms by means of elementary functions.

Another variation of the first standard form had been considered by Ramanujan~\cite{Ramanujan1919} about the same time as myself. By expressing it as double integral, he finds the functional equations satisfied by the integral. He does not solve them except in the particular cases corresponding to my rational $ia/c^2$, or when the parameters are such that the solution is given by iteration of the equations, say the first $m$ times and the second $n$ times.

\section{An Integral Function connected with the $\theta$ Functions}\label{sec:integral_function}

Write as the definition of the thetafunctions,
\begin{align}
\theta_{00}(x, \omega) &= \sum_{m=-\infty}^\infty q^{m^2} e^{2\pi imx}, \label{eq:theta00_def}\\
\theta_{10}(x, \omega) &= \sum_{m \text{ odd}} q^{m^2} e^{2\pi imx}, \label{eq:theta10_def}\\
\theta_{01}(x, \omega) &= \sum_{n=-\infty}^\infty (-1)^n q^{n^2} e^{2\pi inx}, \label{eq:theta01_def}\\
\theta_{11}(x, \omega) &= \sum_{m \text{ odd}} (-1)^{(m-1)/2} q^{m^2} e^{2\pi imx}. \label{eq:theta11_def}
\end{align}

As usual, $q = e^{\pi i\omega}$ with $\Im(\omega) > 0$. We shall sometimes write $\theta_{00}(x)$ instead of $\theta_{00}(x, \omega)$ when the argument $\omega$ is obvious and similarly in other cases. We write
\begin{equation}\label{eq:theta00_short}
\theta_{00} = \theta_{00}(0, \omega) = \sum_{m=-\infty}^\infty q^{m^2}.
\end{equation}

Some of the simpler properties of these functions are typified by
\begin{align}
\theta_{11}(x + 1) &= -\theta_{11}(x), \label{eq:theta_prop1}\\
\theta_{11}(x + \omega) &= -e^{-\pi i(2x+\omega)} \theta_{11}(x), \label{eq:theta_prop2}
\end{align}
and so $\theta_{11}(x)$ has a simple zero at the points $x = a + b\omega$ where $a, b$ are any integers. Further the thetafunction possesses a simple transformation theorem of which particular cases are
\begin{align}
\theta_{11}\left(\frac{x}{\omega}, -\frac{1}{\omega}\right) &= -i\sqrt{-i\omega} e^{\pi ix^2/\omega} \theta_{11}(x, \omega), \label{eq:theta_transform1}\\
\theta_{00}\left(\frac{x}{\omega}, -\frac{1}{\omega}\right) &= \sqrt{-i\omega} e^{\pi ix^2/\omega} \theta_{00}(x, \omega), \label{eq:theta_transform2}
\end{align}
where the radical here, and throughout this paper, denotes that value with a positive real part.

Also
\begin{align}
\theta_{00}(0, \omega) &= \prod_{n=1}^\infty (1-q^{2n})(1+q^{2n-1})^2, \label{eq:theta_product1}\\
\theta_{01}(0, \omega) &= \prod_{n=1}^\infty (1-q^{2n})(1-q^{2n-1})^2, \label{eq:theta_product2}\\
\theta_{10}(0, \omega) &= 2q^{1/4} \prod_{n=1}^\infty (1-q^{2n})(1+q^{2n})^2. \label{eq:theta_product3}
\end{align}

The integral function is defined by the series
\begin{equation}\label{eq:integral_function}
f(x, \omega) = \sum_{m \text{ odd}} \frac{q^{m^2/4} e^{\pi imx}}{1 - q^m}.
\end{equation}

It is of a type which can be defined uniquely by two equations such as
\begin{align}
f(x + 1) + f(x) &= 0, \label{eq:functional_eq_1}\\
f(x + \omega) + f(x) &= \theta_{11}(x). \label{eq:functional_eq_2}
\end{align}

For if two integral functions satisfied these equations, their difference $d(x)$ would satisfy
\begin{align}
d(x + 1) + d(x) &= 0, \label{eq:diff_eq1}\\
d(x + \omega) + d(x) &= 0, \label{eq:diff_eq2}
\end{align}
and so unless $d(x) = 0$, it would have as many poles as zeros in the parallelogram, vertices at $0, 1, \omega, 1+\omega$, as is easily seen by considering the integral
\begin{equation}\label{eq:residue_integral}
\oint \frac{d'(z)}{d(z)} dz
\end{equation}
around the parallelogram.

Some simple properties of $f(x)$ are
\begin{align}
f(-x) + f(x + \omega) &= 0, \label{eq:f_prop1}\\
f(x) - f(-x) &= \theta_{11}(2x), \label{eq:f_prop2}\\
f(a + b\omega) &= (-1)^{a+b} f(0), \label{eq:f_prop3}
\end{align}
if $a$ and $b$ are any integers, as then $\theta_{11}(a + b\omega) = 0$.

Also
\begin{equation}\label{eq:f_zero}
f(0) = \sum_{m \text{ odd}} \frac{1}{1 - q^m}
\end{equation}
as is easily proved by putting $x = \frac{1}{2}$ in the expansion
\begin{equation}\label{eq:f_expansion}
\frac{\theta_{01}(x)}{\theta_{00}(x)} = \frac{\pi}{\omega} \sum_{m \text{ odd}} \frac{q^{m^2/4} e^{\pi imx}}{1 - q^m}.
\end{equation}

Hence from equation~\eqref{eq:theta_product2}
\begin{equation}\label{eq:f_zero_expanded}
f(0) = \sum_{m \text{ odd}} \frac{1}{1 - q^m} = \frac{1}{4\pi i} \frac{d}{d\omega} \log \theta_{00}(0, \omega).
\end{equation}

The importance of functions such as $f(x)$ in class number formulae arises from two reasons. The first is that the derivative $f'(0)$ can be expressed very simply in terms of power series whose general coefficient involves $F(n)$. Thus if we take another function of the type $f(x)$ and define an integral function $f_{01}(x)$ by
\begin{equation}\label{eq:f01_def}
f_{01}(x + 1) = f_{01}(x), \quad f_{01}(x + \omega) + f_{01}(x) = \theta_{01}(x),
\end{equation}
it can be shown~\cite{Mordell1916} that
\begin{equation}\label{eq:f01_series}
\frac{f_{01}'(0)}{\theta_{01}(0)} = -4\pi i \sum_{n=1}^\infty F(n) q^n = -4\pi i \Omega(\omega),
\end{equation}
say. This gives a simple generating function for $\Omega(\omega)$. There are similar functions e.g. $f_{00}(x)$ which is really $f_{01}(x, \omega + 1)$.

The second reason is that the function
\begin{equation}\label{eq:ratio_function}
\frac{f_{01}(x) \theta_s(x)}{\theta_{00}(x)},
\end{equation}
where $\theta_s(x)$ is a thetafunction of order $s$, sometimes permits of a simple expansion from which formulae analogous to equation~\eqref{eq:class_relation} can be found for
\begin{equation}\label{eq:class_relation_general}
F(n) + 2F(n - s \cdot 1^2) + 2F(n - s \cdot 2^2) + 2F(n - s \cdot 3^2) + \cdots.
\end{equation}
See also Section~\ref{sec:problems} in this connection.

The discovery of such relations should be facilitated by the study of the function $\Omega(\omega)$ and the application of the theory of the modular functions when possible. Thus the singularities of $\Omega(\omega)$ are given by the expansion~\cite{Mordell1920}
\begin{equation}\label{eq:omega_expansion}
\Omega\left(\frac{a\tau + b}{c\tau + d}\right) = (c\tau + d) \Omega(\tau) + \frac{c}{12} + \frac{1}{2} \sum_{a,b} \left(\frac{a}{b}\right),
\end{equation}
where the double series is summed first for $a = 0, \pm 2, \pm 4, \ldots$ and then for $b = 1, 3, 5, \ldots$ in this order. The symbol $\left(\frac{a}{b}\right)$ is the Jacobi symbol of quadratic residuacity. If $a$ is not prime to $b$, $\left(\frac{a}{b}\right) = 0$ except that it equals 1 when $a = 0, b = 1$.

It becomes important now to discover some simple relation connecting $\Omega(\omega)$ and $\Omega(-1/\omega)$. I found that
\begin{equation}\label{eq:omega_relation1}
\int_{-\infty}^\infty \frac{te^{\pi i\omega t^2}}{e^{2\pi t} - 1} dt = \frac{i}{2\omega} \left[-2\Omega(\omega) + 2\Omega(-1/\omega) + \frac{1}{4}\log\theta_{00}(0, \omega)\right],
\end{equation}

\begin{equation}\label{eq:omega_relation2}
\int_{-\infty}^\infty \frac{te^{\pi i\omega t^2}}{e^{2\pi t} + 1} dt = \frac{1}{2\omega} \sum_{n=1}^\infty (-1)^{n-1} F(4n-1) q_1^{4n-1} + \frac{1}{2\omega} \sum_{n=1}^\infty (-1)^{n-1} F(2n) q_1^{2n},
\end{equation}
where $q_1 = e^{-\pi i/\omega}$.

From these can be deduced relations such as
\begin{equation}\label{eq:class_number_sum}
\sum_{n=1}^\infty \frac{F(n)}{(z+n)^2} + \sum_{n=1}^\infty \frac{F(n)}{(z+n)^2} = \frac{1}{\pi^2} \sum_{n=1}^\infty \frac{F(n) e^{2\pi i \sqrt{n} z}}{n^{1/2}} + \frac{1}{2} \sum_{n=1}^\infty \frac{4F(n) - 3G(n)}{(z+n)^2}
\end{equation}
\begin{equation}\label{eq:class_number_sum2}
= \frac{1}{4} \sum_{n=0}^\infty \frac{(-1)^n F(4n+3)}{n^2 + (4n+3)^2} + \frac{1}{4} \sum_{n=0}^\infty \frac{1}{n^2 + 1/4},
\end{equation}
where $\Re(z) > 0$.

The consideration of such questions obviously suggests the evaluation of $f'(x/\omega, -1/\omega)$ in terms of $f(x, \omega)$ and led me to the theorem of the next section and then to the general integral. The result~\eqref{eq:omega_relation1} follows on differentiating both sides of equation~\eqref{eq:main_theorem} for $x$ and putting $x = 0$. The integral~\eqref{eq:omega_relation1} had been previously considered by Ramanujan~\cite{Ramanujan1915b} and he proved the characteristic property
\begin{equation}\label{eq:ramanujan_property}
\int_{-\infty}^\infty \frac{te^{-\pi it^2/\omega}}{e^{2\pi t} - 1} dt = f(\omega) \int_{-\infty}^\infty \frac{te^{\pi i\omega t^2}}{e^{2\pi t} - 1} dt
\end{equation}
in a slightly different form and in an entirely different way. He also gives some applications to the Riemann zetafunction, but naturally the connection of the integral with the class number was unknown to him.

\section{The First Standard Form}\label{sec:first_standard_form}

I now proceed to the proof of the

\begin{theorem}\label{thm:main_theorem}
\begin{equation}\label{eq:main_theorem}
\int_{-\infty}^\infty \frac{e^{\pi i\omega t^2 - 2\pi itx}}{e^{2\pi t} - 1} dt = \frac{f(-x/\omega, -1/\omega) + i\omega f(x, \omega)}{\omega \theta_{11}(x, \omega)},
\end{equation}
where the path of integration may be taken as either the real axis of $t$ indented by the lower half of a small circle described about the origin as centre, say the path $(-\infty, \hat{0}, \infty)$, or as a straight line parallel to the real axis of $t$ and below it at a distance less than unity. Such a path may be denoted by $P_{0,-1}$. We remark again that $\Im(\omega) > 0$. The case $\Im(\omega) = 0$ will be treated in Section~\ref{sec:coalesce_case}.
\end{theorem}

\begin{proof}
For consider the function $\Phi(x)$ defined by
\begin{equation}\label{eq:phi_def}
\theta_{11}(x, \omega) \Phi(x) = f(x/\omega, -1/\omega) + i\omega f(x, \omega).
\end{equation}

It is a meromorphic function of $x$ with apparently simple poles at the points $x = a + b\omega$ where $a$ and $b$ are any integers. But when $x = a + b\omega$, the numerator of $\Phi(x)$ is
\begin{align}
&f\left(\frac{a+b\omega}{\omega}, -\frac{1}{\omega}\right) + i\omega f(a + b\omega, \omega) \\
&= (-1)^{a+b} \left[f\left(\frac{b}{\omega}, -\frac{1}{\omega}\right) + i\omega f(0, \omega)\right] = 0
\end{align}
from equation~\eqref{eq:f_zero}, and so $\Phi(x)$ really defines an integral function of $x$.

The function $\Phi(x)$ satisfies two simple functional equations, namely,
\begin{align}
\Phi(x-1) - \Phi(x) &= \sqrt{-i\omega} e^{\pi ix^2/\omega}, \label{eq:phi_functional1}\\
\Phi(x+\omega) e^{\pi i(2x+\omega)} - \Phi(x) &= -i\omega e^{\pi i(2x+\omega)}. \label{eq:phi_functional2}
\end{align}

For
\begin{align}
&\theta_{11}(x-1, \omega) \Phi(x-1) - \theta_{11}(x, \omega) \Phi(x) \\
&= f\left(\frac{x-1}{\omega}, -\frac{1}{\omega}\right) - f\left(\frac{x}{\omega}, -\frac{1}{\omega}\right) + i\omega [f(x-1, \omega) - f(x, \omega)]
\end{align}
and gives equation~\eqref{eq:phi_functional1}.

Next
\begin{align}
&e^{-\pi i(2x+\omega)} \theta_{11}(x, \omega) \Phi(x+\omega) - \theta_{11}(x, \omega) \Phi(x) \\
&= f\left(\frac{x+\omega}{\omega}, -\frac{1}{\omega}\right) + i\omega [f(x+\omega, \omega) - f(x, \omega)] \\
&= f\left(\frac{x}{\omega} + 1, -\frac{1}{\omega}\right) + i\omega [\theta_{11}(x, \omega) - f(x, \omega)] \\
&= \theta_{11}(x, \omega) \Phi(x) + i\omega \theta_{11}(x, \omega),
\end{align}
and gives equation~\eqref{eq:phi_functional2}.

These equations define uniquely the integral function $\Phi(x)$. For if $d(x)$ is the difference of two integral functions satisfying these equations, then
\begin{align}
d(x-1) - d(x) &= 0, \label{eq:d_functional1}\\
d(x+\omega) - e^{\pi i(2x+\omega)} d(x) &= 0. \label{eq:d_functional2}
\end{align}

Hence if $d(x)$ is not identically zero, it must have at least one pole in the parallelogram, vertices at $0, 1, \omega, 1+\omega$, as is easily seen from
\begin{equation}\label{eq:d_residue}
\oint \frac{d'(z)}{d(z)} dz.
\end{equation}

Another form for the solution of equations~\eqref{eq:phi_functional1} and~\eqref{eq:phi_functional2} can be found on noting that if $a$ and $\sqrt{b}$ have positive real parts,
\begin{equation}\label{eq:gaussian_integral}
\int_0^\infty e^{-at - b/t} dt = \sqrt{\frac{\pi}{b}} e^{-2\sqrt{ab}},
\end{equation}
where the path of integration is $P_{0,-1}$. Hence equation~\eqref{eq:phi_functional1} can be written as
\begin{equation}\label{eq:phi_integral_form}
\Phi(x-1) - \Phi(x) = \int_{P_{0,-1}} \frac{e^{\pi i\omega t^2 - 2\pi itx}}{e^{2\pi t} - 1} dt.
\end{equation}

This is evidently satisfied by the following value for $\Phi(x)$,
\begin{equation}\label{eq:phi_solution}
\Phi(x) = \omega \int_{P_{0,-1}} \frac{e^{\pi i\omega t^2 - 2\pi itx}}{e^{2\pi t} - 1} dt,
\end{equation}
the integral being taken along $P_{0,-1}$. But the integral is obviously an integral function of $x$, and it will now be shown that it also satisfies equation~\eqref{eq:phi_functional2}, so that the integral is really another form for $\Phi(x)$.

For consider the integral
\begin{equation}\label{eq:contour_integral}
\oint \frac{e^{\pi i\omega t^2 - 2\pi it(x+\omega)}}{e^{2\pi t} - 1} dt
\end{equation}
taken along the sides of a rectangle with vertices $A, B, C, D$ at the points $\pm X - i\lambda, \pm X + i(1-\lambda)$ where $\lambda$ is any fixed real number with $0 < \lambda < 1$. Make $X \to \infty$. The integrals along the sides $AD, CB \to 0$, for writing $t = \pm X + in$ so that $-\lambda \leq n \leq 1-\lambda$, the modulus of the integrand is
\begin{equation}\label{eq:integrand_bound}
O(e^{\pi i\omega X^2 - \mu X}),
\end{equation}
where $\mu$ is independent of $n$. The integral along $BA \to \Phi(x)$, and that along $DC$ to $-e^{-\pi i(2x+\omega)} \Phi(x+\omega)$ on writing $t + i$ for $t$ and noting that
\begin{equation}\label{eq:shift_identity}
\pi i\omega(t+i)^2 - 2\pi it(x+\omega) = \pi i\omega t^2 - 2\pi itx - \pi i(2x+\omega).
\end{equation}

The integrand is analytic in $ABCD$ except for a simple pole at $t = 0$ with residue $\frac{1}{2\pi i}$.

Hence by Cauchy's theorem
\begin{equation}\label{eq:cauchy_result}
\Phi(x) - e^{-\pi i(2x+\omega)} \Phi(x+\omega) = i\omega,
\end{equation}
and the identification of $\Phi(x)$ with the integral is completed.
\end{proof}

The integral arising when 1 in the denominator is replaced by any constant $d = e^{\pi i\lambda}$, say, has its value given by
\begin{equation}\label{eq:general_constant}
\int_{-\infty}^\infty \frac{e^{\pi i\omega t^2 - 2\pi itx}}{e^{2\pi t} - e^{\pi i\lambda}} dt = e^{-\pi i(x^2/\omega + 2x\lambda/\omega + \lambda^2/\omega)} \frac{f(x + \lambda\omega/\omega, -1/\omega) + i\omega f(x + \lambda\omega, \omega)}{\omega \theta_{11}(x + \lambda\omega, \omega)},
\end{equation}
where there is no loss of generality in assuming that $0 \leq \Re(\lambda) < 1$. The path of integration is the real axis, but if $\lambda$ is purely imaginary, i.e. $\Re(\lambda) = 0$, the real axis is indented by the lower half of a small circle described about the point $t = i\lambda$.

For on writing $i\lambda + t$ for $t$, the integral becomes
\begin{equation}\label{eq:shifted_integral}
e^{\pi i(\lambda^2\omega + 2\lambda x + x^2)} \int_{-i\lambda}^\infty \frac{e^{\pi i\omega t^2 - 2\pi it(x+\lambda\omega)}}{e^{2\pi t} - 1} dt.
\end{equation}

The path of integration is of the type $P_{0,-1}$ and the result is given at once by equation~\eqref{eq:main_theorem}.

\section{The Second Standard Form}\label{sec:second_standard_form}

It will be shown in Section~\ref{sec:standard_forms} that the general integral can be reduced to two standard forms of which equation~\eqref{eq:main_theorem} is the first. The second is the integral
\begin{equation}\label{eq:second_standard}
\int_{-\infty}^\infty \frac{e^{\pi i\omega t^2 - 2\pi itx}}{e^{2\pi\omega t} - 1} dt
\end{equation}
along the path $(-\infty, \hat{0}, \infty)$ or $(-\infty, \hat{\delta}, \infty)$, and which we now evaluate. We still suppose $\Im(\omega) > 0$. The case $\Im(\omega) = 0$ will be dealt with in Section~\ref{sec:coalesce_case}. The imaginary axis of $\omega$ is a line of essential singularities of the integral considered as a function of $\omega$, so that it will be necessary to distinguish the cases when $\Re(\omega) > 0$, $\Re(\omega) < 0$, and $\Re(\omega) = 0$.

Write $q = e^{\pi i\omega}$,
\begin{align}
\theta(x, \omega) &= 1 + q e^{2\pi ix} + q^4 e^{8\pi ix} + q^9 e^{18\pi ix} + \cdots, \label{eq:theta_def}\\
\phi(x, \omega) &= 1 + q e^{-2\pi ix} + q^4 e^{-8\pi ix} + q^9 e^{-18\pi ix} + \cdots, \label{eq:phi_def}
\end{align}
so that $\theta(x, \omega)$, $\phi(x, \omega)$ are parts of the series for $\theta_{00}(x, \omega)$ and
\begin{equation}\label{eq:theta_phi_sum}
\theta(x, \omega) + \phi(x, \omega) = \theta_{00}(x, \omega).
\end{equation}

Then, if $\Re(\omega) > 0$, for the path $(-\infty, \hat{0}, \infty)$,
\begin{equation}\label{eq:second_form_case1}
\int_{-\infty}^\infty \frac{e^{\pi i\omega t^2 - 2\pi itx}}{e^{2\pi\omega t} - 1} dt = \frac{e^{\pi ix^2/\omega} \theta(x, \omega)}{\sqrt{-i\omega}},
\end{equation}
which on noting equations~\eqref{eq:theta_phi_sum} and~\eqref{eq:theta_transform2} can be written as
\begin{equation}\label{eq:second_form_case1_alt}
\int_{-\infty}^\infty \frac{e^{\pi i\omega t^2 - 2\pi itx}}{e^{2\pi\omega t} - 1} dt = \frac{e^{\pi ix^2/\omega} \phi(x, \omega)}{\sqrt{-i\omega}} + \theta_{00}\left(\frac{x}{\omega}, -\frac{1}{\omega}\right).
\end{equation}

It is shown in Section~\ref{sec:coalesce_case} that if $\Im(\omega) = 0$ and $\omega > 0$, then equation~\eqref{eq:second_form_case1} holds when $\Im(x) > 0$, and equation~\eqref{eq:second_form_case1_alt} when $\Im(x) < 0$.

If $\Re(\omega) < 0$, for the path $(-\infty, \hat{\delta}, \infty)$,
\begin{equation}\label{eq:second_form_case2}
\int_{-\infty}^\infty \frac{e^{\pi i\omega t^2 - 2\pi itx}}{e^{2\pi\omega t} - 1} dt = \frac{e^{\pi ix^2/\omega} \phi(x, \omega)}{\sqrt{-i\omega}} + 0,
\end{equation}

\begin{equation}\label{eq:second_form_case2_alt}
\int_{-\infty}^\infty \frac{e^{\pi i\omega t^2 - 2\pi itx}}{e^{2\pi\omega t} - 1} dt = \frac{e^{\pi ix^2/\omega} \theta(x, \omega)}{\sqrt{-i\omega}} - \theta_{00}\left(\frac{x}{\omega}, -\frac{1}{\omega}\right).
\end{equation}

If $\Im(\omega) = 0, \omega < 0$, then equation~\eqref{eq:second_form_case2} holds when $\Im(x) < 0$ and equation~\eqref{eq:second_form_case2_alt} when $\Im(x) > 0$.

Finally if $\Re(\omega) = 0$,
\begin{equation}\label{eq:second_form_case3}
\int_{-\infty}^\infty \frac{e^{\pi i\omega t^2 - 2\pi itx}}{e^{2\pi\omega t} - 1} dt = \frac{e^{\pi ix^2/\omega} \theta(x, \omega)}{\sqrt{-i\omega}},
\end{equation}
where the path of integration is the real axis of $t$ indented by the lower halves of small circles described around the points $ni/\omega$, $(n = 0, \pm 1, \pm 2, \ldots)$.

\section{The Case when the Standard Forms coalesce}\label{sec:coalesce_case}

There is, however, another way of solving equations~\eqref{eq:phi_functional1} and~\eqref{eq:phi_functional2} which has the great advantage of also giving the evaluation of the general integral when $\omega$ is real. Then the functions $f(x, \omega)$, $\theta_{00}(x, \omega)$ no longer exist but some of the results of Sections~\ref{sec:first_standard_form} and~\ref{sec:second_standard_form} are still valid. Clearly the two standard forms are the same since the integrals~\eqref{eq:second_form_case1} to~\eqref{eq:second_form_case3} reduce to equation~\eqref{eq:main_theorem} when $t$ is replaced by $t/\omega$.

Suppose then $\omega$ is real. The integrals~\eqref{eq:second_form_case1} to~\eqref{eq:second_form_case3} (and also equation~\eqref{eq:main_theorem}) converge uniformly in $x$ for all bounded $x$ and hence are integral functions of $x$. Thus take the integral~\eqref{eq:second_form_case1}, say $J$, and consider the behaviour of the integrand at the limits of summation $t = \pm \infty$. Clearly at $t = \infty$, $J$ converges absolutely and uniformly when $\Re(x) = -\omega + \epsilon > -\omega$ and $\epsilon$ is small. At $t = -\infty$ $J$ obviously converges absolutely and uniformly for $\Re(x) \leq -\epsilon < 0$. But $J$ also converges uniformly for $0 \geq \Re(x) \geq -\omega$. Thus near $\Re(x) = -\omega$, the convergence of the integral at only $t = \infty$ need be considered. Write
\begin{equation}\label{eq:splitting}
\frac{e^{\pi i\omega t^2 - 2\pi itx}}{e^{2\pi\omega t} - 1} = \frac{e^{\pi i\omega t^2 - 2\pi t(x+\omega)} + e^{-2\pi t(x+\omega)}}{e^{2\pi\omega t} - 1}.
\end{equation}

Since
\begin{equation}\label{eq:convergence1}
\int_\infty e^{\pi i\omega t^2 - 2\pi t(x+\omega)} dt
\end{equation}
converges uniformly for bounded $x$ and
\begin{equation}\label{eq:bound1}
|e^{-2\pi i(x+\omega)}| \leq e^{2\pi \epsilon t},
\end{equation}
the result follows since $\omega > 0$ for $J$. Similarly near $\Re(x) = 0$.

If $\omega$ is real and positive, the functional equations~\eqref{eq:phi_functional1} and~\eqref{eq:phi_functional2} still hold. This is obvious for equation~\eqref{eq:phi_functional1} from~\eqref{eq:phi_integral_form}. It suffices to prove equation~\eqref{eq:phi_functional2} for $0 > \Re(x) > -\omega$ as both sides are integral functions of $x$.

The path of integration in equation~\eqref{eq:phi_solution} can be deformed into the inclined path $(-\infty e^{i\alpha}, 0, \infty e^{i\alpha})$, where $\alpha$ is any fixed positive acute angle, by the crude argument of absolute convergence, since for real $X$,
\begin{equation}\label{eq:convergence_bound}
|e^{\pi i\omega X^2 e^{2i\alpha}}| \leq 1
\end{equation}
for $0 \leq \alpha \leq \frac{\pi}{2}$.

The argument leading to equation~\eqref{eq:phi_functional2} now applies as the integral along the side joining the points $Xe^{i\alpha}, Xe^{-i\alpha}$ vanishes when $X \to \infty$, since
\begin{equation}\label{eq:vanishing_bound}
|e^{\pi i\omega X^2 e^{2i\alpha}}| \to 0.
\end{equation}

We shall now solve the equations by iteration. Let $m$ and $n$ be positive integers. Then equation~\eqref{eq:phi_functional1} can be written as
\begin{equation}\label{eq:iteration1}
e^{\pi i(x+n\omega)^2/\omega} \chi(x + n\omega) - e^{\pi ix^2/\omega} \chi(x) = -\frac{i}{\omega} \frac{e^{\pi i(x+r\omega)^2/\omega}}{\sqrt{-i\omega}} \sum_{r=1}^n e^{\pi i(2rx + r^2\omega)}.
\end{equation}

So from equation~\eqref{eq:phi_functional2}
\begin{equation}\label{eq:iteration2}
\chi(x-m) - \chi(x) = \frac{i}{\omega} \frac{e^{\pi i(x-r)^2/\omega}}{\sqrt{-i\omega}} \sum_{r=1}^m e^{\pi i(2rx + r^2\omega)}.
\end{equation}

Change $x$ into $x-m$ in equation~\eqref{eq:iteration1} and apply equation~\eqref{eq:iteration2}, then
\begin{equation}\label{eq:iteration_combined}
\chi(x) = -\frac{i}{\omega} \frac{e^{\pi i(x-r)^2/\omega}}{\sqrt{-i\omega}} \sum_{r=1}^m e^{\pi i(2rx + r^2\omega)} + \frac{i}{\omega} \frac{e^{\pi i(x-m+s\omega)^2/\omega}}{\sqrt{-i\omega}} \sum_{s=1}^n e^{\pi i(2sx + s^2\omega)} + e^{\pi i(2nx + n^2\omega)} \chi(x - m + n\omega),
\end{equation}
where
\begin{equation}\label{eq:chi_bound}
e^{\pi i(2nx + n^2\omega)} \chi(x - m + n\omega) = e^{\pi i(x-m)^2/\omega} \int_{-\infty}^\infty \frac{e^{\pi i\omega t^2 - 2\pi t(x-m+n\omega)}}{e^{2\pi\omega t} - 1} dt.
\end{equation}

When $\omega$ is a rational number, say $\omega = \frac{a}{b}$, where $a$ and $b > 0$ are integers, the integrals can be evaluated in finite terms.

For
\begin{equation}\label{eq:rational_sum1}
1 + \theta(x, \omega) = \sum_{n=0}^\infty e^{\pi in^2\omega + 2\pi inx} = \sum_{r=0}^{b-1} e^{\pi ir^2a/b + 2\pi irx} \sum_{N=0}^\infty e^{2\pi iNbx + \pi iN^2b^2a}.
\end{equation}

Put $n = r + Nb$, $(r = 0, 1, 2, \ldots, b-1, N = 0, 1, \ldots)$ then
\begin{equation}\label{eq:rational_sum2}
1 + \theta(x, \omega) = \sum_{r=0}^{b-1} \frac{e^{\pi ir^2a/b + 2\pi irx}}{1 - e^{\pi ib(2x+a)}}.
\end{equation}

Similarly
\begin{equation}\label{eq:rational_sum3}
\phi(x, \omega) = \sum_{r=0}^{b-1} \frac{e^{\pi ir^2a/b - 2\pi irx}}{1 - e^{\pi ib(-2x+a)}}.
\end{equation}

It is easy to see that with these values of $\theta(x, \omega)$ etc. that equation~\eqref{eq:second_form_case1} holds when $\omega$ is a positive rational number not only for $\Im(x) > 0$ but for all $x$. Arguments similar to those for equation~\eqref{eq:second_form_case1} also apply to equations~\eqref{eq:second_form_case1_alt}, \eqref{eq:second_form_case2}, \eqref{eq:second_form_case2_alt}.

Some of these results were given in a different form by Ramanujan~\cite{Ramanujan1915b} but my proofs are entirely different from his.

Finally when $\omega > 0$ and irrational, and $x$ is also real, the uniform convergence of the integral~\eqref{eq:chi_bound} in $x$ shows that if $m$ and $n \to +\infty$ in such a way that $x - m + n\omega \to 0$,
\begin{align}
\chi(x - m + n\omega) &\to \chi(0) = \int_{-\infty}^\infty \frac{e^{\pi i\omega t^2}}{e^{2\pi\omega t} - 1} dt, \label{eq:limit1}\\
&= \frac{i}{2\omega} = \frac{i}{2\sqrt{-i\omega}}, \label{eq:limit2}
\end{align}
as is obvious from the derivation of equation~\eqref{eq:chi_bound}.

Also
\begin{equation}\label{eq:exponential_limit}
e^{\pi i(2nx + n^2\omega)} \sim e^{\pi i(x-m)^2/\omega},
\end{equation}

Hence
\begin{equation}\label{eq:final_limit}
\chi(x) = \lim_{m,n \to \infty} \left[ \sum_{r=1}^m e^{\pi i(x-r)^2/\omega} + \sum_{s=1}^n e^{\pi i(x-m+s\omega)^2/\omega} \right] + e^{\pi i(2nx + n^2\omega)},
\end{equation}
the dashes denoting that the particular terms $r = m$ and $s = n$ have weights $\frac{1}{2}$.

Hence if $\omega > 0$, $x$ is real and the path is $(-\infty, \hat{0}, \infty)$,
\begin{equation}\label{eq:irrational_formula}
\int_{-\infty}^\infty \frac{e^{\pi i\omega t^2 - 2\pi itx}}{e^{2\pi\omega t} - 1} dt = \lim_{m,n \to \infty} \left[ \sum_{r=1}^{m'} \frac{e^{\pi i(x-r)^2/\omega}}{\sqrt{-i\omega}} + \sum_{s=1}^{n'} \frac{e^{\pi i(x-m+s\omega)^2/\omega}}{\sqrt{-i\omega}} \right],
\end{equation}
where $m \to +\infty$, $n \to +\infty$, $x - m + n\omega \to 0$.

By writing $-t$ for $t$, $-i$ for $i$ and $-x$ for $x$, we see that for $\omega < 0$ and the path $(-\infty, \hat{0}, \infty)$,
\begin{equation}\label{eq:negative_omega_formula}
\int_{-\infty}^\infty \frac{e^{\pi i\omega t^2 - 2\pi itx}}{e^{2\pi\omega t} - 1} dt = \lim_{m,n \to \infty} \left[ \sum_{r=1}^{m'} \frac{e^{\pi i(-x-r)^2/\omega}}{\sqrt{-i\omega}} + \sum_{s=1}^{n'} \frac{e^{\pi i(-x+m+s\omega)^2/\omega}}{\sqrt{-i\omega}} \right],
\end{equation}
where $m \to +\infty$, $n \to +\infty$, $x + m + n\omega \to 0$.

\section{Reduction of the Integral to the Standard Forms}\label{sec:standard_forms}

There remains now the reduction to the standard forms in Sections~\ref{sec:first_standard_form}, \ref{sec:second_standard_form}, \ref{sec:coalesce_case} of the integral
\begin{equation}\label{eq:general_integral}
I = \int_{-\infty}^\infty \frac{e^{at^2 + bt}}{e^{ct} + d} dt,
\end{equation}
where the path of integration is the real axis of $t$ indented by the lower halves of any zeros of the denominator. The case $ac = 0$ may be omitted as these results are well known. The convergence of $I$ requires then $\Re(a) \leq 0$.

If $c$ is real, $I$ reduces at once to the first standard form including the case of real $\omega$ in Section~\ref{sec:coalesce_case}. Hence on writing $-t$ for $t$ if need be, we can suppose that $c$ is a complex number with $\Re(c) \geq 0$. We consider now the case when $|\Re(a)| < 0$.

Three cases arise according as the real part of $a/c^2$ is negative, positive, or zero.

Suppose first $\Re(a/c^2) < 0$. This implies $\Re(c) \neq 0$, for $\Re(c) = 0$ would make $\Re(a) > 0$. Put
\begin{equation}\label{eq:substitution1}
t = \frac{2\pi v}{c}, \quad \pi i\omega = \frac{4\pi^2 a}{c^2}, \text{ i.e. } \Im(\omega) > 0,
\end{equation}
and $I$ takes the form
\begin{equation}\label{eq:reduced_form1}
I = \frac{2\pi}{c} \int_{-\infty}^\infty \frac{e^{\pi i\omega v^2 - 2\pi iav}}{e^{2\pi v} + d} dv.
\end{equation}

The path of integration is now the line through the origin indented by the lower halves of the zeros of the denominator and inclined to the real axis of $v$ at an angle $\arg c$ where
\begin{equation}\label{eq:angle_constraint}
\frac{\pi}{2} > \arg c > 0,
\end{equation}
since $\Re(c) > 0$.

Consider now the integral
\begin{equation}\label{eq:contour_deformation}
\oint \frac{e^{\pi i\omega v^2 - 2\pi iav}}{e^{2\pi v} + d} dv
\end{equation}
taken around the contour formed by the two lines (indented if necessary) joining the points $-\rho, \rho$ (i.e. the real axis) and the points $-\rho c, \rho c$ and the two arcs of the circle $|v| = \rho$ joining the points $\rho, \rho c$ and $-\rho, -\rho c$. When $\rho \to \infty$, the integrals along the arcs tend to zero since $\Re(a) < 0$. For putting $v = \rho e^{i\theta}$, then $\Im(\omega v^2) > 0$ if $0 < \arg \omega + 2\arg c < \pi$, i.e. $0 < \arg a/i < \pi$, which holds since $\Re(a) < 0$.

The zeros of the denominator are of the form $v = ni + d$, say where $n = 0, \pm 1, \pm 2, \ldots$ Only a finite number of them lie within the contour of integration since $\Re(c) \neq 0$. Hence the value of the integral
\begin{equation}\label{eq:contour_result}
\oint \frac{e^{\pi i\omega v^2 - 2\pi iav}}{e^{2\pi v} + d} dv
\end{equation}
is given by Cauchy's theorem in terms of the integral
\begin{equation}\label{eq:real_axis_integral}
\int_{-\infty}^\infty \frac{e^{\pi i\omega v^2 - 2\pi iav}}{e^{2\pi v} + d} dv
\end{equation}
which was evaluated in Section~\ref{sec:first_standard_form}.

Suppose next $\Re(a/c^2) > 0$. Put $t = \frac{2\pi v}{c}$ and then $I$ takes the form
\begin{equation}\label{eq:reduced_form2}
\frac{c}{2\pi} \int_{-\infty}^\infty \frac{e^{\pi i\omega v^2 - 2\pi iav}}{e^{c\omega v} + d} dv,
\end{equation}
provided $\pi i\omega = \frac{4\pi^2 a}{c^2}$, i.e.
\begin{equation}\label{eq:omega_relation}
\omega = \frac{ic^2}{4\pi a},
\end{equation}
so that $\Im(\omega) > 0$.

The path of integration can be deformed into the real axis just as before, since $\Re(a) < 0$, and the integral reduces to the second standard form. Allowance must of course be made for the zeros of the denominator which are now of the form $ni/\omega + d$, say, $(n = 0, \pm 1, \pm 2, \ldots)$. There may be an infinite number of them within the contour, e.g. if $0 < \arg(i/\omega) < \arg(c/\omega)$, but the series of residues converges absolutely since
\begin{equation}\label{eq:convergence_condition}
\Im\left[\frac{i}{\omega}\right] = \Re\left(\frac{1}{\omega}\right) > 0,
\end{equation}
and reduces to a thetafunction with the omission of a finite number of terms.

Thirdly, suppose $\Re(a/c^2) = 0$. Either of the two methods of reduction suffices since $\omega$ is real and the integral $I$ reduces to the special cases treated in Section~\ref{sec:coalesce_case}.

There still remains the case when $\Re(a) = 0$. It suffices to assume $\Im(a) > 0$ as the results when $\Im(a) < 0$ can be deduced by changing the sign of $i$ throughout. On writing $-v$ for $v$, we may still suppose $\Re(c) \geq 0$.

When $\Im(c^2) < 0$, (and so $\Re(c) \neq 0$), we use equations~\eqref{eq:substitution1} and~\eqref{eq:reduced_form1} still holds. It is then easy to see that when $0 > \Re(x) > -1$, the deformation of the path of integration into the real axis and the reduction to the first standard form still hold. The result then holds for all $x$ by the usual argument.

When $\Im(c^2) > 0$, (and so $\Re(c) \neq 0$), we use equation~\eqref{eq:reduced_form2} and as above the integral is reduced to the second standard form.

When $\Im(c^2) = 0$, we may suppose $c$ is purely imaginary, since we have already considered the case when $c$ is real. We may use either equations~\eqref{eq:reduced_form1} or~\eqref{eq:reduced_form2} and the integral is reduced to the form
\begin{equation}\label{eq:real_omega_case}
\int_{-\infty}^\infty \frac{e^{\pi i\omega v^2 - 2\pi iav}}{e^{2\pi v} + d} dv,
\end{equation}
where $\omega$ is real. The value of this integral is included in the results of Section~\ref{sec:coalesce_case}.

When $0 > \Re(x) > -1$, the path of integration can be deformed into the real axis from $-\infty$ to $+\infty$, indented if need be. The residues arising from the infinite number of poles of the denominator, i.e. $v = ni + d$ give rise to a convergent series of the type $\theta(x, \omega)$, or $\phi(x, \omega)$. Thus if $\Re(d) > 0$, only the values $n = 0, 1, 2, 3, \ldots$ can arise and then
\begin{equation}\label{eq:residue_series}
\sum_n e^{\pi i\omega(ni + d)^2 - 2\pi ia(ni + d)}
\end{equation}
converges absolutely if $-\Im(\omega)\Re(d) - 2\Im(x) > 0$, i.e. if $2\Im(x) - \omega\Re(d) > 0$. The results for other values of $x$ follow as with equations~\eqref{eq:second_form_case1} to~\eqref{eq:second_form_case3} when $\omega$ is real.

\section{Miscellaneous Results}\label{sec:miscellaneous_results}

We consider in the present section various applications of and remarks about the results of the paper.

The functional equations~\eqref{eq:phi_functional1} and~\eqref{eq:phi_functional2} admit also of solutions of the type~\eqref{eq:phi_def} but with denominator $\theta_{00}(x, \omega)$ etc. instead of $\theta_{11}(x, \omega)$. Thus if in equation~\eqref{eq:general_constant}, $\lambda = -\frac{1}{2}$, we find on noting
\begin{equation}\label{eq:theta_identity}
\theta_{11}\left(x + \frac{1}{4}\omega, \omega\right) = iq^{1/4}e^{-\pi ix}\theta_{01}(x, \omega),
\end{equation}
that
\begin{equation}\label{eq:alternative_form1}
\int_{-\infty}^\infty \frac{e^{\pi i\omega t^2 - 2\pi itx}}{e^{2\pi t} + 1} dt = \frac{f(-x/\omega - \frac{1}{4}, -\frac{1}{\omega}) + i\omega f(x - \frac{\omega}{4}, \omega)}{\omega \theta_{01}(x, \omega)}.
\end{equation}

It is easy to show that one of the alternative forms referred to above of this result is
\begin{equation}\label{eq:alternative_form2}
\int_{-\infty}^\infty \frac{e^{\pi i\omega t^2 - 2\pi itx}}{e^{2\pi t} + 1} dt = \frac{\theta_{00}(x, \omega)}{2} - \frac{1}{2\pi i} \sum_{n=-\infty}^\infty \frac{q^{n^2-1/4} e^{4(2n-1)\pi ix}}{1 - q^{2n-1}}.
\end{equation}

Further equations~\eqref{eq:phi_functional1} and~\eqref{eq:phi_functional2} can also be solved by means of integrals of a type different from equation~\eqref{eq:phi_integral_form}.

This is suggested by writing in equation~\eqref{eq:main_theorem}, $x/\omega$, $-1/\omega$ in place of $x$, $\omega$ respectively. We have
\begin{equation}\label{eq:transformed_integral}
\int_{-\infty}^\infty \frac{e^{-\pi it^2/\omega - 2\pi itx/\omega}}{e^{2\pi t} - 1} dt = \frac{-\omega f(x, \omega) + if(-x/\omega, -1/\omega)}{\theta_{11}(x/\omega, -1/\omega)}
\end{equation}
and on noting equation~\eqref{eq:theta_transform1}, this becomes
\begin{equation}\label{eq:simplified_form}
\int_{-\infty}^\infty \frac{e^{-\pi i(t-ix)^2/\omega}}{e^{2\pi t} - 1} dt = \frac{-\omega f(-x, \omega) - if(-x/\omega, -1/\omega)}{\omega \sqrt{-i\omega}\theta_{11}(x, \omega)}.
\end{equation}

From equation~\eqref{eq:main_theorem} and equation~\eqref{eq:f_prop1}, we have, along the path $(-\infty, \hat{0}, \infty)$
\begin{equation}\label{eq:combined_result}
\int_{-\infty}^\infty \frac{e^{\pi i\omega t^2 - 2\pi itx}}{e^{2\pi t} - 1} dt + i\sqrt{-i\omega} \int_{-\infty}^\infty \frac{e^{-\pi i(t-ix)^2/\omega}}{e^{2\pi t} - 1} dt = -\sqrt{-i\omega}.
\end{equation}

We can deduce from this that
\begin{equation}\label{eq:reciprocal_relation}
\frac{1}{\sqrt{-i\omega}} \int_{-\infty}^\infty \frac{e^{-\pi i(t-ix)^2/\omega}}{e^{2\pi t} - 1} dt = \int_{-\infty}^\infty \frac{e^{\pi i\omega t^2 - 2\pi itx}}{e^{2\pi t} - 1} dt.
\end{equation}

This result can also be proved by noting that both sides are integral solutions of the equations
\begin{align}
\Phi(x + \omega) + e^{\pi i(2x+\omega)} \Phi(x) &= 2\omega e^{\pi ix^2/\omega}, \label{eq:reciprocal_eq1}\\
\Phi(x - 1) + \Phi(x) &= \frac{2e^{\pi i(x-1)^2/\omega}}{\sqrt{-i\omega}}, \label{eq:reciprocal_eq2}
\end{align}
which cannot admit of more than one integral solution.

Formulae of the type equations~\eqref{eq:combined_result} and~\eqref{eq:reciprocal_relation} have been found by Hardy~\cite{Hardy1904} and Ramanujan~\cite{Ramanujan1915b} by considering reciprocal functions.

The work of Lerch~\cite{Lerch1892}, however, shows that the unique integral solution of the equations~\eqref{eq:phi_functional1} and~\eqref{eq:phi_functional2} can be expressed in an infinity of ways. Thus he proves that if
\begin{equation}\label{eq:lerch_function}
R(u, v, \omega) = \sum_{m,n=-\infty}^\infty \frac{1}{u + v + m + n\omega},
\end{equation}
then the value of the integral
\begin{equation}\label{eq:lerch_integral}
\Psi(v, \omega) = \int_{-\infty}^\infty \frac{e^{\pi i\omega t^2 - 2\pi itv}}{e^{2\pi t} + 1} dt
\end{equation}
is given by
\begin{equation}\label{eq:lerch_result}
\Psi(v, \omega) = \frac{\pi i}{\omega} R\left(\frac{u}{\omega}, \frac{u + v}{\omega}, -\frac{1}{\omega}\right),
\end{equation}
where $u$ is arbitrary. The proof along the lines developed in this paper would not be difficult. Thus both sides of equation~\eqref{eq:lerch_result} are integral functions of $v$ since the residues of the two terms in $R$ cancel for the simple poles at the points $v = -u + m - n\omega$ where $m, n$ are any integers. The functional equations for $\Psi(v, \omega)$ analogous to equation~\eqref{eq:reciprocal_eq1} can be deduced, and then a proof of equation~\eqref{eq:lerch_result} would follow from a study of the functional equations satisfied by $R(u, v, \omega)$.

It is easily seen that $R(u, v, \omega)$ is defined uniquely as the meromorphic function of $v$ satisfying the equations
\begin{align}
R(u, v + 1, \omega) - R(u, v, \omega) &= 0, \label{eq:R_functional1}\\
R(u, v, \omega) - e^{\pi i(2v+\omega)} R(u, v + \omega, \omega) &= \theta_{00}(u, \omega), \label{eq:R_functional2}
\end{align}
and whose only singularities are simple poles at the points $m + n\omega$, where $m, n$ are any integers, and which has a residue $1/2\pi i$ at $v = 0$. There is, however, no need to go into details.

\section{The Case when $ia/c^2$ is Rational and the Gauss's Sums}\label{sec:gauss_sums}

The results of Section~\ref{sec:coalesce_case} when $\omega$ is rational contain the formula for the Gauss's sums. As they are also interesting examples on elementary contour integration, it may be desirable to state the results in their simplest forms and to sketch briefly the proofs again, independently of Section~\ref{sec:coalesce_case}.

\begin{theorem}\label{thm:gauss_sums}
If $a$ and $b$ are positive integers, then
\begin{equation}\label{eq:gauss_formula1}
\{1 - e^{\pi i(2x-a)}\} \int_{-\infty}^\infty \frac{e^{-\pi iat^2/b - 2\pi itx}}{e^{2\pi t} - 1} dt = \frac{1}{\sqrt{ib}} \sum_{r=0}^{a-1} e^{\pi i(x+r)^2/a + \pi i(2x-a)r/a},
\end{equation}
where the path of integration is either $(-\infty, \hat{0}, \infty)$, or is a straight line inclined to the real axis of $t$ at an acute negative angle and meeting the imaginary axis between $t=0$ and $t=-i$.

\begin{equation}\label{eq:gauss_formula2}
\{1 - e^{\pi i(2x-a)}\} \int_{-\infty}^\infty \frac{e^{-\pi iat^2/b - 2\pi itx}}{e^{2\pi t} - 1} dt = \frac{e^{\pi ix^2/a}}{\sqrt{ib}} \sum_{r=1}^{a-1} e^{\pi i(x+r)^2/a + \pi i(2x-a)r/a},
\end{equation}
along either the path $(-\infty, \hat{0}, \infty)$, or along a line inclined to the real axis of $t$ at a positive acute angle and meeting the imaginary axis of $t$ between $t = 0$, $t = -i$.
\end{theorem}

\begin{proof}
For write when $\omega$ is real
\begin{equation}\label{eq:psi_definition}
\Psi(x) = \int_{-\infty}^\infty \frac{e^{\pi i\omega t^2 - 2\pi itx}}{e^{2\pi t} - 1} dt,
\end{equation}
with the inclined path of equation~\eqref{eq:gauss_formula1}. On noting equation~\eqref{eq:phi_integral_form}, so that $\Phi(x) = \omega \Psi(x)$, it is obvious that equation~\eqref{eq:phi_functional1} still holds when $\omega$ is real and so
\begin{equation}\label{eq:psi_functional1}
\Psi(x - 1) - \Psi(x) = \frac{e^{\pi ix^2/\omega}}{\sqrt{-i\omega}}.
\end{equation}

Also
\begin{equation}\label{eq:psi_functional2}
\Psi(x + \omega) - e^{\pi i(2x+\omega)} \Psi(x) = ie^{\pi i(2x+\omega)}.
\end{equation}

For the rectangle of Section~\ref{sec:first_standard_form} is replaced by a parallelogram ABCD, vertices at $\pm \infty e^{i\alpha}$, $\pm \infty e^{i\alpha} + i$. The integrals along the short sides of ABCD are zero if $\alpha$ is an acute angle with $\omega > 0$ because of the factor $e^{\pi i\omega t^2}$ with $t = \pm \infty e^{i\alpha} + \xi$, and $|\xi| \leq 1$. Hence equation~\eqref{eq:psi_functional2} follows.

Let now $m, n$ be positive integers. Applying equation~\eqref{eq:psi_functional1}, $m$ times,
\begin{equation}\label{eq:psi_iteration1}
\Psi(x-m) - \Psi(x) = \frac{1}{\sqrt{-i\omega}} \sum_{r=1}^{m-1} e^{\pi i(x-r)^2/\omega}.
\end{equation}

From equation~\eqref{eq:psi_functional2}
\begin{equation}\label{eq:psi_transformation}
e^{-\pi i(x+\omega)^2/\omega} \Psi(x + \omega) - e^{-\pi ix^2/\omega} \Psi(x) = -ie^{-\pi ix^2/\omega},
\end{equation}
and changing $x$ into $x + \omega$, $x + 2\omega, \ldots, x + (n-1)\omega$ and adding
\begin{equation}\label{eq:psi_iteration2}
e^{-\pi i(2nx+n^2\omega)/\omega} \Psi(x + n\omega) - \Psi(x) = -i \sum_{s=0}^{n-1} e^{\pi i(2sx+s^2\omega)/\omega}.
\end{equation}

Suppose now $\omega = -a/b$ where $a, b$ are positive integers. Take $m = a$, $n = b$. Multiply equation~\eqref{eq:psi_iteration2} by $e^{\pi i(2nx+n^2\omega)/\omega}$, write $b-s$ for $s$, subtract, and note $-a = b\omega$, then equation~\eqref{eq:gauss_formula1} follows. It is also easy to see that when $0 > \Re(x) > -1$, the inclined path can be deformed into the path $(-\infty, \hat{0}, \infty)$, and then that equation~\eqref{eq:gauss_formula1} holds for $0 \geq \Re(x) \geq -1$, while equation~\eqref{eq:psi_functional1} shows the integral really converges for all $x$ and so equation~\eqref{eq:gauss_formula1} holds for all $x$. Similarly when $\omega = a/b$, the result equation~\eqref{eq:gauss_formula2} follows. It can also be deduced from equation~\eqref{eq:gauss_formula1} by changing $i$ into $-i$, noting $(-\infty, \hat{0}, \infty)$ becomes $(-\infty, \hat{0}, \infty)$ and then writing $-t$ for $t$ and $-x-1$ for $x$.
\end{proof}

On putting $x = 0$, in equation~\eqref{eq:gauss_formula1}, the well known reciprocity formula for the Gauss's sums follows. This suggested to me the simple proof which I published in the Messenger of Mathematics~\cite{Mordell1918} that
\begin{equation}\label{eq:gauss_sum_reciprocity}
\sum_{r=0}^{n-1} e^{2\pi ir^2/n} = \sqrt{n} \frac{1+i}{\sqrt{2}}.
\end{equation}

For write
\begin{equation}\label{eq:gauss_sum_function}
f(z) = \sum_{r=0}^{n-1} e^{2\pi ir^2/n}(e^{\pi iz} - 1) = \sum_{l=0}^{n-1} e^{2\pi i(l+1)^2/n}.
\end{equation}

Then
\begin{equation}\label{eq:gauss_sum_difference}
f(z + 1) - f(z) = e^{\pi iz/n}(e^{\pi iz} + 1),
\end{equation}
so that
\begin{equation}\label{eq:gauss_sum_integral}
\oint f(z) dz = e^{\pi iz^2/n}.
\end{equation}

Consider now the integral $\oint f(z) dz$ taken around an infinite parallelogram ABCD of which the parallel sides AB, DC are inclined to the real axis of $z$ at any positive acute angle and cut it at $z = -1/2$, $z = 1/2$ respectively. The sides BC and DA are parallel to the axis of $x$ and are at an infinite distance above it and below it respectively.

Then $f(z)$ is analytic within ABCD except for a simple pole at $z=0$ with residue $S/2\pi i$.

The integral taken along each of the sides BC and DA vanishes, while those along the sides AB, CD reduce to
\begin{equation}\label{eq:contour_integral_sides}
\int_A^B (f(z) - f(z + 1)) dz = \int_A^B e^{\pi iz^2/n}(e^{\pi iz} + 1)dz.
\end{equation}

Hence by Cauchy's theorem
\begin{equation}\label{eq:gauss_sum_final}
S = \int_A^B e^{\pi iz^2/n}(e^{\pi iz} + 1)dz.
\end{equation}

The value of this integral is well known, since the path of integration can be deformed into the real axis from $-\infty$ to $\infty$. We need not use this, however, for if in the first part of the integral, we replace $z$ by $z - 1$ and then throughout put $z = y\sqrt{n}$, we have, with $k$ independent of $n$,
\begin{equation}\label{eq:gauss_sum_scaling}
S = k\sqrt{n}(1 + e^{\pi i}).
\end{equation}

Take $n = 1$, whence
\begin{equation}\label{eq:gauss_sum_normalization}
1 = k(1 + e^{\pi i}) = k \cdot 2i,
\end{equation}
and the result.

\section{The Approximate Functional Equation of the Thetafunction}\label{sec:approximate_functional_equation}

The results of Section~\ref{sec:coalesce_case} when $\omega$ is irrational include the approximate functional equation of the thetafunction. This has been the subject of papers by Hardy and Littlewood~\cite{HardyLittlewood1921,HardyLittlewood1923,HardyLittlewood1925}, van der Corput~\cite{vanderCorput1922,vanderCorput1923}, myself~\cite{Mordell1926} and Wilton~\cite{Wilton1927}. Its importance makes desirable a short treatment independent of Section~\ref{sec:coalesce_case}. I prove the

\begin{theorem}\label{thm:approximate_functional}
Let
\begin{equation}\label{eq:F_definition}
F(x) = \int_{-\infty}^\infty \frac{e^{\pi i\omega t^2 - 2\pi itx}}{e^{2\pi t} - 1} dt
\end{equation}
along the path $(-\infty, \hat{0}, \infty)$, where $\omega > 0$ and $x$ is real.

Let $m \geq 0$, $n \geq 0$ be any integers and put
\begin{equation}\label{eq:S_mn_definition}
S_{m,n} = \frac{1}{\sqrt{-i\omega}} \sum_{r=0}^m e^{\pi i(x-r)^2/\omega} + \sum_{s=0}^n e^{\pi i(x-m+s\omega)^2/\omega}.
\end{equation}

Denote by $S'_{m,n}$ the sum when weights $1/2$ are given to the terms $r=m$, $s=n$, and by $S''_{m,n}$ the sum when in addition weights $1/2$ are attached to the terms $r = 0$, $s = 0$.

Then if $m$ and $n$ are positive integers such that $0 \leq \xi \leq 1$, where
\begin{equation}\label{eq:xi_definition}
x - m + n\omega = \xi,
\end{equation}

\begin{equation}\label{eq:approximate_bound}
|F(x) + S'_{m,n}| \leq \frac{2\xi}{\omega} + \min\left(\frac{1}{\omega}, \frac{\pi\xi}{2\omega^2}\right)
\end{equation}
and so the left hand side $\to 0$ if $m, n \to +\infty$ in such a way that $x - m + n\omega \to 0$.

Also if in addition $0 \leq x < 1$,
\begin{equation}\label{eq:approximate_bound2}
|S''_{m,n}| \leq 8\omega^{-1/2}.
\end{equation}

These results hold uniformly for all $\omega > 0$ and not merely for a usual range such as $0 < \omega \leq 2$.
\end{theorem}

\begin{proof}
Write $x - m$ for $x$ in equation~\eqref{eq:psi_iteration2} and add to equation~\eqref{eq:psi_iteration1}. Then
\begin{equation}\label{eq:combined_iteration}
e^{-\pi i(2nx+n^2\omega)/\omega} \Psi(x - m + n\omega) - \Psi(x) = S'_{m-1,n-1}.
\end{equation}

Also by the argument leading to equation~\eqref{eq:second_form_case1_alt},
\begin{equation}\label{eq:psi_zero}
\Psi(0) = \int_{-\infty}^\infty \frac{e^{\pi i\omega t^2}}{e^{2\pi t} - 1} dt = \frac{i}{2\omega} = \frac{i}{2\sqrt{-i\omega}}.
\end{equation}

Write equation~\eqref{eq:combined_iteration} as
\begin{equation}\label{eq:iteration_rewrite}
S'_{m,n} = e^{-\pi i(2nx+n^2\omega)/\omega} \Psi(x - m + n\omega) - \Psi(x) + \frac{i e^{\pi i(x-m)^2/\omega}}{2\sqrt{-i\omega}} - e^{-\pi i(2nx+n^2\omega)/\omega}.
\end{equation}

\begin{align}
&= e^{-\pi i(2nx+n^2\omega)/\omega} (\Psi(x - m + n\omega) - \Psi(0)) - \Psi(x) + T_2, \label{eq:T_decomposition}
\end{align}
where
\begin{equation}\label{eq:T2_definition}
T_2 = \frac{i e^{\pi i(x-m)^2/\omega}}{2\sqrt{-i\omega}} - e^{-\pi i(2nx+n^2\omega)/\omega}.
\end{equation}

Then from equation~\eqref{eq:xi_definition},
\begin{equation}\label{eq:T2_bound}
|2\sqrt{\omega} T_2| = |e^{\pi i(x-m)^2/\omega} - e^{-\pi i(2nx+n^2\omega)/\omega}| = |1 - e^{\pi i \xi^2/\omega}| < \frac{\pi \xi}{\omega},
\end{equation}
and so
\begin{equation}\label{eq:T2_final_bound}
|T_2| \leq \min\left(\frac{1}{\omega}, \frac{\pi\xi}{2\omega^2}\right).
\end{equation}

Also
\begin{equation}\label{eq:T1_definition}
e^{-\pi i(2nx+n^2\omega)/\omega} T_1 = \Psi(0) = \int_0^\infty \frac{e^{\pi i\omega t^2}}{e^{2\pi t} - 1} dt,
\end{equation}
and so
\begin{equation}\label{eq:T1_decomposition}
T_1 = \int_0^\infty \frac{e^{\pi i\omega t^2}}{e^{2\pi t} - 1} dt + \int_0^1 \frac{e^{\pi i\omega t^2}}{1 - e^{-2\pi t}} dt,
\end{equation}
say.

To approximate to these, we require a modified form\footnote{I am indebted to Mr. Davenport for this form which leads to smaller constants than those I found originally.} of the second mean value theorem.

Suppose that for $a \leq x \leq b$, $f(x)$ is positive, monotone decreasing and differentiable, and that $g(x)$ is any continuous function (real or complex) of the real variable $x$. Then
\begin{equation}\label{eq:second_mean_value}
\left|\int_a^b g(x) df(x)\right| \leq f(a) \max_{a \leq x \leq b} \left|\int_a^x g(t) dt\right|.
\end{equation}

This is easily proved by putting $G(x) = \int_a^x g(t) dt$, i.e. $g(x) = G'(x)$ and integrating the left hand side by parts.

From this follows Wilton's result~\cite{Wilton1927} that for $M > 0$
\begin{equation}\label{eq:wilton_bound}
\left|\int_M^\infty \frac{e^{\pi i\omega t^2}}{e^{2\pi t} - 1} dt\right| \leq \min\left(\frac{1}{2\omega}, \frac{1}{\pi\omega M}\right).
\end{equation}

Clearly $I_1 \leq M$, and also
\begin{equation}\label{eq:I1_bound}
I_1 \leq \int_M^\infty \frac{e^{\pi i\omega t^2}}{e^{2\pi t}} dt + \int_M^\infty \frac{e^{\pi i\omega t^2}}{e^{2\pi t} - 1} dt < \frac{1}{2\pi\omega M} + \frac{1}{\pi\omega M},
\end{equation}
since
\begin{equation}\label{eq:integral_bound}
\left|\int_M^\infty e^{\pi i\omega t^2} dt\right| = \left|\int_M^\infty d(e^{\pi i\omega t^2})/(2\pi i\omega t)\right| \leq \frac{1}{\pi\omega M}
\end{equation}
on applying the modified second mean value theorem. Then if $M > \omega^{-1}$,
\begin{equation}\label{eq:combined_bound}
|I_1| \leq \frac{1}{2\omega} < \frac{1}{\pi\omega M} + \frac{1}{\pi\omega} = \frac{1}{\pi\omega}.
\end{equation}

Now $\frac{x}{e^{2\pi x} - 1}$ decreases steadily from $\frac{1}{2\pi}$ to $0$ in $0 \leq x \leq \infty$, taking the obvious definition at $x = 0$. For if $y = e^{2\pi x} \geq 1$, then from
\begin{equation}\label{eq:derivative_condition}
\frac{dy}{dx} \cdot \frac{(y-1) \frac{dy}{dx} - y \frac{d^2y}{dx^2}}{(y-1)^2} = 0,
\end{equation}
i.e.
\begin{equation}\label{eq:simplified_condition}
(y-1) \frac{dy}{dx} - y \frac{d^2y}{dx^2} = 0,
\end{equation}
i.e.
\begin{equation}\label{eq:y_condition}
1 = \frac{5y}{4}.
\end{equation}

But since $0 < \frac{4}{5} < 1$, the left hand side $> \frac{5y}{4}$ except when $y = 1$. Hence by the mean value theorem and equation~\eqref{eq:wilton_bound},
\begin{equation}\label{eq:T1_final_bound}
|T_1| \leq \frac{\xi}{\omega} + \max_{0 \leq t \leq 1} \left|\int_0^t \frac{e^{\pi i\omega s^2}}{e^{2\pi s} - 1} ds\right| \leq \frac{\xi}{\omega}.
\end{equation}

Next in $T_4$ write
\begin{equation}\label{eq:T4_decomposition}
T_4 = \int_0^1 \frac{e^{\pi i\omega t^2}(1 - e^{-2\pi t})}{1 - e^{-2\pi t}} dt = \int_0^1 e^{\pi i\omega t^2}(1 - e^{-2\pi t}) dt.
\end{equation}

Also if
\begin{equation}\label{eq:T5_definition}
T_5 = \int_0^1 e^{\pi i\omega t^2} e^{-2\pi t}(1 - e^{-2\pi t}) dt,
\end{equation}
then
\begin{equation}\label{eq:T5_bound}
|T_5| \leq \xi\omega^{-1/2} + \min\left(\frac{1}{\pi\xi}, \frac{1}{2\omega}, \frac{1}{\pi\omega}\right)
\end{equation}
by the mean value theorem, since $\frac{x(1-x)}{1-x^5}$ decreases steadily from $\frac{1}{5+1}$ to $0$ in $1 \geq x \geq 0$.

For from
\begin{equation}\label{eq:derivative_x}
\frac{d}{dx} \cdot \frac{x(1-x)}{1-x^5} = 0,
\end{equation}
\begin{equation}\label{eq:expanded_derivative}
(1-x^5)(1 - 2x) - x(1-x)(5x^4) = 0,
\end{equation}
\begin{equation}\label{eq:final_condition}
(5+1)x^5(1-x) = 0,
\end{equation}
i.e.
\begin{equation}\label{eq:simplified_final}
x^5 = \frac{1}{5+1}.
\end{equation}

But the left hand side $> (5+1)x$ except when $x = 1$. Hence
\begin{equation}\label{eq:combined_T_bounds}
|T_1| \leq \xi\omega^{-1/2} + \min\left(\frac{\pi\xi}{2\omega^2}, \frac{1}{\pi\omega}\right).
\end{equation}

Hence, and this is equation~\eqref{eq:approximate_bound},
\begin{equation}\label{eq:final_approximation}
|F(x) + S'_{m,n}| \leq 2\xi\omega^{-1/2} + 2\min\left(\frac{\pi\xi}{2\omega^2}, \frac{1}{\pi\omega}\right).
\end{equation}

If now also $0 \leq x < 1$, the formulae~\eqref{eq:T1_decomposition} to~\eqref{eq:combined_T_bounds} on taking $0 \leq \xi < 1$, show
\begin{equation}\label{eq:psi_difference_bound}
|\Psi(x) - \Psi(0)| \leq 2\omega^{-1/2} + \omega^{-1}.
\end{equation}

Also from equation~\eqref{eq:final_approximation}
\begin{equation}\label{eq:F_S_bound}
|F(x) + S'_{m,n}| \leq 2\omega^{-1/2} + 2\omega^{-1/2}.
\end{equation}

Hence
\begin{equation}\label{eq:F_zero_bound}
|F(0) + S'_{m,n}| \leq 7\omega^{-1/2},
\end{equation}
i.e.
\begin{equation}\label{eq:S_final_bound}
|S''_{m,n}| \leq 8\omega^{-1/2}.
\end{equation}
\end{proof}

\section{Some Problems}\label{sec:problems}

There still remains the general transformation formula for $f(x, \omega)$, i.e. a simple result for
\begin{equation}\label{eq:transformation_problem}
f\left(\frac{\alpha\omega + \beta}{\gamma\omega + \delta}, \frac{\alpha\omega + \beta}{\gamma\omega + \delta}\right),
\end{equation}
where $\alpha, \beta, \gamma, \delta$ are any integers satisfying $\alpha\delta - \beta\gamma = 1$. I had hoped to deal with this question fourteen years ago but my attention was diverted elsewhere. It seems to me now that some of the ideas recently developed by Hecke~\cite{Hecke1926a,Hecke1926b} for the transformation of the thetafunctions associated with algebraic fields may be relevant, but I leave this to others.

There is also the question in equation~\eqref{eq:f01_series} of finding a simple expansion for
\begin{equation}\label{eq:expansion_problem}
\frac{f_{00}(x) \theta_s(x)}{\theta_{00}(x)},
\end{equation}
where $\theta_s(x)$ is a thetafunction of order $s$, which would lead to class relation formulae. Thus~\cite{Mordell1916}
\begin{equation}\label{eq:class_relation_example1}
\frac{f_{00}(x)}{\theta_{00}(x)} \theta_{01}(2x, 2\omega) = -\frac{\theta_{00}(x, \omega)}{2} + \sum_{n=1}^\infty B_n q^{2n},
\end{equation}
where for $n \geq 0$,
\begin{align}
B_n &= (-1)^{n+r} q^{(n^2-8r^2)/4}, \label{eq:B_n_definition}\\
B_{-n} &= \sum_{r=1}^{(n-1)/2} (-1)^{n+r} q^{(n^2-8r^2)/4}, \label{eq:B_minus_n_definition}
\end{align}

On differentiating and putting $x = 0$, we find that if $m$ is any integer $> 0$,
\begin{equation}\label{eq:class_relation_result1}
F(m) - 2F(m - 2 \cdot 1^2) + 2F(m - 2 \cdot 2^2) - \cdots = 2 \sum (-1)^{x+y+1},
\end{equation}
where the summation in $x, y$ refers to the solutions in integers of
\begin{equation}\label{eq:constraint1}
\frac{x^2}{2} + 2y^2 = m
\end{equation}
with $x > 0$ and $-\frac{1}{2}(x-1) \leq y \leq 0$, or is zero if no such solutions exist.

Next
\begin{equation}\label{eq:class_relation_example2}
\frac{f_{00}(x)}{\theta_{00}(x)} \theta_{00}(3x, 3\omega) - \frac{1}{2}\theta_{01}(2x, 2\omega) + A\theta_{11}(2x, 2\omega) + \sum_{n=1}^\infty B_n q^{2n},
\end{equation}
where for $n \geq 0$,
\begin{align}
B_n &= (-1)^{n+r} q^{(n^2-8r^2)/4}, \label{eq:B_n_new}\\
B_{-n} &= \sum_{r=1}^{(n-2)/2} (-1)^{n+r} q^{(n^2-8r^2)/4}, \label{eq:B_minus_n_new}
\end{align}
and is also subject to the condition $r \equiv n \pmod{2}$. Also $A$ is independent of $x$, but it does not seem easy to find a simple form for $A$ of use for deducing class-relation formulae. This is, of course, the difficulty arising in the general case and is worth consideration by other investigators.

Here, however, a class relation formula can be deduced as I have already shown. For if we change $x$ into $x + \frac{1}{2}$ and add the two expansions, $A$ is eliminated. Now differentiate and put $x = 0$: we find that
\begin{equation}\label{eq:class_relation_result2}
F(2m) - 2F(2m - 3 \cdot 1^2) + 2F(2m - 3 \cdot 2^2) - \cdots = (-1)^{m+1} 2 \sum,
\end{equation}
where the summation in $x, y$ refers to the solutions in integers of
\begin{equation}\label{eq:constraint2}
\frac{x^2}{3} + 3y^2 = m
\end{equation}
with $x > 0$ and $-\frac{1}{3}(x-1) \leq y \leq 0$.

An extensive set of formulae of this kind has been given by Uspensky and Venkoff~\cite{UspenskyVenkoff1924} which they have investigated by Liouville's arithmetic method.

I may remark finally that there is a possibility that relations such as equations~\eqref{eq:omega_relation1} to~\eqref{eq:class_number_sum2} may lead to results about the order of magnitude of the class number and perhaps supply solutions of questions such as ``Is $F(n) = 1$ for an infinity of values of $n$?''

\begin{thebibliography}{99}

\bibitem{HardyLittlewood1914}
G.H. Hardy and J.E. Littlewood.
\newblock Note on the function $\int_0^\infty \frac{e^{-x^2}}{e^{2\pi x} + 1} dx$.
\newblock {\em Quarterly Journal of Mathematics}, 35:193--207, 1904.

\bibitem{HardyLittlewood1921}
G.H. Hardy and J.E. Littlewood.
\newblock Some problems of diophantine approximation.
\newblock {\em Acta Mathematica}, 37:193--221, 1914.

\bibitem{HardyLittlewood1923}
G.H. Hardy and J.E. Littlewood.
\newblock Some problems of diophantine approximation: an additional note on the trigonometrical series associated with the elliptic theta-functions.
\newblock {\em Acta Mathematica}, 47:189--198, 1925.

\bibitem{HardyLittlewood1925}
G.H. Hardy and J.E. Littlewood.
\newblock The zeros of riemann's zeta-function on the critical line.
\newblock {\em Mathematische Zeitschrift}, 10:283--317, 1921.

\bibitem{Hecke1926a}
E. Hecke.
\newblock Zur theorie der elliptischen modulfunktionen.
\newblock {\em Mathematische Annalen}, 97:210, 1926.

\bibitem{Hecke1926b}
E. Hecke.
\newblock \"Uber das verhalten von $\zeta_K(s, \mathfrak{m})$ und $\zeta_K(s, \chi, \mathfrak{f})$ bei modultransformationen.
\newblock {\em Journal f\"ur die reine und angewandte Mathematik}, 157:159--170, 1926.

\bibitem{Kronecker1889}
L. Kronecker.
\newblock Summierung der gaussschen reihen $\sum_{h=0}^{n-1} e^{2h^2\pi i/n}$.
\newblock {\em Journal f\"ur die reine und angewandte Mathematik}, 105:267--268, 1889.

\bibitem{Lerch1892}
M. Lerch.
\newblock Bemerkungen zur theorie der elliptischen funktionen.
\newblock {\em Rozpravy ceske Akademie cisare Frantiska Josefa pro vedy slovesnost, a umeni}, (II Cl) 1 Nr. (24), 1892.

\bibitem{Mordell1916}
L.J. Mordell.
\newblock On class relation formulae.
\newblock {\em Messenger of Mathematics}, 46:113--135, 1916.

\bibitem{Mordell1918}
L.J. Mordell.
\newblock On a simple summation of the series $\sum_{s=0}^{n-1} e^{2s^2\pi i/n}$.
\newblock {\em Messenger of Mathematics}, 48:54--56, 1918.

\bibitem{Mordell1920}
L.J. Mordell.
\newblock The value of the definite integral $\int_{-\infty}^\infty \frac{e^{at^2+bt}}{e^{ct}+d} dt$.
\newblock {\em Quarterly Journal of Mathematics}, 68:329--342, 1920.

\bibitem{Mordell1926}
L.J. Mordell.
\newblock The approximate functional formula for the thetafunction.
\newblock {\em Journal of the London Mathematical Society}, 1:68--72, 1926.

\bibitem{Ramanujan1915a}
S. Ramanujan.
\newblock Some definite integrals.
\newblock {\em Messenger of Mathematics}, 44:10--18, 1915.

\bibitem{Ramanujan1915b}
S. Ramanujan.
\newblock Some definite integrals connected with gauss's sums.
\newblock {\em Messenger of Mathematics}, 44:75--85, 1915.

\bibitem{Ramanujan1919}
S. Ramanujan.
\newblock Some definite integrals.
\newblock {\em Journal of the Indian Mathematical Society}, 11:81--87, 1919.

\bibitem{Siegel1932}
C.L. Siegel.
\newblock \"Uber riemann's nachlass zur analytischen zahlentheorie.
\newblock {\em Quellen und Studien zur Geschichte der Mathematik, Astronomie und Physik}, 2:45--80, 1932.

\bibitem{UspenskyVenkoff1924}
J.V. Uspensky and B. Venkoff.
\newblock On some new class-number relations.
\newblock {\em Proceedings of the international mathematical congress, Toronto}, pages 315--317, 1924.

\bibitem{vanderCorput1922}
J.G. van der Corput.
\newblock \"Uber summen, die mit den elliptischen $\theta$ funktionen zusammenh\"angen.
\newblock {\em Mathematische Annalen}, 87:66--77, 1922.

\bibitem{vanderCorput1923}
J.G. van der Corput.
\newblock \"Uber summen, die mit den elliptischen $\theta$ funktionen zusammenh\"angen.
\newblock {\em Mathematische Annalen}, 90:1--18, 1923.

\bibitem{Wilton1927}
J.R. Wilton.
\newblock The approximate functional formula for the thetafunction.
\newblock {\em Journal of the London Mathematical Society}, 2:177--180, 1927.

\end{thebibliography}

\end{document}

