\documentclass[12pt]{article}
\usepackage{amsmath,amsthm,amssymb, graphicx, enumitem}
\usepackage[margin=1in]{geometry}
\usepackage[utf8]{inputenc}
\usepackage{hyperref}

\newtheorem{theorem}{Theorem}
\newtheorem{lemma}{Lemma}
\newtheorem{definition}{Definition}

\title{Spectral Moments and Pre-Envelope Covariances of Nonseparable Processes}
\author{Mario Di Paola\thanks{Professor, Dipartimento di Ingegneria Strutturale e Geotecnica, Universita di Palermo, Viale delle Scienze, 90128 Palermo, Italy} \and Giovanni Petrucci\thanks{Consulting Engineer, Istituto di Costruzione di Macchine, Universita di Palermo, Viale delle Scienze, 90128 Palermo, Italy}}

\begin{document}

\maketitle

\begin{abstract}
A critical review of the definition of the spectral moments of a stochastic process in the nonstationary case is presented. An adequate time-domain representation of spectral moments in the stationary case is first established, showing that the spectral moments are related to the variances of the stationary analytical pre-envelope processes. The extension to the nonstationary case is made in the time domain evaluating the covariances of the nonstationary pre-envelope showing the differences between the proposed definition and the classical one made introducing the evolutionary power.
\end{abstract}

\section{Introduction}

Many time-varying loadings to structures are modeled as stochastic processes and the response analysis can be established in a probabilistic sense. The stochastic processes of input and response may often be nonstationary for frequency content and amplitude, as in the case of a strong motion phase during an earthquake \cite{Kameda1975} and can be adequately represented as nonseparable processes \cite{Priestley1965}.

Second-order moments completely define the statistics of the response. However, in many cases such as prediction of the first excursion failure, fatigue failure, etc., we are concerned with the statistics of the envelope process. The above, following Dugundji \cite{Dugundji1958} and Yang \cite{Yang1972}, and Krenk et al. \cite{Krenk1983}, for stationary and nonstationary processes, respectively, can be seen as the modulus of the pre-envelope \cite{Arens1957,Dugundji1958}; i.e., a complex process, the real part of which is the given process while its imaginary part is related to the real one in such a way that the complex process exhibits power only in the positive frequency range. It follows that the statistics of the envelope are related to the covariances of the pre-envelope.

The pre-envelope are, in the stationary cases, strictly related to the so-called spectral moments (hereafter referred to as SM) \cite{Vanmarcke1972}. In particular, the SM, defined as the moments of the one-sided power spectral density function have, in time domain, the meaning of variances of the pre-envelope \cite{DiPaola1985}.

The extension of the SM to the nonstationary case is usually made in the frequency domain as the moments of the one-sided evolutionary power spectral density \cite{Priestley1965,Hammond1968,Shinozuka1970}. However, such definition, with exception of the zeroth one, has no physical meaning in the time domain and enjoyment of unsatisfactory properties, for example, in the transient case of an oscillator subjected to white noise input \cite{Corotis1972}.

In this paper, the connection between spectral moments and pre-envelope covariances (PEC) is presented. In particular, it is shown that only the area of the evolutionary power coincides with the PEC, while all other moments differ from the variances of the various derivatives of the pre-envelope and, as a consequence, the moments of the evolutionary power do not give any information on the statistics of the envelope.

The PEC for a multi-degree-of-freedom linear system subjected to nonstationary, nonseparable processes is also presented, and the numerical aspect on their evaluation is discussed in the application.

For clarity's sake, in the next two sections the complex representation of pre-envelope processes is first discussed.

\section{Stationary Pre-Envelope Process}

Let $\mathbf{F}(t)$ be an $n$-dimensional real stationary stochastic process vector given in the Priestley representation as follows:
\begin{equation}
\mathbf{F}(t)=\int_{-\infty}^{\infty}e^{-i\omega t}\,d\mathbf{Z}(\omega)=\int_{-\infty}^{\infty}e^{i\omega t}\,d\mathbf{Z}^{*}(\omega).
\label{eq:stationary_representation}
\end{equation}

$i$ is the imaginary unit ($i=\sqrt{-1}$), while the asterisk indicates the complex conjugate and $d\mathbf{Z}(\omega)$ is a stochastic vector process having orthogonal increments, i.e.,
\begin{equation}
E[d\mathbf{Z}(\omega_{1})d\mathbf{Z}^{*T}(\omega_{2})]=\delta(\omega_{2}-\omega_{1})\,d\psi(\omega_{1})
\label{eq:orthogonal_increments}
\end{equation}
where $E[\cdot]$ means stochastic average, $\delta(\cdot)$ is the Dirac's delta, the superimposed $T$ means transpose, and $d\psi(\omega)$ is a deterministic Hermitian positive-definite matrix. It is worth noting that in order for equation \eqref{eq:stationary_representation} to be fulfilled, it is necessary that the process vector $d\mathbf{Z}(\omega)$ be complex, such that its real and imaginary parts are even and odd functions of $\omega$, respectively.

Without loss of generality we consider that $\mathbf{F}(t)$ is a zero-mean process and $\psi(\omega)$ is a differentiable matrix, and hence, the following relationship
\begin{equation}
d\psi(\omega)=G(\omega)\,d\omega
\label{eq:spectral_density}
\end{equation}
holds, $G(\omega)$ being the Hermitian power spectral density function matrix defined in both the positive and negative frequency ranges.

Let us consider a new process vector $\tilde{\mathbf{F}}(t)$, derived from $\mathbf{F}(t)$, such that the corresponding power spectral density function matrix $\tilde{G}(\omega)$ is one-sided, i.e.,
\begin{equation}
\tilde{G}(\omega)=2U(\omega)G(\omega)=S(\omega),
\label{eq:one_sided_spectral_density}
\end{equation}
$U(\omega)$ being the unit step function. Equation \eqref{eq:one_sided_spectral_density} is verified if the vector $d\tilde{\mathbf{Z}}(\omega)$, corresponding to $\tilde{\mathbf{F}}(\omega)$, takes on the following form:
\begin{equation}
d\tilde{\mathbf{Z}}(\omega)=\frac{1}{\sqrt{2}}(1+\operatorname{sgn}(\omega))\,d\mathbf{Z}(\omega)=\sqrt{2}\,U(\omega)\,d\mathbf{Z}(\omega)
\label{eq:transformed_increments}
\end{equation}
$\operatorname{sgn}(\omega)$ being the signum function. Equation \eqref{eq:orthogonal_increments}, rewritten for the stochastic process vector $\tilde{\mathbf{Z}}(\omega)$, now gives:
\begin{equation}
E[d\tilde{\mathbf{Z}}(\omega_{1})d\tilde{\mathbf{Z}}^{*T}(\omega_{2})]=2U(\omega_{1})G(\omega_{1})\delta(\omega_{2}-\omega_{1})=S(\omega_{1})\delta(\omega_{2}-\omega_{1}).
\label{eq:transformed_orthogonal}
\end{equation}

Using equation \eqref{eq:transformed_increments}, the appropriate description of the vector $\tilde{\mathbf{F}}(t)$ takes on the form:
\begin{equation}
\tilde{\mathbf{F}}(t)=\int_{-\infty}^{\infty}e^{-i\omega t}\,d\tilde{\mathbf{Z}}(\omega)=\frac{2}{\sqrt{2}}\int_{0}^{\infty}e^{-i\omega t}\,d\mathbf{Z}(\omega).
\label{eq:pre_envelope_stationary}
\end{equation}

This equation shows that $\tilde{\mathbf{F}}(t)$ is a complex vector having frequency content only in the positive frequency range, and it can easily be seen that its real part is proportional to the real process $\mathbf{F}(t)$ defined in equation \eqref{eq:stationary_representation}, while its imaginary part is the signumless Hilbert Transform of the real one, i.e., $\tilde{\mathbf{F}}(t)$ constitutes an analytic process \cite{Papoulis1965,Nigam1982}, and is the so-called ``pre-envelope'' \cite{Arens1957,Dugundji1958}:
\begin{equation}
\tilde{\mathbf{F}}(t)=\frac{2}{\sqrt{2}}[\mathbf{F}(t)-i\bar{\mathbf{F}}(t)]
\label{eq:pre_envelope_definition}
\end{equation}
where the accent means Hilbert Transform:
\begin{equation}
\bar{\mathbf{F}}(t)=\frac{1}{\pi}\int_{-\infty}^{\infty}\frac{\mathbf{F}(p)}{t-p}\,dp.
\label{eq:hilbert_transform}
\end{equation}

In the stationary case the vector $\tilde{\mathbf{F}}(t)$ can be written in the form
\begin{equation}
\tilde{\mathbf{F}}(t)=\frac{2}{\sqrt{2}}\int_{-\infty}^{\infty}e^{-i\omega t}\operatorname{sgn}(\omega)\,d\mathbf{Z}(\omega).
\label{eq:alt_pre_envelope}
\end{equation}

It is worth noting that the modulus of the $j$th entry of the complex process vector defined in equation \eqref{eq:pre_envelope_definition} is proportional to the ``envelope'' of $F_{j}(t)$.

The cross-correlation function matrix of the complex vector $\tilde{\mathbf{F}}(t)$ is given in the form:
\begin{equation}
\mathbf{R}_{\tilde{F}\tilde{F}}(\tau)=E[\tilde{\mathbf{F}}(t)\tilde{\mathbf{F}}^{*T}(t+\tau)]=\int_{0}^{\infty}e^{i\omega\tau}S(\omega)\,d\omega.
\label{eq:cross_correlation_complex}
\end{equation}

From this equation it can easily be shown that the cross-correlation matrix of the vector $\tilde{\mathbf{F}}(t)$ is complex and such that its real part is the cross-correlation function of the real vector $\mathbf{F}(t)$ while its imaginary part is the Hilbert Transform of the real one, i.e.,
\begin{equation}
\mathbf{R}_{\tilde{F}\tilde{F}}(\tau)=\mathbf{R}_{FF}(\tau)+i\mathbf{R}_{\bar{F}F}(\tau).
\label{eq:correlation_decomposition}
\end{equation}

The real and the imaginary parts of the cross-correlation function matrix can be rewritten in equivalent forms as follows:
\begin{equation}
\mathbf{R}_{FF}(\tau)=\mathbf{R}_{\tilde{F}\tilde{F}}^{R}(\tau)=\Re\{\mathbf{R}_{\tilde{F}\tilde{F}}(\tau)\}=\int_{-\infty}^{\infty}e^{i\omega\tau}G(\omega)\,d\omega,
\label{eq:real_part_correlation}
\end{equation}
\begin{equation}
\mathbf{R}_{\bar{F}F}(\tau)=\Im\{\mathbf{R}_{\tilde{F}\tilde{F}}(\tau)\}=\int_{-\infty}^{\infty}e^{i\omega\tau}\operatorname{sgn}(\omega)G(\omega)\,d\omega.
\label{eq:imaginary_part_correlation}
\end{equation}

As a conclusion the process vector $\tilde{\mathbf{F}}(t)$ having the representation given in equation \eqref{eq:pre_envelope_definition} exhibits power only in the positive frequency range and has the complex cross-correlation function defined in equation \eqref{eq:correlation_decomposition}.

\section{Nonstationary Pre-Envelope Process}

Let $\mathbf{F}(t)$ be a real nonstationary nonseparable stochastic process vector. Following Priestley, its representation is given in the form:
\begin{equation}
\mathbf{F}(t)=\int_{-\infty}^{\infty}e^{-i\omega t}A(\omega,t)\,d\mathbf{Z}(\omega)=\int_{-\infty}^{\infty}e^{i\omega t}A^{*}(\omega,t)\,d\mathbf{Z}^{*}(\omega)
\label{eq:nonstationary_representation}
\end{equation}
where $A(\omega,t)$ is a slowly time-varying deterministic function matrix and $d\mathbf{Z}(\omega)$ is the stochastic process vector already defined in equation \eqref{eq:orthogonal_increments}. As for the stationary case, due to the fact that the process $\mathbf{F}(t)$ is real, the real and the imaginary parts of the vector $A(\omega,t)d\mathbf{Z}(\omega)$ must be even and odd functions of $\omega$, respectively.

The complex representation of the nonstationary process vector can be obtained by inserting the process vector $d\mathbf{Z}(\omega)$ defined in equation \eqref{eq:transformed_increments} into equation \eqref{eq:nonstationary_representation}, thus obtaining:
\begin{equation}
\tilde{\mathbf{F}}(t)=\frac{1}{\sqrt{2}}\int_{-\infty}^{\infty}e^{-i\omega t}A(\omega,t)\,d\mathbf{Z}(\omega)=\frac{2}{\sqrt{2}}\int_{0}^{\infty}e^{-i\omega t}A(\omega,t)\,d\mathbf{Z}(\omega).
\label{eq:nonstationary_complex}
\end{equation}

This equation shows that the nonstationary vector process $\tilde{\mathbf{F}}(t)$ is a complex vector having frequency content only in the positive frequency range, and it can easily be seen that its real part is proportional to the real vector process $\mathbf{F}(t)$ defined in equation \eqref{eq:nonstationary_representation}, while its imaginary part will be denoted as $\bar{\mathbf{F}}(t)$. Hence, we can write:
\begin{equation}
\tilde{\mathbf{F}}(t)=\frac{2}{\sqrt{2}}[\mathbf{F}(t)-i\bar{\mathbf{F}}(t)].
\label{eq:nonstationary_pre_envelope}
\end{equation}

It is to be emphasized that $\bar{\mathbf{F}}(t)$ coincides with $\bar{\mathbf{F}}(t)$ from equation \eqref{eq:hilbert_transform} only in the stationary case, while in the nonstationary case it is given as
\begin{equation}
\bar{\mathbf{F}}(t)=\frac{1}{\sqrt{2}}\int_{-\infty}^{\infty}e^{-i\omega t}\operatorname{sgn}(\omega)A(\omega,t)\,d\mathbf{Z}(\omega).
\label{eq:nonstationary_hilbert}
\end{equation}

The modulus of the $j$th entry of the vector $\tilde{\mathbf{F}}(t)$ in equation \eqref{eq:nonstationary_pre_envelope} is proportional to the envelope function of $F_{j}(t)$ defined by Yang \cite{Yang1972}. The complex cross-correlation function matrix of the complex vector $\tilde{\mathbf{F}}(t)$ can be written as follows:
\begin{equation}
\mathbf{R}_{\tilde{F}\tilde{F}}(t_{1},t_{2})=E[\tilde{\mathbf{F}}(t_{1})\tilde{\mathbf{F}}^{*T}(t_{2})]=\int_{0}^{\infty}e^{i\omega(t_{2}-t_{1})}A(\omega,t_{1})S(\omega)A^{*T}(\omega,t_{2})\,d\omega
\label{eq:nonstationary_correlation}
\end{equation}
in which $T=t_{2}-t_{1}$.

The real and the imaginary parts of the correlation matrix $\mathbf{R}_{\tilde{F}\tilde{F}}(t_{1},t_{2})$ can be rewritten in equivalent forms as follows:
\begin{equation}
\mathbf{R}_{FF}(t_{1},t_{2})=\mathbf{R}_{\tilde{F}\tilde{F}}^{R}(t_{1},t_{2})=\int_{-\infty}^{\infty}e^{i\omega(t_{2}-t_{1})}A(\omega,t_{1})G(\omega)A^{*T}(\omega,t_{2})\,d\omega,
\label{eq:nonstationary_real}
\end{equation}
\begin{equation}
\mathbf{R}_{\bar{F}F}(t_{1},t_{2})=\Im\{\mathbf{R}_{\tilde{F}\tilde{F}}(t_{1},t_{2})\}=\int_{-\infty}^{\infty}e^{i\omega(t_{2}-t_{1})}\operatorname{sgn}(\omega)A(\omega,t_{1})G(\omega)A^{*T}(\omega,t_{2})\,d\omega.
\label{eq:nonstationary_imaginary}
\end{equation}

Using equations \eqref{eq:nonstationary_real} and \eqref{eq:nonstationary_imaginary}, the correlation function matrix $\mathbf{R}_{\tilde{F}\tilde{F}}(t_{1},t_{2})$ can be rewritten in the form:
\begin{equation}
\mathbf{R}_{\tilde{F}\tilde{F}}(t_{1},t_{2})=\mathbf{R}_{FF}(t_{1},t_{2})+i\mathbf{R}_{\bar{F}F}(t_{1},t_{2}).
\label{eq:nonstationary_correlation_decomposition}
\end{equation}

In the next section it will be shown that the complex representation of the vector $\tilde{\mathbf{F}}(t)$ given in equations \eqref{eq:pre_envelope_stationary} and \eqref{eq:nonstationary_complex} for the stationary and nonstationary case, respectively, are essential not only for the definition of the envelopes, but also for the correct definition of the spectral moments \cite{Vanmarcke1972} in both stationary and nonstationary cases (see also Di Paola \cite{DiPaola1985}).

\section{Spectral Moments and Pre-Envelope Covariances, Stationary Case}

In this section the covariances of the stationary process defined in equation \eqref{eq:pre_envelope_definition} are presented. In order to do this, let $\tilde{\mathbf{P}}(t)$ the $2n$ dimensional vector of the state variables, be introduced as follows:
\begin{equation}
\tilde{\mathbf{P}}^{T}(t)=[\tilde{\mathbf{F}}^{T}(t),\dot{\tilde{\mathbf{F}}}^{T}(t)]
\label{eq:state_vector}
\end{equation}
where the upper dot means time differentiation and $\tilde{\mathbf{F}}(t)$ is the stationary process vector given in equation \eqref{eq:pre_envelope_stationary}. Using the main properties of the correlation function given in equations \eqref{eq:real_part_correlation} and \eqref{eq:imaginary_part_correlation} evaluated for $\tau=0$, the time-independent Hermitian cross-covariance matrix of the complex vector $\tilde{\mathbf{P}}(t)$, i.e., the PEC matrix, is given as:
\begin{equation}
\boldsymbol{\Sigma}_{\tilde{P}\tilde{P}}=E[\tilde{\mathbf{P}}(t)\tilde{\mathbf{P}}^{*T}(t)]=E[\mathbf{P}(t)\mathbf{P}^{T}(t)]+iE[\mathbf{P}(t)\bar{\mathbf{P}}^{T}(t)]
\label{eq:pec_matrix_stationary}
\end{equation}
where $\mathbf{P}(t)$ is the real vector of state variables
\begin{equation}
\mathbf{P}^{T}(t)=[\mathbf{F}^{T}(t),\dot{\mathbf{F}}^{T}(t)].
\label{eq:real_state_vector}
\end{equation}

Equation \eqref{eq:pec_matrix_stationary} shows that the real part of the matrix $\boldsymbol{\Sigma}_{\tilde{P}\tilde{P}}$ is the traditional covariance matrix of the real process vector $\mathbf{P}(t)$, while the imaginary part is the cross-covariance between the real vector $\mathbf{P}(t)$ and its Hilbert Transform.

The matrix $\boldsymbol{\Sigma}_{\tilde{P}\tilde{P}}$ can be rewritten in an extended form as follows:
\begin{equation}
\boldsymbol{\Sigma}_{\tilde{P}\tilde{P}}=
\begin{bmatrix}
E[\tilde{\mathbf{F}}(t)\tilde{\mathbf{F}}^{*T}(t)] & E[\tilde{\mathbf{F}}(t)\dot{\tilde{\mathbf{F}}}^{*T}(t)]\\
E[\dot{\tilde{\mathbf{F}}}(t)\tilde{\mathbf{F}}^{*T}(t)] & E[\dot{\tilde{\mathbf{F}}}(t)\dot{\tilde{\mathbf{F}}}^{*T}(t)]
\end{bmatrix}.
\label{eq:pec_matrix_extended}
\end{equation}

In previous papers \cite{Borino1988,Muscolino1988}, this matrix has been called, in a less appropriate manner, cross-covariance spectral matrix (CCS matrix).

Using equation \eqref{eq:pre_envelope_stationary} to represent the stochastic vector $\tilde{\mathbf{F}}(t)$, after some simple algebra it can easily be seen that the various block matrices of the matrix $\boldsymbol{\Sigma}_{\tilde{P}\tilde{P}}$ take on the form:
\begin{align}
E[\tilde{\mathbf{F}}(t)\tilde{\mathbf{F}}^{*T}(t)] &=\int_{0}^{\infty}S(\omega)\,d\omega = \mathbf{\Lambda}_{0,FF},
\label{eq:lambda0}\\
E[\tilde{\mathbf{F}}(t)\dot{\tilde{\mathbf{F}}}^{*T}(t)] &= i\int_{0}^{\infty}\omega S(\omega)\,d\omega = i\mathbf{\Lambda}_{1,FF},
\label{eq:lambda1}\\
E[\dot{\tilde{\mathbf{F}}}(t)\dot{\tilde{\mathbf{F}}}^{*T}(t)] &= \int_{0}^{\infty}\omega^{2}S(\omega)\,d\omega = \mathbf{\Lambda}_{2,FF}.
\label{eq:lambda2}
\end{align}

Equations \eqref{eq:lambda0}-\eqref{eq:lambda2} show that the PEC matrix is related to the moments of the one-sided power spectral matrix $S(\omega)$, i.e., to the so-called SM \cite{Vanmarcke1972}.

Inserting equations \eqref{eq:lambda0}-\eqref{eq:lambda2} in \eqref{eq:pec_matrix_extended}, the frequency domain representation of the PEC matrix is given as:
\begin{equation}
\boldsymbol{\Sigma}_{\tilde{P}\tilde{P}}=
\begin{bmatrix}
\displaystyle\int_{0}^{\infty}S(\omega)\,d\omega & i\displaystyle\int_{0}^{\infty}\omega S(\omega)\,d\omega\\[2ex]
-i\displaystyle\int_{0}^{\infty}\omega S^{*}(\omega)\,d\omega & \displaystyle\int_{0}^{\infty}\omega^{2}S(\omega)\,d\omega
\end{bmatrix}.
\label{eq:pec_frequency_domain}
\end{equation}

Comparing equations \eqref{eq:pec_matrix_stationary} and \eqref{eq:pec_frequency_domain}, the important connection between the SM and the PEC is evidenced.

The presence of the imaginary unit in the out-of-diagonal block matrices in equation \eqref{eq:pec_frequency_domain} inverts the roles of the real and imaginary parts of the first spectral moment, with respect to the cross-covariance $E[\tilde{\mathbf{F}}(t)\tilde{\mathbf{F}}^{T}(t)]$.

It is interesting to note that the PEC matrix particularized for the vector $\tilde{\mathbf{F}}(t)$, having only one component, is such that its determinant is related to the bandwidth parameter \cite{Vanmarcke1972}.

\section{Spectral Moments and Pre-Envelope Covariances, Nonstationary Case}

The SM in the nonstationary case are defined in the literature as the moments of the so-called one-sided evolutionary spectral density \cite{Shinozuka1970}:
\begin{equation}
S(\omega,t)=A(\omega,t)S(\omega)A^{*T}(\omega,t)
\label{eq:evolutionary_spectral_density}
\end{equation}
and the extension of the time-dependent SM is usually made in the form \cite{Corotis1972}:
\begin{equation}
\Lambda_{j,FF}(t)=\int_{0}^{\infty}\omega^{j}S(\omega,t)\,d\omega\quad j=0,1,2,\ldots
\label{eq:time_dependent_sm}
\end{equation}

Using the main properties of the correlation function given in equations \eqref{eq:nonstationary_imaginary} and \eqref{eq:nonstationary_correlation_decomposition}, particularized for $t_{1}=t_{2}=t$, it can easily be seen that for $j=0$, equation \eqref{eq:time_dependent_sm} gives:
\begin{align}
\Lambda_{0,FF}(t) &=\int_{0}^{\infty}S(\omega,t)\,d\omega = E[\tilde{\mathbf{F}}(t)\tilde{\mathbf{F}}^{*T}(t)]\notag\\
&=E[\mathbf{F}(t)\mathbf{F}^{T}(t)]+iE[\bar{\mathbf{F}}(t)\mathbf{F}^{T}(t)].
\label{eq:zeroth_moment_correspondence}
\end{align}

This equation shows that the zeroth moment coincides with the covariance of the complex process defined in equation \eqref{eq:nonstationary_pre_envelope}, while for $j$ greater than zero, the moments of the one-sided evolutionary power spectral density function matrix has no analogous correspondence in the time domain of the variance of the pre-envelope processes, as in the stationary case.

The time-dependent PEC matrix is given in the form:
\begin{equation}
\boldsymbol{\Sigma}_{\tilde{P}\tilde{P}}(t)=E[\tilde{\mathbf{P}}(t)\tilde{\mathbf{P}}^{*T}(t)]
\label{eq:time_dependent_pec}
\end{equation}
where $\tilde{\mathbf{P}}(t)$ is defined in equation \eqref{eq:state_vector} and while $\tilde{\mathbf{F}}(t)$ is defined in equation \eqref{eq:nonstationary_pre_envelope}.

The block matrices of $\boldsymbol{\Sigma}_{\tilde{P}\tilde{P}}(t)$ are given in equation \eqref{eq:pec_matrix_extended}, the first block matrix has already been defined in equation \eqref{eq:zeroth_moment_correspondence}, while the other blocks can be written in the form:
\begin{align}
E[\tilde{\mathbf{F}}(t)\dot{\tilde{\mathbf{F}}}^{*T}(t)] &=E[\mathbf{F}(t)\dot{\mathbf{F}}^{T}(t)]+iE[\bar{\mathbf{F}}(t)\dot{\mathbf{F}}^{T}(t)],
\label{eq:second_block}\\
E[\dot{\tilde{\mathbf{F}}}(t)\dot{\tilde{\mathbf{F}}}^{*T}(t)] &=E[\dot{\mathbf{F}}(t)\dot{\mathbf{F}}^{T}(t)]+iE[\bar{\mathbf{F}}(t)\dot{\mathbf{F}}^{T}(t)].
\label{eq:third_block}
\end{align}

Using equation \eqref{eq:nonstationary_complex} to represent the nonstationary vector $\tilde{\mathbf{F}}(t)$, and writing its time differentiation in the form
\begin{equation}
\dot{\tilde{\mathbf{F}}}(t)=\frac{2}{\sqrt{2}}\int_{0}^{\infty}e^{-i\omega t}\mathbf{A}_{1}(\omega,t)\,d\mathbf{Z}(\omega)
\label{eq:time_derivative}
\end{equation}
where
\begin{equation}
\mathbf{A}_{1}(\omega,t)=-i\omega A(\omega,t)+\dot{A}(\omega,t),
\label{eq:a1_definition}
\end{equation}

equations \eqref{eq:time_derivative} and \eqref{eq:a1_definition} can be rewritten as
\begin{align}
E[\tilde{\mathbf{F}}(t)\dot{\tilde{\mathbf{F}}}^{*T}(t)] &=\int_{0}^{\infty}A(\omega,t)S(\omega)A_{1}^{*T}(\omega,t)\,d\omega,
\label{eq:pec_01}\\
E[\dot{\tilde{\mathbf{F}}}(t)\dot{\tilde{\mathbf{F}}}^{*T}(t)] &=\int_{0}^{\infty}A_{1}(\omega,t)S(\omega)A_{1}^{*T}(\omega,t)\,d\omega.
\label{eq:pec_02}
\end{align}

or, in an explicit form
\begin{align}
E[\tilde{\mathbf{F}}(t)\dot{\tilde{\mathbf{F}}}^{*T}(t)] &=i\int_{0}^{\infty}\omega S(\omega,t)\,d\omega+\int_{0}^{\infty}S_{1}(\omega,t)\,d\omega,
\label{eq:pec_01_explicit}\\
E[\dot{\tilde{\mathbf{F}}}(t)\dot{\tilde{\mathbf{F}}}^{*T}(t)] &=\int_{0}^{\infty}\omega^{2}S(\omega,t)\,d\omega+i\int_{0}^{\infty}[\omega S_{1}(\omega,t)-S_{1}^{*}(\omega,t)]\,d\omega\notag\\
&\quad+\int_{0}^{\infty}S_{2}(\omega,t)\,d\omega,
\label{eq:pec_02_explicit}
\end{align}
where
\begin{equation}
\begin{aligned}
\mathbf{S}_{1}(\omega,t) &=A(\omega,t)S(\omega)\dot{A}^{*T}(\omega,t),\\
\mathbf{S}_{2}(\omega,t) &=\dot{A}(\omega,t)S(\omega)\dot{A}^{*T}(\omega,t).
\end{aligned}
\label{eq:s1_s2_definitions}
\end{equation}

Equations \eqref{eq:pec_01_explicit} and \eqref{eq:pec_02_explicit} show that the variances of the nonstationary pre-envelope processes $\tilde{\mathbf{F}}(t)$ and $\dot{\tilde{\mathbf{F}}}(t)$ can be constructed adding to the moments of the one-sided evolutionary spectrum other quantities involving the time derivatives of the function matrix $A(\omega,t)$. Only when $A(\omega,t)$ is a smooth function matrix varying very slowly in $t$, $\dot{A}(\omega,t)$ is approximately equal to zero, is the cross-covariances of the pre-envelope processes proportional to the time-dependent spectral moments defined in equation \eqref{eq:time_dependent_sm} \cite{To1986}. On the other hand, when comparing equations \eqref{eq:lambda0}-\eqref{eq:lambda2} with equations \eqref{eq:zeroth_moment_correspondence}, \eqref{eq:pec_01_explicit}, and \eqref{eq:pec_02_explicit}, it seems to be more reasonable to evaluate the covariances of the pre-envelope in the nonstationary case, in the time domain, i.e., defining the time-dependent PEC as the covariances of the nonstationary complex processes $\tilde{\mathbf{F}}(t)$ and $\dot{\tilde{\mathbf{F}}}(t)$:
\begin{equation}
\begin{aligned}
\tilde{\Lambda}_{0,FF}(t) &=E[\tilde{\mathbf{F}}(t)\tilde{\mathbf{F}}^{*T}(t)],\\
i\tilde{\Lambda}_{1,FF}(t) &=E[\tilde{\mathbf{F}}(t)\dot{\tilde{\mathbf{F}}}^{*T}(t)],\\
\tilde{\Lambda}_{2,FF}(t) &=E[\dot{\tilde{\mathbf{F}}}(t)\dot{\tilde{\mathbf{F}}}^{*T}(t)].
\label{eq:pec_definitions}
\end{aligned}
\end{equation}

Using these quantities instead of the moments of the evolutionary power, some quantities of engineering interest, such as the probability density function of the envelope and the mean rate threshold crossing of the given barrier, can be computed in an exact manner \cite{DiPaola1987,Muscolino1988}, while only approximate expression can be obtained using covariances of the real processes or the moments of the evolutionary power.

\section{Input-Output Relationships}

The equation of motion of an $n$-degree-of-freedom linear structural system is governed by the following equation:
\begin{equation}
M\ddot{\mathbf{X}}+C\dot{\mathbf{X}}+K\mathbf{X}=\mathbf{F}(t)
\label{eq:equation_of_motion}
\end{equation}
where $M$, $C$, and $K$ are the inertia, damping and stiffness matrices, respectively, $\mathbf{X}(t)$ is the vector of displacements, $\mathbf{F}(t)$ is the forcing function vector. The vector solution $\mathbf{X}(t)$ can be obtained in the form:
\begin{equation}
\mathbf{X}(t)=\int_{0}^{t}H(t-\tau)\mathbf{F}(\tau)\,d\tau+G(t)K\mathbf{X}_{0}+H(t)M\dot{\mathbf{X}}_{0},
\label{eq:solution_x}
\end{equation}
$H(t)$ being the impulse response function matrix, $G(t)$ is related to the matrix $H(t)$ in the form:
\begin{equation}
\dot{G}(t)=H(t),
\label{eq:g_and_h}
\end{equation}
and $\mathbf{X}_{0}$, $\dot{\mathbf{X}}_{0}$ are the vectors of initial conditions. For greater convenience, let the state vector
\begin{equation}
\mathbf{U}^{T}(t)=[\mathbf{X}^{T}(t),\dot{\mathbf{X}}^{T}(t)]
\label{eq:state_vector_u}
\end{equation}
be introduced, then the vector solution $\mathbf{X}(t)$ is written in the form
\begin{equation}
\mathbf{U}(t)=\theta(t)\mathbf{U}_{0}+\int_{0}^{t}L(t-\tau)\mathbf{F}(\tau)\,d\tau
\label{eq:solution_u}
\end{equation}
in which we have set
\begin{equation}
\theta(t)=
\begin{bmatrix}
-G(t)K & H(t)M\\
-\dot{G}(t)K & \dot{H}(t)M
\end{bmatrix},
\quad
L(t)=\begin{bmatrix}
H(t)\\
\dot{H}(t)
\end{bmatrix},
\quad
\mathbf{U}_{0}=\begin{bmatrix}
\mathbf{X}_{0}\\
\dot{\mathbf{X}}_{0}
\end{bmatrix}.
\label{eq:matrices_definition}
\end{equation}

Equation \eqref{eq:solution_u} is able to give the state vector solution, $\mathbf{U}(t)$, in the deterministic case. The vector $\mathbf{U}(t)$ is either real or complex depending on whether the forcing vector is real or complex. In order to evaluate the PEC of the vector solution $\mathbf{U}(t)$, the forcing vector $\mathbf{F}(t)$ must be defined as in equation \eqref{eq:pre_envelope_stationary} or \eqref{eq:nonstationary_complex} in the stationary or nonstationary case, respectively.

\subsection{PEC Matrix of the Output-Stationary Case}

Particularizing equation \eqref{eq:solution_u} for stationary condition and complex forcing function defined as in equation \eqref{eq:pre_envelope_stationary}, we obtain the stationary response of the state vector in the form
\begin{equation}
\tilde{\mathbf{U}}(t)=\int_{-\infty}^{t}L(t-\tau)\tilde{\mathbf{F}}(\tau)\,d\tau=\frac{2}{\sqrt{2}}\left\{\int_{0}^{\infty}\left[\int_{-\infty}^{t}L(t-\tau)e^{-i\omega\tau}\,d\tau\right]d\mathbf{Z}(\omega)\right\}.
\label{eq:stationary_response}
\end{equation}

After some easy manipulations, the latter can be rewritten in the form
\begin{equation}
\tilde{\mathbf{U}}(t)=\frac{2}{\sqrt{2}}\int_{0}^{\infty}e^{-i\omega t}L^{*}(\omega)\,d\mathbf{Z}(\omega),
\label{eq:stationary_response_simplified}
\end{equation}
in which $L(\omega)$ is the Fourier Transform of $L(t)$. From this equation it can easily be seen that $\tilde{\mathbf{U}}(t)$ is a complex process such that its imaginary part is the (signumless) Hilbert transform of the corresponding real one, i.e.,
\begin{equation}
\tilde{\mathbf{U}}(t)=\frac{2}{\sqrt{2}}[\mathbf{U}(t)-i\bar{\mathbf{U}}(t)].
\label{eq:pre_envelope_u}
\end{equation}

The PEC matrix of the vector $\mathbf{X}$, according to equation \eqref{eq:pec_matrix_stationary}, is given in the form:
\begin{equation}
\boldsymbol{\Sigma}_{\tilde{U}\tilde{U}}=E[\tilde{\mathbf{U}}(t)\tilde{\mathbf{U}}^{*T}(t)]=
\begin{bmatrix}
\mathbf{\Lambda}_{0,XX} & i\mathbf{\Lambda}_{1,XX}\\
-i\mathbf{\Lambda}_{1,XX}^{*} & \mathbf{\Lambda}_{2,XX}
\end{bmatrix},
\label{eq:pec_output_stationary}
\end{equation}
in which the various block matrices can be written as
\begin{align}
\mathbf{\Lambda}_{0,XX} &=E[\tilde{\mathbf{X}}(t)\tilde{\mathbf{X}}^{*T}(t)]=\int_{0}^{\infty}H^{*}(\omega)S(\omega)H^{T}(\omega)\,d\omega,
\label{eq:lambda0_xx}\\
\mathbf{\Lambda}_{1,XX} &=E[\tilde{\mathbf{X}}(t)\dot{\tilde{\mathbf{X}}}^{*T}(t)]=i\int_{0}^{\infty}\omega H^{*}(\omega)S(\omega)H^{T}(\omega)\,d\omega,
\label{eq:lambda1_xx}\\
\mathbf{\Lambda}_{2,XX} &=E[\dot{\tilde{\mathbf{X}}}(t)\dot{\tilde{\mathbf{X}}}^{*T}(t)]=\int_{0}^{\infty}\omega^{2}H^{*}(\omega)S(\omega)H^{T}(\omega)\,d\omega.
\label{eq:lambda2_xx}
\end{align}

The latter equations give, in compact form, all the envelope covariances of the nodal response in the stationary case, showing the perfect correspondence with the pre-envelope covariances of the response process.

\subsection{PEC Matrix of the Output-Nonstationary Case}

In order to obtain the PEC matrix of the output in the nonstationary case, the forcing function vector in equation \eqref{eq:solution_u} must be defined as in equation \eqref{eq:nonstationary_complex}, it follows that the vector solution is given in the form
\begin{align}
\tilde{\mathbf{U}}(t) &=\theta(t)\tilde{\mathbf{U}}_{0}+\int_{0}^{t}L(t-\tau)\tilde{\mathbf{F}}(\tau)\,d\tau\notag\\
&=\theta(t)\mathbf{U}_{0}+\frac{2}{\sqrt{2}}\int_{0}^{\infty}e^{-i\omega t}N^{*}(\omega,t)\,d\mathbf{Z}(\omega),
\label{eq:nonstationary_response}
\end{align}
in which we have set
\begin{equation}
N(\omega,t)=\int_{0}^{t}L(t-\tau)A(\omega,t)e^{-i\omega(t-\tau)}\,d\tau,
\label{eq:n_definition}
\end{equation}
and $\tilde{\mathbf{U}}_{0}$ is the complex vector of initial conditions given as
\begin{equation}
\tilde{\mathbf{U}}_{0}=\frac{2}{\sqrt{2}}(\mathbf{U}_{0}-i\bar{\mathbf{U}}_{0}),
\label{eq:complex_initial}
\end{equation}
in which $\bar{\mathbf{U}}_{0}$ has the same probability distribution as $\mathbf{U}_{0}$.

The time-dependent PEC matrix of the vector $\tilde{\mathbf{U}}(t)$ is given in the form
\begin{equation}
\boldsymbol{\Sigma}_{\tilde{U}\tilde{U}}(t)=\int_{0}^{\infty}N^{*}(\omega,t)S(\omega)N^{T}(\omega,t)\,d\omega+\theta(t)\boldsymbol{\Sigma}_{\tilde{U}\tilde{U}}(0)\theta^{T}(t)+\mathbf{Q}(t)+\mathbf{Q}^{T}(t),
\label{eq:pec_output_nonstationary}
\end{equation}
in which $\boldsymbol{\Sigma}_{\tilde{U}\tilde{U}}(0)$ is the PEC matrix evaluated at time $t=0$, and $\mathbf{Q}(t)$ is the complex matrix given in the form
\begin{equation}
\mathbf{Q}(t)=\theta(t)\int_{0}^{t}E[\tilde{\mathbf{U}}_{0}\tilde{\mathbf{F}}^{T}(\tau)]L^{T}(t-\tau)\,d\tau.
\label{eq:q_matrix}
\end{equation}

For deterministic zero-start conditions, the PEC matrix is given in the simpler form
\begin{equation}
\boldsymbol{\Sigma}_{\tilde{U}\tilde{U}}(t)=\int_{0}^{\infty}N^{*}(\omega,t)S(\omega)N^{T}(\omega,t)\,d\omega,
\label{eq:pec_zero_start}
\end{equation}
in which the various block matrices are given as
\begin{align}
\mathbf{\Lambda}_{0,XX}(t) &=\int_{0}^{\infty}R_{0}^{*}(\omega,t)S(\omega)R_{0}^{T}(\omega,t)\,d\omega,
\label{eq:lambda0_t}\\
\mathbf{\Lambda}_{1,XX}(t) &=\int_{0}^{\infty}R_{0}^{*}(\omega,t)S(\omega)R_{1}^{T}(\omega,t)\,d\omega,
\label{eq:lambda1_t}\\
\mathbf{\Lambda}_{2,XX}(t) &=\int_{0}^{\infty}R_{1}^{*}(\omega,t)S(\omega)R_{1}^{T}(\omega,t)\,d\omega,
\label{eq:lambda2_t}
\end{align}
where
\begin{equation}
\begin{aligned}
R_{0}(\omega,t) &=\int_{0}^{t}H(t-\tau)A(\omega,t)e^{-i\omega\tau}\,d\tau,\\
R_{1}(\omega,t) &=\int_{0}^{t}\dot{H}(t-\tau)A(\omega,t)e^{-i\omega\tau}\,d\tau.
\end{aligned}
\label{eq:r_definitions}
\end{equation}

Evaluating for each frequency the integrals in equations \eqref{eq:r_definitions} and substituting the latter in equations \eqref{eq:lambda0_t}-\eqref{eq:lambda2_t}, the various block matrices of $\boldsymbol{\Sigma}_{\tilde{U}\tilde{U}}(t)$ can be computed.

\section{Numerical Example}

As an application, a two-degree-of-freedom, classically damped system depicted in Figure~\ref{fig:two_dof_system} has been analyzed. In this case the vector solution $\mathbf{X}$ can be evaluated by means of the mode superposition as follows:
\begin{equation}
\mathbf{X}=\Phi\mathbf{Y},
\label{eq:mode_superposition}
\end{equation}
where $\Phi$ is the modal matrix normalized with respect to $M$, and $\mathbf{Y}$ is the vector solution of the decoupled modal differential equations.

\begin{figure}[h]
\centering

\caption{The two-degrees-of-freedom system}
\label{fig:two_dof_system}
\end{figure}

The examined system is characterized by the following data:
\begin{equation*}
M_{1}=M_{2}=1\,\text{kg};\quad K_{1}=50\,\text{N/cm};\quad K_{2}=33\,\text{N/cm}.
\end{equation*}

The modal analysis provided the following results:
\begin{itemize}
\item Natural radian frequencies: $\omega_{1}=3.76$ rad/s; $\omega_{2}=10.93$ rad/s
\item Modal Matrix:
\begin{equation}
\Phi=
\begin{bmatrix}
0.811 & 0.585\\
0.585 & -0.811
\end{bmatrix}
\end{equation}
\item The damping ratio, here assumed equal for both modes, is $\xi=0.05$.
\end{itemize}

The input process is defined as in equation \eqref{eq:nonstationary_complex}, in which $A(\omega,t)$ is given in the form \cite{Spanos1983}:
\begin{equation}
A(\omega,t)=\sqrt{\frac{b\beta(\omega)}{\pi}}\int_{0}^{t}\exp[-(\beta(\omega)+ib)\tau]\,d\tau.
\end{equation}

The process $d\tilde{Z}(\omega)$ is such that
\begin{equation}
E[d\tilde{Z}(\omega_{1})d\tilde{Z}^{*}(\omega_{2})]=\alpha(\omega_{1})\delta(\omega_{2}-\omega_{1})d\omega_{1}\quad(\omega>0).
\end{equation}

The parameters chosen for the analysis are:
\begin{equation*}
b=0.15\,\text{s}^{-1};\quad \beta(\omega)=\alpha(\omega)=\left(\frac{\omega}{5\pi}\right)^{2}\,\text{s}^{-1}.
\end{equation*}

The spectrum of the input is characterized by a dominant frequency decreasing with time from about $5\pi$ rad/sec to $2\pi$ rad/sec, and by the fact that its total power initially increases with time and then gradually decreases.

In Figure~\ref{fig:modal_covariances_1}, the modal covariances of pre-envelope complex processes $\tilde{Y}_{i}$ ($i=1,2$) are plotted.

\begin{figure}[h]
\centering

\caption{Modal pre-envelope covariances: (a) dashed line $E[Y_{1}^{2}]$, full line $40E[\tilde{Y}_{2}^{2}]$; (b) dashed line $E[\dot{Y}_{1}^{2}]$, full line $4E[\dot{Y}_{2}^{2}]$; (c) dashed line $\Re(E[\tilde{Y}_{1}\dot{\tilde{Y}}_{1}^{*}])$, full line $4\Re(E[\tilde{Y}_{2}\dot{\tilde{Y}}_{2}^{*}])$; (d) dashed line $\Im(E[\tilde{Y}_{1}\dot{\tilde{Y}}_{1}^{*}])$, full line $40\Im(E[\tilde{Y}_{2}\dot{\tilde{Y}}_{2}^{*}])$}
\label{fig:modal_covariances_1}
\end{figure}

In these figures it can be seen that the peaks of the curves of the different modes are located at different instants, according to the behavior of the input process. It is to be emphasized that if the function $A(\omega,t)$ and the power spectral density function had been chosen as real functions, all the moments of the evolutionary power would be real functions, while in the new representation, $E[\tilde{Y}_{i}\dot{\tilde{Y}}_{i}^{*}]$, $i=1,2$ are complex functions.

In Figure~\ref{fig:modal_cross_covariances}, the various modal cross-covariances are plotted.

\begin{figure}[h]
\centering

\caption{Modal pre-envelope covariances: (a) dashed line $\Im(E[\tilde{Y}_{1}\dot{\tilde{Y}}_{1}^{*}])$, full line $\Re(E[\tilde{Y}_{1}\dot{\tilde{Y}}_{1}^{*}])$; (b) dashed line $\Im(E[\tilde{Y}_{2}\dot{\tilde{Y}}_{2}^{*}])$, full line $\Re(E[\tilde{Y}_{2}\dot{\tilde{Y}}_{2}^{*}])$; (c) dashed line $\Im(E[\dot{\tilde{Y}}_{1}\dot{\tilde{Y}}_{1}^{*}])$, full line $\Re(E[\dot{\tilde{Y}}_{1}\dot{\tilde{Y}}_{1}^{*}])$}
\label{fig:modal_cross_covariances}
\end{figure}

In Figure~\ref{fig:nodal_covariances}, the (nodal) covariances of the pre-envelope complex process $\tilde{X}_{2}(t)$ (displacements of the second mass) are plotted.

\begin{figure}[h]
\centering

\caption{Nodal pre-envelope covariances: (a) dashed line $E[X_{2}^{2}]$, full line $E[\tilde{X}_{2}\tilde{X}_{2}^{*}]/10$; (b) dashed line $50\Re(E[\tilde{X}_{2}\dot{\tilde{X}}_{2}^{*}])$, full line $\Im(E[\tilde{X}_{2}\dot{\tilde{X}}_{2}^{*}])$}
\label{fig:nodal_covariances}
\end{figure}

From a practical point of view, the numerical evaluation of the nonstationary PEC needs to be conducted in the following way:
\begin{enumerate}
\item First, in a suitable time interval, depending on the behavior of the input process, an adequate number of instants must be selected.
\item For each instant, the $R_{0}(\omega,t)$, $R_{1}(\omega,t)$ complex coefficients given in equation \eqref{eq:r_definitions} have to be evaluated, and an integration over the instantaneous frequency range of the input process for every covariance must be effected according to equations \eqref{eq:lambda0_t}-\eqref{eq:lambda2_t}.
\end{enumerate}

These integrals are difficult to solve analytically, but are not affected by particular computational problems, so that the most delicate aspect of the numerical problem is the evaluation of the $R_{0}(\omega,t)$ and $R_{1}(\omega,t)$ coefficients. If no analytical solution of such integrals can be found, for each instant considered and for each coefficient, a different numerical integration from $0$ to the current instant must be effected. Such integrals depend essentially on the form of the $A(\omega,t)$ input function.

In the present application a closed-form solution was easily found, but it is not reported for brevity's sake.

\section{Conclusions and Discussion}

The probabilistic structures of a real Gaussian process is fully determined by the first two moments (mean and covariance). In some cases of engineering interest, however, we are concerned with the statistics of the so-called envelope, that is, for narrow band process, a smooth curve joining the peaks of the process. Following Dugundji \cite{Dugundji1958} in the stationary case, and Yang \cite{Yang1972} in the nonstationary case, the envelope is defined as the modulus of the pre-envelope, i.e., a complex process, the real part of which is the given process, while the imaginary part is related to the real ones in such a way that the resulting complex process exhibits power only in the positive frequency range. In order to obtain the statistics of the envelope, the variances of the pre-envelope need to be evaluated, rather than the variances of the given real process.

In this paper, the covariances of the pre-envelope processes have been evaluated, and it is shown that in the stationary case these covariances are strictly related to the so-called spectral moments. In particular, PEC matrix has been defined, the real part of which is the well-known covariance matrix of the real process, while its imaginary part contains the lowest imaginary part of the even SM, and the real part contains the first odd SM.

Because the statistical characterization of the envelope requires both the real and the imaginary parts of the complex process, both the real and the imaginary parts of the PEC matrix are essential for the evaluation of the peak statistics of the real process.

In order to extend the previous concepts to the nonstationary case, the complex representation of the nonstationary processes (introduced by Yang) has been adopted and extended to the vector processes, and the covariances of the pre-envelope process has been evaluated. The pre-envelope covariance coincides with the zeroth-order moment of the evolutionary power, while no analogous correspondence can be obtained between the higher moments of the evolutionary power and the covariances of the derivatives of the pre-envelope.

On the other hand, remembering that the SM are useful quantities for the evaluation of the statistic of the peaks, and the latter are related to the moduli of the complex processes, it seems to be more appropriate to evaluate the higher-order time-dependent pre-envelope covariances of the various derivatives of the nonstationary complex processes, instead of the moments of the evolutionary power.

It is shown that the pre-envelope covariances are given as the sum of the traditional higher-order SM obtained as the moments of the evolutionary power and other similar quantities involving the derivatives of the modulating functions.

The pre-envelope covariance of a multi-degree-of-freedom linear system excited by a nonstationary, nonseparable process has been also discussed and the numerical aspects have been evidenced by means of a numerical example.

\begin{thebibliography}{99}

\bibitem{Kameda1975}
Kameda, A., 1975, ``Evolutionary Spectra of Seismogram by Multifilter,'' \textit{Journal of The Engng. Mech. Div.}, Vol. 101, No. EM6, pp. 787--801.

\bibitem{Priestley1965}
Priestley, M. B., 1965, ``Evolutionary Spectra and Non-Stationary Processes,'' \textit{Journal of the Royal Statistical Society}, Vol. 27, pp. 204--228.

\bibitem{Dugundji1958}
Dugundji, J., 1958, ``Envelope and Pre-Envelope of Real Waveforms,'' \textit{IRE Transaction on Information Theory}, Vol. 4, pp. 53--57.

\bibitem{Yang1972}
Yang, J. N., 1972, ``Non-Stationary Envelope Process and First Excursion Probability,'' \textit{Journal of Structural Mechanics}, Vol. 1, pp. 231--248.

\bibitem{Krenk1983}
Krenk, S., Madsen, H. O., and Madsen, P. H., 1983, ``Stationary and Transient Response Envelopes,'' \textit{Journal of Engng. Mech. Div.}, Vol. 109, No. EMI, pp. 263--277.

\bibitem{Arens1957}
Arens, R., 1957, ``Complex processes for envelopes of normal noise,'' \textit{IRE Trans. on Information Theory}, Vol. 3, pp. 204--207.

\bibitem{Vanmarcke1972}
Vanmarcke, E. H., 1972, ``Properties of Special Moments with Application to Random Vibration,'' \textit{Journal of Eng. Mech. Div.}, Vol. 98, pp. 425--446.

\bibitem{DiPaola1985}
Di Paola, M., 1985, ``Transient Spectral Moments of Linear Systems,'' \textit{S. M. Archives}, Vol. 10, pp. 225--243.

\bibitem{Papoulis1965}
Papoulis, A., 1965, ``Probability Random Variables and Stochastic Processes,'' McGraw-Hill, Kogakusha, Tokyo.

\bibitem{Nigam1982}
Nigam, N. C., 1982, ``Phase Properties of a Class of Random Processes,'' \textit{Earthquake Engineering and Structural Dynamics}, Vol. 10, pp. 711--717.

\bibitem{Hammond1968}
Hammond, J. K., 1968, ``On the Response of Single and Multidegree of Freedom Systems to Non-Stationary Random Excitation,'' \textit{Journal of Sound and Vibration}, Vol. 7, pp. 393--416.

\bibitem{Shinozuka1970}
Shinozuka, M., 1970, ``Random Processes with Evolutionary Power,'' \textit{Journal of Eng. Mech. Div.}, Vol. 96, pp. 543--545.

\bibitem{Corotis1972}
Corotis, R. S., Vanmarcke, E. H., and Cornell, C. A., 1972, ``First Passage of Non-Stationary Random Processes,'' \textit{Journal of Engng. Mech. Div.}, Vol. 98, No. EM2, pp. 401--414.

\bibitem{Borino1988}
Borino, G., Di Paola, M., and Muscolino, G., 1988, ``Non-stationary spectral moments of base excited MDOF systems,'' \textit{Earth. Engng. and Struct. Dyn.}, Vol. 16, pp. 745--756.

\bibitem{Muscolino1988}
Muscolino, G., 1988, ``Non-Stationary Envelope in Random Vibration Theory,'' \textit{Journal of Engng. Mech. Div.}, Vol. 114, No. 8, pp. 1396--1413.

\bibitem{To1986}
To, C. W. S., 1986, ``Response Statistic of Discretized Structures to Non-Stationary Vibration,'' \textit{Journal of Sound and Vibration}, Vol. 105, pp. 217--231.

\bibitem{DiPaola1987}
Di Paola, M., and Muscolino, G., 1987, ``Spectral Moments and Envelope for Non-Stationary Non-Separable Processes,'' \textit{Proc. of the Int. Conf. ICASP 5}, Vol. 1, pp. 55--62.

\bibitem{Spanos1983}
Spanos, P. T. D., and Solomos, G. P., 1983, ``Markov Approximation to Transient Vibration,'' \textit{Journal of Eng. Mech. Div.}, Vol. 1, pp. 1134--1149.

\end{thebibliography}

\end{document}
