\documentclass[12pt]{article}
\usepackage[margin=0.5in]{geometry}  % Fix margins
\usepackage{amsmath, amssymb, amsthm}
\usepackage{enumitem}
\usepackage{hyperref}

% Theorem-like environments
\newtheorem{theorem}{Theorem}[section]
\newtheorem{lemma}[theorem]{Lemma}
\newtheorem{proposition}[theorem]{Proposition}
\newtheorem{corollary}[theorem]{Corollary}
\theoremstyle{definition}
\newtheorem{definition}[theorem]{Definition}
\theoremstyle{remark}
\newtheorem{remark}[theorem]{Remark}

\title{Spectral Measures and the Spectral Theorem}
\author{Sam Raskin\\\texttt{sraskin@uchicago.edu}}
\date{University of Chicago REU 2006}

\begin{document}

\maketitle

\section{Preliminary Facts about Hilbert Spaces}
\label{sec:prelim}

We work in a Hilbert space $\mathcal{H}$ with the inner product of $v$ and $w$ denoted $(v \mid w)$.

\subsection{The Riesz Lemma}
\label{subsec:riesz_linear}

\begin{lemma}[Riesz Representation for Linear Functionals]
\label{lem:riesz_linear}
If $\psi$ is a continuous linear functional on $\mathcal{H}$, then there exists a (not necessarily unique if $\psi = 0$) vector $w \in \mathcal{H}$ such that
\begin{equation}
\psi(v) = (v \mid w)
\label{eq:riesz_linear}
\end{equation}
for all $v \in \mathcal{H}$.
\end{lemma}

\begin{proof}
Let $K = \ker(\psi)$. If $K^\perp = \{0\}$, then $\psi \equiv 0$, and the lemma holds with $w = 0$. Otherwise, choose a nonzero $w \in K^\perp$. Normalize $w$ so that
\begin{equation}
\psi(w) = \|w\|^2.
\label{eq:riesz_normalize}
\end{equation}
For a given vector $v \in \mathcal{H}$, decompose $v$ as
\begin{equation}
v = v_1 + v_2,
\qquad
v_1 = \frac{\psi(v)}{\|w\|^2}\, w,
\qquad
v_2 = v - v_1.
\label{eq:riesz_decomp}
\end{equation}
Then
\begin{equation}
\psi(v_2)
= \psi(v) - \psi\!\left(\frac{\psi(v)}{\|w\|^2} w\right)
= \psi(v) - \frac{\psi(v)}{\|w\|^2} \psi(w)
= \psi(v) - \psi(v) = 0,
\label{eq:riesz_v2}
\end{equation}
so $v_2 \in K$. Hence $v_2 \perp w$, and
\begin{equation}
(v \mid w)
= (v_1 + v_2 \mid w)
= (v_1 \mid w) + (v_2 \mid w)
= (v_1 \mid w)
= \left( \frac{\psi(v)}{\|w\|^2} w \mid w \right)
= \psi(v).
\label{eq:riesz_conclusion}
\end{equation}
Thus $\psi(v) = (v \mid w)$ for all $v$.
\end{proof}

\begin{definition}
\label{def:bounded_sesquilinear}
A bounded sesquilinear functional is a map
\begin{equation}
\varphi: \mathcal{H} \times \mathcal{H} \to \mathbb{C}
\label{eq:sesquilinear_def}
\end{equation}
which is linear in the first variable, conjugate-linear in the second, and such that there exists a constant $M \ge 0$ with
\begin{equation}
|\varphi(v,w)| \leq M \,\|v\|\,\|w\|
\label{eq:sesquilinear_bound}
\end{equation}
for all $v,w \in \mathcal{H}$. The least such $M$ is denoted $\|\varphi\|$.
\end{definition}

\begin{lemma}[Riesz Representation for Bounded Sesquilinear Forms]
\label{lem:riesz_sesquilinear}
If $\varphi$ is a bounded sesquilinear functional on $\mathcal{H}$, then there exists a unique bounded operator $A$ on $\mathcal{H}$ such that
\begin{equation}
\varphi(v,w) = (A v \mid w)
\label{eq:riesz_sesquilinear}
\end{equation}
for all $v,w \in \mathcal{H}$.
\end{lemma}

\begin{proof}
Fix $v \in \mathcal{H}$ and define
\begin{equation}
\psi_v(w) = \overline{\varphi(v,w)}.
\label{eq:psi_v_def}
\end{equation}
Then $\psi_v$ is a bounded linear functional of $w$. By Lemma~\ref{lem:riesz_linear}, there exists a unique vector (which we denote $A v$) such that
\begin{equation}
\psi_v(w) = (w \mid A v)
\quad\text{for all } w \in \mathcal{H}.
\label{eq:psi_v_repr}
\end{equation}
Thus
\begin{equation}
\varphi(v,w)
= \overline{\psi_v(w)}
= \overline{(w \mid A v)}
= (A v \mid w),
\label{eq:varphi_repr}
\end{equation}
so \eqref{eq:riesz_sesquilinear} holds.

Linearity of $A$ in $v$ follows from the linearity of $\varphi$ in the first argument. To see boundedness, take $w = A v$ in \eqref{eq:riesz_sesquilinear}:
\begin{equation}
\|A v\|^2
= |(A v \mid A v)|
= |\varphi(v, A v)|
\le \|\varphi\| \,\|v\|\,\|A v\|.
\label{eq:A_bounded}
\end{equation}
If $A v \neq 0$, cancel $\|A v\|$ to get
\begin{equation}
\|A v\| \le \|\varphi\| \,\|v\|.
\label{eq:A_bounded_final}
\end{equation}
If $A v = 0$, then \eqref{eq:A_bounded_final} is trivial. Hence $\|A\| \le \|\varphi\|$.
\end{proof}

\subsection{Adjoints}
\label{subsec:adjoints}

\begin{theorem}[Existence and Uniqueness of Adjoints]
\label{thm:adjoint}
For a bounded operator $A$ on $\mathcal{H}$, there exists a unique bounded operator $A^*$, the adjoint of $A$, satisfying
\begin{equation}
(A v \mid w) = (v \mid A^* w)
\label{eq:adjoint_def}
\end{equation}
for all $v,w \in \mathcal{H}$.
\end{theorem}

\begin{proof}
Define
\begin{equation}
\varphi(w,v) = (w \mid A v).
\label{eq:adjoint_varphi}
\end{equation}
Then $\varphi$ is a bounded sesquilinear functional. By Lemma~\ref{lem:riesz_sesquilinear}, there exists a unique bounded operator $A^*$ such that
\begin{equation}
\varphi(w,v) = (A^* w \mid v)
\quad\text{for all } v,w \in \mathcal{H}.
\label{eq:adjoint_Riesz}
\end{equation}
Thus
\begin{equation}
(A v \mid w) = (w \mid A v) = \overline{(A^* w \mid v)} = (v \mid A^* w),
\end{equation}
which is exactly \eqref{eq:adjoint_def}.
\end{proof}

\begin{definition}
\label{def:hermitian_normal}
An operator $A$ is called \emph{Hermitian} (or self-adjoint) if $A = A^*$. An operator $A$ is called \emph{normal} if
\begin{equation}
\|A v\| = \|A^* v\|
\label{eq:normal_def}
\end{equation}
for all $v \in \mathcal{H}$.
\end{definition}

\begin{proposition}
\label{prop:normal_commute}
An operator $A$ is normal if and only if
\begin{equation}
A A^* = A^* A.
\label{eq:normal_comm}
\end{equation}
\end{proposition}

\begin{proof}
For any $v \in \mathcal{H}$,
\begin{equation}
\|A v\|^2 = (A v \mid A v) = (A^* A v \mid v),
\qquad
\|A^* v\|^2 = (A^* v \mid A^* v) = (A A^* v \mid v).
\label{eq:norms_normal}
\end{equation}
Thus \eqref{eq:normal_def} is equivalent to
\begin{equation}
(A^* A v \mid v) = (A A^* v \mid v)
\quad\text{for all } v.
\label{eq:normal_form}
\end{equation}
Since the inner product is nondegenerate, \eqref{eq:normal_form} holds for all $v$ if and only if $A^* A = A A^*$, i.e., \eqref{eq:normal_comm} holds.
\end{proof}

\begin{remark}
\label{rem:hermitian_from_symmetric}
In Lemma~\ref{lem:riesz_sesquilinear}, if $\varphi$ is symmetric, i.e.
\begin{equation}
\varphi(v,w) = \overline{\varphi(w,v)},
\label{eq:symmetric_form}
\end{equation}
then the resulting operator $A$ is Hermitian.
\end{remark}

\subsection{Projections}
\label{subsec:projections}

\begin{definition}[Orthogonal Projection]
\label{def:projection}
If $\mathcal{M}$ is a closed subspace of $\mathcal{H}$, then every vector $v \in \mathcal{H}$ admits a unique decomposition
\begin{equation}
v = v_1 + v_2,
\quad
v_1 \in \mathcal{M},
\quad
v_2 \in \mathcal{M}^\perp.
\label{eq:proj_decomp}
\end{equation}
The orthogonal projection onto $\mathcal{M}$ is the map
\begin{equation}
P: \mathcal{H} \to \mathcal{H},
\quad
P v = v_1.
\label{eq:projection_def}
\end{equation}
This $P$ is a bounded linear operator. If $\mathcal{M} = \mathcal{H}$, then $P = I$; if $\mathcal{M} = \{0\}$, then $P = 0$.
\end{definition}

\begin{definition}[Order and Sum of Projections]
\label{def:proj_order_sum}
Let $\{P_i\}_{i \in I}$ be projections onto closed subspaces $\mathcal{M}_i \subset \mathcal{H}$. Define a partial order by
\begin{equation}
P_i \le P_j
\quad\Longleftrightarrow\quad
\mathcal{M}_i \subset \mathcal{M}_j.
\label{eq:proj_order}
\end{equation}
We define
\begin{equation}
\sum_{i \in I} P_i
\label{eq:proj_sum}
\end{equation}
to be the orthogonal projection onto the closed linear span of $\bigcup_{i \in I} \mathcal{M}_i$ (interpreted as a strong-operator limit when $I$ is infinite).
\end{definition}

\begin{theorem}[Characterization of Projections]
\label{thm:proj_characterization}
An operator $P$ on $\mathcal{H}$ is an orthogonal projection if and only if it is Hermitian and idempotent:
\begin{equation}
P^2 = P,
\quad
P^* = P.
\label{eq:proj_char}
\end{equation}
\end{theorem}

\begin{proof}
If $P$ is an orthogonal projection onto a closed subspace $\mathcal{M}$, then $P^2 = P$ is immediate from the definition. To see that $P$ is Hermitian, take $v,w \in \mathcal{H}$ and write $v = v_1 + v_2$, $w = w_1 + w_2$ with $v_1,w_1 \in \mathcal{M}$ and $v_2,w_2 \in \mathcal{M}^\perp$. Then
\begin{equation}
(P v \mid w) = (v_1 \mid w_1 + w_2) = (v_1 \mid w_1) + (v_1 \mid w_2) = (v_1 \mid w_1),
\label{eq:proj_herm1}
\end{equation}
and
\begin{equation}
(v \mid P w) = (v_1 + v_2 \mid w_1) = (v_1 \mid w_1) + (v_2 \mid w_1) = (v_1 \mid w_1).
\label{eq:proj_herm2}
\end{equation}
Thus $(P v \mid w) = (v \mid P w)$ for all $v,w$, so $P$ is Hermitian.

Conversely, suppose $P$ is Hermitian and idempotent. Define
\begin{equation}
\mathcal{M} = \{ v \in \mathcal{H} : P v = v\}.
\label{eq:M_fix}
\end{equation}
We claim $P$ is the orthogonal projection onto $\mathcal{M}$. It suffices to show that
\begin{equation}
(P v \mid v - P v) = 0
\label{eq:proj_orth}
\end{equation}
for all $v \in \mathcal{H}$. Indeed,
\begin{equation}
(P v \mid v - P v)
= (P v \mid v) - (P v \mid P v)
= (P v \mid v) - (P^2 v \mid v)
= (P v \mid v) - (P v \mid v) = 0.
\end{equation}
Thus $P v$ and $v - P v$ are orthogonal, with $P v \in \mathcal{M}$ and $v - P v \in \mathcal{M}^\perp$, so $P$ is the orthogonal projection onto $\mathcal{M}$.
\end{proof}

\begin{corollary}
\label{cor:proj_norm}
If $P$ is an orthogonal projection, then for all $v \in \mathcal{H}$,
\begin{equation}
\|P v\|^2 = (P v \mid v).
\label{eq:proj_norm}
\end{equation}
\end{corollary}

\begin{proof}
Using $P^2 = P$ and Hermitian symmetry,
\begin{equation}
\|P v\|^2 = (P v \mid P v) = (P^2 v \mid v) = (P v \mid v),
\end{equation}
as claimed.
\end{proof}

\section{Spectral Measures}
\label{sec:spectral_measures}

\subsection{Definition and Basic Properties}
\label{subsec:spectral_basic}

Let $\mathcal{B}(\mathbb{C})$ be the Borel $\sigma$-algebra on $\mathbb{C}$ and $P(\mathcal{H})$ the set of orthogonal projections on $\mathcal{H}$.

\begin{definition}[Spectral Measure]
\label{def:spectral_measure}
A (complex) spectral measure is a function
\begin{equation}
E : \mathcal{B}(\mathbb{C}) \to P(\mathcal{H})
\label{eq:E_def}
\end{equation}
satisfying:
\begin{enumerate}[label=(\roman*)]
\item $E(\emptyset) = 0$ and $E(\mathbb{C}) = I$.
\item If $\{B_n\}_{n \in \mathbb{N}}$ is a family of pairwise disjoint Borel sets, then
\begin{equation}
E\Bigl(\bigcup_{n} B_n\Bigr) = \sum_{n} E(B_n),
\label{eq:spec_countable_additivity}
\end{equation}
where the sum is taken in the strong operator topology.
\end{enumerate}
\end{definition}

\begin{remark}
\label{rem:spectral_measure_properties}
Standard measure-theoretic arguments adapt to spectral measures:
\begin{enumerate}[label=(\alph*)]
\item The condition $E(\emptyset) = 0$ follows from \eqref{eq:spec_countable_additivity}.
\item If $B_0 \subset B_1$, then $E(B_0) \le E(B_1)$ in the projection order.
\item Spectral measures are modular:
\begin{equation}
E(B_0 \cup B_1) + E(B_0 \cap B_1) = E(B_0) + E(B_1).
\label{eq:modularity}
\end{equation}
In particular, one obtains
\begin{equation}
E(B_0 \cap B_1) = E(B_0) E(B_1).
\label{eq:intersection_product}
\end{equation}
\end{enumerate}
\end{remark}

\begin{proposition}
\label{prop:spectral_measure_characterization}
Suppose $E : \mathcal{B}(\mathbb{C}) \to P(\mathcal{H})$ is such that for all $v,w \in \mathcal{H}$, the scalar-valued function
\begin{equation}
E^{*}_{v,w}(B) = (E(B) v \mid w)
\label{eq:E_star}
\end{equation}
satisfies
\begin{equation}
E^{*}_{v,w}\Bigl(\bigcup_n B_n\Bigr) = \sum_n E^{*}_{v,w}(B_n)
\label{eq:E_star_additivity}
\end{equation}
for all disjoint families $\{B_n\}$, and $E(\mathbb{C}) = I$. Then $E$ is a spectral measure.
\end{proposition}

\begin{proof}
By Remark~\ref{rem:spectral_measure_properties}, it suffices to verify \eqref{eq:spec_countable_additivity}. Let $\{B_n\}$ be a disjoint family of Borel sets. For any $v \in \mathcal{H}$,
\begin{equation}
\sum_{n} \|E(B_n) v\|^2
= \sum_{n} (E(B_n) v \mid v)
= \sum_{n} E^{*}_{v,v}(B_n)
= E^{*}_{v,v}\Bigl(\bigcup_n B_n\Bigr)
= (E(\bigcup_n B_n) v \mid v).
\label{eq:spectral_Pythag}
\end{equation}
Thus $\sum_n E(B_n) v$ converges in $\mathcal{H}$, and the strong operator limit
\begin{equation}
\sum_n E(B_n)
\label{eq:spectral_sum_strong}
\end{equation}
exists. For disjoint $B,C$ and any $v,w$,
\begin{equation}
(E(B) v + E(C) v \mid w)
= (E(B) v \mid w) + (E(C) v \mid w)
= E^{*}_{v,w}(B) + E^{*}_{v,w}(C)
= E^{*}_{v,w}(B \cup C),
\end{equation}
so
\begin{equation}
(E(B \cup C) v \mid w) = \bigl( (E(B) + E(C)) v \mid w \bigr).
\end{equation}
By induction and taking limits, we obtain \eqref{eq:spec_countable_additivity}.
\end{proof}

% Further sections would continue similarly...

\begin{thebibliography}{9}
\bibitem{Ha}
P.~R.~Halmos,
\textit{Introduction to Hilbert Space},
Chelsea Publishing, New York, 1951.

\bibitem{R1}
W.~Rudin,
\textit{Principles of Mathematical Analysis},
3rd ed., McGraw--Hill, New York, 1976.

\bibitem{R2}
W.~Rudin,
\textit{Real and Complex Analysis},
3rd ed., McGraw--Hill, New York, 1987.
\end{thebibliography}

\end{document}
