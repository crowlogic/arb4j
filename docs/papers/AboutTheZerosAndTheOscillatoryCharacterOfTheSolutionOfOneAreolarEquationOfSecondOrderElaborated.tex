\documentclass[12pt,oneside]{article}
\usepackage[utf8]{inputenc}
\usepackage[T1]{fontenc}
\usepackage[english]{babel}
\usepackage{amsmath}
\usepackage{amssymb}
\usepackage{amsthm}
\usepackage{geometry}
\usepackage{setspace}
\usepackage{hyperref}
\usepackage{lmodern}

\geometry{margin=1.25in}
\onehalfspacing

\theoremstyle{plain}
\newtheorem{theorem}{Theorem}
\newtheorem{lemma}[theorem]{Lemma}
\newtheorem{proposition}[theorem]{Proposition}
\newtheorem{corollary}[theorem]{Corollary}

\theoremstyle{definition}
\newtheorem{definition}[theorem]{Definition}
\newtheorem{example}[theorem]{Example}
\newtheorem{remark}[theorem]{Remark}

\title{Areolar Differential Equations and Complex-Conjugate Analysis: \\
An Exposition on Zeros and Oscillatory Solutions}

\author{}
\date{}

\begin{document}

\maketitle

\begin{abstract}
This document provides a rigorous mathematical investigation into the spectral and oscillatory properties of solutions to second-order linear areolar differential equations with constant coefficients. We bridge classical oscillation theory in real differential equations with the domain of complex-conjugate analysis, revealing fundamental structural correspondences while exposing the distinctive character that emerges when solutions are permitted to vary independently with respect to both $z$ and $\overline{z}$. The work synthesizes the Kolosov-Polozhii operational calculus, characteristic root analysis, and the topology of zero sets within the non-holomorphic complex plane.
\end{abstract}

\section{Introduction: The Areolar Derivative Framework}

The entire construction rests upon the operational calculus of areolar (or Cauchy-Riemann conjugate) derivatives, introduced by G.V.\ Kolosov in 1909 as a tool for solving problems in elasticity theory. Rather than restricting analysis to holomorphic functions (which satisfy the Cauchy-Riemann equations), the areolar derivative formalism permits the independent treatment of $z$ and $\overline{z}$ as separate variables within a unified operator algebra.

For a complex-valued function $W(z, \overline{z}) = u(x,y) + i\nu(x,y)$, the areolar derivatives are defined as:

\begin{equation}
\hat{\frac{\partial W}{\partial z}} = \frac{1}{2}\left[\frac{\partial u}{\partial x} + \frac{\partial \nu}{\partial y} + i\left(\frac{\partial \nu}{\partial x} - \frac{\partial u}{\partial y}\right)\right]
\end{equation}

\begin{equation}
\hat{\frac{\partial W}{\partial \overline{z}}} = \frac{1}{2}\left[\frac{\partial u}{\partial x} - \frac{\partial \nu}{\partial y} + i\left(\frac{\partial \nu}{\partial x} + \frac{\partial u}{\partial y}\right)\right]
\end{equation}

This formalism admits a crucial observation: when $W$ is holomorphic (satisfying Cauchy-Riemann), we have $\hat{\partial W}/\partial \overline{z} = 0$, recovering the classical theory as a special case. Conversely, non-holomorphic functions enter as irreducible degrees of freedom. The second areolar derivative $\hat{\partial^2 W}/\partial \overline{z}^2$ is then constructed as the composition of this operator with itself.

\begin{remark}
The operators $\hat{\partial}/\partial z$ and $\hat{\partial}/\partial \overline{z}$ are known as Wirtinger derivatives in modern literature and form the foundation of complex analysis in multiple complex variables.
\end{remark}

\section{The Central Problem: Oscillation Theory in the Complex Domain}

The classical theory of oscillatory differential equations in $\mathbb{R}$ concerns equations of the form:

\begin{equation}
\frac{d^2x}{dt^2} + a(t)x = 0
\end{equation}

Oscillation is understood in the elementary sense: a solution undergoes repeated sign changes. The oscillatory character emerges when the coefficient function $a(t)$ satisfies positivity and sufficient growth conditions (specifically, $\int_0^\infty a(t)\,dt = \infty$).

Brsakoska poses the fundamental question: does an analogous notion of oscillation exist for complex-conjugate differential equations? The equation under investigation is:

\begin{equation}
\hat{\frac{d^2W}{d\overline{z}^2}} + (\alpha + i\beta)W = 0
\end{equation}

where $K = \alpha + i\beta$ is a complex constant and $W = W(z, \overline{z})$ is the unknown complex-valued function of two independent complex variables.

The difficulty is immediate: oscillation for complex functions cannot mean sign change in the elementary sense---a complex number possesses no ordering. The paper transcends this obstacle through a subtle reinterpretation: oscillatory behavior corresponds to the existence of isolated zeros of $W$, where the topology of the zero set reflects the underlying harmonic structure.

\section{Resolution of the Characteristic Equation and Construction of Particular Solutions}

Brsakoska seeks particular solutions of the form $W = e^{r\overline{z}}$, where $r$ is to be determined. Computing the areolar derivatives:

\begin{equation}
\hat{\frac{dW}{d\overline{z}}} = re^{r\overline{z}}, \quad \hat{\frac{d^2W}{d\overline{z}^2}} = r^2e^{r\overline{z}}
\end{equation}

Substitution into the differential equation yields the characteristic equation:

\begin{equation}
r^2 + (\alpha + i\beta) = 0
\end{equation}

whence:

\begin{equation}
r = (-\alpha - i\beta)^{1/2}
\end{equation}

This square root admits two values in $\mathbb{C}$. Writing $-\alpha - i\beta = \rho e^{i\theta}$ with $\rho = \sqrt{\alpha^2 + \beta^2}$ and $\theta = \arctan(\beta/\alpha)$, the two roots are:

\begin{equation}
r_1 = \sqrt{\frac{\sqrt{\alpha^2+\beta^2} + \alpha}{2}} + i\sqrt{\frac{\sqrt{\alpha^2+\beta^2} - \alpha}{2}}
\end{equation}

\begin{equation}
r_2 = -\sqrt{\frac{\sqrt{\alpha^2+\beta^2} + \alpha}{2}} - i\sqrt{\frac{\sqrt{\alpha^2+\beta^2} - \alpha}{2}}
\end{equation}

Note the crucial relation: $r_2 = -r_1$, and $r_1^2 = r_2^2 = -\alpha - i\beta$.

The two particular solutions are $W_1 = e^{r_1\overline{z}}$ and $W_2 = e^{r_2\overline{z}}$.

\section{The General Solution and Generalized Constants}

\begin{theorem}
The general solution of the areolar equation 
\[
\hat{\frac{d^2W}{d\overline{z}^2}} + (\alpha + i\beta)W = 0
\]
where $\alpha + i\beta$ is a constant with positive real and imaginary part, is given by
\[
W(z, \overline{z}) = C_1(z)e^{r_1\overline{z}} + C_2(z)e^{r_2\overline{z}}
\]
where $C_1(z)$ and $C_2(z)$ are arbitrary analytic functions of the single complex variable $z$, in the role of generalized constants.
\end{theorem}

\begin{proof}
The particular integrals $W_1$ and $W_2$ according to the exponential ansatz satisfy the equation. We verify that with elimination of $C_1(z)$ and $C_2(z)$ from the general form, we recover the original equation and only that equation.

Computing derivatives:
\begin{align}
\hat{\frac{dW}{d\overline{z}}} &= C_1(z)r_1 e^{r_1\overline{z}} + C_2(z)r_2 e^{r_2\overline{z}} \\
\hat{\frac{d^2W}{d\overline{z}^2}} &= C_1(z)r_1^2 e^{r_1\overline{z}} + C_2(z)r_2^2 e^{r_2\overline{z}}
\end{align}

Since $r_1^2 = r_2^2 = -\alpha - i\beta$ by construction, we obtain:
\[
\hat{\frac{d^2W}{d\overline{z}^2}} = (-\alpha - i\beta)\left[C_1(z)e^{r_1\overline{z}} + C_2(z)e^{r_2\overline{z}}\right] = (-\alpha - i\beta)W
\]

Rearranging:
\[
\hat{\frac{d^2W}{d\overline{z}^2}} + (\alpha + i\beta)W = 0
\]

Thus the general form satisfies the equation. Conversely, any solution must be expressible in this form by linear independence of the exponential terms.
\end{proof}

\begin{remark}
This structure reveals a profound distinction from ordinary differential equations. In the standard case, constants are scalars (determined by two initial conditions); here, they are analytic functions of $z$. This reflects the non-locality of the problem: the solution at a point depends on the choice of holomorphic structure throughout a region, not merely on pointwise initial conditions. The dimension of the solution space is infinite-dimensional, indexed by pairs of entire or meromorphic functions.
\end{remark}

\section{The Nature of Zeros: Isolated and Countable}

\begin{theorem}
The areolar equation from second order
\[
\hat{\frac{d^2W}{d\overline{z}^2}} + KW = 0
\]
where $W = W(z, \overline{z})$ and $K$ is an arbitrary complex constant, can have only isolated zeroes, which depend from the analytic functions $C_1(z)$ and $C_2(z)$, in the form of generalized constants and that zeroes are either common zeroes of $C_1(z)$ and $C_2(z)$, or they are countably many zeroes of the complex combination
\[
C_1(z)e^{r_1\overline{z}} + C_2(z)e^{r_2\overline{z}} = 0.
\]
\end{theorem}

\begin{proof}
Setting $W = 0$ in the general solution:
\[
C_1(z)e^{r_1\overline{z}} + C_2(z)e^{r_2\overline{z}} = 0
\]

Rearranging:
\[
C_2(z) = -C_1(z)e^{(r_1-r_2)\overline{z}}
\]

Since $r_1 - r_2 = 2\lambda$ where $\lambda = \sqrt{\frac{\sqrt{\alpha^2+\beta^2} + \alpha}{2}}$ is real and positive (for $\alpha, \beta > 0$), we have:
\[
C_2(z) = -C_1(z)e^{2\lambda\overline{z}}
\]

The decisive observation: since $C_1(z)$ and $C_2(z)$ are analytic functions of the single complex variable $z$, and since $\overline{z}$ is non-analytic in $z$, the equation can be satisfied only when:

\begin{enumerate}
\item $C_1(z) = 0$ (which by continuity implies $C_2(z) = 0$), or
\item At isolated points where the transcendental equation $C_1(z)e^{2\lambda\overline{z}} + C_2(z) = 0$ is satisfied.
\end{enumerate}

An analytic function possesses only isolated zeros (by the open mapping theorem and the uniqueness principle for holomorphic functions). Therefore, the solution $W$ has zeros that are either:
\begin{enumerate}
\item Common zeros of $C_1(z)$ and $C_2(z)$ (isolated by analyticity), or
\item Points where the nonlinear combination $C_1(z)e^{2\lambda\overline{z}} + C_2(z) = 0$ holds (countably many, generically forming discrete point sets).
\end{enumerate}

The zero set cannot form curves or regions except in degenerate cases where special algebraic relationships between the coefficient functions occur, violating the genericity assumption.
\end{proof}

\begin{remark}
This contrasts sharply with real differential equations, where generic solutions are either everywhere nonzero or possess isolated zeros depending on initial conditions. Here, the non-holomorphic structure forces a fundamentally discrete zero set. The topological rigidity of the zero set reflects the incompatibility between the analytic structure of the generalized constants and the anti-analytic structure introduced by dependence on $\overline{z}$.
\end{remark}

\section{Example 1: The Positive Real Case}

Consider the equation:
\[
\hat{\frac{d^2W}{d\overline{z}^2}} + W = 0
\]

Here $\alpha = 1, \beta = 0$, yielding $\rho = 1$ and $\theta = \pi$.

The characteristic roots are:
\[
r_1 = 1^{1/2}\left[\cos(\pi/2) + i\sin(\pi/2)\right] = i
\]
\[
r_2 = 1^{1/2}\left[\cos(\pi/2 + \pi) + i\sin(\pi/2 + \pi)\right] = -i
\]

The general solution is:
\[
W = C_1(z)e^{i\overline{z}} + C_2(z)e^{-i\overline{z}}
\]

Taking $C_1(z) = C_2(z) = 1$ (the simplest case):
\[
W_p = e^{i\overline{z}} + e^{-i\overline{z}} = 2\cos(\overline{z}) = 2\cos(x-iy)
\]

Expanding using complex trigonometric identities:
\[
\cos(x-iy) = \cos x \cosh y + i\sin x \sinh y
\]

Thus:
\[
u(x,y) = \cos x \cosh y, \quad v(x,y) = \sin x \sinh y
\]

The zeros satisfy $u = 0$ and $v = 0$ simultaneously. Since $\cosh y \neq 0$ for all real $y$, we require $\cos x = 0$, giving $x = (2k+1)\pi/2$. Since $\sinh y = 0$ only when $y = 0$, the zeros form a countably infinite discrete set:
\[
\left\{\left((2k+1)\frac{\pi}{2}, 0\right) : k \in \mathbb{Z}\right\}
\]

This is the analog of oscillatory behavior in the complex setting: the solution exhibits oscillatory character in both real and imaginary parts, and the zero set captures this oscillation geometrically as a countable collection of isolated points. 

\textbf{Critical observation:} this case mirrors the classical $y'' + y = 0$ in that the sign of the coefficient is positive.

\section{Example 2: The Negative Real Case}

Consider the equation:
\[
\hat{\frac{d^2W}{d\overline{z}^2}} - W = 0
\]

Here $\alpha = -1, \beta = 0$, so $-\alpha - i\beta = 1$.

The characteristic roots are:
\[
r_1 = 1^{1/2}\left[\cos(0) + i\sin(0)\right] = 1
\]
\[
r_2 = 1^{1/2}\left[\cos(\pi) + i\sin(\pi)\right] = -1
\]

The general solution is:
\[
W = C_1(z)e^{\overline{z}} + C_2(z)e^{-\overline{z}}
\]

With $C_1(z) = C_2(z) = 1$:
\[
W_p = e^{\overline{z}} + e^{-\overline{z}} = 2\cosh(\overline{z}) = 2\cosh(x-iy)
\]

Expanding:
\[
\cosh(x-iy) = \cosh x \cos y - i\sinh x \sin y
\]

Thus:
\[
u(x,y) = \cosh x \cos y, \quad v(x,y) = -\sinh x \sin y
\]

Zeros require $u = 0$ and $v = 0$. Since $\cosh x > 0$ for all real $x$, we need $\cos y = 0$, giving $y = (2k+1)\pi/2$. Since $\sinh x = 0$ only when $x = 0$, the zeros form the set:
\[
\left\{\left(0, (2k+1)\frac{\pi}{2}\right) : k \in \mathbb{Z}\right\}
\]

Again a countable collection, now aligned on the imaginary axis.

\textbf{Brsakoska's critical observation:} Despite the sign change from $+1$ to $-1$, both cases exhibit oscillatory character and countable zero sets. The dichotomy in real differential equations---where $y'' + y = 0$ oscillates while $y'' - y = 0$ exhibits exponential growth---dissolves in the areolar context.

\section{Example 3: The Purely Imaginary Case}

For the equation:
\[
\hat{\frac{d^2W}{d\overline{z}^2}} + iW = 0
\]

we have $\alpha = 0, \beta = 1$, so $\rho = 1$ and $\arg(-i) = -\pi/2 = 3\pi/2$.

The characteristic roots are:
\[
r_1 = 1^{1/2}\left[\cos(3\pi/4) + i\sin(3\pi/4)\right] = -\frac{1}{\sqrt{2}} + i\frac{1}{\sqrt{2}} = \frac{1+i}{\sqrt{2}}
\]
\[
r_2 = -\frac{1+i}{\sqrt{2}}
\]

The general solution is:
\[
W = C_1(z)e^{\frac{1+i}{\sqrt{2}}\overline{z}} + C_2(z)e^{-\frac{1+i}{\sqrt{2}}\overline{z}}
\]

For suitable choices of $C_1$ and $C_2$, the zero set again forms countable isolated points, typically (for generic parameters) along coordinate lines in the $(x,y)$ plane.

The exponential factors decompose as:
\[
\exp\left(\frac{1+i}{\sqrt{2}}(x-iy)\right) = \exp\left(\frac{x}{\sqrt{2}} + \frac{y}{\sqrt{2}}\right) \cdot \exp\left(i\left(\frac{x}{\sqrt{2}} - \frac{y}{\sqrt{2}}\right)\right)
\]

This combines:
\begin{enumerate}
\item Exponential growth in the direction of increasing $x + y$
\item Oscillation in the direction perpendicular to $x = y$
\end{enumerate}

\section{The Profound Structural Insight: Sign Independence of Oscillatority}

The paper's deepest contribution lies in recognizing that \textbf{oscillatory behavior in the areolar setting is not destroyed by sign changes in the coefficient}---a stark departure from real ODEs. The reason is subtle: oscillation is no longer about monotonicity and sign flipping, but about the geometric distribution of zeros in the complex plane.

The non-holomorphic character of $W$ as a function of $\overline{z}$ introduces a \textbf{fundamental rigidity}. Solutions cannot smoothly vary without respecting the coupling between the $C_1(z)e^{r_1\overline{z}}$ and $C_2(z)e^{r_2\overline{z}}$ terms. Even when one term (say, $e^{\overline{z}}$ in Example 2) would normally be considered non-oscillatory in the real sense, the algebraic structure forces the appearance of discrete zeros through the interplay of both components.

In Examples 1 and 2:
\begin{enumerate}
\item Example 1 ($+$ sign): zeros on the \textbf{real axis}
\item Example 2 ($-$ sign): zeros on the \textbf{imaginary axis}
\end{enumerate}

The sign change merely \textbf{rotates} the zero set by $90°$. It does \textbf{NOT} affect whether the solution oscillates or not. Both solutions have oscillatory components in their real and imaginary parts.

\textbf{Conclusion on signs:} The sign of $K$ does not determine oscillatority. Rather, it determines \textbf{WHERE} the zero set lies geometrically in the complex plane.

\section{Mathematical Underpinnings: Analyticity and the Cauchy-Riemann Structure}

The constraint that $C_1(z)$ and $C_2(z)$ are analytic functions of $z$ alone imposes severe restrictions on the solution manifold. An analytic function $f(z) = p(x,y) + iq(x,y)$ satisfies:

\begin{equation}
\frac{\partial p}{\partial x} = \frac{\partial q}{\partial y}, \quad \frac{\partial p}{\partial y} = -\frac{\partial q}{\partial x}
\end{equation}

These are the Cauchy-Riemann equations. Any function violating these cannot be analytic, and any solution involving such a function immediately introduces non-holomorphic character.

The equation $C_2(z) = -C_1(z)e^{2\lambda\overline{z}}$ presents precisely this situation: the right side depends on $\overline{z}$, which explicitly violates Cauchy-Riemann when composed with any non-constant function. This incompatibility forces isolation of the zero set.

\section{Codimension and Topology of the Zero Set}

For a general function $W = u + iv$ in $\mathbb{R}^2$, the condition $W = 0$ requires:
\[
u(x,y) = 0 \quad \text{AND} \quad v(x,y) = 0
\]

This is a \textbf{codimension-2} condition: two independent equations in two variables generically yield a 0-dimensional solution set (isolated points).

By contrast, a single equation $u(x,y) = 0$ defines a codimension-1 set (typically a curve).

For areolar solutions with analytic generalized constants, the transcendental equation $C_1(z)e^{2\lambda\overline{z}} + C_2(z) = 0$ presents a genuinely nonlinear constraint. Generically, this equation is satisfied on a discrete point set, not on curves or regions.

\section{Asymptotic Behavior and Growth Rates}

For the particular solution $W = e^{r\overline{z}}$ where $r = a + ib$ (with $a, b \in \mathbb{R}$):

\begin{equation}
|W| \sim \exp(ax + by)
\end{equation}

The growth is \textbf{directional}: solutions accelerate exponentially along the direction $\arg(r)$ and decay along $\arg(r) + \pi$.

For $K = \alpha + i\beta$ with both $\alpha, \beta > 0$:
\[
r = \sqrt{\frac{\sqrt{\alpha^2+\beta^2} + \alpha}{2}} + i\sqrt{\frac{\sqrt{\alpha^2+\beta^2} - \alpha}{2}}
\]

The real part is always positive: $\Re(r) > 0$, ensuring $x$-direction growth. The imaginary part sign depends on $\beta$:
\begin{enumerate}
\item If $\beta > 0$: $\Im(r) > 0$ $\Rightarrow$ also grows in $+y$ direction
\item If $\beta < 0$: $\Im(r) < 0$ $\Rightarrow$ decays in $+y$ direction
\end{enumerate}

This asymptotic behavior is preserved by the arbitrary analytic factors $C_1(z), C_2(z)$ up to polynomial modulation---the exponential envelope dominates at infinity.

\section{Elliptic Structure and Fredholm Properties}

The operator $\hat{\partial}^2/\partial\overline{z}^2$ is \textbf{elliptic} in the complex sense. Its principal symbol in Fourier space is:
\[
\sigma_2(\xi, \eta) = (\xi - i\eta)^2
\]

This vanishes only when $\xi = i\eta$, which in real coordinates corresponds to the origin in Fourier space.

Ellipticity guarantees:
\begin{enumerate}
\item No characteristic hypersurfaces (unlike hyperbolic operators such as the wave equation)
\item Unique continuation: if $W$ vanishes on an open set, then $W \equiv 0$ everywhere
\item Hypoellipticity: smooth coefficients yield smooth solutions
\item Fredholm property: when restricted appropriately, the operator is Fredholm of index 0
\end{enumerate}

The operator possesses an infinite-dimensional kernel (the solution space parametrized by analytic functions $C_1(z), C_2(z)$) and a finite-dimensional cokernel, a configuration allowed by the Fredholm index being zero.

\section{The Vekua Equation Connection}

The areolar framework connects to the more general Vekua equation:
\[
\hat{\frac{\partial W}{\partial \overline{z}}} = \sigma(z, \overline{z}) \cdot W
\]

For second-order equations, we have generalizations of classical elliptic and hyperbolic theory.

The key structural difference from holomorphic PDE:
\begin{enumerate}
\item Holomorphic functions ($\hat{\partial W}/\partial \overline{z} = 0$): rigid, satisfy maximum principle
\item Areolar functions ($\hat{\partial W}/\partial \overline{z} \neq 0$): flexible, solution manifold is infinite-dimensional
\end{enumerate}

This flexibility explains why the general solution contains arbitrary analytic functions $C_1(z), C_2(z)$ as generalized constants.

\section{Applications to Mathematical Physics}

\subsection{Two-Dimensional Elasticity}

Historically, Kolosov introduced areolar derivatives to solve elasticity problems. In 2D elasticity, stress tensor components $\sigma_{xx}, \sigma_{yy}, \sigma_{xy}$ can be expressed via a complex stress function:
\[
\phi(z, \overline{z}) = \text{complex stress potential}
\]

The equations of equilibrium become:
\[
\hat{\frac{d^2\phi}{d\overline{z}^2}} + \text{boundary conditions} = 0
\]

This is precisely the areolar equation! The oscillatory solutions describe:
\begin{enumerate}
\item Stress wave propagation through materials
\item Resonance phenomena at characteristic frequencies
\item Complex coupling between tensile and shear components
\end{enumerate}

The zero set of the stress function corresponds to lines of zero normal stress---critical features in engineering analysis.

\subsection{Connection to Complex Geometry}

The non-holomorphic structure of areolar equations provides a bridge between complex analysis and real differential geometry. Solutions exhibit properties intermediate between holomorphic functions (rigid, satisfying Cauchy-Riemann) and general smooth functions (flexible, no constraints).

\section{Higher-Order Generalizations}

The pattern extends to $n$-th order equations:
\[
\frac{\hat{\partial}^n W}{\partial \overline{z}^n} + A(z, \overline{z})W = 0
\]

For constant coefficients:
\[
\frac{\hat{\partial}^n W}{\partial \overline{z}^n} + K \cdot W = 0, \quad K \in \mathbb{C}
\]

The characteristic equation becomes:
\[
r^n + K = 0 \implies r = n\text{-th root of } (-K)
\]

There are $n$ distinct roots:
\[
r_k = |K|^{1/n} \exp\left(i\frac{\arg(-K) + 2\pi k}{n}\right), \quad k = 0, 1, \ldots, n-1
\]

The general solution is:
\[
W(z, \overline{z}) = \sum_{k=0}^{n-1} C_k(z)e^{r_k\overline{z}}
\]

with $C_k(z)$ arbitrary analytic functions.

The solution space remains infinite-dimensional, but the zero set structure becomes more complex: differences $r_j - r_k$ can be real or complex, allowing zeros to form curves or higher-dimensional surfaces in special cases.

\section{Regularity Theory and Schauder Estimates}

A fundamental question: how smooth are solutions to the areolar equation?

\begin{theorem}[Regularity Propagation]
If $A(z, \overline{z}) \in C^k$ and $W$ solves $\hat{\partial}^2 W/\partial \overline{z}^2 + AW = 0$, then $W \in C^{k+2}$.

More precisely, by Schauder estimates:
\[
\|W\|_{C^{2,\alpha}} \leq C \left(\|AW\|_{C^{0,\alpha}} + \|W\|_{L^2}\right)
\]
where $C^{2,\alpha}$ is the space of functions with continuous second derivatives and $\alpha$-Holder continuous first derivatives.
\end{theorem}

For areolar equations, the regularity is:
\begin{enumerate}
\item $C^\infty$ smooth (analytic in $z$) if coefficients are analytic
\item Holder continuous with exponent matching the coefficient regularity
\end{enumerate}

\textbf{Critical observation:} For particular solutions $W = C(z) \exp(r\overline{z})$:

If $C(z)$ is analytic, then $W$ is analytic in $x$ but only smooth (not analytic) in $y$. This reflects the anti-holomorphic dependence on $y$ through $\overline{z} = x - iy$.

\section{Unique Continuation}

A fundamental property follows from ellipticity:

\begin{theorem}[Unique Continuation]
If $W$ solves $\hat{\partial}^2 W/\partial \overline{z}^2 + KW = 0$ on a connected domain $D$ and vanishes on an open subset $U \subset D$, then $W \equiv 0$ throughout $D$.
\end{theorem}

This is vastly more restrictive than in real differential equations, where solutions can exhibit exponential decay without global vanishing. The non-holomorphic structure enforces global rigidity.

\section{Microlocal Analysis Perspective}

From the viewpoint of microlocal analysis, the principal symbol of $\hat{\partial}^2/\partial\overline{z}^2$ on the cotangent bundle $T^*\mathbb{R}^2$ is:
\[
\sigma_2(x, y, \xi, \eta) = (\xi - i\eta)^2
\]

The characteristic set is:
\[
\text{Char}(\hat{\partial}^2/\partial\overline{z}^2) = \{(\xi, \eta) \in \mathbb{R}^4 : (\xi - i\eta)^2 = 0\} \cap \mathbb{R}^2 = \{0\}
\]

Characteristics exist only at the origin in real frequency space, confirming strict ellipticity. This ensures:
\begin{enumerate}
\item Wave front regularity properties
\item Justification for the infinite-dimensional smooth solution manifold
\item Propagation of singularities is trivial
\end{enumerate}

This contrasts with hyperbolic operators (e.g., wave equation) where characteristic surfaces form nontrivial cones.

\section{Spectral Theory and Fredholm Operators}

For the operator $T[W] = \hat{\partial}^2 W/\partial\overline{z}^2 + KW$ on appropriate Hilbert spaces:

The kernel is:
\[
\ker(T) = \{C_1(z)e^{r_1\overline{z}} + C_2(z)e^{r_2\overline{z}} : C_1, C_2 \text{ analytic}\}
\]

This is infinite-dimensional but of \textbf{finite codimension}.

By Fredholm theory, the cokernel $\text{coker}(T) = L^2(\mathbb{C})^*/\text{Im}(T)$ is finite-dimensional, and:
\[
\text{index}(T) = \dim(\ker(T)) - \dim(\text{coker}(T)) = 0
\]

For the resolvent operator $(T - \lambda)^{-1}$:
\begin{enumerate}
\item For $\lambda \notin \sigma(T)$: the resolvent exists and is bounded
\item For $\lambda \in \sigma(T)$: the resolvent is unbounded or does not exist
\end{enumerate}

The spectrum is contained in a half-plane of $\mathbb{C}$, a consequence of the elliptic structure.

\section{Gauge Invariance and Hidden Symmetries}

The areolar equation admits gauge transformations. If $W$ solves $\hat{\partial}^2 W/\partial\overline{z}^2 + KW = 0$, then:
\[
W' = e^{\phi(z, \overline{z})} \cdot W
\]
also solves a modified equation, provided $\phi$ satisfies compatibility conditions.

This suggests that the solution space possesses \textbf{hidden symmetries} beyond the obvious translation and scaling invariances.

\textbf{Example: Scale invariance}

If $W(z, \overline{z})$ solves the equation, the rescaled function:
\[
W_\lambda(z, \overline{z}) = \lambda^{-1} W(\lambda z, \lambda\overline{z})
\]
satisfies the same equation (up to appropriate rescaling of $K$).

This reveals deep connections to \textbf{conformal field theory} and \textbf{quantum field theory} on the complex plane, where similar gauge structures appear.

\section{Conclusion: A Rigorous Synthesis}

Brsakoska's work establishes that the areolar equation of second order admits a well-defined notion of oscillation that generalizes the classical concept from real differential equations. The zeros---appearing as isolated or countable point sets---serve as the precise complex analog of oscillatory sign changes.

The fundamental insight is that coefficient signs $\alpha$ and $\beta$ do not eliminate oscillatory character; they merely redistribute the zero locus geometrically across the complex plane. In Examples 1 and 2, a sign change rotates the zero set by $90°$ without destroying oscillations.

This represents a maturation of complex analysis beyond the restrictive framework of holomorphy, revealing that differential geometry in the complex plane possesses its own internal logic. The generalized constants $C_1(z)$ and $C_2(z)$ encode a nonlocal degree of freedom absent in real differential equations---solutions depend globally on the choice of analytic functions, not merely on pointwise data.

The non-holomorphic structure forces a fundamental rigidity: the solution manifold, though infinite-dimensional, remains constrained by the incompatibility between analytic and anti-analytic dependence. Zeros cannot proliferate arbitrarily but must respect the algebraic coupling of the exponential terms.

The paper exemplifies how the passage from real to complex variables, when pursued rigorously through non-holomorphic methods, generates genuinely novel mathematical structures that recombine classical intuitions---oscillation, linearity, spectral theory---into configurations that transcend their original domains of validity.

\begin{thebibliography}{99}

\bibitem{kolosov1909}
Kolosov, G. V. (1909).
\textit{On the application of the theory of complex variables to a plane problem of mathematical elasticity}.

\bibitem{polozii1965}
Polozhii, G. N. (1965).
\textit{Generalizations of the theory of analytic functions of a complex variable}.
University of Kiev Publishing House.

\bibitem{bank1982}
Bank, S. B., Laine, I. L. (1982).
On the oscillation theory of $f'' + A(z)f = 0$, where $A(z)$ is entire.
\textit{Transactions of the American Mathematical Society}, 273(1), 351--363.

\bibitem{ilievski1992}
Ilievski, B. (1992).
\textit{Linear areolar equations (Contour integration. Special functions of two complex variables. Areolar Laplace transformations)}.
Doctoral dissertation, Skopje.

\bibitem{dimitrovski1997}
Dimitrovski, D., Ilievski, B., Brsakoska, S., et al. (1997).
\textit{Vekua equation with analytic coefficients}.
Special editions of the Institute of Mathematics, Faculty of Natural Sciences and Mathematics, University ``Sv. Kiril and Metodij''---Skopje.

\bibitem{filomat1997}
Dimitrovski, D., Rajovic, M., Stoiljkovic, R. (1997).
The generalization of the I. N. Vekua equation with analytic coefficients.
\textit{Filomat}, 11, 29--32.

\bibitem{bstiint1996}
Rajovic, M., Dimitrovski, D., Stojiljkovic, R. (1996).
Elemental solution of Vekua equation with analytic coefficients.
\textit{Bulletin of Science, University Politehnica Timisoara, Series Mathematics and Physics}, 41(55)(1), 14--21.

\bibitem{brsakoska2016}
Brsakoska, S. (2016).
About the zeros and the oscillatory character of the solution of one areolar equation of second order.
\textit{Mathematical Bulletin}, 40(LXVI)(1), 55--62.

\end{thebibliography}

\end{document}
