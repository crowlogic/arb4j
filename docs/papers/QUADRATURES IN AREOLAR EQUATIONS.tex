\documentclass[12pt]{article}
\usepackage{amsmath, amsthm, amssymb}
\usepackage[utf8]{inputenc}
\usepackage{enumitem}
\usepackage{array}
\usepackage{fancyhdr}
\usepackage[hidelinks]{hyperref}

% Theorem environments
\newtheorem{theorem}{Theorem}
\newtheorem{definition}{Definition}

% Custom enumeration for degree symbols
\renewcommand{\theenumi}{\arabic{enumi}\textdegree}
\renewcommand{\labelenumi}{\theenumi}

% Page style for header information
\fancypagestyle{firstpage}{
  \fancyhf{}
  \fancyhead[C]{\small Zbornik radova Filozofskog fakulteta u Nišu\\Serija Matematika 6(1992), 59--63\\FILOMAT-20, Niš, September 26--28, 1991}
  \renewcommand{\headrulewidth}{0pt}
}

\title{\textbf{QUADRATURES IN AREOLAR EQUATIONS}}
\author{DRAGAN DIMITROVSKI\thanks{Mathematički institut, PMF, p. f. 162, 91 000 Skopje, Yugoslavia} \and BORKO ILIJEVSKI\footnotemark[1]}
\date{}

\begin{document}

\thispagestyle{firstpage}
\maketitle

\begin{abstract}
There are many information about the quadratures for the generalizations of the areolar equation \eqref{eq:areolar-general} and the Vekua equation \eqref{eq:vekua-linear}. We give a new characterisation by quadratures of the equation \eqref{eq:conjugated-general} with a new operation \(I'\). In this paper we give a preliminary result for the quadrature treatment of the equation \eqref{eq:conjugated-simplest}.
\end{abstract}

\section{Introduction}

If \(W(z,\bar{z})\) is a differentiable function of two complex variables \(z\) and \(\bar{z}\), then areolar equation of order one is an equation of the form
\begin{equation}\label{eq:areolar-general}
W_{\bar{z}} = f(z,W),
\end{equation}
where \(f\) is an analytic function of its arguments. It is known that there is an analogy between the equations \eqref{eq:areolar-general} and the ordinary differential equations \(y' = f(x,y)\) especially in the quadratures, where in \eqref{eq:areolar-general} the role of the integral constant plays an arbitrary analytic (in the sense of Cauchy-Riemann conditions) function \(\Phi(z)\) \cite{vekua1958,polozhii1965}.

Also, it is known that distinctions in the quadratures appear if in \eqref{eq:areolar-general}, in \(f\), conjugate of the unknown function \(W\) appears. Our aim will be to find quadratures solutions of the equation of the form
\begin{equation}\label{eq:conjugated-general}
W_{\bar{z}} = F(z,\bar{z},W,\bar{W}),
\end{equation}
where \(F\) is also an analytic function of its arguments (in the sense that it has convergent power series in the all arguments).

Specially, the simplest linear equation
\begin{equation}\label{eq:vekua-linear}
W_{\bar{z}} = A(z,\bar{z})W + B(z,\bar{z})\bar{W}
\end{equation}
in \(W,\bar{W}\), is called I.N. Vekua equation and it is a topic in the monography \cite{vekua1958}.

However, there is not quadratures information for it but only existential theorems. The general result in the quadrature sense has been also given by I.N. Vekua in \cite{vekua1952}. It applies to the Vekua equation
\begin{equation}\label{eq:vekua-analytic-coeffs}
W_{\bar{z}} = A(z)W + F(z)
\end{equation}
with coefficients \(A\) and \(F\) analytic in \(z\). Its solution is
\begin{equation}\label{eq:vekua-solution}
W = \Phi(z)e^{\omega(z)} + e^{\omega(z)}\int e^{-\omega(t)}\Phi(t)F_2(z,t)\,dt,
\end{equation}
where
\begin{equation}\label{eq:omega-def}
\omega(z) = \int_{\gamma_2(z)} \frac{A(t)}{t-z}\,dt.
\end{equation}
Here, \(T(z,t)\), \(K_1(z,t)\) are elements, which we obtain from the approximations
\begin{equation}\label{eq:kernel-series1}
F_2(z,t) = \sum_{j=1}^{\infty} K_{2j}(z,t), \quad K_1(z,t) = \sum_{j=1}^{\infty} K_{2j+1}(z,t),
\end{equation}
and \(K_n\) are kernels defined by
\begin{equation}\label{eq:kernel-base}
K_1(z,t) = \frac{A(t)}{\pi(t-z)},
\end{equation}
\begin{equation}\label{eq:kernel-recursive}
K_n(z,t) = \int_D K_1(z,\xi)K_{n-1}(\xi,t)\,d\xi\quad n=2,3,\ldots
\end{equation}
This means that the equation \eqref{eq:vekua-analytic-coeffs} is solved by iterations, but we do not obtain a general solution so we can not see the dependence of the integral ``constant'' \(\Phi(z)\). But it is known that the approximation iterations in all cases can not replace the general solution (general physical laws, many boundary problems etc.). So our goal will be to solve equation \eqref{eq:conjugated-general} in the quadrature form, with explicitly given dependence on the integral ``constant''.

B. Iievski \cite{ilievski1990} solved the primary and simplest conjugated Vekua equation
\begin{equation}\label{eq:conjugated-simplest}
W_{\bar{z}} = \bar{W}
\end{equation}
with the ordinary method by the areolar power series
\begin{equation}\label{eq:power-series}
W(z,\bar{z}) = \sum_{i,k=0}^{\infty} C_{ik}z^i \bar{z}^k
\end{equation}
and got solution in the form of generalized exponential power series
\begin{equation}\label{eq:generalized-exponential}
W(z,\bar{z}) = \sum_{k=0}^{\infty} \left[ C_{k0}\bar{z}^k \sum_{n=0}^{\infty} \frac{(z\bar{z})^n}{(n!)^2(n+1)\cdots(n+k)} + C_{0k}\bar{z}^{k+1} \sum_{n=1}^{\infty} \frac{(z\bar{z})^n}{(n!)^2(n+1)\cdots(n+k)(n+k+1)} \right],
\end{equation}
where \(C_{k0}, C_{0k}\) are arbitrary coefficients such that the power series \(\Phi(z) = \sum_{k=0}^{\infty} C_{k0}z^k\) converge. This means that arbitrary integral ``constant'' \(\Phi(z)\) and \(\bar{\Phi}(z)\) appear implicitly in the solution \eqref{eq:generalized-exponential}. Our aim will be to get explicit dependence.

\section{Main Result}

\begin{theorem}\label{thm:main}
Let \(\Phi(z)\) be a function analytic in \(z\). Then the general solution of \eqref{eq:conjugated-simplest} is given by the quadratures
\begin{equation}\label{eq:quadrature-solution}
W(z,\bar{z}) = \Phi(z) + \bar{z}\int \Phi(z)\,dz + \bar{z}^2\int \frac{\Phi(z)}{z}\,dz + \bar{z}^3\int \frac{\Phi(z)}{z^2}\,dz + \cdots
\end{equation}
\end{theorem}

\begin{proof}
It follows from \eqref{eq:generalized-exponential} forming triangular schema of the coefficients \(C_{k0}\) and \(C_{0k}\) and its ordering. If \(\Phi(z)\) is analytic, it is easy to show that
\begin{equation}\label{eq:remainder-limit}
\lim_{n\to\infty} R_n = \lim_{n\to\infty} \frac{\bar{z}^{n+1}}{n!}\int \frac{\Phi(z)}{(dz)^n} = 0.
\end{equation}
We get kind of a general exponential function
\begin{equation}\label{eq:general-exp-form}
W(z,\bar{z}) = \Phi(z) + \sum_{n=1}^{\infty} \bar{z}^n\int \frac{\Phi(z)}{(dz)^n}.
\end{equation}
\end{proof}

\begin{definition}\label{def:conjugate-exp}
We introduce a general conjugate exponential function \(W = \operatorname{cap} z\) as a solution of \eqref{eq:general-exp-form} or \eqref{eq:quadrature-solution} of the equation \eqref{eq:conjugated-simplest}
\begin{equation}\label{eq:conjugate-exp-def}
W = I'W(z,\bar{z}) = \operatorname{cap} z = e^{\bar{z}\int dz}.
\end{equation}
It is a source of many known but also some new special functions.
\end{definition}

\section{Special Cases}

\begin{enumerate}
\item If we choose arbitrary integral ``constant'' \(\Phi(z) = e^z\), we get for the particular solution the ordinary exponential function \(W = e^z\).

\item If \(\Phi(z) = 0\), we get a combination of ordinary and conjugate exponent and polynomial:
\begin{equation}\label{eq:poly-combination}
W(z,\bar{z}) = \sum_{k=0}^{m} P_k\bar{z}^k + \sum_{k=0}^{n} P_{k+1}z^k,
\end{equation}
where \(P_m = \text{const}\), \(P_{k+1} = \text{const}\) are allowed.

\item If we choose \(\Phi(z) = 1\), we get a special complex function
\begin{equation}\label{eq:special-function-phi1}
W(z,\bar{z}) = S(z\bar{z}) + G(z,\bar{z}),
\end{equation}
where, if \(z = \bar{z}\), then \(S'' + S' - 4zS = 0\) (Bessel equation) and \(S\) is summable
\begin{equation}\label{eq:bessel-integral}
S = \int_0^{\pi/2} \cosh(2\alpha\cos t) I_0(c\sin^2 t)\,dt = J_0(2ic) - \text{cylindrical function},
\end{equation}
and
\begin{equation}\label{eq:g-function}
G(z,\bar{z}) = \int_0^{\pi/2} \frac{2uv}{2\sqrt{zauv-1}}\,du \cdot 2uz\sqrt{1-u^2},
\end{equation}
and many other special functions.

\item In the same way using \eqref{eq:areolar-general} could be solved conjugate equations
\begin{align}
W_{\bar{z}} &= \bar{W}^2, \label{eq:conjugate-w2}\\
W_{\bar{z}} &= W\bar{W}, \label{eq:conjugate-ww}\\
W_{\bar{z}} &= \bar{W}^n, \label{eq:conjugate-wn}\\
W_{\bar{z}} &= W\bar{W}^n, \label{eq:conjugate-wwn}
\end{align}
and linear equations
\begin{align}
W_{\bar{z}} &= \bar{W} + f(z), \label{eq:linear-fz}\\
W_{\bar{z}} &= \bar{W} + f(z,\bar{z}), \label{eq:linear-fzz}\\
W_{\bar{z}} &= \bar{W} + f(z)\bar{W}, \label{eq:linear-fzw}
\end{align}
and equations which can be transformed by them. For example
\begin{align}
W_{\bar{z}} &= \bar{W}^2 + F(z), \label{eq:transformed-w2}\\
W_{\bar{z}} &= 2W\bar{W} + T(z), \label{eq:transformed-2ww}\\
W_{\bar{z}} &= W(nW) + T(z), \label{eq:transformed-nw}
\end{align}
etc. (we use known theorems formula for the nonconjugated areolar equation \(W_{\bar{z}} = A(z)W\)).

\item It is possible to construct a trigonometry of the functions.
\end{enumerate}

\begin{thebibliography}{9}
\bibitem{vekua1958} I.N. Vekua, \textit{Generalized analytic functions}, Monograph, Moscow, Fizmatgiz, 1958.
\bibitem{polozhii1965} G.N. Polozhii, \textit{Generalized theory of analytic functions of complex variable, p-analytic and (p,q)-analytic functions}, Kiev, 1965.
\bibitem{vekua1952} I.N. Vekua, \textit{Systems of differential equations of first order of elliptic type and boundary value problems with applications in mechanics}, Mathematical collection 31(73), 2 Moscow, 1952, p. 217--314.
\bibitem{ilievski1990} B.S. Ilevski, \textit{Nekoi analitički rešenija na edna klasa ravenki Vekua}, Matematički bilten, kn. 14(XL), Skopje, 1990, str. 79--86.
\bibitem{canak1983} M. Canak, \textit{Systeme von differentialgleichungen erster ordnung vom elliptischen typus mit analytischen koeffizienten und methode der verallgemeinerten areolaren reihen}, Publ. L'Inst. Math., 33(47), (1983), 35--39.
\end{thebibliography}

\end{document}
