\documentclass{article}
\usepackage[english]{babel}
\usepackage{amssymb,mathrsfs}

%%%%%%%%%% Start TeXmacs macros
\newcommand{\tmaffiliation}[1]{\\ #1}
\newtheorem{definition}{Definition}
%%%%%%%%%% End TeXmacs macros

\begin{document}

\title{Evolutionary Spectra and Non-Stationary Processes}

\author{
  M.B. Priestley
  \tmaffiliation{February 3rd, 1965}
}

\maketitle

\begin{abstract}
  We develop an approach to the spectral analysis of non-stationary processes
  which is based on the concept of ``evolutionary spectra''; that is, spectral
  functions which are time dependent, and have a physical interpretation as
  local energy distributions over frequency. It is shown that the notion of
  evolutionary spectra generalizes the usual definition of spectra for
  stationary processes, and that, under certain conditions, the evolutionary
  spectrum at each instant of time may be estimated from a single realization
  of a process. By such means it is possible to study processes with
  continuously changing ``spectral patterns''.
\end{abstract}

\

\

{\tableofcontents}

\section*{1. Introduction}

In the classical approach to statistical spectral analysis it is always
assumed that the process under study, $X_t$, is stationary, at least up to the
second order. That is, we assume that $E (X_t) = \mu$, a constant (independent
of $t$) which we may take to be zero, and that, for each $s$ and $t$, the
covariance
\begin{equation}
  \label{eq:covariance} R_{s, t} = E \{ (X_s - \mu)  (X_t - \mu)^{\ast} \}
\end{equation}
(* denoting the complex conjugate) is a function of $|s - t|$ only. In this
case it is well known that $R_{s, t}$ has a spectral representation of the
form
\begin{equation}
  \label{eq:spectral_rep} R_{s, t} = \int e^{i \omega (s - t)} dF (\omega)
\end{equation}
where $F (\omega)$ is some function having the properties of a distribution
function, and the range of integration is $(- \infty, \infty)$ for a
continuous parameter process, and $(- \pi, \pi)$ in the discrete case.

Corresponding to \eqref{eq:spectral_rep}, $\{X_t \}$ has a spectral
representation of the form
\begin{equation}
  \label{eq:process_rep} X_t = \int e^{i \omega t} dZ (\omega)
\end{equation}
where $Z (\omega)$ is an orthogonal process with $E \{|dZ (\omega) |^2 \} = dF
(\omega)$. When $\{X_t \}$ represents some physical process, the spectral
density function $f (\omega) = F' (\omega)$ (when it exists) describes the
distribution (over the frequency range) of the energy (per unit time)
dissipated by the process, and given a sample record of $\{X_t \}$, there are
several methods of estimating $f (\omega)$ (see, e.g.,
{\cite{grenander1957}}).

In practice, however, it often happens that the assumption of stationarity is
a very doubtful one. For example, records of atmospheric turbulence exhibit
marked changes over periods of time, and in such cases classical spectral
analysis based on a stationary model can hardly be carried through with
conviction. The question arises, therefore, as to whether it might be possible
to formulate a spectral theory for non-stationary processes within the
framework of classical concepts such as ``energy'' and ``frequency'', so that
a spectral function (however defined) would still possess a meaningful and
useful physical interpretation. Intuitively it seems obvious that if no
restrictions (other than finite first and second moments) are placed on the
class of non-stationary processes considered, no useful inferences may be
drawn from a single sample record. On the other hand, if one considers a
non-stationary process of the form
\begin{equation}
  \label{eq:piecewise_stat} X_t = \left\{\begin{array}{ll}
    X_t^{(1)} & (t \leqslant t_0)\\
    X_t^{(2)} & (t > t_0)
  \end{array}\right.
\end{equation}
where both $\{X_t^{(1)} \}$ and $\{X_t^{(2)} \}$ are stationary but with
different autocovariance functions, then it is clear that given a sample
record, say from $t = t_0 - T$ to $t = t_0 + T$, it is certainly possible to
infer ``something'' about the spectral content of $X_t$. If, in the above
example, $t_0$ were known, one would presumably estimate two spectral density
functions, one for $\{X_t^{(1)} \}$ and one for $\{X_t^{(2)} \}$. If now we
try to generalize this approach, we are led to the notion of a continuously
changing spectrum, or more precisely, a time-dependent spectrum. Clearly, in
such a case, we could never hope to estimate the spectrum at a particular
instant of time, but if we assume that the spectrum is changing slowly over
time, then by using estimates which involve only local functions of $\{X_t
\}$, we may attempt to estimate some form of ``average'' spectrum of $X_t$ in
the neighbourhood of any particular time-instant. We therefore consider a
class of processes whose non-stationary characteristics are changing slowly
over time, and in this respect our approach is conceived in the same spirit as
Jowett's study of ``smoothly heteromorphic'' processes {\cite{jowett1957}}.

\section*{2. Non-stationary Processes}

There have been several attempts to define a spectrum for a non-stationary
process, but in each case the object was to obtain a single function whose
properties depended on the behaviour of the process over the whole parameter
space. Cram{\'e}r {\cite{cramer1960}} considered the class of processes which
are harmonizable (in the Lo{\`e}ve sense), that is, have a representation of
the form \eqref{eq:process_rep} but without the restriction that $Z (\omega)$
must be orthogonal, and he defined the integrated spectrum (now a function of
two variables) by
\begin{equation}
  \label{eq:integrated_spectrum} dF (\omega, \nu) = E \{|dZ (\omega) dZ^{\ast}
  (\nu) |\}
\end{equation}
On the other hand, Hatanaka and Suzuki (unpublished) define the spectrum (or
more precisely, spectral density function) of any non-stationary process as
the limit of the expected value of the periodogram as the sample size tends to
infinity. In our approach, however, we define a spectral quantity whose
physical interpretation is similar to that of the spectrum of a stationary
process. A somewhat related idea was developed by Page {\cite{page1952}} who
introduced the idea of ``instantaneous power spectra''. In effect Page defines
the spectrum in the same way as Hatanaka and Suzuki, i.e. as
\begin{equation}
  \label{eq:page_spectrum1} f^{\ast} (\omega) = \lim_{T \rightarrow \infty}
  f_T^{\ast} (\omega),
\end{equation}
where
\begin{equation}
  \label{eq:page_spectrum2} f_T^{\ast} (\omega) = E \left| \int_0^T X_t e^{- i
  \omega t} dt \right|^2
\end{equation}
and then defines the instantaneous power spectrum $\rho_t (\omega)$ by
writing, for each $\omega$,
\begin{equation}
  \label{eq:page_spectrum3} f_T^{\ast} (\omega) = \int_0^T \rho_t (\omega) dt
\end{equation}
so that
\begin{equation}
  \label{eq:page_spectrum4} \rho_t (\omega) = \frac{d}{dt} \{f_t^{\ast}
  (\omega)\}
\end{equation}
and
\begin{equation}
  \label{eq:page_spectrum5} f^{\ast} (\omega) = \int_0^{\infty} \rho_t
  (\omega) dt
\end{equation}
Thus, the instantaneous power spectrum, $\rho_t (\omega)$, represents the
difference between the spectral content of the process over the interval $(0,
t + \delta t)$ and the interval $(0, t)$. This is in contrast with the
approach developed below, whose object (roughly speaking) is to study the
spectral content of the process within the interval $(t, t + \delta t)$. We
feel that this latter quantity is the more relevant one as far as physical
interpretation is concerned.

\section*{3. Spectral Theory for a Class of Non-stationary Processes:
Oscillatory Processes}

Consider a continuous parameter (complex-valued) stochastic process $\{X_t
\}$, $- \infty < t < \infty$. (Most of the following discussion will, with the
usual modifications, apply equally well to discrete parameter processes.) We
assume that the process is ``trend free'', that is, we may write $E (X_t) =
0$, all $t$, and define the autocovariance function by
\begin{equation}
  \label{eq:autocovariance} R_{s, t} = E (X_s X_t^{\ast})
\end{equation}
(there seems little point in discussing processes for which $R_{s, t}$ is a
function of $|t - s|$ only, but $E (X_t)$ varies with $t$, i.e. the
non-stationary character is confined to the mean, since in this case the
process may be studied by a combination of regression analysis and classical
spectral analysis (cf. {\cite{grenander1957}}, Ch. 7).

We now restrict attention to the class of process for which there exists a
family $\mathscr{F}$ of functions $\{\phi_t (\omega)\}$ defined on the real
line, and indexed by the suffix $t$, and a measure $\mu (\omega)$ on the real
line, such that for each $s, t$, the covariance function $R_{s, t}$ admits a
representation of the form
\begin{equation}
  \label{eq:covar_representation} R_{s, t} = \int_{- \infty}^{\infty} \phi_s
  (\omega) \phi_t^{\ast} (\omega) d \mu (\omega)
\end{equation}
When the parameter space is limited to a finite interval, say $0 \leqslant t
\leqslant T$, it is always possible to obtain a representation of the form
\eqref{eq:covar_representation} in terms of the eigenfunctions of the
covariance kernel $\{R_{s, t} \}$ (Parzen, unpublished). It should be noted
that although we have described $\mathscr{F}$ as a family of functions, each
defined on the $\omega$-axis and indexed by the parameter $t$, we may also
think of $\mathscr{F}$ as a family of functions $\{\phi_{\omega} (t)\}$, say,
each defined on the $t$-axis and indexed by the parameter $\omega$. In fact,
when we study the properties of various families (Section 7), it is convenient
to adopt the latter description.

In order for $\mathrm{var} (X_t)$ to be finite for each $t$, $\phi_t (\omega)$
must be quadratically integrable with respect to the measure $\mu$, for each
$t$. It may then be shown (see, e.g., {\cite{bartlett1955}}, p. 143;
{\cite{grenander1957}}, p. 27) that whenever $R_{s, t}$ has the representation
\eqref{eq:covar_representation}, the process $\{X_t \}$ admits a
representation of the form
\begin{equation}
  \label{eq:process_rep2} X_t = \int_{- \infty}^{\infty} \phi_t (\omega) dZ
  (\omega)
\end{equation}
where $Z (\omega)$ is an orthogonal process, with
\begin{equation}
  \label{eq:orthogonal_process} E |dZ (\omega) |^2 = d \mu (\omega)
\end{equation}
The measure $\mu (\omega)$ here plays the same role as the integrated spectrum
$F (\omega)$ does in the case of stationary processes, so that the analogous
situation to the case of an absolutely continuous spectrum is obtained by
assuming that the measure $\mu (\omega)$ is absolutely continuous with respect
to Lebesgue measure.

Parzen (unpublished) has pointed out that if there exists a representation of
$\{X_t \}$ of the form \eqref{eq:process_rep2}, then there is a multitude of
different representations of the process, each representation based on a
different family of functions. (The situation is in some ways similar to the
selection of a basis for a vector space.) When the process is stationary, one
valid choice of functions is the complex exponential family given by
\begin{equation}
  \label{eq:complex_exp} \phi_t (\omega) = e^{i \omega t}
\end{equation}
This family provides the well-known spectral decomposition (cf.
\eqref{eq:process_rep}) in terms of sine and cosine ``waves'', and forms the
basis of the physical interpretation of spectral analysis as an ``energy
distribution over frequency''. However, if the process is nonstationary this
choice of family of functions is no longer valid (since the representation
\eqref{eq:process_rep} implies that $\{X_t \}$ is stationary), and the
physical concept of ``frequency'' would appear to be no longer directly
relevant. This is hardly surprising, since the sine and cosine waves are
themselves ``stationary'' and it is natural that they should form the ``basic
elements'' used in building up models of stationary processes. If we wish to
introduce the notion of frequency in the analysis of non-stationary processes,
we are led to seeking new ``basic elements'' which, although
``non-stationary'', have an oscillatory form, and in which the notion of
``frequency'' is still dominant. One class of basic elements (or more
precisely, family of functions) which possess the required structure may be
obtained as follows. Suppose that, for each fixed $\omega$, $\phi_t (\omega)$
(considered as a function of $t$) possesses a (generalized) Fourier transform
whose modulus has an absolute maximum at frequency $\theta (\omega)$, say.
Then we may regard $\phi_t (\omega)$ as an amplitude modulated sine wave with
frequency $\theta (\omega)$, and write $\phi_t (\omega)$ in the form
\begin{equation}
  \label{eq:amplitude_modulated} \phi_t (\omega) = A_t (\omega) e^{i \theta
  (\omega) t}
\end{equation}
where the modulating function $A_t (\omega)$ is such that the modulus of its
(generalized) Fourier transform has an absolute maximum at the origin (i.e.
zero frequency). We now formalize this approach in the following definition.

\begin{definition}
  \label{def:oscillatory_function}The function of $t$, $\phi_t (\omega)$, will
  be said to be an oscillatory function if, for some (necessarily unique)
  $\theta (\omega)$ it may be written in the form
  \eqref{eq:amplitude_modulated}, where $A_t (\omega)$ is of the form
  \begin{equation}
    \label{eq:modulating_function} A_t (\omega) = \int_{- \infty}^{\infty}
    e^{it \theta} dH_{\omega} (\theta)
  \end{equation}
  with $|dH_{\omega} (\theta) |$ having an absolute maximum at $\theta = 0$.
  (The function $A_t (\omega)$ may be regarded as the ``envelope'' of $\phi_t
  (\omega)$.) If, further, the family $\{\phi_t (\omega)\}$ is such that
  $\theta (\omega)$ is a single-valued function of $\omega$ (i.e. if no two
  distinct members of the family have Fourier transforms whose maxima occur at
  the same point), then we may transform the variable in the integral in
  \eqref{eq:covar_representation} from $\omega$ to $\theta (\omega)$, and by
  suitably redefining $A_t (\omega)$ and the measure $\mu (\omega)$, write
  \begin{equation}
    \label{eq:covar_representation2} R_{s, t} = \int_{- \infty}^{\infty} A_s
    (\omega) A_t^{\ast} (\omega) e^{i \omega (s - t)} d \mu (\omega)
  \end{equation}
  and correspondingly
  \begin{equation}
    \label{eq:process_rep3} X_t = \int_{- \infty}^{\infty} A_t (\omega) e^{i
    \omega t} dZ (\omega)
  \end{equation}
  where
  \begin{equation}
    E |dZ (\omega) |^2 = d \mu (\omega)
  \end{equation}
\end{definition}

\begin{definition}
  \label{def:oscillatory_process}If there exists a family of oscillatory
  functions $\{\phi_t (\omega)\}$, in terms of which the process $\{X_t \}$
  has a representation of the form \eqref{eq:covar_representation}, $\{X_t \}$
  will be termed an ``oscillatory process''.
\end{definition}

It follows that any oscillatory process also has a representation of the form
\eqref{eq:process_rep3}, where the family $A_t (\omega)$ satisfies the
condition of definition \eqref{def:oscillatory_function}, and that, without
loss of generality, we may write any family of oscillatory functions in the
form
\begin{equation}
  \label{eq:oscillatory_functions} \phi_t (\omega) = A_t (\omega) e^{i \omega
  t}
\end{equation}
We may note that, since \eqref{eq:complex_exp} is a particular case of
\eqref{eq:amplitude_modulated} (with $A_t (\omega) \equiv 1$, all $t, \omega$,
and $\theta (\omega) \equiv \omega$), the class of oscillatory processes
certainly includes all second-order stationary processes.

\section*{4. Evolutionary (Power) Spectra}

Consider an oscillatory process of the form \eqref{eq:process_rep3}, with
autocovariance function, $R_{s, t}$, of the form
\eqref{eq:covar_representation2}. For any particular process $\{X_t \}$ there
will, in general, be a large number of different families of oscillatory
functions in terms of each of which $\{X_t \}$ has a representation of the
form \eqref{eq:process_rep3}, with each family inducing a different measure
$\mu (\omega)$. For a particular family, $\mathscr{F}$, of spectral functions
$\{\phi_t (\omega)\}$, it is tempting to define the spectrum of $\{X_t \}$
(with respect to $\mathscr{F}$) simply as the measure $\mu (\omega)$. However,
such a definition would not have the interpretation of an ``energy
distribution over frequency''. For, from \eqref{eq:covar_representation2}, we
may write
\begin{equation}
  \label{eq:total_energy} \mathrm{var} X_t \equiv R_{t, t} = \int_{-
  \infty}^{\infty} |A_t (\omega) |^2 d \mu (\omega)
\end{equation}
Since var $X_t$ may be interpreted as a measure of the ``total energy'' of the
process at time $t$, \eqref{eq:total_energy} gives a decomposition of total
energy in which the contribution from ``frequency'' $\omega$ is $\{|A_t
(\omega) |^2 d \mu (\omega)\}$. This result is consistent with the
interpretation of equation \eqref{eq:process_rep3} as an expression for $X_t$
as the limiting form of a ``sum'' of sine waves with different frequencies and
time-varying random amplitudes $\{A_t (\omega) dZ (\omega)\}$. We are thus led
to the following definition.

\begin{definition}
  \label{def:evol_spectrum}Let $\mathscr{F}$ denote a particular family of
  oscillatory functions, $\{\phi_t (\omega)\} \equiv \{A_t (\omega) e^{i
  \omega t} \}$, and let $\{X_t \}$ be an oscillatory process having a
  representation of the form \eqref{eq:process_rep3} in terms of the family
  $\mathscr{F}$. We define the evolutionary power spectrum at time $t$ with
  respect to the family $\mathscr{F}$, $dF_t (\omega)$, by
  \begin{equation}
    \label{eq:evol_spectrum} dF_t (\omega) = |A_t (\omega) |^2 d \mu (\omega)
  \end{equation}
\end{definition}

Note that when $\{X_t \}$ is stationary, and $\mathscr{F}$ is chosen to be the
family of complex exponentials, $dF_t (\omega)$ reduces to the standard
definition of the (integrated) spectrum. The evolutionary spectrum has the
same physical interpretation as the spectrum of a stationary process, namely,
that it describes a distribution of energy over frequency, but whereas the
latter is determined by the behaviour of the process over all time, the former
represents specifically the spectral content of the process in the
neighbourhood of the time instant $t$.

Although, according to definition \eqref{def:evol_spectrum}, the evolutionary
spectrum, $dF_t (\omega)$, depends on the choice of family $\mathscr{F}$, it
follows from equation \eqref{eq:total_energy} that
\begin{equation}
  \label{eq:integral_spectrum} \mathrm{var} (X_t) = \int_{- \infty}^{\infty}
  dF_t (\omega)
\end{equation}
so that the value of the integral of $dF_t (\omega)$ is independent of the
particular family $\mathscr{F}$, and, for all families, represents the total
energy of the process at time $t$.

It is now convenient to ``standardize'' the functions $A_t (\omega)$ so that,
for all $\omega$,
\begin{equation}
  \label{eq:standardize} A_0 (\omega) = 1
\end{equation}
i.e. we incorporate $|A_0 (\omega) |$ in the measure $\mu (\omega)$. With this
convention, $d \mu (\omega)$ represents the evolutionary spectrum at $t = 0$,
and $|A_t (\omega) |^2$ represents the change in the spectrum, relative to
zero time. We now have, for each $\omega$,
\begin{equation}
  \label{eq:unit_integral} \int_{- \infty}^{\infty} dH_{\omega} (\theta) = 1
\end{equation}
so that the Fourier transforms of the $\{A_t (\omega)\}$ are normalized to
have unit integrals.

There is an interesting alternative interpretation of oscillatory processes in
terms of time-varying filters. Let $\{X_t \}$ be of the form
\eqref{eq:process_rep3} and suppose that for each fixed $t$ we may write
(formally)
\begin{equation}
  \label{eq:filter_fourier} A_t (\omega) = \int_{- \infty}^{\infty} e^{i
  \omega u} h_t (u) du
\end{equation}
Then from \eqref{eq:process_rep3}
\begin{equation}
  \label{eq:filter_representation} X_t = \int_{- \infty}^{\infty} S_{t - u}
  h_t (u) du
\end{equation}
where
\begin{equation}
  \label{eq:stationary_process} S_t = \int_{- \infty}^{\infty} e^{i \omega t}
  dZ (\omega)
\end{equation}
is a stationary process with spectrum $d \mu (\omega)$. Thus $X_t$ may be
interpreted as the result of passing a stationary process through a
time-varying filter $\{h_t (u)\}$. Conversely, any process of the form
\eqref{eq:filter_representation} (with $h_t (u)$ chosen so that $A_t (\omega)$
is of the required form) may be written in the form \eqref{eq:process_rep3}.
Thus the evolutionary spectrum at time $t$, $|A_t (\omega) |^2 d \mu
(\omega)$, may be interpreted as the spectrum (in the classical sense) of the
stationary process which we would have obtained if the filter $\{h_t (u)\}$
was held fixed in the state which it attained at the time instant $t$.

\

\begin{thebibliography}{99}
  {\bibitem{bartlett1955}}Bartlett, M. S. (1955), An Introduction to
  Stochastic Processes with Special Reference to Methods and Applications.
  Cambridge University Press.
  
  {\bibitem{cramer1960}}Cram{\'e}r, H. (1960), ``On some classes of
  non-stationary processes'', Proc. Fourth Berkeley Symposium Math. Statist.
  and Prob., 2, 57-78. University of California Press.
  
  {\bibitem{grenander1957}}Grenander, U. and Rosenblatt, M. (1957),
  Statistical Analysis of Stationary Time Series. New York: Wiley.
  
  {\bibitem{herbst1963a}}Herbst, L. J. (1963a), ``Periodogram analysis and
  variance fluctuations'', J. R. statist. Soc. B, 25, 442-450.
  
  {\bibitem{herbst1963b}}Herbst, L. J. (1963b), ``A test for variance
  heterogeneity in the residuals of a Gaussian moving average'', J. R.
  statist. Soc. B, 25, 451-454.
  
  {\bibitem{herbst1963c}}Herbst, L. J. (1963c), ``Almost periodic variances'',
  Ann. math. Statist., 34, 1549-1557.
  
  {\bibitem{jowett1957}}Jowett, G. H. (1957), ``Statistical analysis using
  local properties of smoothly heteromorphic stochastic series'', Biometrika,
  44, 454-463.
  
  {\bibitem{lomnicki1957}}Lomnicki, Z. A. and Zaremba, S. K. (1957), ``On
  estimating the spectral density function of a stochastic process'', J. R.
  statist. Soc. B, 19, 13-37.
  
  {\bibitem{page1952}}Page, C. H. (1952), ``Instantaneous power spectra'', J.
  appl. Phys., 23, 103-106.
  
  {\bibitem{priestley1962}}Priestley, M. B. (1962), ``Basic consideration in
  the estimation of spectra'', Technometrics, 4, 551-563.
\end{thebibliography}

\end{document}
