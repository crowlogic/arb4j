\documentclass[12pt]{article}
\usepackage{amsmath}
\usepackage{amssymb}
\usepackage{amsthm}
\usepackage{enumitem}
\usepackage{geometry}
\usepackage{fancyhdr}
\usepackage{graphicx}

\geometry{margin=1in}
\pagestyle{fancy}
\fancyhf{}
\fancyhead[C]{Evolutionary Spectra and Non-Stationary Processes}
\fancyfoot[C]{\thepage}

\newtheorem{theorem}{Theorem}[section]
\newtheorem{lemma}[theorem]{Lemma}
\newtheorem{definition}[theorem]{Definition}
\newtheorem{corollary}[theorem]{Corollary}

\numberwithin{equation}{section}

\begin{document}

\title{Evolutionary Spectra and Non-Stationary Processes}
\author{M. B. Priestley}
\date{University of Manchester}
\maketitle

\begin{abstract}
We develop an approach to the spectral analysis of non-stationary processes which is based on the concept of "evolutionary spectra"; that is, spectral functions which are time dependent, and have a physical interpretation as local energy distributions over frequency. It is shown that the notion of evolutionary spectra generalizes the usual definition of spectra for stationary processes, and that, under certain conditions, the evolutionary spectrum at each instant of time may be estimated from a single realization of a process. By such means it is possible to study processes with continuously changing "spectral patterns".
\end{abstract}

\section{Introduction}
\label{sec:introduction}

In the classical approach to statistical spectral analysis it is always assumed that the process under study, $X_t$, is stationary, at least up to the second order. That is, we assume that $E(X_t) = \mu$, a constant (independent of $t$) which we may take to be zero, and that, for each $s$ and $t$, the covariance

\begin{equation}
\label{eq:covariance_def}
R_{s,t} = E\{(X_s - \mu)(X_t - \mu)^*\}
\end{equation}

(* denoting the complex conjugate) is a function of $|s - t|$ only. In this case it is well known that $R_{s,t}$ has a spectral representation of the form

\begin{equation}
\label{eq:spectral_representation}
R_{s,t} = \int e^{i\omega(s-t)} dF(\omega)
\end{equation}

where $F(\omega)$ is some function having the properties of a distribution function, and the range of integration is $(-\infty, \infty)$ for a continuous parameter process, and $(-\pi, \pi)$ in the discrete case.

Corresponding to \eqref{eq:spectral_representation}, $\{X_t\}$ has a spectral representation of the form

\begin{equation}
\label{eq:process_spectral_representation}
X_t = \int e^{i\omega t} dZ(\omega)
\end{equation}

where $Z(\omega)$ is an orthogonal process with $E\{|dZ(\omega)|^2\} = dF(\omega)$. When $\{X_t\}$ represents some physical process, the spectral density function $f(\omega) = F'(\omega)$ (when it exists) describes the distribution (over the frequency range) of the energy (per unit time) dissipated by the process, and given a sample record of $\{X_t\}$, there are several methods of estimating $f(\omega)$ (see, e.g., Grenander and Rosenblatt, 1957, Ch. 4).

In practice, however, it often happens that the assumption of stationarity is a very doubtful one. For example, records of atmospheric turbulence exhibit marked changes over periods of time, and in such cases classical spectral analysis based on a stationary model can hardly be carried through with conviction. The question arises, therefore, as to whether it might be possible to formulate a spectral theory for non-stationary processes within the framework of classical concepts such as "energy" and "frequency", so that a spectral function (however defined) would still possess a meaningful and useful physical interpretation.

Intuitively it seems obvious that if no restrictions (other than finite first and second moments) are placed on the class of non-stationary processes considered, no useful inferences may be drawn from a single sample record. On the other hand, if one considers a non-stationary process of the form

\begin{equation}
\label{eq:piecewise_stationary}
X_t = \begin{cases}
X_t^{(1)} & (t \leq t_0) \\
X_t^{(2)} & (t > t_0)
\end{cases}
\end{equation}

where both $\{X_t^{(1)}\}$ and $\{X_t^{(2)}\}$ are stationary but with different autocovariance functions, then it is clear that given a sample record, say from $t = t_0 - T$ to $t = t_0 + T$, it is certainly possible to infer "something" about the spectral content of $X_t$. If, in the above example, $t_0$ were known, one would presumably estimate two spectral density functions, one for $\{X_t^{(1)}\}$ and one for $\{X_t^{(2)}\}$. If now we try to generalize this approach, we are led to the notion of a continuously changing spectrum, or more precisely, a time-dependent spectrum.

Clearly, in such a case, we could never hope to estimate the spectrum at a particular instant of time, but if we assume that the spectrum is changing slowly over time, then by using estimates which involve only local functions of $\{X_t\}$, we may attempt to estimate some form of "average" spectrum of $X_t$ in the neighbourhood of any particular time-instant. We therefore consider a class of processes whose non-stationary characteristics are changing slowly over time, and in this respect our approach is conceived in the same spirit as Jowett's study of "smoothly heteromorphic" processes (Jowett, 1957).

\section{Non-Stationary Processes}
\label{sec:non_stationary}

There have been several attempts to define a spectrum for a non-stationary process, but in each case the object was to obtain a single function whose properties depended on the behaviour of the process over the whole parameter space. Cramer (1960) considered the class of processes which are harmonizable (in the Loève sense), that is, have a representation of the form \eqref{eq:process_spectral_representation} but without the restriction that $Z(\omega)$ must be orthogonal, and he defined the integrated spectrum (now a function of two variables) by

\begin{equation}
\label{eq:cramer_spectrum}
dF(\omega, \nu) = E\{dZ(\omega) dZ^*(\nu)\}
\end{equation}

On the other hand, Hatanaka and Suzuki (unpublished) define the spectrum (or more precisely, spectral density function) of any non-stationary process as the limit of the expected value of the periodogram as the sample size tends to infinity. In our approach, however, we define a spectral quantity whose physical interpretation is similar to that of the spectrum of a stationary process.

A somewhat related idea was developed by Page (1952) who introduced the idea of "instantaneous power spectra". In effect Page defines the spectrum in the same way as Hatanaka and Suzuki, i.e. as

\begin{equation}
\label{eq:page_spectrum}
f^*(\omega) = \lim_{T \to \infty} f_T(\omega)
\end{equation}

where

\begin{equation}
\label{eq:periodogram_def}
f_T(\omega) = \frac{1}{2\pi T} \left|\int_0^T X_t e^{-i\omega t} dt\right|^2
\end{equation}

and then defines the instantaneous power spectrum $P_t(\omega)$ by writing, for each $\omega$,

\begin{equation}
\label{eq:instantaneous_power}
f_T(\omega) = \frac{1}{T} \int_0^T P_t(\omega) dt
\end{equation}

so that

\begin{equation}
\label{eq:pt_definition}
P_t(\omega) = \frac{d}{dt}\{tf_T(\omega)\}
\end{equation}

Thus, the instantaneous power spectrum, $P_t(\omega)$, represents the difference between the spectral content of the process over the interval $(0, t + \delta t)$ and the interval $(0, t)$. This is in contrast with the approach developed below, whose object (roughly speaking) is to study the spectral content of the process within the interval $(t, t + \delta t)$. We feel that this latter quantity is the more relevant one as far as physical interpretation is concerned.

\section{Spectral Theory for a Class of Non-Stationary Processes: Oscillatory Processes}
\label{sec:oscillatory_processes}

Consider a continuous parameter (complex-valued) stochastic process $\{X_t\}$, $-\infty < t < \infty$. (Most of the following discussion will, with the usual modifications, apply equally well to discrete parameter processes.) We assume that the process is "trend free", that is, we may write $E(X_t) = 0$, all $t$, and define the autocovariance function by

\begin{equation}
\label{eq:autocovariance_def}
R_{s,t} = E(X_s X_t^*)
\end{equation}

We now restrict attention to the class of process for which there exists a family of functions $\{\phi_t(\omega)\}$ defined on the real line, and indexed by the suffix $t$, and a measure $\mu(\omega)$ on the real line, such that for each $s, t$, the covariance function $R_{s,t}$ admits a representation of the form

\begin{equation}
\label{eq:covariance_representation}
R_{s,t} = \int_{-\infty}^{\infty} \phi_s(\omega) \phi_t^*(\omega) d\mu(\omega)
\end{equation}

When the parameter space is limited to a finite interval, say $0 < t < T$, it is always possible to obtain a representation of the form \eqref{eq:covariance_representation} in terms of the eigenfunctions of the covariance kernel $\{R_{s,t}\}$ (Parzen, unpublished). It should be noted that although we have described $\Phi$ as a family of functions, each defined on the $\omega$-axis and indexed by the parameter $t$, we may also think of $\Phi$ as a family of functions $\{\psi_\omega(t)\}$, say, each defined on the $t$-axis and indexed by the parameter $\omega$. In fact, when we study the properties of various families (Section 7), it is convenient to adopt the latter description.

In order for var$(X_t)$ to be finite for each $t$, $\phi_t(\omega)$ must be quadratically integrable with respect to the measure $\mu$ for each $t$. It may then be shown (see, e.g., Bartlett, 1955, P. 143; Grenander and Rosenblatt, 1957, P. 27) that whenever $R_{s,t}$ has the representation \eqref{eq:covariance_representation}, the process $\{X_t\}$ admits a representation of the form

\begin{equation}
\label{eq:process_representation}
X_t = \int \phi_t(\omega) dZ(\omega)
\end{equation}

where $Z(\omega)$ is an orthogonal process, with

\begin{equation}
\label{eq:orthogonal_measure}
E|dZ(\omega)|^2 = d\mu(\omega)
\end{equation}

The measure $\mu(\omega)$ here plays the same role as the integrated spectrum $F(\omega)$ does in the case of stationary processes, so that the analogous situation to the case of an absolutely continuous spectrum is obtained by assuming that the measure $\mu(\omega)$ is absolutely continuous with respect to Lebesgue measure.

Parzen (unpublished) has pointed out that if there exists a representation of $\{X_t\}$ of the form \eqref{eq:process_representation}, then there is a multitude of different representations of the process, each representation based on a different family of functions. (The situation is in some ways similar to the selection of a basis for a vector space.) When the process is stationary, one valid choice of functions is the complex exponential family given by

\begin{equation}
\label{eq:stationary_family}
\phi_t(\omega) = e^{i\omega t}
\end{equation}

This family provides the well-known spectral decomposition (cf. \eqref{eq:process_spectral_representation}) in terms of sine and cosine "waves", and forms the basis of the physical interpretation of spectral analysis as an "energy distribution over frequency". However, if the process is non-stationary this choice of family of functions is no longer valid (since the representation \eqref{eq:process_spectral_representation} implies that $\{X_t\}$ is stationary), and the physical concept of "frequency" would appear to be no longer directly relevant. This is hardly surprising, since the sine and cosine waves are themselves "stationary" and it is natural that they should form the "basic elements" used in building up models of stationary processes. If we wish to introduce the notion of frequency in the analysis of non-stationary processes, we are led to seeking new "basic elements" which, although "non-stationary", have an oscillatory form, and in which the notion of "frequency" is still dominant.

One class of basic elements (or more precisely, family of functions) which possess the required structure may be obtained as follows. Suppose that, for each fixed $\omega$, $\phi_t(\omega)$ (considered as a function of $t$) possesses a (generalized) Fourier transform whose modulus has an absolute maximum at frequency $\theta(\omega)$, say. Then we may regard $\phi_t(\omega)$ as an amplitude modulated sine wave with frequency $\theta(\omega)$, and write $\phi_t(\omega)$ in the form

\begin{equation}
\label{eq:oscillatory_form}
\phi_t(\omega) = A_t(\omega) e^{i\theta(\omega)t}
\end{equation}

where the modulating function $A_t(\omega)$ is such that the modulus of its (generalized) Fourier transform has an absolute maximum at the origin (i.e. zero frequency). We now formalize this approach in the following definition.

\begin{definition}
\label{def:oscillatory_function}
The function of $t$, $\phi_t(\omega)$, will be said to be an oscillatory function if, for some (necessarily unique) $\theta(\omega)$ it may be written in the form \eqref{eq:oscillatory_form}, where $A_t(\omega)$ is of the form

\begin{equation}
\label{eq:amplitude_form}
A_t(\omega) = \int e^{i\theta t} dH_\omega(\theta)
\end{equation}

with $|dH_\omega(\theta)|$ having an absolute maximum at $\theta = 0$. (The function $A_t(\omega)$ may be regarded as the "envelope" of $\phi_t(\omega)$.)
\end{definition}

If, further, the family $\{\phi_t(\omega)\}$ is such that $\theta(\omega)$ is a single-valued function of $\omega$ (i.e. if no two distinct members of the family have Fourier transforms whose maxima occur at the same point), then we may transform the variable in the integral in \eqref{eq:covariance_representation} from $\omega$ to $\theta(\omega)$, and by suitably redefining $A_t(\omega)$ and the measure $\mu(\omega)$, write

\begin{equation}
\label{eq:transformed_covariance}
R_{s,t} = \int_{-\infty}^{\infty} A_s(\omega) A_t^*(\omega) e^{i\omega(s-t)} d\mu(\omega)
\end{equation}

and correspondingly

\begin{equation}
\label{eq:transformed_process}
X_t = \int A_t(\omega) e^{i\omega t} dZ(\omega)
\end{equation}

where $E|dZ(\omega)|^2 = d\mu(\omega)$.

\begin{definition}
\label{def:oscillatory_process}
If there exists a family of oscillatory functions $\{\phi_t(\omega)\}$, in terms of which the process $\{X_t\}$ has a representation of the form \eqref{eq:covariance_representation}, $\{X_t\}$ will be termed an "oscillatory process".
\end{definition}

It follows that any oscillatory process also has a representation of the form \eqref{eq:transformed_process}, where the family $A_t(\omega)$ satisfies the condition of Definition~\ref{def:oscillatory_function}, and that, without loss of generality, we may write any family of oscillatory functions in the form

\begin{equation}
\label{eq:standard_oscillatory_form}
\phi_t(\omega) = A_t(\omega) e^{i\omega t}
\end{equation}

We may note that, since \eqref{eq:stationary_family} is a particular case of \eqref{eq:oscillatory_form} (with $A_t(\omega) \equiv 1$, all $t, \omega$, and $\theta(\omega) = \omega$), the class of oscillatory processes certainly includes all second-order stationary processes.

\section{Evolutionary (Power) Spectra}
\label{sec:evolutionary_spectra}

Consider an oscillatory process of the form \eqref{eq:transformed_process}, with autocovariance function, $R_{s,t}$, of the form \eqref{eq:transformed_covariance}. For any particular process $\{X_t\}$ there will, in general, be a large number of different families of oscillatory functions in terms of each of which $\{X_t\}$ has a representation of the form \eqref{eq:transformed_process}, with each family inducing a different measure $\mu(\omega)$. For a particular family, $\mathcal{F}$, of spectral functions $\{\phi_t(\omega)\}$, it is tempting to define the spectrum of $\{X_t\}$ (with respect to $\mathcal{F}$) simply as the measure $\mu(\omega)$. However, such a definition would not have the interpretation of an "energy distribution over frequency". For, from \eqref{eq:transformed_covariance}, we may write

\begin{equation}
\label{eq:variance_decomposition}
\text{var}(X_t) = R_{t,t} = \int_{-\infty}^{\infty} |A_t(\omega)|^2 d\mu(\omega)
\end{equation}

Since var$(X_t)$ may be interpreted as a measure of the "total energy" of the process at time $t$, \eqref{eq:variance_decomposition} gives a decomposition of total energy in which the contribution from "frequency" $\omega$ is $\{|A_t(\omega)|^2 d\mu(\omega)\}$. This result is consistent with the interpretation of equation \eqref{eq:transformed_process} as an expression for $X_t$ as the limiting form of a "sum" of sine waves with different frequencies and time-varying random amplitudes $\{A_t(\omega)dZ(\omega)\}$.

We are thus led to the following definition.

\begin{definition}
\label{def:evolutionary_spectrum}
Let $\mathcal{F}$ denote a particular family of oscillatory functions, let $\{X_t\}$ be an oscillatory process having a representation of the form \eqref{eq:transformed_process} in terms of the family $\mathcal{F}$. We define the evolutionary power spectrum at time $t$ with respect to the family $\mathcal{F}$, $dF_t(\omega)$, by

\begin{equation}
\label{eq:evolutionary_spectrum_def}
dF_t(\omega) = |A_t(\omega)|^2 d\mu(\omega)
\end{equation}
\end{definition}

Note that when $\{X_t\}$ is stationary, and $\mathcal{F}$ is chosen to be the family of complex exponentials, $dF_t(\omega)$ reduces to the standard definition of the (integrated) spectrum. The evolutionary spectrum has the same physical interpretation as the spectrum of a stationary process, namely, that it describes a distribution of energy over frequency, but whereas the latter is determined by the behaviour of the process over all time, the former represents specifically the spectral content of the process in the neighbourhood of the time instant $t$.

Although, according to Definition~\ref{def:evolutionary_spectrum}, the evolutionary spectrum, $dF_t(\omega)$, depends on the choice of family $\mathcal{F}$, it follows from equation \eqref{eq:variance_decomposition} that

\begin{equation}
\label{eq:total_energy_invariance}
\text{var}(X_t) = \int dF_t(\omega)
\end{equation}

so that the value of the integral of $dF_t(\omega)$ is independent of the particular family $\mathcal{F}$, and, for all families, represents the total energy of the process at time $t$.

It is now convenient to "standardize" the functions $A_t(\omega)$ so that, for all $\omega$,

\begin{equation}
\label{eq:standardization}
A_0(\omega) = 1
\end{equation}

i.e. we incorporate $|A_0(\omega)|^2$ in the measure $\mu(\omega)$. With this convention, $dF_t(\omega)$ represents the evolutionary spectrum at $t = 0$, and $|A_t(\omega)|^2$ represents the change in the spectrum, relative to zero time. We now have, for each $\omega$,

\begin{equation}
\label{eq:normalized_integral}
\int_{-\infty}^{\infty} |dH_\omega(\theta)| = 1
\end{equation}

so that the Fourier transforms of the $\{A_t(\omega)\}$ are normalized to have unit integrals.

There is an interesting alternative interpretation of oscillatory processes in terms of time-varying filters. Let $\{X_t\}$ be of the form \eqref{eq:transformed_process} and suppose that for each fixed $t$ we may write (formally)

\begin{equation}
\label{eq:filter_representation}
A_t(\omega) = \int_{-\infty}^{\infty} e^{i\omega u} h_t(u) du
\end{equation}

Then from \eqref{eq:transformed_process}

\begin{equation}
\label{eq:filtered_process}
X_t = \int_{-\infty}^{\infty} S_{t-u} h_t(u) du
\end{equation}

where

\begin{equation}
\label{eq:stationary_component}
S_t = \int e^{i\omega t} dZ(\omega)
\end{equation}

is a stationary process with spectrum $d\mu(\omega)$. Thus $X_t$ may be interpreted as the result of passing a stationary process through a time-varying filter $\{h_t(u)\}$. Conversely, any process of the form \eqref{eq:filtered_process} (with $h_t(u)$ chosen so that $A_t(\omega)$ is of the required form) may be written in the form \eqref{eq:transformed_process}. Thus the evolutionary spectrum at time $t$, $|A_t(\omega)|^2 d\mu(\omega)$, may be interpreted as the spectrum (in the classical sense) of the stationary process which we would have obtained if the filter $\{h_t(u)\}$ was held fixed in the state which it attained at the time instant $t$.

In Section~\ref{sec:estimation} we show how, for a certain class of processes, evolutionary spectra may be estimated from a sample record of $\{X_t\}$, and by examining the variations of $dF_t(\omega)$ over time we are enabled to study continuously the changing spectral pattern of the process.

\section{The Uniformly Modulated Process}
\label{sec:uniformly_modulated}

One interesting example of a non-stationary process satisfying the model \eqref{eq:transformed_process} is the following

\begin{equation}
\label{eq:uniformly_modulated_def}
X_t = c(t) X_t^{(0)}
\end{equation}

where $\{X_t^{(0)}\}$ is a stationary process with zero mean and spectrum $dF(\omega)$, and the function $c(t)$ (with $c(0) = 1$) has a (generalized) Fourier transform whose modulus has an absolute maximum at the origin. (For example, $c(t)$ may be any non-negative real-valued function whose Fourier transform exists.) Processes of the form \eqref{eq:uniformly_modulated_def} have been studied by Herbst (1963a, b, c). Since $\{X_t^{(0)}\}$ is stationary, we may write

\begin{equation}
\label{eq:stationary_representation}
X_t^{(0)} = \int e^{i\omega t} dZ(\omega)
\end{equation}

where $Z(\omega)$ is orthogonal with $E|dZ(\omega)|^2 = dF(\omega)$, so that

\begin{equation}
\label{eq:modulated_representation}
X_t = \int c(t) e^{i\omega t} dZ(\omega)
\end{equation}

We may note that since $\mathcal{F}_0 = \{c(t) e^{i\omega t}\}$ is a family of oscillatory functions, the process defined by \eqref{eq:uniformly_modulated_def} is an oscillatory process and, with respect to $\mathcal{F}_0$, has evolutionary spectrum

\begin{equation}
\label{eq:uniformly_modulated_spectrum}
dF_t(\omega) = |c(t)|^2 dF(\omega)
\end{equation}

It should be observed, however, that the process defined by \eqref{eq:uniformly_modulated_def} is a very special case of the model \eqref{eq:transformed_process}, in that all the spectral components (with respect to $\omega$) are varying over time in exactly the same way. More specifically, for any pair of frequencies $\omega_1, \omega_2$, and time instants $t_1, t_2$,

\begin{equation}
\label{eq:uniform_modulation_ratio}
\frac{dF_{t_1}(\omega_1)}{dF_{t_1}(\omega_2)} = \frac{dF_{t_2}(\omega_1)}{dF_{t_2}(\omega_2)}
\end{equation}

A process for which there exists a family $\mathcal{F}$ such that the evolutionary spectrum (with respect to $\mathcal{F}$) has the above property will be called a uniformly modulated process.

\section{Effect of Filters}
\label{sec:filters}

One of the most useful features of the spectral representation of stationary processes is that it enables the effect of linear transformations (i.e. "filters") to be described purely in terms of the effect on individual spectral components. Thus, if we consider a linear transformation of a stationary process $\{X_t\}$, of the form

\begin{equation}
\label{eq:filter_def}
Y_t = \int_{-\infty}^{\infty} g(u) X_{t-u} du
\end{equation}

then it is well known that (with an obvious notation) the spectra of $\{X_t\}$ and $\{Y_t\}$ are related by

\begin{equation}
\label{eq:filter_spectrum}
dF^{(Y)}(\omega) = |\Gamma(\omega)|^2 dF^{(X)}(\omega)
\end{equation}

where

\begin{equation}
\label{eq:transfer_function}
\Gamma(\omega) = \int_{-\infty}^{\infty} g(u) e^{-i\omega u} du
\end{equation}

is termed the transfer function of the filter $\{g(u)\}$. Hence $dF^{(Y)}(\omega_1)$, say, is determined purely by $dF^{(X)}(\omega_1)$ and $\Gamma(\omega_1)$ and is not affected by $dF^{(X)}(\omega)$ at other frequencies.

We now show that this property holds (in an approximate sense) for evolutionary spectra when we consider linear transformations of non-stationary processes.

Suppose that $\{X_t\}$ satisfies a model of the form \eqref{eq:transformed_process}, and consider a slightly more general form of the transformation \eqref{eq:filter_def}, namely

\begin{equation}
\label{eq:generalized_filter}
Y_t = \int_{-\infty}^{\infty} g(u) X_{t-u} e^{-i\omega_0(t-u)} du
\end{equation}

where $\omega_0$ is any constant frequency. Then we may write

\begin{equation}
\label{eq:filtered_representation}
Y_t = \int \Gamma_{t,\omega_0}(\omega) A_t(\omega + \omega_0) e^{i\omega t} dZ(\omega + \omega_0)
\end{equation}

where, for any $t, \lambda$,

\begin{equation}
\label{eq:generalized_transfer}
\Gamma_{t,\lambda}(\omega) = \int_{-\infty}^{\infty} g(u) \left\{\frac{A_{t-u}(\lambda)}{A_t(\lambda)}\right\} e^{-i\omega u} du
\end{equation}

The function $\Gamma_{t,\lambda}(\omega)$ will be termed the generalized transfer function of the filter $\{g(u)\}$ with respect to the family $\mathcal{F}$.

Now the representation of $\{Y_t\}$ given by \eqref{eq:filtered_representation} is not necessarily of the form \eqref{eq:transformed_process}, since the modulus of the (generalized) Fourier transform of $\{\Gamma_{t,\omega_0}(\omega) A_t(\omega + \omega_0)\}$ may not have an absolute maximum at zero frequency. If not, then the function

\begin{equation}
\label{eq:modified_family}
\psi_{t,\omega_0}(\omega) = \Gamma_{t,\omega_0}(\omega) A_t(\omega + \omega_0) e^{i\omega t}
\end{equation}

will still, in general, be oscillatory, but its "dominant" frequency will be slightly shifted from $\omega$.

There is, however, an important case where, for each $t, \lambda$, the function $\psi_{t,\lambda}(\omega)$ reduces approximately to $\Gamma(\omega)$, namely when $A_{t-u}(\lambda)$ is, for each $t, \lambda$, slowly varying compared with the function $g(u)$. Thus, we assume that $g(u)$ decays rapidly to zero as $|u| \to \infty$, and that $A_{t-u}(\lambda)$ is approximately constant over the range of $u$ for which $g(u)$ is non-negligible. In this case, we may write heuristically (for each $t, \lambda$)

\begin{equation}
\label{eq:approximate_transfer}
\Gamma_{t,\lambda}(\omega) \approx \Gamma(\omega)
\end{equation}

so that using \eqref{eq:filtered_representation} we may write $\{Y_t\}$ in the form

\begin{equation}
\label{eq:approximate_filtered_process}
Y_t = \int \Gamma(\omega_0) A_t(\omega + \omega_0) e^{i\omega t} dZ(\omega)
\end{equation}

where

\begin{equation}
\label{eq:orthogonal_measure_shifted}
E|dZ(\omega)|^2 = |\Gamma(\omega_0)|^2 d\mu(\omega + \omega_0)
\end{equation}

Thus, we have

\begin{equation}
\label{eq:filtered_evolutionary_spectrum}
dF_t^{(Y)}(\omega) = |\Gamma(\omega_0)|^2 dF_t^{(X)}(\omega + \omega_0)
\end{equation}

where the evolutionary spectra $dF_t^{(Y)}(\omega)$ and $dF_t^{(X)}(\omega)$ are both defined with respect to the same family of oscillatory functions $\{A_t(\omega) e^{i\omega t}\}$.

In order to define more precisely the notion of a "slowly varying" function $A_{t-u}(\lambda)$, and to examine in more detail the approximation \eqref{eq:approximate_transfer}, we now introduce the notion of "semi-stationary processes".

\section{Semi-Stationary Processes}
\label{sec:semi_stationary}

Let $\{X_t\}$ be an oscillatory process whose non-stationary characteristics are changing "slowly" over time. Then we may expect that there will exist a family $\mathcal{F}$ of oscillatory functions $\phi_t(\omega) = A_t(\omega) e^{i\omega t}$ in terms of which $\{X_t\}$ has a representation of the form \eqref{eq:transformed_process}, and which are such that, for each $\omega$, $A_t(\omega)$ is (in some sense) a slowly varying function of $t$. Now there are, of course, various ways of defining a slowly varying function, but for our purposes the most convenient characterization is obtained by specifying that its Fourier transform must be "highly concentrated" in the region of zero frequency.

For each family $\mathcal{F}$, define the function $B_{\mathcal{F}}(\omega)$ by

\begin{equation}
\label{eq:bandwidth_def}
B_{\mathcal{F}}(\omega) = \int |\theta| |dH_\omega(\theta)|
\end{equation}

(Note that $B_{\mathcal{F}}(\omega)$ is a measure of the "width" of $|dH_\omega(\theta)|$.)

\begin{definition}
\label{def:semi_stationary_family}
A family $\mathcal{F}$ of oscillatory functions will be termed semi-stationary if the function $B_{\mathcal{F}}(\omega)$ is bounded for all $\omega$, and the constant, $B_{\mathcal{F}}$, defined by

\begin{equation}
\label{eq:characteristic_width}
B_{\mathcal{F}} = \left[\sup_{\omega} \{B_{\mathcal{F}}(\omega)\}\right]^{-1}
\end{equation}

will be termed the characteristic width of the family $\mathcal{F}$.
\end{definition}

\begin{definition}
\label{def:semi_stationary_process}
A semi-stationary process $\{X_t\}$ is now defined as one for which there exists a semi-stationary family $\mathcal{F}$ in terms of which $\{X_t\}$ has a representation of the form \eqref{eq:transformed_process}.
\end{definition}

For example, the uniformly modulated process, defined in Section~\ref{sec:uniformly_modulated}, is a semi-stationary process, since the family $\mathcal{F}_0 = \{c(t) e^{i\omega t}\}$ is semi-stationary. (Note that, since $c(t)$ is independent of $\omega$, $B_{\mathcal{F}_0}(\omega)$ is independent of $\omega$.)

For a particular semi-stationary process $\{X_t\}$ consider the class $\mathcal{C}$ of semi-stationary families $\mathcal{F}$, in terms of each of which $\{X_t\}$ admits a spectral representation. We define the characteristic width of the process $\{X_t\}$, $B_X$, by

\begin{equation}
\label{eq:process_characteristic_width}
B_X = \sup_{\mathcal{F} \in \mathcal{C}} (B_{\mathcal{F}})
\end{equation}

Roughly speaking, $B_X$ may be interpreted as the maximum interval over which the process may be treated as "approximately stationary". Note that for stationary processes the class $\mathcal{C}$ contains the family of complex exponentials, which has infinite characteristic width. Consequently, all stationary processes have infinite characteristic width.

Now let $\mathcal{C}^* \subset \mathcal{C}$ denote the sub-class of families whose characteristic widths are each equal to $B_X$, and let $\mathcal{F}^*$ denote any family $\in \mathcal{C}^*$. For example, if $\{X_t\}$ is stationary, $\mathcal{C}^*$ contains only one family, namely the complex exponentials, so that $\mathcal{F}^*$ is uniquely determined as this family. (However, as far as the theory of evolutionary spectra is concerned, the uniqueness of $\mathcal{F}^*$ is not required—see Section~\ref{sec:determination}.) If $\mathcal{C}^*$ is empty, let $\mathcal{F}^*$ denote any family whose characteristic width is arbitrarily close to $B_X$.

We now consider the spectral representation of $\{X_t\}$ in terms of the family $\mathcal{F}^*$. Thus, we write

\begin{equation}
\label{eq:optimal_representation}
X_t = \int A_t^*(\omega) e^{i\omega t} dZ^*(\omega)
\end{equation}

where $E|dZ^*(\omega)|^2 = d\mu^*(\omega)$, say, and the functions $\Phi_t^*(\omega) = \{A_t^*(\omega) e^{i\omega t}\} \in \mathcal{F}^*$.

It is now clear that if the evolutionary spectrum of $\{X_t\}$ is defined with respect to $\mathcal{F}^*$, \eqref{eq:filtered_evolutionary_spectrum} will be a valid approximation provided that the "width" of $g(u)$ is much smaller than $B_X$, i.e. provided that, for each $\omega$, $dH^*(\theta)$ (the Fourier transform of $A_t(\omega)$) "behaves as a $\delta$-function with respect to $\Gamma(\omega)$". To define this notion more precisely, we introduce the following definition.

\begin{definition}
\label{def:pseudo_delta}
We will say that $u(x)$ is a pseudo $\delta$-function of order $\epsilon$ with respect to $v(x)$ if, for any $k$, there exists $\epsilon$ ($< 1$) independent of $k$ such that

\begin{equation}
\label{eq:pseudo_delta_condition}
\left|\int u(x) v(x+k) dx - v(k) \int u(x) dx\right| < \epsilon
\end{equation}
\end{definition}

Now suppose that
\begin{enumerate}[label=(\alph*)]
\item the filter $\{g(u)\}$ is square integrable and normalized so that

\begin{equation}
\label{eq:filter_normalization}
2\pi \int_{-\infty}^{\infty} |g(u)|^2 du = \int_{-\infty}^{\infty} |\Gamma(\omega)|^2 d\omega = 1
\end{equation}

\item

\begin{equation}
\label{eq:filter_bandwidth}
\int_{-\infty}^{\infty} |u| |g(u)| du = B_g \quad \text{(say)}
\end{equation}

(Note that $B_g$ is a measure of the "width" of $\{g(u)\}$.)
\end{enumerate}

\begin{lemma}
\label{lemma:pseudo_delta}
Let $\{\Phi_t\}$ be a semi-stationary family with characteristic width $B_{\mathcal{F}}$. Then, for each $t, \omega$, $\{e^{it\theta} dH_\omega(\theta)\}$ is a pseudo $\delta$-function of order $(B_g/B_{\mathcal{F}})$ with respect to $\Gamma(\theta)$.
\end{lemma}

\begin{proof}
For any $k$, write

\begin{equation}
\label{eq:pseudo_delta_expansion}\int e^{i\theta k} \Gamma(\theta + k) dH_\omega(\theta) = \Gamma(k) \int e^{i\theta k} dH_\omega(\theta) + R(k)
\end{equation}

in which

\begin{equation}
\label{eq:remainder_term}
R(k) = \int e^{i\theta k} [\Gamma(\theta + k) - \Gamma(k)] dH_\omega(\theta)
\end{equation}

But

\begin{equation}
\label{eq:remainder_bound}
|R(k)| \leq \sup_{\theta} |\Gamma'(\theta)| \int |\theta| |dH_\omega(\theta)| = B_g/B_{\mathcal{F}}
\end{equation}

in virtue of \eqref{eq:filter_bandwidth}—the result follows.
\end{proof}

We are now in a position to derive a more exact form of the relation \eqref{eq:filtered_evolutionary_spectrum}.

\begin{theorem}
\label{thm:filter_approximation}
Let $\{g(u)\}$ be a filter satisfying \eqref{eq:filter_normalization}, \eqref{eq:filter_bandwidth}, and $\Gamma_{t,\lambda}(\omega)$ its generalized transfer function with respect to a semi-stationary family $\mathcal{F}$ of characteristic width $B_{\mathcal{F}}$. If, for any $\epsilon (> 0)$, we choose $\{g(u)\}$ so that $B_g < \epsilon B_{\mathcal{F}}$, then

\begin{equation}
\label{eq:transfer_approximation}
|A_t(\lambda)| \left|\Gamma_{t,\lambda}(\omega) - \Gamma(\omega)\right| < \epsilon
\end{equation}

for all $t, \lambda, \omega$.
\end{theorem}

\begin{proof}
We have, from \eqref{eq:generalized_transfer},

\begin{equation}
\label{eq:transfer_expansion}
A_t(\lambda) \Gamma_{t,\lambda}(\omega) = \int_{-\infty}^{\infty} g(u) A_{t-u}(\lambda) e^{-i\omega u} du
\end{equation}

Substituting for $A_{t-u}(\lambda)$ in terms of its Fourier transform $dH_\lambda(\theta)$, we obtain

\begin{equation}
\label{eq:transfer_fourier}
A_t(\lambda) \Gamma_{t,\lambda}(\omega) = \int_{-\infty}^{\infty} \int_{-\infty}^{\infty} g(u) e^{-i\omega u} e^{i(t-u)\theta} dH_\lambda(\theta) du
\end{equation}

on interchanging the order of integration. However, according to Lemma~\ref{lemma:pseudo_delta}, $\{e^{it\theta} dH_\lambda(\theta)\}$ is a pseudo $\delta$-function of order $(B_g/B_{\mathcal{F}})$ with respect to $\Gamma(\theta)$. Thus, if $\{g(u)\}$ is chosen so that, for given $\epsilon$, $B_g < \epsilon B_{\mathcal{F}}$, then

\begin{equation}
\label{eq:pseudo_delta_approximation}
\left|\int_{-\infty}^{\infty} e^{it\theta} \Gamma(\omega + \theta) dH_\lambda(\theta) - \Gamma(\omega) \int_{-\infty}^{\infty} e^{it\theta} dH_\lambda(\theta)\right| < \epsilon
\end{equation}

Noting that

\begin{equation}
\label{eq:amplitude_integral}
\int_{-\infty}^{\infty} e^{it\theta} dH_\lambda(\theta) = A_t(\lambda)
\end{equation}

the result follows.
\end{proof}

\section{Determination of Evolutionary Spectra}
\label{sec:determination}

Let $\{X_t\}$ be a semi-stationary process with characteristic width $B_X$, and $\{g(u)\}$ a filter satisfying \eqref{eq:filter_normalization}, \eqref{eq:filter_bandwidth}, with width $B_g$. For any frequency $\omega_0$, define the process $\{Y_t\}$ as in \eqref{eq:generalized_filter}, i.e. write

\begin{equation}
\label{eq:filtered_process_def}
Y_t = \int_{-\infty}^{\infty} g(u) X_{t-u} e^{-i\omega_0(t-u)} du
\end{equation}

Using the representation of $\{X_t\}$ in terms of the family $\mathcal{F}^*$ (given by \eqref{eq:optimal_representation}), it follows from \eqref{eq:filtered_representation} that we may write

\begin{equation}
\label{eq:optimal_filtered_representation}
Y_t = \int_{-\infty}^{\infty} \Gamma_{t,\omega_0}^*(\omega) A_t^*(\omega + \omega_0) e^{i\omega t} dZ^*(\omega + \omega_0)
\end{equation}

where $\Gamma_{t,\omega_0}^*(\omega)$ is the generalized transfer function of $\{g(u)\}$ with respect to the family $\mathcal{F}^*$. Due to the orthogonality of $Z^*(\omega)$, it follows that

\begin{equation}
\label{eq:filtered_variance}
E|Y_t|^2 = \int_{-\infty}^{\infty} |\Gamma_{t,\omega_0}^*(\omega)|^2 |A_t^*(\omega + \omega_0)|^2 d\mu^*(\omega + \omega_0)
\end{equation}

Now suppose that $\{g(u)\}$ is chosen so that $B_g < \epsilon B_X$. Then according to Theorem~\ref{thm:filter_approximation} (remembering that the characteristic width of $\mathcal{F}^*$ is either equal to, or arbitrarily close to, $B_X$), we may write

\begin{equation}
\label{eq:transfer_approximation_detailed}
\Gamma_{t,\omega_0}^*(\omega) = \Gamma(\omega) + \epsilon(t, \omega_0, \omega)
\end{equation}

where

\begin{equation}
\label{eq:error_bound}
|\epsilon(t, \omega_0, \omega)| < \epsilon/|A_t^*(\omega + \omega_0)|
\end{equation}

Thus we obtain from \eqref{eq:filtered_variance}

\begin{equation}
\label{eq:variance_expansion}
E|Y_t|^2 = \int_{-\infty}^{\infty} |\Gamma(\omega)|^2 |A_t^*(\omega + \omega_0)|^2 d\mu^*(\omega + \omega_0) + I_1 + I_2 + I_3
\end{equation}

say, where

\begin{align}
I_1 &= \int_{-\infty}^{\infty} \Gamma^*(\omega) \epsilon(t, \omega_0, \omega) |A_t^*(\omega + \omega_0)|^2 d\mu^*(\omega + \omega_0) \\
I_2 &= \int_{-\infty}^{\infty} \Gamma(\omega) \epsilon^*(t, \omega_0, \omega) |A_t^*(\omega + \omega_0)|^2 d\mu^*(\omega + \omega_0) \\
I_3 &= \int_{-\infty}^{\infty} |\epsilon(t, \omega_0, \omega)|^2 |A_t^*(\omega + \omega_0)|^2 d\mu^*(\omega + \omega_0)
\end{align}

Now

\begin{equation}
\label{eq:I3_bound}
|I_3| < \epsilon^2 \int_{-\infty}^{\infty} d\mu^*(\omega) = O(\epsilon^2)
\end{equation}

and

\begin{equation}
\label{eq:I2_bound}
|I_2| < \epsilon \int_{-\infty}^{\infty} |\Gamma(\omega)| |A_t^*(\omega + \omega_0)| d\mu^*(\omega + \omega_0)
\end{equation}

To show that $|I_2| = O(\epsilon)$, it remains to prove that the integral on the right-hand side of the above inequality remains finite as $B_g \to 0$. To demonstrate this fact, let the set $\Omega$ be defined by

\begin{equation}
\label{eq:omega_set}
\Omega = \{\omega: |\Gamma(\omega)| |A_t^*(\omega + \omega_0)| \geq 1\}
\end{equation}

Then

\begin{equation}
\label{eq:integral_split}
\int_{-\infty}^{\infty} |\Gamma(\omega)| |A_t^*(\omega + \omega_0)| d\mu^*(\omega + \omega_0) < \int_{\Omega} d\mu^*(\omega + \omega_0) + \int_{\Omega^c} |\Gamma(\omega)|^2 |A_t^*(\omega + \omega_0)|^2 d\mu^*(\omega + \omega_0)
\end{equation}

The first term is finite, since $d\mu^*(\omega)$ is the evolutionary spectrum at zero time with respect to $\mathcal{F}^*$, and the second term is finite since $\Gamma(\omega)$ is normalized so that

\begin{equation}
\label{eq:transfer_normalization}
\int_{-\infty}^{\infty} |\Gamma(\omega)|^2 d\omega = 1
\end{equation}

The term $I_1$ may be treated similarly, so that in terms of

\begin{equation}
\label{eq:evolutionary_spectrum_optimal}
dF_t(\omega) = |A_t^*(\omega)|^2 d\mu^*(\omega)
\end{equation}

the evolutionary spectrum of $\{X_t\}$ with respect to the family $\mathcal{F}^*$, we have

\begin{theorem}
\label{thm:evolutionary_determination}
\begin{equation}
\label{eq:evolutionary_determination}
E|Y_t|^2 = \int_{-\infty}^{\infty} |\Gamma(\omega)|^2 dF_t(\omega + \omega_0) + O(\epsilon)
\end{equation}

where $O(\epsilon)$ denotes a term which may be made arbitrarily small by choosing $B_g$ sufficiently small relative to $B_X$.
\end{theorem}

Now consider the case where the measure $\mu^*(\omega)$ is absolutely continuous with respect to Lebesgue measure, so that for each $t$ we may write

\begin{equation}
\label{eq:spectral_density}
dF_t(\omega) = f_t(\omega) d\omega, \quad \text{all } \omega
\end{equation}

where $f_t(\omega)$, the evolutionary spectral density function, exists for all $\omega$. Then rewriting Theorem~\ref{thm:evolutionary_determination} in terms of $f_t(\omega)$, we have, to $O(\epsilon)$,

\begin{equation}
\label{eq:density_determination}
E|Y_t|^2 = \int_{-\infty}^{\infty} |\Gamma(\omega)|^2 f_t(\omega + \omega_0) d\omega
\end{equation}

So far we have worked with the representation of $\{X_t\}$ in terms of the family $\mathcal{F}^*$. However, as the validity of \eqref{eq:density_determination} depends only on the condition $B_g < B_{\mathcal{F}}$, it is clear that, for fixed $B_g$, \eqref{eq:density_determination} will still be approximately true if instead we work with a representation of $\{X_t\}$ in terms of any other semi-stationary family $\mathcal{F}$ whose characteristic width $B_{\mathcal{F}} \gg B_g$. Thus, if $dF_t(\omega) = f_t(\omega) d\omega$ is the evolutionary spectrum of $\{X_t\}$ with respect to such a family, then \eqref{eq:density_determination} will still hold approximately if we substitute $f_t(\omega)$ for $f_t^*(\omega)$.

However, it must be remembered that \eqref{eq:density_determination} is only an approximation. In fact, the exact value of $E|Y_t|^2$ is given by

\begin{equation}
\label{eq:exact_variance}
E|Y_t|^2 = \int_{-\infty}^{\infty} |\Gamma_{t,\omega_0}(\omega)|^2 dF_t(\omega + \omega_0)
\end{equation}

if we work in terms of a general family $\mathcal{F}$. Thus, the exact value of $E|Y_t|^2$ is an average of $dF_t(\omega)$ over both frequency and time, and we note that, since $E|Y_t|^2$ is independent of the choice of $\mathcal{F}$, the value of this average of $dF_t(\omega)$ (over time and frequency) must also be independent of $\mathcal{F}$. Thus the right-hand side of \eqref{eq:exact_variance} has an unambiguous interpretation as an "average" of the total energy of the process contained within a band of frequencies in the region of $\omega_0$ and an interval of time in the neighbourhood of $t$.

Now in writing \eqref{eq:density_determination}, we have assumed that the effect of the "time-averaging" is negligible, since the condition $B_g < B_{\mathcal{F}}$ implies that $dF_t(\omega)$ is changing very slowly over the effective range of the filter $\{g(u)\}$. However, the degree of accuracy of \eqref{eq:density_determination} depends on the ratio $(B_g/B_{\mathcal{F}})$. For example, if $B_g = 0$, i.e. $g(u) = \delta(u)$, then \eqref{eq:density_determination} is exact for all $\mathcal{F}$, and reduces to \eqref{eq:total_energy_invariance}, namely

\begin{equation}
\label{eq:exact_energy}
E|Y_t|^2 = \int_{-\infty}^{\infty} dF_t(\omega)
\end{equation}

However, if

\begin{equation}
\label{eq:wide_filter}
g(u) = \lim_{T \to \infty} \{g_T(u)\}
\end{equation}

where

\begin{equation}
\label{eq:boxcar_filter}
g_T(u) = \begin{cases}
1/\sqrt{2T} & (|u| < T) \\
0 & (|u| \geq T)
\end{cases}
\end{equation}

so that $B_g = \infty$, then it may be shown that in this case

\begin{equation}
\label{eq:infinite_bandwidth}
E|Y_t|^2 = \lim_{T \to \infty} \int_{-\infty}^{\infty} |G_{T,t,\omega_0}(\omega)|^2 dF(\omega + \omega_0)
\end{equation}

where

\begin{equation}
\label{eq:finite_fourier}
G_{T,t,\omega_0}(\omega) = \frac{1}{\sqrt{2T}} \int_{-T}^{T} e^{i\omega u} A_{t-u}(\omega_0) du
\end{equation}

Note that $E|Y_t|^2$ is independent of $t$, and reduces to the classical definition of the spectrum of $\{X_t\}$ (if it were stationary), but that \eqref{eq:density_determination} is now invalid for all families.

A comparison of equations \eqref{eq:exact_energy} and \eqref{eq:infinite_bandwidth} is interesting. The right-hand side of \eqref{eq:exact_energy} is a function only of the evolutionary spectrum at time $t$, and does not involve its values at other instants of time, but it provides no information on the distribution of $dF_t(\omega)$ over frequency, since it is completely independent of $\omega_0$. However, assuming that, for each $T$, $|G_{T,t,\omega_0}(\omega)|^2$ is highly concentrated in the region $\omega = 0$ (as will generally be the case since we have assumed that, for all $\omega$, $A_t(\omega)$ is a slowly varying function of $t$), the right-hand side of \eqref{eq:infinite_bandwidth} will be approximately equal to

\begin{equation}
\label{eq:average_spectrum}
dF_{\omega_0} \left\{\lim_{T \to \infty} \int_{-\infty}^{\infty} |G_{T,t,\omega_0}(\omega)|^2 d\omega\right\}
\end{equation}

and this quantity, being completely independent of $t$, may be interpreted as a form of "average" over $t$ of the values of $dF_t(\omega_0)$ for $-\infty < t < \infty$.

Thus we see that the more accurately we try to determine $dF_t(\omega)$ as a function of time, the less accurately we determine it as a function of frequency, and vice versa. This feature suggests a form of UNCERTAINTY PRINCIPLE, namely, "In determining evolutionary spectra, one cannot obtain simultaneously a high degree of resolution in both the time domain and frequency domain".

Suppose now that we fix the degree of resolution in the frequency domain, i.e. we set a lower bound to $B_g$. Then for a particular family $\mathcal{F}$, the resolution in the time domain will be determined by the value of $B_g/B_{\mathcal{F}}$. Clearly then, we obtain the maximum possible resolution in time by working in terms of the family with the maximum characteristic width.

Thus, if $\mathcal{C}^*$ contains only one member, i.e. if $\mathcal{F}^*$ is uniquely determined, then $\mathcal{F}^*$ provides the natural representation for $\{X_t\}$, and is the family in terms of which we can most precisely express the time-varying spectral pattern of the process. In particular, we now see why the natural representation of stationary processes is given in terms of the complex exponential family—the reason being simply that in this case $\mathcal{C}^*$ is unique and is just this family.

If $\mathcal{C}^*$ contains several families, $\mathcal{F}_1^*, \mathcal{F}_2^*, \ldots$ then we may say that each $\mathcal{F}_i^*$ ($i = 1, 2, \ldots$) provides a natural representation for the process. For, let $dF_t^{(i)*}(\omega)$ denote the evolutionary spectrum with respect to $\mathcal{F}_i^*$, and let $\tilde{f}_t^{(i)*}(\omega_0)$ denote the "smoothed" evolutionary spectrum, given by

\begin{equation}
\label{eq:smoothed_spectrum}
\tilde{f}_t^{(i)*}(\omega_0) = \int_{-\infty}^{\infty} |\Gamma(\omega)|^2 dF_t^{(i)*}(\omega + \omega_0)
\end{equation}

Then we have, for each $i$,

\begin{equation}
\label{eq:smoothed_equivalence}
E(|Y_t|^2) = \tilde{f}_t^{(i)*}(\omega_0) + O(B_g/B_X)
\end{equation}

If now we fix the filter width $B_g$, then to $O(B_g/B_X)$ the "smoothed" evolutionary spectra with respect to each $\mathcal{F}_i^*$ are identical. Consequently, as "smoothed" spectra are the most we can determine, we may regard the representations with respect to each $\mathcal{F}_i^*$ as equivalent—at least as far as their corresponding spectra are concerned.

Finally, consider the case when $\mathcal{C}^*$ is empty, i.e. there is no family $\mathcal{F} \in \mathcal{C}$ with characteristic width $B_X$. Then there is no natural representation for the process since now, given any family $\mathcal{F} \in \mathcal{C}$, there always exists another member of $\mathcal{C}$ with larger characteristic width. However, for this case we may redefine the sub-class $\mathcal{C}^*$ so that it includes all families whose characteristic widths lie between $B_X - \eta$ and $B_X$, where $\eta$ is arbitrarily small. Then by the argument used above it follows that all families $\in \mathcal{C}^*$ give rise to almost identical evolutionary spectra.

From now on, we will consider only the representation of $\{X_t\}$ in terms of a particular family $\mathcal{F}^* \in \mathcal{C}^*$, and when we refer to the evolutionary spectrum of $\{X_t\}$ (without reference to any particular family) we will mean $|A_t^*(\omega)|^2 d\mu^*(\omega)$, the evolutionary spectrum with respect to the family $\mathcal{F}^*$.

\section{Estimation of Evolutionary Spectra}
\label{sec:estimation}

Suppose we are given a sample record of $\{X_t\}$, say for $0 < t < T$. We now consider the problem of estimating the evolutionary spectrum $dF_t(\omega)$, for $0 \leq t < T$, from the sample. (We omit the asterisks in $A_t(\omega)$, $d\mu^*(\omega)$, and $dF_t(\omega)$, it being understood that all functions are now defined with respect to the family $\mathcal{F}^*$.) We will treat here only the case where the measure $\mu(\omega)$ is absolutely continuous with respect to Lebesgue measure, so that for each $t$ we may write

\begin{equation}
\label{eq:spectral_density_estimation}
dF_t(\omega) = f_t(\omega) d\omega, \quad \text{all } \omega
\end{equation}

where $f_t(\omega)$, the evolutionary spectral density function, exists for all $\omega$.

Let $\{g(u)\}$ be a filter of width $B_g$ satisfying the conditions of Section~\ref{sec:determination}, and write, for any frequency $\omega_0$,

\begin{equation}
\label{eq:filtered_estimate}
U_t = \int_{-\infty}^{\infty} g(u) X_{t-u} e^{-i\omega_0(t-u)} du
\end{equation}

We assume that $B_g < B_X \ll T$, so that for $t > 0$, the limits in the above integral may be replaced effectively by $(-\infty, \infty)$, when $U_t$ becomes identical with the process $\{Y_t\}$ defined in \eqref{eq:filtered_process_def}. In fact, the difference between $U_t$ and $Y_t$ is due to "transients" (or "end-effects") in the filter output, and we assume that these are negligible for $t$ sufficiently greater than zero. It then follows from Theorem~\ref{thm:evolutionary_determination} that

\begin{equation}
\label{eq:expected_filtered_power}
E|U_t|^2 = \int_{-\infty}^{\infty} |\Gamma(\omega)|^2 f_t(\omega + \omega_0) d\omega + O(B_g/B_X)
\end{equation}

At this point it is interesting to note an important difference between the estimation of evolutionary spectra and the estimation of spectra for stationary processes. In the latter case we may still employ the technique described above, but the bandwidth of $|\Gamma(\omega)|^2$ is chosen as a function of $T$ which tends to zero as $T \to \infty$. However, in dealing with evolutionary spectra, the bandwidth of $|\Gamma(\omega)|^2$ (which varies inversely with $B_g$) is limited by the restriction $B_g < B_X$. In other words, since we have chosen the filter so that it operates only locally on $\{X_t\}$, thereby assuring a high degree of resolution in the time domain, we must sacrifice some degree of resolution in the frequency domain. Thus, in order to estimate $f_t(\omega)$ we must assume that, for each $t$, it is smooth compared with $|\Gamma(\omega)|^2$, i.e. that its bandwidth is substantially larger than the bandwidth of $|\Gamma(\omega)|^2$ (cf. Priestley, 1962). In this case $|\Gamma(\omega)|^2$ is a pseudo $\delta$-function with respect to $f_t(\omega)$ (the order being the ratio of the bandwidths), and we may write

\begin{equation}
\label{eq:approximate_unbiasedness}
E|U_t|^2 \approx f_t(\omega_0)
\end{equation}

(remembering that

\begin{equation}
\label{eq:transfer_normalization_estimation}\int_{-\infty}^{\infty} |\Gamma(\omega)|^2 d\omega = 1.)
\end{equation}

Thus, $|U_t|^2$ is an (approximately) unbiased estimate of $f_t(\omega_0)$. Now a straightforward calculation shows that, for $\{X_t\}$ a normal process,

\begin{equation}
\label{eq:variance_estimate}
\text{var}(|U_t|^2) = \left\{\int_{-\infty}^{\infty} |\Gamma(\omega)|^2 f_t(\omega + \omega_0) d\omega\right\}^2 (1 + \delta_{0,\omega_0})
\end{equation}

which, being independent of $T$, means that $|U_t|^2$ will not be a very useful estimate of $f_t(\omega_0)$ in practice. (This is completely analogous to the behaviour of the periodogram in classical spectral analysis.) However, to reduce sampling fluctuations we may "smooth" the values of $|U_t|^2$ for neighbouring values of $t$. In so doing, we increase the precision of our estimates by sacrificing some degree of resolvability in the time domain.

We, therefore, consider a weight-function $W_{T'}(t)$, depending on the parameter $T'$, which satisfies
\begin{enumerate}[label=(\alph*)]
\item $W_{T'}(t) \geq 0$, all $t, T'$,
\item $W_{T'}(t)$ decays to zero as $|t| \to \infty$, all $T'$,
\item $\int_{-\infty}^{\infty} W_{T'}(t) dt = 1$, all $T'$,
\item $\int_{-\infty}^{\infty} \{W_{T'}(t)\}^2 dt < \infty$, all $T'$.
\end{enumerate}

Write

\begin{equation}
\label{eq:weight_transform}
W_{T'}^*(\lambda) = \int_{-\infty}^{\infty} e^{i\lambda t} W_{T'}(t) dt
\end{equation}

We assume that there exists a constant $C$ such that

\begin{equation}\label{eq:weight_normalization}
\lim_{T' \to \infty} T' \int_{-\infty}^{\infty} |W_{T'}^*(\lambda)|^2 d\lambda = C
\end{equation}

Now let

\begin{equation}
\label{eq:smoothed_estimate}
V_t = \int_{-\infty}^{\infty} W_{T'}(u) |U_{t-u}|^2 du
\end{equation}

Again, we assume that $W_{T'}(u)$ decays sufficiently fast so that the above integral may be evaluated from a finite length of $|U_t|^2$. It follows from \eqref{eq:expected_filtered_power} that

\begin{equation}
\label{eq:expected_smoothed}
E(V_t) = \int_{-\infty}^{\infty} \int_{-\infty}^{\infty} W_{T'}(u) f_{t-u}(\omega + \omega_0) |\Gamma(\omega)|^2 du d\omega + O(B_g/B_X)
\end{equation}

\begin{equation}
\label{eq:smoothed_expectation}
= \int_{-\infty}^{\infty} \tilde{f}_t(\omega + \omega_0) |\Gamma(\omega)|^2 d\omega + O(B_g/B_X)
\end{equation}

where

\begin{equation}
\label{eq:smoothed_density}
\tilde{f}_t(\omega + \omega_0) = \int_{-\infty}^{\infty} W_{T'}(u) f_{t-u}(\omega + \omega_0) du
\end{equation}

From \eqref{eq:smoothed_expectation} we see that $E(V_t)$ is a "smoothed" form of $f_t(\omega_0)$, smoothed over both time and frequency. If we now assume that $f_t(\omega)$ is smooth compared with $|\Gamma(\omega)|^2$ and $W_{T'}(t)$, then

\begin{equation}
\label{eq:final_approximation}
E(V_t) \approx f_t(\omega_0)
\end{equation}

and $V_t$ provides a useful estimate of $f_t(\omega_0)$.

\begin{thebibliography}{99}

\bibitem{bartlett1955} Bartlett, M. S. (1955). {\em An Introduction to Stochastic Processes}. Cambridge University Press.

\bibitem{cramer1960} Cramér, H. (1960). On the theory of stationary random processes. {\em Arkiv för Matematik}, 4, 375-378.

\bibitem{grenander1957} Grenander, U. and Rosenblatt, M. (1957). {\em Statistical Analysis of Stationary Time Series}. Wiley, New York.

\bibitem{herbst1963a} Herbst, L. (1963a). A test for quasi-stationarity. {\em Annals of Mathematical Statistics}, 34, 206-211.

\bibitem{herbst1963b} Herbst, L. (1963b). On the asymptotic behaviour of processes with time-varying spectra. {\em Annals of Mathematical Statistics}, 34, 212-216.

\bibitem{herbst1963c} Herbst, L. (1963c). The statistical analysis of non-stationary time series. {\em Skandinavisk Aktuarietidskrift}, 46, 1-24.

\bibitem{jowett1957} Jowett, G. H. (1957). Statistical analysis using local properties of smoothly heteromorphic stochastic series. {\em Biometrika}, 44, 454-463.

\bibitem{page1952} Page, C. H. (1952). Instantaneous power spectra. {\em Journal of Applied Physics}, 23, 103-106.

\bibitem{parzen_unpublished} Parzen, E. (unpublished). On the representation of stochastic processes. Technical Report, Stanford University.

\bibitem{priestley1962} Priestley, M. B. (1962). The analysis of stationary processes with mixed spectra. {\em Technometrics}, 4, 551-565.

\end{thebibliography}

\end{document}