\documentclass{article}
\usepackage[english]{babel}
\usepackage{geometry,amsmath,amssymb,latexsym,theorem}
\geometry{letterpaper}

%%%%%%%%%% Start TeXmacs macros
\newcommand{\assign}{:=}
\newcommand{\cdummy}{\cdot}
\newcommand{\tmaffiliation}[1]{\\ #1}
\newcommand{\tmdummy}{$\mbox{}$}
\newcommand{\tmop}[1]{\ensuremath{\operatorname{#1}}}
\newcommand{\tmtextbf}[1]{\text{{\bfseries{#1}}}}
\newcommand{\tmtextit}[1]{\text{{\itshape{#1}}}}
\newenvironment{proof}{\noindent\textbf{Proof\ }}{\hspace*{\fill}$\Box$\medskip}
\newtheorem{definition}{Definition}
\newtheorem{lemma}{Lemma}
{\theorembodyfont{\rmfamily}\newtheorem{remark}{Remark}}
\newtheorem{theorem}{Theorem}
%%%%%%%%%% End TeXmacs macros

\begin{document}

\title{The sampling theorem, Dirichlet series and Hankel transforms}

\author{
  Dieter Klusch
  \tmaffiliation{Institut f{\"u}r Mathematik I (WE 1), Freie Universit{\"a}t
  Berlin, Germany}
}

\date{Received 9 April 1991\\
Revised 19 December 1991}

\maketitle

\begin{abstract}
  Some very surprising relations between fundamental theorems and formulas of
  signal analysis, of analytic number theory and of applied analysis are
  presented. It is shown that generalized forms of the classical
  Whittaker-Kotelnikov-Shannon sampling theorem as well as of the
  Brown-Butzer-Splettst{\"o}{\ss}er approximate sampling expansion for
  non-band-limited signal functions can be deduced via the theory of Dirichlet
  series with functional equations from a new summation formula for Hankel
  transforms. This counterpart to Poisson's summation formula is shown to be
  essentially ``equivalent'' to the famous functional equation of Riemann's
  zeta-function, to the ``modular relation'' of the theta-function, to the
  Nielsen-Doetsch summation formula for Bessel functions and to the partial
  fraction expansion of the periodic Hilbert kernel.
\end{abstract}

{\noindent}\tmtextbf{Keywords:} Approximate sampling theorem; Whittaker's
cardinal series; Hankel transforms; Dirichlet series; Riemann's zeta-function;
theta-transformation; Bessel functions.

{\tableofcontents}

\section{Introduction}\label{sec:intro}

The Whittaker-Shannon sampling theorem states that every signal function $f :
\mathbb{R} \to \mathbb{C}$ that is band-limited to $[- \pi W, \pi W]$ for some
$W > 0$, i.e., its Fourier transform $\hat{f} (\nu)$ vanishes for almost all
$| \nu | > \pi W$, can be completely reconstructed from its sampled values $f
(n / W)$, $n \in \mathbb{Z}$, in terms of Whittaker's cardinal series (cf.
{\cite{28,32}})
\begin{equation}
  f (t) = \sum_{n = - \infty}^{+ \infty} f \left( \frac{n}{W} \right)
  \mathrm{sinc} (Wt - n) \forall t \in \mathbb{R}
  \label{eq:whittaker_cardinal}
\end{equation}
where $\mathrm{sinc} (t) \assign \frac{\sin \pi t}{\pi t}$, $t \neq 0$, and
$\mathrm{sinc} (0) \assign 1$ is the Dirichlet kernel. There are various
extensions of this Lagrange-type interpolation formula which is the
theoretical basis of modern pulse-code modulation communication systems. It
has been proved by Butzer et al. (cf. {\cite{7,8,9,10}}) that the
Brown-Butzer-Splettst{\"o}{\ss}er approximate sampling theorem for not
necessarily band limited functions
\begin{equation}
  f (t) = \sum_{n = - \infty}^{+ \infty} f \left( \frac{n}{W} \right)
  \mathrm{sinc} \{Wt - n\} \forall t \in \mathbb{R} \label{eq:brown_butzer}
\end{equation}
is essentially ``equivalent'' to three fundamental theorems in three different
fields, namely to the Poisson summation formula of Fourier analysis, to a
particular form of Cauchy's integral formula in complex function theory, as
well as to the Euler-Maclaurin summation formula of numerical analysis. Hence
the sampling theorem plays a unique role in various branches of analysis (cf.
{\cite{5}}).

In the present paper, it is proved by means of the theory of Dirichlet series
that generalized forms of the sampling expansions
\eqref{eq:whittaker_cardinal} and \eqref{eq:brown_butzer} can also be deduced
via a Poisson-type duality formula (cf. Theorem~\ref{thm:poisson_type} below)
from five important theorems of the theory of integral transforms, of analytic
number theory and of the theory of special functions of mathematical physics.
In particular, \eqref{eq:whittaker_cardinal} and \eqref{eq:brown_butzer} are
deduced from a general summation formula for ordinary Hankel integral
transforms (cf. Theorems~\ref{thm:hankel_shannon} and~\ref{thm:riemann_hankel}
below), which is shown to be ``equivalent'' to the famous functional equation
of Riemann's zeta-function (cf. Theorem~\ref{thm:riemann_functional} below),
to the ``modular relation'' of Jacobi's theta-function (cf.
Lemma~\ref{lem:theta_relation} below), to the well-known Nielsen-Doetsch
summation formula for Bessel functions of the first kind (cf.
Lemma~\ref{lem:bessel_summation} below) as well as to the partial fraction
expansion of the periodic ``Hilbert-kernel'' (cf.
Lemma~\ref{lem:hilbert_kernel} below).

Results of this special type on nontrivial relations between fundamental
theorems of signal analysis, analytic number theory and applied analysis are
obviously new. Demonstrably they are unparalleled in the vast literature of
signal theory and Fourier analysis (cf. {\cite{8,21}}).

On the other hand, they also extend well-known results of the theory of
Dirichlet series with functional equations and related arithmetical identities
due to many authors (see, e.g., {\cite{2,3,4,11,12,13,14,22}}). In fact, none
of these papers contains any explicit contribution to modern signal analysis
or to the ``equivalent'' characterisation of Riemann's functional equation by
summation properties of Hankel transforms (cf.
Theorem~\ref{thm:riemann_functional} below). The same fact holds for the
classical deep works {\cite{15,17,19,20,27,29}}, which form the basis of our
considerations. By the way, we too extend an interesting result of {\cite{18}}
(see also {\cite{1,26}}) on the famous Voronoi summation formula and its
relation to Dirichlet series.

Concerning the preliminaries, let $L^p (\mathbb{R})$, $1 \leq p \leq \infty$,
denote the space of all complex-valued Lebesgue measurable functions $f$
defined on $\mathbb{R}$ for which the norms
\begin{equation}
  \|f\|_p \assign \left( \int_{\mathbb{R}} |f (u) |^p  \hspace{0.17em} du
  \right)^{\frac{1}{p}} \forall 1 \leq p < \infty \quad \|f\|_{\infty} \assign
  \textrm{\tmop{ess} \hspace{0.17em} \sup}_{u \in \mathbb{R}} |f (u) |
  \label{eq:lp_norm}
\end{equation}
are finite. By $C (\mathbb{R})$, denote the space of all uniformly continuous
and bounded functions on $\mathbb{R}$ endowed with the supremum norm $\|
\cdummy \|_C$. The Fourier transform $\hat{f}$ of $f \in L^1 (\mathbb{R})$ is
defined by
\begin{equation}
  \hat{f} (u) \assign \int_{\mathbb{R}} f (u) e^{- iuu}  \hspace{0.17em} du
  \forall u \in \mathbb{R} \label{eq:fourier_transform}
\end{equation}
and of $f \in L^2 (\mathbb{R})$ by the limit in the $L^2 (\mathbb{R})$-norm of
\begin{equation}
  \int_{- R}^R f (u) e^{- iuu}  \hspace{0.17em} du \forall R \to \infty
  \label{eq:fourier_l2}
\end{equation}
If $f \in L^p (\mathbb{R})$, $p = 1$ or $2$, is such that $\hat{f} \in L^1
(\mathbb{R})$, then the Fourier inversion formula
\begin{equation}
  f (t) = \frac{1}{\sqrt{2 \pi}}  \int_{\mathbb{R}} \hat{f} (u) e^{iut} 
  \hspace{0.17em} du \forall t \in \mathbb{R} \label{eq:fourier_inversion}
\end{equation}
holds at each point of continuity of $f$. The bilateral Laplace transform
$\mathcal{L}_z \{f\}$ of $f \in L^1 (\mathbb{R})$ and $f$ being of bounded
exponential growth is defined in its strip of convergence $S$ by
\begin{equation}
  \mathcal{L}_z \{f\} \assign \int_{\mathbb{R}} f (u) e^{- zu} 
  \hspace{0.17em} du \forall z \in S \label{eq:laplace_transform}
\end{equation}
The Hankel transform of $f \in L^1 (\mathbb{R}_+)$ is defined for $s \in
\mathbb{R}_+$ by
\begin{equation}
  (H_{\nu} f) (s) \assign \int_0^{\infty} f (t) \sqrt{st} J_{\nu}  (2
  \sqrt{st})  \hspace{0.17em} dt \label{eq:hankel_transform}
\end{equation}
where $J_{\nu}$ is the Bessel function of the first kind of order $\nu \geq
0$.

\section{Riemann's functional equation and Hankel
transforms}\label{sec:riemann}

As remarked above, one aim is to deduce the sampling theorems from Riemann's
functional equation. In view of Fourier's inversion theorem (cf.
\eqref{eq:fourier_inversion} below) it is quite sufficient to base further
investigations on the following generalized functional equation of Riemann's
type.

\begin{definition}
  \label{def:riemann_functional}Let $\{\lambda_n \}$ and $\{\mu_n \}$ be two
  sequences of positive numbers strictly increasing to infinity, and $\{a_n
  \}$ and $\{b_n \}$ two sequences of complex numbers not identically zero.
  Consider the functions $\phi (s)$ and $\psi (s)$ represented as absolutely
  convergent Dirichlet series in the half-plane $\mathrm{Re} s > 1$:
  \begin{equation}
    \phi (s) = \sum_{n \geq 1} \frac{a_n}{\lambda_n^s}, \quad \psi (s) =
    \sum_{n \geq 1} \frac{b_n}{\mu_n^s} . \label{eq:dirichlet_series}
  \end{equation}
  State that $\phi$ and $\psi$ satisfy the functional equation if there exists
  a meromorphic function $\chi$ with the following properties:
  \begin{enumerate}
    \item
    
    \begin{align}
      \chi (s) = \pi^{- s / 2} \Gamma \left( \frac{s}{2} \right) \phi (s), &
      \quad \mathrm{Re} s > 1,  \label{eq:chi_right}\\
      \chi (s) = \pi^{- (1 - s) / 2} \Gamma \left( \frac{1 - s}{2} \right)
      \psi (1 - s), & \quad \mathrm{Re} s < 0.  \label{eq:chi_left}
    \end{align}
    
    \item $\chi (s)$ has only simple poles in $\mathbb{C}$ at $s = 0$ and $s =
    1$, i.e., $\phi (s)$ is analytic in $\mathbb{C} \setminus \{1\}$ and $(s -
    1) \phi (s)$ is an entire function whose order one regards as finite.
  \end{enumerate}
\end{definition}

In view of \eqref{eq:chi_right}, let
\begin{equation}
  a_0 \assign - 2 \psi (0), \quad b_0 \assign \mathrm{res}_{s = 1} \phi (s),
  \quad \lambda_0 = \mu_0 \assign 0, \label{eq:residue_definition}
\end{equation}
\begin{equation}
  a_{- n} \assign a_n, \quad b_{- n} \assign b_n, \quad \lambda_{- n} \assign
  - \lambda_n, \quad \mu_{- n} \assign - \mu_n \label{eq:symmetric_extension}
\end{equation}
$\forall n \in \mathbb{N}$. The following lemmas are originally due to
Hamburger (cf. {\cite{4,11,19}}).

\begin{lemma}
  \label{lem:theta_relation}Riemann's functional equation
  \eqref{eq:chi_right}--\eqref{eq:chi_left} is ``equivalent'' to the
  theta-relation
  \begin{equation}
    \sum_{n = - \infty}^{\infty} a_n \exp (- \pi \lambda_n^2 \tau) = \tau^{- 1
    / 2}  \sum_{n = - \infty}^{\infty} b_n \exp (- \pi \mu_n^2 \tau^{- 1})
    \forall \quad \mathrm{Re} \tau > 0. \label{eq:theta_relation_eq}
  \end{equation}
\end{lemma}

The behaviour of \eqref{eq:theta_relation_eq} under the one-sided Laplace
transform is given by the generalized Nielsen-Doetsch formula for
Schl{\"o}milch series (cf. {\cite{11,15,16,23,24,31}}):

\begin{lemma}
  \label{lem:bessel_summation}The theta-relation \eqref{eq:theta_relation_eq}
  is ``equivalent'' to the Bessel summation formula
  
  \begin{equation}
    \frac{\Gamma (\nu + 1)}{\pi^{\nu}}  \left[ b_0 t^{\nu} + 2 \sum_{n \geq 1}
    b_n \mu_n^{\nu} J_{\nu} (2 \pi \mu_n t) \right] = \frac{t^{\nu - 1 /
    2}}{\pi^{\nu + 1 / 2}}  \left[ \frac{a_0}{\Gamma (\nu + 1)} e^{\mu t} + 2
    \sum_{n \in \mathbb{N}} a_n (t - \lambda_n)_+^{\nu - 1 / 2} \right]
    \label{eq:bessel_summation_eq}
  \end{equation}
  
  where $t \in \mathbb{R}_+, \nu \geq 0$.
\end{lemma}

We now transfer \eqref{eq:bessel_summation_eq} to the space of Hankel
transforms. More generally, prove the following obviously new ``equivalence''
theorem, which extends, e.g., researches of {\cite{11}}.

\begin{theorem}
  \label{thm:riemann_functional}Let $f : \mathbb{R}_+ \to \mathbb{C}$ be such
  that, for $\xi > 0$ and $q \geq \frac{1}{2}$,
  \begin{equation}
    g_{\xi, 0} (t) \assign f (t) t^{1 / 4}  (t - \xi)^{q - 1 / 4} \in L^2
    (\mathbb{R}_+) \cap C (\mathbb{R}_+) \label{eq:hankel_condition_1}
  \end{equation}
  and, for $\xi > 0$ and $q > \frac{3}{2}$,
  \begin{equation}
    g_{\xi, q} (t) = O (t^{- \nu - 1}) \forall t \to \infty .
    \label{eq:hankel_condition_2}
  \end{equation}
  Define the operator
  \begin{equation}
    (T_{\nu} f) (\xi) \assign \int_0^{\infty} g_{\xi, q} (t)  \hspace{0.17em}
    dt \label{eq:operator_definition}
  \end{equation}
  Then Riemann's functional equation \eqref{eq:chi_right}--\eqref{eq:chi_left}
  is ``equivalent'' to the Hankel summation formula
  \begin{equation}
    \begin{array}{l}
      \text{\qquad} \frac{\Gamma (\nu + 1)}{\pi^{\nu}}  \left[ b_0 (T_{\nu} f)
      (0) + 2 \sum_{n \geq 1} b_n \mu_n^{\nu} (H_{\nu} f) (T_2 \mu_n^2)
      \right]\\
      = \frac{\pi^{\nu - 1 / 2}}{\Gamma (\nu + 1 / 2)}  \left[ a_0 (T_{\nu} f)
      (0) + 2 \sum_{n \geq 1} a_n (T_{\nu} f) (\lambda_n) \right]
    \end{array} \label{eq:hankel_summation_formula}
  \end{equation}
  valid for $\nu > \frac{1}{2}$.
\end{theorem}

\begin{proof}
  Show that \eqref{eq:theta_relation_eq} and
  \eqref{eq:hankel_summation_formula} are ``equivalent''. The assertion then
  follows from Lemma~\ref{lem:theta_relation}.
  
  Multiply \eqref{eq:bessel_summation_eq} by $g_{\xi, q} (t) = f (t) \in L^1
  (\mathbb{R}_+) \cap C (\mathbb{R}_+)$. Integration over $\mathbb{R}_+$ then
  yields by \eqref{eq:operator_definition}:
  \begin{equation}
    \begin{array}{l}
      \text{} \frac{\pi^{\nu}}{\Gamma (\nu + 1)}  (T_{\nu} f) (0) + 2 \sum_{n
      \geq 1} b_n \mu_n^{\nu}  \int_0^{\infty} f (t) J_{\nu}  (2 \pi \mu_n t) 
      \hspace{0.17em} dt\\
      = \frac{\pi^{\nu + 1 / 2}}{\Gamma (\nu + 1 / 2)}  \left[ a_0 (T_{\nu} f)
      (0) + 2 \sum_{m \in \mathbb{N}} a_n (T_{\nu} f) (\lambda_n) \right]
    \end{array} \label{eq:hankel_proof_1}
  \end{equation}
  For $\nu > 0$ the asymptotic equality
  \begin{equation}
    J_{\nu} (z) = \sqrt{\frac{2}{\pi z}} \cos \left( z - \frac{\nu \pi}{2} -
    \frac{\pi}{4} \right) \quad \forall z \to \infty
    \label{eq:bessel_asymptotic}
  \end{equation}
  holds, and since $g_{\xi, q} (t) = t^{- 1 / 4} f (t) \in L^1 (\mathbb{R}_+)
  \cap C (\mathbb{R}_+)$, exchange by \eqref{eq:operator_definition} the order
  of summation and integration on the left-hand side of
  \eqref{eq:hankel_proof_1}. By condition \eqref{eq:hankel_condition_2} and in
  view of \eqref{eq:dirichlet_series} one has for $\nu > \frac{1}{2}$ and some
  constant $c > 0$,
  \begin{equation}
    c \left| \sum_{m \in \mathbb{N}} a_n g_{\xi, q} (t) \hspace{0.17em} dt
    \right| \leq c \sum_{m \geq 1} |a_n | \lambda_n^{2 \mu - 1} .
    \label{eq:hankel_bound}
  \end{equation}
  Hence the interchange of summation and integration on the right-hand side of
  \eqref{eq:hankel_proof_1} is permissible and:
  \begin{equation}
    \begin{array}{l}
      \frac{\Gamma (\nu + 1)}{\pi^{\nu}}  \left[ b_0 (T_{\nu} f) (0) + 2
      \sum_{n \geq 1} b_n \mu_n^{\nu}  \int_0^{\infty} f (t) J_{\nu} (2 \pi
      \mu_n t) \hspace{0.17em} dt \right]\\
      = \frac{\pi^{\nu - 1 / 2}}{\Gamma (\nu + 1 / 2)}  \left[ a_0 (T_{\nu} f)
      (0) + 2 \sum_{n \geq 1} a_n (T_{\nu} f) (\lambda_n) \right]
    \end{array} \label{eq:hankel_proof_2}
  \end{equation}
  which by \eqref{eq:hankel_transform} and \eqref{eq:operator_definition}
  yields \eqref{eq:hankel_summation_formula}. Now to the converse direction.
  Prove a generalized version of \eqref{eq:theta_relation_eq} by applying
  \eqref{eq:hankel_summation_formula} with $\nu = 2 m > \frac{1}{2}$ to the
  function
  \begin{equation}
    f (t) = t^{2 \mu - m - 1 / 2} \exp \{- \pi t \tau^{- 1} \} \forall t \in
    \mathbb{R}_+ \label{eq:test_function}
  \end{equation}
  where $\mu > 0$ and $\mathrm{Re} \tau > 0$. Verify directly that
  \begin{equation}
    g_{2 m} (t) = f (t) t^{- \mu}  (t - \xi)^{2 m - 1 / 2} \forall \xi > 0,
    \label{eq:test_satisfies_1}
  \end{equation}
  satisfies \eqref{eq:hankel_condition_1} and \eqref{eq:hankel_condition_2}.
  
  By \eqref{eq:hankel_transform} one has for $s \in \mathbb{R}_+$,
  \begin{equation}
    (H_{\nu} f) (s) = \left( \frac{\pi}{2} \right)^{2 \mu - m} 
    \int_0^{\infty} t^{2 \mu - m - 1 / 2} e^{- \pi t \tau^{- 1}} J_{2 m}  (2
    \sqrt{st})  \hspace{0.17em} dt. \label{eq:hankel_computed}
  \end{equation}
  Using Hankel's generalisation of Weber's first exponential integral (cf.
  {\cite{31}}), obtain the Hankel transform of \eqref{eq:test_function}:
  \begin{equation}
    (H_{\nu} f) (s) = 2^{2 \mu - m} \left( \frac{\pi}{\tau} \right)^{\mu}
    s^{\mu - 1 / 2} M_{2 m - \mu, \mu}  [2 \pi s \tau^{- 1}],
    \label{eq:hankel_whittaker}
  \end{equation}
  where $M (z)$ is the Whittaker function of the first kind. By
  \eqref{eq:operator_definition} one has for $\xi \geq 0$,
  \begin{equation}
    (T_{2 m} f) (\xi) = \left( \frac{\pi}{2 \mu} \right)^2  \int_{\max (0,
    \xi)}^{\infty} t^{2 (\mu - m) - 1 / 2} e^{- \pi t \tau^{- 1}} 
    \hspace{0.17em} dt \label{eq:operator_computed}
  \end{equation}
  For the case $\xi = 0$ in \eqref{eq:operator_computed}, obtain by Euler's
  representation of the $\Gamma$-function,
  \begin{equation}
    (T_{2 m} f) (0) = \left( \frac{\pi}{2 \mu} \right)^2 \Gamma (2 \mu), \quad
    (T_{2 m + 1} f) (0) = \left( \frac{\pi}{2 \mu + 1 / 2} \right)^2 \Gamma (2
    \mu + 1 / 2) \label{eq:operator_zero}
  \end{equation}
  For the case $\xi > 0$ in \eqref{eq:operator_computed}, the Whittaker
  integral representation of the Whittaker function $W (z)$ of the second kind
  is needed (cf. {\cite{33}}):
  \begin{equation}
    W_{\kappa, \mu} (z) = \frac{e^{- \frac{z}{2}} z^{\kappa +
    \frac{1}{2}}}{\Gamma (\mu - \kappa + 1)}  \int_0^{\infty} e^{- zt} t^{\mu
    - \kappa - \frac{1}{2}}  (1 + t)^{\kappa + \mu}  \hspace{0.17em} dt
    \label{eq:whittaker_integral}
  \end{equation}
  valid for $\mathrm{Re} z > 0$ and $\mathrm{Re} (\mu - \kappa) > -
  \frac{1}{2}$. Now substitute $t = \xi + \frac{\tau \nu}{\pi}$, $\nu > 0$, in
  \eqref{eq:operator_computed}. Then in view of \eqref{eq:whittaker_integral},
  obtain by straightforward computation:
  \begin{equation}
    (T_{2 m} f) (\xi) = \left( \frac{\pi}{2 \mu + \frac{1}{2}} \right)^2
    \xi^{\mu - \frac{1}{2}} e^{- \frac{\pi \xi}{\tau}} W_{2 m - \mu, \mu} 
    [\pi \xi \tau^{- 1}] \label{eq:operator_xi}
  \end{equation}
  Inserting \eqref{eq:hankel_whittaker}, \eqref{eq:operator_zero} and
  \eqref{eq:operator_xi} into \eqref{eq:hankel_summation_formula}, obtain the
  Whittaker transformation formula
  \begin{equation}
    \begin{array}{l}
      b_0  \sqrt{\tau} \pi + (\tau \pi)^{\mu}  \sum_{n \geq 1} b_n \mu_n^{2 m
      - 1} e^{- \frac{\mu_n \pi}{\tau}} M_{2 m - \mu, \mu}  [\mu_n \pi \tau]\\
      = \frac{\Gamma (2 m + 1) \Gamma (2 \mu)}{\Gamma (2 m + 1 / 2) \Gamma (2
      \mu + 1 / 2)}  \left[ a_0 \pi + \sum_{n \geq 1} a_n \lambda_n^{\mu -
      \frac{1}{2}} e^{- \frac{\lambda_n \pi}{\tau}} W_{2 m - \mu, \mu}
      [\lambda_n \pi \tau^{- 1}] \right] \label{eq:whittaker_theta}
    \end{array}
  \end{equation}
  valid for $\mathrm{Re} \tau > 0$, $\mu > 0$ and $m > \frac{1}{2}$. Finally,
  observe that (cf. {\cite{16}})
  \begin{equation}
    M_{\frac{1}{2} + m, m} (z) = z^{\frac{1}{2} + m} e^{- \frac{z}{2}}
    \label{eq:whittaker_special}
  \end{equation}
  and
  \begin{equation}
    W_{\frac{1}{4} - m, \frac{1}{4} + m} (z) = z^{\frac{1}{4} - m} e^z
  \end{equation}
  $\forall \mathrm{Re} z > 0, m \in \mathbb{R}_+ .$ Hence the Whittaker
  theta-relation \eqref{eq:whittaker_theta} reduces for $\mu = m +
  \frac{1}{4}$ to \eqref{eq:theta_relation_eq} and
  Theorem~\ref{thm:riemann_functional} is proved.
\end{proof}

\begin{remark}
  \label{rem:remarks_2_5}{\tmdummy}
  
  \begin{enumerate}
    \item In the case $a_n = b_n = 1$ and $\lambda_n = \mu_n = n$, equation
    \eqref{eq:chi_right}--\eqref{eq:chi_left} becomes Riemann's functional
    equation for the ordinary $\zeta$-function while
    \eqref{eq:theta_relation_eq} reduces to the linear transformation formula
    for Jacobi's elliptic theta-function $\vartheta_3 (0| \tau)$,
    \eqref{eq:bessel_summation_eq} is the classical Nielsen-Doetsch summation
    formula for Bessel functions (cf. {\cite{15,31}}) and
    \eqref{eq:hankel_summation_formula} degenerates to a series transformation
    formula which at first has been studied without rigorous proof by Erdelyi
    (cf. {\cite{1,17}}) in the theory of special functions of mathematical
    physics.
    
    \item Note that the Whittaker theta-relation \eqref{eq:whittaker_theta} is
    a far-reaching extension of the ``modular relation''
    \eqref{eq:theta_relation_eq}. Hence the Hankel formula
    \eqref{eq:hankel_summation_formula} can be regarded as a counterpart to
    Poisson's duality formula for Fourier transforms.
  \end{enumerate}
\end{remark}

\section{Poisson formula}\label{sec:poisson}

The following ``equivalent'' characterisation of \eqref{eq:theta_relation_eq}
is originally due to Hamburger and Siegel (cf. {\cite{4,11,19,29}}).

\begin{lemma}
  \label{lem:hilbert_kernel}The theta-relation \eqref{eq:theta_relation_eq} is
  ``equivalent'' to the partial fraction expansion
  \begin{equation}
    F (z) \assign a_0 + 2 \sum_{n \geq 1} a_n e^{- 2 \pi \lambda_n^2 z} =
    \frac{\pi z}{\pi z} + \sum_{n \geq 1} \frac{2 b_n z}{(\pi z)^2 + \mu_n^2},
    \quad \mathrm{Re} z > 0. \label{eq:partial_fraction}
  \end{equation}
\end{lemma}

For further investigations, only the fact that \eqref{eq:theta_relation_eq}
implies the expansion \eqref{eq:partial_fraction} of the generalized periodic
``Hilbert kernel'' is needed (cf. {\cite{6}}). In view of
\eqref{eq:partial_fraction}, give the next definition.

\begin{definition}
  \label{def:class_An}For $z = x + iy \in \mathbb{C}$ and fixed $\eta > 0$,
  consider the strip $S$: $|x| < \eta$, $y \in \mathbb{R}$.
  
  Denote by $A_{\eta}$ the class of functions $\Omega : \mathbb{C} \to
  \mathbb{C}$ with the following properties:
  \begin{enumerate}
    \item $\Omega (z)$ is analytic in $S$.
    
    \item There exists a positive number $\delta < \eta$ so that $\Omega (\pm
    \delta + iy) \in L^1 (\mathbb{R})$.
    
    \item Two sequences of positive numbers $\{\alpha_{\mu} \}$ and
    $\{\beta_{\mu} \}$ strictly increasing to infinity can be determined so
    that uniformly for $|x| \leq \delta$,
    \begin{equation}
      \Omega (x + i \alpha_{\mu}) = O (1), \quad \Omega (x - i \beta_{\mu}) =
      O (1), \quad \mu \to \infty, \label{eq:hankel_class_1}
    \end{equation}
    and with $F (z)$ defined by \eqref{eq:partial_fraction},
    \begin{equation}
      \Omega (x + i \alpha_{\mu}) F (x + i \alpha_{\mu}) = O (1), \quad \Omega
      (x - i \beta_{\mu}) F (x - i \beta_{\mu}) = O (1), \quad \mu \to \infty
      . \label{eq:hankel_class_2}
    \end{equation}
  \end{enumerate}
\end{definition}

For the class $A_{\eta}$, prove the following theorem (cf. {\cite{19}}).

\begin{theorem}
  \label{thm:residue_formula}Let $\Omega \in A_{\eta}$. Then the Hankel
  summation formula \eqref{eq:hankel_summation_formula} implies the residue
  formula
  \begin{equation}
    \sum_n \Omega (i \mu_n) = \sum_n a_n e^{- 2 \pi i \lambda_n y} +
    \frac{1}{2 \pi i}  \int_{(- \delta)} \Omega (z) F (z)  \hspace{0.17em} dz.
    \label{eq:residue_formula_eq}
  \end{equation}
\end{theorem}

\begin{proof}
  Function $F (z)$ defined by \eqref{eq:partial_fraction} is analytic in
  $\mathbb{C}$ except for simple poles at $z = i \mu_n$, $n \in \mathbb{Z}$.
  For $0 < \delta < \eta$, consider the fixed straight lines $(\pm \delta)
  \assign (\pm \delta - i \infty, \pm \delta + i \infty)$.
  
  By \eqref{eq:hankel_class_2}, the Phragm{\'e}n-Lindel{\"o}f principle and
  Cauchy's theorem:
  \begin{equation}
    \int_{(\delta)} \Omega (z) F (z)  \hspace{0.17em} dz = \int_{(- \delta)}
    \Omega (z) F (z)  \hspace{0.17em} dz + 2 \pi i \sum_n b_n \Omega (i \mu_n)
    . \label{eq:residue_proof_1}
  \end{equation}
  Since $F (z)$ is an odd function of $z$:
  \begin{equation}
    \frac{1}{\pi i}  \sum_n b_n \Omega (i \mu_n) = \frac{1}{2 \pi i} 
    \int_{(\delta)} F (z)  \{\Omega (z) + \Omega (- z)\}  \hspace{0.17em} dz.
    \label{eq:residue_proof_2}
  \end{equation}
  By Lemma~\ref{lem:hilbert_kernel}, $F (z)$ is also represented for
  $\mathrm{Re} z > 0$ by the absolutely convergent Dirichlet series on the
  left-hand side of \eqref{eq:partial_fraction}. Hence inserting in
  \eqref{eq:residue_proof_2}, with \eqref{eq:hankel_class_1} and Lebesgue's
  dominated convergence theorem:
  \begin{equation}
    \begin{array}{ll}
      \frac{1}{\pi i}  \sum_n b_n \Omega (i \mu_n) & = \frac{1}{2 \pi i} 
      \int_{(\delta)} a_0  \{\Omega (z) + \Omega (- z)\}  \hspace{0.17em} dz\\
      & + \frac{1}{\pi i}  \sum_{n \geq 1} a_n  \int_{(\delta)} e^{- 2 \pi
      \lambda_n}  \{\Omega (z) + \Omega (- z)\}  \hspace{0.17em} dz
    \end{array} \label{eq:residue_proof_3}
  \end{equation}
  Now by \eqref{eq:hankel_class_1} relation \eqref{eq:residue_proof_3} also
  holds for the imaginary axis. Hence:
  \begin{equation}
    \sum_n b_n \Omega (i \mu_n) = a_0  \int_{\mathbb{R}} \Omega (iy) 
    \hspace{0.17em} dy + \sum_{n \neq 0} a_n  \int_{\mathbb{R}} e^{- 2 \pi i
    \lambda_n y} \Omega (iy)  \hspace{0.17em} dy. \label{eq:residue_proof_4}
  \end{equation}
  Hence by Lemma~\ref{lem:theta_relation} and
  Theorem~\ref{thm:riemann_functional} the assertion of
  Theorem~\ref{thm:residue_formula} is proved.
\end{proof}

\begin{remark}
  \label{rem:theta_special}As an illustration of
  Theorem~\ref{thm:residue_formula}, take $\Omega (z) = \exp \{z^2 \pi \tau^{-
  1} \}$, $\mathrm{Re} \tau > 0$, and $a_n = b_n = 1$, $\lambda_n = \mu_n =
  n$. Then $F (z) = i \cot (i \pi z)$ and $\Omega \in A_{\eta}$. Hence
  \eqref{eq:residue_formula_eq} yields an important special case of
  \eqref{eq:theta_relation_eq}, i.e., the well-known linear transformation
  formula for the theta-function $\vartheta_3 (0| \tau)$.
\end{remark}

Now use Theorem~\ref{thm:residue_formula} to deduce a Poisson formula.

Denote by $A$ the class of all functions $g : \mathbb{R} \to \mathbb{C}$ with
the properties
\begin{equation}
  g \in L^1 (\mathbb{R}) \cap C (\mathbb{R}) \quad \text{with } \hat{g} \in
  L^1 (\mathbb{R}), \label{eq:class_A_prop_1}
\end{equation}
\begin{equation}
  \mathcal{L}_{zu}  \{g (au)\} \forall a > 0 \text{is defined in } S,
  \text{and satisfies } \eqref{eq:hankel_class_1} \text{and }
  \eqref{eq:hankel_class_2} . \label{eq:class_A_prop_2}
\end{equation}
\begin{theorem}
  \label{thm:poisson_type}Let $g \in A$. Then the Hankel summation formula
  \eqref{eq:hankel_summation_formula} implies the Poisson-type summation
  formula
  \begin{equation}
    \sqrt{2 \pi}  \sum_n a_n  \hat{g} (2 \pi \lambda_n a) = a^{- 1}  \sum_n
    b_n  \hat{g} (\mu_n \pi a^{- 1}) \forall a \in \mathbb{R}_+
    \label{eq:poisson_summation}
  \end{equation}
\end{theorem}

\begin{proof}
  By well-known facts from the theory of bilateral Laplace transforms (cf.
  {\cite{16}}), function
  \begin{equation}
    \Omega (z) \assign \mathcal{L}_{zu}  \{g (au)\} = a^{- 1} 
    \int_{\mathbb{R}} g (u) e^{- za^{- 1} u}  \hspace{0.17em} du
    \label{eq:poisson_proof_1}
  \end{equation}
  is analytic in its strip of convergence $S$, and the Riemann-Lebesgue lemma
  yields
  \begin{equation}
    \lim_{|y| \to \infty} \Omega (x + iy) = 0 \label{eq:riemann_lebesgue}
  \end{equation}
  uniformly for $|x| \leq \delta < \eta$. By \eqref{eq:riemann_lebesgue}, have
  \begin{equation}
    \Omega (iy) = a^{- 1}  \sqrt{2 \pi}  \hat{g} (ya^{- 1})
    \label{eq:poisson_proof_2}
  \end{equation}
  Hence by Theorem~\ref{thm:residue_formula} and Fourier's inversion formula
  \eqref{eq:fourier_inversion}:
  \begin{equation}
    \begin{array}{ll}
      \sum_n b_n  \hat{g} (\mu_n \pi a^{- 1}) & = \sum_n a_n e^{- 2 \pi i
      \lambda_n y}  \hat{g} (ya^{- 1})  \hspace{0.17em} dy\\
      & = a \sqrt{2 \pi}  \sum_n a_n  \hat{g} (2 \pi \lambda_n a)
    \end{array} \label{eq:poisson_proof_3}
  \end{equation}
  This proves \eqref{eq:poisson_summation}.
\end{proof}

\begin{remark}
  \label{rem:poisson_remarks}{\tmdummy}
  
  \begin{enumerate}
    \item The special case $a = 1$ of Theorem~\ref{thm:poisson_type} with the
    Hankel formula \eqref{eq:hankel_summation_formula} replaced by Riemann's
    functional equation \eqref{eq:chi_right}--\eqref{eq:chi_left} has been
    proved by means of the theory of almost periodic Schwarz distributions via
    Cauchy's theorem in {\cite{22}}.
    
    \item The special case of \eqref{eq:poisson_summation} for even functions
    $g$, $a = 1$ and $\lambda_n = \mu_n = n$ has been deduced from Riemann's
    functional equation \eqref{eq:chi_right}--\eqref{eq:chi_left} by Ferrar
    (cf. {\cite{1,18}}) and Patterson {\cite{26}}. Their methods of proof are
    mainly based upon Mellin's inversion formula and Cauchy's residue theorem.
  \end{enumerate}
\end{remark}

\section{Sampling theorems and Hankel transforms}\label{sec:sampling}

For $\sigma \geq 0$ and $1 \leq p < \infty$, let $B_{\sigma}^p$ be the class
of entire functions $f : \mathbb{C} \to \mathbb{C}$ of exponential type
$\sigma$, i.e.,
\begin{equation}
  |f (z) | < e^{\sigma |y|} \|f\|_C \forall z = x + iy \in \mathbb{C},
  \label{eq:exponential_type}
\end{equation}
which belong to $L^p (\mathbb{R})$ when restricted to $\mathbb{R}$. Have
\begin{equation}
  B_{\sigma}^p \subseteq B_{\sigma}^q \forall 1 \leq p \leq q \leq \infty
  \label{eq:bsp_inclusion}
\end{equation}
and the Paley-Wiener theorem states that a function $f \in L^p (\mathbb{R})$,
$1 < p < 2$, has an extension to $\mathbb{C}$ as an element of $B_{\sigma}^p$
if and only if $\hat{f} (\nu) = 0$ for $| \nu | \geq \sigma$. Hence in view of
\eqref{eq:bsp_inclusion}, say that $f$ is band-limited to $\pi W$ if $f \in
B_W^p$ for some $W > 0$ and $1 < p \leq \infty$ (cf. {\cite{8}}).

Now use Theorem~\ref{thm:poisson_type} to deduce from
\eqref{eq:hankel_summation_formula} the classical form of the
Whittaker-Shannon sampling theorem (cf. {\cite{8}}).

\begin{theorem}
  \label{thm:hankel_shannon}For some $W > 0$, $1 < p < \infty$, let $f \in
  B_W^p$ be such that \eqref{eq:class_A_prop_2} holds with $a = (2 \pi W)^{-
  1}$. Then the Hankel summation formula \eqref{eq:hankel_summation_formula}
  implies
  \begin{equation}
    f (t) = \sum_{n = - \infty}^{+ \infty} f \left( \frac{n}{W} \right)
    \mathrm{sinc} \{Wt - n\}, \quad t \in \mathbb{R},
    \label{eq:hankel_shannon_result}
  \end{equation}
  the series being absolutely and uniformly convergent.
\end{theorem}

\begin{proof}
  Trivially,
  \begin{equation}
    f (t) = \sum_{n = - \infty}^{+ \infty} f \left( \frac{n}{W} \right)
    \mathrm{sinc} (n), \quad t \in \mathbb{R}.
    \label{eq:hankel_shannon_proof_1}
  \end{equation}
  Now let $g \in B_W^p$ with \eqref{eq:class_A_prop_2}. Then the Paley-Wiener
  theorem yields $\hat{g} (\nu) = 0$ for $| \nu | \geq 2 \pi W$ and by
  Theorem~\ref{thm:poisson_type} with $a = b_n = 1$, $\lambda_n = \mu_n = n$,
  $n \in \mathbb{N}$:
  \begin{equation}
    \sqrt{2 \pi}  \sum_{n = - \infty}^{+ \infty} \hat{g} \left( \frac{n}{W}
    \right) = \hat{g} (0) = \frac{1}{\sqrt{2 \pi}}  \int_{\mathbb{R}} g (u) 
    \hspace{0.17em} du \label{eq:hankel_shannon_proof_2}
  \end{equation}
  Let $f_1, f_2 \in B_W^p$ and apply \eqref{eq:hankel_shannon_proof_2} to $g_1
  (u) \assign f_1 (u) f_2  (t - u) \in B_W^p$ with \eqref{eq:class_A_prop_2}.
  Then the convolution integral $f_1 \ast f_2$ can be replaced by the discrete
  version (cf. {\cite{8}}):
  \begin{equation}
    \begin{array}{ll}
      (f_1 \ast f_2) (t) & = \frac{1}{\sqrt{2 \pi}}  \int_{\mathbb{R}} f_1 (u)
      f_2  (t - u)  \hspace{0.17em} du\\
      & = \sum_{n = - \infty}^{+ \infty} f_1 \left( \frac{n}{W} \right) f_2 
      \left( t - \frac{n}{W} \right)
    \end{array} \label{eq:hankel_shannon_proof_3} \forall t \in \mathbb{R}
  \end{equation}
  Take $f_1 = f$ and $f_2 (\cdummy) = \mathrm{sinc} \{W \cdot\}$ in
  \eqref{eq:hankel_shannon_proof_3}. Then the commutativity of the integral
  $f_1 \ast f_2$ yields
  \begin{equation}
    \sum_{n = - \infty}^{+ \infty} f \left( \frac{n}{W} \right) \mathrm{sinc}
    \{Wt - n\} = f (t) \mathrm{sinc} (0) \label{eq:hankel_shannon_proof_4}
  \end{equation}
  and in view of \eqref{eq:hankel_shannon_proof_1} obtain
  \eqref{eq:hankel_shannon_result}.
  
  Finally, the convergence assertion follows from a theorem of Nikol'skii
  {\cite{25}}, which states that for any $h > 0$ and $f \in B_{\sigma}^p$, $1
  < p \leq \infty$:
  \begin{equation}
    \sup_{u \in \mathbb{R}} |f (u - hn) |^p < \sum_n \|f\|_p^p
    \label{eq:nikol_skii}
  \end{equation}
  Hence Theorem~\ref{thm:hankel_shannon} is proved.
\end{proof}

Now generalize Theorem~\ref{thm:hankel_shannon} by applying
Theorem~\ref{thm:poisson_type} to a special function and to the case of
Dirichlet series \eqref{eq:dirichlet_series} with $\lambda_n = \mu_n = n$.

\begin{theorem}
  \label{thm:riemann_hankel}For $t \in \mathbb{R}$ fixed and $W > 0$, let
  \begin{equation}
    g_1 (u) \assign f (u) \mathrm{sinc} \{W (t - u)\} \in A,
    \label{eq:riemann_hankel_setup}
  \end{equation}
  where
  \begin{equation}
    f \in L^2 (\mathbb{R}) \cap C (\mathbb{R}), \quad \text{with } \hat{f} \in
    L^1 (\mathbb{R}) . \label{eq:riemann_hankel_condition}
  \end{equation}
  Then the Hankel summation formula \eqref{eq:hankel_summation_formula}
  implies for $t \in \mathbb{R}$,
  \begin{equation}
    f (t) = \sum_{n = - \infty}^{+ \infty} a_n  \hat{f} \left( \frac{n}{W}
    \right) \mathrm{sinc} \{Wt - n\} + R (f, \phi, \psi, W) (t),
    \label{eq:riemann_hankel_result}
  \end{equation}
  where the remainder
  \begin{equation}
    R (f, \phi, \psi, W) (t) \assign \frac{1}{\sqrt{2 \pi}}  \sum_{n = -
    \infty}^{+ \infty} \{ 1 - b_n e^{- 2 \pi iny} \}  \int_{(2 n - 1) \pi
    W}^{(2 n + 1) \pi W} \hat{f} (v) e^{itv}  \hspace{0.17em} dv
    \label{eq:remainder_formula}
  \end{equation}
  is uniformly bounded in $t \in \mathbb{R}$.
\end{theorem}

\begin{proof}
  Need the generalized Parseval formula
  \begin{equation}
    \int_{\mathbb{R}} f_1 (u) \overline{f_2 (u)} \hspace{0.17em} du =
    \int_{\mathbb{R}} \hat{f}_1 (v) \overline{\hat{f}_2 (v)} \hspace{0.17em}
    dv, \quad f_1, f_2 \in L^2 (\mathbb{R}), \label{eq:parseval}
  \end{equation}
  where the bar indicates complex conjugates.
  
  Take $f_1 (u) = f (u) e^{- iyu}$, $y \in \mathbb{R}$, and $f_2 (u) =
  \mathrm{sinc} \{W (t - u)\}$. Since
  \begin{equation}
    \hat{f}_1 (v) = \hat{f}  (v + y) \label{eq:proof_f1_hat}
  \end{equation}
  and
  \begin{equation}
    \hat{f}_2 (v) = \left\{\begin{array}{ll}
      \frac{1}{\sqrt{2 \pi}} W^{- 1} e^{- itv}, & |v| < \pi W,\\
      0, & |v| > \pi W,
    \end{array}\right. \label{eq:proof_f2_hat}
  \end{equation}
  obtain from \eqref{eq:parseval},
  \begin{equation}
    \hat{f} (y) = e^{- iyt}  \frac{1}{\sqrt{2 \pi W}}  \int_{- \pi W + y}^{+
    \pi W + y} \hat{f} (v) e^{itv}  \hspace{0.17em} dv.
    \label{eq:proof_convolution}
  \end{equation}
  Directly show that $g_t \in L^1 (\mathbb{R}) \cap C (\mathbb{R})$ and
  $\hat{g}_t \in L^1 (\mathbb{R})$. Hence by Theorem~\ref{thm:poisson_type}
  with $a = (2 \pi W)^{- 1}$, $\lambda_n = \mu_n = n$ and by
  \eqref{eq:proof_convolution}:
  \begin{equation}
    \begin{array}{ll}
      \sum_{n = - \infty}^{+ \infty} a_n g_t \left( \frac{n}{W} \right) & = W
      \sqrt{2 \pi}  \sum_{n = - \infty}^{+ \infty} b_n  \hat{g}_t  (2 \pi
      Wn)\\
      & = \frac{1}{\sqrt{2 \pi}}  \sum_{n = - \infty}^{+ \infty} \int_{(2 n -
      1) \pi W}^{(2 n + 1) \pi W} \hat{f} (v) e^{itv}  \hspace{0.17em} dve^{-
      2 \pi inWt}
    \end{array} \label{eq:proof_poisson}
  \end{equation}
  Now split off the integral in Fourier's inversion formula
  \eqref{eq:fourier_inversion} in the form
  \begin{equation}
    f (t) = \frac{1}{\sqrt{2 \pi}}  \sum_{n = - \infty}^{+ \infty} \int_{(2 n
    - 1) \pi W}^{(2 n + 1) \pi W} \hat{f} (v) e^{itv}  \hspace{0.17em} dv
    \label{eq:proof_fourier_split}
  \end{equation}
  Hence subtraction of \eqref{eq:proof_poisson} from
  \eqref{eq:proof_fourier_split} leads to the required result
  \eqref{eq:riemann_hankel_result} with the remainder
  \eqref{eq:remainder_formula}. Finally, observe that by the Hamburger-Siegel
  theorem (cf. {\cite{19,29,30}}) the sequence $\{b_n \}$ is bounded. Hence
  there is an absolute constant $c \geq 2$ so that
  \begin{equation}
    |R (f, \phi, \psi, W) (t) | < \frac{c}{\sqrt{2 \pi}}  \sum_{n = -
    \infty}^{+ \infty} \int_{(2 n - 1) \pi W}^{(2 n + 1) \pi W} | \hat{f} (v)
    |  \hspace{0.17em} dv = c \| \hat{f} \|_1 \label{eq:remainder_bound}
  \end{equation}
  This proves Theorem~\ref{thm:riemann_hankel}.
\end{proof}

\begin{remark}
  \label{rem:sampling_remarks}{\tmdummy}
  
  \begin{enumerate}
    \item If $b_0 = \mathrm{res}_{s = 1} \zeta (s) = 1$, then with an absolute
    constant $c \geq 2$,
    \begin{equation}
      |R (f, \phi, \psi, W) (t) | < \frac{c}{\sqrt{2 \pi}}  \sum_{n \neq 0}
      \int_{(2 n - 1) \pi W}^{(2 n + 1) \pi W} | \hat{f} (v) | \hspace{0.17em}
      dv = \frac{c}{\sqrt{2 \pi}}  \int_{|v| > \pi W} | \hat{f} (v) | 
      \hspace{0.17em} dv \label{eq:remainder_zeta}
    \end{equation}
    Hence in this case, $\lim_{W \to \infty} R (f, \phi, \psi, W) (t) = 0$
    uniformly in $t \in \mathbb{R}$ and \eqref{eq:riemann_hankel_result}
    becomes
    \begin{equation}
      f (t) = \lim_{W \to \infty}  \sum_{n = - \infty}^{+ \infty} a_n  \hat{f}
      \left( \frac{n}{W} \right) \mathrm{sinc} \{Wt - n\}
      \label{eq:limit_form}
    \end{equation}
    uniformly in $t \in \mathbb{R}$.
    
    \item If, in addition, $f$ is band-limited to $[- \pi W, \pi W]$, i.e.,
    $\hat{f} (\nu) = 0$ for almost all $| \nu | > \pi W$, then
    \eqref{eq:limit_form} admits the form
    \begin{equation}
      f (t) = \sum_{n = - \infty}^{+ \infty} a_n  \hat{f} \left( \frac{n}{W}
      \right) \mathrm{sinc} \{Wt - n\} \forall t \in \mathbb{R}
      \label{eq:band_limited_form}
    \end{equation}
    Since the sequence $\{a_n \}$ is bounded, it follows from
    \eqref{eq:nikol_skii} that the series in \eqref{eq:band_limited_form}
    converges absolutely and uniformly.
    
    \item Consider the special case $a_n = b_n = 1$ of
    Theorem~\ref{thm:riemann_hankel}. By definition
    \eqref{eq:residue_definition} and well-known facts from the theory of the
    $\zeta$-function (cf. {\cite{30}}), have
    \begin{equation}
      a_0 = - 2 \zeta (0) = 1 \text{\quad and\quad} b_0 = \mathrm{res}_{s = 1}
      \zeta (s) = 1 \label{eq:zeta_values}
    \end{equation}
    Hence the special Hankel summation formula (cf. {\cite{17,24}})
    \begin{equation}
      \begin{array}{l}
        \frac{\pi}{\Gamma (\nu + 1)}  (T_{\nu + 1} f) (0) + 2 \sum_{n \geq 1}
        n^{- \nu}  (H_{\nu} f)  (\pi^2 n^2)\\
        = \frac{\pi^{\nu - 1 / 2}}{\Gamma (\nu + 1 / 2)}  \left[ (T_{1, \nu}
        f) (0) + 2 \sum_{n \geq 1} (T_{1, \nu} f) (n^2) \right]
      \end{array} \label{eq:special_hankel}
    \end{equation}
    valid for $\nu > \frac{1}{2}$, implies the classical forms
    \eqref{eq:whittaker_cardinal} and \eqref{eq:brown_butzer} of the sampling
    theorem.
  \end{enumerate}
\end{remark}

\begin{thebibliography}{99}
  {\bibitem{1}}R. Bellman, \tmtextit{A Brief Introduction to Theta Functions},
  Holt, Rinehart \& Winston, New York, 1961.
  
  {\bibitem{2}}B.C. Berndt, Identities involving the coefficients of a class
  of Dirichlet series I, \tmtextit{Trans. Amer. Math. Soc.} \tmtextbf{137}
  (1969) 345--359.
  
  {\bibitem{3}}S. Bochner, Some properties of modular relations,
  \tmtextit{Ann. of Math.} \tmtextbf{53} (1951) 332--363.
  
  {\bibitem{4}}S. Bochner and K. Chandrasekharan, On Riemann's functional
  equation, \tmtextit{Ann. of Math.} \tmtextbf{63} (1956) 336--359.
  
  {\bibitem{5}}P.L. Butzer, M. Hauss and R.L. Stens, The sampling theorem and
  its unique role in various branches of mathematics, \tmtextit{Mitt. Math.
  Ges. Hamburg} \tmtextbf{12}(3) (1991) 523--547.
  
  {\bibitem{6}}P.L. Butzer and R.J. Nessel, \tmtextit{Fourier Analysis and
  Approximation I}, Birkh{\"a}user, Basel, 1971.
  
  {\bibitem{7}}P.L. Butzer, S. Ries and R.L. Stens, Shannon's sampling
  theorem, Cauchy's integral formula and related results, in: P.L. Butzer,
  R.L. Stens and B.Sz. Nagy, Eds., \tmtextit{Anniversary Volume on
  Approximation Theory and Functional Analysis}, Internat. Ser. Numer. Math.
  \tmtextbf{65}, Birkh{\"a}user, Basel, 1984, pp. 363--377.
  
  {\bibitem{8}}P.L. Butzer, W. Splettst{\"o}{\ss}er and R.L. Stens, The
  sampling theorem and linear prediction in signal analysis,
  \tmtextit{Jahresber. Deutsch. Math.-Verein.} \tmtextbf{90} (1988) 1--70.
  
  {\bibitem{9}}P.L. Butzer and R.L. Stens, The Poisson summation formula,
  Whittaker's cardinal series and approximate integration, in: \tmtextit{Proc.
  Second Edmonton Conf. on Approximation Theory}, Canad. Math. Soc. Proc.
  \tmtextbf{3}, Amer. Mathematical Soc., Providence, RI, 1983, pp. 19--36.
  
  {\bibitem{10}}P.L. Butzer and R.L. Stens, The Euler-Maclaurin summation
  formula, the sampling theorem, and approximate integration over the real
  axis, \tmtextit{Linear Algebra Appl.} \tmtextbf{52/53} (1983) 141--155.
  
  {\bibitem{11}}K. Chandrasekharan and H. Joris, Dirichlet series with
  functional equations and related arithmetical identities, \tmtextit{Acta
  Arith.} \tmtextbf{24} (1973) 165--191.
  
  {\bibitem{12}}K. Chandrasekharan and S. Mandelbrojt, On Riemann's functional
  equation, \tmtextit{Ann. of Math.} \tmtextbf{66} (1957) 285--296.
  
  {\bibitem{13}}K. Chandrasekharan and R. Narasimhan, Hecke's functional
  equation and arithmetical identities, \tmtextit{Ann. of Math.} \tmtextbf{74}
  (1961) 1--23.
  
  {\bibitem{14}}K. Chandrasekharan and R. Narasimhan, Functional equations
  with multiple gamma factors and the average order of arithmetical functions,
  \tmtextit{Ann. of Math.} \tmtextbf{76} (1962) 93--136.
  
  {\bibitem{15}}G. Doetsch, Summatorische Eigenschaften der Besselschen
  Funktionen und andere Funktionalrelationen, die mit der linearen
  Transformationsformel der Thetafunktion {\"a}quivalent sind,
  \tmtextit{Compositio Math.} \tmtextbf{1} (1935) 85--97.
  
  {\bibitem{16}}G. Doetsch, \tmtextit{Handbuch der Laplace-Transformation},
  Vols. I--III, Birkh{\"a}user, Basel, 1955.
  
  {\bibitem{17}}A. Erdelyi, Gewisse Reihentransformationen, die mit der
  linearen Transformationsformel der Thetafunktion zusammenh{\"a}ngen,
  \tmtextit{Compositio Math.} \tmtextbf{4} (1937) 406--423.
  
  {\bibitem{18}}W.L. Ferrar, Summation formulae and their relation to
  Dirichlet Series II, \tmtextit{Compositio Math.} \tmtextbf{4} (1936--37)
  394--405.
  
  {\bibitem{19}}H. Hamburger, {\"U}ber einige Beziehungen, die mit der
  Funktionalgleichung der Riemannschen $\zeta$-Funktion {\"a}quivalent sind,
  \tmtextit{Math. Ann.} \tmtextbf{85} (1922) 129--140.
  
  {\bibitem{20}}E. Hecke, \tmtextit{Dirichlet Series, Modular Functions and
  Quadratic Forms}, Princeton Univ. Press, Princeton, NJ, 1938.
  
  {\bibitem{21}}J.R. Higgins, Five short stories about the cardinal series,
  \tmtextit{Bull. Amer. Math. Soc.} \tmtextbf{12} (1985) 45--89.
  
  {\bibitem{22}}J.P. Kahane and S. Mandelbrojt, L'{\'e}quation fonctionelle de
  Riemann et la formule sommatoire de Poisson, \tmtextit{Ann. Sci. {\'E}cole
  Norm. Sup.} \tmtextbf{75} (1958) 57--80.
  
  {\bibitem{23}}D. Klusch, The sampling theorem, Dirichlet series and Bessel
  functions, \tmtextit{Math. Nachr.} \tmtextbf{154} (1991) 129--139.
  
  {\bibitem{24}}D. Klusch, On the summation of Bessel functions and Hankel
  transforms, \tmtextit{J. Math. Anal. Appl.} \tmtextbf{160} (1991) 303--313.
  
  {\bibitem{25}}S.M. Nikol'skii, \tmtextit{Approximation of Functions of
  Several Variables and Imbedding Theorems}, Springer, Berlin, 1975.
  
  {\bibitem{26}}S.J. Patterson, \tmtextit{An Introduction to the Theory of the
  Riemann Zeta-function}, Cambridge Univ. Press, Cambridge, 1989.
  
  {\bibitem{27}}B. Riemann, {\"U}ber die Anzahl der Primzahlen unter einer
  gegebenen Gr{\"o}sse, \tmtextit{Monatsber. Akad. Berlin} (1859) 671--680.
  
  {\bibitem{28}}C.E. Shannon, Communication in the presence of noise,
  \tmtextit{Proc. IRE} \tmtextbf{37} (1949) 10--21.
  
  {\bibitem{29}}C.L. Siegel, Bemerkung zu einem Satz von Hamburger {\"u}ber
  die Funktionalgleichung der Riemannschen Zetafunktion, \tmtextit{Math. Ann.}
  \tmtextbf{86} (1922) 276--279.
  
  {\bibitem{30}}E.C. Titchmarsh, \tmtextit{The Theory of the Riemann
  Zeta-function}, Oxford Univ. Press, Oxford, 1967.
  
  {\bibitem{31}}G.N. Watson, \tmtextit{A Treatise on the Theory of Bessel
  Functions}, Cambridge Univ. Press, Cambridge, 1922.
  
  {\bibitem{32}}E.T. Whittaker, On the functions which are represented by the
  expansion of the interpolation-theory, \tmtextit{Proc. Roy. Soc. Edinburgh}
  \tmtextbf{35} (1915) 181--194.
  
  {\bibitem{33}}E.T. Whittaker and G.N. Watson, \tmtextit{A Course of Modern
  Analysis}, Cambridge Univ. Press, Cambridge, 4th ed., 1973.
\end{thebibliography}

\end{document}
