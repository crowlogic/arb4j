\documentclass{article}
\usepackage[english]{babel}
\usepackage{geometry,amsmath,amssymb}
\geometry{letterpaper}

%%%%%%%%%% Start TeXmacs macros
\catcode`\<=\active \def<{
\fontencoding{T1}\selectfont\symbol{60}\fontencoding{\encodingdefault}}
\catcode`\>=\active \def>{
\fontencoding{T1}\selectfont\symbol{62}\fontencoding{\encodingdefault}}
\newcommand{\nequiv}{\mathrel{\not\equiv}}
\newcommand{\tmaffiliation}[1]{\\ #1}
\newcommand{\tmem}[1]{{\em #1\/}}
\newcommand{\tmemail}[1]{\\ \textit{Email:} \texttt{#1}}
\newcommand{\tmnote}[1]{\thanks{\textit{Note:} #1}}
\newcommand{\tmrsub}[1]{\ensuremath{_{\textrm{#1}}}}
\newcommand{\tmrsup}[1]{\textsuperscript{#1}}
\newcommand{\tmtextit}[1]{\text{{\itshape{#1}}}}
\newtheorem{corollary}{Corollary}
%%%%%%%%%% End TeXmacs macros

\newtheorem{tm}{Theorem}
\newtheorem{exmp}{Example}
\newtheorem{exmps}{Examples}
\newtheorem{alg}{Algorithm}
\newtheorem{rems}{Remarks}
{\newenvironmentstar{mylist}{1}{ }}
\newcommand{\up}{uncertainty principle}
\newcommand{\tfa}{time-frequency analysis}
\newcommand{\tfr}{time-frequency representation}
\newcommand{\ft}{Fourier transform}
\newcommand{\stft}{short-time Fourier transform}
\newcommand{\qm}{quantum mechanics}
\newcommand{\tf}{time-frequency}
\newcommand{\fa}{Fourier analysis}
\newcommand{\fct}{function}
\newcommand{\fif}{if and only if}
\newcommand{\tfs}{time-frequency shift}
\newcommand{\fs}{Fourier series}
\newcommand{\on}{orthonormal}
\newcommand{\onb}{orthonormal basis}
\newcommand{\fco}{Fourier coefficient}
\newcommand{\sa}{signal analysis}
\newcommand{\psf}{Poisson summation formula}
\newcommand{\bdl}{bandlimited}
\newcommand{\prob}{probability}
\newcommand{\Wd}{Wigner distribution}
\newcommand{\gf}{Gabor frame}
\newcommand{\frop}{frame operator}
\newcommand{\gfrop}{Gabor frame operator}
\newcommand{\coeff}{coefficient}
\newcommand{\repr}{representation}
\newcommand{\rb}{Riesz basis}
\newcommand{\mdl}{modulation}
\newcommand{\distr}{distribution}
\newcommand{\modsp}{modulation space}
\newcommand{\psdo}{pseudodifferential operator}
\newcommand{\hei}{Heisenberg group}
\newcommand{\rep}{representation}
\newcommand{\ha}{harmonic analysis}
\newcommand{\lcg}{locally compact group}
\newcommand{\lcal}{locally compact abelian}
\newcommand{\srep}{Schr{\"o}dinger representation}
\newcommand{\svn}{Stone--von Neumann theorem}
\newcommand{\iur}{irreducible unitary representation}
\newcommand{\gs}{Gabor system}
\newcommand{\zt}{Zak transform}
\newcommand{\blt}{Balian--Low theorem}
\newcommand{\wb}{Wilson bas}
\newcommand{\ws}{Wilson system}
\newcommand{\wt}{wavelet transform}
\newcommand{\wth}{wavelet theory}
\newcommand{\gex}{Gabor expansion}
\newcommand{\wl}{Wiener's lemma}
\newcommand{\opal}{operator algebra}
\newcommand{\trpo}{trigonometric polynomial}
\providecommand{\op}{operator}
\newcommand{\pde}{partial differential equation}
\newcommand{\knc}{Kohn--Nirenberg correspondence}
\newcommand{\fre}{frequency}
\newtheorem{tm}{Theorem}
\newtheorem{exmp}{Example}
\newtheorem{exmps}{Examples}
\newtheorem{alg}{Algorithm}
\newtheorem{rems}{Remarks}
\newcommand{\beqa}{}
\newcommand{\eeqa}{}
\providecommand{\*}{{\opl}}
\newcommand{\field}[1]{#1}
\newcommand{\bR}{\field{R}}
\newcommand{\bN}{\field{N}}
\newcommand{\bZ}{\field{Z}}
\newcommand{\bC}{\field{C}}
\newcommand{\bQ}{{\field{Q}}}
\newcommand{\bT}{{\field{T}}}
\newcommand{\bH}{{\field{H}}}
\newcommand{\mylistlabel}[1]{#1{\hspace*{\fill}}}
{\newenvironmentstar{mylist}{1}{ }}
\newcommand{\Hr}{{\bH}{\hspace{0.17em}} \tmrsub{r}}
\newcommand{\bh}{\mathbf{h}}
\newcommand{\arro}{\ensuremath{\Rightarrow}}
\newcommand{\al}{{\alpha}}
\newcommand{\be}{{\beta}}
\newcommand{\ga}{{\gamma}}
\newcommand{\Ga}{{\Gamma}}
\newcommand{\de}{{\delta}}
\newcommand{\De}{{\Delta}}
\newcommand{\la}{{\lambda}}
\newcommand{\La}{{\Lambda}}
\newcommand{\vp}{{\varphi}}
\newcommand{\ve}{{\varepsilon}}
\newcommand{\e}{{\epsilon}}
\newcommand{\ka}{{\kappa}}
\newcommand{\om}{{\omega}}
\newcommand{\Om}{{\Omega}}
\newcommand{\si}{{\sigma}}
\newcommand{\Si}{{\Sigma}}
\newcommand{\cF}{\ensuremath{\mathcal{F}}}
\newcommand{\cS}{\ensuremath{\mathcal{S}}}
\newcommand{\cD}{\ensuremath{\mathcal{D}}}
\newcommand{\cH}{\ensuremath{\mathcal{H}}}
\newcommand{\cB}{\ensuremath{\mathcal{B}}}
\newcommand{\cG}{\ensuremath{\mathcal{G}}}
\newcommand{\cM}{\ensuremath{\mathcal{M}}}
\newcommand{\cK}{\ensuremath{\mathcal{K}}}
\newcommand{\cU}{\ensuremath{\mathcal{U}}}
\newcommand{\cA}{\ensuremath{\mathcal{A}}}
\newcommand{\cE}{\ensuremath{\mathcal{E}}}
\newcommand{\cJ}{\ensuremath{\mathcal{J}}}
\newcommand{\cI}{\ensuremath{\mathcal{I}}}
\newcommand{\cC}{\ensuremath{\mathcal{C}}}
\newcommand{\cW}{\ensuremath{\mathcal{W}}}
\newcommand{\cO}{\mathcal{O}}
\newcommand{\cN}{\ensuremath{\mathcal{N}}}
\newcommand{\cP}{\ensuremath{\mathcal{P}}}
\newcommand{\cR}{\ensuremath{\mathcal{R}}}
\newcommand{\cT}{\ensuremath{\mathcal{T}}}
\newcommand{\cX}{\ensuremath{\mathcal{X}}}
\newcommand{\cZ}{\ensuremath{\mathcal{Z}}}
\newcommand{\tma}{{\aleph}}
\newcommand{\fhat}{\^{f}}
\newcommand{\hg}{\^{g}}
\newcommand{\bg}{\={g}}
\newcommand{\hvp}{\^{{\vp}}}
\newcommand{\wh}[1]{\^{#1}}
\newcommand{\tW}{\~{W}}
\newcommand{\tif}{\~{f}}
\newcommand{\tig}{\~{g}}
\newcommand{\tih}{\~{h}}
\newcommand{\tz}{\~{{\zeta}}}
\newcommand{\Lac}{{\Lambda}\tmrsup{{\circ}}}
\newcommand{\vgf}{V\tmrsub{g}f}
\newcommand{\cano}{\tmrsup{{\circ}}}
\newcommand{\rd}{{\bR}\tmrsup{d}}
\newcommand{\cd}{{\bC}\tmrsup{d}}
\newcommand{\rdd}{{\bR}\tmrsup{2d}}
\newcommand{\zdd}{{\bZ}\tmrsup{2d}}
\newcommand{\ellzd}{{\ell}\tmrsup{1}({\zd})}
\newcommand{\lr}{L\tmrsup{2}({\bR})}
\newcommand{\lrd}{L\tmrsup{2}({\rd})}
\newcommand{\lrdd}{L\tmrsup{2}({\rdd})}
\newcommand{\zd}{{\bZ}\tmrsup{d}}
\newcommand{\td}{{\bT}\tmrsup{d}}
\newcommand{\mvv}{M\tmrsub{v}\tmrsup{1}}
\newcommand{\intrd}{\int\tmrsub{{\rd}}}
\newcommand{\intrdd}{\int\tmrsub{{\rdd}}}
\newcommand{\intcd}{\int\tmrsub{{\cd}}}
\newcommand{\infint}{\int\tmrsub{-{\infty}}\tmrsup{{\infty}}}
\newcommand{\dint}{\int{\hspace{-0.17em}}{\hspace{-0.17em}}{\hspace{-0.17em}}\int}
\newcommand{\tmL}{(}
\newcommand{\R}{)}
\newcommand{\tml}{{\langle}}
\newcommand{\tmr}{{\rangle}}
\newcommand{\<}{}
\newcommand{\>}{}
\newcommand{\ra}{{\rightarrow}}
\newcommand{\lra}{{\longrightarrow}}
\newcommand{\Lra}{{\Longrightarrow}}
\newcommand{\Llra}{{\Longleftrightarrow}}
\newcommand{\lms}{{\longmapsto}}
\newcommand{\sset}{{\subset}}
\newcommand{\sseq}{{\subseteq}}
\newcommand{\IN}{{\hspace{-0.17em}}{\in}{\hspace{-0.17em}}}
\newcommand{\hats}{{\hspace{-0.17em}}\^{}}
\newcommand{\prt}{{\partial}}
\newcommand{\absl}{}
\newcommand{\absr}{}
\newcommand{\tamb}{T\tmrsub{{\al}k}M\tmrsub{{\be}n}}
\newcommand{\mbta}{M\tmrsub{{\be}n}T\tmrsub{{\al}k}}
\newcommand{\dsum}{\underset{k,n {\in}{\zd}}{\sum{\hspace{-0.17em}}\sum}}
\newcommand{\lnsum}{\underset{l,n {\in}{\zd}}{\sum{\hspace{-0.17em}}\sum}}
\newcommand{\invbe}{{\frac{1}{{\be}}}}
\newcommand{\inval}{{\frac{1}{{\al}}}}
\newcommand{\inv}{^{- 1}}
\newcommand{\ud}{{\hspace{0.17em}} \mathrm{d}}
\newcommand{\bks}{{\backslash}}
\newcommand{\nat}{{\hspace{0.17em}}{\natural}{\hspace{0.17em}}}
\newcommand{\intqa}{\int\tmrsub{Q\tmrsub{1/{\al}}}}
\newcommand{\mv}{M\tmrsub{v}\tmrsup{1}}
\newcommand{\alp}{{\tfrac{{\alpha}}{p}}{\hspace{0.17em}}}
\newcommand{\zaa}{Z\tmrsub{{\alpha}}}
\newcommand{\Lmpq}{L\tmrsub{m}\tmrsup{p,q}}
\newcommand{\lmpq}{{\ell}\tmrsub{\~{m}}\tmrsup{p,q}}
\newcommand{\ltmpq}{{\ell}\tmrsub{\~{m}}\tmrsup{p,q}}
\newcommand{\Mmpq}{M\tmrsub{m}\tmrsup{p,q}}
\newcommand{\phas}{(x,{\omega})}
\newcommand{\Hpol}{{\bH}{\hspace{0.17em}} \tmrsup{pol}}
\newcommand{\sgg}{S\tmrsub{g,g}}
\providecommand{\qed}{{\hspace*{\fill}}{\rule{7pt}{8pt}} .2truein}
\newcommand{\abs}[1]{{\lvert}#1{\rvert}}
\newcommand{\norm}[1]{{\lVert}#1{\rVert}}
\newcommand{\tp}{totally positive}
\newcommand{\lpc}{Laguerre-Polya class}
\newcommand{\pff}{Polya frequency function}

\begin{document}

\title{Schoenberg's Theory of Totally Positive Functions and the Riemann Zeta
Function}

\author{
  Karlheinz Gr{\"o}chenig
  \tmnote{K. G. was supported in part by the project P31887-N32 of the
  Austrian Science Fund (FWF)}
  \tmaffiliation{Faculty of Mathematics\\
  University of Vienna\\
  Oskar-Morgenstern-Platz 1\\
  A-1090 Vienna, Austria}
  \tmemail{karlheinz.groechenig@univie.ac.at}
}

\date{}

\maketitle

\begin{abstract}
  We review Schoenberg's characterization of totally positive functions and
  its connection to the Laguerre-Polya class. This characterization yields a
  new condition that is equivalent to the truth of the Riemann hypothesis.
\end{abstract}

In a series of papers in the 1950s Schoenberg investigated the properties of
{\tp} functions~{\cite{CS66,sch47,Sch50,sch51,SW53}}. He found several
characterizations and used total positivity to prove fundamental properties of
splines~{\cite{SW53,Sch73}}. The purpose of this note is to survey some
aspects of Schoenberg's work on {\tp} functions, to advertize the connection
between {\tp} functions and the Riemann hypothesis, and to provide some
mathematical entertainment.

One may speculate whether Schoenberg himself thought about the Riemann zeta
function. He was the son-in-law of the eminent number theorist Edmund Landau,
he collaborated with Polya, he knew deeply the work of Polya and Schur about
the {\lpc} of entire functions that remains influential in the study of the
Riemann hypothesis. Yet, to my knowledge, he never mentioned any number theory
in his work on {\tp} functions and splines; by the same token, Schoenberg's
name is not mentioned in analytic number theory.

{\tmem{Totally positive functions.}} A measurable function $\Lambda$ on $\bR$
is {\tp}, if for every $n \in \bN$ and every two sets of increasing numbers
$x_1 < x_2 < \ldots < x_n$ and $y_1 < y_2 < \ldots < y_n$ the matrix $(\Lambda
(x_j - y_k))_{j, k = 1, \ldots, n}$ has non-negative determinant:
\begin{equation}
  \label{eq:1} \det (\Lambda (x_j - y_k))_{j, k = 1, \ldots, n} \geq 0
\end{equation}
If in addition $\Lambda$ is integrable, then $\Lambda$ is called a Polya
frequency function.

If $\Lambda$ is {\tp} and not equal to $e^{ax + b}$, there exist an
exponential $e^{cx}$, such that $\Lambda_1 (x) = e^{cx} \Lambda (x)$ is a
{\pff}, i.e., $\Lambda_1$ is {\tp} and integrable~{\cite[Lemma~4]{sch51}}. It
is usually no loss of generality to restrict to {\pff}s.

The class of {\tp} functions played and plays an important role in
approximation theory, in particular in spline theory~{\cite{SW53}}, and in
statistics~{\cite{Efr65,karlin}}. In a different and rather surprising
direction, {\tp} functions appear in the representation theory of infinite
dimensional motion groups~{\cite{Pick91}}. Recently, {\tp}

functions appeared in sampling theory and in {\tfa} {\cite{GS13,GRS18,GRS20}},
where they were instrumental in the derivation of optimal results.

{\tmem{The Laguerre-Polya class.}} An entire function $\Psi$ of order at most
$2$ belongs to the Laguerre-Polya class, if its Hadamard factorization is of
the form
\begin{equation}
  \label{eq:2} \Psi (s) = Cs^m e^{- \gamma s^2 + \delta s}  \hspace{0.17em}
  \prod_{j = 1}^{\infty} (1 + \delta_j s) e^{- \delta_j s}  \qquad s \in \bC
  \hspace{0.17em}
\end{equation}
where $\delta_j \inv \in \bR$ are the zeros of $\Psi$, $m$ is the order of the
zero at $0$, $\gamma \geq 0$, $\delta \in \bR$, and
\begin{equation}
  \label{eq:3} 0 < \gamma + \sum_{j = 1}^{\infty} \delta_j^2 < \infty
  \hspace{0.17em}
\end{equation}
Thus the {\lpc} consists of entire functions of order two with convergence
exponent two with only real zeros. While the study of the distribution of
zeros of entire functions is a perennial topic in complex analysis and of
interest in its own right~{\cite{Levin80}}, the {\lpc} has gained special
prominence in analytic number theory: the Riemann hypothesis says that a
relative of the Riemann zeta function belongs to the {\lpc}.

{\tmem{Schoenberg's characterization of {\tp} functions.}} The fundamental
results about {\tp} functions were derived by Schoenberg in a series of
papers~{\cite{sch47,Sch50,sch51,SW53}}. A comprehensive treatment is contained
in Karlin's monograph~{\cite[Ch.~7]{karlin}}.

The notions of {\tp} functions and {\lpc} are seemingly unrelated, yet there
is a deep connection between them through the following characterization of
Schoenberg~{\cite{sch51}}.

\begin{tm}
  \label{tm:tp}(i) If $\Lambda$ is a Polya frequency function, then its
  (two-sided) Laplace transform converges in a vertical strip $\{z \in \bC :
  \alpha < \mathrm{Re} \hspace{0.17em} z < \beta\}, \alpha < 0 < \beta$, and
  \begin{equation}
    \label{eq:5} \int_{- \infty}^{\infty} \Lambda (x) e^{- sx} 
    \hspace{0.17em} dx = \frac{1}{\Psi (s)}
  \end{equation}
  is the reciprocal of a function $\Psi$ in the {\lpc} with $\Psi (0) > 0$.
  
  (ii) Conversely, if $\Psi$ is in the {\lpc} with $\Psi (0) > 0$, then its
  reciprocal $1 / \Psi$ is the Laplace transform of a Polya frequency function
  $\Lambda$.
\end{tm}

This is a fascinating theorem, because it relates two function classes that
seem to bear absolutely no resemblance to each other. Schoenberg's theorem
establishes a bijection between the class of {\pff}s, the {\lpc}, and yields a
parametrization by the set $(0, \infty) \times \bR \times \ell^2 (\bZ)$.

By using the Fourier transform instead of the Laplace transform, Schoenberg's
theorem can be recast as follows: A function $\Lambda$ is {\tp} and
integrable, {\fif} its Fourier transform possesses the factorization
\begin{equation}
  \label{eq:6} \hat{\Lambda} (\tau) = Ce^{- \gamma \tau^2 + 2 \pi i \delta
  \tau}  \hspace{0.17em} \prod_{j = 1}^{\infty} (1 + 2 \pi i \delta_j \tau)
  \inv e^{2 \pi i \delta_j \tau}
\end{equation}
where $C > 0$, $\gamma \geq 0$, $\delta, \delta_j \in \bR$ and $\sum_{j =
1}^{\infty} \delta_j^2 < \infty$ (and the product in \eqref{eq:6} may also be
finite).

If we drop the condition of integrability and exclude exponential functions,
then the representation \eqref{eq:6} still holds for every {\tp} function, but
their Laplace transform of $\Lambda$ converges in some vertical strip $\{z \in
\bC : \alpha < \mathrm{Re} \hspace{0.17em} z < \beta\}$ that does not contain
$0$.

A similar result holds for one-sided {\tp} functions~{\cite[Thm.~2]{sch51}}.

\begin{tm}
  \label{tm:onesided}(i) If $\Lambda$ is a Polya frequency function with
  support in $[0, \infty)$, then its Laplace transform converges in a
  half-plane $\{z \in \bC : - \alpha < \mathrm{Re} \hspace{0.17em} z\}, \alpha
  > 0$, and
  \begin{equation}
    \label{eq:5} \int_0^{\infty} \Lambda (x) e^{- sx}  \hspace{0.17em} dx =
    \frac{1}{\Psi (s)}
  \end{equation}
  is the reciprocal of an entire function $\Psi$ with Hadamard factorization
  \begin{equation}
    \label{eq:ne1} \Psi (s) = Ce^{\delta s}  \prod_{j = 1}^{\infty} (1 +
    \delta_j s) \hspace{0.17em},
  \end{equation}
  with $\delta \in \bR, \delta_j \geq 0, \sum \delta_j < \infty$.
  
  (ii) Conversely, if $\Psi$ possesses the factorization \eqref{eq:ne1}, then
  its reciprocal $1 / \Psi$ is the Laplace transform of a Polya frequency
  function $\Lambda$ with support in $[0, \infty)$.
\end{tm}

{\vspace{3mm}}

{\tmem{Elementary examples.}} If $\hat{\Lambda} (\tau) = (1 + 2 \pi i \delta
\tau) \inv$, then $\Lambda (x) = \delta \inv e^{- x / \delta} \chi_{[0,
\infty)} (x)$ is the one-sided exponential function. For $\hat{\Lambda} (\tau)
= e^{- \pi \gamma \tau^2}$ for $\gamma > 0$, we obtain the Gaussian $\Lambda
(x) = \gamma^{- 1 / 2} e^{- \pi x^2 / \gamma}$. In both cases, it is easy to
check directly that these functions are {\tp}.

The proof of the implication (ii) of Theorem~\ref{tm:tp} is based on the
(non-trivial) fact that the convolution $\Lambda = \Lambda_1 \ast \Lambda_2$
of two {\pff}s $\Lambda_1, \Lambda_2$ is again a {\pff}. The converse in
Theorem~\ref{tm:tp} lies much deeper, and Schoenberg used heavily several
results of Polya about functions in the {\lpc}~{\cite{Pol15,PS1914}}. See the
end of this note for the essential step of the argument.

Schoenberg's motivation was the characterization and deeper understanding of
{\tp} functions, and thus the implication (i) and the factorization
\eqref{eq:6} can be considered his main insight about {\tp} functions.
However, instead of reading Schoenberg's theorem as a characterization of
{\tp} functions, one may read it as a characterization of the {\lpc}. {\tmem{A
function $\Psi$ with $\Psi (0) > 0$ and $\Psi \neq e^{as + b}$ is in the
{\lpc}, {\fif} the Fourier transform of $1 / \Psi$ is a {\pff}.}}

{\tmem{The Riemann hypothesis and {\tp} functions.}} Let $\zeta (s) = \sum_{n
= 1}^{\infty} n^{- s}$ for $s \in \bC, \mathrm{Re} \hspace{0.17em} s > 1$, be
the Riemann zeta function and let

\begin{align}
  \xi (s) & = \tfrac{1}{2} s (s - 1) \pi^{- s / 2} \Gamma \left( \frac{s}{2}
  \right)  \hspace{0.17em} \zeta (s)  \label{eq:6b}\\
  \Xi (s) & = \xi \left( \frac{1}{2} + is \right)  \label{eq:7}
\end{align}

be the Riemann xi-functions (where $\Gamma$ is the usual gamma function). Then
the functional equation for the Riemann zeta function is expressed by the
symmetry
\begin{equation}
  \label{eq:8} \xi (s) = \xi (1 - s)  \qquad \text{and } \qquad \Xi (s) = \Xi
  (- s)
\end{equation}
for the xi-functions.

The Riemann hypothesis conjectures that all non-trivial zeros of the zeta
function lie on the critical line $1 / 2 + it$. See the
monographs~{\cite{Iwa14,Ivi03,Tit86}}, the two volumes about equivalents of
the Riemann hypothesis~{\cite{Br17-1,Br17-2}} or the survey
articles~{\cite{Bom10,Conrey03}}.

Expressed in terms of the xi-functions, the Riemann hypothesis states that
$\Xi$ has only real zeros, in other words, {\tmem{$\Xi$ belongs to the
{\lpc}.}} Thus many investigations of the zeta function involve complex
analysis related to the {\lpc}. Schoenberg's theorem immediately leads to the
following equivalent condition for the Riemann hypothesis to hold.

\begin{tm}
  \label{tm:equi1}The Riemann hypothesis holds, {\fif} there exists a {\pff}
  $\Lambda$, such that
  \begin{equation}
    \label{eq:9} \frac{1}{\Xi (s)} = \int_{- \infty}^{\infty} \Lambda (x) e^{-
    sx}  \hspace{0.17em} dx \qquad \forall s \in \bC, | \mathrm{Re}
    \hspace{0.17em} s| < t_0 \hspace{0.17em}
  \end{equation}
  where $1 / 2 + it_0$ is the first zero of the zeta function on the critical
  line.
\end{tm}

Let us make this statement a bit more explicit by taking the Fourier transform
instead of the Laplace transform.

\begin{tm}
  \label{tm:equi2}The Riemann hypothesis holds, {\fif}
  \begin{equation}
    \label{eq:10} \Lambda (x) = \frac{\int_{- \infty}^{\infty} \frac{e^{- ix
    \tau}}{\xi (\tfrac{1}{2} + \tau)}  \hspace{0.17em} d \tau}{2 \pi} 
  \end{equation}
  is a Polya frequency function.
\end{tm}

The growth of $\xi$ in the complex plane is~{\cite{Tit86}} $| \xi (s) | = \cO
(e^{A|s| \ln |s|}) \hspace{0.17em}$, and on the positive real line
\begin{equation}
  \ln \xi (\sigma) \asymp \tfrac{1}{2} \sigma \log \sigma \qquad \sigma > 1
\end{equation}
Consequently $\frac{1}{\xi (\sigma)} \leq Ce^{- | \sigma | \log | \sigma | /
2}$ decays super-exponentially. Since $\zeta$ and thus $\xi$ do not have any
real zeros in the interval $[0, 1]$ and $\zeta > 0$ on $(1, \infty)$, the
function $\xi$ is therefore strictly positive on $\bR$ and $1 / \xi$ is
integrable. Thus its Fourier transform is well-defined.

Using $s = 2 \pi i \tau$, we can rewrite \eqref{eq:9} as a Fourier transform.
The inversion formula for the Fourier transform now yields
\begin{equation}
  \begin{array}{ll}
    \Lambda (x) & = \int_{- \infty}^{\infty} \frac{e^{2 \pi ix \tau}}{\Xi (2
    \pi i \tau)}  \hspace{0.17em} d \tau\\
    & = \int_{- \infty}^{\infty} \frac{e^{2 \pi ix \tau}}{\xi \left(
    \frac{1}{2} - 2 \pi \tau \right)}  \hspace{0.17em} d \tau
    \hspace{0.17em}\\
    & = \frac{\int_{- \infty}^{\infty} \frac{e^{- ix \tau}}{\xi (\tfrac{1}{2}
    + \tau)}  \hspace{0.17em} d \tau}{2 \pi}
  \end{array}
\end{equation}
which is \eqref{eq:10}.

Using the symmetry of $\Xi$, there is an alternative formulation of
Theorem~\ref{tm:equi1} with the restricted {\lpc} defined in \eqref{eq:ne1}.
Since $\Xi$ is symmetric, it can be written as $\Xi (s) = \Xi_1  (- s^2)$ for
an entire function $\Xi_1$ of order $1 / 2$. Furthermore, $\Xi$ has only real
zeros, {\fif} $\Xi_1$ has only negative zeros (with convergence exponent at
most $1$). The characterization of one-sided {\pff}s yields the following
equivalence.

\begin{tm}
  \label{tm:ones}The Riemann hypothesis holds, {\fif} there exists a {\pff}
  $\Lambda$ with support in $[0, \infty)$, such that
  \begin{equation}
    \label{eq:9} \frac{1}{\Xi_1 (s)} = \int_0^{\infty} \Lambda (x) e^{- sx} 
    \hspace{0.17em} dx \qquad \text{{\forall}} s \in \bC, \mathrm{Re}
    \hspace{0.17em} s > \alpha
  \end{equation}
  for some $\alpha < 0$.
\end{tm}

These equivalences seem to be new. Schoenberg's name is not even mentioned in
~{\cite{Br17-1,Br17-2}} on equivalents of the Riemann hypothesis.

It is interesting that the characterization of Theorem~\ref{tm:equi2} is
``orthogonal'' to most research on $\zeta$ and to the well-known criteria for
the Riemann hypothesis. Theorem~\ref{tm:equi2} requires only {\tmem{the values
of $\zeta$ on the real line}} to probe the secrets of $\zeta$ in the critical
strip. This fact is remarkable, but the price to pay is the added difficulty
to extract any meaningful statements about $\xi$ on the critical strip from
its restriction to $\bR$. This seems much harder, if not impossible.

To work with Theorem~\ref{tm:equi2}, one would need a viable expression for
the Fourier-Laplace transform of $1 / \xi$, but there seems to be none. The
$1$-positivity in \eqref{eq:1} says that $\Lambda \geq 0$, which is equivalent
to the Fourier transform $\hat{\Lambda} = 1 / \xi$ to be positive definite by
Bochner's theorem. Explicitly, we would need to know that, for all choices of
$c_j \in \bC, \tau_j \in \bR, j = 1, \ldots, n$, and all $n \in \bN$, we have
$\sum_{j, k = 1}^n c_j \overline{c_k} \xi (\tfrac{1}{2} + \tau_j - \tau_k)
\inv \geq 0$. Not even this property of $1 / \xi$ seems to be known. It is
therefore unlikely that much is gained by Theorems~\ref{tm:equi1} --
\ref{tm:ones}.

By contrast, the Fourier transform of $\Xi (x)$ on the critical line (!) was
already known to Riemann (see~{\cite[2.16.1]{Tit86}}) and is the starting
point of a program to prove the Riemann hypothesis that goes back to
Polya~{\cite{Pol26}}. After important work of de Bruijn, Hejhal, and Newman
this line of thought has recently culminated in the resolution of the Newman
conjecture by Rodgers and Tao~{\cite{RT20}}.

{\tmem{Some non-trivial {\pff}s.}} Perhaps Schoenberg had also the Riemann
hypothesis in mind, when he investigated Polya frequency functions. The
examples in~{\cite{sch47,sch51}} of {\tp} functions smell of the zeta
function.

(i) The zero set $\{0, - 1, - 2, \ldots\}$ with multiplicity one yields the
entire function
\begin{equation}
  \label{eq:n1} \Psi (s) = e^{\gamma s} s \prod_{n = 1}^{\infty} (1 +
  \frac{s}{n})  \hspace{0.17em} e^{- s / n} \hspace{0.17em}
\end{equation}
where $\gamma$ is the Euler constant. By a classical result $\Psi$ is the
reciprocal of the $\Gamma$-function $\Gamma (s) = \int_0^{\infty} x^{s - 1}
e^{- x}  \hspace{0.17em} dx$. Consequently, the Laplace transform of $\Psi (s)
\inv = \Gamma (s)$ is a {\tp} function. Indeed, using the substitution $x =
e^{- t}$ in the definition of $\Gamma$, one obtains
\begin{equation}
  \label{eq:n2} \Gamma (s) = \int_{- \infty}^{\infty} e^{- e^{- x}} e^{- sx} 
  \hspace{0.17em} dx \hspace{0.17em} \qquad \mathrm{Re} \hspace{0.17em} s > 0
  \hspace{0.17em}
\end{equation}
Theorem~\ref{tm:tp} implies that
\[ \Lambda (x) = e^{- e^{- x}} \]
is {\tp}. By removing the pole of $\Gamma$ at $0$, we obtain $\forall
\mathrm{Re} s > - 1$
\begin{equation}
  \begin{array}{ll}
    s \Gamma (s) & = \int_{- \infty}^{\infty} \Lambda' (x) e^{- sx} 
    \hspace{0.17em} dx\\
    & = \int_{- \infty}^{\infty} e^{- x}  \hspace{0.17em} e^{- e^{- x}} e^{-
    sx}  \hspace{0.17em} dx \hspace{0.17em}
  \end{array}
\end{equation}
Consequently
\begin{equation}
  \Lambda_1 (x) = e^{- x - e^{- x}}
\end{equation}
is a {\pff}.

(ii) The zero set $\bZ$ with simple zeros yields
\begin{equation}
  \Psi (s) = \frac{\sin \pi s}{\pi}
\end{equation}
. By Theorem~\ref{tm:tp}, $1 / \Psi$ is the Laplace transform of a {\tp}
function on a suitable strip of convergence. Schoenberg's calculation yields
the {\tp} function
\begin{equation}
  \Lambda (x) = \frac{1}{1 + e^{- x}} \hspace{0.17em}
\end{equation}
(iii) Finally the zero set $\{- n^2 : n \in \bN \}$ yields the entire function
\begin{equation}
  \begin{array}{ll}
    \Psi (s) & = s \prod_{n = 1}^{\infty} (1 + \frac{s}{n^2})\\
    & = - \frac{1}{\pi}  \sqrt{- s} \sin \pi \sqrt{- s} \hspace{0.17em}
  \end{array}
\end{equation}
The associated {\tp} function is the Jacobi theta function
\begin{equation}
  \Lambda (x) = \left\{\begin{array}{ll}
    \sum_{j = - \infty}^{\infty} (- 1)^j e^{- j^2 x} & \text{for } x > 0\\
    0 & \text{for } x \leq 0 \hspace{0.17em}
  \end{array}\right.
\end{equation}
All three functions show up prominently in the treatment of the functional
equation of the zeta function: $\Gamma$ is contained in the definition of the
xi-function, $\sin$ in the formulation of the functional equation, and a
Jacobi theta function is used in several proofs of the functional equation
(Riemann's original proof, see~{\cite{Tit86}}).

{\tmem{Intrinsic characterization of {\pff}s.}} The fundamental property of
{\pff}s is their smoothing property or {\tmem{variation diminishing
property}}. The relevance of smoothing properties for many applications is
outlined in Schoenberg's survey~{\cite{Sch53}}. In this context the variation
of a real-valued function on $\bR$ is measured either by the number of sign
changes or by the number of {\tmem{real}} zeros. Formally, given $f : \bR \to
\bR$ let
\begin{equation}
  \label{eq:v1} v (f) = \max \# \{n \in \bN : \exists x_j \in \bR, x_0 < x_1 <
  \ldots < x_n  \text{with } f (x_j) f (x_{j + 1}) < 0\} \hspace{0.17em},
\end{equation}
and let $N (f)$ be the number of {\tmem{real}} zeros of $f$ counted with
multiplicity.

Given a function $\Lambda$, let $T_{\Lambda}$ be the convolution operator
$T_{\Lambda} f = f \ast \Lambda$. Schoenberg's second characterization of
{\pff}s is as follows~{\cite{Sch50}}.

\begin{tm}
  \label{tm:vd}Let $\Lambda$ be integrable and continuous. Then $\Lambda$ is
  variation diminishing, i.e.,
  \[ v (T_{\Lambda} f) \leq v (f) \]
  for all functions that are locally Riemann integrable, {\fif} either
  $\Lambda$ or $- \Lambda$ is a {\pff}.
\end{tm}

This characterization is ``intrinsic'' in the sense that it uses only the
properties of the matrices occurring in the definition \eqref{eq:1} of total
positivity.

With a perturbation argument one can replace sign changes with zeros and
obtains the following consequence.

\begin{corollary}
  \label{zdim}Let $\Lambda$ be a {\pff}. Then for every real-valued polynomial
  $p$ the convolution $T_{\Lambda}$ is zero-decreasing, i.e.,
  \[ N (T_{\Lambda} p) \leq N (p) \hspace{0.17em} . \]
\end{corollary}

{\vspace{3mm}}

{\tmem{Intrinsic characterizations of the {\lpc}.}} There are several
characterizations of the {\lpc} that require only their properties as entire
functions. This is part of classical complex analysis and the results are due
to Polya and Schur~{\cite{Pol15,PS1914}} building on work of Laguerre,
Hadamard, and many others. These results relate the properties of the zero set
to properties of the power series expansion of an entire function. Before
formulating a sequence of equivalences, we note that every formal power series
$F (s) \sim \sum_{j = 0}^{\infty} a_j s^j$ yields a differential operator $F
(D) p (x) = \sum_{j = 0}^{\infty} a_j D^j p (x)$ with $D = \frac{d}{dx}$. The
differential operator is well-defined at least on polynomials, and the mapping
$F \mapsto F (D)$ is an algebra homomorphism and thus provides a simple
functional calculus.

\begin{tm}
  \label{tm:lpcc}Let $\Psi (s) = \sum_{j = 0}^{\infty} \frac{\beta_j}{j!} s^j$
  be an entire function. Then the following are equivalent:
  
  (i) $\Psi$ belongs to the {\lpc}.
  
  (ii) $\Psi$ can be approximated uniformly on compact sets by polynomials
  with only real zeros.
  
  (iii) For all $n \in \bN$ the polynomials
  \begin{equation}
    p_n (x) = \sum_{j = 0}^n \beta_j \binom{n}{j} x^j
  \end{equation}
  and
  \begin{equation}
    q_n (x) = \sum_{j = 0}^n \beta_j \binom{n}{j} x^{n - j}
  \end{equation}
  have only real zeros.
  
  (iv) If
  \begin{equation}
    p (x) = \sum_{j = 0}^m c_j x^j
  \end{equation}
  is a polynomial with only real, {\tmem{non-positive}} zeros, then the
  polynomial
  \begin{equation}
    q (x) = \sum \beta_j c_j x^j
  \end{equation}
  has only real zeros.
  
  \
  
  If, in addition, $\Psi (0) > 0$ and
  \begin{equation}
    \frac{1}{\Psi (s)} = \sum_{j = 0}^{\infty} \frac{\gamma_j}{j!} s^j
  \end{equation}
  , then the following property is equivalent to (i) -- (iv).
  
  (v) The transform $p \mapsto \frac{1}{\Psi (D)} p$ is zero-decreasing, i.e.,
  the polynomial
  \begin{equation}
    q (x) = \frac{1}{\Psi (D)}
  \end{equation}
  \begin{equation}
    p (x) = \sum_{j = 0}^{\infty} \frac{\gamma_j}{j!} p^{(j)} (x)
  \end{equation}
  
  
  \ has at most as many real zeros as $p$ (real-valued):
  \begin{equation}
    N \left( \frac{1}{\Psi (D)} p \right) \leq N (p)
  \end{equation}
\end{tm}

Applying condition (iv) to the polynomials $x^{n - 1}  (1 + x)^2$, one obtains
a necessary condition on the Taylor coefficients of a function in the {\lpc},
namely the so-called Turan inequalities.

\begin{corollary}
  \label{turan}If
  \begin{equation}
    \Psi (s) = \sum_{j = 0}^{\infty} \frac{\beta_j}{j!} s^j
  \end{equation}
  belongs to the {\lpc}, then
  \[ \beta_n^2 - \beta_{n - 1} \beta_{n + 1} \geq 0 \qquad \text{for all } n
     \in \bN \]
\end{corollary}

Applying condition (v) to polynomials of the form
\begin{equation}
  p (x) = \left( \sum_{k = 1}^n a_k x^k \right)^2
\end{equation}
and working out $\Psi (D) \inv p$, one obtains the following necessary
condition for the {\lpc}~{\cite[p.~235]{Pol15}}.

\begin{corollary}
  \label{hankel}Assume that $\Psi$ belongs to the {\lpc}, $\Psi (0) > 0$,
  $\Psi (s) \neq e^{as + b}$ and $1 / \Psi$ has the Taylor expansion
  \begin{equation}
    \frac{1}{\Psi (s)} = \sum_{j = 0}^{\infty} \frac{\gamma_j}{j!} s^j
  \end{equation}
  . Then for every $n \in \bN$ the $n \times n$ Hankel matrix $(\gamma_{j +
  k})_{j, k = 0, \ldots, n - 1}$ is positive definite (and thus invertible).
\end{corollary}

However, the positivity of the Hankel matrices is not sufficient for $\Psi$ to
be in the {\lpc}, as was proved already by Hamburger~{\cite{Ham20}}.

Theorem~\ref{tm:lpcc} and its corollaries are all contained in the seminal
papers of Polya and Schur~{\cite{Pol15,PS1914}} from 1914 and 1915 and have
inspired a century of exciting mathematics. Each of the equivalent conditions
in Theorem~\ref{tm:lpcc} is a point of departure for the study of the Riemann
hypothesis.

No list can do justice to all contributions between 1914 and 2020, so let us
mention only a few directions whose origin is in Polya's work. Further
references and more detailed history can be found in the cited articles.

Condition (iii) applied to the Riemann function $\Xi$ yields an important
equivalence of the Riemann hypothesis. The polynomials in condition (iii) are
nowadays called Jensen polynomials. In modern language (iii) says that ``the
Jensen polynomials for the Riemann function $\Xi (s)$ must be hyperbolic''.
Significant recent progress on this equivalence is reported in~{\cite{GORZ}}.

The relations between the Jensen polynomials, the multiplier sequences of
condition (iv), and the Turan inequalities and their generalizations have been
studied in depth by Craven, Csordas, and Varga~{\cite{Cso15,CC89,CV90}} who
found many additional equivalences to the Riemann hypothesis. A particular
highlight is their proof that $\Xi$, or rather the Taylor coefficients of $\Xi
(\sqrt{s})$ satisfy the Turan inequalities~{\cite{CNV86}}, thereby resolving a
60 year old conjecture going back to --- Polya.

Finally let us mention that total positivity enters the investigation of the
{\lpc} in yet another way. A entire function belongs to the restricted {\lpc}
defined by \eqref{eq:ne1}, {\fif} the sequence of its Taylor coefficients
$(a_n)$ is a Polya frequency {\tmem{sequence}}~{\cite{AS52}}. This means that
the infinite upper triangular Toeplitz matrix $A$ with entries $A_{jk} = a_{k
- j}$, if $k \geq j$ and $A_{jk} = 0$, if $k < j$ has only positive minors.
This aspect of total positivity has been used in~{\cite{Kat07,Nut13}} for the
investigation of the zeta function.

{\tmem{From total positivity to the {\lpc}.}} By comparing the two intrinsic
characterizations in Theorems~\ref{tm:vd} and \ref{tm:lpcc} one may guess that
the respective conditions on zero diminishing must play the decisive role in
the proof of Theorem~\ref{tm:tp}(i). To give the gist of this argument, we
cannot do better than repeat Schoenberg's beautiful argument.

First, since $\Lambda$ is assumed to be a {\pff}, $\Lambda$ must decay
exponentially~{\cite[Lemma~2]{sch51}}, therefore its moments of all orders
exist. Let
\begin{equation}
  \mu_n = \int_{\bR} x^n \Lambda (x)  \hspace{0.17em} dx
\end{equation}
be the $n$-th moment. By expanding the exponential
\begin{equation}
  e^{- sx} = \sum_{j = 1}^{\infty} \frac{(- s)^j}{j!} x^j
\end{equation}
we express the Laplace transform of $\Lambda$ as a power series
\begin{equation}
  \label{eq:nn2} \begin{array}{ll}
    \int_{\bR} e^{- sx} \Lambda (x)  \hspace{0.17em} dx & = \sum_{j =
    0}^{\infty} \frac{(- 1)^j}{j!} s^j \mu_j  \hspace{0.17em}\\
    & = \sum_{j = 0}^{\infty} \frac{(- 1)^j}{j!} s^j \int_{\bR} x^j \Lambda
    (x)  \hspace{0.17em} dx\\
    & = F (s)
  \end{array}
\end{equation}
Since $\Lambda \nequiv 0$ and $\Lambda \geq 0$, we have $F (0) > 0$, and its
reciprocal also possesses a power series expansion around $0$ with a positive
radius of convergence
\begin{equation}
  \Psi (s) = \frac{1}{F (s)} = \sum_{j = 0}^{\infty} \frac{\beta_j}{j!} 
  \hspace{0.17em} s^j \hspace{0.17em}
\end{equation}
Next, we consider the convolution of $\Lambda$ with a polynomial $p$ of degree
$N$ and relate it to the moments of $\Lambda$:
\begin{equation}
  \begin{array}{ll}
    q (x) & = (\Lambda \ast p) (x)\\
    & = \int_{\bR} p (x - t) \Lambda (t)  \hspace{0.17em} dt\\
    & = \int_{\mathbb{R}} \left( \sum_{j = 0}^N \frac{(- t)^j}{j!} p^{(j)}
    (x) \right) \Lambda (t)  \hspace{0.17em} dt\\
    & = \sum_{j = 0}^n \frac{(- 1)^j}{j!} \mu_j  \hspace{0.17em} p^{(j)}
    (x)\\
    & = F (D) p (x)
  \end{array}
\end{equation}
By Corollary~\ref{zdim} the number of real zeros of $q$ (counted with
multiplicity) does not exceed the number of real zeros of $p$,
\begin{equation}
  \label{eq:nn3} N (q) = N (F (D) p) \leq N (p)
\end{equation}
Using the functional calculus, we can invert $F (D)$ and recover $p$ from $q =
\Lambda \ast p$ via
\begin{equation}
  \begin{array}{ll}
    p (x) & = \frac{q (x)}{F (D)} \\
    & = \Psi (D)
  \end{array}
\end{equation}
\begin{equation}
  q (x) = \sum_{j = 0}^{\infty} \frac{\beta_j}{j!}  \hspace{0.17em} q^{(j)}
  (x) \hspace{0.17em}
\end{equation}
For the monomial $q (x) = x^n$ we obtain the polynomial
\begin{equation}
  \begin{array}{ll}
    q_n (x) \hspace{0.17em} & = \Psi (D) x^n\\
    & = \sum_{j = 0}^n \beta_j \binom{n}{j} x^{n - j}
  \end{array}
\end{equation}
of degree $n$. Since $x^n = F (D) q_n$, \eqref{eq:nn3} implies the count of
zeros (with multiplicities)
\begin{equation}
  n = N (x^n) \leq N (q_n) \leq n \hspace{0.17em}
\end{equation}
For every $n$, $q_n$ therefore has only real zeros. This is precisely
condition (iii) of Theorem~\ref{tm:lpcc}, and we conclude that $\Psi$ is in
the {\lpc}.

{\tmem{Summary.}} Schoenberg's characterization of {\tp} functions implies a
condition equivalent to the Riemann hypothesis. The characterization is
interesting in itself because it involves only the values of the Riemann zeta
function on the real axis. To the best of our knowledge, the characterization
of the {\lpc} by means of {\tp} functions has not yet been tested on the
Riemann zeta function.

\begin{thebibliography}{10}
  {\bibitem{AS52}}M.~Aissen, I.~J. Schoenberg, and A.~M. Whitney.
  {\newblock}On the generating functions of totally positive sequences. I.
  {\newblock}\tmtextit{J. Analyse Math.}, 2:93--103, 1952.
  
  {\bibitem{Bom10}}E.~Bombieri. {\newblock}The classical theory of zeta and
  $L$-functions. {\newblock}\tmtextit{Milan J. Math.}, 78(1):11--59, 2010.
  
  {\bibitem{Br17-1}}K.~Broughan. {\newblock}\tmtextit{Equivalents of the
  Riemann hypothesis. Vol. 1}, volume 164 of \tmtextit{Encyclopedia of
  Mathematics and its Applications}. {\newblock}Cambridge University Press,
  Cambridge, 2017. {\newblock}Arithmetic equivalents.
  
  {\bibitem{Br17-2}}K.~Broughan. {\newblock}\tmtextit{Equivalents of the
  Riemann hypothesis. Vol. 2}, volume 165 of \tmtextit{Encyclopedia of
  Mathematics and its Applications}. {\newblock}Cambridge University Press,
  Cambridge, 2017. {\newblock}Analytic equivalents.
  
  {\bibitem{Conrey03}}J.~B. Conrey. {\newblock}The Riemann hypothesis.
  {\newblock}\tmtextit{Notices Amer. Math. Soc.}, 50(3):341--353, 2003.
  
  {\bibitem{CC89}}T.~Craven and G.~Csordas. {\newblock}Jensen polynomials and
  the Tur{\'a}n and Laguerre inequalities. {\newblock}\tmtextit{Pacific J.
  Math.}, 136(2):241--260, 1989.
  
  {\bibitem{Cso15}}G.~Csordas. {\newblock}Fourier transforms of positive
  definite kernels and the Riemann $\xi$-function.
  {\newblock}\tmtextit{Comput. Methods Funct. Theory}, 15(3):373--391, 2015.
  
  {\bibitem{CNV86}}G.~Csordas, T.~S. Norfolk, and R.~S. Varga. {\newblock}The
  Riemann hypothesis and the Tur{\'a}n inequalities.
  {\newblock}\tmtextit{Trans. Amer. Math. Soc.}, 296(2):521--541, 1986.
  
  {\bibitem{CV90}}G.~Csordas and R.~S. Varga. {\newblock}Necessary and
  sufficient conditions and the Riemann hypothesis. {\newblock}\tmtextit{Adv.
  in Appl. Math.}, 11(3):328--357, 1990.
  
  {\bibitem{CS66}}H.~B. Curry and I.~J. Schoenberg. {\newblock}On P{\'o}lya
  frequency functions. IV. The fundamental spline functions and their limits.
  {\newblock}\tmtextit{J. Analyse Math.}, 17:71--107, 1966.
  
  {\bibitem{Efr65}}B.~Efron. {\newblock}Increasing properties of P{\'o}lya
  frequency functions. {\newblock}\tmtextit{Ann. Math. Statist.}, 36:272--279,
  1965.
  
  {\bibitem{GORZ}}M.~Griffin, K.~Ono, L.~Rolen, and D.~Zagier.
  {\newblock}Jensen polynomials for the Riemann zeta function and other
  sequences. {\newblock}\tmtextit{Proc. Natl. Acad. Sci. USA},
  116(23):11103--11110, 2019.
  
  {\bibitem{GRS18}}K.~Gr{\"o}chenig, J.~L. Romero, and J.~St{\"o}ckler.
  {\newblock}Sampling theorems for shift-invariant spaces, Gabor frames, and
  totally positive functions. {\newblock}\tmtextit{Invent. Math.},
  211(3):1119--1148, 2018.
  
  {\bibitem{GRS20}}K.~Gr{\"o}chenig, J.~L. Romero, and J.~St{\"o}ckler.
  {\newblock}Sharp results on sampling with derivatives in shift-invariant
  spaces and multi-window Gabor frames. {\newblock}\tmtextit{Constr. Approx.},
  51(1):1--25, 2020.
  
  {\bibitem{GS13}}K.~Gr{\"o}chenig and J.~St{\"o}ckler. {\newblock}Gabor
  frames and totally positive functions. {\newblock}\tmtextit{Duke Math. J.},
  162(6):1003--1031, 2013.
  
  {\bibitem{Ham20}}H.~Hamburger. {\newblock}Bemerkungen zu einer Fragestellung
  des Herrn P{\'o}lya. {\newblock}\tmtextit{Math. Z.}, 7(1-4):302--322, 1920.
  
  {\bibitem{Ivi03}}A.~Ivi{\'c}. {\newblock}\tmtextit{The Riemann
  zeta-function}. {\newblock}Dover Publications, Inc., Mineola, NY, 2003.
  {\newblock}Theory and applications, Reprint of the 1985 original [Wiley, New
  York; MR0792089 (87d:11062)].
  
  {\bibitem{Iwa14}}H.~Iwaniec. {\newblock}\tmtextit{Lectures on the Riemann
  zeta function}, volume~62 of \tmtextit{University Lecture Series}.
  {\newblock}American Mathematical Society, Providence, RI, 2014.
  
  {\bibitem{karlin}}S.~Karlin. {\newblock}\tmtextit{Total positivity. Vol. I}.
  {\newblock}Stanford University Press, Stanford, Calif, 1968.
  
  {\bibitem{Kat07}}O.~M. Katkova. {\newblock}Multiple positivity and the
  Riemann zeta-function. {\newblock}\tmtextit{Comput. Methods Funct. Theory},
  7(1):13--31, 2007.
  
  {\bibitem{Levin80}}B.~J. Levin. {\newblock}\tmtextit{Distribution of zeros
  of entire functions}, volume~5 of \tmtextit{Translations of Mathematical
  Monographs}. {\newblock}American Mathematical Society, Providence, R.I.,
  revised edition, 1980. {\newblock}Translated from the Russian by R. P. Boas,
  J. M. Danskin, F. M. Goodspeed, J. Korevaar, A. L. Shields and H. P.
  Thielman.
  
  {\bibitem{Nut13}}J.~Nuttall. {\newblock}Wronskians, cumulants, and the
  Riemann hypothesis. {\newblock}\tmtextit{Constr. Approx.}, 38(2):193--212,
  2013.
  
  {\bibitem{Pick91}}D.~Pickrell. {\newblock}Mackey analysis of infinite
  classical motion groups. {\newblock}\tmtextit{Pacific J. Math.},
  150(1):139--166, 1991.
  
  {\bibitem{Pol15}}G.~P{\'o}lya. {\newblock}Algebraische Untersuchungen
  {\"u}ber ganze Funktionen vom Geschlechte Null und Eins.
  {\newblock}\tmtextit{J. Reine Angew. Math.}, 145: 224--249, 1915.
  
  {\bibitem{Pol26}}G.~P{\'o}lya. {\newblock}Bemerkung {\"U}ber die
  Integraldarstellung der Riemannschen $\xi$-Funktion.
  {\newblock}\tmtextit{Acta Math.}, 48(3-4):305--317, 1926.
  
  {\bibitem{Polcoll}}G.~P{\'o}lya. {\newblock}\tmtextit{Collected papers}.
  {\newblock}The MIT Press, Cambridge, Mass.-London, 1974. {\newblock}Vol. II:
  Location of zeros, Edited by R. P. Boas, Mathematicians of Our Time, Vol. 8.
  
  {\bibitem{PS1914}}G.~P{\'o}lya and J.~Schur. {\newblock}{\"U}ber zwei Arten
  von Faktorenfolgen in der Theorie der algebraischen Gleichungen.
  {\newblock}\tmtextit{J. Reine Angew. Math.}, 144:89--113, 1914.
  
  {\bibitem{RT20}}B.~Rodgers and T.~Tao. {\newblock}The de Bruijn--Newman
  constant is non-negative. {\newblock}\tmtextit{Forum Math. Pi}, 8:e6, 62,
  2020.
  
  {\bibitem{sch47}}I.~J. Schoenberg. {\newblock}On totally positive functions,
  Laplace integrals and entire functions of the Laguerre-Polya-Schur type.
  {\newblock}\tmtextit{Proc. Nat. Acad. Sci. U. S. A.}, 33:11--17, 1947.
  
  {\bibitem{Sch50}}I.~J. Schoenberg. {\newblock}On P{\'o}lya frequency
  functions. II. Variation-diminishing integral operators of the convolution
  type. {\newblock}\tmtextit{Acta Sci. Math. (Szeged)}, 12:97--106, 1950.
  
  {\bibitem{sch51}}I.~J. Schoenberg. {\newblock}On P{\'o}lya frequency
  functions. I. The totally positive functions and their Laplace transforms.
  {\newblock}\tmtextit{J. Analyse Math.}, 1:331--374, 1951.
  
  {\bibitem{Sch53}}I.~J. Schoenberg. {\newblock}On smoothing operations and
  their generating functions. {\newblock}\tmtextit{Bull. Amer. Math. Soc.},
  59:199--230, 1953.
  
  {\bibitem{Sch73}}I.~J. Schoenberg. {\newblock}\tmtextit{Cardinal spline
  interpolation}. {\newblock}Society for Industrial and Applied Mathematics,
  Philadelphia, Pa., 1973. {\newblock}Conference Board of the Mathematical
  Sciences Regional Conference Series in Applied Mathematics, No. 12.
  
  {\bibitem{SW53}}I.~J. Schoenberg and A.~Whitney. {\newblock}On P{\'o}lya
  frequence functions. III. The positivity of translation determinants with an
  application to the interpolation problem by spline curves.
  {\newblock}\tmtextit{Trans. Amer. Math. Soc.}, 74:246--259, 1953.
  
  {\bibitem{Tit86}}E.~C. Titchmarsh. {\newblock}\tmtextit{The theory of the
  Riemann zeta-function}. {\newblock}The Clarendon Press, Oxford University
  Press, New York, second edition, 1986. {\newblock}Edited and with a preface
  by D. R. Heath-Brown.
\end{thebibliography}

The very same results of Polya~ were also the point of departure for the many
investigations of the zeros of the Riemann zeta function after Polya, de
Bruijn, Newman, Hejhal and many others~. Yet, in none of the investigations of
the zeros of the Riemann zeta function via zeros of integral transforms and
the factorization~\eqref{eq:2} one finds a hint to {\tp} functions or the work
of Schoenberg.

\end{document}
