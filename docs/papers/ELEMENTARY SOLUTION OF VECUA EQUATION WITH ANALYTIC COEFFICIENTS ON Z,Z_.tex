\documentclass{article}
\usepackage{amsmath,amssymb,amsthm}
\usepackage{enumitem}
\usepackage{hyperref}
\usepackage{fancyhdr}
\usepackage{geometry}
\geometry{margin=0.25in}

\pagestyle{fancy}
\fancyhead[L]{MATEMATIČKI VESNIK}
\fancyhead[R]{UDK ..originalni naučni rad research paper}
\fancyfoot[C]{AMS Subject Classification: A0,0D02}

\newtheorem{theorem}{Theorem}
\newtheorem{lemma}{Lemma}
\newtheorem{definition}{Definition}

\theoremstyle{remark}
\newtheorem{remark}{Remark}

\begin{document}

\title{ELEMENTARY SOLUTION OF VEKUA EQUATION WITH ANALYTIC COEFFICIENTS ON \(z, \bar{z}\)}
\author{Miloje Rajović, Dragan Dimitrovski, Rade Stojiljković}
\date{}
\maketitle

\begin{abstract}
An explicit solution of Vekua equation with analytic coefficients on \(z, \bar{z}\) is given in the form of a series which depends on the coefficients \(A, B, F\) of the equation, and an analytic function \(\Phi(z)\) in the role of an arbitrary integration constant.
\end{abstract}

\section{Introduction}

The well-known I.N. Vekua equation

\begin{equation}
\frac{\partial W}{\partial \bar{z}} = A(z, \bar{z})W + B(z, \bar{z})\overline{W} + F(z, \bar{z}),
\label{eq:vekua}
\end{equation}

where \(A, B, F \in L_p(G)\), \(p > 2\), \(W(z, \bar{z}) = W = U(x, y) + iV(x, y)\), \(z = x + iy\);

where \(G\) is a bounded closed domain of the complex plane, with a smooth curve \(\Gamma\) as its boundary, and where the differentiation operator is \(\partial_{\bar{z}} = \frac{1}{2}(\frac{\partial}{\partial x} + i\frac{\partial}{\partial y})\), that is

\begin{equation}
\frac{\partial W}{\partial \bar{z}} = \frac{1}{2}\left[\left(\frac{\partial U}{\partial x} - \frac{\partial V}{\partial y}\right) + i\left(\frac{\partial U}{\partial y} + \frac{\partial V}{\partial x}\right)\right],
\label{eq:derivative_operator}
\end{equation}

is solved in the monograph \cite{vekua1959} and in the article \cite{vekua1962}, too, in the general case when the coefficients are some measurable functions in the domain \(G\), firstly by iteration method, but only in that domain.

There are numerous systems of partial equations of the first order and elliptic type with two unknown functions \(U(x, y)\) and \(V(x, y)\), which have the feature of reducing to equation \eqref{eq:vekua}; for example, for the well-known Carleman's system

\begin{equation}
\begin{aligned}
\frac{\partial U}{\partial x} - \frac{\partial V}{\partial y} + aU + bV &= f, \\
\frac{\partial U}{\partial y} + \frac{\partial V}{\partial x} + cU + dV &= g,
\end{aligned}
\label{eq:carleman}
\end{equation}

in the case of analytic coefficients \(a, b, c, d, f, g\), it is possible to get the solution of the system \eqref{eq:carleman}, written in the form \eqref{eq:vekua}, by the iteration methods of I.N. Vekua, using the integral equation

\begin{equation}
W(z, \bar{z}) = -\frac{1}{\pi}\iint_G \frac{A(\zeta)W(\zeta) + B(\zeta)\overline{W(\zeta)}}{\zeta - z}\,d\xi\,d\eta + \frac{1}{2\pi i}\int_\Gamma \frac{W(\zeta)\,d\zeta}{\zeta - z},
\label{eq:integral_representation}
\end{equation}

\(\zeta = \xi + i\eta\), but those iterations have only symbolic meaning and no practical validity to solve the given system; they only give one way of qualitative judgment of the solution.

This is the main reason why we now give one direct approach to solving the equation \eqref{eq:vekua} in the form of an explicit formula, which is a generalization of quadratures, where the solution will depend only on coefficients, using the method of areolar series.

\section{Homogeneous Equation}

\begin{theorem}
The incomplete homogeneous I.N. Vekua equation with conjugation of the unknown function \(W\), and with analytic coefficient \(A(z)\) which depends only on a complex variable \(z\),

\begin{equation}
\frac{\partial W}{\partial \bar{z}} = A(z)\overline{W},
\label{eq:homogeneous}
\end{equation}

has general solution in the form of series which depends on \(A(z)\) and \(\Phi(z)\), where the last function is analytic only on \(z\) and has the role of an arbitrary integration constant:

\begin{equation}
\begin{aligned}
W(z, \bar{z}) &= \Phi(z) + A(z)\Biggl[\int \overline{\Phi}\,d\bar{z} + \int \overline{A}\,d\bar{z}\int \Phi\,dz \\
&\quad + \int \overline{A}\,d\bar{z}\int A\,dz\int \overline{\Phi}\,d\bar{z} + \int \overline{A}\,d\bar{z}\int A\,dz\int \overline{A}\,d\bar{z}\int \Phi\,dz + \cdots \\
&\quad + \int \overline{A}\,d\bar{z}\int A\,dz\int \overline{A}\,d\bar{z}\int A\,dz\cdots\int \Phi(z)\,dz + \cdots\Biggr].
\end{aligned}
\label{eq:series_solution}
\end{equation}
\end{theorem}

The formula was proved for the first time by M. \v{C}anak \cite{canak1979} (see also \cite{canak1980}). A proof by the method of areolar series is given in the papers \cite{dimitrovski1978} and \cite{canak1979}.

Taking the series

\begin{equation}
A(z) = \sum_{k=0}^\infty a_k z^k, \qquad W(z, \bar{z}) = \sum_{p,q=0}^\infty C_{p,q}z^p\bar{z}^q,
\label{eq:areolar_series}
\end{equation}

and using the fact that an areolar equation with analytic coefficients has only analytic solutions \cite{dimitrovski1978} (Cauchy type problem, proved by Dimitrovski-Ilijevski), we get the coefficients in polynomial form

\begin{equation}
C_{p,q} = P_{p,q}(a_0, a_1, a_2, \ldots, a_k); \qquad k \leq p-1, q-1.
\label{eq:coefficients}
\end{equation}

After the grouping and condensation of coefficients \(C_{p,q}\) corresponding to the powers \(z^p\bar{z}^q\), it can be seen that it is possible to write these terms in the integral form as

\begin{equation}
C_{p,q} = \int \overline{A}\,d\bar{z}\int A\,dz\int \overline{\Phi}\,d\bar{z},
\label{eq:coefficient_integral}
\end{equation}

where the coefficient \(C_{p,0}\) is arbitrary, i.e. it determines an arbitrary analytic function

\begin{equation}
\Phi(z) = \sum_{p=0}^\infty C_{p,0}z^p.
\label{eq:phi_series}
\end{equation}

\begin{proof}
The proof of these facts needs a detailed technical procedure and we think there is no need to reproduce it here.
\end{proof}

The solution \eqref{eq:series_solution} satisfies the equation \eqref{eq:homogeneous} where \(A(z) = U(x, y) + iV(x, y)\) is an analytic function for which Cauchy-Riemann's conditions \(U_x = V_y\), \(U_y = -V_x\) are valid. This inspires us how to solve a more general Vekua equation.

\section{General Equation}

Let the equation

\begin{equation}
\frac{\partial \overline{W}}{\partial \bar{z}} = A(z, \bar{z})\overline{W},
\label{eq:general}
\end{equation}

be given, where \(A(z, \bar{z})\) is the given analytic function on \(z, \bar{z}\). Since Cauchy-Riemann's condition on \(A\) need not be valid now, the problem is much more general. Also, the method of areolar series

\begin{equation}
A(z, \bar{z}) = \sum_{p,q=0}^\infty a_{p,q}z^p\bar{z}^q; \qquad W(z, \bar{z}) = \sum_{p,q=0}^\infty C_{p,q}z^p\bar{z}^q,
\label{eq:double_series}
\end{equation}

and their summing, which would be natural to use here too, is too much complicated and it would be hard to get the solution in the form \eqref{eq:series_solution}. That is the reason why we shall use the operator method here.

Let \(R^*\) denote the reverse operation to conjugated differentiation \(\partial / \partial \bar{z}\), i.e.

\begin{equation}
\frac{\partial W}{\partial \bar{z}} = F \iff W = \int^* F = \int^* A(z, \bar{z})\overline{W},
\label{eq:operator}
\end{equation}

where the right-hand side of \eqref{eq:operator} is a complex integral equation. But as analytic (in wider sense) equation has an analytic solution \eqref{eq:series_solution}, then it is

\begin{equation}
W = \int A(z, \bar{z})\overline{W}\,d\bar{z} + \Phi(z),
\label{eq:integral_form}
\end{equation}

where \(\Phi(z)\) is an arbitrary analytic function, the integration constant.

\begin{theorem}
The operator \(R^* F\) is a contraction operator, if \(F\) is an analytic function, defined in a simple bounded region \(D\).
\label{thm:contraction}
\end{theorem}

\begin{proof}
Denote the right-hand side of \eqref{eq:integral_form} by

\begin{equation}
T(W) = \int A\overline{W}\,d\bar{z} + \Phi(z).
\label{eq:T_operator}
\end{equation}

We obtain

\begin{equation}
\begin{aligned}
|T W_1 - T W_2| &= \left|\int A\overline{W}_1\,d\bar{z} + \Phi - \int A\overline{W}_2\,d\bar{z} - \Phi\right| \\
&< \int |A(z, \bar{z})| \cdot |\overline{W}_1 - \overline{W}_2| \cdot |d\bar{z}| \\
&\leq \max_D |A| \cdot \max_D |\overline{W_1 - W_2}| \cdot |z| < M m h,
\end{aligned}
\label{eq:contraction_estimate}
\end{equation}

where \(M = \max_D |A(z, \bar{z})|\), \(m = \max_D |\overline{W_1 - W_2}|\), \(h = \max_D |z|\). Iterating we get

\begin{equation}
|T^2 W_1 - T^2 W_2| = |T(T W_1) - T(T W_2)| = |T(T W_1 - T W_2)|,
\label{eq:iteration2}
\end{equation}

and so on. By induction we obtain that

\begin{equation}
|T^n W_1 - T^n W_2| \leq M^n m \frac{h^n}{n!},
\label{eq:iterationn}
\end{equation}

and as we can choose \(n\) so that \((M h)^n / n!\) becomes arbitrarily small, we can obtain that

\begin{equation}
\|T^n W_1 - T^n W_2\| \leq q\|W_1 - W_2\|
\label{eq:contraction_final}
\end{equation}

with \(q < 1\), that is the operator \(T\) determined by \eqref{eq:T_operator} is a contraction operator.
\end{proof}

\begin{theorem}
The Vekua equation \eqref{eq:general} has the general solution written in the form of the third approximation

\begin{equation}
\begin{aligned}
W &= W_3 = \Phi + \overline{\Phi}\int A(z, \bar{z})\,d\bar{z} + \int A(z, \bar{z})\,d\bar{z}\int \overline{A}\Phi(z)\,dz \\
&\quad + \int A(z, \bar{z})\,d\bar{z}\int \overline{A}(z, \bar{z})\overline{\Phi}(z)\,dz\int A(z, \bar{z})\,d\bar{z} + R_3,
\end{aligned}
\label{eq:third_approximation}
\end{equation}

where the residue has the form

\begin{equation}
R_3 = \int A\,d\bar{z}\int \overline{A}\,dz\int A\,d\bar{z}\int \overline{A}W(z, \bar{z})\,dz.
\label{eq:residue}
\end{equation}
\end{theorem}

\begin{proof}
Starting with iterations, from \eqref{eq:integral_form} we get \(W = \int^* A(z, \bar{z})W\,d\bar{z} + \Phi(z)\), where \(W = W_0\) is an arbitrary initial value. Now define the second approximation

\begin{equation}
W_2 = \int A(z, \bar{z})\overline{W}_1\,d\bar{z} + \overline{\Phi}(z),
\label{eq:second_approx}
\end{equation}

which is the base for

\begin{equation}
W_1 = \int A(z, \bar{z})W\overline{d\bar{z}} + \Phi(z),
\label{eq:first_approx}
\end{equation}

and \(\overline{W} = \overline{W}_0\). Then

\begin{equation}
\begin{aligned}
W_2 &= \int \overline{A}\left(\int \overline{A}\overline{W}\,d\bar{z} + \Phi\right)d\bar{z} + \Phi \\
&= \int A\int[A W\,dz + \overline{\Phi}]\,d\bar{z} + \Phi \\
&= \Phi + \overline{\Phi}\int A\,d\bar{z} + \int A\,d\bar{z}\int \overline{A}W\,dz.
\end{aligned}
\label{eq:w2_computation}
\end{equation}

In the same way we define

\begin{equation}
W_3 = \Phi + \overline{\Phi}\int A\,d\bar{z}\int \overline{A}W_2\,dz,
\label{eq:w3_definition}
\end{equation}

and so on. We can go arbitrarily far in this direction. The process converges by the fixed point principle, using the analyticity assumed.
\end{proof}

This inspires us to solve the general equation \eqref{eq:vekua} with arbitrary analytic coefficients. Introducing the operator \(R^*\) and starting from \eqref{eq:operator}, we have

\begin{equation}
W = \int^*(A W + B\overline{W} + F),
\label{eq:general_integral}
\end{equation}

and since an analytic equation of the first order has an analytic solution \(W(z, \bar{z}, A, B, F)\) we get

\begin{equation}
W = \int (A W + B\overline{W} + F)\,d\bar{z} + \Phi.
\label{eq:general_form}
\end{equation}

But, since

\begin{equation}
\overline{W} = \int \overline{(A W + B\overline{W} + F)}\,d\bar{z} + \overline{\Phi} = \int (\overline{A W} + \overline{B}W + \overline{F})\,dz + \overline{\Phi},
\label{eq:conjugate_w}
\end{equation}

by iteration it is easy to obtain the following:

\begin{theorem}
The Vekua equation \eqref{eq:vekua} with analytic coefficients has the following representation of the general solution in the form of series of integrals of the coefficients and arbitrary integration elements:

\begin{equation}
\begin{aligned}
W(z, \bar{z}) &= \Phi + \int A\Phi\,dz + \int A\,dz\int A\Phi\,d\bar{z} + \int A\,dz\int A\,d\bar{z}\int A\Phi\,d\bar{z} \\
&\quad + \cdots + \int B\overline{\Phi}\,dz + \int B\,dz\int \overline{B}\overline{\Phi}\,dz + \int B\,dz\int \overline{B}\,dz\int \overline{B}\Phi\,dz \\
&\quad + \cdots + \int A\,dz\int B\overline{\Phi}\,dz\int B\,dz\int A\overline{\Phi}\,dz + \int A\,dz\int A\,dz\int B\overline{B}\overline{\Phi}\,dz \\
&\quad + \int B\,dz\int \overline{B}\,dz\int \overline{A}\,dz + \cdots + \int F\,d\bar{z} + \int A\,dz\int F\,d\bar{z} + \cdots
\end{aligned}
\label{eq:full_series}
\end{equation}
\end{theorem}

Or, stated in another way:

\begin{theorem}
The solution of Vekua equation \eqref{eq:vekua} can be written as a sum of four summands

\begin{equation}
W(z, \bar{z}) = W_{A,\Phi} + W_{B,\Phi} + W_{A,B,\Phi} + W_{A,B,F},
\label{eq:summands}
\end{equation}

whose forms are given in the formula \eqref{eq:full_series}, where one can see that each of the coefficients \(A, B, F\) particularly has an influence to the solution, while \(\Phi\) is an arbitrary analytic function in the role of the integration constant.
\end{theorem}

We see that there is no part with \(F\) alone, i.e. \(W_{F,\Phi} = 0\) and so it follows:

\begin{theorem}
Unhomogeneity \(F\) has no influence to the form of the general solution.
\label{thm:unhomogeneity}
\end{theorem}

\section{Applications}

As Carleman's system \eqref{eq:carleman} can be easily reduced to \eqref{eq:vekua}, where \(A, B, F\) simply depend on \(a, b, c, d, f, g\), a great number of systems \eqref{eq:carleman} of elliptic type with analytic coefficients can be solved through the functions

\begin{equation}
U(x, y) = \operatorname{Re}W, \quad V(x, y) = \operatorname{Im}W,
\label{eq:uv_parts}
\end{equation}

in the sense of general solution, where \(\Phi(z) = \alpha(x, y) + i\beta(x, y)\), and where \(\alpha\) and \(\beta\) are arbitrary real harmonic functions which fulfill the Cauchy-Riemann's conditions \(\alpha_x = \beta_y\), \(\alpha_y = -\beta_x\).

By elimination of \(U\) (or \(V\)), a great number of partial equations of the second order

\begin{equation}
\frac{\partial^2 U}{\partial x^2} + \frac{\partial^2 U}{\partial y^2} + P\frac{\partial U}{\partial x} + Q\frac{\partial U}{\partial y} + RU = S,
\label{eq:second_order}
\end{equation}

can be solved, too (and similarly, equations depending on the conjugated function \(V(x, y)\)).

\section{Conclusion}

A simple procedure, formal-mathematical and iterative, solves Vekua equation easier than it was done in the well-known article \cite{vekua1962}, through an integral symmetrical formula, which is simple for appraisal and approximation, which we didn't find in literature.

\begin{remark}
Fundamental theorem is also formulated in a different way in \cite{canak1979}.
\end{remark}

\begin{thebibliography}{9}
\bibitem{vekua1959}
I.N. Vekua, \emph{Ob obshchenie analiticheskie funkcii}, Nauka, Moscow.

\bibitem{vekua1962}
I.N. Vekua, \emph{Sistemy differentsial'nyh uravneni v elliptichesko go tipa i granichnye zadachi s primeneniem v teorii obolochek}, Mat. Sbornik.

\bibitem{ilijevski1970}
B. Ilijevski, \emph{Nekoi analitichki reshenia na nekoi klasi ravenki od tip I.N. Vekua}, Mat. Bilten Drush voto mat.inf.R.Makedonija (XL).

\bibitem{ilijevski1975}
B. Ilijevski, \emph{Linearni areolarni ravenki}, Ph.D. thesis, Skopje.

\bibitem{dimitrovski1978}
D. Dimitrovski, B. Ilijevski, \emph{L'equation differentielle lineaire areolaire analytique}, Prilozi MANU, sec. Math. Techn.

\bibitem{canak1979}
M. \v{C}anak, \emph{Systeme von Differentialgleichungen erster Ordnung vom elliptischen Typus mit analytischen Koeffizienten und Methode der verallgemeinerten areolaren Reihen}, Publ. de l'Inst. Math.

\bibitem{canak1980}
M. \v{C}anak, \emph{Uber Existenz und Einzigkeit der L\"osung einer areolaren Differentialgleichung erster Ordnung}, Math. Balkanica.
\end{thebibliography}

\noindent
Miloje Rajović, Dept. of Mathematics, Technical Faculty, Kraljevo, Yugoslavia

\noindent
Dragan Dimitrovski, Mathematical Institute, Faculty of Sciences, 000 Skopje, Rep. of Macedonia, P.O. Box

\noindent
Rade Stojiljković, High Pedagogical School, Gnjilane, Yugoslavia

\end{document}
