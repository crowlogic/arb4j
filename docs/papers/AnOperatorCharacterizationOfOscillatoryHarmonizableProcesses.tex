\documentclass{article}
\usepackage[english]{babel}
\usepackage{geometry,amsmath,amssymb,latexsym}
\geometry{letterpaper}

%%%%%%%%%% Start TeXmacs macros
\newcommand{\cdummy}{\cdot}
\newcommand{\tmaffiliation}[1]{\\ #1}
\newcommand{\tmtextbf}[1]{\text{{\bfseries{#1}}}}
\newcommand{\tmtextit}[1]{\text{{\itshape{#1}}}}
\newenvironment{proof}{\noindent\textbf{Proof\ }}{\hspace*{\fill}$\Box$\medskip}
\newtheorem{corollary}{Corollary}
\newtheorem{definition}{Definition}
\newtheorem{proposition}{Proposition}
\newtheorem{theorem}{Theorem}
%%%%%%%%%% End TeXmacs macros

\begin{document}

\title{An Operator Characterization of Oscillatory Harmonizable Processes}

\author{
  Randall J. Swift
  \tmaffiliation{Department of Mathematics\\
  Western Kentucky University, Bowling Green, Kentucky}
}

\date{}

\maketitle

\begin{center}
  Dedicated to Professor M. M. Rao, advisor and friend, on the occasion of his
  65th birthday.
\end{center}

{\tableofcontents}

\section{INTRODUCTION}

A class of nonstationary stochastic processes which are encountered in some
applications is the class of modulated stationary processes $X (t)$. These
processes are obtained when a stationary process $X_0 (t)$ is multiplied by
some nonrandom modulating function $A (t)$:
\begin{equation}
  \label{eq:modulated-process} X (t) = A (t) X_0 (t)
\end{equation}
This class of processes has been investigated by Joyeux {\cite{Joyeux1987}}
and Priestley {\cite{Priestley1981}}. The book by Yaglom {\cite{Yaglom1987}}
provides a nice treatment of these processes. In particular, if $A (t)$ armits
a generalized Fourier transform, the class of oscillatory processes, studied
by Priestley {\cite{Priestley1981}} is obtained. In some physical situations,
the assumption of stationarity for the process $X_0 (t)$ is unrealistic
{\cite{Rao1982}}. If this condition is relaxed, and $X_0 (t)$ is assumed to be
harmonizable and if $A (t)$ armits a generalized Fourier transform, the
process $X (t)$ is not oscillatory, but is oscillatory harmonizable.

This paper investigates the properties of oscillatory harmonizable processes.
Section~\ref{sec:preliminaries} recalls the basic theory of harmonizable
processes required for the subsequent analysis.
Section~\ref{sec:oscillatory-harmonizable} introduces and develops the class
of oscillatory harmonizable processes. In this section, the spectral
representation of oscillatory harmonizable processes is obtained. This
representation is used to deduce relationships between the oscillatory
harmonizable processes and other classes of nonstationary processes.
Section~\ref{sec:operator-characterization} obtains an important and useful
operator characterization for oscillatory harmonizable processes.

\section{PRELIMINARIES}\label{sec:preliminaries}

In the following work, there is always an underlying probability space,
$(\Omega, \Sigma, P)$, whether this is explicitly stated or not.

\begin{definition}
  \label{def:lp-space}For $p \geq 1$, define $L_0^p (P)$ to be the set of all
  complex valued $f \in L^p (\Omega, \Sigma, P)$ such that $E (f) = 0$, where
  $E (f) = \int f (\omega) dP (\omega)$ is the expectation.
\end{definition}

In this paper, we will consider second order stochastic processes. More
specifically, mappings $X : \mathbb{R} \to L_0^2 (P)$.

\begin{definition}
  \label{def:stationary}A stochastic process $X : \mathbb{R} \to L_0^2 (P)$ is
  stationary (stationary in the wide or Khintchine sense) if its covariance $r
  (s, t) = E (X (s) \cdot \overline{X (t)})$ is continuous and is a function
  of the difference of its arguments, so that
  \begin{equation}
    \label{eq:stationary-covariance-form} r (s, t) = r (s - t)
  \end{equation}
\end{definition}

An equivalent definition of a stationary process is one whose covariance
function can be represented as
\begin{equation}
  \label{eq:stationary-covariance} r (t) = \int_{\mathbb{R}} e^{i \lambda t} 
  \hspace{0.17em} dF (\lambda)
\end{equation}
for a unique non-negative bounded Borel measure $F (\cdummy)$. This alternate
definition is a consequence of a classical theorem of Bochner's
{\cite{GihmanSkorohod1974}}, and motivates the following definition.

\begin{definition}
  \label{def:weakly-harmonizable}A stochastic process $X : \mathbb{R} \to
  L_0^2 (P)$ is weakly harmonizable if its covariance $r (\cdummy, \cdummy)$
  is expressible as
  \begin{equation}
    \label{eq:harmonizable-covariance} r (s, t) = {\int_{- \infty}^{\infty}} 
    {\int_{- \infty}^{\infty}}  e^{i \lambda s - i \lambda' t} 
    \hspace{0.17em} dF (\lambda, \lambda')
  \end{equation}
  where $F : \mathbb{R} \times \mathbb{R} \to \mathbb{C}$ is a positive
  semi-definite bimeasure, hence of finite Fr{\'e}chet variation. The
  integrals in \eqref{eq:harmonizable-covariance} are strict Morse-Transue
  {\cite{ChangRao1986}}. A stochastic process, $X (\cdummy)$, is strongly
  harmonizable if the bimeasure $F (\cdummy, \cdummy)$ in
  \eqref{eq:harmonizable-covariance} extends to a complex measure and hence is
  of bounded Vitali variation. In either case, $F (\cdummy, \cdummy)$ is
  termed the spectral bi-measure (or spectral measure) of the harmonizable
  process.
\end{definition}

Comparison of equation \eqref{eq:harmonizable-covariance} with equation
\eqref{eq:stationary-covariance} shows that when $F (\cdummy, \cdummy)$
concentrates on the diagonal $\lambda = \lambda'$, both the weak and strong
harmonizability concepts reduce to the stationary concept. Harmonizable
processes retain the powerful Fourier analytic methods inherent with
stationary processes, as seen in Bochner's theorem,
\eqref{eq:stationary-covariance}; but they relax the requirement of
stationarity.

The structure and properties of harmonizable processes has been investigated
and developed extensively by M.M. Rao and others. The following sources are
listed here to provide a partial summary of the literature. The papers by Rao
{\cite{Rao1978,Rao1982,Rao1989,Rao1991,Rao1994}} provide a basis for the
theory. Chang and Rao {\cite{ChangRao1986}} develop the necessary bi-measure
theory. A study of sample path behavior for harmonizable processes is
considered by Swift {\cite{Swift1996b}}. Some results on moving average
representations were obtained by Mehlman {\cite{Mehlman1992}}. The structure
of harmonizable isotropic random fields and some applications has been
considered by Swift {\cite{Swift1994,Swift1995,Swift1996a}}. Second order
processes with harmonizable increments has been investigated also by Swift
{\cite{Swift1996c}}. The forthcoming book by Kakihara {\cite{Kakihara}} gives
a general treatment of multidimensional second order processes which include
the harmonizable class.

\section{OSCILLATORY HARMONIZABLE
PROCESSES}\label{sec:oscillatory-harmonizable}

M. B. Priestley {\cite{Priestley1981}} introduced and studied a generalization
of the class of stationary processes. This generalization is given by:

\begin{definition}
  \label{def:oscillatory}A stochastic process $X : \mathbb{R} \to L_0^2 (P)$
  is oscillatory if it has representation
  \begin{equation}
    \label{eq:oscillatory-representation} X (t) = \int_{\mathbb{R}} A (t,
    \lambda) e^{i \lambda t}  \hspace{0.17em} dZ (\lambda)
  \end{equation}
  where $Z (\cdummy)$ is a stochastic measure with orthogonal increments and
  \begin{equation}
    \label{eq:oscillatory-modulating-function} A (t, \lambda) =
    \int_{\mathbb{R}} e^{i \lambda t} H (\lambda, dx)
  \end{equation}
  with $H (\cdummy, B)$ a Borel function on $\mathbb{R}$, $H (\lambda, \cdot)$
  a signed measure and $A (t, \lambda)$ having an absolute maximum at $\lambda
  = 0$ independent of $t$.
\end{definition}

Using this representation the covariance of an oscillatory process is
\begin{equation}
  \label{eq:oscillatory-covariance} r (s, t) = \int_{\mathbb{R}} A (s,
  \lambda) A (t, \lambda) e^{i \lambda (s - t)}  \hspace{0.17em} d \Phi
  (\lambda)
\end{equation}
The idea of definition~\ref{def:weakly-harmonizable} provides the motivation
for the following definition.

\begin{definition}
  \label{def:oscillatory-weakly-harmonizable}A stochastic process $X :
  \mathbb{R} \to L_0^2 (P)$ is oscillatory weakly harmonizable if its
  covariance has representation
  \begin{equation}
    \label{eq:oscillatory-harmonizable-covariance} r (s, t) = {\int_{-
    \infty}^{\infty}}  {\int_{- \infty}^{\infty}}  A (s, \lambda) A (t,
    \lambda') e^{i \lambda s - i \lambda' t}  \hspace{0.17em} dF (\lambda,
    \lambda')
  \end{equation}
  where $F (\cdummy, \cdummy)$ is a function of bounded Fr{\'e}chet variation,
  and
  \begin{equation}
    A (t, \lambda) = \int_{\mathbb{R}} e^{i \lambda t} H (\lambda, dx)
  \end{equation}
  with $H (\cdummy, B)$ a Borel function on $\mathbb{R}$, $H (\lambda, \cdot)$
  a signed measure and $A (t, \lambda)$ having an absolute maximum at $\lambda
  = 0$ independent of $t$.
\end{definition}

Note that if $A (t, \lambda) = 1$, this class coincides with the weakly
harmonizable processes. As Priestley's definition provides an extension to the
class of stationary processes,
definition~\ref{def:oscillatory-weakly-harmonizable} provides an cxtension to
the class of weakly harmonizable processes.

Observe, further, that in this definition, for $F (\cdummy, \cdummy)$
concentrating on the diagonal, $\lambda = \lambda'$, the oscillatory processes
are obtained. Thus the oscillatory harmonizable processes also provide an
extension to the class introduced by Priestley, which we will now term
oscillatory stationary.

Using this definition, it is possible to obtain the spectral representation of
an oscillatory harmonizable process $X (\cdummy)$

\begin{theorem}
  \label{thm:spectral-representation}The spectral representation of an
  oscillatory weakly harmonizable stochastic process is:
  \begin{equation}
    \label{eq:spectral-representation} X (t) = \int_{\mathbb{R}} A (t,
    \lambda) e^{i \lambda t}  \hspace{0.17em} dZ (\lambda)
  \end{equation}
  where $Z (\cdummy)$ is a stochastic measure satisfying
  \begin{equation}
    \label{eq:spectral-measure-property} E (Z (B_1) \overline{Z (B_2)}) = F
    (B_1, B_2)
  \end{equation}
  with $F (\cdummy, \cdummy)$ a function of bounded Fr{\'e}chet variation.
\end{theorem}

\begin{proof}
  Let $X (\cdummy)$ be an oscillatory weakly harmonizable process. Then, the
  covariance $r (\cdummy, \cdummy)$ has representation
  \eqref{eq:oscillatory-harmonizable-covariance}. Applying a form of
  Karhunen's theorem {\cite{Yaglom1987}} gives the spectral representation of
  $X (\cdummy)$ as \eqref{eq:spectral-representation}, which is the desired
  result.
\end{proof}

The following condition on the signed measure $H$, for oscillatory strongly
harmonizable processes shows these processes are actually a subclass of the
strongly harmonizable processes. A similar result was obtained by R. Joyeux
{\cite{Joyeux1987}}, for the oscillatory stationary processes.

\begin{theorem}
  \label{thm:strongly-harmonizable-condition}If $X (\cdummy)$ is an
  oscillatory strongly harmonizable process with
  \begin{equation}
    \label{eq:h-condition} \int_{\mathbb{R}} |H| (\lambda, dr) < \infty
  \end{equation}
  uniformly in $\lambda \in \mathbb{R}$, then $X (\cdummy)$ is strongly
  harmonizable.
\end{theorem}

\begin{proof}
  Let
  \begin{equation}
    \label{eq:z-tilde} \tilde{Z} (A) = \int_{\mathbb{R}} H (\lambda, A -
    \lambda)  \hspace{0.17em} dZ (\lambda)
  \end{equation}
  where $A$ is a Borel set of $\mathbb{R}$ and $A - \lambda = \{x - \lambda :
  x \in A\}$. $\tilde{Z} (\cdummy)$ is a stochastic measure since $H (\lambda,
  \cdot)$ is a signed measure, and uniformly bounded by $K$.
  
  Now set
  \begin{equation}
    \label{eq:x-tilde} \tilde{X} (t) = \int_{\mathbb{R}} e^{i \omega t} 
    \hspace{0.17em} \tilde{Z} (d \omega)
  \end{equation}
  \tmtextbf{Claim:} $\tilde{X} (\cdummy)$ is a strongly harmonizable process.
  
  If one lets $\tilde{F} (A, B) = E (\tilde{Z} (A) \overline{\tilde{Z} (B)})$
  for Borel sets $A, B$ of $\mathbb{R}$, it must be shown that
  \begin{equation}
    \label{eq:finite-variation-condition} {\int_{- \infty}^{\infty}}  {\int_{-
    \infty}^{\infty}}  | \tilde{F} (d \omega, d \omega') | < \infty
  \end{equation}
  Now
  
  \begin{align}
    E (\tilde{Z} (d \omega) \overline{\tilde{Z} (d \omega')}) & = {\int_{-
    \infty}^{\infty}}  {\int_{- \infty}^{\infty}}  H (\lambda, d (\omega -
    \lambda)) \overline{H (\lambda', d (\omega' - \lambda'))} E (Z (d \lambda)
    \overline{Z (d \lambda')}) \nonumber\\
    & = {\int_{- \infty}^{\infty}}  {\int_{- \infty}^{\infty}}  H (\lambda, d
    (\omega - \lambda)) \overline{H (\lambda', d (\omega' - \lambda'))} F (d
    \lambda, d \lambda')  \label{eq:f-tilde-expression}
  \end{align}
  
  where $F (A, B) = E (Z (A) \overline{Z (B)})$ is of finite Vitali variation
  since $X (t)$ is strongly harmonizable.
  
  Thus,
  
  \begin{align}
    {\int_{- \infty}^{\infty}}  {\int_{- \infty}^{\infty}}  | \tilde{F} (d
    \omega, d \omega') | & = {\int_{- \infty}^{\infty}}  {\int_{-
    \infty}^{\infty}}  | {\int_{- \infty}^{\infty}}  {\int_{-
    \infty}^{\infty}}  H (\lambda, d (\omega - \lambda)) \overline{H
    (\lambda', d (\omega' - \lambda'))} F (d \lambda, d \lambda') |
    \nonumber\\
    & \leq {\int_{- \infty}^{\infty}}  {\int_{- \infty}^{\infty}}  |H|
    (\lambda, d (\omega - \lambda)) |H| (\lambda', d (\omega' - \lambda')) |F|
    (d \lambda, d \lambda') \nonumber\\
    & < \infty  \label{eq:finite-variation-result}
  \end{align}
  
  since $\int |H| (\lambda, \mathbb{R})$ is bounded, proving the claim. Now
  \begin{equation}
    \begin{array}{ll}
      \tilde{X} (t) & = \int_{\mathbb{R}} e^{i \omega t}  \hspace{0.17em}
      \tilde{Z} (d \omega) \text{}\\
      & = {\int_{- \infty}^{\infty}}  {\int_{- \infty}^{\infty}}  e^{i \omega
      t} H (\lambda, d (\omega - \lambda))  \hspace{0.17em} Z (d \lambda)\\
      & = \int e^{i \lambda t} A (t, \lambda)  \hspace{0.17em} Z (d
      \lambda)\\
      & = X (t)
    \end{array} \label{eq:x-equality}
  \end{equation}
  
  
  So $X (t)$ is strongly harmonizable, which completes the proof of the
  theorem.
\end{proof}

An additional class of processes related to the oscillatory processes is given
by:

\begin{definition}
  \label{def:slowly-changing}An oscillatory weakly harmonizable stochastic
  process $X : \mathbb{R} \to L_0^2 (P)$ is $\varepsilon$-slowly changing
  weakly harmonizable if
  \begin{equation}
    \label{eq:slowly-changing-condition} B (\lambda) = \int_{\mathbb{R}} |x|
    |H| (\lambda, dx) \leq \varepsilon, \quad \lambda \in \mathbb{R}.
  \end{equation}
\end{definition}

Slowly changing stationary processes where first considered by Priestley
{\cite{Priestley1981}} and are of interest not only in engineering but also in
economics. Priestley showed that it is possible to define a spectral measure
for these processes. The class of slowly changing harmonizable processes
introduced above extend the class of slowly changing stationary processes. The
following corollary shows that it is possible to consider a similar concept
for the slowly changing harmonizable class.

\begin{corollary}
  \label{cor:slowly-changing}Slowly changing strongly harmonizable processes
  form a subclass of strongly harmonizable processes.
\end{corollary}

\begin{proof}
  The assumption is \eqref{eq:slowly-changing-condition}.
  
  \tmtextbf{Claim:}
  \begin{equation}
    \label{eq:h-finite} \int_{\mathbb{R}} |H| (\lambda, dr) < \infty
  \end{equation}
  In fact,
  
  \begin{align}
    |H| (\lambda, \mathbb{R}) & = \int_{\mathbb{R}} |H| (\lambda, dr)
    \nonumber\\
    & = \int_{|x| \leq 1} |H| (\lambda, dx) + \int_{|x| > 1} |H| (\lambda,
    dx) \nonumber\\
    & \leq K + \int_{|x| > 1} |H| (\lambda, dx) \nonumber\\
    & \leq K + \int_{|x| > 1} |x| |H| (\lambda, dx) \nonumber\\
    & \leq K + \int_{\mathbb{R}} |x| |H| (\lambda, dx) \nonumber\\
    & \leq K + \varepsilon < \infty  \label{eq:h-bound}
  \end{align}
  
  which is the claim. Now since $K$ is finite, by
  Theorem~\ref{thm:strongly-harmonizable-condition} $X (t)$ is strongly
  harmonizable proving the corollary.
\end{proof}

\section{AN OPERATOR CHARACTERIZATION}\label{sec:operator-characterization}

Using oscillatory harmonizable processes, it is possible to obtain a
representation of a broader class of processes on $\mathbb{R}$.

\begin{definition}
  \label{def:weak-class-c}Let $S$ be a locally compact space with
  $\mathcal{B}_0$ as the $\sigma$-ring generated by the bounded Borel sets of
  $S$. If $T$ is any index set, $\{X (t), t \in T\} \subset L_0^2 (P)$ a
  second order process, $r$ its covariance and $\beta : \mathcal{B}_0 \times
  \mathcal{B}_0 \to \mathbb{C}$, a bimeasure having locally bounded
  Fr{\'e}chet variation then $X (\cdummy)$ is said to be (locally) weakly of
  class (C) when $\beta$ is positive definite and
  \begin{equation}
    \label{eq:class-c-covariance} r (s, t) = \iint_{S \times S} g_s (x)
    \overline{g_t (y)} \hspace{0.17em} \beta (dx, dy) \quad \forall (s, t) \in
    T \times T \text{(strict MT-integral)}
  \end{equation}
  where $g_t : S \to \mathbb{C}, t \in T$ a family of Borel functions for
  which the integral exists. If $\beta$ has locally finite Vitali variation,
  then the process is termed of class (C) relative to $\{g_s, s \in T\}$ and
  $\beta$.
\end{definition}

Weak class (C) processes are considered extensively in Chang and Rao
{\cite{ChangRao1986}}. Oscillatory harmonizable processes affords this broad
class of processes to have a simple representation on $\mathbb{R}$ as seen in
the following:

\begin{proposition}
  \label{prop:class-c}The class of oscillatory weakly harmonizable processes
  $\{X (t), t \in \mathbb{R}\} \subset L_0^2 (P)$ coincides with the class of
  weak class (C) processes indexed on $\mathbb{R}$.
\end{proposition}

\begin{proof}
  This follows by setting $g_s (\lambda) = e^{i \lambda s} A (s, \lambda)$ in
  definition~\ref{def:weak-class-c} since $F (\cdummy, \cdummy)$ always has
  finite Fr{\'e}chet variation.
\end{proof}

Using this simple identification, an operator representation of weak class (C)
processes indexed on $\mathbb{R}$ is possible. This result is an extension of
that given in Chang and Rao {\cite{ChangRao1988}} for the oscillatory
stationary class.

\begin{theorem}
  \label{thm:operator-characterization}$X (\cdummy)$ is an oscillatory weakly
  harmonizable process iff it is representable as
  \begin{equation}
    \label{eq:operator-representation} X (t) = a (t) T (t) Y (0) \quad \forall
    t \in \mathbb{R},
  \end{equation}
  where $Y_0 = Y (0)$ is some point in
  \begin{equation}
    \label{eq:hilbert-space} H (X) = \text{sp} \{X (t), t \in \mathbb{R}\}
  \end{equation}
  with $a (t)$ a densely defined closed operator in $H (X)$ for each $t \in
  \mathbb{R}$ and $\{T (s), s \in \mathbb{R}\}$ a weakly continuous family of
  positive definite contractive operators in $H (X)$ which commutes with each
  $a (t), t \in \mathbb{R}$.
\end{theorem}

\begin{proof}
  Suppose $X (t)$ is oscillatory weakly harmonizable. Then
  \begin{equation}
    \label{eq:x-spectral} X (t) = \int_{\mathbb{R}} A (t, \lambda) e^{i
    \lambda t}  \hspace{0.17em} dZ (\lambda)
  \end{equation}
  where $Z (\cdummy)$ is a stochastic measure satisfying $E (Z (B_1)
  \overline{Z (B_2)}) = F (B_1, B_2)$ with $F (\cdummy, \cdummy)$ of bounded
  Fr{\'e}chet variation.
  
  Let
  \begin{equation}
    \label{eq:y-process} Y (t) = \int_{\mathbb{R}} e^{i \lambda t} 
    \hspace{0.17em} dZ (\lambda)
  \end{equation}
  Then $Y (\cdummy)$ is weakly harmonizable.
  
  Now by a theorem of Rao {\cite{Rao1982}} there is a weakly continuous family
  of positive definite contractive operators $\{T (t), t \in \mathbb{R}\}$ on
  $H (X) = \text{sp} \{X (t), t \in \mathbb{R}\}$ so that
  \begin{equation}
    \label{eq:y-representation} Y (t) = T (t) Y_0
  \end{equation}
  Using the spectral theorem for this family of operators (cf. Rao
  {\cite{Rao1982}}),
  \begin{equation}
    \label{eq:spectral-theorem} T (t) = \int_{\mathbb{R}} e^{i \lambda t} E (d
    \lambda), \quad t \in \mathbb{R}
  \end{equation}
  where $\{E (\cdot), \mathcal{B}\}$ is the resolution of the identity of $\{T
  (t), t \in \mathbb{R}\}$ with $\mathcal{B}$ as the Borel $\sigma$-algebra of
  $\mathbb{R}$. So
  \begin{equation}
    \label{eq:z-relation} Z (\Lambda) = E (\Lambda) Y_0, \quad \Lambda \in
    \mathcal{B}.
  \end{equation}
  Now define
  \begin{equation}
    \label{eq:operator-a} a (t) = \int_{\mathbb{R}} A (t, \lambda) E (d
    \lambda), \quad t \in \mathbb{R}.
  \end{equation}
  It follows that $a (t)$ is closed and densely defined on $H (Y)$ with its
  domain containing $\{Y (s), s \in \mathbb{R}\}$. Now since $T (t)$ and $E
  (D)$ commute for all $t$ and $D$, then $a (t)$ and $\{E (D), D \in
  \mathcal{B}\}$ commute so that $a (t)$ and $\{T (s), s \in \mathbb{R}\}$
  commute for each $t$.
  
  Thus
  
  \begin{align}
    X (t) & = \int_{\mathbb{R}} A (t, \lambda) e^{i \lambda t} 
    \hspace{0.17em} dZ (\lambda) \nonumber\\
    & = \int_{\mathbb{R}} A (t, \lambda) e^{i \lambda t} E (d \lambda) Y_0
    \nonumber\\
    & = \int_{\mathbb{R}} A (t, \lambda) E (d \lambda)  \int_{\mathbb{R}}
    e^{i \omega t} E (d \omega) Y_0 \nonumber\\
    & = a (t) T (t) Y_0 \nonumber\\
    & = a (t) T (t) Y (0)  \label{eq:final-representation}
  \end{align}
  
  where \eqref{eq:final-representation} follows since
  \begin{equation}
    E (D)  \int e^{i \omega t} Y_0 = \int_D e^{i \omega t} E (d \omega) Y_0, D
    \in \mathcal{B}
  \end{equation}
  Thus if $X (t)$ is oscillatory weakly harmonizable, then $X (t) = a (t) T
  (t) Y (0)$ where $Y_0 = Y (0)$ is some point in $H (X) = \text{sp} \{X (t),
  t \in \mathbb{R}\}$, $a (t)$ is a densely defined closed operator in $H (X)$
  for each $t \in \mathbb{R}$ and $\{T (s), s \in \mathbb{R}\}$ is a weakly
  continuous family of positive definite contractive operators in $H (X)$
  which commutes with each $a (t), t \in \mathbb{R}$.
  
  Now suppose $X (t)$ can be represented as $X (t) = a (t) T (t) Y (0)$ with
  $a (t)$, $T (t)$, and $Y (0)$ as stated in the theorem. Then, using a
  classical result of von Neumann and F. Riesz {\cite{RieszNagy1990}}, $a (t)$
  is a function $g (t, \cdot)$ of $T (t)$ and further
  \begin{equation}
    \label{eq:operator-a-function} a (t) = g (t, T (t)) = \int_{\mathbb{R}} g
    (t, \lambda) E (d \lambda)
  \end{equation}
  Thus
  
  \begin{align}
    X (t) & = a (t) T (t) Y (0) \nonumber\\
    & = \int_{\mathbb{R}} g (t, \lambda) E (d \lambda)  \int_{\mathbb{R}}
    e^{i \omega t} E (d \omega) Y_0 \nonumber\\
    & = \int_{\mathbb{R}} g (t, \lambda) E (d \lambda) Y_0 \nonumber\\
    & = \int_{\mathbb{R}} g (t, \lambda)  \hspace{0.17em} dZ (\lambda) 
    \label{eq:final-g representation}
  \end{align}
  
  but this is the representation of an oscillatory weakly harmonizable
  process.
\end{proof}

\section*{ACKNOWLEDGEMENTS}

The author expresses his thanks to Professor M. M. Rao for his advice and
encouragement during the work of this project. The author also expresses his
gratitude to the Mathematics department at Western Kentucky University for
release time during the Spring 1995 semester, during which this work was
completed.

\begin{thebibliography}{19}
  {\bibitem{ChangRao1986}}D. K. Chang and M. M. Rao, \tmtextit{Bimeasures and
  Nonstationary Processes}, in Real and Stochastic Analysis, John Wiley and
  Sons, New York, 1986, p.~7.
  
  {\bibitem{ChangRao1988}}D. K. Chang and M. M. Rao, \tmtextit{Special
  Representations of Weakly Harmonizable Processes}, Stochastic Analysis and
  Applications, 6(2):169, 1988.
  
  {\bibitem{GihmanSkorohod1974}}I. I. Gihman and A. V. Skorohod, \tmtextit{The
  Theory of Stochastic Processes I}, Springer-Verlag, New York, 1974.
  
  {\bibitem{Joyeux1987}}R. Joyeux, \tmtextit{Slowly Changing Processes and
  Harmonizability}, Journal of Time Series Analysis, 8, No.~4, 1987.
  
  {\bibitem{Kakihara}}Y. Kakihara, \tmtextit{Multidimensional Second Order
  Stochastic Processes}, World Scientific, In preparation.
  
  {\bibitem{Mehlman1992}}M. H. Mehlman, \tmtextit{Prediction and Fundamental
  Moving Averages for Discrete Multidimensional Harmonizable Processes},
  Journal of Multivariate Analysis, 43, No.~1, 1992.
  
  {\bibitem{Priestley1981}}M. B. Priestley, \tmtextit{Spectral Analysis and
  Time Series}, Vol.~1 and 2, Academic Press, London, 1981.
  
  {\bibitem{Rao1978}}M. M. Rao, \tmtextit{Covariance Analysis of Non
  Stationary Time Series}, Developments in Statistics, 1, p.~171, 1978.
  
  {\bibitem{Rao1982}}M. M. Rao, \tmtextit{Harmonizable Processes: Structure
  Theory}, L'Enseignement Math{\'e}matique, 28, p.~295, 1982.
  
  {\bibitem{Rao1989}}M. M. Rao, \tmtextit{Harmonizable Signal Extraction,
  Filtering and Sampling}, in Topics in Non-Gaussian Signal Processing, (E. J.
  Wegman, S. C. Schwartz, J. B. Thomas, eds.), Springer-Verlag, New York,
  1989.
  
  {\bibitem{Rao1991}}M. M. Rao, \tmtextit{Sampling and Prediction for
  Harmonizable Isotropic Random Fields}, Journal of Combinatorics, Information
  and System Sciences, 16, No.~2-3, p.~207, 1991.
  
  {\bibitem{Rao1994}}M. M. Rao, \tmtextit{Harmonizable processes and
  inference: unbiased prediction for stochastic flows}, Journal of Statistical
  Planning and Inference, 39, p.~187, 1994.
  
  {\bibitem{RieszNagy1990}}F. Riesz and B. Sz-Nagy, \tmtextit{Functional
  Analysis}, Dover, New York, 1990.
  
  {\bibitem{Swift1994}}R. Swift, \tmtextit{The Structure of Harmonizable
  Isotropic Random Fields}, Stochastic Analysis and Applications, 12, No.~5,
  p.~583, 1994.
  
  {\bibitem{Swift1995}}R. Swift, \tmtextit{Representation and Prediction for
  Locally Harmonizable Isotropic Random Fields}, Journal of Applied
  Mathematics and Stochastic Analysis, VIII, p.~101, 1995.
  
  {\bibitem{Swift1996a}}R. Swift, \tmtextit{A Class of Harmonizable Isotropic
  Random Fields}, Journal of Combinatorics, Information and System Sciences,
  (to appear), 1996.
  
  {\bibitem{Swift1996b}}R. Swift, \tmtextit{Almost Periodic Harmonizable
  Processes}, Georgian Mathematical Journal, (to appear), 1996.
  
  {\bibitem{Swift1996c}}R. Swift, \tmtextit{Stochastic Processes with
  Harmonizable Increments}, Journal of Combinatorics, Information and System
  Sciences, (to appear), 1996.
  
  {\bibitem{Yaglom1987}}A. M. Yaglom, \tmtextit{Correlation Theory of
  Stationary and Related Random Functions}, Vol.~1 and 2, Springer-Verlag, New
  York, 1987.
\end{thebibliography}

\end{document}
