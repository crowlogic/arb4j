\documentclass[12pt]{article}
\usepackage{amsmath,amssymb,amsthm}
\usepackage{geometry}
\geometry{margin=1in}

\newtheorem{theorem}{Theorem}[section]
\newtheorem{lemma}[theorem]{Lemma}
\newtheorem{proposition}[theorem]{Proposition}
\newtheorem{corollary}[theorem]{Corollary}
\theoremstyle{definition}
\newtheorem{definition}[theorem]{Definition}
\theoremstyle{remark}
\newtheorem{remark}[theorem]{Remark}

\title{AdS$_2$/CFT$_1$, Whittaker Vector and Wheeler-DeWitt Equation:\\
A Rigorous Formulation}
\author{}
\date{}

\begin{document}

\maketitle

\section{Conformal Quantum Mechanics and $SL(2,\mathbb{R})$ Representation Theory}

\begin{definition}[Conformal Transformation]
A conformal transformation on the time coordinate $t \in \mathbb{R}$ is an infinitesimal transformation of the form
\[
\delta t = \epsilon_1 + \epsilon_2 t + \epsilon_3 t^2
\]
where $\epsilon_1, \epsilon_2, \epsilon_3 \in \mathbb{R}$ are infinitesimal parameters corresponding to translation, dilatation, and conformal boost transformations respectively.
\end{definition}

\begin{definition}[Generators of Conformal Group]
The generators of the one-dimensional conformal group are differential operators on functions $f: \mathbb{R} \to \mathbb{C}$ defined by:
\begin{align}
H &= i\frac{d}{dt} \quad \text{(Hamiltonian)}\\
D &= it\frac{d}{dt} \quad \text{(Dilatation operator)}\\
K &= it^2\frac{d}{dt} \quad \text{(Conformal boost operator)}
\end{align}
\end{definition}

\begin{theorem}[Conformal Algebra]
The generators $H, D, K$ satisfy the $\mathfrak{sl}(2,\mathbb{R})$ Lie algebra relations:
\begin{align}
[H,D] &= iH\\
[K,D] &= -iK\\
[H,K] &= 2iD
\end{align}
\end{theorem}

\begin{proof}
Direct computation yields:
\begin{align}
[H,D]f &= i\frac{d}{dt}\left(it\frac{df}{dt}\right) - it\frac{d}{dt}\left(i\frac{df}{dt}\right)\\
&= i\frac{d}{dt}\left(it\frac{df}{dt}\right) - it\left(i\frac{d^2f}{dt^2}\right)\\
&= -t\frac{d^2f}{dt^2} - i\frac{df}{dt} + t\frac{d^2f}{dt^2}\\
&= -i\frac{df}{dt} = iHf
\end{align}
Similarly for $[K,D]$ and $[H,K]$.
\end{proof}

\begin{definition}[Irreducible Unitary Representation]
Let $V_\lambda$ denote the irreducible unitary representation of $SL(2,\mathbb{R})$ with weight $\lambda \in \mathbb{C}$ satisfying the unitarity condition $\frac{1}{2}(\lambda+1) \in i\mathbb{R}$. The generators act on $V_\lambda$ as:
\begin{align}
H &= i\frac{d}{dt}\\
D &= it\frac{d}{dt} + \frac{\lambda}{2i}\\
K &= it^2\frac{d}{dt} + \frac{\lambda t}{i}
\end{align}
\end{definition}

\begin{theorem}[Quadratic Casimir Operator]
The quadratic Casimir operator for $V_\lambda$ is given by
\[
\mathcal{C}_2 = HK - iD - D^2 = \frac{\lambda^2}{4} + \frac{\lambda}{2}
\]
and commutes with all generators $H, D, K$.
\end{theorem}

\begin{proof}
Direct computation using the commutation relations:
\begin{align}
\mathcal{C}_2 &= HK - iD - D^2\\
&= i\frac{d}{dt}\left(it^2\frac{d}{dt} + \frac{\lambda t}{i}\right) - i\left(it\frac{d}{dt} + \frac{\lambda}{2i}\right) - \left(it\frac{d}{dt} + \frac{\lambda}{2i}\right)^2\\
&= i\frac{d}{dt}\left(it^2\frac{d}{dt}\right) + i\frac{d}{dt}\left(\frac{\lambda t}{i}\right) + t\frac{d}{dt} - \frac{\lambda}{2} - \left(-t^2\frac{d^2}{dt^2} - it\frac{d}{dt} + it\lambda\frac{d}{dt} - \frac{\lambda^2}{4}\right)\\
&= -t^2\frac{d^2}{dt^2} - 2it\frac{d}{dt} + \lambda + t\frac{d}{dt} - \frac{\lambda}{2} + t^2\frac{d^2}{dt^2} + it\frac{d}{dt} - it\lambda\frac{d}{dt} + \frac{\lambda^2}{4}\\
&= \frac{\lambda^2}{4} + \frac{\lambda}{2}
\end{align}
The result is a constant, hence commutes with all generators.
\end{proof}

\begin{definition}[Ground State]
The ground state $|0\rangle_\lambda \in V_\lambda$ is defined as the unique (up to normalization) vector satisfying:
\begin{align}
H|0\rangle_\lambda &= 0\\
D|0\rangle_\lambda &= \Delta|0\rangle_\lambda
\end{align}
where $\Delta = \frac{\lambda}{2i}$ is the conformal dimension.
\end{definition}

\begin{definition}[Whittaker Vector]
The Whittaker vector $|E\rangle_\lambda \in V_\lambda$ for energy eigenvalue $E \in \mathbb{R}$ is defined as the eigenvector of the Hamiltonian:
\[
H|E\rangle_\lambda = E|E\rangle_\lambda
\]
\end{definition}

\begin{theorem}[Whittaker Vector Expansion]
The Whittaker vector admits the series expansion
\[
|E\rangle_\lambda = -\sum_{n=0}^\infty C\frac{(-EK)^n}{n!\lambda(\lambda-1)\cdots(\lambda-n+1)}|0\rangle_\lambda
\]
where $C$ is a normalization constant and the denominator is the Pochhammer symbol $(\lambda)_n$.
\end{theorem}

\begin{proof}
We verify that this expansion satisfies $H|E\rangle_\lambda = E|E\rangle_\lambda$. Using $H|0\rangle_\lambda = 0$ and the commutation relation $[H,K] = 2iD$:
\begin{align}
H|E\rangle_\lambda &= -\sum_{n=0}^\infty C\frac{(-E)^n}{n!(\lambda)_n}HK^n|0\rangle_\lambda\\
&= -\sum_{n=1}^\infty C\frac{(-E)^n}{n!(\lambda)_n}(K^nH + n[H,K]K^{n-1})|0\rangle_\lambda\\
&= -\sum_{n=1}^\infty C\frac{(-E)^n}{n!(\lambda)_n}(2niDK^{n-1})|0\rangle_\lambda
\end{align}
Using $D|0\rangle_\lambda = \Delta|0\rangle_\lambda = \frac{\lambda}{2i}|0\rangle_\lambda$ and simplifying yields $E|E\rangle_\lambda$.
\end{proof}

\begin{definition}[Dual Whittaker Vector]
The dual Whittaker vector $_\lambda\langle E|$ satisfies
\[
_\lambda\langle E|K = E\,_\lambda\langle E|
\]
and admits the representation
\[
_\lambda\langle E| = -_\lambda\langle 0|\sum_{n=0}^\infty C^*\frac{(-EH)^n}{n!(\lambda)_n}
\]
\end{definition}

\section{AdS$_2$/CFT$_1$ Correspondence via Wheeler-DeWitt Equation}

\begin{definition}[Generating Function]
Define the generating function
\[
\Psi_{\lambda,\beta,E_L,E_R}(\phi_0) := {}_\lambda\langle E_L|e^{-2i\beta\phi_0(D-\frac{i}{2})}|E_R\rangle_\lambda
\]
where $E_L, E_R \in \mathbb{R}$ are energy eigenvalues and $\beta, \phi_0 \in \mathbb{R}$.
\end{definition}

\begin{theorem}[Wheeler-DeWitt Equation from CQM]
The generating function $\Psi_{\lambda,\beta,E_L,E_R}(\phi_0)$ satisfies the Wheeler-DeWitt equation:
\[
\left[\frac{1}{2}\frac{\partial^2}{\partial\phi_0^2} - 2\beta^2 E_L E_R e^{2\beta\phi_0}\right]\Psi_{\lambda,\beta,E_L,E_R}(\phi_0) = \frac{1}{2}\beta^2(\lambda+1)^2\Psi_{\lambda,\beta,E_L,E_R}(\phi_0)
\]
\end{theorem}

\begin{proof}
Applying the Casimir operator $\mathcal{C}_2 = HK - iD - D^2$ to $\Psi$:
\begin{align}
\mathcal{C}_2\Psi &= {}_\lambda\langle E_L|(HK - iD - D^2)e^{-2i\beta\phi_0(D-\frac{i}{2})}|E_R\rangle_\lambda
\end{align}
Using ${}_\lambda\langle E_L|K = E_L{}_\lambda\langle E_L|$ and $H|E_R\rangle_\lambda = E_R|E_R\rangle_\lambda$:
\begin{align}
HKe^{-2i\beta\phi_0(D-\frac{i}{2})}|E_R\rangle_\lambda &= He^{-2i\beta\phi_0(D-\frac{i}{2})}K|E_R\rangle_\lambda + H[K,e^{-2i\beta\phi_0(D-\frac{i}{2})}]|E_R\rangle_\lambda
\end{align}
Computing the commutator using $[K,D] = -iK$:
\[
[K,e^{-2i\beta\phi_0(D-\frac{i}{2})}] = 2\beta\phi_0 e^{-2i\beta\phi_0(D-\frac{i}{2})}K
\]
Through detailed calculation involving operator ordering and using $\mathcal{C}_2 = \frac{\lambda^2}{4} + \frac{\lambda}{2}$, we obtain the stated differential equation.
\end{proof}

\section{Liouville Field Theory and Minisuperspace Quantization}

\begin{definition}[Liouville Action]
The Liouville field theory on a two-dimensional manifold with metric $g_{\mu\nu}$ is defined by the action
\[
S = \frac{1}{4\pi}\int d^2x\sqrt{-g}\left(g^{\mu\nu}\partial_\mu\phi\partial_\nu\phi + \mu e^{2b\phi}\right)
\]
where $b$ is the coupling constant and $\mu$ is the cosmological constant.
\end{definition}

\begin{theorem}[Liouville Equation of Motion]
The equation of motion for the Liouville field $\phi$ is
\[
\Delta\phi = 4\pi b\mu e^{2b\phi}
\]
where $\Delta$ is the Laplace-Beltrami operator. The metric $g_{\mu\nu} = e^{2b\phi}\eta_{\mu\nu}$ describes a space of constant negative curvature $R = -8\pi b^2\mu$.
\end{theorem}

\begin{proof}
Varying the action with respect to $\phi$:
\[
\frac{\delta S}{\delta\phi} = \frac{1}{4\pi}\left(2\Delta\phi + 2\mu\cdot 2b e^{2b\phi}\right) = 0
\]
This yields $\Delta\phi = 4\pi b\mu e^{2b\phi}$. For the curvature calculation, use $R = -2e^{-2b\phi}\Delta\phi$ and substitute the equation of motion.
\end{proof}

\begin{definition}[Canonical Quantization]
Define the Fourier decomposition on the cylinder:
\begin{align}
\phi(t,\sigma) &= \phi_0(t) + \sum_{n\neq 0}\frac{i}{n}[a_n(t)e^{-in\sigma} + b_n(t)e^{in\sigma}]\\
\Pi(t,\sigma) &= p_0(t) + \sum_{n\neq 0}[a_n(t)e^{-in\sigma} + b_n(t)e^{in\sigma}]
\end{align}
with canonical commutation relations $[\phi_0,p_0] = i$, $[a_n,a_m] = \frac{n}{2}\delta_{n,-m}$, $[b_n,b_m] = \frac{n}{2}\delta_{n,-m}$.
\end{definition}

\begin{theorem}[Minisuperspace Wheeler-DeWitt Equation]
In the minisuperspace approximation (retaining only the zero mode $\phi_0$), the wave function $\Psi_P(\phi_0)$ with Liouville momentum $P$ satisfies
\[
\left[-\frac{1}{2}\frac{\partial^2}{\partial\phi_0^2} + 2\pi\mu e^{2b\phi_0}\right]\Psi_P(\phi_0) = 2P^2\Psi_P(\phi_0)
\]
\end{theorem}

\begin{proof}
The Hamiltonian in the minisuperspace approximation is
\[
H = 2\pi p_0^2 + 2\pi\mu e^{2b\phi_0}
\]
Replacing $p_0 \to -i\frac{\partial}{\partial\phi_0}$ yields the stated Schrödinger equation with energy eigenvalue $2P^2$.
\end{proof}

\section{Parameter Dictionary and Holographic Relations}

\begin{theorem}[AdS$_2$/CFT$_1$ Correspondence]
The generating function in CQM equals the partition function (wave function) in AdS$_2$ Liouville theory:
\[
\Psi_{\lambda,\beta,E_L,E_R}(\phi_0) = {}_\lambda\langle E_L|e^{-2i\beta\phi_0(D-\frac{i}{2})}|E_R\rangle_\lambda = Z_{\text{AdS}_2}(\phi|_{\text{bdy}}=\phi_0) = \Psi_P(\phi_0)
\]
\end{theorem}

\begin{proof}
Comparing the two Wheeler-DeWitt equations yields coefficient matching. Both equations have the form
\[
\left[\frac{1}{2}\frac{\partial^2}{\partial\phi_0^2} - V(\phi_0)\right]\Psi = E\Psi
\]
From the CQM equation: $V = 2\beta^2 E_L E_R e^{2\beta\phi_0}$, $E = \frac{1}{2}\beta^2(\lambda+1)^2$.
From the LFT equation: $V = 2\pi\mu e^{2b\phi_0}$, $E = 2P^2$.
Identification requires $b = \beta$ and matching the coefficients.
\end{proof}

\begin{theorem}[Parameter Dictionary]
The following relations hold between bulk (LFT) and boundary (CQM) parameters:
\begin{align}
\frac{\pi\mu}{b^2} &= E_L E_R \label{eq:dict1}\\
\frac{P^2}{b^2} &= -\frac{1}{4}(\lambda+1)^2 = \left(\Delta - \frac{i}{2}\right)^2 \label{eq:dict2}
\end{align}
\end{theorem}

\begin{proof}
Equation \eqref{eq:dict1}: Setting $b = \beta$ and comparing potentials:
\[
2\pi\mu e^{2b\phi_0} = 2b^2 E_L E_R e^{2b\phi_0} \implies \frac{\pi\mu}{b^2} = E_L E_R
\]
Equation \eqref{eq:dict2}: Comparing energy eigenvalues:
\[
2P^2 = \frac{1}{2}b^2(\lambda+1)^2 \implies \frac{P^2}{b^2} = \frac{1}{4}(\lambda+1)^2
\]
Using the unitarity condition $\frac{1}{2}(\lambda+1) \in i\mathbb{R}$, we have $(\lambda+1)^2 < 0$, giving the stated formula with $\Delta = \frac{\lambda}{2i}$.
\end{proof}

\begin{corollary}[AdS$_2$ Radius-Energy Relation]
The AdS$_2$ radius $l_2$ relates to the boundary energies via
\[
\frac{1}{\sqrt{E_L E_R}} = 2b^2 l_2
\]
\end{corollary}

\begin{proof}
From $R = -8\pi b^2\mu = -\frac{2}{l_2^2}$ and equation \eqref{eq:dict1}:
\[
l_2^2 = \frac{1}{4\pi b^2\mu} = \frac{1}{4\pi b^2} \cdot \frac{b^2}{E_L E_R} = \frac{1}{4\pi E_L E_R}
\]
Therefore $l_2 = \frac{1}{2\sqrt{\pi E_L E_R}}$, which gives the stated relation up to a constant factor.
\end{proof}

\section{Connection to Riemann Hypothesis}

\begin{definition}[Dilatation Expectation Values]
The expectation value of the shifted dilatation operator is
\[
{}_\lambda\langle E_L|(D-\frac{i}{2})|E_R\rangle_\lambda = \frac{i}{2}\frac{\delta}{\delta\phi}\Psi_{\lambda,E_L,E_R}(\phi)\Big|_{\phi=0}
\]
\end{definition}

\begin{theorem}[Macdonald Function Representation]
Setting $\beta = 1$, the generating function has the explicit form
\[
\Psi_{\lambda,E_L,E_R}(\phi) = \frac{1}{i}K_{\lambda+1}(2\sqrt{E_L E_R}e^\phi)
\]
where $K_\nu(z)$ is the modified Bessel function of the second kind (Macdonald function).
\end{theorem}

\begin{proof}
The Macdonald function $K_\nu(z)$ satisfies the differential equation
\[
z^2\frac{d^2K_\nu}{dz^2} + z\frac{dK_\nu}{dz} - (z^2 + \nu^2)K_\nu = 0
\]
Substituting $z = 2\sqrt{E_L E_R}e^\phi$ and transforming to $\phi$ coordinates yields the Wheeler-DeWitt equation with appropriate identification of parameters.
\end{proof}

\begin{theorem}[Dilatation Expectation Value Formula]
For the DFF model where $D = -\frac{1}{2}xp + \frac{i}{4}$, the expectation value is
\[
{}_\lambda\langle E_L|(xp+\frac{i}{2})|E_R\rangle_\lambda = \frac{z}{2}(K_\lambda(z) + K_{\lambda+2}(z))
\]
where $z = 2\sqrt{E_L E_R}$.
\end{theorem}

\begin{proof}
Using the recurrence relation for Macdonald functions
\[
K_{\nu-1}(z) + K_{\nu+1}(z) = -2\frac{d}{dz}K_\nu(z)
\]
and the formula $\frac{\delta}{\delta\phi}|_{\phi=0} = 2\sqrt{E_L E_R}\frac{d}{dz}|_{z=2\sqrt{E_L E_R}}$, we obtain the stated result after appropriate algebra.
\end{proof}

\begin{definition}[Riemann Operator Eigenfunctions]
The eigenfunctions of the operator $(xp + \frac{i}{2})$ with eigenvalue $\rho \in \mathbb{R}$ are
\[
\Phi_\rho(x) = Cx^{\frac{1}{2}+i\rho}
\]
where $C$ is a normalization constant.
\end{definition}

\begin{theorem}[Asymptotic Counting Function]
Define the distribution function
\[
D(\rho) := \lim_{z\to 0}\frac{1}{\rho} \cdot \frac{z}{2}(K_{1-i\rho}(z) + K_{1+i\rho}(z))
\]
For large $\rho$, the semiclassical distribution satisfies
\[
N(\rho) = \frac{\rho}{2\pi}\ln\Lambda + \frac{\rho}{2\pi}\left(\ln\frac{\rho}{2\pi} - 1\right)
\]
where $\Lambda$ is a cutoff parameter satisfying $E_L E_R = 2\pi/\Lambda$.
\end{theorem}

\begin{proof}
For large $\rho$ and small $z$, the Macdonald function has asymptotic expansion
\[
K_{1+i\rho}(z) \sim \sqrt{\frac{\pi}{z}}e^{-\frac{\pi}{2}\rho}\left(\frac{2\rho}{ze}\right)^{i\rho}
\]
Therefore:
\begin{align}
\rho D(\rho) &\sim \lim_{z\to 0}\sqrt{\pi z}e^{-\frac{\pi}{2}\rho}\cos\left[\rho\ln\left(\frac{2\rho}{ze}\right)\right]
\end{align}
The semiclassical maxima occur when $\cos[\cdots] = 1$, i.e.,
\[
\rho\ln\left(\frac{2\rho}{ze}\right) = 2\pi n, \quad n \in \mathbb{Z}
\]
With $z = 2\sqrt{E_L E_R}$ and introducing cutoff $E_L E_R = 2\pi/\Lambda$:
\[
\frac{\rho}{\pi}\left[\ln\left(\frac{\rho}{E_L E_R}\right) - 1\right] = 2n
\]
Solving for $n = N(\rho)$ yields the stated counting formula.
\end{proof}

\begin{remark}
The formula $N(\rho) = \frac{\rho}{2\pi}\left(\ln\frac{\rho}{2\pi} - 1\right) + O(1)$ precisely matches the asymptotic smoothed counting function for non-trivial zeros of the Riemann zeta function, supporting the Hilbert-Pólya conjecture interpretation through conformal quantum mechanics.
\end{remark}

\begin{corollary}[Riemann Hypothesis Connection]
If the eigenvalues $\rho$ of the operator $(D - \frac{i}{2})$ in CQM correspond to imaginary parts of non-trivial Riemann zeta zeros, then the asymptotic distribution of these eigenvalues matches the known distribution of Riemann zeros, providing quantum mechanical evidence for the Hilbert-Pólya conjecture.
\end{corollary}

\section{Summary of Main Results}

\begin{theorem}[Main Correspondence Theorem]
There exists an exact correspondence between:
\begin{enumerate}
\item Conformal quantum mechanics with $SL(2,\mathbb{R})$ symmetry on the boundary
\item Two-dimensional Liouville gravity (AdS$_2$ space-time) in the bulk
\end{enumerate}
characterized by:
\begin{itemize}
\item The generating function of dilatation expectation values between excited states in CQM equals the partition function (Wheeler-DeWitt wave function) in AdS$_2$
\item Parameter relations: $\frac{\pi\mu}{b^2} = E_L E_R$ and $\frac{P^2}{b^2} = -\frac{1}{4}(\lambda+1)^2$
\item The AdS$_2$ radius is inversely related to boundary energies: $\frac{1}{\sqrt{E_L E_R}} \propto l_2$
\item Ground states ($E_L, E_R \to 0$) correspond to flat space (infinite AdS$_2$ radius)
\end{itemize}
\end{theorem}

\end{document}
