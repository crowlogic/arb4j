\documentclass{article}
\usepackage[english]{babel}
\usepackage{geometry,amsmath,amssymb,hyperref}
\geometry{letterpaper}

%%%%%%%%%% Start TeXmacs macros
\newcommand{\assign}{:=}
\newcommand{\nobracket}{}
\newcommand{\tmrsub}[1]{\ensuremath{_{\textrm{#1}}}}
\newcommand{\tmtextbf}[1]{\text{{\bfseries{#1}}}}
\newcommand{\tmtextit}[1]{\text{{\itshape{#1}}}}
\newcommand{\tmtexttt}[1]{\text{{\ttfamily{#1}}}}
%%%%%%%%%% End TeXmacs macros

\begin{document}

\begin{center}
  {\Large AdS\tmrsub{$2$}/CFT\tmrsub{$1$}, Whittaker vector and Wheeler-DeWitt
  equation}
  
  Tadashi Okazaki\footnote{tadashiokazaki@phys.ntu.edu.tw}
  
  \tmtextit{Department of Physics and Center for Theoretical Sciences
  
  National Taiwan University, Taipei 10617, Taiwan }
\end{center}

\begin{abstract}
  We study the energy representation of conformal quantum mechanics as the
  Whittaker vector without specifying classical Lagrangian. We show that a
  generating function of expectation values among two excited states of the
  dilatation operator in conformal quantum mechanics is a solution to the
  Wheeler-DeWitt equation and it corresponds to the AdS\tmrsub{$2$} partition
  function evaluated as the minisuperspace wave function in Liouville field
  theory. We also show that the dilatation expectation values in conformal
  quantum mechanics lead to the asymptotic smoothed counting function of the
  Riemann zeros.
\end{abstract}

The holographic principle {\cite{'tHooft:1993gx,Susskind:1994vu}} states that
quantum gravity on $(d + 1)$-dimensional manifold can be described by a theory
on its $d$-dimensional boundary. The AdS\tmrsub{$d + 1$}/CFT\tmrsub{$d$}
correspondence {\cite{Maldacena:1997re}} which is one of the greatest
productions of string theory provides the most successful realization as the
relationship between effective gauge theories of the brane dynamics and string
theory on the near horizon AdS geometry. The AdS\tmrsub{$2$}/CFT\tmrsub{$1$}
could conceivably be the most significant case in that all the extremal black
holes contain an AdS\tmrsub{$2$} factor in their near horizon geometry
{\cite{Kunduri:2007vf,Figueras:2008qh}}. In spite of a lot of interesting
works
{\cite{Strominger:1998yg,Maldacena:1998uz,Nakatsu:1998st,Townsend:1998qp,Spradlin:1999bn,Cadoni:1999ja,Blum:1999pc,NavarroSalas:1999up,Caldarelli:2000xk,Cadoni:2000gm,Bellucci:2002va,Strominger:2003tm,Leiva:2003kd,Giveon:2004zz,Azeyanagi:2007bj,Alishahiha:2008tv,Gupta:2008ki,Hartman:2008dq,Galajinsky:2008ce,Sen:2008vm,Sen:2008yk,Sen:2011cn,Chamon:2011xk,MolinaVilaplana:2012xe,Jackiw:2012ur}}
it would be fair to say that we have not yet gained a fully satisfactory
understanding of the correspondence due to the peculiarities of the
AdS\tmrsub{$2$}/CFT\tmrsub{$1$}, including the fact that only the
AdS\tmrsub{$2$} has two disconnected boundaries and it is a long-standing
question whether the dual CFT\tmrsub{$1$} description is a single CFT or two
systems living on the two boundaries.

In this letter we study conformal quantum mechanics (CQM) without specifying
classical Lagrangian description. One important consequence is a generic
evidence of the AdS\tmrsub{$2$}/CFT\tmrsub{$1$} as the relationship between a
generating function of the dilatation expectation values in the boundary CQM
and a partition function of the bulk AdS\tmrsub{$2$}. We show that the
expectation values are not associated with the ground state but with two
excited states in the correspondence. Our result is in favor of the statement
{\cite{Zamolodchikov:2001ah,Sen:2008vm,Balasubramanian:1998sn,Balasubramanian:1998de,Balasubramanian:2010ys}}
that the dual CFT\tmrsub{$1$}'s on two boundaries of AdS\tmrsub{$2$}
space-time are excited and entangled. We also claim that the energies on the
boundary CQM would be responsible for the AdS\tmrsub{$2$} radius so that the
ground state would correspond to a flat space in the bulk with an infinitely
large AdS\tmrsub{$2$} radius.

Another intriguing thing is the speculative implication of the resulting
expectation values of the dilatation operator in CQM. When we consider the
DFF-model {\cite{deAlfaro:1976je}} as CQM, the dilatation operator takes the
form of $xp$. Such operator has been proposed as a strong candidate for the
realization of the Hilbert-P{\'o}lya conjecture that the imaginary part of the
non-trivial Riemann zeros are eigenvalues of a self-adjoint operator (see
{\cite{MR1694895,berry1999h,MR1684543,MR2443603,Sierra:2011tb,MR2812337}} and
references therein). Unlike a lot of efforts undertaken thus far we here
propose a novel approach to obtain the counts of the Riemann zeros from CQM
point of view. The fact that the operator $xp$ is the dilatation operator in
CQM rather than the Hamiltonian enables us to jump into a fairly general
setting beyond the operator with the form of $xp$. We show that the
expectation values of the dilatation operator in CQM naturally lead to the
asymptotic form of the smoothed counting function of the Riemann zeros.

We shall begin by considering CQM that is invariant under the conformal
symmetric transformation of the time coordinate $t$ {\cite{deAlfaro:1976je}}

\begin{align}
  \delta t & = \epsilon_1 + \epsilon_2 t + \epsilon_3 t^2  \label{conf1c}
\end{align}

where $\epsilon_1$, $\epsilon_2$ and $\epsilon_3$ are identified with the
infinitesimal parameters of the translation, the dilatation and the conformal
boost transformation respectively. The corresponding generators, the
Hamiltonian $H = id / dt$, the dilatation operator $D = it (d / dt)$, and the
conformal boost operator $K = it^2  (d / dt)$ obey the commutation relations

\begin{align}
  {}[H, D] & = iH  \label{conf1e1}\\
  {}[K, D] & = - iK  \label{conf1e2}\\
  {}[H, K] & = 2 iD  \label{conf1e3}
\end{align}

of $SL (2, \mathbb{R})$, which we call the one-dimensional conformal group.
Hence the Hilbert space of conformal quantum mechanical system exhibits the
$SL (2, \mathbb{R})$ conformal symmetry and the physical states would be
classified by its irreducible representation. Since we wish to make all
integrals convergent, we require the unitarity of the representation. The
classification of the irreducible unitary representations of $SL (2,
\mathbb{R})$ was studied in {\cite{Bargmann:1946me}} and the irreducible
unitary representation of $SL (2, \mathbb{R})$ conformal group as a function
of the continuous time coordinate $t$ can be generalized by taking the
principal spherical series of the representation which is induced by the
one-dimensional representation of the Borel subgroup. Let $V_{\lambda}$ be a
set of the irreducible unitary representations of $SL (2, \mathbb{R})$ with
weight $\lambda$. Then the Hamiltonian $H$, the dilatation operator $D$ and
the conformal boost operator $K$ can be expressed as

\begin{align}
  H & = i \frac{d}{dt}  \label{conf1d1}\\
  D & = it \frac{d}{dt} + \frac{\lambda}{2 i}  \label{conf1d2}\\
  K & = it^2  \frac{d}{dt} + \frac{1}{i} \lambda t  \label{conf1d3}
\end{align}

satisfying the commutation relations (\ref{conf1e1}), (\ref{conf1e2}) and
(\ref{conf1e3}). The unitarity implies that $\frac{1}{2}  (\lambda + 1)$ is
pure imaginary. The finite conformal transformation is

\begin{align}
  t^n & \rightarrow \frac{(\alpha t + \beta)^n}{(\gamma t + \delta)^{\lambda +
  n}}  \label{conf1a}
\end{align}

where the parameters $\alpha$, $\beta$, $\gamma$ and $\delta$ are the elements
of real two by two matrices with determinant one

\begin{align}
  A & = \left( \begin{array}{cc}
    \alpha & \gamma\\
    \beta & \delta
  \end{array} \right), & \alpha \delta - \beta \gamma & = 1  \label{conf1b}
\end{align}

which forms the one-dimensional conformal group $SL (2, \mathbb{R})$. The
quadratic Casimir operator is given by

\begin{align}
  \label{conf1e} \mathcal{C}_2 & = HK - iD - D^2 \nonumber\\
  & = \frac{\lambda^2}{4} + \frac{\lambda}{2} 
\end{align}

Let us define the ground state $|0 \rangle_{\Delta}$ by

\begin{align}
  H| 0 \rangle_{\Delta} & = 0  \label{conf1f1}\\
  D| 0 \rangle_{\Delta} & = \Delta |0 \rangle_{\Delta}  \label{conf1f2}
\end{align}

From the equations (\ref{conf1e})-(\ref{conf1f2}) we see that the eigenvalue
$\Delta$ of the ground state $|0 \rangle_{\Delta}$ is $\Delta =
\frac{\lambda}{2 i}$ and we thus write the ground state as $|0
\rangle_{\lambda}$. Now consider the energy eigenstate obeying

\begin{align}
  H|E \rangle = E|E \rangle .  \label{conf1f3}
\end{align}

This eigenvector $|E \rangle$ is known as the Whittaker vector in the
representation theory of $SL (2, \mathbb{R})$
{\cite{MR0271275,MR0311838,MR507800}}. Given eigenvalue $E$ and irreducible
representation with weight $\lambda$, there is a unique Whittaker vector and
we can write

\begin{align}
  |E \rangle_{\lambda} & = - \sum_k \sum_n C_k  \frac{(- EK)^n}{n! \lambda
  (\lambda - 1) \cdots (\lambda - n + 1)} |0 \rangle_{\lambda, k} 
  \label{conf1g3}
\end{align}

where $k$ parametrizes the degenerate ground states which might arise from
some symmetries in the theories and $C_k$ is the weighted coefficient for
$k$-th ground state. For simplicity let us assume the vanishing of tunneling
amplitudes; $_{\lambda, l} \langle 0|0 \rangle_{\lambda, k} = \delta_{l, k}$
and the normalization of the ground states; $\sum_k |C_k |^2 = 1$ so that we
shall omit the indices $k$. Correspondingly we can take the dual energy
eigenstate vector $_{\lambda} \langle E|$ as the dual Whittaker vector which
satisfies the relation \footnote{The choice of pair of the Whittaker vector
and the dual Whittaker vector results in the consistent Hamiltonian reduction.
}

\begin{align}
  _{\lambda} \langle E|K & = E_{\lambda} \langle E|  \label{conf1h1}
\end{align}

and it can be represented by

\begin{align}
  _{\lambda} \langle E| \assign -_{\lambda} \langle 0| \sum_{k, n} C_k^{\ast} 
  \frac{(- EH)^n}{n! \lambda (\lambda - 1) \cdots (\lambda - n + 1)} 
  \label{conf1h2}
\end{align}

With the Whittaker vector $|E \rangle_{\lambda}$ and its dual vector
$_{\lambda} \langle E|$ as the energy eigenstate vector and its dual vector in
conformal quantum mechanics, we will consider the situation where the theory
is coupled to two-dimensional bulk theory to investigate the
AdS\tmrsub{$2$}/CFT\tmrsub{$1$} correspondence. Let us consider the function
defined by

\begin{align}
  W_{\lambda, E_L, E_R} (\phi) & \assign_{\lambda} \langle E_L |e^{- 2 i \phi
  D} |E_R \rangle_{\lambda}  \label{conf1h3}
\end{align}

where $E_L, E_R$ are the eigenvalues of the Whittaker vector (\ref{conf1f3})
and the dual Whittaker vector (\ref{conf1h1}) respectively. Here $\phi$ is
regarded as the restriction of some bulk field in AdS\tmrsub{$2$} space-time
on the boundary that is coupled to the dilatation operator $D$ in conformal
quantum mechanics on the boundaries. By acting the quadratic Casimir operator
(\ref{conf1e}) on the function (\ref{conf1h3}), one obtains the differential
equation

\begin{align}
  \left[ \frac{1}{2}  \frac{\partial^2}{\partial \phi^2} -
  \frac{\partial}{\partial \phi} - 2 E_L E_R e^{2 \phi} \right] W_{\lambda,
  E_L, E_R} (\phi) & = \left( \frac{\lambda^2}{2} + \lambda \right)
  W_{\lambda, E_L, E_R} (\phi)  \label{conf1h4}
\end{align}

The function $W_{\lambda, E_L, E_R} (\phi)$ is known as the $SL (2,
\mathbb{R})$ Whittaker function
{\cite{MR0271275,MR0311838,MR507800,MR1729357}}. At first sight one might
expect that the Whittaker function $W_{\lambda, E_L, E_R}$ plays a role of the
generating function of the expectation values of the operator $D$. However,
the unitarity asserts that the eigenvalue $\Delta$ of the operator $D$
associated with the ground state $|0 \rangle_{\lambda}$ are not real-valued
observables. Alternatively if we consider a shifted operator $\left( D -
\frac{i}{2} \right)$, the corresponding eigenvalues would provide real-valued
physical quantities. Furthermore in a more precise treatment one can express
the bulk filed $\phi$ as $\beta \phi_0$ where $\phi_0$ is the time-dependent
part while $\beta$ is the time-independent part. Instead of the Whittaker
function (\ref{conf1h3}) let us consider the function

\begin{align}
  \Psi_{\lambda, \beta, E_L, E_R} (\phi_0) \assign_{\lambda} \langle E_L |e^{-
  2 i \beta \phi_0 (D - \frac{i}{2})} |E_R \rangle_{\lambda}  \label{conf1h5}
\end{align}

Applying the quadratic Casimir operator (\ref{conf1e}) on the function
(\ref{conf1h5}), we get

\begin{align}
  \left[ \frac{1}{2}  \frac{\partial^2}{\partial \phi_0^2} - 2 \beta^2 E_L E_R
  e^{2 \beta \phi_0} \right] \Psi_{\lambda, \beta, E_L, E_R} (\phi_0) & =
  \frac{1}{2} \beta^2  (\lambda + 1)^2 \Psi_{\lambda, \beta, E_L, E_R}
  (\phi_0)  \label{conf1h6}
\end{align}

We see that the equation (\ref{conf1h6}) is the Wheeler-DeWitt equation that
is encountered in the minisuperspace approximation of Liouville field theory
(LFT).

Local properties of LFT can be described by the Lagrangian

\begin{align}
  \mathcal{L} & = \frac{1}{4 \pi} \partial^{\mu} \phi \partial_{\mu} \phi +
  \mu e^{2 b \phi}  \label{lft1a1}
\end{align}

where $b$ is the dimensionless coupling constant and $\mu$ is the cosmological
coupling constant. The equation of motion is

\begin{align}
  \Delta \phi & = 4 \pi b \mu e^{2 b \phi}  \label{lft1a2}
\end{align}

In two dimensions it is always possible to make any metric $g_{\mu \nu}$
conformally flat by coordinate redefinition $g_{\mu \nu} = e^{2 b \phi}
\eta_{\mu \nu}$. Furthermore, in two dimensions, the curvature can be
determined by the scalar curvature. Equation (\ref{lft1a2}) asserts that the
curvature is a negative constant $- 8 \pi b^2 \mu$ and $g_{\mu \nu}$ describes
a two-dimensional surface with constant negative curvature, thus the
corresponding Lorentzian surface can be identified with AdS\tmrsub{$2$}
space-time locally.

In order to quantize LFT via canonical quantization, we shall employ the
Fourier decomposition of the Liouville field $\phi$ and its canonical momentum
$\Pi$ on the cylinder

\begin{align}
  \label{lft1a3} \phi (t, \sigma) & = \phi_0 (t) + \sum_{n \neq 0} \frac{i}{n}
  [a_n (t) e^{- in \sigma} + b_n (t) e^{in \sigma}] \nonumber\\
  \Pi (t, \sigma) & = p_0 (t) + \sum_{n \neq 0} [a_n (t) e^{- in \sigma} + b_n
  (t) e^{in \sigma}] 
\end{align}

with $a_n^{\dag} = a_{- n}$, $b_n^{\dag} = b_{- n}$. The canonical relation

\begin{align}
  {}[\phi (t, \sigma), \Pi (t, \sigma')] = i \delta (\sigma - \sigma') 
  \label{lft1a4}
\end{align}

leads to the commutation relations

\begin{align}
  {}[\phi_0, p_0] & = i, & [a_n, a_m] & = \frac{n}{2} \delta_{n, - m}, & [b_n,
  b_m] & = \frac{n}{2} \delta_{n, - m}  \label{lft1a5}
\end{align}

which imply that $a_n, b_n$ are creation operators while $a_{- n}, b_{- n}$
are annihilation operators. The spectrum of LFT has been discussed in the
minisuperspace approximation {\cite{Seiberg:1990eb}}. The minisuperspace
approximation was originally proposed in quantum gravity
{\cite{DeWitt:1967yk,Misner:1972js}} where the problem is simplified by only
treating the zero mode and truncating the higher excited modes. Whether
physics in minisuperspace quantization gives a faithful properties of quantum
gravity remains an open question, however, it has been discussed that the
minisuperspace approximation would be exact in pure two-dimensional gravity
{\cite{Moore:1991ir,Fateev:2000ik}}. Replacing the canonical momentum $p_0 =
\frac{\dot{\phi}_0}{2 \pi}$ with differential operator $- i (\partial /
\partial \phi_0)$, we obtain the minisuperspace Schr{\"o}dinger equation

\begin{align}
  \left[ - \frac{1}{2}  \frac{\partial^2}{\partial \phi_0^2} + 2 \pi \mu e^{2
  b \phi_0} \right] \Psi_P (\phi_0) & = 2 P^2 \Psi_P (\phi_0)  \label{lft1a6}
\end{align}

where $P$ is the Liouville momentum, the eigenvalue of the Hamiltonian and
$\Psi_P (\phi_0)$ is the minisuperspace wave function. For $\phi_0 \rightarrow
- \infty$ the interaction is small and the wave function is a linear
combination of $e^{\pm iP \phi_0}$. Because of the complete reflection
potential at $\phi_0 \rightarrow 0$, $P$ is restricted to be positive so that
the incoming wave uniquely determines the reflected wave.

The function $\Psi_{\lambda, \beta, E_L, E_R} (\phi_0)$ in CQM plays a role of
the generating function of the expectation values evaluated among two excited
states. On the other hand, the Liouville wave function $\Psi_P (\phi_0)$ would
be regarded as the partition function of AdS\tmrsub{$2$}, which is a function
of the boundary values in the sense that a partition function generically
transforms as a wave function under a change of polarization on field space
specified at a boundary
{\cite{Aganagic:2003qj,KashaniPoor:2006nc,Cheng:2010yw,Beem:2012mb}} and LFT
describes AdS\tmrsub{$2$} space-time in the classical solution. Therefore we
come to the interesting conclusion that AdS\tmrsub{$2$} bulk mode $\phi$ that
behaves as the zero-mode $\phi_0$ near the boundary is dual to a source term
$\phi_0  (D - \frac{i}{2})$ in the CQM on the boundary of AdS\tmrsub{$2$} via
AdS\tmrsub{$2$}/CFT\tmrsub{$1$} correspondence

\begin{align}
  \label{ads2cft1a0} \Psi_{\lambda, \beta, E_L, E_R} (\phi_0) & = \left\langle
  E_L \left| e^{- 2 i \beta \phi_0  \left( D - \frac{i}{2} \right)} \right|
  E_R \right\rangle_{\mathrm{CQM}} \nonumber\\
  & = Z_{\mathrm{AdS}_2} (\phi |_{\mathrm{bdy}} = \phi_0) = \Psi_P (\phi_0) 
\end{align}

We observe that the expectation values in the CQM are evaluated between two
excited states defined by the Whittaker vector (\ref{conf1f3}) and the dual
Whittaker vector (\ref{conf1h1}) in the correspondence (\ref{ads2cft1a0}). The
two distinct states in the correspondence (\ref{ads2cft1a0}) would would
enable us to have two independent dynamical systems. This is consistent to the
statement
{\cite{Azeyanagi:2007bj,Sen:2008vm,Balasubramanian:2010ys,Chamon:2011xk}} that
unlike higher dimensional cases AdS\tmrsub{$2$} in the global coordinate has
two boundaries and the dual conformal field theory of asymptotically
AdS\tmrsub{$2$} is realized as two systems or two copies of CFT\tmrsub{$1$} on
the two boundaries although we cannot exclude the possibility of a single
CFT\tmrsub{$1$} as the dual description. The appearance of the excited states
could also be in favor of the statement that in the Lorentzian AdS/CFT
operator expectation values in excited CFT states differ from vacuum
expectation values due to the existence of normalizable propagating states in
the bulk {\cite{Balasubramanian:1998sn,Balasubramanian:1998de}}.

The comparison of the two Wheeler-DeWitt equations (\ref{conf1h6}) and
(\ref{lft1a6}) identifies the coupling constant $b$ in LFT with the constant
parameter $\beta$ in CQM and establishes the dictionary of parameters between
the bulk LFT of AdS\tmrsub{$2$} and the boundary CQM as

\begin{align}
  \frac{\pi \mu}{b^2} & = E_L E_R  \label{cor1}\\
  \frac{P^2}{b^2} & = - \frac{1}{4}  (\lambda + 1)^2 = \left( \Delta -
  \frac{i}{2} \right)^2  \label{cor2}
\end{align}

Therefore the relations (\ref{cor1}) and (\ref{cor2}) indicate that the
quantum gravity on AdS\tmrsub{$2$} can be described by two conformal quantum
mechanical systems on the boundary with energies $E_L$ and $E_R$ as a
holographic principle {\cite{'tHooft:1993gx,Susskind:1994vu}}.

The equation (\ref{cor1}) says that the excited states with non-vanishing
energy eigenvalues $E_L$, $E_R$ in the boundary CQM are needed to realize
negative constant curvature of AdS\tmrsub{$2$} space-time which is generated
by the non-trivial interaction term with finite and non-vanishing parameters
$b$, $\mu$ in two-dimensional gravity theory. It is illustrative to compare
our result with the analogous statement in the AdS\tmrsub{$3$}/CFT\tmrsub{$2$}
that the AdS\tmrsub{$3$} radius $l_3$ is represented by the central charge $c$
of the CFT\tmrsub{$2$} through the Brown-Henneaux relation $c = \frac{3 l_3}{2
G_3}$ {\cite{Brown:1986nw}} where $G_3$ is the three-dimensional Newton
constant. Instead of the central charge the energies of the states play an
important role in the AdS\tmrsub{$2$}/CFT\tmrsub{$1$}, however, the relation
between the AdS\tmrsub{$2$} radius and the energies is even more attractive in
that one of the other parameters in LFT necessarily comes about. In terms of
the coupling constant $b$ controlling the quantum effect in LFT we can write
the AdS\tmrsub{$2$} radius $l_2$ as

\begin{align}
  \frac{1}{\sqrt{E_L E_R}} & = 2 b^2 l_2  \label{cor3}
\end{align}

The semiclassical analysis in LFT is valid for small $b$, and accordingly our
formula (\ref{cor3}) would reflect the fact that the ground states with
vanishing energies in the boundary CQM force the AdS\tmrsub{$2$} radius to go
to infinity and the classical AdS\tmrsub{$2$} geometry then becomes flat
space-time.

We see from the equation (\ref{cor2}) that the Liouville momentum $P$ in
two-dimensional gravity theory corresponds to the conformal dimension of the
ground state in the dual CQM. As we have discussed, the unitarity condition in
CQM requires that $\frac{1}{2}  (\lambda + 1)$ is pure imaginary. This is
consistent to the fact that the Liouville momentum $P$ is real in the dual
two-dimensional gravity theory.

We would like to emphasize that the correspondence (\ref{ads2cft1a0}) and the
dictionaries (\ref{cor1}), (\ref{cor2}), (\ref{cor3}) are quite universal
since we have not specified the conformal quantum mechanical systems so far.
However, if we contain more specific information characterizing dynamical
properties and symmetries, there would be more fruitful statements (the
GKP-Witten relation {\cite{Gubser:1998bc,Witten:1998qj}}) in the
AdS\tmrsub{$2$}/CFT\tmrsub{$1$} as the extension of the relation
(\ref{ads2cft1a0})

\begin{align}
  \langle e^{h_0 \mathcal{O}} \rangle_{\mathrm{CQM}} & = Z_{\mathrm{AdS}_2}
  (h|_{\mathrm{bdy}} = h_0)  \label{ads2cft1b0}
\end{align}

where $h_0$ is some function of the boundary values for the bulk field $h$
while $\mathcal{O}$ is the dual operator in CQM. For a non-flat space the left
values in the correspondence (\ref{ads2cft1b0}) are generally presumed to be
computed between two excited states from the relation (\ref{cor3}). It would
be interesting that there has been proposals for such relation associated with
the DFF-model in {\cite{Chamon:2011xk}} and with the counting of microstates
of BPS extremal black holes in {\cite{Sen:2008yk,Sen:2008vm}}.

We now consider the resulting expectation values

\begin{align}
  _{\lambda} \langle E_L | \left( D - \frac{i}{2} \right) |E_R
  \rangle_{\lambda} & = \frac{i}{2}  \frac{\delta}{\delta \phi} \Psi_{\lambda,
  E_L, E_R} (\phi) |_{\phi = 0} \nobracket  \label{conf1h7b}
\end{align}

and its possible application to one of the deepest mathematical problem, the
Riemann hypothesis. The equation (\ref{conf1h6}) has two linearly independent
solutions, which are known to be cylindric functions
{\cite{gradshteyn2000table}}. By requiring the unitarity we can write the
solutions as

\begin{align}
  \Psi_{\lambda, E_L, E_R} (\phi) & = \frac{1}{i} K_{\lambda + 1}  \left( 2
  \sqrt{E_L E_R} e^{\phi} \right)  \label{conf1h8}
\end{align}

where $K_{\nu} (z)$ is the Macdonald function. We should note that the
Macdonald functions of purely imaginary order $\lambda + 1$ with positive
argument are real. The prefactor $\frac{1}{i}$ in the generating function
(\ref{conf1h8}) guarantees the reality condition of the expectation values in
CQM. Making use of the recurrence relation

\begin{align}
  K_{\nu - 1} (z) + K_{\nu + 1} (z) & = - 2 \frac{d}{dz} K_{\nu} (z) 
  \label{conf1h9}
\end{align}

and the formula (\ref{conf1h7b}) we can write the expectation values between
the excited states as

\begin{align}
  _{\lambda} \langle E_L | \left( D - \frac{i}{2} \right) |E_R
  \rangle_{\lambda} & = - \frac{z}{4}  (K_{\lambda} (z) + K_{\lambda + 2} (z))
  \label{conf1i1}
\end{align}

where $z = 2 \sqrt{E_L E_R}$. Let us now consider a one particle conformal
quantum mechanical model known as the DFF-model {\cite{deAlfaro:1976je}}
\begin{equation}
  S = \frac{1}{2}  \int  \left( \dot{x} (t)^2 - \frac{g}{x (t)^2} \right) dt
  \label{dff1}
\end{equation}
with $g$ being a dimensionless coupling constant. For the DFF-model the
dilatation operator can be expressed as $D = - \frac{1}{2} xp + \frac{i}{4}$
where $p$ is the canonical momentum and the equation (\ref{conf1i1}) becomes

\begin{align}
  _{\lambda} \langle E_L | \left( xp + \frac{i}{2} \right) |E_R
  \rangle_{\lambda} & = \frac{z}{2}  (K_{\lambda} (z) + K_{\lambda + 2} (z)) 
  \label{conf1i2}
\end{align}

It is speculated that the Riemann zeros would be realized as eigenvalues of
the operator which takes the form of $xp$
{\cite{MR1694895,berry1999h,MR1684543}} as a promising candidate of the
Riemann operator in the Hilbert-P{\'o}lya conjecture whose eigenvalues are the
imaginary part of the non-trivial Riemann zeros. Berry and Keating
{\cite{berry1999h,MR1684543}} identified the operator $xp$ with the
Hamiltonian and imposed the conditions $|x| \ge l_x$, $|p| \ge l_p$ so that
$l_x l_p = 2 \pi \hbar$ in the phase space. Then they found that the
semiclassical number $N (E)$ of states with the energy between $0$ and $E$ is
given by the area in the phase space divided by the Planck cell $h = 2 \pi$

\begin{align}
  N (E) & = \frac{E}{2 \pi}  \left( \log \frac{E}{2 \pi} - 1 \right)
  +\mathcal{O} (1)  \label{rh1a}
\end{align}

and observed that (\ref{rh1a}) precisely coincides with the asymptotics of the
smoothed counting function of the number of Riemann zeros {\cite{MR0466039}}.
Connes {\cite{MR1694895}} introduced the constraints $|x| \le \Lambda$, $|p|
\le \Lambda$ where $\Lambda$ is a common cutoff and counted the number of such
semiclassical states as

\begin{align}
  N (E) & = \frac{E}{\pi} \log \Lambda - \frac{E}{2 \pi}  \left( \log
  \frac{E}{2 \pi} - 1 \right)  \label{rh1b}
\end{align}

He interpreted the counting formula (\ref{rh1b}) as missing spectral lines
associated to the smooth Riemann zeros which arise in the limit $\Lambda
\rightarrow \infty$, however, it was reinterpreted as a finite size correction
from a physical system later in {\cite{MR2443603}}. According to these
semiclassical proposals of the Hilbert-P{\'o}lya conjecture it has been
desirable to replace these artificially imposed semiclassical regularizations
of the operator $xp$ with the proper quantum treatment which naturally
generates a discrete spectrum. In order to obtain the discrete spectrum via
quantization, there has been proposed various attempts including the
modifications of the $xp$ operator (see, e.g.,
{\cite{Sierra:2011tb,MR2812337}}) and the adoption of the regularization
methods (see, e.g., {\cite{MR2443603}}). Nevertheless, these attempts seem to
be quite artificial and difficult fo find the quantum mechanical explanation
to follow these ideas.

Here we wish to propose a novel perspective to acquire the distribution of the
spectrum of the Riemann operator from conformal quantum mechanics point of
view. Consider now the eigenfunction $\Phi_{\rho} (x)$ of the operator $\left(
xp + \frac{i}{2} \right)$ in the equation (\ref{conf1i2}) satisfying
\begin{equation}
  \left[ \frac{1}{i} x \frac{d}{dx} + \frac{i}{2} \right] \Phi_{\rho} (x) =
  \rho \Phi_{\rho} (x) \label{conf1i3}
\end{equation}
They take the form
\begin{equation}
  \Phi_{\rho} (x) = Cx^{\frac{1}{2} + i \rho} \label{conf1i4}
\end{equation}


with $C$ a constant of integration. The non-trivial zeros of the Riemann zeta
function $\zeta (s)$ which is conjectured to be $s = \frac{1}{2} + i \rho$,
$\rho \in \mathbb{R}$ in the Riemann hypothesis appears in the power $x^s$ of
the eigenfunction $\Phi_{\rho} (x)$. The eigenvalues $\rho$ of the operator
$\left( xp + \frac{i}{2} \right)$ which would be the candidates of the Riemann
zeros can be continuous in the position eigenfunction. This is the same
situation as has been already discussed in many literatures.

However, in CQM the operator $\left( xp + \frac{i}{2} \right)$ should not be
recognized as the Hamiltonian but rather as the dilatation operator whose
expectation values can be measured by the energy eigenstates as

\begin{align}
  D (z ; \rho) & = \frac{z}{2}  (K_{1 - i \rho} (z) + K_{1 + i \rho} (z)) 
  \label{conf1i50}
\end{align}

Note that the expression (\ref{conf1i50}) can evidently be lifted to arbitrary
conformal quantum mechanical systems by qualifying $\rho$ as the eigenvalue of
the operator $- 2 \left( D - \frac{i}{2} \right)$. Although almost all quantum
approaches so far have tried to identify the operator $xp$ with the
Hamiltonian and simultaneously diagonalize it with the position $x$ or the
momentum $p$, CQM would provide an alternative avenue to the Riemann
hypothesis as the diagonalization of the dilatation operator. We observe that
the ground state $|0 \rangle_{\lambda}$ is the eigenfunction of both of the
Hamiltonian and the dilatation operator. Since the excited states are not
eigenvectors of the dilatation operator, the limit $E_L$, $E_R \rightarrow 0$
of the expectation values (\ref{conf1i50}) would naturally give rise to the
eigenfunction of the operator $- 2 \left( D - \frac{i}{2} \right)$ multiplied
by its eigenvalue $\rho$. In other words, the limit in which $z = 2 \sqrt{E_L
E_R}$ goes to zero yields the definite eigenvalue $\rho$ and the distribution
function $D (\rho)$ of the ground state. Hence the expectation values
(\ref{conf1i50}) are in some sense the regularized functions which produce the
distribution of the eigenvalues $\rho$ as

\begin{align}
  \rho D (\rho) & = \lim_{z \rightarrow 0}  \frac{z}{2}  (K_{1 - i \rho} (z) +
  K_{1 + i \rho} (z))  \label{conf1i5}
\end{align}

where we have used the relation $\lambda = - (1 + i \rho)$ and the formula
$K_{\nu} (z) = K_{- \nu} (z)$. The asymptotic behavior of the Macdonald
function

\begin{align}
  K_{1 + i \rho} (z) \sim \sqrt{\frac{\pi}{z}} e^{- \frac{\pi}{2} \rho} \left(
  \frac{2 \rho}{ze} \right)^{i \rho}  \label{conf1i6}
\end{align}

for large $\rho$ allows us to write (\ref{conf1i5}) as

\begin{align}
  \rho D (\rho) & = \lim_{z \rightarrow 0}  \sqrt{\pi z} e^{- \frac{\pi}{2}
  \rho} \cos \left[ \rho \ln \left( \frac{2 \rho}{ze} \right) \right] 
  \label{conf1i7}
\end{align}

The semiclassical distribution of (\ref{conf1i7}) for large $\rho$ is realized
when the cosine function is at its maximum

\begin{align}
  \cos \left[ \rho \ln \left( \frac{2 \rho}{ze} \right) \right] = 1 
  \label{conf1i8}
\end{align}

so that

\begin{align}
  \frac{\rho}{\pi}  \left[ \ln \left( \frac{\rho}{E_L E_R} \right) - 1 \right]
  & = 2 n, \forall n \in \mathbb{Z}.  \label{conf1i9}
\end{align}

Since the expression (\ref{conf1i9}) diverges when $E_L, E_R \rightarrow 0$, a
low energy cutoff is required to make sense of the expression (\ref{conf1i9}).
Let us introduce the cutoff $\Lambda$ such that $E_L E_R = 2 \pi / \Lambda$.
Then we obtain the behavior of the large eigenvalues $\rho$ as

\begin{align}
  N (\rho) & = \frac{\rho}{2 \pi} \ln \Lambda + \frac{\rho}{2 \pi}  \left( \ln
  \frac{\rho}{2 \pi} - 1 \right)  \label{conf1i10}
\end{align}

Remarkably the first term is a continuum in the limit $\Lambda \rightarrow
\infty$ while the second term leads to the asymptotics of the counting
function of the Riemann zeros as in (\ref{rh1a}) and (\ref{rh1b}). It would be
interesting to note that the equation (\ref{conf1i10}) also counts the large
conformal dimensions for the ground state in CQM. Combining the semiclassical
realization (\ref{conf1i10}) of the counting Riemann zeros with our proposed
holographic correspondence (\ref{cor2}) would indicate underlying profound
relation among essential ingredients in number theory, in quantum mechanics
and in gravity.

\subsection*{Acknowledgments}

The author would like to thank Pei-Ming Ho, Kazuo Hosomichi, Takeo Inami and
Dharmesh Jain for communications and discussions and Yu Nakayama for
enlightening comments and remarks about Liouville field theory and Michael
Berry, Jon Keating, Paul Townsend and especially Germ{\'a}n Sierra for helpful
comments and explanations of their works on Riemann zeros. This work was
supported by National Taiwan University and the National Center for
Theoretical Science (NCTS).

\begin{thebibliography}{10}
  \bibitem[1]{'tHooft:1993gx}G.~'t~Hooft, ``Dimensional reduction in quantum
  gravity,'' in \tmtextit{Salamfest 1993:0284-296}, pp.~0284--296, 1993.
  {\newblock}\href{http://xxx.lanl.gov/abs/gr-qc/9310026}{\tmtexttt{gr-qc/9310026}}.
  
  \bibitem[2]{Susskind:1994vu}L.~Susskind, ``The World as a hologram,''
  \tmtextit{J.Math.Phys.} \tmtextbf{36} (1995) 6377--6396,
  \href{http://xxx.lanl.gov/abs/hep-th/9409089}{\tmtexttt{hep-th/9409089}}.
  
  \bibitem[3]{Maldacena:1997re}J.~M. Maldacena, ``The Large N limit of
  superconformal field theories and supergravity,''
  \tmtextit{Adv.Theor.Math.Phys.} \tmtextbf{2} (1998) 231--252,
  \href{http://xxx.lanl.gov/abs/hep-th/9711200}{\tmtexttt{hep-th/9711200}}.
  
  \bibitem[4]{Kunduri:2007vf}H.~K. Kunduri, J.~Lucietti, and H.~S. Reall,
  ``Near-horizon symmetries of extremal black holes,'' \tmtextit{Class. Quant.
  Grav.} \tmtextbf{24} (2007) 4169--4190,
  \href{http://xxx.lanl.gov/abs/0705.4214}{\tmtexttt{0705.4214}}.
  
  \bibitem[5]{Figueras:2008qh}P.~Figueras, H.~K. Kunduri, J.~Lucietti, and
  M.~Rangamani, ``Extremal vacuum black holes in higher dimensions,''
  \tmtextit{Phys.Rev.} \tmtextbf{D78} (2008) 044042,
  \href{http://xxx.lanl.gov/abs/0803.2998}{\tmtexttt{0803.2998}}.
  
  \bibitem[6]{Strominger:1998yg}A.~Strominger, ``AdS(2) quantum gravity and
  string theory,'' \tmtextit{JHEP} \tmtextbf{9901} (1999) 007,
  \href{http://xxx.lanl.gov/abs/hep-th/9809027}{\tmtexttt{hep-th/9809027}}.
  
  \bibitem[7]{Maldacena:1998uz}J.~M. Maldacena, J.~Michelson, and
  A.~Strominger, ``Anti-de Sitter fragmentation,'' \tmtextit{JHEP}
  \tmtextbf{9902} (1999) 011,
  \href{http://xxx.lanl.gov/abs/hep-th/9812073}{\tmtexttt{hep-th/9812073}}.
  
  \bibitem[8]{Nakatsu:1998st}T.~Nakatsu and N.~Yokoi, ``Comments on
  Hamiltonian formalism of AdS / CFT correspondence,''
  \tmtextit{Mod.Phys.Lett.} \tmtextbf{A14} (1999) 147--160,
  \href{http://xxx.lanl.gov/abs/hep-th/9812047}{\tmtexttt{hep-th/9812047}}.
  
  \bibitem[9]{Townsend:1998qp}P.~Townsend, ``The M(atrix) model / AdS(2)
  correspondence,''
  \href{http://xxx.lanl.gov/abs/hep-th/9903043}{\tmtexttt{hep-th/9903043}}.
  
  \bibitem[10]{Spradlin:1999bn}M.~Spradlin and A.~Strominger, ``Vacuum states
  for AdS(2) black holes,'' \tmtextit{JHEP} \tmtextbf{9911} (1999) 021,
  \href{http://xxx.lanl.gov/abs/hep-th/9904143}{\tmtexttt{hep-th/9904143}}.
  
  \bibitem[11]{Cadoni:1999ja}M.~Cadoni and S.~Mignemi, ``Asymptotic symmetries
  of AdS(2) and conformal group in d = 1,'' \tmtextit{Nucl. Phys.}
  \tmtextbf{B557} (1999) 165--180,
  \href{http://xxx.lanl.gov/abs/hep-th/9902040}{\tmtexttt{hep-th/9902040}}.
  
  \bibitem[12]{Blum:1999pc}J.~D. Blum, ``Supersymmetric quantum mechanical
  description of four-dimensional black holes,'' \tmtextit{JHEP}
  \tmtextbf{0001} (2000) 006,
  \href{http://xxx.lanl.gov/abs/hep-th/9907101}{\tmtexttt{hep-th/9907101}}.
  
  \bibitem[13]{NavarroSalas:1999up}J.~Navarro-Salas and P.~Navarro, ``AdS(2) /
  CFT(1) correspondence and near extremal black hole entropy,''
  \tmtextit{Nucl. Phys.} \tmtextbf{B579} (2000) 250--266,
  \href{http://xxx.lanl.gov/abs/hep-th/9910076}{\tmtexttt{hep-th/9910076}}.
  
  \bibitem[14]{Caldarelli:2000xk}M.~Caldarelli, G.~Catelani, and L.~Vanzo,
  ``Action, Hamiltonian and CFT for 2-D black holes,'' \tmtextit{JHEP}
  \tmtextbf{10} (2000) 005,
  \href{http://xxx.lanl.gov/abs/hep-th/0008058}{\tmtexttt{hep-th/0008058}}.
  
  \bibitem[15]{Cadoni:2000gm}M.~Cadoni, P.~Carta, D.~Klemm, and S.~Mignemi,
  ``AdS(2) gravity as conformally invariant mechanical system,''
  \tmtextit{Phys. Rev.} \tmtextbf{D63} (2001) 125021,
  \href{http://xxx.lanl.gov/abs/hep-th/0009185}{\tmtexttt{hep-th/0009185}}.
  
  \bibitem[16]{Bellucci:2002va}S.~Bellucci, A.~Galajinsky, E.~Ivanov, and
  S.~Krivonos, ``AdS(2) / CFT(1), canonical transformations and superconformal
  mechanics,'' \tmtextit{Phys.Lett.} \tmtextbf{B555} (2003) 99--106,
  \href{http://xxx.lanl.gov/abs/hep-th/0212204}{\tmtexttt{hep-th/0212204}}.
  
  \bibitem[17]{Strominger:2003tm}A.~Strominger, ``A Matrix model for AdS(2),''
  \tmtextit{JHEP} \tmtextbf{0403} (2004) 066,
  \href{http://xxx.lanl.gov/abs/hep-th/0312194}{\tmtexttt{hep-th/0312194}}.
  
  \bibitem[18]{Leiva:2003kd}C.~Leiva and M.~S. Plyushchay, ``Conformal
  symmetry of relativistic and nonrelativistic systems and Ads / CFT
  correspondence,'' \tmtextit{Annals Phys.} \tmtextbf{307} (2003) 372--391,
  \href{http://xxx.lanl.gov/abs/hep-th/0301244}{\tmtexttt{hep-th/0301244}}.
  
  \bibitem[19]{Giveon:2004zz}A.~Giveon and A.~Sever, ``Strings in a 2-d
  extremal black hole,'' \tmtextit{JHEP} \tmtextbf{0502} (2005) 065,
  \href{http://xxx.lanl.gov/abs/hep-th/0412294}{\tmtexttt{hep-th/0412294}}.
  
  \bibitem[20]{Azeyanagi:2007bj}T.~Azeyanagi, T.~Nishioka, and T.~Takayanagi,
  ``Near Extremal Black Hole Entropy as Entanglement Entropy via
  AdS(2)/CFT(1),'' \tmtextit{Phys.Rev.} \tmtextbf{D77} (2008) 064005,
  \href{http://xxx.lanl.gov/abs/0710.2956}{\tmtexttt{0710.2956}}.
  
  \bibitem[21]{Alishahiha:2008tv}M.~Alishahiha and F.~Ardalan, ``Central
  Charge for 2D Gravity on AdS(2) and AdS(2)/CFT(1) Correspondence,''
  \tmtextit{JHEP} \tmtextbf{08} (2008) 079,
  \href{http://xxx.lanl.gov/abs/0805.1861}{\tmtexttt{0805.1861}}.
  
  \bibitem[22]{Gupta:2008ki}R.~K. Gupta and A.~Sen, ``Ads(3)/CFT(2) to
  Ads(2)/CFT(1),'' \tmtextit{JHEP} \tmtextbf{0904} (2009) 034,
  \href{http://xxx.lanl.gov/abs/0806.0053}{\tmtexttt{0806.0053}}.
  
  \bibitem[23]{Hartman:2008dq}T.~Hartman and A.~Strominger, ``Central Charge
  for AdS(2) Quantum Gravity,'' \tmtextit{JHEP} \tmtextbf{04} (2009) 026,
  \href{http://xxx.lanl.gov/abs/0803.3621}{\tmtexttt{0803.3621}}.
  
  \bibitem[24]{Galajinsky:2008ce}A.~Galajinsky, ``Particle dynamics on AdS(2)
  x S**2 background with two-form flux,'' \tmtextit{Phys.Rev.} \tmtextbf{D78}
  (2008) 044014,
  \href{http://xxx.lanl.gov/abs/0806.1629}{\tmtexttt{0806.1629}}.
  
  \bibitem[25]{Sen:2008vm}A.~Sen, ``Quantum Entropy Function from
  AdS(2)/CFT(1) Correspondence,'' \tmtextit{Int.J.Mod.Phys.} \tmtextbf{A24}
  (2009) 4225--4244,
  \href{http://xxx.lanl.gov/abs/0809.3304}{\tmtexttt{0809.3304}}.
  
  \bibitem[26]{Sen:2008yk}A.~Sen, ``Entropy Function and AdS(2) / CFT(1)
  Correspondence,'' \tmtextit{JHEP} \tmtextbf{0811} (2008) 075,
  \href{http://xxx.lanl.gov/abs/0805.0095}{\tmtexttt{0805.0095}}.
  
  \bibitem[27]{Sen:2011cn}A.~Sen, ``State Operator Correspondence and
  Entanglement in $AdS_2 / CFT_1$,'' \tmtextit{Entropy} \tmtextbf{13} (2011)
  1305--1323, \href{http://xxx.lanl.gov/abs/1101.4254}{\tmtexttt{1101.4254}}.
  
  \bibitem[28]{Chamon:2011xk}C.~Chamon, R.~Jackiw, S.-Y. Pi, and L.~Santos,
  ``Conformal quantum mechanics as the CFT\tmrsub{$1$} dual to
  AdS\tmrsub{$2$},'' \tmtextit{Phys.Lett.} \tmtextbf{B701} (2011) 503--507,
  \href{http://xxx.lanl.gov/abs/1106.0726}{\tmtexttt{1106.0726}}.
  
  \bibitem[29]{MolinaVilaplana:2012xe}J.~Molina-Vilaplana and G.~Sierra, ``An
  $xp$ model on $AdS_2$ spacetime,'' \tmtextit{Nucl. Phys.} \tmtextbf{B877}
  (2013) 107--123,
  \href{http://xxx.lanl.gov/abs/1212.2436}{\tmtexttt{1212.2436}}.
  
  \bibitem[30]{Jackiw:2012ur}R.~Jackiw and S.-Y. Pi, ``Conformal Blocks for
  the 4-Point Function in Conformal Quantum Mechanics,'' \tmtextit{Phys.Rev.}
  \tmtextbf{D86} (2012) 045017,
  \href{http://xxx.lanl.gov/abs/1205.0443}{\tmtexttt{1205.0443}}.
  
  \bibitem[31]{Zamolodchikov:2001ah}A.~B. Zamolodchikov and A.~B.
  Zamolodchikov, ``Liouville field theory on a pseudosphere,''
  \href{http://xxx.lanl.gov/abs/hep-th/0101152}{\tmtexttt{hep-th/0101152}}.
  
  \bibitem[32]{Balasubramanian:1998sn}V.~Balasubramanian, P.~Kraus, and A.~E.
  Lawrence, ``Bulk versus boundary dynamics in anti-de Sitter space-time,''
  \tmtextit{Phys. Rev.} \tmtextbf{D59} (1999) 046003,
  \href{http://xxx.lanl.gov/abs/hep-th/9805171}{\tmtexttt{hep-th/9805171}}.
  
  \bibitem[33]{Balasubramanian:1998de}V.~Balasubramanian, P.~Kraus, A.~E.
  Lawrence, and S.~P. Trivedi, ``Holographic probes of anti-de Sitter
  space-times,'' \tmtextit{Phys. Rev.} \tmtextbf{D59} (1999) 104021,
  \href{http://xxx.lanl.gov/abs/hep-th/9808017}{\tmtexttt{hep-th/9808017}}.
  
  \bibitem[34]{Balasubramanian:2010ys}V.~Balasubramanian, J.~Parsons, and
  S.~F. Ross, ``States of a chiral 2d CFT,'' \tmtextit{Class. Quant. Grav.}
  \tmtextbf{28} (2011) 045004,
  \href{http://xxx.lanl.gov/abs/1011.1803}{\tmtexttt{1011.1803}}.
  
  \bibitem[35]{deAlfaro:1976je}V.~de~Alfaro, S.~Fubini, and G.~Furlan,
  ``Conformal Invariance in Quantum Mechanics,'' \tmtextit{Nuovo Cim.}
  \tmtextbf{A34} (1976) 569.
  
  \bibitem[36]{MR1694895}A.~Connes, ``Trace formula in noncommutative geometry
  and the zeros of the Riemann zeta function,'' \tmtextit{Selecta Math.
  (N.S.)} \tmtextbf{5} (1999), no.~1 29--106.
  
  \bibitem[37]{berry1999h}M.~V. Berry and J.~P. Keating, ``H= xp and the
  riemann zeros,'' in \tmtextit{Supersymmetry and Trace Formulae},
  pp.~355--367. {\newblock}Springer, 1999.
  
  \bibitem[38]{MR1684543}M.~V. Berry and J.~P. Keating, ``The Riemann zeros
  and eigenvalue asymptotics,'' \tmtextit{SIAM Rev.} \tmtextbf{41} (1999),
  no.~2 236--266 (electronic).
  
  \bibitem[39]{MR2443603}G.~Sierra and P.~K. Townsend, ``Landau levels and
  Riemann zeros,'' \tmtextit{Phys. Rev. Lett.} \tmtextbf{101} (2008), no.~11
  110201, 4.
  
  \bibitem[40]{Sierra:2011tb}G.~Sierra and J.~Rodriguez-Laguna, ``The H=xp
  model revisited and the Riemann zeros,'' \tmtextit{Phys.Rev.Lett.}
  \tmtextbf{106} (2011) 200201,
  \href{http://xxx.lanl.gov/abs/1102.5356}{\tmtexttt{1102.5356}}.
  
  \bibitem[41]{MR2812337}M.~V. Berry and J.~P. Keating, ``A compact
  Hamiltonian with the same asymptotic mean spectral density as the Riemann
  zeros,'' \tmtextit{J. Phys. A} \tmtextbf{44} (2011), no.~28 285203, 14.
  
  \bibitem[42]{Bargmann:1946me}V.~Bargmann, ``Irreducible unitary
  representations of the Lorentz group,'' \tmtextit{Annals Math.}
  \tmtextbf{48} (1947) 568--640.
  
  \bibitem[43]{MR0271275}H.~Jacquet, ``Fonctions de Whittaker associ{\'e}es
  aux groupes de Chevalley,'' \tmtextit{Bull. Soc. Math. France} \tmtextbf{95}
  (1967) 243--309.
  
  \bibitem[44]{MR0311838}G.~Schiffmann, ``Int{\'e}grales d'entrelacement et
  fonctions de Whittaker,'' \tmtextit{Bull. Soc. Math. France} \tmtextbf{99}
  (1971) 3--72.
  
  \bibitem[45]{MR507800}B.~Kostant, ``On Whittaker vectors and representation
  theory,'' \tmtextit{Invent. Math.} \tmtextbf{48} (1978), no.~2 101--184.
  
  \bibitem[46]{MR1729357}P.~Etingof, ``Whittaker functions on quantum groups
  and $q$-deformed Toda operators,'' in \tmtextit{Differential topology,
  infinite-dimensional Lie algebras, and applications}, vol.~194 of
  \tmtextit{Amer. Math. Soc. Transl. Ser. 2}, pp.~9--25. {\newblock}Amer.
  Math. Soc., Providence, RI, 1999.
  
  \bibitem[47]{Seiberg:1990eb}N.~Seiberg, ``Notes on quantum Liouville theory
  and quantum gravity,'' \tmtextit{Prog.Theor.Phys.Suppl.} \tmtextbf{102}
  (1990) 319--349.
  
  \bibitem[48]{DeWitt:1967yk}B.~S. DeWitt, ``Quantum Theory of Gravity. 1. The
  Canonical Theory,'' \tmtextit{Phys. Rev.} \tmtextbf{160} (1967) 1113--1148.
  
  \bibitem[49]{Misner:1972js}C.~W. Misner, ``MINISUPERSPACE,''.
  
  \bibitem[50]{Moore:1991ir}G.~W. Moore, N.~Seiberg, and M.~Staudacher, ``From
  loops to states in 2-D quantum gravity,'' \tmtextit{Nucl. Phys.}
  \tmtextbf{B362} (1991) 665--709.
  
  \bibitem[51]{Fateev:2000ik}V.~Fateev, A.~B. Zamolodchikov, and A.~B.
  Zamolodchikov, ``Boundary Liouville field theory. 1. Boundary state and
  boundary two point function,''
  \href{http://xxx.lanl.gov/abs/hep-th/0001012}{\tmtexttt{hep-th/0001012}}.
  
  \bibitem[52]{Aganagic:2003qj}M.~Aganagic, R.~Dijkgraaf, A.~Klemm, M.~Marino,
  and C.~Vafa, ``Topological strings and integrable hierarchies,''
  \tmtextit{Commun.Math.Phys.} \tmtextbf{261} (2006) 451--516,
  \href{http://xxx.lanl.gov/abs/hep-th/0312085}{\tmtexttt{hep-th/0312085}}.
  
  \bibitem[53]{KashaniPoor:2006nc}A.-K. Kashani-Poor, ``The Wave Function
  Behavior of the Open Topological String Partition Function on the
  Conifold,'' \tmtextit{JHEP} \tmtextbf{04} (2007) 004,
  \href{http://xxx.lanl.gov/abs/hep-th/0606112}{\tmtexttt{hep-th/0606112}}.
  
  \bibitem[54]{Cheng:2010yw}M.~C.~N. Cheng, R.~Dijkgraaf, and C.~Vafa,
  ``Non-Perturbative Topological Strings And Conformal Blocks,''
  \tmtextit{JHEP} \tmtextbf{09} (2011) 022,
  \href{http://xxx.lanl.gov/abs/1010.4573}{\tmtexttt{1010.4573}}.
  
  \bibitem[55]{Beem:2012mb}C.~Beem, T.~Dimofte, and S.~Pasquetti,
  ``Holomorphic Blocks in Three Dimensions,'' \tmtextit{JHEP} \tmtextbf{1412}
  (2014) 177, \href{http://xxx.lanl.gov/abs/1211.1986}{\tmtexttt{1211.1986}}.
  
  \bibitem[56]{Brown:1986nw}J.~D. Brown and M.~Henneaux, ``Central Charges in
  the Canonical Realization of Asymptotic Symmetries: An Example from
  Three-Dimensional Gravity,'' \tmtextit{Commun.Math.Phys.} \tmtextbf{104}
  (1986) 207--226.
  
  \bibitem[57]{Gubser:1998bc}S.~Gubser, I.~R. Klebanov, and A.~M. Polyakov,
  ``Gauge theory correlators from noncritical string theory,''
  \tmtextit{Phys.Lett.} \tmtextbf{B428} (1998) 105--114,
  \href{http://xxx.lanl.gov/abs/hep-th/9802109}{\tmtexttt{hep-th/9802109}}.
  
  \bibitem[58]{Witten:1998qj}E.~Witten, ``Anti-de Sitter space and
  holography,'' \tmtextit{Adv.Theor.Math.Phys.} \tmtextbf{2} (1998) 253--291,
  \href{http://xxx.lanl.gov/abs/hep-th/9802150}{\tmtexttt{hep-th/9802150}}.
  
  \bibitem[59]{gradshteyn2000table}I.~S. Gradshteyn and I.~Ryzhik, ``Table of
  integrals, series, and products. translated from the russian. translation
  edited and with a preface by alan jeffrey and daniel zwillinger,'' 2000.
  
  \bibitem[60]{MR0466039}H.~M. Edwards, \tmtextit{Riemann's zeta function}.
  {\newblock}Academic Press [A subsidiary of Harcourt Brace Jovanovich,
  Publishers], New York-London, 1974. {\newblock}Pure and Applied Mathematics,
  Vol. 58.
\end{thebibliography}

\end{document}
