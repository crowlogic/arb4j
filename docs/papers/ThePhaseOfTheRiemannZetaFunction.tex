\documentclass[11pt]{article}
\usepackage{amsmath}
\usepackage{amssymb}
\usepackage{amsfonts}
\usepackage{graphicx}
\usepackage{enumitem}
\usepackage{amsthm}

\theoremstyle{plain}
\newtheorem{theorem}{Theorem}
\newtheorem{lemma}{Lemma}
\newtheorem{definition}{Definition}

\title{The phase of the Riemann zeta function}
\author{Avinash Khare\\
Institute of Physics, Sachivalaya Marg, Bhubaneswar 751005, India}
\date{}

\begin{document}

\maketitle

\begin{abstract}
We offer an alternative interpretation of the Riemann zeta function $\zeta(s)$ as a scattering amplitude and its nontrivial zeros as the resonances in the scattering amplitude. We also look at several different facets of the phase of the $\zeta$ function. For example, we show that the smooth part of the function along the line of the zeros is related to the quantum density of states of an inverted oscillator. On the other hand, for $\Re s > 1/2$, we show that the memory of the zeros fades only gradually through a Lorentzian smoothing of the delta functions. The corresponding trace formula for $\Re s \gg 1$ is shown to be of the same form as generated by a one-dimensional harmonic oscillator in one direction along with an inverted oscillator in the transverse direction. Quite remarkably for this simple model, the Gutzwiller trace formula can be obtained analytically and is found to agree with the quantum result.
\end{abstract}

\textbf{Keywords:} Riemann zeta function; Gutzwiller trace formula.\\
\textbf{PACS Nos:} 03.65; 05.30

\section{Introduction}

The Riemann zeta function $\zeta(s)$ of the complex variable $s = \sigma + it$ is defined for $\sigma > 1$ by \cite{edwards1974}
\begin{equation}
\zeta(s) = \sum_{n=1}^{\infty} \frac{1}{n^s} = \prod_{p \text{ primes}} \frac{1}{1-p^{-s}}
\label{eq:zeta_def}
\end{equation}
and for $\sigma < 1$ by analytic continuation. As is well known, $\zeta(s)$ has so called trivial zeros at $s = -2n$ with $n = 1,2,3,\ldots$. All other zeros are at complex values of $s$. According to Riemann's celebrated hypothesis made in 1856, the nontrivial zeros of $\zeta$ all lie symmetrically on the line $\sigma = 1/2$, i.e. $\zeta(1/2 \pm it_n) = 0$. Riemann's hypothesis is supported by numerical tests up to very large values of $t_n$ but mathematicians are still unable to prove or disprove it rigorously. It is also well known that there are an infinite number of zeros on the half-line $\sigma = 1/2$ \cite{hardy1914}. These so-called nontrivial zeros are of great interest to mathematicians from a number-theoretic point of view and to physicists interested in quantum chaos and periodic orbit theory \cite{berry1985,gutzwiller1991,wu1993}. In particular, the so-called nontrivial zeros of $\zeta(s)$ exhibit an intrinsically random distribution of the GUE type. Further, assuming that the famous Riemann hypothesis is correct, the density of zeros can be shown to obey a sum rule which is analogous to the famous Gutzwiller formula for the level density. All this is taken to suggest that there may be a classical chaotic dynamical system without time-reversal invariance and $t_n$ are the eigenvalues of the quantum Hamiltonian which is obtained by quantizing this classical chaotic system \cite{berry1985b,berry1986,sieber1990}. However, to date, no such classical system has been found.

We have offered an alternative interpretation of the $\zeta$ function as a scattering amplitude and its nontrivial zeros as resonances in the scattering amplitude \cite{bhaduri1995a,bhaduri1995b}. As a bonus, this interpretation directly leads us to an approximate rule for the location of the zeros with an error which is at most 3 per cent.

We have also looked in detail at several different facets of the phase of the Riemann $\zeta$ function. For example, we show that the smooth part of the $\zeta$ function along the line of the zeros (i.e. $\sigma = 1/2$) is related to the quantum density of states of an inverted oscillator. On the other hand, for $\sigma > 1/2$, we show that the memory of the zeros fades only gradually through a Lorentzian smoothing of the delta functions. The corresponding trace formula for $\sigma \gg 1$ is shown to be of the same form as generated by a one-dimensional harmonic oscillator in one direction along with an inverted oscillator in the transverse direction. Quite remarkably, for this simple model, the Gutzwiller trace formula can be obtained analytically and is found to agree with the quantum result. As far as I am aware this is the first instance of a scattering system, for which the Gutzwiller formula is exact. Before I come to these issues, I briefly review some properties of the $\zeta$ function.

\section{Some Properties of the Riemann Zeta function}

The Riemann zeta function as defined in Eq.~\eqref{eq:zeta_def} can also be written as
\begin{equation}
\zeta(s) = \sum_{n=0}^{\infty} \exp(-E_n \beta)
\label{eq:partition}
\end{equation}
where $\beta = 1$, $s = \xi \beta$ and $E_n = \xi \ln(n + 1)$, with the constant $\xi$ setting the energy scale. In this form it is clearly the canonical partition function of a quantum system with an energy spectrum $E_n$ as given above. Without any loss of generality, we set $\xi = 1$ throughout this article.

Actually, $\zeta$ can also be looked upon as a grand partition function; since any integer $n$ is a unique product of primes $p_i$ ($\ln(n) = \sum_i a_i \ln(p_i)$ where $a_i$ are a set of positive integers or zero). Hence the zeta function may also be written as
\begin{equation}
\zeta(s) = \prod_i \frac{1}{1 - \exp(-\ln(p_i)s)}
\label{eq:grand_partition}
\end{equation}
Written in this form, it can clearly be regarded as a bosonic grand partition function with chemical potential $\mu = 0$. In this article however we shall make use of the canonical partition function interpretation of the $\zeta$.

The quantum density of states of the system as given by \eqref{eq:partition} with the energy spectrum $E_n = \ln(n+1)$ is defined as
\begin{equation}
\rho(E) = \sum_{n=0}^{\infty} \delta(E-E_n) = \sum_{n=0}^{\infty} \delta(E - \ln(n + 1))
\label{eq:density_def}
\end{equation}

Following Jennings \cite{jennings1974}, let us derive an expression for the smooth and oscillating parts of $\rho(E)$ for a nondegenerate spectrum. Using the fact that
\begin{equation}
\sum_{n=0}^{\infty} \delta(x - n) = 1 + 2 \sum_{m=1}^{\infty} \cos(2\pi mx)
\label{eq:delta_sum}
\end{equation}
we obtain
\begin{equation}
\rho(E) = \frac{dG}{dE} \left[ 1 + 2 \sum_{m=1}^{\infty} \cos(2\pi m G(E)) \right]
\label{eq:density_expansion}
\end{equation}

Here $G(E)$ satisfies the inverse relation $n = G(E_n)$. For the spectrum, $E_n = \ln(n + 1)$, we have $G(E) = (e^E - 1)$ and hence the density of states can be written as
\begin{equation}
\rho(E) = e^E \left[ 1 + 2 \sum_{m=1}^{\infty} \cos(2\pi m e^E) \right]
\label{eq:density_explicit}
\end{equation}
which is exponentially rising! For an exponentially rising density of states like the one here, the entropy $S(E) = E$, and the limiting inverse temperature (which is the analogue of the Hagedorn temperature \cite{hagedorn1965} in high energy physics) is given by $\beta_c = \partial S/\partial E = 1$. This is not surprising in view of the composite nature of the system and the fact that the number of elementary constituents (primes) tend to increase with increasing $E$ (corresponding to large $n$). A similar situation is encountered in high energy physics \cite{hagedorn1965} where the hadrons may be considered to be the composite of more elementary constituents whose number also increases with energy.

Let us now mention some standard results for the Riemann zeta Function. The most important is the functional equation satisfied by it \cite{edwards1974}
\begin{equation}
\zeta(s) = 2^s \pi^{s-1} \sin(s\pi/2) \Gamma(1 - s) \zeta(1 - s)
\label{eq:functional}
\end{equation}

I would like to point out the analogy \cite{julia1994} between this functional equation and the Kramers-Wannier duality of the canonical partition function per site for the ``infinite'' planar square Ising model \cite{kramers1941} as given by
\begin{equation}
Z(K_s) = Z(K'_s)/\sinh(2K'_s)
\label{eq:kramers_wannier}
\end{equation}
with
\begin{equation}
\sinh(2K_s) \sinh(2K'_s) = 1
\label{eq:duality_condition}
\end{equation}

It is worth emphasizing that the Kramers-Wannier duality is rather special and the fixed point of the duality equation turns out to be the critical temperature. Also note that both $\zeta$ and $Z$ are real for real $s$ and their sets of zeros are invariant under both duality and complex conjugation. I would also like to emphasize the striking similarity between the result of Fisher \cite{fisher1965} on the location of all the zeros of $Z$ on the circle $|\sinh(2K_s)| = 1$, with the Riemann hypothesis that all nontrivial zeros are on the fixed line $\sigma = 1/2$. In last two years, duality has played a very profound role in supersymmetric gauge field theories as well as string theories \cite{phystoday1995} and it would be worthwhile to enquire if some of those ideas could also be useful in the case of $\zeta$ function.

Using the fundamental functional relationship between $\zeta(s)$ and $\zeta(1 - s)$ as given by Eq.~\eqref{eq:functional}, it is easy to show that
\begin{equation}
\zeta(1/2 - it) = \left(\frac{\pi}{t}\right)^{-it} \frac{\Gamma(1/2 + it/2)}{\Gamma(-it/2)} \zeta(1/2 + it)
\label{eq:symmetry}
\end{equation}
where $\Gamma(z)$ denotes the gamma function of the argument $z$. We may further write
\begin{equation}
\zeta(1/2 + it) = Z(t) \exp(-i\theta(t))
\label{eq:phase_def}
\end{equation}
where $Z(t)$ is real, and $\theta(t)$ is the phase angle, with the convention that $\theta(0) = \pi$. Using Eqs.~\eqref{eq:zeta_def} and \eqref{eq:partition}, it follows that
\begin{equation}
\exp(2i\theta(t)) = \exp(-it \ln \pi) \frac{\Gamma(1/2 + it/2)}{\Gamma(-it/2)}
\label{eq:phase_relation}
\end{equation}

The phase $\theta$, as defined above, is smooth in the sense that it does not include the jumps by $\pi$ due to the zeros of $Z(t)$. Nevertheless, the number of zeros between 0 and $t$ on the $\sigma = 1/2$ line is counted fairly accurately by $\theta(t)$, as will become clear from the Argand diagram. Note that
\begin{equation}
\frac{\theta(t)}{\pi} = -\frac{t}{2\pi} \ln \pi + \frac{1}{\pi} \Im \left[ \ln \Gamma\left(\frac{1}{4} + \frac{it}{2}\right) - \ln \Gamma\left(\frac{1}{4} - \frac{it}{2}\right) \right] + 1
\label{eq:phase_formula}
\end{equation}
which satisfies the condition that $\theta(0) = \pi$. The density of zeros is given by
\begin{equation}
\frac{1}{\pi} \frac{d\theta}{dt} = \frac{1}{2\pi} \left[ -\ln \pi + \psi\left(\frac{1}{4} + \frac{it}{2}\right) - \psi\left(\frac{1}{4} - \frac{it}{2}\right) \right]
\label{eq:zero_density}
\end{equation}
where the digamma function is defined as $\psi(z) = \Gamma'(z)/\Gamma(z)$. From the above, the asymptotic expression for $\theta(t)$ may be obtained immediately by making asymptotic expansion of the $\Gamma$ functions. We denote this by $\vartheta(t)$, and it is given by
\begin{equation}
\vartheta(t) = \frac{1}{2} \left[ \frac{t}{2\pi} \ln\left(\frac{t}{2\pi e}\right) + \frac{7}{8} + \frac{1}{48\pi t} + \ldots \right]
\label{eq:asymptotic_phase}
\end{equation}

\section{Argand diagram and analogy with the scattering amplitude}

To bring out some characteristics of the function $\zeta(1/2 + it)$, we plot its Argand diagram in figure 1(a) in the range $t = 9$ to $t = 50$. This shows a sequence of closed loops, one for every zero of the zeta function. At a zero of $\zeta(1/2 + it)$, both its real and imaginary parts are zero at the same value of $t$, and therefore every loop converges at the origin. The intercepts on the real axis are the so-called ``Gram points'' where only imaginary part of $\zeta(s)$ is zero due to the phase angle $\theta(t) = n\pi$. With infrequent exceptions, there is one Gram point between two consecutive zeros of the $\zeta$ function. The first two exceptions to this rule occur for the 126th and the 134th zeros at $t = 282.455$ and $295.584$ respectively \cite{edwards1974}. In figures 1(b) and 1(c), Argand diagrams are drawn away from the 1/2-axis, for $\sigma = 0.6$ and $\sigma = 1$ respectively. These clearly show the defocussing at the origin due to the absence of the zeros in the Zeta function. Moreover, the number of intercepts along the real-axis in the Argand diagrams now show a large increase compared to the $\sigma = 1/2$ case, whereas the intercepts on the imaginary axis are few or nonexistant. This is a reflection of the change in the behaviour of the phase $\theta(t)$ away from the $\sigma = 1/2$ line. In figure 2(a), the phase angle $\theta(t)$, as determined by eq.~\eqref{eq:phase_formula}, is plotted as a function of $t$ on the 1/2-axis. This phase angle is a smooth function of $t$ because the jumps by $\pi$ at every zero (due to the change in the sign of the $\zeta$-function) is not registered by it. These discontinuities are shown separately in figure 2(b). The smooth phase keeps increasing monotonically with $t$, since the curve in the complex plane passes through the origin at every zero.

Finally, in figures 3(a) and 3(b), the Argand diagrams of the Zeta function are drawn for a much larger range of $t$, from 1 to 500, on and off the 1/2-axis. Note that the scale for $\sigma = 1$ is expanded compared to that for $\sigma = 1/2$. Borrowing from the terminology of the motion of a particle in phase space, it is as if there is an ``attractor'' at the origin for $\sigma = 1/2$ (figure 3(a)), which is absent from the more disorderly tracks of figure 3(b), which is drawn along the $\sigma = 1$ line. The latter figure also shows that the real part of the function is always positive for $\sigma = 1$ for this entire range of $t$.

The loop structure of the function at $\sigma = 1/2$, with some near-circular shapes, is reminiscent of the Argand plots for the scattering amplitudes of different partial waves in the analysis of resonances, for example in pion-nucleon scattering \cite{bhaduri1988}. Consider the partial wave amplitude $f_l(k)$, defined in terms of the partial wave phase-shift $\delta_l(k)$ and the inelasticity parameter $\eta_l(k)$,
\begin{equation}
f_l(k) = \frac{\eta_l \exp(2i\delta_l) - 1}{2ik}
\label{eq:scattering_amplitude}
\end{equation}

Here $l$ refers to the angular momentum, and $k$ the wave number. Note that $\Im f_l(k)$ is never negative, since the inelasticity parameter $\eta_l$ is always less than one. One generally plots an Argand diagram with $2k\Im f_l(k)$ along the y-axis and $2k \Re f_l(k)$ along the x-axis for various values of $k$. For the case of no inelasticity ($\eta_l = 1$) and a single resonance, the Argand diagram is a perfect circle with unit radius, with the center on the imaginary axis at $i$. By comparing this with figure 1(a) at $\sigma = 1/2$, we see that the real and the imaginary parts are interchanged in the latter, but otherwise there is a strong similarity, with many of the loops having inelasticity. This analogy is flawed, however, since $\Re\zeta(1/2 + it)$ does become negative in small islands of $t$. Nevertheless, if these islands are ignored, then the phase shift $\delta_l$ may be identified with the phase angle $\theta + \pi/2$, with each closed loop in figure 1(a) being regarded as an isolated resonance. In this approximation, the Gram points occur as before for $\sin \theta = 0$, while the zeros of $\zeta(1/2+ it)$ are given by the condition
\begin{equation}
\cos \theta = 0, \quad \theta = (m + 1/2)\pi, \quad m = 1,2,\ldots
\label{eq:zero_condition}
\end{equation}

This condition for the location of the zeros was also obtained by Berry \cite{berry1985} from the first term in his approximate formula. Eq.~\eqref{eq:zero_condition} has roots that yield the zeros on the 1/2-axis with an error of at most 3 per cent.

\section{The inverted harmonic oscillator}

On the $\alpha = 1/2$ line, our analogy with the scattering amplitude suggests that the phase angle $\theta(t)$ is related to a scattering phase shift. We now demonstrate that the scattering of a nonrelativistic particle by an inverted harmonic oscillator with a hard wall at the origin generates a phase shift that is closely related to $\vartheta(t)$. Indeed, we show that the quantum density of states for this problem is essentially the same as Eq.~\eqref{eq:zero_density} for the density of the zeros. Consider the Schrödinger equation for $x \geq 0$,
\begin{equation}
-\frac{\hbar^2}{2m} \frac{d^2\Phi}{dx^2} - \frac{1}{2}M\omega^2x^2\Phi = E\Phi
\label{eq:schrodinger}
\end{equation}
and impose the boundary condition that the wave function $\Phi$ vanishes at the origin. Putting $x^2 = y$, $\Phi = y^{-1/4}\psi$, it becomes
\begin{equation}
\frac{d^2\psi}{dy^2} - \frac{k^2}{4y^2}\psi - \frac{1}{4y}\psi + t\psi = 0
\label{eq:transformed_schrodinger}
\end{equation}

In the above equation,
\begin{align}
l &= \frac{1}{4}, \quad k^2 = \frac{m\omega}{\hbar}, \quad t = \frac{E}{\hbar\omega}
\label{eq:parameters}
\end{align}

This is effectively a three dimensional Schrödinger equation for a repulsive Coulomb potential in the variable $y$. To obtain the phase shift, we write the asymptotic solution of the above equation as
\begin{equation}
\psi(y) \sim \sin\left(ky - l\ln(2ky) - \frac{\pi l}{2} + \eta_l\right)
\label{eq:asymptotic_solution}
\end{equation}
where $\eta_l$ is the phase shift with respect to the distorted Coulomb wave, given by $\arg \Gamma(l + 1 + it/2)$. For our one-dimensional problem, only $l = -1/4$ is relevant. For this case, omitting the subscript $l$, the phase shift $\eta$ is
\begin{equation}
\eta = \arg \Gamma\left(\frac{3}{4} + \frac{it}{2}\right) = \frac{\pi\nu}{2}, \quad \frac{\Gamma(1/4 + i\nu)}{\Gamma(1/4 - i\nu)} = \frac{\cosh \pi\nu + i \sinh \pi\nu}{|\Gamma(1/4 + i\nu)|^2}
\label{eq:phase_shift}
\end{equation}

the number of quantum states $n(t)$, between 0 and $t$, is then given by
\begin{equation}
n(t) = \frac{\eta(t)}{\pi} = \frac{1}{\pi} \left[ C + \frac{1}{2\pi} \ln t + \frac{1}{4} - \frac{1}{2\pi} \Im \ln \Gamma\left(\frac{1}{4} + \frac{it}{2}\right) \right]
\label{eq:quantum_states}
\end{equation}

In the above equation, $C$ is a smooth function given by
\begin{equation}
C = \frac{5\pi}{8} - \tan^{-1}(\operatorname{cosech} \pi t)
\label{eq:smooth_function}
\end{equation}

Note from above that the expression for $n(t)$ is not quite identical to $\theta(t)$ as defined in Eq.~\eqref{eq:phase_formula}. However, their derivatives, the quantum density of states, only differ by a constant and an exponentially small term. It should also be pointed out that even if we had started with a full inverted harmonic oscillator (rather than the half-oscillator), the same conclusion would be reached, even though some nonuniqueness may arise in the choice of the boundary condition. The inverted harmonic oscillator problem has been studied by a number of authors in the past in relation to time-delay \cite{ford1959,barton1986} and string theory \cite{brezin1990,gross1990,parisi1990,ginsparg1990}. No connection, however, was made to the phase of $\zeta(1/2+ it)$.

\section{The phase of the Zeta function for $\sigma > 1/2$}

We define the phase of $\zeta(s)$ along the imaginary axis $t$ for a fixed $\sigma$ to be $\theta_\sigma(t)$, given by
\begin{equation}
\zeta(\sigma + it) = |\zeta(s)|\exp(-i\theta_\sigma(t))
\label{eq:phase_general}
\end{equation}

When $\sigma = 1/2$, we shall simply drop the subscript, and denote the phase by $\theta$. Since $\zeta(s)$ is always positive, the discontinuous jumps due to the sign changes are included in the phase $\theta_\sigma$. This is different from the definition of the phase given in Eq.~\eqref{eq:phase_def}, where $\theta$ denoted the smooth part only. From the above definition~\eqref{eq:phase_general}, it follows that
\begin{equation}
\frac{d\theta_\sigma}{dt} = -\frac{d}{dt}(\Im \ln \zeta(s))
\label{eq:phase_derivative}
\end{equation}

It is the derivative of the phase $\theta_\sigma$ with respect to $t$ that contains the information about the density of the zeros on the line $s = 1/2 + it$. We wish to examine the same derivative for $\sigma > 1/2$. To this end, it is convenient to define \cite{edwards1974} the entire function $\xi(s)$ which has the same zeros on the complex plane as $\zeta(s)$,
\begin{equation}
\xi(s) = \frac{1}{2} s(s-1) \pi^{-s/2} \Gamma(s/2) \zeta(s)
\label{eq:xi_function}
\end{equation}

Because $\xi(s)$ is an entire function, it may be expressed as \cite{edwards1974}
\begin{equation}
\xi(s) = \xi(0) \prod_n \left(1 - \frac{s}{s_n}\right)
\label{eq:xi_product}
\end{equation}
where $s_n$ are the zeros of $\xi(s)$ on the complex plane. Some straight-forward algebra then yields
\begin{equation}
\frac{d}{dt} \Im \ln \xi(s) = \sum_n \frac{(\sigma - 1/2)}{(\sigma-1/2)^2 + (t-t_n)^2}
\label{eq:xi_derivative}
\end{equation}

In the above, we have assumed the Riemann hypothesis, that the only zeros of $\xi(s)$ are at $s = s_n = 1/2+ it_n$. Noting the representation of the Dirac delta function
\begin{equation}
\delta(t - t_n) = \lim_{\gamma \to 0} \frac{1}{\pi} \frac{\gamma}{(t-t_n)^2 + \gamma^2}
\label{eq:delta_representation}
\end{equation}
we immediately see that the derivative of the phase changes from delta function spikes to Lorentzian as we move away from the $\sigma = 1/2$ line, with a width $\gamma = (\sigma - 1/2)$. Let us denote the density of the zeros of $\zeta(1/2 + it)$ for $t > 0$ by
\begin{equation}
\rho(t) = \sum_n \delta(t-t_n)
\label{eq:zero_density_def}
\end{equation}

Then the Lorentz smoothed density is expressed as
\begin{equation}
\rho_\sigma(t) = \sum_n \frac{1}{\pi} \frac{(\sigma-1/2)}{(\sigma-1/2)^2 + (t-t_n)^2}
\label{eq:lorentz_smoothed}
\end{equation}

The complete expression for the phase derivative $d\theta_\sigma/dt$ can now be obtained by using the asymptotic Stirling's formula for the large arguments of the $\Gamma$ function. The final expression, after some algebra, is given by
\begin{equation}
\frac{1}{\pi} \frac{d\theta_\sigma}{dt} = \frac{1}{\pi} \frac{d}{dt}(\Im \ln \zeta(s)) = -[\rho_\sigma(t) - \rho_\sigma^{(0)}(t)] = -\delta\rho_\sigma(t)
\label{eq:phase_derivative_final}
\end{equation}
where
\begin{equation}
\rho_\sigma^{(0)}(t) = \frac{1}{2\pi} \ln \left(\frac{|(\sigma-1/2) + it|}{2\pi}\right)^{1/2}
\label{eq:smooth_density}
\end{equation}
and $\rho_\sigma$ is given by Eq.~\eqref{eq:lorentz_smoothed}. In the above equation, we have neglected terms of order $t^{-2}$ in $\rho_\sigma^{(0)}(t)$. For $\sigma = 1/2$, Eq.~\eqref{eq:phase_derivative_final} reduces to
\begin{equation}
\frac{1}{\pi} \frac{d\theta}{dt} = \frac{1}{\pi} \delta\rho(t) = \frac{1}{2\pi} \left[\ln \left(\frac{t}{2\pi}\right) - \psi\left(\frac{1}{4} + \frac{it}{2}\right) - \psi\left(\frac{1}{4} - \frac{it}{2}\right)\right] \sum_n \delta(t-t_n)
\label{eq:phase_derivative_half}
\end{equation}

This sharply discontinuous function, according to Eqs.~\eqref{eq:lorentz_smoothed} and~\eqref{eq:phase_derivative_final}, gets Lorentz-smoothed by a width $(\sigma - 1/2)$ as the derivative of the phase is computed along the imaginary axis at $\sigma > 1/2$. Nevertheless, the position of the Riemann zeros are still remembered, with the memory gradually fading with increasing $\sigma$, as shown in figure 4.

We next consider the Euler product form for $\zeta(s)$ with a view to construct a dynamical model for large $\sigma$. This is given by \cite{edwards1974}
\begin{equation}
\zeta(\sigma + it) = \prod_p \left(1 - \frac{1}{p^{\sigma+it}}\right)^{-1}
\label{eq:euler_product}
\end{equation}
where the product is over all the primes $p$. For $\sigma > 1$, the above expression is convergent. Following \cite{berry1985}, one may then obtain the convergent trace formula
\begin{equation}
\delta\rho_\sigma(t) = -\sum_p \sum_{k=1}^{\infty} \frac{1}{\pi k p^{k\sigma}} \cos(kt \ln p)
\label{eq:trace_formula}
\end{equation}

For $\sigma \gg 1$, only the $p = 2$ term dominates in the above trace-formula, and the contribution of the higher harmonics ($k > 1$) may be neglected. Then Eq.~\eqref{eq:trace_formula} may be written as \cite{berry1986}
\begin{equation}
\delta\rho_\sigma(t) = -\frac{1}{\pi \omega_1} \frac{\cos(2\pi t/\omega_1)}{\sinh(\pi\omega_\sigma/\omega_1)}
\label{eq:simplified_trace}
\end{equation}
where
\begin{equation}
\omega_1 = \frac{2\pi}{\ln 2}, \quad \omega_\sigma = 2\sigma
\label{eq:frequencies}
\end{equation}

We shall now present a dynamical toy model which will generate a semiclassical trace formula, which for its lowest harmonic $k = 1$ is the same as Eq.~\eqref{eq:simplified_trace}. Moreover, in this situation, since only one term survives in the Euler product formula~\eqref{eq:euler_product}, the Argand diagram for $\zeta(s)$ is a circle. The analogy with the Argand diagram for the scattering amplitude is now vivid, with one elastic resonance only.

\section{The density of states for a Toy-model}

\subsection{The Gutzwiller trace formula}

In this section, we first derive semiclassically the trace formula for a particle executing unstable simple harmonic motion in one direction, perched on the edge of an inverted harmonic oscillator in a transverse direction. We also perform the quantum-mechanical calculation to test the semiclassical formula. Consider a particle described by the Hamiltonian ($m = 1$)
\begin{equation}
H = \frac{1}{2}(p_x^2 + \omega_1^2 x^2 + p_y^2 - \omega_2^2 y^2)
\label{eq:hamiltonian}
\end{equation}

In the present problem the situation is rather simple, since there is only one primitive orbit along $x$. The trace formula is of the form
\begin{equation}
\delta\rho_{sc}(E) = \sum_{k=1}^{\infty} \frac{T_1}{\hbar\pi} \frac{1}{\sqrt{|\det(M_1^{(k)}-1)|}} \cos\left(\frac{kS_1(E)}{\hbar} - \frac{\sigma_1^{(k)}\pi}{2}\right)
\label{eq:gutzwiller_formula}
\end{equation}

Here $T_1 = 2\pi/\omega_1$ is the period of the isolated orbit along $x$, and $S_1(E) = 2\pi E/\omega_1$ is the corresponding action. $M_1^{(k)}$ is the $2 \times 2$-monodromy matrix, and $\sigma_1^{(k)}$ the Maslov index. The monodromy calculation is straight-forward, and confirms that the orbit is unstable (hyperbolic).

To establish the notation for this, and the subsequent Maslov index calculation we follow Creagh et al \cite{creagh1990}. We start with a periodic orbit with coordinates given by $q = (x,y)$, and momenta $p = (p_x,p_y)$. The stability of the orbit is determined by the propagation of small perturbations $\delta q(t)$, $\delta p(t)$ away from the periodic solution. The time evolution of these perturbations using the linearized equations of motion may be written in terms of a $4 \times 4$ matrix (called the matrizant) $X(t)$:
\begin{equation}
\begin{pmatrix}
\delta q(t) \\
\delta p(t)
\end{pmatrix} = X(t) \begin{pmatrix}
\delta q(0) \\
\delta p(0)
\end{pmatrix}
\label{eq:matrizant}
\end{equation}
with the initial condition $X(0) = I$. The value of the matrizant after one period (along the $x$-axis $T_1$) is the full monodromy matrix $M_1$:
\begin{equation}
M_1 = X(T_1)
\label{eq:full_monodromy}
\end{equation}

Two of the eigenvalues of $M_1$ are unity, and the $4 \times 4$ matrix $M$ may be written in uncoupled blocks of a $2 \times 2$ monodromy matrix $M_1$ and a $2 \times 2$ unit matrix. It is this reduced matrix $M_1$ that is given above in Eq.~\eqref{eq:gutzwiller_formula}, the superscript $k$ denoting the matrix for $k$-cycles. The monodromy matrix after one complete period is given by (we drop the superscript $k = 1$)
\begin{equation}
M_1(T_1) = \begin{pmatrix}
\cosh(2\pi\omega_2/\omega_1) & \frac{1}{\omega_2}\sinh(2\pi\omega_2/\omega_1) \\
\omega_2\sinh(2\pi\omega_2/\omega_1) & \cosh(2\pi\omega_2/\omega_1)
\end{pmatrix}
\label{eq:monodromy_matrix}
\end{equation}

The eigenvalues are obtained from $\det(M_1 - \lambda I) = 0$ and are found to be $\exp(\pm 2\pi\omega_2/\omega_1)$. Therefore
\begin{equation}
\sqrt{|\det(M_1 - I)|} = 2 \sinh\left(\frac{\pi\omega_2}{\omega_1}\right)
\label{eq:determinant}
\end{equation}

We next calculate the Maslov index $\sigma_1^{(k)}$ occuring in Eq.~\eqref{eq:gutzwiller_formula}. For an unstable (hyperbolic) orbit, the Maslov index for the $k$th harmonic is simply $k\sigma_1$. For our case, $\sigma_1$ can be calculated analytically and one can show that the Maslov index is simply $2k$. The semiclassical trace formula given by Eq.~\eqref{eq:gutzwiller_formula} now reduces to
\begin{equation}
\delta\rho_{sc}(E) = \sum_{k=1}^{\infty} \frac{1}{\pi\omega_1} \frac{(-1)^k \cos(2\pi kE/\hbar\omega_1)}{\sinh(\pi k\omega_2/\omega_1)}
\label{eq:semiclassical_trace}
\end{equation}

For $\omega_2/\omega_1 \gg 1$, the higher harmonics are severely damped, and the above equation is of the same form as given by Eq.~\eqref{eq:simplified_trace}. Note from Eq.~\eqref{eq:trace_formula}, however, that there is an overall negative sign in the trace formula for the zeros of the Riemann Zeta function, indicating that the Maslov index for every periodic orbit, irrespective of the harmonic, is $\pi$. Our dynamical formula~\eqref{eq:semiclassical_trace}, on the other hand, has alternating sign with the harmonic number $k$, since our Maslov index is $k\pi$. This anomaly in the Maslov index has been noted in the literature \cite{berry1990}. Berry has speculated that it may not be possible to overcome this difficulty with a Hamiltonian that is separable into kinetic and potential terms.

Our next task is to perform a quantum-mechanical calculation of $\delta\rho(E)$ to check the validity of the semiclassical result.

\subsection{The quantum density of states}

We seek to obtain the quantum density of states for the two-dimensional system given by Eq.~\eqref{eq:hamiltonian}. Since there is no interaction coupling the two degrees of freedom, the desired density of states may be readily obtained by convoluting the quantum density of the harmonic oscillator in the $x$-direction with that of the inverted oscillator in the $y$-direction. The density of states of the harmonic oscillator with oscillator constant $\omega_1$ is given by
\begin{equation}
\rho_1(E) = \sum_{n=0}^{\infty} \delta(E - (n + 1/2)\hbar\omega_1)
\label{eq:harmonic_density}
\end{equation}
where the energy $E > 0$. The quantum density of states of an inverted harmonic oscillator has been examined carefully by Barton \cite{barton1986} and is given by
\begin{equation}
\rho_2(E) = \frac{1}{\pi\hbar\omega_2} \Re\psi(1/2+ iE/\hbar\omega_2)
\label{eq:inverted_density}
\end{equation}
where $\psi(z) = d/dz\ln\Gamma(z)$ is the digamma function. In the above equation, we have omitted a constant term $2/\pi \ln L$ that arises from confining the system in a large box of length $L$. To compare with the semiclassical density of states given by Eq.~\eqref{eq:semiclassical_trace}, we convolute the quantum densities~\eqref{eq:harmonic_density} and~\eqref{eq:inverted_density} for a given energy $E > 0$. This total energy $E$ is shared between the two degrees of freedom of this conservative system. The convoluted density $\rho(E)$ of the system is
\begin{equation}
\rho(E) = \int_{-\infty}^{E} \rho_1(E-E')\rho_2(E') dE'
\label{eq:convolution}
\end{equation}

Note that the lower limit in the integration is $-\infty$, since a negative energy $E'$ of the inverted oscillator may add up with the energy $(E - E')$ of the harmonic oscillator to still yield the energy $E > 0$ of the total system. Substituting from Eqs.~\eqref{eq:harmonic_density} and~\eqref{eq:inverted_density} above, we get
\begin{equation}
\rho(E) = \frac{1}{\pi\hbar\omega_2} \sum_{n=0}^{\infty} \Re\psi\left(\frac{1}{2} + i\frac{E - (n+1/2)\hbar\omega_1}{\hbar\omega_2}\right)
\label{eq:quantum_density}
\end{equation}

We proceed to examine to what extent the semiclassical formula, given by Eq.~\eqref{eq:semiclassical_trace}, agrees with the quantum result, Eq.~\eqref{eq:quantum_density}.

\subsection{Comparison of the semiclassical and quantum formulae}

The quantum formula given above is for the full density of states, comprising both the smooth and the oscillating parts, whereas the semiclassical result~\eqref{eq:semiclassical_trace} is only for the oscillating part $\delta\rho(E)$. It is therefore necessary to subtract off the smooth part of the density of states from the quantum result before a comparison is made. The smooth part is most readily obtained numerically by performing a Strutinsky-type Gaussian smoothing \cite{strutinsky1966,ross1972} on the quantum density. The method consists of smoothing the quantum density of states, given by Eq.~\eqref{eq:quantum_density}, by a Gaussian of width $\gamma$, and modulated by a curvature function $P_n[(E - E')/\gamma]$:
\begin{equation}
\tilde{\rho}(E) = \frac{1}{\gamma\sqrt{\pi}} \int_{-\infty}^{\infty} \rho(E') \exp\left(-\frac{(E-E')^2}{\gamma^2}\right) P_n\left(\frac{E-E'}{\gamma}\right) dE'
\label{eq:smoothing}
\end{equation}
where the smoothing width $\gamma$ is larger than the spacing $\hbar\omega_1$ of the oscillator shells. The curvature function $P_n$ for Gaussian smoothing is given by the associated Laguerre polynomial of order $2n$:
\begin{equation}
P_n(x) = L_n^{(1/2)}(x^2)
\label{eq:curvature_function}
\end{equation}

The curvature function $P_n(x)$ ensures the internal consistency of the smoothing procedure if $\rho$ is a polynomial of order $(2n+1)$. In our calculation, a curvature function with $2n = 2$ is sufficient to yield an accurate $\tilde{\rho}(E)$. The smoothing width $\gamma$ was taken to be 20 and the oscillator spacing $\hbar\omega_1$ was kept fixed at 10 (for $\hbar = 1$). The oscillating part of the density of states $\delta\rho(E)$ is obtained by taking $\rho(E) - \tilde{\rho}(E)$. This is compared with the semiclassical form~\eqref{eq:semiclassical_trace} as a function of $E$ for various values of the ratio $\omega_2/\omega_1$. The results of the calculation are displayed in figures 5 and 6. The agreement between the quantum and semiclassical densities is excellent, and the difference is almost negligible.

An instructive way to analyze the semiclassical formula~\eqref{eq:semiclassical_trace} is to construct a dynamical (or Selberg) zeta function $\zeta_s(s)$ from it, and quantize its energies by setting $\zeta_s$ to zero. Let us define
\begin{equation}
\zeta_s(E) = |\zeta_s(E)| \exp(-i\theta_s)
\label{eq:selberg_def}
\end{equation}

Hence we may write, following Bogomolny \cite{bogomolny1992}
\begin{equation}
\frac{1}{\pi} \frac{d\theta_s}{dE} = \frac{1}{\pi} \frac{d}{dE}(\Im \ln \zeta_s(E)) = \delta\rho_{sc}(E)
\label{eq:selberg_phase}
\end{equation}
where $\delta\rho_{sc}(E)$ is given by Eq.~\eqref{eq:semiclassical_trace}. Note that the sign of $\delta\rho_{sc}$ in the above equation is opposite to that of the analogous Eq.~\eqref{eq:phase_derivative_final} for the Riemann zeta function. In the latter, as was discussed in the earlier section, there is an overall minus sign in the trace formula~\eqref{eq:trace_formula} for $\delta\rho_\sigma$, which may be regarded as arising from a Maslov index of $\pi$ common to every periodic orbit \cite{berry1990}. This extra negative sign is disregarded in~\eqref{eq:selberg_phase}, since such a common Maslov index does not arise in the dynamical case.

From Eqs.~\eqref{eq:semiclassical_trace} and~\eqref{eq:selberg_phase}, we deduce that
\begin{align}
\theta_s(E) &= \frac{1}{2} \sum_{k=1}^{\infty} \frac{(-1)^k \sin(2\pi kE/\hbar\omega_1)}{k \sinh(k\pi\omega_2/\omega_1)} \nonumber \\
&= \frac{1}{2} \Im \ln \prod_{l=0}^{\infty} \left( 1 + \frac{(-1)^l \exp(2\pi ilE/\hbar\omega_1)}{k \sinh(k\pi\omega_2/\omega_1)} \right)
\label{eq:selberg_phase_explicit}
\end{align}

Making use of the identity
\begin{equation}
\frac{1}{2 \sinh(x/2)} = \sum_{l=0}^{\infty} \exp\left(-\left(l + \frac{1}{2}\right)x\right)
\label{eq:sinh_identity}
\end{equation}
and summing the series over $k$, we obtain
\begin{equation}
\theta_s(E) = -\Im \ln \prod_{l=0}^{\infty} \left( 1 + \exp\left( \frac{2\pi i}{ℏ\omega_1}\left( E + \left(l + \frac{1}{2}\right)\hbar\omega_2 \right) \right) \right)
\label{eq:selberg_phase_final}
\end{equation}

From Eq.~\eqref{eq:selberg_def}, $\theta_s$ is the same as $-\Im \ln \zeta_s$, so we obtain for the Selberg zeta function the expression
\begin{equation}
\zeta_s(E) = \prod_{l=0}^{\infty} \left( 1 + \exp\left( \frac{2\pi i}{\hbar\omega_1}\left( E + \left(l + \frac{1}{2}\right)\hbar\omega_2 \right) \right) \right)
\label{eq:selberg_zeta}
\end{equation}

The quantized energies are obtained by setting this equal to zero, giving
\begin{equation}
E = (n + 1/2)\hbar\omega_1 - i(l + 1/2)\hbar\omega_2
\label{eq:quantized_energies}
\end{equation}

where $l, n$ take on positive integral values including zero. Note that in our case, the quantized energies $E$ are complex. It is gratifying to see that the semiclassical Gutzwiller trace formula is so successful in describing the oscillating density of states in the continuum. To our knowledge, such an application has not been made before. It is all the more interesting because this toy-model allows us to construct a convergent dynamical zeta function. Furthermore, the model itself has some relevance, for asymptotically large $\sigma$, to the Riemann zeta function.

\begin{thebibliography}{21}
\bibitem{edwards1974} H. M. Edwards, \textit{Riemann's zeta function} (Academic Press, New York, 1974); A. Ivic, \textit{The Riemann zeta function} (John Wiley, New York, 1985)

\bibitem{hardy1914} G. H. Hardy, Comptes Rendus \textbf{CLVIII}, 280 (1914)

\bibitem{berry1985} M. V. Berry, Proc. R. Soc. London \textbf{A400}, 229 (1985); M. C. Gutzwiller, \textit{Chaos in classical and quantum mechanics} (Springer Verlag, New York, 1991); Hua Wu, and D. W. L. Sprung, Phys. Rev. \textbf{E48}, 2595 (1993)

\bibitem{berry1985b} M. V. Berry, Proc. R. Soc. London \textbf{A400}, 229 (1985)

\bibitem{berry1986} M. V. Berry, in \textit{Quantum chaos and statistical nuclear physics} (edited by T. H. Seligman and H. Nishioka, Lecture Notes in Physics, No. 263, Springer Verlag, New York 1986) p.~1

\bibitem{sieber1990} M. Sieber and F. Steiner, Physica \textbf{D44}, 248 (1990); Phys. Lett. \textbf{A148}, 415 (1990); A. M. Odlyzko, Math. Comput. \textbf{48}, 273 (1987)

\bibitem{bhaduri1995a} R. K. Bhaduri, A. Khare and J. Law, Phys. Rev. \textbf{E52}, 486 (1995)

\bibitem{bhaduri1995b} R. K. Bhaduri, A. Khare, S. M. Reimann and E. L. Tomusiak, Bhubaneswar Preprint IP-BBSR/95-57; Ann. Phys. (N.Y.) (in press)

\bibitem{jennings1974} B. K. Jennings, (1974) Unpublished

\bibitem{hagedorn1965} R. Hagedorn, Suppl. Nuovo Cimento \textbf{3}, 147 (1965)

\bibitem{julia1994} See for example, B. L. Julia, Physica \textbf{A203}, 425 (1994)

\bibitem{kramers1941} H. A. Kramers and G. H. Wannier, Phys. Rev. \textbf{60}, 263 (1941)

\bibitem{fisher1965} M. E. Fisher, Lectures in theoretical physics \textbf{C7}, (1965) 1, Univ. of Colorado Press, Boulder

\bibitem{phystoday1995} see for example, Phys. Today, p.~17, March 95 Issue and references therein

\bibitem{bhaduri1988} See for example, R. K. Bhaduri, \textit{Models of the nucleon} (Addison Wesley, Reading, MA, 1988) p.~27

\bibitem{ford1959} K. W. Ford, D. L. Hill, M. Wakano, and J. A. Wheeler, Ann. Phys. (N.Y.) \textbf{7}, 239 (1959)

\bibitem{barton1986} G. Barton, Ann. Phys. (N.Y.) \textbf{166}, 322 (1986)

\bibitem{brezin1990} E. Brézin, V. A. Kazakov and Al. B. Zamolodchikov, Nucl. Phys. \textbf{B338}, 673 (1990)

\bibitem{gross1990} D. J. Gross and N. Miljković, Phys. Lett. \textbf{B238}, 217 (1990)

\bibitem{parisi1990} G. Parisi, Phys. Lett. \textbf{B238}, 213 (1990)

\bibitem{ginsparg1990} P. Ginsparg and J. Zinn-Justin, Phys. Lett. \textbf{B240}, 333 (1990)

\bibitem{creagh1990} S. C. Creagh, J. M. Robbins and R. G. Littlejohn, Phys. Rev. \textbf{A42}, 1907 (1990)

\bibitem{berry1990} M. V. Berry and J. P. Keating, J. Phys. \textbf{A23}, 4839 (1990)

\bibitem{strutinsky1966} V. M. Strutinsky, Yad. Fiz. \textbf{3}, 614 (1966), Nucl. Phys. \textbf{A95}, 420 (1967)

\bibitem{ross1972} C. K. Ross and R. K. Bhaduri, Nucl. Phys. \textbf{A188}, 566 (1972)

\bibitem{bogomolny1992} E. B. Bogomolny, Nonlinearity \textbf{5}, 805 (1992)

\end{thebibliography}

\end{document}
