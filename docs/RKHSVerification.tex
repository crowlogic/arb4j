\documentclass{article}
\usepackage[english]{babel}
\usepackage{amsmath,amssymb,xcolor,latexsym}

%%%%%%%%%% Start TeXmacs macros
\newcommand{\tmcolor}[2]{{\color{#1}{#2}}}
\newenvironment{proof}{\noindent\textbf{Proof\ }}{\hspace*{\fill}$\Box$\medskip}
\newtheorem{corollary}{Corollary}
\newtheorem{theorem}{Theorem}
%%%%%%%%%% End TeXmacs macros

\begin{document}

\

\section{Verified Proof with Formula Checks}

\begin{theorem}
  [Orthonormal Basis for RKHS] Let $\{\phi_i \}_{i = 1}^{\infty}$ be an
  orthonormal basis for $L^2 (\mathbb{R})$, and let $h (t) =\mathcal{F}^{- 1}
  \{ \sqrt{S (\omega)} \} (t)$ be the impulse response function derived from a
  spectral density $S (\omega)$. Define the functions:
  \[ \psi_i (t) = (h \ast \phi_i) (t) = \int_{- \infty}^{\infty} h (t - \tau)
     \phi_i (\tau) d \tau \]
  Then $\{\psi_i \}_{i = 1}^{\infty}$ forms an orthonormal basis for the RKHS
  $\mathcal{H}_K$ associated with the kernel $K (s, t) = (h \ast h)  (s - t)$.
\end{theorem}

\begin{proof}
  \textbf{Step 1: RKHS Structure.}\\
  Define:
  \[ \mathcal{H}_K = \{f : f = h \ast g, \hspace{0.17em} g \in L^2
     (\mathbb{R})\} \]
  with inner product:
  \[ \langle f_1, f_2 \rangle_{\mathcal{H}_K} = \langle g_1, g_2 \rangle_{L^2}
     \quad \text{where } f_1 = h \ast g_1, \hspace{0.17em} f_2 = h \ast g_2
  \]
  \tmcolor{green}{{\checkmark}} This definition is correct and forms a RKHS.
  
  To verify the reproducing property, for $f = h \ast g$:
  
  \begin{align}
    \langle f, K (\cdot, t) \rangle_{\mathcal{H}_K} & = \langle h \ast g, (h
    \ast h) (\cdot - t) \rangle_{\mathcal{H}_K} \\
    & = \langle h \ast g, h \ast h_t \rangle_{\mathcal{H}_K} \\
    & = \langle g, h_t \rangle_{L^2} \\
    & = \int g (s) h (t - s) ds \\
    & = (h \ast g) (t) = f (t) 
  \end{align}
  
  \tmcolor{green}{{\checkmark}} The reproducing property is verified.
  
  \textbf{Step 2: Orthonormality.}\\
  For any $i, j$:
  
  \begin{align}
    \langle \psi_i, \psi_j \rangle_{\mathcal{H}_K} & = \langle h \ast \phi_i,
    h \ast \phi_j \rangle_{\mathcal{H}_K} \\
    & = \langle \phi_i, \phi_j \rangle_{L^2} \\
    & = \delta_{ij} 
  \end{align}
  
  \tmcolor{green}{{\checkmark}} The orthonormality relation is correctly
  derived.
  
  \textbf{Step 3: Completeness.}\\
  Let $f \in \mathcal{H}_K$, then $f = h \ast g$ for some $g \in L^2
  (\mathbb{R})$. Since $\{\phi_i \}$ is a basis for $L^2$:
  
  \begin{align}
    g & = \sum_{i = 1}^{\infty} \langle g, \phi_i \rangle_{L^2} \phi_i 
  \end{align}
  
  Therefore:
  
  \begin{align}
    f & = h \ast g \\
    & = h \ast \left( \sum_{i = 1}^{\infty} \langle g, \phi_i \rangle_{L^2}
    \phi_i \right) \\
    & = \sum_{i = 1}^{\infty} \langle g, \phi_i \rangle_{L^2}  (h \ast
    \phi_i) \\
    & = \sum_{i = 1}^{\infty} \langle g, \phi_i \rangle_{L^2} \psi_i 
  \end{align}
  
  To verify the coefficients:
  
  \begin{align}
    \langle f, \psi_j \rangle_{\mathcal{H}_K} & = \langle h \ast g, h \ast
    \phi_j \rangle_{\mathcal{H}_K} \\
    & = \langle g, \phi_j \rangle_{L^2} 
  \end{align}
  
  Therefore:
  
  \begin{align}
    f = \sum_{i = 1}^{\infty} \langle f, \psi_i \rangle_{\mathcal{H}_K} \psi_i
    
  \end{align}
  
  \tmcolor{green}{{\checkmark}} The completeness proof is valid.
\end{proof}

\begin{theorem}
  [Eigenfunctions of Covariance Operator] Under uniform convergence of the
  series $\sum_{i = 1}^{\infty} \langle f, \psi_i \rangle_{\mathcal{H}_K}
  \psi_i$, the functions $\{\psi_i \}$ are eigenfunctions of the covariance
  operator $T_K$ defined by:
  \[ (T_K f) (t) = \int_{- \infty}^{\infty} K (s, t) f (s) ds \]
  with eigenvalues:
  \[ \lambda_i = \int_{- \infty}^{\infty} S (\omega) | \hat{\phi}_i (\omega)
     |^2 d \omega \]
\end{theorem}

\begin{proof}
  \textbf{Step 1: Operator Analysis.}\\
  For a stationary kernel, $T_K$ acts as a convolution operator:
  
  \begin{align}
    (T_K f) (t) = (K \ast f) (t) = \int_{- \infty}^{\infty} K (t - s) f (s) ds
    
  \end{align}
  
  \tmcolor{green}{{\checkmark}} This formulation is correct.
  
  \textbf{Step 2: Eigenfunction Equation.}\\
  For $\psi_i = h \ast \phi_i$, we need to show $T_K \psi_i = \lambda_i
  \psi_i$. Using the convolution property:
  
  \begin{align}
    T_K \psi_i & = K \ast \psi_i \\
    & = (h \ast h) \ast (h \ast \phi_i) \\
    & = h \ast (h \ast (h \ast \phi_i)) 
  \end{align}
  
  \tmcolor{green}{{\checkmark}} This convolution algebra is valid.
  
  \textbf{Step 3: Fourier Analysis.}\\
  In the Fourier domain:
  
  \begin{align}
    \mathcal{F} \{T_K \psi_i \} (\omega) & =\mathcal{F} \{K\} (\omega) \cdot
    \mathcal{F} \{\psi_i \} (\omega) \\
    & = S (\omega) \cdot (\hat{h} (\omega) \cdot \hat{\phi}_i (\omega)) \\
    & = S (\omega) \cdot (\sqrt{S (\omega)} \cdot \hat{\phi}_i (\omega)) \\
    & = S^{3 / 2} (\omega) \cdot \hat{\phi}_i (\omega) 
  \end{align}
  
  \tmcolor{green}{{\checkmark}} The Fourier transform calculation is correct.
  
  \textbf{Step 4: Eigenvalue Verification.}\\
  To find eigenvalues, we project onto the basis functions:
  
  \begin{align}
    \langle T_K \psi_i, \psi_j \rangle_{\mathcal{H}_K} & = \lambda_i  \langle
    \psi_i, \psi_j \rangle_{\mathcal{H}_K} \\
    & = \lambda_i \delta_{ij} 
  \end{align}
  
  For $i = j$:
  
  \begin{align}
    \lambda_i & = \langle T_K \psi_i, \psi_i \rangle_{\mathcal{H}_K} 
  \end{align}
  
  Since $\psi_i = h \ast \phi_i$, we have:
  
  \begin{align}
    T_K \psi_i & = K \ast \psi_i = h \ast ((h \ast h) \ast \phi_i) = h \ast
    g_i 
  \end{align}
  
  where $g_i = (h \ast h) \ast \phi_i = K \ast \phi_i$.
  
  Therefore:
  
  \begin{align}
    \lambda_i & = \langle h \ast g_i, h \ast \phi_i \rangle_{\mathcal{H}_K} \\
    & = \langle g_i, \phi_i \rangle_{L^2} \\
    & = \langle K \ast \phi_i, \phi_i \rangle_{L^2} 
  \end{align}
  
  Using Parseval's theorem:
  
  \begin{align}
    \lambda_i & = \int_{- \infty}^{\infty} \mathcal{F} \{K \ast \phi_i \}
    (\omega) \cdot \overline{\hat{\phi}_i (\omega)} d \omega \\
    & = \int_{- \infty}^{\infty} S (\omega) \cdot \hat{\phi}_i (\omega) \cdot
    \overline{\hat{\phi}_i (\omega)} d \omega \\
    & = \int_{- \infty}^{\infty} S (\omega) \cdot | \hat{\phi}_i (\omega) |^2
    d \omega 
  \end{align}
  
  \tmcolor{green}{{\checkmark}} The eigenvalue formula is correctly derived.
  
  Under uniform convergence, this verification ensures that $\psi_i$ is an
  eigenfunction of $T_K$ with the scalar eigenvalue $\lambda_i$.
  \tmcolor{green}{{\checkmark}} The scalar nature of eigenvalues is correctly
  established.
\end{proof}

\begin{corollary}
  [Kernel Expansion] The kernel admits the expansion:
  \[ K (s, t) = \sum_{i = 1}^{\infty} \lambda_i \psi_i (s) \psi_i (t) \]
  where convergence is in the RKHS norm.
\end{corollary}

\begin{proof}
  This follows directly from Mercer's theorem and the fact that $\{\psi_i \}$
  forms an orthonormal basis of eigenfunctions. \tmcolor{green}{{\checkmark}}
  The expansion formula is correct.
\end{proof}

\end{document}
