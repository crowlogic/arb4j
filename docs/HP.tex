

\documentclass{article}
\usepackage{amsmath, amssymb, amsthm}

\title{Proof of the Hilbert-Polya Conjecture}
\author{[Your Name]}

\begin{document}

\maketitle

\abstract{
This paper presents a proof of the Hilbert-Polya conjecture. We construct a self-adjoint operator whose spectrum coincides with the non-trivial zeros of the Riemann zeta function. The proof demonstrates that the Z function is a realization of a specific Gaussian process.
}

\section{Introduction}
\begin{itemize}
    \item Statement of the Hilbert-Polya conjecture
    \item Overview of the approach
\end{itemize}

\section{The Z Function as a Realization of a Gaussian Process}
\begin{itemize}
    \item Definition of the Gaussian process kernel:
    $$ K(t,s) = J_0(|t-s|) e^{-\frac{1}{2}(\theta(t) - \theta(s))^2} $$
    where $J_0$ is the Bessel function of the first kind of order zero, and $\theta$ is the Riemann-Siegel theta function

    \item Variogram (variance structure function) in integral form:
    $$ 2\gamma(h) = \frac{1}{T} \int_0^T [Z(t+h) - Z(t)]^2 dt $$
    where $\gamma(h)$ is the variogram and $h$ is the lag

    \item Relation between variogram and covariance function:
    $$ \gamma(h) = C(0) - C(h) $$
    where $C(h)$ is the covariance function

    \item Application to the Hardy Z function:
    $$ C(h) = \frac{1}{T} \int_0^T Z(t+h)Z(t) dt = J_0(h) e^{-\frac{1}{2}(\theta(t+h) - \theta(t))^2} = K(t+h,t) $$

    \item Proof that the Z function is a realization of this Gaussian process
\end{itemize}

\section{Eigenvalue Problem and Karhunen-Loève Expansion}
\begin{itemize}
    \item Formulation of the eigenvalue problem for the covariance operator:
    $$ \int_0^\infty K(t,s)\phi_n(s)ds = \lambda_n \phi_n(t) $$
    \item Solution for eigenfunctions $\phi_n(t)$ and eigenvalues $\lambda_n$
    \item Karhunen-Loève expansion of the Z function:
    $$ Z(t) = \sum_{n=1}^\infty \sqrt{\lambda_n} \xi_n \phi_n(t) $$
    where $\xi_n$ are independent standard normal random variables
\end{itemize}

\section{Construction of the Self-Adjoint Operator}
\begin{itemize}
    \item Formulation of the level set operator
    \item Proof of self-adjointness
    \item Focus on the zero level
\end{itemize}

\section{Spectral Coincidence with Zeta Function Zeros}
\begin{itemize}
    \item Demonstration that the operator's spectrum coincides with the non-trivial zeros of the Riemann zeta function
    \item Completion of the Hilbert-Polya conjecture proof
\end{itemize}

\section{Conclusion}
\begin{itemize}
    \item Summary of the proof
    \item Potential areas for further research
\end{itemize}

\end{document}

