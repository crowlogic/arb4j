\documentclass{article}
\usepackage[english]{babel}
\usepackage{amssymb,latexsym,theorem}

%%%%%%%%%% Start TeXmacs macros
\newcommand{\assign}{:=}
\newcommand{\tmem}[1]{{\em #1\/}}
\newenvironment{proof}{\noindent\textbf{Proof\ }}{\hspace*{\fill}$\Box$\medskip}
\newtheorem{definition}{Definition}
\newtheorem{proposition}{Proposition}
{\theorembodyfont{\rmfamily}\newtheorem{remark}{Remark}}
\newtheorem{theorem}{Theorem}
%%%%%%%%%% End TeXmacs macros

\begin{document}

\title{Pushforward and Pullback Operations in Measure Theory}

\date{}

\maketitle

\section{Introduction}

Pushforward and pullback operations constitute essential techniques in measure
theory, enabling the transfer of measures and functions between measurable
spaces via measurable transformations. These constructs are fundamentally
connected to the behavior of $\sigma$-algebras and the preservation of
measurability under mapping, providing indispensable tools for analysis,
probability theory, and related fields.

\section{Preliminaries}

\begin{definition}
  Let $(X, \mathcal{A})$ and $(Y, \mathcal{B})$ be measurable spaces, meaning
  $X$ and $Y$ are sets, and $\mathcal{A}, \mathcal{B}$ are $\sigma$-algebras
  of subsets of $X$ and $Y$, respectively. A map $f : X \to Y$ is said to be
  {\tmem{measurable}} if for all $B \in \mathcal{B}$, the preimage $f^{- 1}
  (B) \in \mathcal{A}$.
\end{definition}

Let $\mu$ be a measure on $(X, \mathcal{A})$ and $\nu$ a measure on $(Y,
\mathcal{B})$.

\section{The Pushforward (Image) Measure}

\begin{definition}
  [Pushforward Measure] Let $f : X \to Y$ be a measurable function and $(X,
  \mathcal{A}, \mu)$ a measure space. The {\tmem{pushforward}} (or image)
  measure $f_{\ast} \mu$ on $(Y, \mathcal{B})$ is defined by
  \begin{equation}
    (f_{\ast} \mu) (B) = \mu (f^{- 1} (B)) \forall B \in \mathcal{B}
  \end{equation}
\end{definition}

\begin{theorem}
  Let $(X, \mathcal{A}, \mu)$ be a measure space, $f : X \to Y$ a measurable
  function, and $(Y, \mathcal{B})$ a measurable space. Then $f_{\ast} \mu$ is
  a measure on $(Y, \mathcal{B})$.
\end{theorem}

\begin{proof}
  It suffices to verify the properties of a measure:
  \begin{enumerate}
    \item {\tmem{Non-negativity}}:
    \begin{equation}
      (f_{\ast} \mu) (B) = \mu (f^{- 1} (B)) \geq 0 \forall B \in \mathcal{B}
    \end{equation}
    since $\mu$ is a measure.
    
    \item {\tmem{Null empty set}}:
    \begin{equation}
      (f_{\ast} \mu) (\emptyset) = \mu (f^{- 1} (\emptyset)) = \mu (\emptyset)
      = 0
    \end{equation}
    \item {\tmem{Countable additivity}}: Let $(B_n)_{n \in \mathbb{N}} \subset
    \mathcal{B}$ be pairwise disjoint. Then
    \begin{equation}
      \begin{array}{cl}
        (f_{\ast} \mu) \left( \bigcup_{n = 1}^{\infty} B_n \right) & = \mu
        \left( f^{- 1} \left( \bigcup_{n = 1}^{\infty} B_n \right) \right)\\
        & = \mu \left( \bigcup_{n = 1}^{\infty} f^{- 1} (B_n) \right)\\
        & = \sum_{n = 1}^{\infty} \mu (f^{- 1} (B_n))\\
        & = \sum_{n = 1}^{\infty} (f_{\ast} \mu) (B_n)
      \end{array}
    \end{equation}
    where the third equality uses measurability of $f$ and the fact that
    preimages preserve unions and disjointness.
  \end{enumerate}
  Thus, $f_{\ast} \mu$ is a measure.
\end{proof}

\begin{remark}
  If $\mu$ is a probability measure, then so is $f_{\ast} \mu$. In this
  context, $f_{\ast} \mu$ describes the distribution of the random variable
  $f$ induced by $\mu$.
\end{remark}

\section{The Pullback Operation for Measurable Functions}

The pullback operation allows the transfer of functions and, in more elaborate
contexts, measures across measurable spaces.

\begin{definition}
  [Pullback of a Function] Let $f : X \to Y$ be a measurable function and $g :
  Y \to \mathbb{R}$ a $\mathcal{B}$-measurable function. The {\tmem{pullback}}
  of $g$ along $f$, denoted $f^{\ast} g$, is defined by
  \begin{equation}
    f^{\ast} g \assign g \circ f, \quad x \mapsto g (f (x))
  \end{equation}
  for $x \in X$.
\end{definition}

\begin{theorem}
  If $g : Y \to \mathbb{R}$ is $\mathcal{B}$-measurable and $f : X \to Y$ is
  $\mathcal{A}$-$\mathcal{B}$-measurable, then $f^{\ast} g = g \circ f$ is
  $\mathcal{A}$-measurable.
\end{theorem}

\begin{proof}
  Let $B \in \mathcal{B} (\mathbb{R})$, the Borel $\sigma$-algebra on
  $\mathbb{R}$. Then
  \begin{equation}
    (f^{\ast} g)^{- 1} (B) = \{x \in X : g (f (x)) \in B\} = f^{- 1} (g^{- 1}
    (B))
  \end{equation}
  Since $g$ is $\mathcal{B}$-measurable, $g^{- 1} (B) \in \mathcal{B}$. Since
  $f$ is $\mathcal{A}$-$\mathcal{B}$-measurable, $f^{- 1} (g^{- 1} (B)) \in
  \mathcal{A}$. Thus, $f^{\ast} g$ is $\mathcal{A}$-measurable.
\end{proof}

\section{Pullback of a Measure: Theoretical Caveat}

Generally, the pullback of a measure via a function is not always well
defined. In particular, given a measure $\nu$ on $(Y, \mathcal{B})$ and a
measurable $f : X \to Y$, the set function
\begin{equation}
  \mu (A) \assign \nu (f (A)) \forall A \in \mathcal{A}
\end{equation}
is not, in general, a measure. Issues arise due to the failure of countable
additivity unless $f$ is injective or further structure is present.

\begin{remark}
  A legitimate pullback of measures (under the name {\tmem{inverse image
  measure}}) exists in the context of differentiable manifolds, or via the
  theory of signed measures and distributions, but not in general for
  arbitrary measure spaces.
\end{remark}

\section{The Push-Pull Formula (Change of Variables)}

\begin{theorem}
  [Pushforward and Integration (Change of Variables)] Let $(X, \mathcal{A},
  \mu)$ be a measure space, $(Y, \mathcal{B})$ a measurable space, and $f : X
  \to Y$ a measurable map. Let $g : Y \to [0, + \infty]$ be
  $\mathcal{B}$-measurable. Then
  \begin{equation}
    \int_Y g \hspace{0.17em} d (f_{\ast} \mu) = \int_X g \circ f
    \hspace{0.17em} d \mu
  \end{equation}
\end{theorem}

\begin{proof}
  Consider first $g = \textbf{1}_B$ for $B \in \mathcal{B}$. Then
  \begin{equation}
    \int_Y \textbf{1}_B  \hspace{0.17em} d (f_{\ast} \mu) = (f_{\ast} \mu) (B)
    = \mu (f^{- 1} (B)) = \int_X \textbf{1}_{f^{- 1} (B)}  \hspace{0.17em} d
    \mu = \int_X (\textbf{1}_B \circ f)  \hspace{0.17em} d \mu
  \end{equation}
  By linearity and monotone convergence, the result extends to all
  non-negative $\mathcal{B}$-measurable functions $g$.
\end{proof}

\section{Interrelationships and Further Properties}

\begin{proposition}
  The assignments $f \mapsto f_{\ast}$ and $g \mapsto f^{\ast} g$ are
  functorial in the sense that
  \begin{enumerate}
    \item For measurable maps $f : X \to Y$ and $g : Y \to Z$, $(g \circ
    f)_{\ast} = g_{\ast} \circ f_{\ast}$ as assignments on measures.
    
    \item For measurable maps $f : X \to Y$ and $g : Y \to Z$, $(g \circ
    f)^{\ast} = f^{\ast} \circ g^{\ast}$ as assignments on functions.
  \end{enumerate}
\end{proposition}

\begin{proof}
  
  \begin{enumerate}
    \item Let $\mu$ be a measure on $X$. For $C \in \mathcal{C}$ (where
    $\mathcal{C}$ is a $\sigma$-algebra on $Z$):
    \begin{equation}
      \begin{array}{ll}
        ((g \circ f)_{\ast} \mu) (C) & = \mu ((g \circ f)^{- 1} (C))\\
        & = \mu (f^{- 1} (g^{- 1} (C)))\\
        & = (f_{\ast} \mu) (g^{- 1} (C))\\
        & = (g_{\ast} f_{\ast} \mu) (C)
      \end{array}
    \end{equation}
    \item Let $h : Z \to \mathbb{R}$. Then for $x \in X$,
    \begin{equation}
      \begin{array}{ll}
        (g \circ f)^{\ast} h (x) & = h (g (f (x)))\\
        & = (g^{\ast} h) (f (x))\\
        & = f^{\ast}  (g^{\ast} h) (x)
      \end{array}
    \end{equation}
  \end{enumerate}
\end{proof}

\section{Conclusion}

Pushforward and pullback operations are pivotal in connecting the structures
of different measurable spaces via measurable functions. The pushforward
provides a mechanism to transfer measures in a functorial and natural way,
while the pullback appropriately lifts functions, preserving measurability.
The integration change-of-variables formula encapsulates the deep relationship
between these operations, forming a cornerstone of modern analysis.

\end{document}
