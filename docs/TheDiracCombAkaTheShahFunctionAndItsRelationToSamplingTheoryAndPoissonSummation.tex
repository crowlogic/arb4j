\documentclass{article}
\usepackage[english]{babel}
\usepackage{geometry,amsmath,amssymb,latexsym}
\geometry{letterpaper}

%%%%%%%%%% Start TeXmacs macros
\newcommand{\cdummy}{\cdot}
\newcommand{\tmaffiliation}[1]{\\ #1}
\newcommand{\tmrsub}[1]{\ensuremath{_{\textrm{#1}}}}
\newcommand{\tmtextit}[1]{\text{{\itshape{#1}}}}
\newenvironment{proof}{\noindent\textbf{Proof\ }}{\hspace*{\fill}$\Box$\medskip}
\newtheorem{corollary}{Corollary}
\newtheorem{definition}{Definition}
\newtheorem{proposition}{Proposition}
\newtheorem{theorem}{Theorem}
%%%%%%%%%% End TeXmacs macros

\begin{document}

\title{The Shah Function: Properties, Fourier Analysis, and Applications to
Sampling Theory}

\author{
  Stephen Crowley
  \tmaffiliation{August 16, 2025}
}

\maketitle

\begin{abstract}
  This document presents the mathematical properties of the Shah function
  (Dirac comb), including its definition as a periodic distribution of delta
  functions, fundamental algebraic identities, Fourier transform properties,
  and connections to sampling theory and Poisson summation. The analysis
  establishes the role of the Shah function in relating continuous and
  discrete signal processing through sampling and periodization operations.
\end{abstract}

{\tableofcontents}

\section{Definition and Basic Properties}

\begin{definition}
  \label{def:shah_function}The Shah function with unit period is defined as
  the distribution
  \begin{equation}
    \label{eq:shah_unit} \text{III} (x) = \sum_{k \in \mathbb{Z}} \delta (x -
    k)
  \end{equation}
  where $\delta$ is the Dirac delta function. For arbitrary period $T > 0$,
  the scaled Shah function is
  \begin{equation}
    \label{eq:shah_period_T} \text{III}_T (x) = \sum_{k \in \mathbb{Z}} \delta
    (x - kT)
  \end{equation}
\end{definition}

\begin{proposition}
  \label{prop:shah_support}The Shah function III$(x)$ has support on the
  integer lattice $\mathbb{Z}$. For any test function $\phi \in \mathcal{S}
  (\mathbb{R})$ (Schwartz space), the action is given by
  \begin{equation}
    \label{eq:shah_action} \langle \text{III}, \phi \rangle = \sum_{k \in
    \mathbb{Z}} \phi (k)
  \end{equation}
\end{proposition}

\begin{proof}
  By definition \eqref{eq:shah_unit} and the sifting property of the delta
  function:
  
  \begin{align}
    \langle \text{III}, \phi \rangle & \left. = \left\langle \sum_{k \in
    \mathbb{Z}} \delta \right( \cdot - k), \phi \right\rangle 
    \label{eq:shah_action_expand}\\
    & = \sum_{k \in \mathbb{Z}} \langle \delta (\cdot - k), \phi \rangle 
    \label{eq:linearity_functional}\\
    & = \sum_{k \in \mathbb{Z}} \phi (k)  \label{eq:delta_sifting}
  \end{align}
  
  where step \eqref{eq:linearity_functional} uses the linearity of the
  distributional pairing and step \eqref{eq:delta_sifting} applies the sifting
  property $\langle \delta (\cdot - k), \phi \rangle = \phi (k)$.
\end{proof}

\section{Scaling and Periodicity Properties}

\begin{theorem}
  \label{thm:shah_scaling}The Shah function satisfies the scaling relation
  \begin{equation}
    \label{eq:shah_scaling} \text{III}_T (x) = \frac{1}{T} \text{III} \left(
    \frac{x}{T} \right)
  \end{equation}
\end{theorem}

\begin{proof}
  Starting with definition \eqref{eq:shah_period_T}:
  
  \begin{align}
    \text{III}_T (x) & = \sum_{k \in \mathbb{Z}} \delta (x - kT) 
    \label{eq:scaling_start}
  \end{align}
  
  For the right-hand side of \eqref{eq:shah_scaling}:
  
  \begin{align}
    \frac{1}{T} \text{III} \left( \frac{x}{T} \right) & = \frac{1}{T}  \sum_{k
    \in \mathbb{Z}} \delta \left( \frac{x}{T} - k \right) 
    \label{eq:scaling_rhs}\\
    & = \frac{1}{T}  \sum_{k \in \mathbb{Z}} \delta \left( \frac{x - kT}{T}
    \right)  \label{eq:factor_out}\\
    & = \sum_{k \in \mathbb{Z}} \delta (x - kT)  \label{eq:delta_scaling}
  \end{align}
  
  where step \eqref{eq:delta_scaling} uses the scaling property $\delta (ax) =
  \frac{1}{|a|} \delta (x)$ with $a = 1 / T > 0$.
  
  Comparing \eqref{eq:scaling_start} and \eqref{eq:delta_scaling} establishes
  \eqref{eq:shah_scaling}.
\end{proof}

\begin{proposition}
  \label{prop:shah_periodicity}The Shah function is periodic with period $T$:
  \begin{equation}
    \label{eq:shah_periodic} \text{III}_T (x + T) = \text{III}_T (x)
  \end{equation}
\end{proposition}

\begin{proof}
  \begin{align}
    \text{III}_T (x + T) & = \sum_{k \in \mathbb{Z}} \delta (x + T - kT) 
    \label{eq:period_shift}\\
    & = \sum_{k \in \mathbb{Z}} \delta (x - (k - 1) T) 
    \label{eq:index_adjust}\\
    & = \sum_{j \in \mathbb{Z}} \delta (x - jT)  \label{eq:reindex}\\
    & = \text{III}_T (x)  \label{eq:period_result}
  \end{align}
  
  where step \eqref{eq:reindex} substitutes $j = k - 1$ and uses the fact that
  as $k$ ranges over all integers, so does $j$.
\end{proof}

\section{Sampling and Replication Properties}

\begin{theorem}
  \label{thm:sampling_property}For any function $f$ and the Shah function, the
  sampling property states
  \begin{equation}
    \label{eq:sampling} f (x) \cdot \text{III}_T (x) = \sum_{k \in \mathbb{Z}}
    f (kT) \delta (x - kT)
  \end{equation}
\end{theorem}

\begin{proof}
  The product of distributions is defined through the action on test
  functions. For $\phi \in \mathcal{S} (\mathbb{R})$:
  
  \begin{align}
    \langle f \cdot \text{III}_T, \phi \rangle & = \langle f, \text{III}_T
    \cdot \phi \rangle  \label{eq:product_definition}\\
    & = \langle f, \phi \sum_{k \in \mathbb{Z}} \delta (\cdot - kT) \rangle 
    \label{eq:shah_expand}\\
    & = \sum_{k \in \mathbb{Z}} \langle f, \phi \delta (\cdot - kT) \rangle 
    \label{eq:distribute_sum}\\
    & = \sum_{k \in \mathbb{Z}} \langle f \delta (\cdot - kT), \phi \rangle 
    \label{eq:product_delta}\\
    & = \sum_{k \in \mathbb{Z}} \langle f (kT) \delta (\cdot - kT), \phi
    \rangle  \label{eq:evaluate_at_k}\\
    & = \sum_{k \in \mathbb{Z}} f (kT) \langle \delta (\cdot - kT), \phi
    \rangle  \label{eq:factor_constant}\\
    & = \sum_{k \in \mathbb{Z}} f (kT) \phi (kT)  \label{eq:delta_action}
  \end{align}
  
  Step \eqref{eq:evaluate_at_k} uses the fact that $f \delta (x - kT) = f (kT)
  \delta (x - kT)$ since the delta function localizes the product to $x = kT$.
  
  This establishes that $f \cdot \text{III}_T$ has the same distributional
  action as $\sum_k f (kT) \delta (\cdummy - kT)$.
\end{proof}

\begin{theorem}
  \label{thm:replication_property}The replication property of the Shah
  function under convolution is
  \begin{equation}
    \label{eq:replication} (f \ast \text{III}_T) (x) = \sum_{k \in \mathbb{Z}}
    f (x - kT)
  \end{equation}
  where $\ast$ denotes convolution.
\end{theorem}

\begin{proof}
  By definition of convolution with a distribution:
  
  \begin{align}
    (f \ast \text{III}_T) (x) & = \langle \text{III}_T (\cdot), f (x - \cdot)
    \rangle  \label{eq:convolution_def}\\
    & = \left\langle \sum_{k \in \mathbb{Z}} \delta \right( \cdot - kT), f (x
    - \cdot) \rangle  \label{eq:shah_in_convolution}\\
    & = \sum_{k \in \mathbb{Z}} \langle \delta (\cdot - kT), f (x - \cdot)
    \rangle  \label{eq:sum_linearity}\\
    & = \sum_{k \in \mathbb{Z}} f (x - kT)  \label{eq:sifting_convolution}
  \end{align}
  
  where step \eqref{eq:sifting_convolution} applies the sifting property at
  $kT$.
\end{proof}

\section{Fourier Transform Properties}

\begin{theorem}
  \label{thm:shah_fourier}The Fourier transform of the Shah function with
  period $T$ is
  \begin{equation}
    \label{eq:shah_fourier_transform} \mathcal{F} [\text{III}_T] (\omega) =
    \frac{2 \pi}{T}  \sum_{n \in \mathbb{Z}} \delta \left( \omega - \frac{2
    \pi n}{T} \right) = \frac{2 \pi}{T} \text{III}_{\frac{2 \pi}{T}} (\omega)
  \end{equation}
  using the angular frequency convention $\mathcal{F} [f] (\omega) = \int_{-
  \infty}^{\infty} f (x) e^{- i \omega x} dx$.
\end{theorem}

\begin{proof}
  The Fourier transform of III\tmrsub{$T$} is computed using the Poisson
  summation formula. First, consider the fundamental period function
  \begin{equation}
    \label{eq:fundamental_period} g_T (x) = \left\{\begin{array}{ll}
      1 & \text{if } |x| < T / 2\\
      0 & \text{if } |x| > T / 2
    \end{array}\right.
  \end{equation}
  The Shah function can be expressed as the limit:
  \begin{equation}
    \label{eq:shah_as_limit} \text{III}_T (x) = \lim_{\epsilon \to 0^+} 
    \frac{1}{T}  \sum_{k \in \mathbb{Z}} g_{\epsilon}  (x - kT)
  \end{equation}
  The Fourier transform of $g_{\epsilon}$ is:
  
  \begin{align}
    \mathcal{F} [g_{\epsilon}] (\omega) & = \int_{- \epsilon / 2}^{\epsilon /
    2} e^{- i \omega x} dx  \label{eq:g_epsilon_fourier}\\
    & = \frac{2 \sin (\omega \epsilon / 2)}{\omega}  \label{eq:sinc_result}
  \end{align}
  
  For $\epsilon \to 0$, this approaches $\epsilon$ for $\omega = 0$ and $0$
  for $\omega \neq 0$, giving $\mathcal{F} [g_{\epsilon}] (\omega) \to
  \epsilon \cdot 2 \pi \delta (\omega)$ in the distributional limit.
  
  Alternatively, use the direct approach with the Poisson summation formula.
  The periodic distribution III\tmrsub{$T$} has Fourier series representation:
  \begin{equation}
    \label{eq:fourier_series_shah} \text{III}_T (x) = \frac{1}{T}  \sum_{n \in
    \mathbb{Z}} e^{i 2 \pi nx / T}
  \end{equation}
  Taking the Fourier transform term by term:
  
  \begin{align}
    \mathcal{F} [\text{III}_T] (\omega) & =\mathcal{F} \left[ \frac{1}{T} 
    \sum_{n \in \mathbb{Z}} e^{i 2 \pi nx / T} \right] 
    \label{eq:fourier_series_transform}\\
    & = \frac{1}{T}  \sum_{n \in \mathbb{Z}} \mathcal{F} [e^{i 2 \pi nx / T}]
    (\omega)  \label{eq:linearity_fourier}\\
    & = \frac{1}{T}  \sum_{n \in \mathbb{Z}} 2 \pi \delta \left( \omega -
    \frac{2 \pi n}{T} \right)  \label{eq:exponential_fourier}\\
    & = \frac{2 \pi}{T}  \sum_{n \in \mathbb{Z}} \delta \left( \omega -
    \frac{2 \pi n}{T} \right)  \label{eq:factor_out_2pi}
  \end{align}
  
  Step \eqref{eq:exponential_fourier} uses $\mathcal{F} [e^{i \omega_0 x}]
  (\omega) = 2 \pi \delta (\omega - \omega_0)$.
\end{proof}

\begin{corollary}
  \label{cor:shah_self_reciprocal}The Shah function is self-reciprocal under
  Fourier transform up to scaling:
  \begin{equation}
    \label{eq:self_reciprocal} \mathcal{F} [\text{III}_T] = \frac{2 \pi}{T}
    \text{III}_{\frac{2 \pi}{T}}
  \end{equation}
\end{corollary}

\section{Sampling Theorem and Spectral Replication}

\begin{theorem}
  \label{thm:sampling_spectral}When a function $f (x)$ is multiplied by III$_T
  (x)$, its Fourier transform becomes
  \begin{equation}
    \label{eq:sampling_spectrum} \mathcal{F} [f \cdummy \text{III}_T] (\omega)
    = \frac{1}{T}  \sum_{n \in \mathbb{Z}} \mathcal{F} [f]  \left( \omega -
    \frac{2 \pi n}{T} \right)
  \end{equation}
\end{theorem}

\begin{proof}
  Using the convolution theorem and result \eqref{eq:shah_fourier_transform}:
  
  \begin{align}
    \mathcal{F} [f \cdummy \text{III}_T] (\omega) & = \frac{1}{2 \pi} 
    (\mathcal{F}[f] \ast \mathcal{F}[\text{III}_T]) (\omega) 
    \label{eq:convolution_theorem}\\
    & = \frac{1}{2 \pi} \mathcal{F} [f] \ast \left( \frac{2 \pi}{T}
    \text{III}_{\frac{2 \pi}{T}} \right) (\omega) 
    \label{eq:substitute_shah_fourier}\\
    & = \frac{1}{T} (\mathcal{F} [f] \ast \text{III}_{\frac{2 \pi}{T}})
    (\omega)  \label{eq:factor_constants}
  \end{align}
  
  Applying the replication property \eqref{eq:replication} to the convolution:
  
  \begin{align}
    (\mathcal{F} [f] \ast \text{III}_{\frac{2 \pi}{T}}) (\omega) & = \sum_{n
    \in \mathbb{Z}} \mathcal{F} [f]  \left( \omega - n \cdot \frac{2 \pi}{T}
    \right)  \label{eq:replication_applied}\\
    & = \sum_{n \in \mathbb{Z}} \mathcal{F} [f]  \left( \omega - \frac{2 \pi
    n}{T} \right)  \label{eq:index_order}
  \end{align}
  
  Combining with \eqref{eq:factor_constants}:
  \begin{equation}
    \label{eq:final_sampling_spectrum} \mathcal{F} [f \cdummy \text{III}_T]
    (\omega) = \frac{1}{T}  \sum_{n \in \mathbb{Z}} \mathcal{F} [f]  \left(
    \omega - \frac{2 \pi n}{T} \right)
  \end{equation}
\end{proof}

\begin{theorem}
  \label{thm:periodization_spectrum}When a function $f (x)$ is convolved with
  III$_T (x)$, its Fourier transform becomes
  \begin{equation}
    \label{eq:periodization_spectrum} \mathcal{F}[f \ast \text{III}_T]
    (\omega) =\mathcal{F} [f] (\omega) \cdot \mathcal{F} [\text{III}_T]
    (\omega) = \frac{2 \pi}{T} \mathcal{F} [f] (\omega)  \sum_{n \in
    \mathbb{Z}} \delta \left( \omega - \frac{2 \pi n}{T} \right)
  \end{equation}
\end{theorem}

\begin{proof}
  Direct application of the convolution theorem:
  
  \begin{align}
    \mathcal{F}[f \ast \text{III}_T] (\omega) & =\mathcal{F} [f] (\omega)
    \cdot \mathcal{F} [\text{III}_T] (\omega)  \label{eq:product_theorem}\\
    & =\mathcal{F} [f] (\omega) \cdot \frac{2 \pi}{T}  \sum_{n \in
    \mathbb{Z}} \delta \left( \omega - \frac{2 \pi n}{T} \right) 
    \label{eq:substitute_shah_transform}\\
    & = \frac{2 \pi}{T} \mathcal{F} [f] (\omega)  \sum_{n \in \mathbb{Z}}
    \delta \left( \omega - \frac{2 \pi n}{T} \right)  \label{eq:distribute_f}
  \end{align}
  
  This shows that the spectrum is sampled at the reciprocal lattice points
  $\frac{2 \pi n}{T}$.
\end{proof}

\section{Two-Dimensional Shah Function}

\begin{definition}
  \label{def:shah_2d}The two-dimensional Shah function on a rectangular
  lattice with periods $T_x, T_y > 0$ is
  \begin{equation}
    \label{eq:shah_2d} \text{III}_{T_x, T_y} (x, y) = \sum_{m, n \in
    \mathbb{Z}} \delta (x - mT_x) \delta (y - nT_y)
  \end{equation}
\end{definition}

\begin{theorem}
  \label{thm:shah_2d_fourier}The two-dimensional Fourier transform of the
  rectangular lattice Shah function is
  \begin{equation}
    \label{eq:shah_2d_fourier} \mathcal{F} [\text{III}_{T_x, T_y}] (k_x, k_y)
    = \frac{(2 \pi)^2}{T_x T_y}  \sum_{m, n \in \mathbb{Z}} \delta \left( k_x
    - \frac{2 \pi m}{T_x} \right) \delta \left( k_y - \frac{2 \pi n}{T_y}
    \right)
  \end{equation}
\end{theorem}

\begin{proof}
  The two-dimensional Fourier transform factorizes:
  
  \begin{align}
    & \mathcal{F} [\text{III}_{T_x, T_y}] (k_x, k_y) \nonumber\\
    & = \iint \text{III}_{T_x, T_y} (x, y) e^{- i (k_x x + k_y y)} dxdy 
    \label{eq:2d_fourier_def}\\
    & = \iint \sum_{m, n \in \mathbb{Z}} \delta (x - mT_x) \delta (y - nT_y)
    e^{- i (k_x x + k_y y)} dxdy  \label{eq:substitute_2d_shah}\\
    & = \sum_{m, n \in \mathbb{Z}} \iint \delta (x - mT_x) \delta (y - nT_y)
    e^{- i (k_x x + k_y y)} dxdy  \label{eq:interchange_sum_integral}\\
    & = \sum_{m, n \in \mathbb{Z}} e^{- i (k_x mT_x + k_y nT_y)} 
    \label{eq:delta_sifting_2d}\\
    & = \sum_{m \in \mathbb{Z}} e^{- ik_x mT_x}  \sum_{n \in \mathbb{Z}} e^{-
    ik_y nT_y}  \label{eq:factor_exponentials}
  \end{align}
  
  Each sum is a one-dimensional Shah transform:
  
  \begin{align}
    \sum_{m \in \mathbb{Z}} e^{- ik_x mT_x} & = \frac{2 \pi}{T_x}  \sum_{j \in
    \mathbb{Z}} \delta \left( k_x - \frac{2 \pi j}{T_x} \right) 
    \label{eq:x_direction_shah}\\
    \sum_{n \in \mathbb{Z}} e^{- ik_y nT_y} & = \frac{2 \pi}{T_y}  \sum_{\ell
    \in \mathbb{Z}} \delta \left( k_y - \frac{2 \pi \ell}{T_y} \right) 
    \label{eq:y_direction_shah}
  \end{align}
  
  Taking the product:
  
  \begin{align}
    & \mathcal{F} [\text{III}_{T_x, T_y}] (k_x, k_y) \nonumber\\
    & = \frac{2 \pi}{T_x}  \sum_{j \in \mathbb{Z}} \delta \left( k_x -
    \frac{2 \pi j}{T_x} \right) \cdot \frac{2 \pi}{T_y}  \sum_{\ell \in
    \mathbb{Z}} \delta \left( k_y - \frac{2 \pi \ell}{T_y} \right) 
    \label{eq:product_1d_transforms}\\
    & = \frac{(2 \pi)^2}{T_x T_y}  \sum_{j, \ell \in \mathbb{Z}} \delta
    \left( k_x - \frac{2 \pi j}{T_x} \right) \delta \left( k_y - \frac{2 \pi
    \ell}{T_y} \right)  \label{eq:final_2d_result}
  \end{align}
  
  Relabeling $j = m, \ell = n$ gives equation \eqref{eq:shah_2d_fourier}.
\end{proof}

\section{Weighted Shah Functions}

\begin{definition}
  \label{def:weighted_shah}A weighted two-dimensional Shah function
  incorporates reliability weights $w_{m, n}$, density weights $\rho (x, y)$,
  and local tapers $\tau (x, y)$:
  \begin{equation}
    \label{eq:weighted_shah} \text{III}_{w, \rho, \tau} (x, y) = \sum_{m, n
    \in \mathbb{Z}} w_{m, n} \rho (mT_x, nT_y) \tau (x - mT_x, y - nT_y)
    \delta (x - mT_x) \delta (y - nT_y)
  \end{equation}
\end{definition}

\begin{theorem}
  \label{thm:weighted_shah_properties}The weighted Shah function satisfies the
  sampling property
  \begin{equation}
    \label{eq:weighted_sampling} f (x, y) \cdot \text{III}_{w, \rho, \tau} (x,
    y) = \sum_{m, n} w_{m, n} \rho (mT_x, nT_y) f (mT_x, nT_y) \tau (0, 0)
    \delta (x - mT_x) \delta (y - nT_y)
  \end{equation}
\end{theorem}

\begin{proof}
  For each term in the weighted sum:
  
  \begin{align}
    & f (x, y) \cdot w_{m, n} \rho (mT_x, nT_y) \tau (x - mT_x, y - nT_y)
    \delta (x - mT_x) \delta (y - nT_y) \nonumber\\
    & = w_{m, n} \rho (mT_x, nT_y) f (x, y) \tau (x - mT_x, y - nT_y) \delta
    (x - mT_x) \delta (y - nT_y)  \label{eq:rearrange_factors}\\
    & = w_{m, n} \rho (mT_x, nT_y) f (mT_x, nT_y) \tau (0, 0) \delta (x -
    mT_x) \delta (y - nT_y)  \label{eq:evaluate_at_lattice}
  \end{align}
  
  Step \eqref{eq:evaluate_at_lattice} uses the fact that the delta functions
  force evaluation at $(x, y) = (mT_x, nT_y)$, so $f (x, y) = f (mT_x, nT_y)$
  and $\tau (x - mT_x, y - nT_y) = \tau (0, 0)$.
  
  Summing over all lattice points $(m, n)$ gives equation
  \eqref{eq:weighted_sampling}.
\end{proof}

\section{Connection to Poisson Summation}

\begin{theorem}
  \label{thm:poisson_shah_connection}The Poisson summation formula can be
  expressed using Shah functions as
  \begin{equation}
    \label{eq:poisson_shah} \sum_{k \in \mathbb{Z}} f (kT) = \frac{1}{T} 
    \sum_{n \in \mathbb{Z}} \mathcal{F} [f] \left( \frac{2 \pi n}{T} \right)
  \end{equation}
  This identity follows from the duality between multiplication by
  III\tmrsub{$T$} (sampling) and convolution with III\tmrsub{$2 \pi / T$}
  (replication) in the frequency domain.
\end{theorem}

\begin{proof}
  Consider the identity:
  \begin{equation}
    \label{eq:shah_pairing} \langle \text{III}_T, f \rangle = \langle
    \text{III}, f (T \cdot) \rangle = \sum_{k \in \mathbb{Z}} f (kT)
  \end{equation}
  By Parseval's theorem for distributions:
  \begin{equation}
    \label{eq:parseval_shah} \langle \text{III}_T, f \rangle = \frac{1}{2 \pi}
    \langle \mathcal{F}[\text{III}_T], \mathcal{F}[f] \rangle
  \end{equation}
  Substituting the Fourier transform of the Shah function from equation
  \eqref{eq:shah_fourier_transform}:
  
  \begin{align}
    \frac{1}{2 \pi}  \langle \mathcal{F}[\text{III}_T], \mathcal{F}[f] \rangle
    & = \frac{1}{2 \pi}  \left\langle \frac{2 \pi}{T}  \sum_{n \in \mathbb{Z}}
    \delta \left( \cdot - \frac{2 \pi n}{T} \right), \mathcal{F}[f]
    \right\rangle  \label{eq:substitute_shah_fourier_full}\\
    & = \frac{1}{T}  \sum_{n \in \mathbb{Z}} \left\langle \delta \left( \cdot
    - \frac{2 \pi n}{T} \right), \mathcal{F}[f] \right\rangle 
    \label{eq:factor_constants_poisson}\\
    & = \frac{1}{T}  \sum_{n \in \mathbb{Z}} \mathcal{F} [f] \left( \frac{2
    \pi n}{T} \right)  \label{eq:delta_evaluation}
  \end{align}
  
  Equating \eqref{eq:shah_pairing} and \eqref{eq:delta_evaluation} yields the
  Poisson summation formula \eqref{eq:poisson_shah}.
\end{proof}

\section{Applications to Discrete Fourier Transform}

\begin{theorem}
  \label{thm:dft_shah_connection}The discrete Fourier transform of a finite
  sequence can be understood as the result of sampling and periodization
  operations using Shah functions. For a sequence $\{x_n \}_{n = 0}^{N - 1}$
  with $x_n = f (nT)$ for some continuous function $f$, the DFT coefficients
  satisfy
  \begin{equation}
    \label{eq:dft_shah} X_k = T \sum_{m \in \mathbb{Z}} \mathcal{F} [f] 
    \left( \frac{2 \pi k}{NT} - \frac{2 \pi m}{T} \right)
  \end{equation}
\end{theorem}

\begin{proof}
  The finite sequence corresponds to sampling $f$ with a windowed Shah
  function:
  \begin{equation}
    \label{eq:windowed_sampling} f_s (x) = f (x)  \sum_{n = 0}^{N - 1} \delta
    (x - nT) = f (x) \cdot W_N (x) \cdot \text{III}_T (x)
  \end{equation}
  where $W_N (x) = \sum_{n = 0}^{N - 1} \delta (x - nT) / \text{III}_T (x)$ is
  the windowing function.
  
  The DFT coefficient is:
  
  \begin{align}
    X_k & = \sum_{n = 0}^{N - 1} f (nT) e^{- 2 \pi ikn / N} 
    \label{eq:dft_definition}\\
    & = T \sum_{n = 0}^{N - 1} f (nT) e^{- 2 \pi iknT / (NT)} 
    \label{eq:factor_T}\\
    & = T \int_{- \infty}^{\infty} f_s (x) e^{- 2 \pi ikx / (NT)} dx 
    \label{eq:integral_form}
  \end{align}
  
  Step \eqref{eq:integral_form} recognizes the sum as a Riemann sum
  approximation to the integral with spacing $T$.
  
  The Fourier transform of $f_s$ involves convolution with the Shah transform:
  
  \begin{align}
    \mathcal{F} [f_s] \left( \frac{2 \pi k}{NT} \right) & = \frac{1}{T} 
    \sum_{m \in \mathbb{Z}} \mathcal{F} [f \cdot W_N]  \left( \frac{2 \pi
    k}{NT} - \frac{2 \pi m}{T} \right)  \label{eq:sampling_effect}
  \end{align}
  
  For large $N$, the windowing effect becomes negligible for most frequencies,
  giving:
  \begin{equation}
    \label{eq:asymptotic_dft} X_k \approx T \sum_{m \in \mathbb{Z}}
    \mathcal{F} [f]  \left( \frac{2 \pi k}{NT} - \frac{2 \pi m}{T} \right)
  \end{equation}
\end{proof}

\section{Normalization and Integral Properties}

\begin{proposition}
  \label{prop:shah_normalization}The Shah function satisfies the normalization
  condition that for any interval of length $T$ containing exactly one lattice
  point,
  \begin{equation}
    \label{eq:shah_integral} \int_a^{a + T} \text{III}_T (x) \phi (x) dx =
    \phi (kT)
  \end{equation}
  where $kT \in (a, a + T)$ is the unique lattice point in the interval and
  $\phi$ is any test function.
\end{proposition}

\begin{proof}
  Since III$_T (x) = \sum_{j \in \mathbb{Z}} \delta (x - jT)$, only one term
  contributes to the integral:
  
  \begin{align}
    \int_a^{a + T} \text{III}_T (x) \phi (x) dx & = \int_a^{a + T} \sum_{j \in
    \mathbb{Z}} \delta (x - jT) \phi (x) dx  \label{eq:expand_integral}\\
    & = \sum_{j \in \mathbb{Z}} \int_a^{a + T} \delta (x - jT) \phi (x) dx 
    \label{eq:interchange_sum_integral_norm}\\
    & = \int_a^{a + T} \delta (x - kT) \phi (x) dx  \label{eq:single_term}\\
    & = \phi (kT)  \label{eq:sifting_normalization}
  \end{align}
  
  Step \eqref{eq:single_term} uses the fact that only $j = k$ contributes
  since $kT$ is the unique lattice point in $(a, a + T)$.
\end{proof}

This normalization ensures that the Shah function acts as a proper sampling
operator, extracting function values at lattice points with unit weight per
period.

{\cite{grafakosClassicalFourierAnalysis}}{\cite{hormanderI}}{\cite{rudinRealComplexAnalysis}}{\cite{katznelsonHarmonicAnalysis}}{\cite{steinWeissEuclidean}}{\cite{evansPDE}}{\cite{courantHilbertII}}{\cite{follandFourierApplications}}{\cite{ahlforsComplexAnalysis}}{\cite{mumfordTataI}}

\begin{thebibliography}{10}
  \bibitem[1]{ahlforsComplexAnalysis}Lars~V.~Ahlfors.
  {\newblock}\tmtextit{Complex Analysis}. {\newblock}McGraw-Hill, 3rd 
  edition, 1979.{\newblock}
  
  \bibitem[2]{courantHilbertII}Richard Courant  and  David Hilbert.
  {\newblock}\tmtextit{Methods of Mathematical Physics, Vol. II: Partial
  Differential Equations}. {\newblock}Wiley--VCH, 1962.{\newblock}
  
  \bibitem[3]{evansPDE}Lawrence~C.~Evans. {\newblock}\tmtextit{Partial
  Differential Equations}. {\newblock}American Mathematical Society, 2nd 
  edition, 2010.{\newblock}
  
  \bibitem[4]{follandFourierApplications}Gerald~B.~Folland.
  {\newblock}\tmtextit{Fourier Analysis and Its Applications}.
  {\newblock}American Mathematical Society, 2009.{\newblock}
  
  \bibitem[5]{grafakosClassicalFourierAnalysis}Loukas Grafakos.
  {\newblock}\tmtextit{Classical Fourier Analysis}. {\newblock}Springer, 3rd 
  edition, 2014.{\newblock}
  
  \bibitem[6]{hormanderI}Lars H{\"o}rmander. {\newblock}\tmtextit{The Analysis
  of Linear Partial Differential Operators I: Distribution Theory and Fourier
  Analysis}. {\newblock}Springer, 2nd  edition, 1990.{\newblock}
  
  \bibitem[7]{katznelsonHarmonicAnalysis}Yitzhak Katznelson.
  {\newblock}\tmtextit{An Introduction to Harmonic Analysis}.
  {\newblock}Cambridge University Press, 3rd  edition, 2004.{\newblock}
  
  \bibitem[8]{mumfordTataI}David Mumford. {\newblock}\tmtextit{Tata Lectures
  on Theta I}. {\newblock}Birkh{\"a}user, 1983.{\newblock}
  
  \bibitem[9]{rudinRealComplexAnalysis}Walter Rudin. {\newblock}\tmtextit{Real
  and Complex Analysis}. {\newblock}McGraw-Hill, 3rd  edition,
  1987.{\newblock}
  
  \bibitem[10]{steinWeissEuclidean}Elias~M.~Stein  and  Guido Weiss.
  {\newblock}\tmtextit{Introduction to Fourier Analysis on Euclidean Spaces}.
  {\newblock}Princeton University Press, 1971.{\newblock}
\end{thebibliography}

\end{document}
