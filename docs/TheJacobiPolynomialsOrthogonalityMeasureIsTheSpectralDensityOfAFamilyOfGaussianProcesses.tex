\documentclass{article}
\usepackage{amsmath}
\usepackage{amssymb}
\usepackage{amsthm}

\title{Fourier Transform of the Jacobi Weight Function}
\author{}
\date{}

\begin{document}
\maketitle

\begin{theorem}
For $\alpha, \beta > -1$, the Fourier transform of the Jacobi weight function
\[ w(x) = (1-x)^\alpha(1+x)^\beta \quad \text{on } [-1,1] \]
is given by
\[ \hat{w}(t) = 2^{\alpha+\beta+1} \Gamma(\alpha+1) \Gamma(\beta+1) \frac{J_{\alpha+\beta+1}(2t)}{(2t)^{\alpha+\beta+1}} e^{it} \]
where $J_\nu$ denotes the Bessel function of the first kind of order $\nu$.
\end{theorem}

\begin{proof}
\textbf{1. Initial Setup and Conditions:}\\
The conditions $\alpha, \beta > -1$ ensure:
\begin{itemize}
\item The weight function is integrable on $[-1,1]$
\item The Beta function $B(\alpha+1,\beta+1)$ is well-defined
\item The resulting Bessel function expression converges
\end{itemize}

We need to compute the Fourier transform:
\[ \hat{w}(t) = \int_{-1}^1 (1-x)^\alpha (1+x)^\beta e^{-ixt} \, dx \]

\textbf{2. Change of Variables:}\\
Let $u = (1+x)/2$, then:
\begin{align*}
x &= 2u-1 \\
dx &= 2du \\
\text{when } x &= -1, u = 0 \\
\text{when } x &= 1, u = 1
\end{align*}

The integral becomes:
\[ \hat{w}(t) = 2^{1+\alpha+\beta} \int_0^1 (1-u)^\alpha u^\beta e^{-i(2u-1)t} \, du \]

\textbf{3. Exponential Splitting:}
\[ e^{-i(2u-1)t} = e^{-i2ut} e^{it} \]

\textbf{4. Connection to Hypergeometric Functions:}\\
The integral now takes the form:
\[ 2^{1+\alpha+\beta} e^{it} \int_0^1 (1-u)^\alpha u^\beta e^{-i2ut} \, du \]

This integral relates to the generalized hypergeometric function $_1F_1$ through:
\[ \int_0^1 u^\beta (1-u)^\alpha e^{-i2ut} \, du = B(\alpha+1,\beta+1) {}_1F_1(\beta+1;\alpha+\beta+2;-2it) \]

\textbf{5. Transformation to Bessel Functions:}\\
The hypergeometric function transforms to Bessel form through three key steps:

First, applying the Kummer transformation:
\[ {}_1F_1(a;b;z) = e^z {}_1F_1(b-a;b;-z) \]

Second, using the limiting relation between confluent hypergeometric and Bessel functions:
\[ J_\nu(z) = \frac{(z/2)^\nu}{\Gamma(\nu+1)} {}_0F_1(;\nu+1;-z^2/4) \]

Finally, through Hankel's contour integral representation:
\[ J_\nu(z) = \frac{z^\nu}{2^\nu \Gamma(\nu+1)} {}_0F_1\left(;\nu+1;-\frac{z^2}{4}\right) \]

These transformations yield:
\[ \int_0^1 (1-u)^\alpha u^\beta e^{-i2ut} \, du = B(\alpha+1, \beta+1) \frac{J_{\alpha+\beta+1}(2t)}{(2t)^{\alpha+\beta+1}} \]

\textbf{6. Final Result:}\\
Combining all terms:
\[ \hat{w}(t) = 2^{1+\alpha+\beta} B(\alpha+1, \beta+1) \frac{J_{\alpha+\beta+1}(2t)}{(2t)^{\alpha+\beta+1}} e^{it} \]

Using the Beta function relation $B(a,b) = \frac{\Gamma(a)\Gamma(b)}{\Gamma(a+b)}$, we obtain our final result:
\[ \hat{w}(t) = 2^{\alpha+\beta+1} \Gamma(\alpha+1) \Gamma(\beta+1) \frac{J_{\alpha+\beta+1}(2t)}{(2t)^{\alpha+\beta+1}} e^{it} \]

The $e^{it}$ term carries the essential phase information of the Fourier transform, completing our proof.
\end{proof}

\end{document}
